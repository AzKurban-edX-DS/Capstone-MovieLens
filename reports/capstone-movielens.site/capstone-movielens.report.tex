% Options for packages loaded elsewhere
\PassOptionsToPackage{unicode}{hyperref}
\PassOptionsToPackage{hyphens}{url}
\PassOptionsToPackage{dvipsnames,svgnames,x11names}{xcolor}
\documentclass[
]{article}
\usepackage{xcolor}
\usepackage[margin=1in]{geometry}
\usepackage{amsmath,amssymb}
\setcounter{secnumdepth}{5}
\usepackage{iftex}
\ifPDFTeX
  \usepackage[T1]{fontenc}
  \usepackage[utf8]{inputenc}
  \usepackage{textcomp} % provide euro and other symbols
\else % if luatex or xetex
  \usepackage{unicode-math} % this also loads fontspec
  \defaultfontfeatures{Scale=MatchLowercase}
  \defaultfontfeatures[\rmfamily]{Ligatures=TeX,Scale=1}
\fi
\usepackage{lmodern}
\ifPDFTeX\else
  % xetex/luatex font selection
\fi
% Use upquote if available, for straight quotes in verbatim environments
\IfFileExists{upquote.sty}{\usepackage{upquote}}{}
\IfFileExists{microtype.sty}{% use microtype if available
  \usepackage[]{microtype}
  \UseMicrotypeSet[protrusion]{basicmath} % disable protrusion for tt fonts
}{}
\makeatletter
\@ifundefined{KOMAClassName}{% if non-KOMA class
  \IfFileExists{parskip.sty}{%
    \usepackage{parskip}
  }{% else
    \setlength{\parindent}{0pt}
    \setlength{\parskip}{6pt plus 2pt minus 1pt}}
}{% if KOMA class
  \KOMAoptions{parskip=half}}
\makeatother
\usepackage{color}
\usepackage{fancyvrb}
\newcommand{\VerbBar}{|}
\newcommand{\VERB}{\Verb[commandchars=\\\{\}]}
\DefineVerbatimEnvironment{Highlighting}{Verbatim}{commandchars=\\\{\}}
% Add ',fontsize=\small' for more characters per line
\usepackage{framed}
\definecolor{shadecolor}{RGB}{248,248,248}
\newenvironment{Shaded}{\begin{snugshade}}{\end{snugshade}}
\newcommand{\AlertTok}[1]{\textcolor[rgb]{0.94,0.16,0.16}{#1}}
\newcommand{\AnnotationTok}[1]{\textcolor[rgb]{0.56,0.35,0.01}{\textbf{\textit{#1}}}}
\newcommand{\AttributeTok}[1]{\textcolor[rgb]{0.13,0.29,0.53}{#1}}
\newcommand{\BaseNTok}[1]{\textcolor[rgb]{0.00,0.00,0.81}{#1}}
\newcommand{\BuiltInTok}[1]{#1}
\newcommand{\CharTok}[1]{\textcolor[rgb]{0.31,0.60,0.02}{#1}}
\newcommand{\CommentTok}[1]{\textcolor[rgb]{0.56,0.35,0.01}{\textit{#1}}}
\newcommand{\CommentVarTok}[1]{\textcolor[rgb]{0.56,0.35,0.01}{\textbf{\textit{#1}}}}
\newcommand{\ConstantTok}[1]{\textcolor[rgb]{0.56,0.35,0.01}{#1}}
\newcommand{\ControlFlowTok}[1]{\textcolor[rgb]{0.13,0.29,0.53}{\textbf{#1}}}
\newcommand{\DataTypeTok}[1]{\textcolor[rgb]{0.13,0.29,0.53}{#1}}
\newcommand{\DecValTok}[1]{\textcolor[rgb]{0.00,0.00,0.81}{#1}}
\newcommand{\DocumentationTok}[1]{\textcolor[rgb]{0.56,0.35,0.01}{\textbf{\textit{#1}}}}
\newcommand{\ErrorTok}[1]{\textcolor[rgb]{0.64,0.00,0.00}{\textbf{#1}}}
\newcommand{\ExtensionTok}[1]{#1}
\newcommand{\FloatTok}[1]{\textcolor[rgb]{0.00,0.00,0.81}{#1}}
\newcommand{\FunctionTok}[1]{\textcolor[rgb]{0.13,0.29,0.53}{\textbf{#1}}}
\newcommand{\ImportTok}[1]{#1}
\newcommand{\InformationTok}[1]{\textcolor[rgb]{0.56,0.35,0.01}{\textbf{\textit{#1}}}}
\newcommand{\KeywordTok}[1]{\textcolor[rgb]{0.13,0.29,0.53}{\textbf{#1}}}
\newcommand{\NormalTok}[1]{#1}
\newcommand{\OperatorTok}[1]{\textcolor[rgb]{0.81,0.36,0.00}{\textbf{#1}}}
\newcommand{\OtherTok}[1]{\textcolor[rgb]{0.56,0.35,0.01}{#1}}
\newcommand{\PreprocessorTok}[1]{\textcolor[rgb]{0.56,0.35,0.01}{\textit{#1}}}
\newcommand{\RegionMarkerTok}[1]{#1}
\newcommand{\SpecialCharTok}[1]{\textcolor[rgb]{0.81,0.36,0.00}{\textbf{#1}}}
\newcommand{\SpecialStringTok}[1]{\textcolor[rgb]{0.31,0.60,0.02}{#1}}
\newcommand{\StringTok}[1]{\textcolor[rgb]{0.31,0.60,0.02}{#1}}
\newcommand{\VariableTok}[1]{\textcolor[rgb]{0.00,0.00,0.00}{#1}}
\newcommand{\VerbatimStringTok}[1]{\textcolor[rgb]{0.31,0.60,0.02}{#1}}
\newcommand{\WarningTok}[1]{\textcolor[rgb]{0.56,0.35,0.01}{\textbf{\textit{#1}}}}
\usepackage{longtable,booktabs,array}
\usepackage{calc} % for calculating minipage widths
% Correct order of tables after \paragraph or \subparagraph
\usepackage{etoolbox}
\makeatletter
\patchcmd\longtable{\par}{\if@noskipsec\mbox{}\fi\par}{}{}
\makeatother
% Allow footnotes in longtable head/foot
\IfFileExists{footnotehyper.sty}{\usepackage{footnotehyper}}{\usepackage{footnote}}
\makesavenoteenv{longtable}
\usepackage{graphicx}
\makeatletter
\newsavebox\pandoc@box
\newcommand*\pandocbounded[1]{% scales image to fit in text height/width
  \sbox\pandoc@box{#1}%
  \Gscale@div\@tempa{\textheight}{\dimexpr\ht\pandoc@box+\dp\pandoc@box\relax}%
  \Gscale@div\@tempb{\linewidth}{\wd\pandoc@box}%
  \ifdim\@tempb\p@<\@tempa\p@\let\@tempa\@tempb\fi% select the smaller of both
  \ifdim\@tempa\p@<\p@\scalebox{\@tempa}{\usebox\pandoc@box}%
  \else\usebox{\pandoc@box}%
  \fi%
}
% Set default figure placement to htbp
\def\fps@figure{htbp}
\makeatother
\setlength{\emergencystretch}{3em} % prevent overfull lines
\providecommand{\tightlist}{%
  \setlength{\itemsep}{0pt}\setlength{\parskip}{0pt}}
\usepackage[backend=biber,natbib=true,style=numeric,backref=true,sorting=none]{biblatex}
\addbibresource{references.bib}
\usepackage{awesomebox}
\usepackage{hyperref}
\usepackage{amsmath}
\hypersetup{backref, pdfpagemode=Normal, colorlinks=true, implicit=false}
\usepackage{bookmark}
\IfFileExists{xurl.sty}{\usepackage{xurl}}{} % add URL line breaks if available
\urlstyle{same}
\hypersetup{
  pdftitle={Capstone Movielens Report},
  pdfauthor={Azamat Kurbanaev},
  colorlinks=true,
  linkcolor={red},
  filecolor={Maroon},
  citecolor={blue},
  urlcolor={Blue},
  pdfcreator={LaTeX via pandoc}}

\title{Capstone Movielens Report}
\author{Azamat Kurbanaev}
\date{2026-02-26}

\begin{document}
\maketitle

{
\hypersetup{linkcolor=}
\setcounter{tocdepth}{4}
\tableofcontents
}
\newenvironment{infobox}[1]
  {
  \begin{itemize}
  \renewcommand{\labelitemi}{
    \raisebox{-.7\height}[0pt][0pt]{
      {\setkeys{Gin}{width=3em,keepaspectratio}
        \includegraphics{images/#1}}
    }
  }
  \setlength{\fboxsep}{1em}
  \begin{blackbox}
  \item
  }
  {
  \end{blackbox}
  \end{itemize}
  }

\newpage

\section{Overview and Executive Summary}\label{intro}

The goal of the project is to build a Recommendation System using a \href{http://grouplens.org/datasets/movielens/10m/}{10M version of the MovieLens dataset}.
Following the \href{https://archive.nytimes.com/bits.blogs.nytimes.com/2009/09/21/netflix-awards-1-million-prize-and-starts-a-new-contest/index.html}{Netflix Grand Prize Contest} requirements, we will evaluate the \emph{Root Mean Square Error} (hereafter \emph{RMSE} for short) score, which, as shown in \href{https://rafalab.dfci.harvard.edu/dsbook-part-2/highdim/regularization.html\#sec-netflix-loss-function}{Section 24.1 \emph{Case study: recommendation systems / Loss function}} of the \emph{Course Textbook (New Edition)}, is defined as:
\[
\mbox{RMSE} = \sqrt{\frac{1}{N} \sum_{i,j}^{N} (y_{i,j} - \hat{y}_{i,j})^2}
\]

with \(N\) being the number of user/movie combinations for which we make predictions and the sum occurring over all these combinations\autocite{IDS2_24-1_LF}.

Our goal is to achieve a value of less than 0.86490 (compare with the \emph{Netflix Grand Prize} requirement: of at least 0.8563\autocite{BigChaosSln}).

\subsection{Datasets Overview}\label{datasets-overview}

To start with we have to generate two datasets derived from the \emph{MovieLens} dataset mentioned above:

\begin{itemize}
\tightlist
\item
  \textbf{edx:} we use it to develop and train our algorithms;
\item
  \textbf{final\_holdout\_test:} according to the course requirements, we use it exclusively to evaluate the \emph{\textbf{RMSE}} of our final algorithm.
\end{itemize}

For this purpose, a dedicated package, \href{https://github.com/AzKurban-edX-DS/edx.capstone.movielens.data/blob/main/README.md}{edx.capstone.movielens.data}, has been developed by the author of \emph{this Report}. The source code of the package is available on \href{https://github.com/AzKurban-edX-DS/edx.capstone.movielens.data}{GitHub} (the main \texttt{R} script that prepares the necessary datasets is located in the file \href{https://github.com/AzKurban-edX-DS/edx.capstone.movielens.data/blob/main/data-raw/init-movielens-data.R}{init-movielens-data.R})\autocite{edx_capstone_movielens_data}.

We install the development version of this package from the \emph{GitHub repository} using the following \href{https://github.com/AzKurban-edX-DS/Capstone-MovieLens/blob/main/r/_startup/1.init.R\#L16}{code snippet}:

\begin{Shaded}
\begin{Highlighting}[]
\DocumentationTok{\#\# Load Source Datasets from Specially Designed Package {-}{-}{-}{-}{-}{-}{-}{-}{-}{-}{-}{-}{-}{-}{-}{-}{-}{-}{-}{-}{-}{-}{-}{-}}
\ControlFlowTok{if}\NormalTok{(}\SpecialCharTok{!}\FunctionTok{require}\NormalTok{(edx.capstone.movielens.data)) \{}
\NormalTok{  start }\OtherTok{\textless{}{-}} \FunctionTok{put\_start\_date}\NormalTok{()}
\NormalTok{  pak}\SpecialCharTok{::}\FunctionTok{pak}\NormalTok{(}\StringTok{"AzKurban{-}edX{-}DS/edx.capstone.movielens.data"}\NormalTok{)}
  \FunctionTok{put\_end\_date}\NormalTok{(start)}
\NormalTok{\}}

\FunctionTok{put\_log}\NormalTok{(}\StringTok{"Dataset loaded from \textasciigrave{}edx.capstone.movielens.data\textasciigrave{} package: edx"}\NormalTok{)}
\end{Highlighting}
\end{Shaded}

\begin{noteblock}
In the code snippet above, we use the user-defined helper function \hyperref[func.put_log]{put\_log} described in section \hyperref[appndx_a.CHF.logging-funcs]{Logging Functions} of \hyperref[appndx_a]{Appendix A}.

\end{noteblock}

Next, we attach the correspondent library to the global environment using the following \href{https://github.com/AzKurban-edX-DS/Capstone-MovieLens/blob/main/r/src/support-functions/data.helper.functions.R\#L235}{line of code}:

\begin{Shaded}
\begin{Highlighting}[]
\FunctionTok{library}\NormalTok{(edx.capstone.movielens.data)}
\end{Highlighting}
\end{Shaded}

\begin{noteblock}
The line of code above is used in the \hyperref[func.init_source_datasets]{init\_source\_datasets} user-defined helper function described in section \hyperref[initializing-input-datasets]{Initializing Input Datasets} of \hyperref[appndx_a]{Appendix A}.

\end{noteblock}

Finally, we have the datasets listed above:

\begin{Shaded}
\begin{Highlighting}[]
\FunctionTok{summary}\NormalTok{(edx)}
\end{Highlighting}
\end{Shaded}

\begin{verbatim}
##      userId         movieId          rating        timestamp            title              genres         
##  Min.   :    1   Min.   :    1   Min.   :0.500   Min.   :7.897e+08   Length:9000055     Length:9000055    
##  1st Qu.:18124   1st Qu.:  648   1st Qu.:3.000   1st Qu.:9.468e+08   Class :character   Class :character  
##  Median :35738   Median : 1834   Median :4.000   Median :1.035e+09   Mode  :character   Mode  :character  
##  Mean   :35870   Mean   : 4122   Mean   :3.512   Mean   :1.033e+09                                        
##  3rd Qu.:53607   3rd Qu.: 3626   3rd Qu.:4.000   3rd Qu.:1.127e+09                                        
##  Max.   :71567   Max.   :65133   Max.   :5.000   Max.   :1.231e+09
\end{verbatim}

\begin{Shaded}
\begin{Highlighting}[]
\FunctionTok{summary}\NormalTok{(final\_holdout\_test)}
\end{Highlighting}
\end{Shaded}

\begin{verbatim}
##      userId         movieId          rating        timestamp            title              genres         
##  Min.   :    1   Min.   :    1   Min.   :0.500   Min.   :7.897e+08   Length:999999      Length:999999     
##  1st Qu.:18096   1st Qu.:  648   1st Qu.:3.000   1st Qu.:9.467e+08   Class :character   Class :character  
##  Median :35768   Median : 1827   Median :4.000   Median :1.035e+09   Mode  :character   Mode  :character  
##  Mean   :35870   Mean   : 4108   Mean   :3.512   Mean   :1.033e+09                                        
##  3rd Qu.:53621   3rd Qu.: 3624   3rd Qu.:4.000   3rd Qu.:1.127e+09                                        
##  Max.   :71567   Max.   :65133   Max.   :5.000   Max.   :1.231e+09
\end{verbatim}

\newpage

\subsubsection{\texorpdfstring{\texttt{edx} Dataset}{edx Dataset}}\label{edx-dataset}

\hfill\break
Let's look into the details of the \texttt{edx} dataset:

\begin{Shaded}
\begin{Highlighting}[]
\FunctionTok{str}\NormalTok{(edx)}
\end{Highlighting}
\end{Shaded}

\begin{verbatim}
## 'data.frame':    9000055 obs. of  6 variables:
##  $ userId   : int  1 1 1 1 1 1 1 1 1 1 ...
##  $ movieId  : int  122 185 292 316 329 355 356 362 364 370 ...
##  $ rating   : num  5 5 5 5 5 5 5 5 5 5 ...
##  $ timestamp: int  838985046 838983525 838983421 838983392 838983392 838984474 838983653 838984885 838983707 838984596 ...
##  $ title    : chr  "Boomerang (1992)" "Net, The (1995)" "Outbreak (1995)" "Stargate (1994)" ...
##  $ genres   : chr  "Comedy|Romance" "Action|Crime|Thriller" "Action|Drama|Sci-Fi|Thriller" "Action|Adventure|Sci-Fi" ...
\end{verbatim}

Note that we have 9000055 rows and six columns in there:

\begin{Shaded}
\begin{Highlighting}[]
\NormalTok{dim\_edx }\OtherTok{\textless{}{-}} \FunctionTok{dim}\NormalTok{(edx)}
\FunctionTok{print}\NormalTok{(dim\_edx)}
\end{Highlighting}
\end{Shaded}

\begin{verbatim}
## [1] 9000055       6
\end{verbatim}

First, let's note that we have 10677 different movies:

\begin{Shaded}
\begin{Highlighting}[]
\NormalTok{n\_movies }\OtherTok{\textless{}{-}} \FunctionTok{n\_distinct}\NormalTok{(edx}\SpecialCharTok{$}\NormalTok{movieId)}
\FunctionTok{print}\NormalTok{(n\_movies)}
\end{Highlighting}
\end{Shaded}

\begin{verbatim}
## [1] 10677
\end{verbatim}

and 69878 different users in the dataset:

\begin{Shaded}
\begin{Highlighting}[]
\NormalTok{n\_users }\OtherTok{\textless{}{-}} \FunctionTok{n\_distinct}\NormalTok{(edx}\SpecialCharTok{$}\NormalTok{userId)}
\FunctionTok{print}\NormalTok{(n\_users)}
\end{Highlighting}
\end{Shaded}

\begin{verbatim}
## [1] 69878
\end{verbatim}

Now, note the expressions below which confirm the fact explained in \href{https://rafalab.dfci.harvard.edu/dsbook/large-datasets.html\#movielens-data}{Section 33.7.1 \emph{Movielens data}} of the \emph{Course Textbook (First Edition)} that not every user rated every movie\autocite{IDS_E1_33-7-1}:

\begin{Shaded}
\begin{Highlighting}[]
\NormalTok{max\_possible\_ratings }\OtherTok{\textless{}{-}}\NormalTok{ n\_movies}\SpecialCharTok{*}\NormalTok{n\_users}
\FunctionTok{sprintf}\NormalTok{(}\StringTok{"Maximum possible ratings: \%s"}\NormalTok{, max\_possible\_ratings)}
\end{Highlighting}
\end{Shaded}

\begin{verbatim}
## [1] "Maximum possible ratings: 746087406"
\end{verbatim}

\begin{Shaded}
\begin{Highlighting}[]
\FunctionTok{sprintf}\NormalTok{(}\StringTok{"Rows in \textasciigrave{}edx\textasciigrave{} dataset: \%s"}\NormalTok{, dim\_edx[}\DecValTok{1}\NormalTok{])}
\end{Highlighting}
\end{Shaded}

\begin{verbatim}
## [1] "Rows in `edx` dataset: 9000055"
\end{verbatim}

\begin{Shaded}
\begin{Highlighting}[]
\FunctionTok{sprintf}\NormalTok{(}\StringTok{"Not every movie was rated: \%s"}\NormalTok{, max\_possible\_ratings }\SpecialCharTok{\textgreater{}}\NormalTok{ dim\_edx[}\DecValTok{1}\NormalTok{])}
\end{Highlighting}
\end{Shaded}

\begin{verbatim}
## [1] "Not every movie was rated: TRUE"
\end{verbatim}

As also explained in \href{https://rafalab.dfci.harvard.edu/dsbook/large-datasets.html\#movielens-data}{Section 33.7.1 of the \emph{Textbook}}, we can think of these data as a very large matrix, with users on the rows and movies on the columns, with many empty cells. Therefore, we can think of a recommendation system as filling in the \texttt{NA}s in the dataset for the movies that some or all the users do not rate. A sample from the \texttt{edx} data below illustrates this idea\autocite{IDS_E1_33-7-1}:

\begin{Shaded}
\begin{Highlighting}[]
\NormalTok{keep }\OtherTok{\textless{}{-}}\NormalTok{ edx }\SpecialCharTok{|\textgreater{}} 
\NormalTok{  dplyr}\SpecialCharTok{::}\FunctionTok{count}\NormalTok{(movieId) }\SpecialCharTok{|\textgreater{}} 
  \FunctionTok{top\_n}\NormalTok{(}\DecValTok{4}\NormalTok{, n) }\SpecialCharTok{|\textgreater{}} 
  \FunctionTok{pull}\NormalTok{(movieId)}

\NormalTok{tab }\OtherTok{\textless{}{-}}\NormalTok{ edx }\SpecialCharTok{|\textgreater{}} 
  \FunctionTok{filter}\NormalTok{(movieId }\SpecialCharTok{\%in\%}\NormalTok{ keep) }\SpecialCharTok{|\textgreater{}} 
  \FunctionTok{filter}\NormalTok{(userId }\SpecialCharTok{\%in\%} \FunctionTok{c}\NormalTok{(}\DecValTok{13}\SpecialCharTok{:}\DecValTok{20}\NormalTok{)) }\SpecialCharTok{|\textgreater{}} 
  \FunctionTok{select}\NormalTok{(userId, title, rating) }\SpecialCharTok{|\textgreater{}} 
  \FunctionTok{mutate}\NormalTok{(}\AttributeTok{title =} \FunctionTok{str\_remove}\NormalTok{(title, }\StringTok{", The"}\NormalTok{),}
         \AttributeTok{title =} \FunctionTok{str\_remove}\NormalTok{(title, }\StringTok{":.*"}\NormalTok{)) }\SpecialCharTok{|\textgreater{}}
  \FunctionTok{pivot\_wider}\NormalTok{(}\AttributeTok{names\_from =} \StringTok{"title"}\NormalTok{, }\AttributeTok{values\_from =} \StringTok{"rating"}\NormalTok{)}

\FunctionTok{print}\NormalTok{(tab)}
\end{Highlighting}
\end{Shaded}

\begin{verbatim}
## # A tibble: 5 x 5
##   userId `Pulp Fiction (1994)` `Jurassic Park (1993)` `Silence of the Lambs (1991)` `Forrest Gump (1994)`
##    <int>                 <dbl>                  <dbl>                         <dbl>                 <dbl>
## 1     13                     4                     NA                            NA                    NA
## 2     16                    NA                      3                            NA                    NA
## 3     17                    NA                     NA                             5                    NA
## 4     18                     5                      3                             5                    NA
## 5     19                    NA                      1                            NA                     4
\end{verbatim}

The following plot of the matrix for a random sample of 100 movies and 100 users with yellow indicating a user/movie combination for which we have a rating shows how \emph{sparse} the matrix is\autocite{IDS_E1_33-7-1}:

\begin{Shaded}
\begin{Highlighting}[]
\NormalTok{users }\OtherTok{\textless{}{-}} \FunctionTok{sample}\NormalTok{(}\FunctionTok{unique}\NormalTok{(edx}\SpecialCharTok{$}\NormalTok{userId), }\DecValTok{100}\NormalTok{)}

\NormalTok{rafalib}\SpecialCharTok{::}\FunctionTok{mypar}\NormalTok{()}
\NormalTok{edx}\SpecialCharTok{|\textgreater{}} 
  \FunctionTok{filter}\NormalTok{(userId }\SpecialCharTok{\%in\%}\NormalTok{ users) }\SpecialCharTok{|\textgreater{}} 
  \FunctionTok{select}\NormalTok{(userId, movieId, rating) }\SpecialCharTok{|\textgreater{}}
  \FunctionTok{mutate}\NormalTok{(}\AttributeTok{rating =} \DecValTok{1}\NormalTok{) }\SpecialCharTok{|\textgreater{}}
  \FunctionTok{pivot\_wider}\NormalTok{(}\AttributeTok{names\_from =}\NormalTok{ movieId, }\AttributeTok{values\_from =}\NormalTok{ rating) }\SpecialCharTok{|\textgreater{}} 
\NormalTok{  (\textbackslash{}(mat) mat[, }\FunctionTok{sample}\NormalTok{(}\FunctionTok{ncol}\NormalTok{(mat), }\DecValTok{100}\NormalTok{)])() }\SpecialCharTok{|\textgreater{}}
  \FunctionTok{as.matrix}\NormalTok{() }\SpecialCharTok{|\textgreater{}} 
  \FunctionTok{t}\NormalTok{() }\SpecialCharTok{|\textgreater{}}
  \FunctionTok{image}\NormalTok{(}\DecValTok{1}\SpecialCharTok{:}\DecValTok{100}\NormalTok{, }\DecValTok{1}\SpecialCharTok{:}\DecValTok{100}\NormalTok{, }\AttributeTok{z =}\NormalTok{ \_ , }\AttributeTok{xlab =} \StringTok{"Movies"}\NormalTok{, }\AttributeTok{ylab =} \StringTok{"Users"}\NormalTok{)}
\end{Highlighting}
\end{Shaded}

\includegraphics[width=0.4\linewidth]{capstone-movielens.report_files/figure-latex/sparsity-of-movie-recs-1}

Further observations highlighted there that, as we can see from the distributions the author presented, some movies get rated more than others, and some users are more active than others in rating movies\autocite{IDS_E1_33-7-1}:

\begin{Shaded}
\begin{Highlighting}[]
\NormalTok{p1 }\OtherTok{\textless{}{-}}\NormalTok{ edx }\SpecialCharTok{|\textgreater{}} 
  \FunctionTok{count}\NormalTok{(movieId) }\SpecialCharTok{|\textgreater{}} 
  \FunctionTok{ggplot}\NormalTok{(}\FunctionTok{aes}\NormalTok{(n)) }\SpecialCharTok{+} 
  \FunctionTok{geom\_histogram}\NormalTok{(}\AttributeTok{bins =} \DecValTok{30}\NormalTok{, }\AttributeTok{color =} \StringTok{"black"}\NormalTok{) }\SpecialCharTok{+} 
  \FunctionTok{scale\_x\_log10}\NormalTok{() }\SpecialCharTok{+} 
  \FunctionTok{ggtitle}\NormalTok{(}\StringTok{"Movies"}\NormalTok{)}

\NormalTok{p2 }\OtherTok{\textless{}{-}}\NormalTok{ edx }\SpecialCharTok{|\textgreater{}} 
  \FunctionTok{count}\NormalTok{(userId) }\SpecialCharTok{|\textgreater{}} 
  \FunctionTok{ggplot}\NormalTok{(}\FunctionTok{aes}\NormalTok{(n)) }\SpecialCharTok{+} 
  \FunctionTok{geom\_histogram}\NormalTok{(}\AttributeTok{bins =} \DecValTok{30}\NormalTok{, }\AttributeTok{color =} \StringTok{"black"}\NormalTok{) }\SpecialCharTok{+} 
  \FunctionTok{scale\_x\_log10}\NormalTok{() }\SpecialCharTok{+} 
  \FunctionTok{ggtitle}\NormalTok{(}\StringTok{"Users"}\NormalTok{)}

\NormalTok{gridExtra}\SpecialCharTok{::}\FunctionTok{grid.arrange}\NormalTok{(p2, p1, }\AttributeTok{ncol =} \DecValTok{2}\NormalTok{)}
\end{Highlighting}
\end{Shaded}

\pandocbounded{\includegraphics[keepaspectratio]{capstone-movielens.report_files/figure-latex/movie-id-and-user-hists-1.pdf}}

\newpage

Finally, the following code snippet demonstrates that no movies have a rating of 0. Movies are rated from 0.5 to 5.0 in 0.5 increments:

\begin{Shaded}
\begin{Highlighting}[]
\CommentTok{\#library(dplyr)}
\NormalTok{s }\OtherTok{\textless{}{-}}\NormalTok{ edx }\SpecialCharTok{|\textgreater{}} \FunctionTok{group\_by}\NormalTok{(rating) }\SpecialCharTok{|\textgreater{}}
  \FunctionTok{summarise}\NormalTok{(}\AttributeTok{n =} \FunctionTok{n}\NormalTok{())}
\FunctionTok{print}\NormalTok{(s)}
\end{Highlighting}
\end{Shaded}

\begin{verbatim}
## # A tibble: 10 x 2
##    rating       n
##     <dbl>   <int>
##  1    0.5   85374
##  2    1    345679
##  3    1.5  106426
##  4    2    711422
##  5    2.5  333010
##  6    3   2121240
##  7    3.5  791624
##  8    4   2588430
##  9    4.5  526736
## 10    5   1390114
\end{verbatim}

\newpage

\section{Methods and Analysis}\label{methods-and-analysis}

\begin{noteblock}
All the source code of the \texttt{R} scripts utilized in this project is available on the project's \href{https://github.com/AzKurban-edX-DS/Capstone-MovieLens/tree/main/r}{GitHub repository}\autocite{edx_capstone_movielens}.

\end{noteblock}

\subsection{Preparing train and set datasets}\label{preparing-train-and-set-datasets}

\hfill\break

We will split the \texttt{edx} dataset into a training set, which we will use to build and train our models, and a test set in which we will compute the accuracy of our predictions, the way described in \href{https://rafalab.dfci.harvard.edu/dsbook-part-2/highdim/regularization.html\#sec-recommendation-systems}{Section 24.1 \emph{Case study: recommendation systems}} of the \emph{Course Textbook (New Edition)}\autocite{IDS2_24-1}.We will also use the \emph{Cross-Validation} method as described in \href{https://rafalab.dfci.harvard.edu/dsbook-part-2/ml/resampling-methods.html\#cross-validation}{Section 29.6 \emph{Cross validation}} of the \emph{Course Textbook (New Edition)}\autocite{IDS2_29-6}.

To prepare datasets for processing, we will use the following functions (described in detail in section \hyperref[initializing-input-datasets]{Initializing Input Datasets} of \hyperref[appndx_a]{Appendix A}), specifically designed for these operations:

\begin{itemize}
\item
  \hyperref[func.make_source_datasets]{make\_source\_datasets};
\item
  \hyperref[func.init_source_datasets]{init\_source\_datasets}.
\end{itemize}

As a result of executing these functions, we will obtain the following datasets (detailed in \hyperref[appndx_b]{Appendix B: Models Training Datasets}), which we will use throughout \emph{this Project}:

\begin{itemize}
\item
  \hyperref[appndx_b.edx.mx]{\texttt{edx.mx} Matrix Object}: The \texttt{edx} dataset converted to a matrix;
\item
  \hyperref[appndx_b.edx.sgr]{\texttt{edx.sgr} Object}: A dataset derived from the \texttt{edx} by splitting each row corresponding to a movie belonging to multiple genres;
\item
  \hyperref[appndx_b.movie_map]{\texttt{movie\_map} Object}: A map establishing a correspondence between a movie ID and its title.;
\item
  \hyperref[appndx_b.date_days_map]{\texttt{date\_days\_map} Object}: A map between the rating date and the number of days since the earliest entry in the edx dataset;
\item
  \hyperref[appndx_b.edx_CV]{\texttt{edx\_CV} Object}: A list of objects created from samples made from the \texttt{edx} dataset to use them for the \emph{K-Fold Cross-Validation} in the way explained in \href{https://rafalab.dfci.harvard.edu/dsbook-part-2/ml/resampling-methods.html\#k-fold-cross-validation}{Section 29.6 \emph{Cross validation / K-fold cross validation}} of the \emph{Course Textbook (New Edition)}\autocite{IDS2_29-6_KFCV}.
\end{itemize}

\begin{noteblock}
The \(K\) value for the \emph{K-Fold Cross-Validation} we use in \emph{this Project} is determined by the length of the list composing the \hyperref[appndx_b.edx_CV]{\texttt{edx\_CV} Object}. In our case, the length of the \texttt{edx\_CV} is 5, meaning we use \emph{5-Fold Cross-Validation}.

\end{noteblock}

\newpage

\subsection{Overall Mean Rating (OMR) Model}\label{overall-mean-rating-omr-model}

\hfill\break

Let's begin our analysis by evaluating the simplest model described in \href{https://rafalab.dfci.harvard.edu/dsbook/large-datasets.html\#a-first-model}{Section 33.7.4: \emph{A first model}} of the \emph{Course Textbook (First Edition)}\autocite{IDS_E1_33-7-4}, and then gradually refine it through further research.

\subsubsection{Mathematical Description of the OMR Model}\label{mathematical-description-of-the-omr-model}

\hfill\break

The \emph{Overal Mean Rating} model that assumes the same rating for all movies and users with all the differences explained by random variation would look as follows:

\[
Y_{i,j} = \mu + \varepsilon_{i,j}
\]

with \(\varepsilon_{i,j}\) independent errors sampled from the same distribution centered at 0 and \(\mu\) is the average rating for all movies.

\subsubsection{OMR Model Building}\label{omr-model-building}

\hfill\break

\begin{noteblock}
The complete source code of the \emph{Overall Mean Rating} computation described in this section is available in the \href{https://github.com/AzKurban-edX-DS/Capstone-MovieLens/blob/main/r/src/capstone-movielens.main.R\#L66}{Overall Mean Rating (Naive) Model} section of the \href{https://github.com/AzKurban-edX-DS/Capstone-MovieLens/blob/main/r/src/capstone-movielens.main.R\#L66}{capstone-movielens.main.R} script on \emph{GitHub}.

\end{noteblock}

We know that the estimate that minimizes the RMSE is the least squares estimate of \(\mu\) and, in this case, is the average of all ratings. We calculate this using the following \href{https://github.com/AzKurban-edX-DS/Capstone-MovieLens/blob/main/r/src/capstone-movielens.main.R\#L106}{line of code}:

\begin{Shaded}
\begin{Highlighting}[]
\NormalTok{mu }\OtherTok{\textless{}{-}} \FunctionTok{mean}\NormalTok{(edx}\SpecialCharTok{$}\NormalTok{rating)}
\NormalTok{mu}
\end{Highlighting}
\end{Shaded}

\begin{verbatim}
## [1] 3.512465
\end{verbatim}

If we predict all unknown ratings with \(\hat{\mu}\), we can use the following \href{https://github.com/AzKurban-edX-DS/Capstone-MovieLens/blob/main/r/src/capstone-movielens.main.R\#L108}{line of code} to compute \emph{Mean Squared Errors} (hereafter \emph{MSE} for short) with the use of \emph{K-Fold Cross-Validation}, where the \(K\) is the length of the \hyperref[appndx_b.edx_CV]{\texttt{edx\_CV} Object} (described in detail in \hyperref[appndx_b]{Appendix B: Models Training Datasets}) to which the \emph{K-Fold Cross-Validation} is applied (in \emph{this Project}, we use \(K = 5\)):

\begin{Shaded}
\begin{Highlighting}[]
\NormalTok{mu.MSEs }\OtherTok{\textless{}{-}} \FunctionTok{naive\_model\_MSEs}\NormalTok{(mu)}
\end{Highlighting}
\end{Shaded}

\begin{noteblock}
In the code snippet above, we use the user-defined function \hyperref[func.naive_model_MSEs]{naive\_model\_MSEs} described in section \hyperref[omr-model-helper-functions]{OMR Model: Helper Functions} of \hyperref[appndx_a]{Appendix A}.

\end{noteblock}

\newpage

The following \href{https://github.com/AzKurban-edX-DS/Capstone-MovieLens/blob/main/r/src/capstone-movielens.main.R\#L110}{code snippet} provides a visual representation of the results we have just obtained:

\begin{Shaded}
\begin{Highlighting}[]
\FunctionTok{data.frame}\NormalTok{(}\AttributeTok{fold\_No =} \DecValTok{1}\SpecialCharTok{:}\DecValTok{5}\NormalTok{, }\AttributeTok{MSE =}\NormalTok{ mu.MSEs) }\SpecialCharTok{|\textgreater{}}
  \FunctionTok{data.plot}\NormalTok{(}\AttributeTok{title =} \StringTok{"MSE resuls of the 5{-}fold CV method applied to the Overall Mean Rating Model"}\NormalTok{,}
              \AttributeTok{xname =} \StringTok{"fold\_No"}\NormalTok{, }
              \AttributeTok{yname =} \StringTok{"MSE"}\NormalTok{)}
\end{Highlighting}
\end{Shaded}

\pandocbounded{\includegraphics[keepaspectratio]{capstone-movielens.report_files/figure-latex/main-script@l110-1.pdf}}

\begin{noteblock}
In the code snippet above, we use the custom data visualization function \hyperref[func.data.plot]{data.plot} described in section \hyperref[data-visualization-functions]{Data Visualization Functions} of \hyperref[appndx_a]{Appendix A}.

\end{noteblock}

We can now calculate the \emph{Root Mean Squared Error (RMSE)} for the current model as a \emph{square root} of the average of the \emph{MSE} values we obtained above, using the following \href{https://github.com/AzKurban-edX-DS/Capstone-MovieLens/blob/main/r/src/capstone-movielens.main.R\#L117}{line of code}:

\begin{Shaded}
\begin{Highlighting}[]
\NormalTok{mu.RMSE }\OtherTok{\textless{}{-}} \FunctionTok{sqrt}\NormalTok{(}\FunctionTok{mean}\NormalTok{(mu.MSEs))}
\NormalTok{mu.RMSE}
\end{Highlighting}
\end{Shaded}

\begin{verbatim}
## [1] 1.060346
\end{verbatim}

\newpage

\subsubsection{OMR Value Is the Best for the Current Model}\label{omr-value-is-the-best-for-the-current-model}

\hfill\break

If we plug in any other number, we will get a higher \emph{RMSE}. Let's prove this with a small experiment using the following \href{https://github.com/AzKurban-edX-DS/Capstone-MovieLens/blob/main/r/src/capstone-movielens.main.R\#L155}{code snippet}:

\begin{Shaded}
\begin{Highlighting}[]
\NormalTok{  deviation }\OtherTok{\textless{}{-}} \FunctionTok{seq}\NormalTok{(}\DecValTok{0}\NormalTok{, }\DecValTok{6}\NormalTok{, }\FloatTok{0.1}\NormalTok{) }\SpecialCharTok{{-}} \DecValTok{3}

\NormalTok{  deviation.RMSE }\OtherTok{\textless{}{-}} \FunctionTok{sapply}\NormalTok{(deviation, }\ControlFlowTok{function}\NormalTok{(delta)\{}
    \FunctionTok{naive\_model\_RMSE}\NormalTok{(mu }\SpecialCharTok{+}\NormalTok{ delta)}
\NormalTok{  \})}
\end{Highlighting}
\end{Shaded}

\begin{noteblock}
In the code snippet above, we use the function \hyperref[func.naive_model_RMSE]{naive\_model\_RMSE} described in section \hyperref[omr-model-helper-functions]{OMR Model: Helper Functions} of \hyperref[appndx_a]{Appendix A} to calculate \emph{RMSEs} for the sequence of the deviation values.

\end{noteblock}

\newpage

Let's quickly analyze the resulting \texttt{deviation.RMSE} vector we have just got. The following \href{https://github.com/AzKurban-edX-DS/Capstone-MovieLens/blob/main/r/src/capstone-movielens.main.R\#L189}{data visualization code} will help us with this:

\begin{Shaded}
\begin{Highlighting}[]
\FunctionTok{data.frame}\NormalTok{(}\AttributeTok{delta =}\NormalTok{ deviation, }
           \AttributeTok{delta.RMSE =}\NormalTok{ deviation.RMSE) }\SpecialCharTok{|\textgreater{}} 
\FunctionTok{data.plot}\NormalTok{(}\AttributeTok{title =} \FunctionTok{TeX}\NormalTok{(r}\StringTok{\textquotesingle{}[RMSE as a function of deviation ($\textbackslash{}delta$) from the Overall Mean Rating ($\textbackslash{}hat\{mu\}$)]\textquotesingle{}}\NormalTok{),}
              \AttributeTok{xname =} \StringTok{"delta"}\NormalTok{, }
              \AttributeTok{yname =} \StringTok{"delta.RMSE"}\NormalTok{, }
              \AttributeTok{xlabel =} \FunctionTok{TeX}\NormalTok{(r}\StringTok{\textquotesingle{}[$\textbackslash{}delta$]\textquotesingle{}}\NormalTok{), }
              \AttributeTok{ylabel =} \StringTok{"RMSE"}\NormalTok{)}
\end{Highlighting}
\end{Shaded}

\pandocbounded{\includegraphics[keepaspectratio]{capstone-movielens.report_files/figure-latex/main-script@l189-1.pdf}}

\begin{noteblock}
In the code snippet above, we use the custom data visualization function \hyperref[func.data.plot]{data.plot} described in section \hyperref[data-visualization-functions]{Data Visualization Functions} of \hyperref[appndx_a]{Appendix A}.

\end{noteblock}

\newpage

\phantomsection\label{chunk.main_script.l177m}
The \href{https://github.com/AzKurban-edX-DS/Capstone-MovieLens/blob/main/r/src/capstone-movielens.main.R\#L177}{piece of code} below prints a quick recap of the results:

\begin{Shaded}
\begin{Highlighting}[]
\NormalTok{which\_min\_deviation }\OtherTok{\textless{}{-}}\NormalTok{ deviation[}\FunctionTok{which.min}\NormalTok{(deviation.RMSE)]}
\NormalTok{min\_rmse }\OtherTok{=} \FunctionTok{min}\NormalTok{(deviation.RMSE)}

\FunctionTok{print\_log1}\NormalTok{(}\StringTok{"Minimum RMSE is achieved when the deviation from the mean is: \%1"}\NormalTok{,}
\NormalTok{         which\_min\_deviation)}
\end{Highlighting}
\end{Shaded}

\begin{verbatim}
## Minimum RMSE is achieved when the deviation from the mean is: 0
\end{verbatim}

\begin{Shaded}
\begin{Highlighting}[]
\FunctionTok{print\_log1}\NormalTok{(}\StringTok{"Is the previously computed RMSE the best for the current model: \%1"}\NormalTok{,}
\NormalTok{         mu.RMSE }\SpecialCharTok{==}\NormalTok{ min\_rmse)}
\end{Highlighting}
\end{Shaded}

\begin{verbatim}
## Is the previously computed RMSE the best for the current model: TRUE
\end{verbatim}

\begin{noteblock}
To print the log, we use the function \hyperref[func.print_log1]{print\_log1} described in section \hyperref[appndx_a.CHF.logging-funcs]{Logging Functions} of \hyperref[appndx_a]{Appendix A}.

\end{noteblock}

Finally, we add the \emph{RMSE} value calculated for the \emph{OMR Model} to our \emph{Result Table} and print the table using the following \href{https://github.com/AzKurban-edX-DS/Capstone-MovieLens/blob/main/r/src/capstone-movielens.main.R\#L200}{piece of code}:

\begin{Shaded}
\begin{Highlighting}[]
\NormalTok{RMSEs.ResultTibble.OMR }\OtherTok{\textless{}{-}}\NormalTok{ RMSEs.ResultTibble }\SpecialCharTok{|\textgreater{}} 
  \FunctionTok{RMSEs.AddRow}\NormalTok{(}\StringTok{"OMR Model"}\NormalTok{, }
\NormalTok{               mu.RMSE,}
               \AttributeTok{comment =} \StringTok{"Overall Mean Rating (OMR) Model"}\NormalTok{)}
\end{Highlighting}
\end{Shaded}

\begin{Shaded}
\begin{Highlighting}[]
\FunctionTok{RMSE\_kable}\NormalTok{(RMSEs.ResultTibble.OMR)}
\end{Highlighting}
\end{Shaded}

\begin{tabular}{>{\raggedright\arraybackslash}p{15em}>{\centering\arraybackslash}p{5em}>{\raggedright\arraybackslash}p{30em}}
\toprule
\textbf{Method} & \textbf{RMSE} & \textbf{Comment}\\
\midrule
Project Objective & 0.864900 & \\
OMR Model & 1.060346 & Overall Mean Rating (OMR) Model\\
\bottomrule
\end{tabular}

\begin{noteblock}
In the code snippet above, we use the \hyperref[func.RMSEs.AddRow]{RMSEs.AddRow} and \hyperref[func.RMSE_kable]{RMSE\_kable} functions described in section \hyperref[result-rmses-tibble-functions]{Result RMSEs Tibble Functions} of \hyperref[appndx_a]{Appendix A}.

\end{noteblock}

To win the grand prize of \$1,000,000, a participating team had to get an RMSE of at least 0.8563\autocite{BigChaosSln}. So we can definitely do better!\autocite{IDS_E1_33-7-4}

\newpage

\subsection{User Effect (UE) Model}\label{user-effect-ue-model}

\hfill\break

\subsubsection{User Effect Analysis}\label{sec.user_effect_analysis}

\hfill\break

To improve our model let's now take into consideration user effects as explained in \href{https://rafalab.dfci.harvard.edu/dsbook-part-2/highdim/regularization.html\#user-effects}{Section 24.1 \emph{Case study: recommendation systems / User effects}} of the \emph{Course Textbook (New Edition)}\autocite{IDS2_24-1_UE}.

Let's start our analysis by calculating the average rating for each user using the following \href{https://github.com/AzKurban-edX-DS/Capstone-MovieLens/blob/main/r/src/capstone-movielens.main.R\#L231}{code snippet}:

\begin{Shaded}
\begin{Highlighting}[]
  \FunctionTok{put\_log}\NormalTok{(}\StringTok{"Computing Average Ratings per User (User Mean Ratings)..."}\NormalTok{)}
\NormalTok{  user.mean\_ratings }\OtherTok{\textless{}{-}} \FunctionTok{rowMeans}\NormalTok{(edx.mx, }\AttributeTok{na.rm =} \ConstantTok{TRUE}\NormalTok{)}
\NormalTok{  user\_ratings.n }\OtherTok{\textless{}{-}} \FunctionTok{rowSums}\NormalTok{(}\SpecialCharTok{!}\FunctionTok{is.na}\NormalTok{(edx.mx))}

\NormalTok{  edx.user\_mean\_ratings }\OtherTok{\textless{}{-}} 
    \FunctionTok{data.frame}\NormalTok{(}\AttributeTok{userId =} \FunctionTok{names}\NormalTok{(user.mean\_ratings), }
               \AttributeTok{mean\_rating =}\NormalTok{ user.mean\_ratings,}
               \AttributeTok{n =}\NormalTok{ user\_ratings.n)}
  
  \FunctionTok{put\_log}\NormalTok{(}\StringTok{"User Mean Ratings have been computed."}\NormalTok{)}
\end{Highlighting}
\end{Shaded}

\begin{Shaded}
\begin{Highlighting}[]
\FunctionTok{str}\NormalTok{(edx.user\_mean\_ratings)}
\end{Highlighting}
\end{Shaded}

\begin{verbatim}
## 'data.frame':    69878 obs. of  3 variables:
##  $ userId     : chr  "1" "2" "3" "4" ...
##  $ mean_rating: num  5 3.29 3.94 4.06 3.92 ...
##  $ n          : num  19 17 31 35 74 39 96 727 21 112 ...
\end{verbatim}

\begin{noteblock}
In the code snippet above, we use the \hyperref[appndx_b.edx.mx]{\texttt{edx.mx} Matrix Object} (described in detail in \hyperref[appndx_b]{Appendix B: Models Training Datasets}), which, in fact, is the \texttt{edx} dataset converted to a matrix as described earlier in section \hyperref[preparing-train-and-set-datasets]{Preparing train and set datasets}. \newline
We also use the user-defined helper function \hyperref[func.put_log]{put\_log} described in section \hyperref[appndx_a.CHF.logging-funcs]{Logging Functions} of \hyperref[appndx_a]{Appendix A}.

\end{noteblock}

The \href{https://github.com/AzKurban-edX-DS/Capstone-MovieLens/blob/main/r/src/capstone-movielens.main.R\#L263}{code snippet} below visualizes the frequency of the users' average ratings. As the author of the \emph{Course Textbook} shows, such a visualization allows us to see that there is substantial variability in the average ratings across users\autocite{IDS2_24-1_UE}:

\begin{Shaded}
\begin{Highlighting}[]
\FunctionTok{hist}\NormalTok{(edx.user\_mean\_ratings}\SpecialCharTok{$}\NormalTok{mean\_rating,}
     \AttributeTok{xlab =} \StringTok{"User Average Rating"}\NormalTok{,}
     \AttributeTok{main =} \StringTok{"Histogram of User Average Rating"}\NormalTok{,}
     \AttributeTok{nclass =} \DecValTok{30}\NormalTok{)}
\end{Highlighting}
\end{Shaded}

\pandocbounded{\includegraphics[keepaspectratio]{capstone-movielens.report_files/figure-latex/main-script@l263-1.pdf}}

\newpage

\subsubsection{Mathematical Description of the UE Model}\label{mathematical-description-of-the-ue-model}

\hfill\break

Following the further explanation provided in \href{https://rafalab.dfci.harvard.edu/dsbook-part-2/highdim/regularization.html\#user-effects}{Section 24.1} of the \emph{Course Textbook} (mentioned above in the \hyperref[sec.user_effect_analysis]{previous Section}), to account for this variability, we use a linear model with a \emph{treatment effect} \(\alpha_i\) for each user. The sum \(\mu+\alpha_i\) can be interpreted as the typical rating user \(i\) gives to movies. So we write the model as follows:
\[
Y_{i,j} = \mu + \alpha_i + \varepsilon_{i,j}
\]

Statistics textbooks refer to the \(\alpha\)s as treatment effects. In the Netflix challenge papers, they refer to them as \emph{bias}\autocite{IDS2_24-1_UE,MFT_RS}.

\subsubsection{UE Model Building}\label{ue-model-building}

\hfill\break

\begin{noteblock}
The complete source code of the \emph{User Effect} computation described in this section is available in the \href{https://github.com/AzKurban-edX-DS/Capstone-MovieLens/blob/main/r/src/capstone-movielens.main.R\#L210}{User Effect (UE) Model} section of the \href{https://github.com/AzKurban-edX-DS/Capstone-MovieLens/blob/main/r/src/capstone-movielens.main.R\#L210}{capstone-movielens.main.R} script on \emph{GitHub}.

\end{noteblock}

As it is stated in \href{https://rafalab.dfci.harvard.edu/dsbook/large-datasets.html\#user-effects}{Section 33.7.6: \emph{User effects}} of the \emph{Course Textbook (First Edition)}, it can be shown that the least squares estimate \(\hat{\alpha}_i\) is just the average of \(y_{i,j} - \hat{\mu}\) for each user \(i\). So we can compute them this way\autocite{IDS_E1_33-7-6}:

\begin{Shaded}
\begin{Highlighting}[]
\NormalTok{a }\OtherTok{\textless{}{-}} \FunctionTok{rowMeans}\NormalTok{(y }\SpecialCharTok{{-}}\NormalTok{ mu, }\AttributeTok{na.rm =} \ConstantTok{TRUE}\NormalTok{)}
\end{Highlighting}
\end{Shaded}

These considerations allow us to compute a \emph{User Mean Ratings} using the following \href{https://github.com/AzKurban-edX-DS/Capstone-MovieLens/blob/main/r/src/capstone-movielens.main.R\#L231}{code snippet}:

\begin{Shaded}
\begin{Highlighting}[]
\FunctionTok{put\_log}\NormalTok{(}\StringTok{"Computing Average Ratings per User (User Mean Ratings)..."}\NormalTok{)}
\NormalTok{user.mean\_ratings }\OtherTok{\textless{}{-}} \FunctionTok{rowMeans}\NormalTok{(edx.mx, }\AttributeTok{na.rm =} \ConstantTok{TRUE}\NormalTok{)}
\NormalTok{user\_ratings.n }\OtherTok{\textless{}{-}} \FunctionTok{rowSums}\NormalTok{(}\SpecialCharTok{!}\FunctionTok{is.na}\NormalTok{(edx.mx))}
  
\NormalTok{edx.user\_mean\_ratings }\OtherTok{\textless{}{-}} 
  \FunctionTok{data.frame}\NormalTok{(}\AttributeTok{userId =} \FunctionTok{names}\NormalTok{(user.mean\_ratings), }
             \AttributeTok{mean\_rating =}\NormalTok{ user.mean\_ratings,}
             \AttributeTok{n =}\NormalTok{ user\_ratings.n)}
  
\FunctionTok{put\_log}\NormalTok{(}\StringTok{"User Mean Ratings have been computed."}\NormalTok{)}
\end{Highlighting}
\end{Shaded}

\begin{Shaded}
\begin{Highlighting}[]
\FunctionTok{str}\NormalTok{(edx.user\_mean\_ratings)}
\end{Highlighting}
\end{Shaded}

\begin{verbatim}
## 'data.frame':    69878 obs. of  3 variables:
##  $ userId     : chr  "1" "2" "3" "4" ...
##  $ mean_rating: num  5 3.29 3.94 4.06 3.92 ...
##  $ n          : num  19 17 31 35 74 39 96 727 21 112 ...
\end{verbatim}

\newpage

And then we compute a \emph{User Effect} using the following \href{https://github.com/AzKurban-edX-DS/Capstone-MovieLens/blob/main/r/src/capstone-movielens.main.R\#L291}{piece of code}:

\begin{Shaded}
\begin{Highlighting}[]
\NormalTok{edx.user\_effect }\OtherTok{\textless{}{-}}\NormalTok{ edx.user\_mean\_ratings }\SpecialCharTok{|\textgreater{}}
  \FunctionTok{mutate}\NormalTok{(}\AttributeTok{userId =} \FunctionTok{as.integer}\NormalTok{(userId),}
         \AttributeTok{a =}\NormalTok{ mean\_rating }\SpecialCharTok{{-}}\NormalTok{ mu)}
\end{Highlighting}
\end{Shaded}

\begin{Shaded}
\begin{Highlighting}[]
\FunctionTok{str}\NormalTok{(edx.user\_effect)}
\end{Highlighting}
\end{Shaded}

\begin{verbatim}
## 'data.frame':    69878 obs. of  4 variables:
##  $ userId     : int  1 2 3 4 5 6 7 8 9 10 ...
##  $ mean_rating: num  5 3.29 3.94 4.06 3.92 ...
##  $ n          : num  19 17 31 35 74 39 96 727 21 112 ...
##  $ a          : num  1.488 -0.218 0.423 0.545 0.406 ...
\end{verbatim}

The following \href{https://github.com/AzKurban-edX-DS/Capstone-MovieLens/blob/main/r/src/capstone-movielens.main.R\#L336}{code snippet} plots a histogram that provides a visual representation of the variability of the \emph{User Effect} across users in the \emph{data frame object} we have just obtained:

\begin{Shaded}
\begin{Highlighting}[]
\FunctionTok{par}\NormalTok{(}\AttributeTok{cex =} \FloatTok{0.7}\NormalTok{)}
\FunctionTok{hist}\NormalTok{(edx.user\_effect}\SpecialCharTok{$}\NormalTok{a, }\DecValTok{30}\NormalTok{, }\AttributeTok{xlab =} \FunctionTok{TeX}\NormalTok{(r}\StringTok{\textquotesingle{}[$\textbackslash{}hat\{alpha\}\_\{i\}$]\textquotesingle{}}\NormalTok{),}
     \AttributeTok{main =} \FunctionTok{TeX}\NormalTok{(r}\StringTok{\textquotesingle{}[Histogram of $\textbackslash{}hat\{alpha\}\_\{i\}$]\textquotesingle{}}\NormalTok{))}
\end{Highlighting}
\end{Shaded}

\pandocbounded{\includegraphics[keepaspectratio]{capstone-movielens.report_files/figure-latex/main-script@l336-1.pdf}}

Now, we are ready to compute the \emph{Mean Squared Errors (MSEs)} for the samples used in \emph{K-Fold Cross-Validation} (additionally using the \hyperref[func.clamp]{clamp} helper function described in section \hyperref[appndx_a.CHF.utility_funcs]{Utility Functions} of \hyperref[appndx_a]{Appendix A}).

\begin{noteblock}
The \(K\) value for the \emph{K-Fold Cross-Validation} is determined by the length of the \hyperref[appndx_b.edx_CV]{\texttt{edx\_CV} Object} described below in \hyperref[appndx_b]{Appendix B: Models Training Datasets} (in \emph{this Project}, we use \(K = 5\)).

\end{noteblock}

The \href{https://github.com/AzKurban-edX-DS/Capstone-MovieLens/blob/main/r/src/capstone-movielens.main.R\#L345}{code snippet} that performs this operation is shown below:

\begin{Shaded}
\begin{Highlighting}[]
\FunctionTok{put\_log}\NormalTok{(}\StringTok{"Computing the RMSE taking into account user effects..."}\NormalTok{)}
\NormalTok{edx.user\_effect.MSEs }\OtherTok{\textless{}{-}} \FunctionTok{sapply}\NormalTok{(edx\_CV, }\ControlFlowTok{function}\NormalTok{(cv\_fold\_dat)\{}
\NormalTok{  cv\_fold\_dat}\SpecialCharTok{$}\NormalTok{validation\_set }\SpecialCharTok{|\textgreater{}}
    \FunctionTok{left\_join}\NormalTok{(edx.user\_effect, }\AttributeTok{by =} \StringTok{"userId"}\NormalTok{) }\SpecialCharTok{|\textgreater{}}
    \FunctionTok{mutate}\NormalTok{(}\AttributeTok{resid =}\NormalTok{ rating }\SpecialCharTok{{-}} \FunctionTok{clamp}\NormalTok{(mu }\SpecialCharTok{+}\NormalTok{ a)) }\SpecialCharTok{|\textgreater{}} 
    \FunctionTok{pull}\NormalTok{(resid) }\SpecialCharTok{|\textgreater{}} \FunctionTok{mse}\NormalTok{()}
\NormalTok{\})}
\end{Highlighting}
\end{Shaded}

\begin{noteblock}
In the code snippet above, we also use the user-defined function \hyperref[func.mse]{mse} described in section \hyperref[root-mean-squared-error-calculation]{(Root) Mean Squared Error Calculation} of \hyperref[appndx_a]{Appendix A}.

\end{noteblock}

The following \href{https://github.com/AzKurban-edX-DS/Capstone-MovieLens/blob/main/r/src/capstone-movielens.main.R\#L355}{code snippet} visualizes the results obtained above:

\begin{Shaded}
\begin{Highlighting}[]
\FunctionTok{data.frame}\NormalTok{(}\AttributeTok{fold\_No =} \DecValTok{1}\SpecialCharTok{:}\DecValTok{5}\NormalTok{, }\AttributeTok{MSE =}\NormalTok{ edx.user\_effect.MSEs) }\SpecialCharTok{|\textgreater{}}
  \FunctionTok{data.plot}\NormalTok{(}\AttributeTok{title =} \StringTok{"MSE resuls of the 5{-}fold CV method applied to the User Effect Model"}\NormalTok{,}
            \AttributeTok{xname =} \StringTok{"fold\_No"}\NormalTok{, }
            \AttributeTok{yname =} \StringTok{"MSE"}\NormalTok{)}
\end{Highlighting}
\end{Shaded}

\pandocbounded{\includegraphics[keepaspectratio]{capstone-movielens.report_files/figure-latex/main-script@l355-1.pdf}}

\begin{noteblock}
In the code snippet above, we use the custom data visualization function \hyperref[func.data.plot]{data.plot} described in section \hyperref[data-visualization-functions]{Data Visualization Functions} of \hyperref[appndx_a]{Appendix A}.

\end{noteblock}

\newpage

We can now calculate the \emph{RMSE} for the current model as a \emph{square root} of the average of the \emph{MSE} values we obtained above, using the following \href{https://github.com/AzKurban-edX-DS/Capstone-MovieLens/blob/main/r/src/capstone-movielens.main.R\#L364}{line of code}:

\begin{Shaded}
\begin{Highlighting}[]
\NormalTok{edx.user\_effect.RMSE }\OtherTok{\textless{}{-}} \FunctionTok{sqrt}\NormalTok{(}\FunctionTok{mean}\NormalTok{(edx.user\_effect.MSEs))}
\end{Highlighting}
\end{Shaded}

Finally, we add the \emph{RMSE} value calculated for the \emph{UE Model} to our \emph{Result Table} and print the table using the following \href{https://github.com/AzKurban-edX-DS/Capstone-MovieLens/blob/main/r/src/capstone-movielens.main.R\#L372}{code snippet}:

\begin{Shaded}
\begin{Highlighting}[]
\NormalTok{RMSEs.ResultTibble.UE }\OtherTok{\textless{}{-}}\NormalTok{ RMSEs.ResultTibble.OMR }\SpecialCharTok{|\textgreater{}} 
  \FunctionTok{RMSEs.AddRow}\NormalTok{(}\StringTok{"UE Model"}\NormalTok{, }
\NormalTok{               edx.user\_effect.RMSE,}
               \AttributeTok{comment =} \StringTok{"User Effect (UE) Model"}\NormalTok{)}
\end{Highlighting}
\end{Shaded}

\begin{Shaded}
\begin{Highlighting}[]
\FunctionTok{RMSE\_kable}\NormalTok{(RMSEs.ResultTibble.UE)}
\end{Highlighting}
\end{Shaded}

\begin{tabular}{>{\raggedright\arraybackslash}p{15em}>{\centering\arraybackslash}p{5em}>{\raggedright\arraybackslash}p{30em}}
\toprule
\textbf{Method} & \textbf{RMSE} & \textbf{Comment}\\
\midrule
Project Objective & 0.8649000 & \\
OMR Model & 1.0603462 & Overall Mean Rating (OMR) Model\\
UE Model & 0.9697962 & User Effect (UE) Model\\
\bottomrule
\end{tabular}

\begin{noteblock}
In the code snippet above, we use the \hyperref[func.RMSEs.AddRow]{RMSEs.AddRow} and \hyperref[func.RMSE_kable]{RMSE\_kable} functions described in section \hyperref[result-rmses-tibble-functions]{Result RMSEs Tibble Functions} of \hyperref[appndx_a]{Appendix A}.

\end{noteblock}

\newpage

\subsection{User+Movie Effect (UME) Model}\label{usermovie-effect-ume-model}

\hfill\break

\begin{noteblock}
The complete source code of the \emph{User+Movie Effect} (hereafter \emph{UM Effect} or \emph{UME} for short) computation described in this section is available in the \href{https://github.com/AzKurban-edX-DS/Capstone-MovieLens/blob/main/r/src/capstone-movielens.main.R\#L382}{User+Movie Effect (UME) Model} section of the \href{https://github.com/AzKurban-edX-DS/Capstone-MovieLens/blob/main/r/src/capstone-movielens.main.R\#L382}{capstone-movielens.main.R} script on \emph{GitHub}.

\end{noteblock}

\subsubsection{Movie Effect Analysis}\label{movie-effect-analysis}

\hfill\break

\paragraph{Movies' Popularity}\label{movies-popularity}

\hfill\break

\phantomsection\label{textbook.24.1.movie_effects}
In \href{https://rafalab.dfci.harvard.edu/dsbook-part-2/highdim/regularization.html\#sec-movie-effects}{Section 24.1 \emph{Case study: recommendation systems / Movie effects}} of the \emph{Course Textbook (New Edition)}, the author draws our attention to the fact that some movies are generally rated higher than others\autocite{IDS2_24-1_ME}.

To prove this fact, we can find out the movies with the highest number of ratings using the following \href{https://github.com/AzKurban-edX-DS/Capstone-MovieLens/blob/main/r/src/capstone-movielens.main.R\#L413}{code snippet}:

\begin{Shaded}
\begin{Highlighting}[]
\NormalTok{edx.ordered\_movie\_ratings }\OtherTok{\textless{}{-}}\NormalTok{ edx }\SpecialCharTok{|\textgreater{}} \FunctionTok{group\_by}\NormalTok{(movieId, title) }\SpecialCharTok{|\textgreater{}}
  \FunctionTok{summarize}\NormalTok{(}\AttributeTok{number\_of\_ratings =} \FunctionTok{n}\NormalTok{()) }\SpecialCharTok{|\textgreater{}}
  \FunctionTok{arrange}\NormalTok{(}\FunctionTok{desc}\NormalTok{(number\_of\_ratings))}
\end{Highlighting}
\end{Shaded}

\begin{Shaded}
\begin{Highlighting}[]
\FunctionTok{print}\NormalTok{(}\FunctionTok{head}\NormalTok{(edx.ordered\_movie\_ratings))}
\end{Highlighting}
\end{Shaded}

\begin{verbatim}
## # A tibble: 6 x 3
## # Groups:   movieId [6]
##   movieId title                            number_of_ratings
##     <int> <chr>                                        <int>
## 1     296 Pulp Fiction (1994)                          31362
## 2     356 Forrest Gump (1994)                          31079
## 3     593 Silence of the Lambs, The (1991)             30382
## 4     480 Jurassic Park (1993)                         29360
## 5     318 Shawshank Redemption, The (1994)             28015
## 6     110 Braveheart (1995)                            26212
\end{verbatim}

\newpage

Now, we can figure out the most given ratings in order from most to least. The \href{https://github.com/AzKurban-edX-DS/Capstone-MovieLens/blob/main/r/src/capstone-movielens.main.R\#L422}{code snippet} below performs this operation:

\begin{Shaded}
\begin{Highlighting}[]
\NormalTok{edx.rating\_groups }\OtherTok{\textless{}{-}}\NormalTok{ edx }\SpecialCharTok{|\textgreater{}}  \FunctionTok{group\_by}\NormalTok{(rating) }\SpecialCharTok{|\textgreater{}}
     \FunctionTok{summarise}\NormalTok{(}\AttributeTok{count =} \FunctionTok{n}\NormalTok{()) }\SpecialCharTok{|\textgreater{}}
     \FunctionTok{arrange}\NormalTok{(}\FunctionTok{desc}\NormalTok{(count))}
\end{Highlighting}
\end{Shaded}

\begin{Shaded}
\begin{Highlighting}[]
\FunctionTok{print}\NormalTok{(edx.rating\_groups)}
\end{Highlighting}
\end{Shaded}

\begin{verbatim}
## # A tibble: 10 x 2
##    rating   count
##     <dbl>   <int>
##  1    4   2588430
##  2    3   2121240
##  3    5   1390114
##  4    3.5  791624
##  5    2    711422
##  6    4.5  526736
##  7    1    345679
##  8    2.5  333010
##  9    1.5  106426
## 10    0.5   85374
\end{verbatim}

\newpage

\paragraph{Rating Distribution}\label{rating-distribution}

\hfill\break

The following \href{https://github.com/AzKurban-edX-DS/Capstone-MovieLens/blob/main/r/src/capstone-movielens.main.R\#L430}{line of code} allows us to summarize that in general, half-star ratings are less common than whole-star ratings (e.g., there are fewer ratings of 3.5 than there are ratings of 3 or 4, etc.):

\begin{Shaded}
\begin{Highlighting}[]
\FunctionTok{print}\NormalTok{(edx }\SpecialCharTok{|\textgreater{}} \FunctionTok{group\_by}\NormalTok{(rating) }\SpecialCharTok{|\textgreater{}} \FunctionTok{summarize}\NormalTok{(}\AttributeTok{count =} \FunctionTok{n}\NormalTok{()))}
\end{Highlighting}
\end{Shaded}

\begin{verbatim}
## # A tibble: 10 x 2
##    rating   count
##     <dbl>   <int>
##  1    0.5   85374
##  2    1    345679
##  3    1.5  106426
##  4    2    711422
##  5    2.5  333010
##  6    3   2121240
##  7    3.5  791624
##  8    4   2588430
##  9    4.5  526736
## 10    5   1390114
\end{verbatim}

We can give even more visibility for the \emph{rating distribution} by plotting the data for the \emph{rating groups} (the \texttt{edx.rating\_groups} object) we obtained above using the following \href{https://github.com/AzKurban-edX-DS/Capstone-MovieLens/blob/main/r/src/capstone-movielens.main.R\#L433}{code snippet}:

\begin{Shaded}
\begin{Highlighting}[]
\NormalTok{edx.rating\_groups }\SpecialCharTok{|\textgreater{}}
  \FunctionTok{ggplot}\NormalTok{(}\FunctionTok{aes}\NormalTok{(}\AttributeTok{x =}\NormalTok{ rating, }\AttributeTok{y =}\NormalTok{ count)) }\SpecialCharTok{+}
  \FunctionTok{geom\_line}\NormalTok{() }
\end{Highlighting}
\end{Shaded}

\pandocbounded{\includegraphics[keepaspectratio]{capstone-movielens.report_files/figure-latex/main-script@l433-1.pdf}}

\newpage

\subsubsection{Mathematical Description of the UME Model}\label{mathematical-description-of-the-ume-model}

\hfill\break

As explained in the \href{https://rafalab.dfci.harvard.edu/dsbook-part-2/highdim/regularization.html\#sec-movie-effects}{Section 24.1, Subsection \emph{Movie effects}} of the \emph{Textbook} (mentioned earlier in section \hyperref[movies-popularity]{Movies' Popularity}), for the \emph{Movie effects} we can use a linear model with a \emph{treatment effect} \(\beta_j\) for each movie, which can be interpreted as the movie effect, or the difference between the average rating for movie \(j\) and the overall average \(\mu\)\autocite{IDS2_24-1_ME}:
\[
Y_{i,j} = \mu + \alpha_i + \beta_j +\varepsilon_{i,j}
\]

The author then gives us an idea of how to use an approximation by first computing the least square estimate \(\hat{\mu}\) and \(\hat{\alpha}_i\), and then estimating \(\hat{\beta}_j\) as the average of the residuals \(y_{i,j} - \hat{\mu} - \hat{\alpha}_i\).

Therefore, we will use this method in the \emph{next Section}.

\subsubsection{UME Model Building}\label{ume-model-building}

\hfill\break

\begin{noteblock}
The complete source code of the \emph{User+Movie (UM) Effect} computation described in this section is available in the \href{https://github.com/AzKurban-edX-DS/Capstone-MovieLens/blob/main/r/src/capstone-movielens.main.R\#L456}{UME Model Building} section of the \href{https://github.com/AzKurban-edX-DS/Capstone-MovieLens/blob/main/r/src/capstone-movielens.main.R\#L456}{capstone-movielens.main.R} script on \emph{GitHub}.

\end{noteblock}

Below is the \href{https://github.com/AzKurban-edX-DS/Capstone-MovieLens/blob/main/r/src/capstone-movielens.main.R\#L472}{line of code} that trains our model with the use of \emph{K-Fold Cross-Validation}, where the \(K\) is the length of the \hyperref[appndx_b.edx_CV]{\texttt{edx\_CV} Object} (described in detail in \hyperref[appndx_b]{Appendix B: Models Training Datasets}) to which the \emph{K-Fold Cross-Validation} is applied (in \emph{this Project}, we use \(K = 5\)):

\begin{Shaded}
\begin{Highlighting}[]
\NormalTok{  cv.UM\_effect }\OtherTok{\textless{}{-}} \FunctionTok{train\_user\_movie\_effect.cv}\NormalTok{()}
\end{Highlighting}
\end{Shaded}

\begin{noteblock}
In the code snippet above we use the \hyperref[func.train_user_movie_effect.cv]{train\_user\_movie\_effect.cv} helper function described in section \hyperref[ume-model-utility-functions]{UME Model: Utility Functions} of \hyperref[appndx_a]{Appendix A}.

\end{noteblock}

\begin{Shaded}
\begin{Highlighting}[]
\FunctionTok{str}\NormalTok{(cv.UM\_effect)}
\end{Highlighting}
\end{Shaded}

\begin{verbatim}
## tibble [10,677 x 3] (S3: tbl_df/tbl/data.frame)
##  $ movieId: int [1:10677] 1 2 3 4 5 6 7 8 9 10 ...
##  $ b      : num [1:10677] 0.335 -0.306 -0.365 -0.598 -0.444 ...
##  $ n      : num [1:10677] 18907 8593 5574 1253 5065 ...
\end{verbatim}

The following \href{https://github.com/AzKurban-edX-DS/Capstone-MovieLens/blob/main/r/src/capstone-movielens.main.R\#L507}{code snippet} plots a histogram that provides a visual representation of the variability of the \emph{UM Effect} across movies in the \emph{data frame object} we have just obtained:

\begin{Shaded}
\begin{Highlighting}[]
\FunctionTok{par}\NormalTok{(}\AttributeTok{cex =} \FloatTok{0.7}\NormalTok{)}
\FunctionTok{hist}\NormalTok{(cv.UM\_effect}\SpecialCharTok{$}\NormalTok{b, }\DecValTok{30}\NormalTok{, }\AttributeTok{xlab =} \FunctionTok{TeX}\NormalTok{(r}\StringTok{\textquotesingle{}[$\textbackslash{}hat\{beta\}\_\{j\}$)]\textquotesingle{}}\NormalTok{),}
     \AttributeTok{main =} \FunctionTok{TeX}\NormalTok{(r}\StringTok{\textquotesingle{}[Histogram of $\textbackslash{}hat\{beta\}\_\{j\}$]\textquotesingle{}}\NormalTok{))}
\end{Highlighting}
\end{Shaded}

\pandocbounded{\includegraphics[keepaspectratio]{capstone-movielens.report_files/figure-latex/main-script@l507-1.pdf}}

We can now construct predictors and calculate the \emph{RMSE} score.

As we already outlined above, to accomplish such tasks, we use \emph{K-Fold Cross-Validation}, where the \(K\) is the length of the \hyperref[appndx_b.edx_CV]{\texttt{edx\_CV} Object} (described in detail in \hyperref[appndx_b]{Appendix B: Models Training Datasets}).

\begin{noteblock}
In \emph{this Project}, we use \(K = 5\).

\end{noteblock}

The following \href{https://github.com/AzKurban-edX-DS/Capstone-MovieLens/blob/main/r/src/capstone-movielens.main.R\#L513}{line of code} validates our model by computing the \emph{RMSE} score for the predictors constructed based on the \emph{UM Effect}:

\begin{Shaded}
\begin{Highlighting}[]
\NormalTok{cv.UM\_effect.RMSE }\OtherTok{\textless{}{-}} \FunctionTok{calc\_user\_movie\_effect\_RMSE.cv}\NormalTok{(cv.UM\_effect)}
\end{Highlighting}
\end{Shaded}

\begin{Shaded}
\begin{Highlighting}[]
\NormalTok{cv.UM\_effect.RMSE}
\end{Highlighting}
\end{Shaded}

\begin{verbatim}
## [1] 0.8732081
\end{verbatim}

\begin{noteblock}
In the code snippet above, we use the \hyperref[func.calc_user_movie_effect_RMSE.cv]{calc\_user\_movie\_effect\_RMSE.cv} function described in section \hyperref[ume-model-utility-functions]{UME Model: Utility Functions} of \hyperref[appndx_a]{Appendix A}.

\end{noteblock}

\newpage

Finally, we add the \emph{RMSE} value obtained above to our \emph{Result Table} and print the table using the following \href{https://github.com/AzKurban-edX-DS/Capstone-MovieLens/blob/main/r/src/capstone-movielens.main.R\#L516}{code snippet}:

\begin{Shaded}
\begin{Highlighting}[]
\NormalTok{RMSEs.ResultTibble.UME }\OtherTok{\textless{}{-}}\NormalTok{ RMSEs.ResultTibble.UE }\SpecialCharTok{|\textgreater{}} 
  \FunctionTok{RMSEs.AddRow}\NormalTok{(}\StringTok{"UME Model"}\NormalTok{, }
\NormalTok{               cv.UM\_effect.RMSE,}
               \AttributeTok{comment =} \StringTok{"User+Movie Effect (UME) Model"}\NormalTok{)}
\end{Highlighting}
\end{Shaded}

\begin{Shaded}
\begin{Highlighting}[]
\FunctionTok{RMSE\_kable}\NormalTok{(RMSEs.ResultTibble.UME)}
\end{Highlighting}
\end{Shaded}

\begin{tabular}{>{\raggedright\arraybackslash}p{15em}>{\centering\arraybackslash}p{5em}>{\raggedright\arraybackslash}p{30em}}
\toprule
\textbf{Method} & \textbf{RMSE} & \textbf{Comment}\\
\midrule
Project Objective & 0.8649000 & \\
OMR Model & 1.0603462 & Overall Mean Rating (OMR) Model\\
UE Model & 0.9697962 & User Effect (UE) Model\\
UME Model & 0.8732081 & User+Movie Effect (UME) Model\\
\bottomrule
\end{tabular}

\begin{noteblock}
In the code snippet above, we use the \hyperref[func.RMSEs.AddRow]{RMSEs.AddRow} and \hyperref[func.RMSE_kable]{RMSE\_kable} functions described in section \hyperref[result-rmses-tibble-functions]{Result RMSEs Tibble Functions} of \hyperref[appndx_a]{Appendix A}.

\end{noteblock}

\newpage

\subsubsection{UME Model Regularization}\label{ume-model-regularization}

\hfill\break

\begin{noteblock}
The complete version of the source code provided in this section can be found in the \href{https://github.com/AzKurban-edX-DS/Capstone-MovieLens/blob/main/r/src/capstone-movielens.main.R\#L526}{UME Model Regularization} section of the \href{https://github.com/AzKurban-edX-DS/Capstone-MovieLens/blob/main/r/src/capstone-movielens.main.R\#L526}{capstone-movielens.main.R} script.

\end{noteblock}

We begin this section with the concept's mathematical description presented below in the subsection \hyperref[ume-model-regularization-mathematical-description]{UME Model Regularization: \emph{Mathematical Description}}.

Next, we will implement the \emph{UME Model Regularization} in the following three steps:

\begin{enumerate}
\def\labelenumi{\arabic{enumi}.}
\tightlist
\item
  \textbf{Pre-configuration:} (Described in subsection \hyperref[ume-model-regularization-pre-configuration]{UME Model Regularization: \emph{Pre-configuration}}) Preliminary determination of the optimal range of \emph{regularization parameter} \(\lambda\) values for the \emph{K-Fold Cross-Validation} samples, where the \(K\) is the length of the \hyperref[appndx_b.edx_CV]{\texttt{edx\_CV} Object} (described in detail in \hyperref[appndx_b]{Appendix B: Models Training Datasets}) to which the \emph{K-Fold Cross-Validation} is applied (in \emph{this Project}, we use \(K = 5\));
\item
  \textbf{Fine-tuning:} (Described in subsection \hyperref[ume-model-regularization-fine-tuning]{UME Model Regularization: \emph{Fine-tuning}}) Determining the best value of \(\lambda\) with the highest possible accuracy that minimizes the \emph{RMSE} score for the model.
\item
  \textbf{Retraining:} (Described in subsection \hyperref[ume-model-regularization-retraining-on-the-edx-with-the-best-lambda]{UME Model Regularization: \emph{Retraining on the \texttt{edx} with the best} \(\lambda\)}) Retraining the model on the entire \texttt{edx} dataset with the best value of \(\lambda\) determined in the previous step.
\end{enumerate}

\paragraph{\texorpdfstring{UME Model Regularization: \emph{Mathematical Description}}{UME Model Regularization: Mathematical Description}}\label{ume-model-regularization-mathematical-description}

\hfill\break

\href{https://rafalab.dfci.harvard.edu/dsbook-part-2/highdim/regularization.html\#penalized-least-squares}{Section 24.2 \emph{Penalized least squares}} of the \emph{Course Textbook (New Edition)} explains why and how we should use \emph{Penalized least squares} to improve our predictions. The author also explains that the general idea of penalized regression is to control the total variability of the movie effects\autocite{IDS2_24-2}:
\[
\sum_{j=1}^{n} \beta_j^2
\]
Specifically, instead of minimizing the least squares equation, we minimize an equation that adds a penalty:

\begin{equation}
\sum_{i,j} \left(y_{u,i} - \mu - \alpha_i - \beta_j \right)^2 + \lambda \sum_{j} \beta_j^2 
\label{eq:ME-penalty}
\end{equation}

The first term is just the sum of squares and the second is a penalty that gets larger when many \(\beta_i\)s are large. Using calculus, we can actually show that the values of \(\beta_i\) that minimize this equation are:

\begin{equation}
\hat{\beta}_j(\lambda) = \frac{1}{\lambda + n_j} \sum_{i=1}^{n_j} \left(Y_{i,j} - \mu - \alpha_i\right)
\label{eq:ME-regularized}
\end{equation}

where \(n_j\) is the number of ratings made for movie \(j\).

This approach will have our desired effect: when our sample size \(n_j\) is very large, we obtain a stable estimate and the penalty \(\lambda\) is effectively ignored since \(n_j+\lambda \approx n_j\). Yet when the \(n_j\) is small, then the estimate \(\hat{\beta}_i(\lambda)\) is shrunken towards 0. The larger the \(\lambda\), the more we shrink\autocite{IDS_E1_33-9-2}.

\newpage

\paragraph{\texorpdfstring{UME Model Regularization: \emph{Pre-configuration}}{UME Model Regularization: Pre-configuration}}\label{ume-model-regularization-pre-configuration}

\hfill\break

\begin{noteblock}
The complete version of the source code provided in this section can be found in the \href{https://github.com/AzKurban-edX-DS/Capstone-MovieLens/blob/main/r/src/capstone-movielens.main.R\#L554}{UME Model Regularization: Pre-configuration} section of the \href{https://github.com/AzKurban-edX-DS/Capstone-MovieLens/blob/main/r/src/capstone-movielens.main.R\#L554}{capstone-movielens.main.R} script.

\end{noteblock}

We need to do some pre-configuration to determine a suitable range of \(\lambda\) for subsequent fine-tuning of our current model.
For this purpose, we will use the user-defined helper function \hyperref[func.tune.model_param]{tune.model\_param}, passing the \hyperref[func.regularize.test_lambda.UM_effect.cv]{regularize.test\_lambda.UM\_effect.cv} (another custom helper function) as the value of its argument \hyperref[func.tune.model_param.args.fn_tune.test.param_value]{fn\_tune.test.param\_value}.

\begin{noteblock}
The functions \hyperref[func.tune.model_param]{tune.model\_param} and \hyperref[func.regularize.test_lambda.UM_effect.cv]{regularize.test\_lambda.UM\_effect.cv} are described in the sections \hyperref[appndx_a.CHF.model_tuning_utils]{Model Tuning Utils} and \hyperref[ume-model-regularization-1]{UME Model: Regularization}, respectively, of \hyperref[appndx_a]{Appendix A}.

\end{noteblock}

The following \href{https://github.com/AzKurban-edX-DS/Capstone-MovieLens/blob/main/r/src/capstone-movielens.main.R\#L571}{code snippet} performs this operation:

\begin{Shaded}
\begin{Highlighting}[]
\NormalTok{lambdas }\OtherTok{\textless{}{-}} \FunctionTok{seq}\NormalTok{(}\DecValTok{0}\NormalTok{, }\DecValTok{1}\NormalTok{, }\FloatTok{0.1}\NormalTok{)}
\NormalTok{cv.UME.preset.result }\OtherTok{\textless{}{-}} 
  \FunctionTok{tune.model\_param}\NormalTok{(lambdas, regularize.test\_lambda.UM\_effect.cv)}

\FunctionTok{put\_log1}\NormalTok{(}\StringTok{"Preliminary regularization set{-}up of \textasciigrave{}lambda\textasciigrave{}s range for the UME Model has been completed}
\StringTok{for the \%1{-}Fold Cross{-}Validation samples."}\NormalTok{,}
\NormalTok{CVFolds\_N)}
\end{Highlighting}
\end{Shaded}

\begin{Shaded}
\begin{Highlighting}[]
\FunctionTok{str}\NormalTok{(cv.UME.preset.result)}
\end{Highlighting}
\end{Shaded}

\begin{verbatim}
## List of 2
##  $ tuned.result:'data.frame':    8 obs. of  2 variables:
##   ..$ RMSE           : num [1:8] 0.873 0.873 0.873 0.873 0.873 ...
##   ..$ parameter.value: num [1:8] 0 0.1 0.2 0.3 0.4 0.5 0.6 0.7
##  $ best_result : Named num [1:2] 0.4 0.873
##   ..- attr(*, "names")= chr [1:2] "param.best_value" "best_RMSE"
\end{verbatim}

\begin{Shaded}
\begin{Highlighting}[]
\NormalTok{cv.UME.preset.result}\SpecialCharTok{$}\NormalTok{best\_result}
\end{Highlighting}
\end{Shaded}

\begin{verbatim}
## param.best_value        best_RMSE 
##        0.4000000        0.8732057
\end{verbatim}

\newpage

Now, let's visualize the results of the \(\lambda\) range pre-configuration using the following \href{https://github.com/AzKurban-edX-DS/Capstone-MovieLens/blob/main/r/src/capstone-movielens.main.R\#L596}{code snippet}:

\begin{Shaded}
\begin{Highlighting}[]
\NormalTok{cv.UME.preset.result}\SpecialCharTok{$}\NormalTok{tuned.result }\SpecialCharTok{|\textgreater{}}
  \FunctionTok{data.plot}\NormalTok{(}\AttributeTok{title =} \FunctionTok{TeX}\NormalTok{(r}\StringTok{\textquotesingle{}[UME Model Regularization: $\textbackslash{}lambda$ Range Pre{-}configuration]\textquotesingle{}}\NormalTok{),}
              \AttributeTok{xname =} \StringTok{"parameter.value"}\NormalTok{, }
              \AttributeTok{yname =} \StringTok{"RMSE"}\NormalTok{, }
              \AttributeTok{xlabel =} \FunctionTok{TeX}\NormalTok{(r}\StringTok{\textquotesingle{}[$\textbackslash{}lambda$]\textquotesingle{}}\NormalTok{), }
              \AttributeTok{ylabel =} \StringTok{"RMSE"}\NormalTok{)}
\end{Highlighting}
\end{Shaded}

\pandocbounded{\includegraphics[keepaspectratio]{capstone-movielens.report_files/figure-latex/main-script@l596-1.pdf}}

\begin{noteblock}
In the code snippet above, we use the custom data visualization function \hyperref[func.data.plot]{data.plot} described in section \hyperref[data-visualization-functions]{Data Visualization Functions} of \hyperref[appndx_a]{Appendix A}.

\end{noteblock}

\newpage

\paragraph{\texorpdfstring{UME Model Regularization: \emph{Fine-tuning}}{UME Model Regularization: Fine-tuning}}\label{ume-model-regularization-fine-tuning}

\hfill\break

\begin{noteblock}
The complete version of the source code provided in this section can be found in the \href{https://github.com/AzKurban-edX-DS/Capstone-MovieLens/blob/main/r/src/capstone-movielens.main.R\#L607}{UME Model Regularization: Fine-tuning} section of the \href{https://github.com/AzKurban-edX-DS/Capstone-MovieLens/blob/main/r/src/capstone-movielens.main.R\#L607}{capstone-movielens.main.R} script on \emph{GitHub}.

\end{noteblock}

We are now ready to perform the \emph{fine-tuning step} of our model \emph{regularization} process to determine the best value for the \(\lambda\) parameter.

The following \href{https://github.com/AzKurban-edX-DS/Capstone-MovieLens/blob/main/r/src/capstone-movielens.main.R\#L608}{piece of code} prepares the interval of values for the \(\lambda\) parameter, over which the operation has to be done:

\begin{Shaded}
\begin{Highlighting}[]
\NormalTok{endpoints }\OtherTok{\textless{}{-}} 
  \FunctionTok{get\_fine\_tune.param.endpoints}\NormalTok{(cv.UME.preset.result}\SpecialCharTok{$}\NormalTok{tuned.result)}

\NormalTok{UM\_effect.loop\_starter }\OtherTok{\textless{}{-}} \FunctionTok{c}\NormalTok{(endpoints[}\StringTok{"start"}\NormalTok{], }
\NormalTok{                            endpoints[}\StringTok{"end"}\NormalTok{], }
                            \DecValTok{8}\NormalTok{)}
\end{Highlighting}
\end{Shaded}

\begin{verbatim}
## *** Values of the endpoints and the divisor for the interval of `lambda` values ***
\end{verbatim}

\begin{Shaded}
\begin{Highlighting}[]
\NormalTok{UM\_effect.loop\_starter}
\end{Highlighting}
\end{Shaded}

\begin{verbatim}
## start   end       
##   0.3   0.5   8.0
\end{verbatim}

\begin{noteblock}
The helper function \hyperref[func.get_fine_tune.param.endpoints]{get\_fine\_tune.param.endpoints} used in the code snippet above is described in the sections \hyperref[appndx_a.CHF.model_tuning_utils]{Model Tuning Utils} of \hyperref[appndx_a]{Appendix A}.

\end{noteblock}

And the next \href{https://github.com/AzKurban-edX-DS/Capstone-MovieLens/blob/main/r/src/capstone-movielens.main.R\#L617}{code snippet} below accomplishes the task of \emph{fine-tuning} the model:

\begin{Shaded}
\begin{Highlighting}[]
\NormalTok{UME.rglr.fine\_tune.cache.base\_name }\OtherTok{\textless{}{-}} \StringTok{"UME.rglr.fine{-}tune"}

\NormalTok{UME.rglr.fine\_tune.results }\OtherTok{\textless{}{-}} 
  \FunctionTok{model.tune.param\_range}\NormalTok{(UM\_effect.loop\_starter,}
\NormalTok{                         UME.rglr.fine\_tune.cache.path,}
\NormalTok{                         UME.rglr.fine\_tune.cache.base\_name,}
\NormalTok{                         regularize.test\_lambda.UM\_effect.cv)}

\NormalTok{UME.rglr.fine\_tune.RMSE.best }\OtherTok{\textless{}{-}}\NormalTok{ UME.rglr.fine\_tune.results}\SpecialCharTok{$}\NormalTok{best\_result[}\StringTok{"best\_RMSE"}\NormalTok{]}
\end{Highlighting}
\end{Shaded}

\begin{noteblock}
The custom functions \hyperref[func.model.tune.param_range]{model.tune.param\_range} and \hyperref[func.regularize.test_lambda.UM_effect.cv]{regularize.test\_lambda.UM\_effect.cv} used in the code snippet above are described in sections \hyperref[appndx_a.CHF.model_tuning_utils]{Model Tuning Utils} and \hyperref[ume-model-regularization-1]{UME Model: Regularization}, respectively, of \hyperref[appndx_a]{Appendix A}.

\end{noteblock}

\newpage

Below are the results of fine-tuning the \emph{UME Model}:

\begin{verbatim}
## *** Path to the cache directory for intermediate fine-tuning results ***
\end{verbatim}

\begin{Shaded}
\begin{Highlighting}[]
\NormalTok{UME.rglr.fine\_tune.cache.path}
\end{Highlighting}
\end{Shaded}

\begin{verbatim}
## [1] "data/regularization/1.UM-effect/fine-tune"
\end{verbatim}

\begin{verbatim}
## *** Fine-tuning results object data structure ***
\end{verbatim}

\begin{Shaded}
\begin{Highlighting}[]
\FunctionTok{str}\NormalTok{(UME.rglr.fine\_tune.results)}
\end{Highlighting}
\end{Shaded}

\begin{verbatim}
## List of 3
##  $ best_result           : Named num [1:2] 0.387 0.873
##   ..- attr(*, "names")= chr [1:2] "param.best_value" "best_RMSE"
##  $ param_values.endpoints: Named num [1:3] 3.87e-01 3.87e-01 9.54e-07
##   ..- attr(*, "names")= chr [1:3] "" "" ""
##  $ tuned.result          :'data.frame':  9 obs. of  2 variables:
##   ..$ parameter.value: num [1:9] 0.387 0.387 0.387 0.387 0.387 ...
##   ..$ RMSE           : num [1:9] 0.873 0.873 0.873 0.873 0.873 ...
\end{verbatim}

\begin{verbatim}
## *** Fine-tuning: best results ***
\end{verbatim}

\begin{Shaded}
\begin{Highlighting}[]
\NormalTok{UME.rglr.fine\_tune.results}\SpecialCharTok{$}\NormalTok{best\_result}
\end{Highlighting}
\end{Shaded}

\begin{verbatim}
## param.best_value        best_RMSE 
##        0.3874500        0.8732057
\end{verbatim}

\begin{Shaded}
\begin{Highlighting}[]
\NormalTok{UME.rglr.fine\_tune.RMSE.best}
\end{Highlighting}
\end{Shaded}

\begin{verbatim}
## best_RMSE 
## 0.8732057
\end{verbatim}

\newpage

The following \href{https://github.com/AzKurban-edX-DS/Capstone-MovieLens/blob/main/r/src/capstone-movielens.main.R\#L636}{code snippet} provides a visual representation of the \emph{fine-tuning results} we have just computed:

\begin{Shaded}
\begin{Highlighting}[]
\NormalTok{UME.rglr.fine\_tune.results}\SpecialCharTok{$}\NormalTok{tuned.result }\SpecialCharTok{|\textgreater{}}
  \FunctionTok{data.plot}\NormalTok{(}\AttributeTok{title =} \StringTok{"UME Model Regularization: Fine{-}tuned result"}\NormalTok{,}
              \AttributeTok{xname =} \StringTok{"parameter.value"}\NormalTok{,}
              \AttributeTok{yname =} \StringTok{"RMSE"}\NormalTok{,}
              \AttributeTok{xlabel =} \FunctionTok{TeX}\NormalTok{(r}\StringTok{\textquotesingle{}[$\textbackslash{}lambda$]\textquotesingle{}}\NormalTok{),}
              \AttributeTok{ylabel =} \FunctionTok{str\_glue}\NormalTok{(}\StringTok{"Deviation from the best RMSE value ("}\NormalTok{,}
                                \FunctionTok{as.character}\NormalTok{(}\FunctionTok{round}\NormalTok{(UME.rglr.fine\_tune.RMSE.best, }\AttributeTok{digits =} \DecValTok{7}\NormalTok{)), }
                                \StringTok{")"}\NormalTok{),}
              \AttributeTok{normalize =} \ConstantTok{TRUE}\NormalTok{)}
\end{Highlighting}
\end{Shaded}

\pandocbounded{\includegraphics[keepaspectratio]{capstone-movielens.report_files/figure-latex/main-script@l636-1.pdf}}

\begin{noteblock}
Note that in the code snippet above, we use the custom data visualization function \hyperref[func.data.plot]{data.plot} (described in section \hyperref[data-visualization-functions]{Data Visualization Functions} of \hyperref[appndx_a]{Appendix A}) with the argument \hyperref[func.data.plot.args.normalize]{normalize} set to \texttt{TRUE}, which means that deviations from the minimum \(y\) value are used to plot, rather than the \(y\) values themselves.

\end{noteblock}

\newpage

\paragraph{\texorpdfstring{UME Model Regularization: \emph{Retraining on the \texttt{edx} with the best} \(\lambda\)}{UME Model Regularization: Retraining on the edx with the best \textbackslash lambda}}\label{ume-model-regularization-retraining-on-the-edx-with-the-best-lambda}

\hfill\break

\begin{noteblock}
The complete version of the source code provided in this section are available in the \href{https://github.com/AzKurban-edX-DS/Capstone-MovieLens/blob/main/r/src/capstone-movielens.main.R\#L650}{UME Model Regularization: Re-training with the best \texttt{lambda}} section of the \href{https://github.com/AzKurban-edX-DS/Capstone-MovieLens/blob/main/r/src/capstone-movielens.main.R\#L650}{capstone-movielens.main.R} script on \emph{GitHub}.

\end{noteblock}

Now, we can refine our \emph{User+Movie Effect Model} by retraining on the entire \texttt{edx} dataset with the best value of the \(\lambda\) parameter we just figured out (let's call it \emph{Regularized User+Movie Effect Model} or \emph{Regularized UME Model} for short), for the definitive \emph{RMSE} calculation and use in subsequent models.

The following \href{https://github.com/AzKurban-edX-DS/Capstone-MovieLens/blob/main/r/src/capstone-movielens.main.R\#L664}{code snippet} performs this operation:

\begin{Shaded}
\begin{Highlighting}[]
\NormalTok{best\_result }\OtherTok{\textless{}{-}}\NormalTok{ UME.rglr.fine\_tune.results}\SpecialCharTok{$}\NormalTok{best\_result}
\NormalTok{UME.rglr.best\_lambda }\OtherTok{\textless{}{-}}\NormalTok{ best\_result[}\StringTok{"param.best\_value"}\NormalTok{]}

\FunctionTok{put\_log1}\NormalTok{(}\StringTok{"Re{-}training Regularized User+Movie Effect Model for the best \textasciigrave{}lambda\textasciigrave{}: \%1..."}\NormalTok{,}
\NormalTok{         UME.rglr.best\_lambda)}

\NormalTok{rglr.UM\_effect }\OtherTok{\textless{}{-}} \FunctionTok{train\_user\_movie\_effect}\NormalTok{(edx, UME.rglr.best\_lambda)}
\end{Highlighting}
\end{Shaded}

\begin{noteblock}
In the code snippet above we use the \hyperref[func.train_user_movie_effect]{train\_user\_movie\_effect} function described in section \hyperref[ume-model-utility-functions]{UME Model: Utility Functions} of \hyperref[appndx_a]{Appendix A}.

\end{noteblock}

\begin{verbatim}
## *** The Best UM Effect Fine-tuning Results ***
\end{verbatim}

\begin{Shaded}
\begin{Highlighting}[]
\NormalTok{UME.rglr.fine\_tune.results}\SpecialCharTok{$}\NormalTok{best\_result}
\end{Highlighting}
\end{Shaded}

\begin{verbatim}
## param.best_value        best_RMSE 
##        0.3874500        0.8732057
\end{verbatim}

\begin{verbatim}
## *** Regularized UM Effect Structure ***
\end{verbatim}

\begin{Shaded}
\begin{Highlighting}[]
\FunctionTok{str}\NormalTok{(rglr.UM\_effect)}
\end{Highlighting}
\end{Shaded}

\begin{verbatim}
## tibble [10,677 x 3] (S3: tbl_df/tbl/data.frame)
##  $ movieId: int [1:10677] 1 2 3 4 5 6 7 8 9 10 ...
##  $ b      : num [1:10677] 0.331 -0.305 -0.364 -0.599 -0.443 ...
##  $ n      : int [1:10677] 23790 10779 7028 1577 6400 12346 7259 821 2278 15187 ...
\end{verbatim}

\begin{Shaded}
\begin{Highlighting}[]
\FunctionTok{print\_log1}\NormalTok{(}\StringTok{"Regularized UME Model has been re{-}trained for the best \textasciigrave{}lambda\textasciigrave{}: \%1."}\NormalTok{,}
\NormalTok{         UME.rglr.best\_lambda)}
\end{Highlighting}
\end{Shaded}

\begin{verbatim}
## Regularized UME Model has been re-trained for the best `lambda`: 0.38745002746582.
\end{verbatim}

\newpage

Now, we are ready to construct predictors and calculate the \emph{RMSE} score for the ultimately \emph{Regularized UME Model} using the following \href{https://github.com/AzKurban-edX-DS/Capstone-MovieLens/blob/main/r/src/capstone-movielens.main.R\#L712}{line of code}:

\begin{Shaded}
\begin{Highlighting}[]
\NormalTok{UME.rglr.retrain.RMSE }\OtherTok{\textless{}{-}} \FunctionTok{calc\_user\_movie\_effect\_RMSE.cv}\NormalTok{(rglr.UM\_effect)}
\end{Highlighting}
\end{Shaded}

\begin{Shaded}
\begin{Highlighting}[]
\FunctionTok{print\_log1}\NormalTok{(}\StringTok{"The best RMSE for the UME Model after being regularized: \%1"}\NormalTok{,}
\NormalTok{         UME.rglr.retrain.RMSE)}
\end{Highlighting}
\end{Shaded}

\begin{verbatim}
## The best RMSE for the UME Model after being regularized: 0.872972999076497
\end{verbatim}

\begin{noteblock}
In the code snippet above, we use the \hyperref[func.calc_user_movie_effect_RMSE.cv]{calc\_user\_movie\_effect\_RMSE.cv} function described in section \hyperref[ume-model-utility-functions]{UME Model: Utility Functions} of \hyperref[appndx_a]{Appendix A}.

\end{noteblock}

Finally, we add the definitive \emph{RMSE} value obtained above to our \emph{Result Table} and print the table using the following \href{https://github.com/AzKurban-edX-DS/Capstone-MovieLens/blob/main/r/src/capstone-movielens.main.R\#L720}{code snippet}:

\begin{Shaded}
\begin{Highlighting}[]
\NormalTok{RMSEs.ResultTibble.rglr.UME }\OtherTok{\textless{}{-}}\NormalTok{ RMSEs.ResultTibble.UME }\SpecialCharTok{|\textgreater{}} 
  \FunctionTok{RMSEs.AddRow}\NormalTok{(}\StringTok{"Regularized User+Movie Effect Model"}\NormalTok{, }
\NormalTok{               UME.rglr.retrain.RMSE,}
               \AttributeTok{comment =} \StringTok{"Computed for \textasciigrave{}lambda\textasciigrave{} = \%1"} \SpecialCharTok{|\textgreater{}}
                 \FunctionTok{msg.glue}\NormalTok{(UME.rglr.best\_lambda))}
\end{Highlighting}
\end{Shaded}

\begin{Shaded}
\begin{Highlighting}[]
\FunctionTok{RMSE\_kable}\NormalTok{(RMSEs.ResultTibble.rglr.UME)}
\end{Highlighting}
\end{Shaded}

\begin{tabular}{>{\raggedright\arraybackslash}p{15em}>{\centering\arraybackslash}p{5em}>{\raggedright\arraybackslash}p{30em}}
\toprule
\textbf{Method} & \textbf{RMSE} & \textbf{Comment}\\
\midrule
Project Objective & 0.8649000 & \\
OMR Model & 1.0603462 & Overall Mean Rating (OMR) Model\\
UE Model & 0.9697962 & User Effect (UE) Model\\
UME Model & 0.8732081 & User+Movie Effect (UME) Model\\
Regularized UME Model & 0.8729730 & Computed for `lambda` = 0.38745002746582\\
\bottomrule
\end{tabular}

\begin{noteblock}
In the code snippet above, we use the \hyperref[func.RMSEs.AddRow]{RMSEs.AddRow} and \hyperref[func.RMSE_kable]{RMSE\_kable} functions described in section \hyperref[result-rmses-tibble-functions]{Result RMSEs Tibble Functions} of \hyperref[appndx_a]{Appendix A}.

\end{noteblock}

\newpage

\subsection{User+Movie+Genre Effect (UMGE) Model}\label{usermoviegenre-effect-umge-model}

\hfill\break

\begin{noteblock}
The complete source code of the \emph{User+Movie+Genre Effect} computation described in this section is available in the \href{https://github.com/AzKurban-edX-DS/Capstone-MovieLens/blob/main/r/src/capstone-movielens.main.R\#L730}{User+Movie+Genre Effect (UMGE) Model} section of the \href{https://github.com/AzKurban-edX-DS/Capstone-MovieLens/blob/main/r/src/capstone-movielens.main.R\#L730}{capstone-movielens.main.R} script on \emph{GitHub}.

\end{noteblock}

As we can see from the \texttt{edx} dataset structure, the \texttt{Movielens} dataset also has a genres column. This column includes every genre that applies to the movie (some movies fall under several genres):

\begin{Shaded}
\begin{Highlighting}[]
\FunctionTok{str}\NormalTok{(edx)}
\end{Highlighting}
\end{Shaded}

\begin{verbatim}
## 'data.frame':    9000055 obs. of  6 variables:
##  $ userId   : int  1 1 1 1 1 1 1 1 1 1 ...
##  $ movieId  : int  122 185 292 316 329 355 356 362 364 370 ...
##  $ rating   : num  5 5 5 5 5 5 5 5 5 5 ...
##  $ timestamp: int  838985046 838983525 838983421 838983392 838983392 838984474 838983653 838984885 838983707 838984596 ...
##  $ title    : chr  "Boomerang (1992)" "Net, The (1995)" "Outbreak (1995)" "Stargate (1994)" ...
##  $ genres   : chr  "Comedy|Romance" "Action|Crime|Thriller" "Action|Drama|Sci-Fi|Thriller" "Action|Adventure|Sci-Fi" ...
\end{verbatim}

\subsubsection{Movie Genres Effect Analysis}\label{movie-genres-effect-analysis}

\hfill\break

The plot below shows strong evidence of a genre effect (for illustrative purposes, the plot shows only categories with more than 40, 000 ratings).

\begin{Shaded}
\begin{Highlighting}[]
\CommentTok{\# Preparing data for plotting:}
\NormalTok{genre\_ratins\_grp }\OtherTok{\textless{}{-}}\NormalTok{ edx }\SpecialCharTok{|\textgreater{}} 
  \FunctionTok{mutate}\NormalTok{(}\AttributeTok{genre\_categories =} \FunctionTok{as.factor}\NormalTok{(genres)) }\SpecialCharTok{|\textgreater{}}
  \FunctionTok{group\_by}\NormalTok{(genre\_categories) }\SpecialCharTok{|\textgreater{}}
  \FunctionTok{summarize}\NormalTok{(}\AttributeTok{n =} \FunctionTok{n}\NormalTok{(), }\AttributeTok{rating\_avg =} \FunctionTok{mean}\NormalTok{(rating), }\AttributeTok{se =} \FunctionTok{sd}\NormalTok{(rating)}\SpecialCharTok{/}\FunctionTok{sqrt}\NormalTok{(}\FunctionTok{n}\NormalTok{())) }\SpecialCharTok{|\textgreater{}}
  \FunctionTok{filter}\NormalTok{(n }\SpecialCharTok{\textgreater{}} \DecValTok{40000}\NormalTok{) }\SpecialCharTok{|\textgreater{}} 
  \FunctionTok{mutate}\NormalTok{(}\AttributeTok{genres =} \FunctionTok{reorder}\NormalTok{(genre\_categories, rating\_avg)) }\SpecialCharTok{|\textgreater{}}
  \FunctionTok{select}\NormalTok{(genres, rating\_avg, se, n)}

\CommentTok{\# Creating plot:}
\NormalTok{genre\_ratins\_grp }\SpecialCharTok{|\textgreater{}} 
  \FunctionTok{ggplot}\NormalTok{(}\FunctionTok{aes}\NormalTok{(}\AttributeTok{x =}\NormalTok{ genres, }\AttributeTok{y =}\NormalTok{ rating\_avg, }\AttributeTok{ymin =}\NormalTok{ rating\_avg }\SpecialCharTok{{-}} \DecValTok{2}\SpecialCharTok{*}\NormalTok{se, }\AttributeTok{ymax =}\NormalTok{ rating\_avg }\SpecialCharTok{+} \DecValTok{2}\SpecialCharTok{*}\NormalTok{se)) }\SpecialCharTok{+} 
  \FunctionTok{geom\_point}\NormalTok{() }\SpecialCharTok{+}
  \FunctionTok{geom\_errorbar}\NormalTok{() }\SpecialCharTok{+} 
  \FunctionTok{ggtitle}\NormalTok{(}\StringTok{"Average rating per Genre"}\NormalTok{) }\SpecialCharTok{+}
  \FunctionTok{ylab}\NormalTok{(}\StringTok{"Average rating"}\NormalTok{) }\SpecialCharTok{+}
  \FunctionTok{theme}\NormalTok{(}\AttributeTok{axis.text.x =} \FunctionTok{element\_text}\NormalTok{(}\AttributeTok{angle =} \DecValTok{90}\NormalTok{, }\AttributeTok{hjust =} \DecValTok{1}\NormalTok{))}
\end{Highlighting}
\end{Shaded}

\pandocbounded{\includegraphics[keepaspectratio]{capstone-movielens.report_files/figure-latex/unnamed-chunk-114-1.pdf}}

Below are worst and best ratings categories:

\begin{verbatim}
## [1] "The worst ratings are for the genre category: Horror"
\end{verbatim}

\begin{verbatim}
## [1] "The best ratings are for the genre category: Comedy|Drama|Romance|War"
\end{verbatim}

\subsubsection{Mathematical Description of the UMGE Model}\label{mathematical-description-of-the-umge-model}

\hfill\break

If we define a \emph{genre treatment effect} \(g_{i,j}\) for user's \(i\) rating of movie \(j\), we can use the following models to account for the \texttt{genre} effect:

\begin{equation}
Y_{i,j} = \mu + \alpha_i + \beta_j + g_{i,j} + \varepsilon_{i,j}
\label{eq:UMGE-model}
\end{equation}

where \(g_{i,j}\) is an \emph{aggregation function} which is explained in detail in \emph{Section 22.3: ``Review of Aggregation Functions'' of ``Recommender Systems Handbook''} (\emph{Chapter 22: ``Aggregation of Preferences in Recommender Systems''}, p.~712) book\autocite{RRSK_RS_HB}.

In the formula above \(g_{i,j}\) denotes a \emph{genre effect} for user's \(i\) rating of movie \(j\), so that:

\[
g_{i,j} = \sum_{k=1}^K x_{i,j}^k \gamma_k
\]

with \(x^k_{i,j} = 1\) if \(g_{i,j}\) includes genre \(k\), and \(x^k_{i,j} = 0\) otherwise.

Therefore, for our current model, we can compute a predicted value
\[
\hat{g}_{i,j} = g_{i,j} + \varepsilon_{i,j}
\]
as a residual:

\begin{equation}
\hat{g}_{i,j} = Y_{i,j} - (\mu + \alpha_i + \beta_j)
\label{eq:genre-effect}
\end{equation}

\subsubsection{UMGE Model Building}\label{umge-model-building}

\hfill\break

\begin{noteblock}
The complete source code of builing and training the current model is available in the \href{https://github.com/AzKurban-edX-DS/Capstone-MovieLens/blob/main/r/src/capstone-movielens.main.R\#L878}{UMGE Model Building} section of the \href{https://github.com/AzKurban-edX-DS/Capstone-MovieLens/blob/main/r/src/capstone-movielens.main.R\#L878}{capstone-movielens.main.R} script on \emph{GitHub}.

\end{noteblock}

Below is the \href{https://github.com/AzKurban-edX-DS/Capstone-MovieLens/blob/main/r/src/capstone-movielens.main.R\#L902}{line of code} that trains our model with the use of \emph{K-Fold Cross Validation} method, where the \(K\) is the length of the \hyperref[appndx_b.edx_CV]{\texttt{edx\_CV} Object} (described in detail in \hyperref[appndx_b]{Appendix B: Models Training Datasets}) to which the \emph{K-Fold Cross-Validation} is applied (in \emph{this Project}, we use \(K = 5\)):

\begin{Shaded}
\begin{Highlighting}[]
\NormalTok{cv.UMG\_effect }\OtherTok{\textless{}{-}} \FunctionTok{train\_user\_movie\_genre\_effect.cv}\NormalTok{()}
\end{Highlighting}
\end{Shaded}

\begin{noteblock}
In the code snippet above we use the \hyperref[func.train_user_movie_genre_effect.cv]{train\_user\_movie\_genre\_effect.cv} function described in section \hyperref[umge-model-utility-functions]{UMGE Model: Utility Functions} of \hyperref[appndx_a]{Appendix A}.

\end{noteblock}

\begin{Shaded}
\begin{Highlighting}[]
\FunctionTok{str}\NormalTok{(cv.UMG\_effect)}
\end{Highlighting}
\end{Shaded}

\begin{verbatim}
## tibble [10,677 x 2] (S3: tbl_df/tbl/data.frame)
##  $ movieId: int [1:10677] 1 2 3 4 5 6 7 8 9 10 ...
##  $ g      : num [1:10677] -9.28e-06 -7.23e-05 -1.07e-04 -6.46e-05 4.56e-05 ...
\end{verbatim}

\newpage

The following \href{https://github.com/AzKurban-edX-DS/Capstone-MovieLens/blob/main/r/src/capstone-movielens.main.R\#L925}{code snippet} plots a histogram that provides a visual representation of the variability of the \emph{UMG Effect} across movies in the \emph{data frame object} we have just obtained:

\begin{Shaded}
\begin{Highlighting}[]
\FunctionTok{par}\NormalTok{(}\AttributeTok{cex =} \FloatTok{0.7}\NormalTok{)}
\FunctionTok{hist}\NormalTok{(cv.UMG\_effect}\SpecialCharTok{$}\NormalTok{g, }\DecValTok{30}\NormalTok{, }\AttributeTok{xlab =} \FunctionTok{TeX}\NormalTok{(r}\StringTok{\textquotesingle{}[$\textbackslash{}hat\{g\}\_\{i,j\}$]\textquotesingle{}}\NormalTok{),}
     \AttributeTok{main =} \FunctionTok{TeX}\NormalTok{(r}\StringTok{\textquotesingle{}[Histogram of $\textbackslash{}hat\{g\}\_\{i,j\}$]\textquotesingle{}}\NormalTok{))}
\end{Highlighting}
\end{Shaded}

\pandocbounded{\includegraphics[keepaspectratio]{capstone-movielens.report_files/figure-latex/main-script@l925-1.pdf}}

We can now construct predictors and calculate the \emph{RMSE} score for the \emph{UMGE Model} using the
following \href{https://github.com/AzKurban-edX-DS/Capstone-MovieLens/blob/main/r/src/capstone-movielens.main.R\#L932}{line of code}:

\begin{Shaded}
\begin{Highlighting}[]
\NormalTok{cv.UMG\_effect.RMSE }\OtherTok{\textless{}{-}} \FunctionTok{calc\_user\_movie\_genre\_effect\_RMSE.cv}\NormalTok{(cv.UMG\_effect)}
\end{Highlighting}
\end{Shaded}

\begin{Shaded}
\begin{Highlighting}[]
\NormalTok{cv.UMG\_effect.RMSE}
\end{Highlighting}
\end{Shaded}

\begin{verbatim}
## [1] 0.872973
\end{verbatim}

\begin{noteblock}
In the code snippet above, we use the \hyperref[func.calc_user_movie_genre_effect_RMSE.cv]{calc\_user\_movie\_genre\_effect\_RMSE.cv} function described in section \hyperref[umge-model-utility-functions]{UMGE Model: Utility Functions} of \hyperref[appndx_a]{Appendix A}.

\end{noteblock}

\newpage

Finally, we add the \emph{RMSE} value obtained above to our \emph{Result Table} and print the table using the following \href{https://github.com/AzKurban-edX-DS/Capstone-MovieLens/blob/main/r/src/capstone-movielens.main.R\#L937}{code snippet}:

\begin{Shaded}
\begin{Highlighting}[]
\NormalTok{RMSEs.ResultTibble.UMGE }\OtherTok{\textless{}{-}}\NormalTok{ RMSEs.ResultTibble.rglr.UME }\SpecialCharTok{|\textgreater{}} 
  \FunctionTok{RMSEs.AddRow}\NormalTok{(}\StringTok{"UMGE Model"}\NormalTok{, }
\NormalTok{               cv.UMG\_effect.RMSE,}
               \AttributeTok{comment =} \StringTok{"User+Movie+Genre Effect (UMGE) Model"}\NormalTok{)}
\end{Highlighting}
\end{Shaded}

\begin{Shaded}
\begin{Highlighting}[]
\FunctionTok{RMSE\_kable}\NormalTok{(RMSEs.ResultTibble.UMGE)}
\end{Highlighting}
\end{Shaded}

\begin{tabular}{>{\raggedright\arraybackslash}p{15em}>{\centering\arraybackslash}p{5em}>{\raggedright\arraybackslash}p{30em}}
\toprule
\textbf{Method} & \textbf{RMSE} & \textbf{Comment}\\
\midrule
Project Objective & 0.8649000 & \\
OMR Model & 1.0603462 & Overall Mean Rating (OMR) Model\\
UE Model & 0.9697962 & User Effect (UE) Model\\
UME Model & 0.8732081 & User+Movie Effect (UME) Model\\
Regularized UME Model & 0.8729730 & Computed for `lambda` = 0.38745002746582\\
\addlinespace
UMGE Model & 0.8729730 & User+Movie+Genre Effect (UMGE) Model\\
\bottomrule
\end{tabular}

\begin{noteblock}
In the code snippet above, we use the \hyperref[func.RMSEs.AddRow]{RMSEs.AddRow} and \hyperref[func.RMSE_kable]{RMSE\_kable} functions described in section \hyperref[result-rmses-tibble-functions]{Result RMSEs Tibble Functions} of \hyperref[appndx_a]{Appendix A}.

\end{noteblock}

Unfortunately, for some reason, we do not see any improvement here yet.

\newpage

\subsubsection{UMGE Model Regularization}\label{umge-model-regularization}

\hfill\break

\begin{noteblock}
The complete version of the source code provided in this section can be found in the \href{https://github.com/AzKurban-edX-DS/Capstone-MovieLens/blob/main/r/src/capstone-movielens.main.R\#L948}{UMGE Model Regularization} section of the \href{https://github.com/AzKurban-edX-DS/Capstone-MovieLens/blob/main/r/src/capstone-movielens.main.R\#L948}{capstone-movielens.main.R} script.

\end{noteblock}

We begin this section with the concept's mathematical description presented below in the subsection \hyperref[umge-model-regularization-mathematical-description]{UMGE Model Regularization: \emph{Mathematical Description}}.

Next, we will implement the \emph{UMGE Model Regularization} in the following three steps:

\begin{enumerate}
\def\labelenumi{\arabic{enumi}.}
\item
  \textbf{Pre-configuration:} (Described in subsection \hyperref[umge-model-regularization-pre-configuration]{UMGE Model Regularization: \emph{Pre-configuration}}) Preliminary determination of the optimal range of \emph{regularization parameter} \(\lambda\) values for the \emph{K-Fold Cross-Validation} samples, where the \(K\) is the length of the \hyperref[appndx_b.edx_CV]{\texttt{edx\_CV} Object} (described in detail in \hyperref[appndx_b]{Appendix B: Models Training Datasets}) to which the \emph{K-Fold Cross-Validation} is applied (in \emph{this Project}, we use \(K = 5\));
\item
  \textbf{Fine-tuning:} (Described in the subsection \hyperref[umge-model-regularization-fine-tuning]{UMGE Model Regularization: \emph{Fine-tuning}}) Determining the best value of \(\lambda\) with the highest possible accuracy that minimizes the \emph{RMSE} score for the model.
\item
  \textbf{Retraining:} (Described in subsection \hyperref[umge-model-regularization-retraining-on-the-edx-with-the-best-lambda]{UMGE Model Regularization: \emph{Retraining on the \texttt{edx} with the best} \(\lambda\)}) Retraining the model on the entire \texttt{edx} dataset with the best value of \(\lambda\) determined in the previous step.
\end{enumerate}

\paragraph{\texorpdfstring{UMGE Model Regularization: \emph{Mathematical Description}}{UMGE Model Regularization: Mathematical Description}}\label{umge-model-regularization-mathematical-description}

\hfill\break

We have already explained the idea of the \emph{Linear Model Regularization} earlier in section \hyperref[ume-model-regularization]{UME Model Regularization}. Let's extend the concept outlined there to our current model.

In this case, the formula \eqref{eq:ME-penalty} for adding a penalty takes the form:

\begin{equation}
\sum_{i,j} \left(y_{u,i} - \mu - \alpha_i - \beta_j - g_{i,j}\right)^2 + \lambda \sum_{i,j} g_{i,j}^2
\label{eq:GE-penalty}
\end{equation}

And the formula \eqref{eq:ME-regularized} for calculating the values of the \emph{treatment effect} that minimizes the equation will take the form:

\begin{equation}
\hat{g}_{i,j}(\lambda) = \frac{1}{\lambda + n_g} \sum_{r=1}^{n_{g}} \left(Y_{i,j} - \mu - \alpha_i - \beta_j\right)
\label{eq:GE-regularized}
\end{equation}

where \(n_{g}\) is the number of ratings made for genre \(g\).

\newpage

\paragraph{\texorpdfstring{UMGE Model Regularization: \emph{Pre-configuration}}{UMGE Model Regularization: Pre-configuration}}\label{umge-model-regularization-pre-configuration}

\hfill\break

\begin{noteblock}
The complete version of the source code provided in this section can be found in the \href{https://github.com/AzKurban-edX-DS/Capstone-MovieLens/blob/main/r/src/capstone-movielens.main.R\#L961}{UMGE Model Regularization: Pre-configuration} section of the \href{https://github.com/AzKurban-edX-DS/Capstone-MovieLens/blob/main/r/src/capstone-movielens.main.R\#L961}{capstone-movielens.main.R} script.

\end{noteblock}

As we explained earlier in the \hyperref[ume-model-regularization-pre-configuration]{UME Model Regularization: \emph{Pre-configuration}}, we need to do some pre-configuration to determine a suitable range of \(\lambda\) for subsequent fine-tuning of our current model.

As before, we are going to use the helper function \hyperref[func.tune.model_param]{tune.model\_param}, but this time, passing another function as its argument \hyperref[func.tune.model_param.args.fn_tune.test.param_value]{fn\_tune.test.param\_value}: the function \hyperref[func.regularize.test_lambda.UMG_effect.cv]{regularize.test\_lambda.UMG\_effect.cv} specifically design for the \emph{UMGE Model}.

\begin{noteblock}
The functions \hyperref[func.tune.model_param]{tune.model\_param} and \hyperref[func.regularize.test_lambda.UMG_effect.cv]{regularize.test\_lambda.UMG\_effect.cv} are described in the sections \hyperref[appndx_a.CHF.model_tuning_utils]{Model Tuning Utils} and \hyperref[umge-model-regularization-1]{UMGE Model: Regularization}, respectively, of \hyperref[appndx_a]{Appendix A}.

\end{noteblock}

The following \href{https://github.com/AzKurban-edX-DS/Capstone-MovieLens/blob/main/r/src/capstone-movielens.main.R\#L978}{code snippet} performs this operation:

\begin{Shaded}
\begin{Highlighting}[]
\NormalTok{lambdas }\OtherTok{\textless{}{-}} \FunctionTok{seq}\NormalTok{(}\DecValTok{0}\NormalTok{, }\FloatTok{0.2}\NormalTok{, }\FloatTok{0.01}\NormalTok{)}
\NormalTok{cv.UMGE.preset.result }\OtherTok{\textless{}{-}} 
  \FunctionTok{tune.model\_param}\NormalTok{(lambdas, regularize.test\_lambda.UMG\_effect.cv)}

\FunctionTok{put\_log1}\NormalTok{(}\StringTok{"Preliminary regularization set{-}up of \textasciigrave{}lambda\textasciigrave{}s range for the UMGE Model has been completed}
\StringTok{for the \%1{-}Fold Cross Validation samples."}\NormalTok{,}
\NormalTok{CVFolds\_N)}
\end{Highlighting}
\end{Shaded}

\begin{Shaded}
\begin{Highlighting}[]
\FunctionTok{str}\NormalTok{(cv.UMGE.preset.result)}
\end{Highlighting}
\end{Shaded}

\begin{verbatim}
## List of 2
##  $ tuned.result:'data.frame':    8 obs. of  2 variables:
##   ..$ RMSE           : num [1:8] 0.873 0.873 0.873 0.873 0.873 ...
##   ..$ parameter.value: num [1:8] 0 0.01 0.02 0.03 0.04 0.05 0.06 0.07
##  $ best_result : Named num [1:2] 0.04 0.873
##   ..- attr(*, "names")= chr [1:2] "param.best_value" "best_RMSE"
\end{verbatim}

\begin{Shaded}
\begin{Highlighting}[]
\NormalTok{cv.UMGE.preset.result}\SpecialCharTok{$}\NormalTok{best\_result}
\end{Highlighting}
\end{Shaded}

\begin{verbatim}
## param.best_value        best_RMSE 
##         0.040000         0.872973
\end{verbatim}

\newpage

Now, let's visualize the results of the \(\lambda\) range pre-configuration using the following \href{https://github.com/AzKurban-edX-DS/Capstone-MovieLens/blob/main/r/src/capstone-movielens.main.R\#L1004}{code snippet}:

\begin{Shaded}
\begin{Highlighting}[]
\NormalTok{cv.UMGE.preset.result}\SpecialCharTok{$}\NormalTok{tuned.result }\SpecialCharTok{|\textgreater{}}
  \FunctionTok{data.plot}\NormalTok{(}\AttributeTok{title =} \FunctionTok{TeX}\NormalTok{(r}\StringTok{\textquotesingle{}[UMGE Model Regularization: $\textbackslash{}lambda$ Range Pre{-}configuration]\textquotesingle{}}\NormalTok{),}
              \AttributeTok{xname =} \StringTok{"parameter.value"}\NormalTok{, }
              \AttributeTok{yname =} \StringTok{"RMSE"}\NormalTok{, }
              \AttributeTok{xlabel =} \FunctionTok{TeX}\NormalTok{(r}\StringTok{\textquotesingle{}[$\textbackslash{}lambda$]\textquotesingle{}}\NormalTok{), }
              \AttributeTok{ylabel =} \FunctionTok{str\_glue}\NormalTok{(}\StringTok{"Deviation from the best RMSE value ("}\NormalTok{,}
                                \FunctionTok{as.character}\NormalTok{(}\FunctionTok{round}\NormalTok{(cv.UMGE.preset.result}\SpecialCharTok{$}\NormalTok{best\_result[}\StringTok{"best\_RMSE"}\NormalTok{], }
                                                   \AttributeTok{digits =} \DecValTok{7}\NormalTok{)),}
                                \StringTok{")"}\NormalTok{),}
              \AttributeTok{normalize =} \ConstantTok{TRUE}\NormalTok{)}
\end{Highlighting}
\end{Shaded}

\pandocbounded{\includegraphics[keepaspectratio]{capstone-movielens.report_files/figure-latex/main-script@l1004-1.pdf}}

\begin{noteblock}
In the code snippet above, we use the custom data visualization function \hyperref[func.data.plot]{data.plot} described in section \hyperref[data-visualization-functions]{Data Visualization Functions} of \hyperref[appndx_a]{Appendix A}.

\end{noteblock}

\newpage

\paragraph{\texorpdfstring{UMGE Model Regularization: \emph{Fine-tuning}}{UMGE Model Regularization: Fine-tuning}}\label{umge-model-regularization-fine-tuning}

\hfill\break

\begin{noteblock}
The complete version of the source code provided in this section can be found in the \href{https://github.com/AzKurban-edX-DS/Capstone-MovieLens/blob/main/r/src/capstone-movielens.main.R\#L1019}{UMGE Model Regularization: Fine-tuning} section of the \href{https://github.com/AzKurban-edX-DS/Capstone-MovieLens/blob/main/r/src/capstone-movielens.main.R\#L1019}{capstone-movielens.main.R} script on \emph{GitHub}.

\end{noteblock}

We are now ready to perform the fine-tuning step of our model \emph{regularization} process to determine the best value for the \(\lambda\) parameter.

The following \href{https://github.com/AzKurban-edX-DS/Capstone-MovieLens/blob/main/r/src/capstone-movielens.main.R\#L1020}{piece of code} prepares the interval of values for the \(\lambda\) parameter, over which the operation has to be done:

\begin{Shaded}
\begin{Highlighting}[]
\NormalTok{endpoints }\OtherTok{\textless{}{-}} 
  \FunctionTok{get\_fine\_tune.param.endpoints}\NormalTok{(cv.UMGE.preset.result}\SpecialCharTok{$}\NormalTok{tuned.result)}

\NormalTok{UMG\_effect.loop\_starter }\OtherTok{\textless{}{-}} \FunctionTok{c}\NormalTok{(endpoints[}\StringTok{"start"}\NormalTok{], }
\NormalTok{                            endpoints[}\StringTok{"end"}\NormalTok{], }
                            \DecValTok{8}\NormalTok{)}
\end{Highlighting}
\end{Shaded}

\begin{verbatim}
## *** Values of the endpoints and the divisor for the interval of `lambda` values ***
\end{verbatim}

\begin{Shaded}
\begin{Highlighting}[]
\NormalTok{UMG\_effect.loop\_starter}
\end{Highlighting}
\end{Shaded}

\begin{verbatim}
## start   end       
##  0.03  0.05  8.00
\end{verbatim}

\begin{noteblock}
The helper function \hyperref[func.get_fine_tune.param.endpoints]{get\_fine\_tune.param.endpoints} used in code snippet above is described in the sections \hyperref[appndx_a.CHF.model_tuning_utils]{Model Tuning Utils} of \hyperref[appndx_a]{Appendix A}.

\end{noteblock}

And the next \href{https://github.com/AzKurban-edX-DS/Capstone-MovieLens/blob/main/r/src/capstone-movielens.main.R\#L1029}{code snippet} below accomplishes the task of \emph{fine-tuning} the model:

\begin{Shaded}
\begin{Highlighting}[]
\NormalTok{UMGE.rglr.fine\_tune.cache.base\_name }\OtherTok{\textless{}{-}} \StringTok{"UMGE.rglr.fine{-}tuning"}

\NormalTok{UMGE.rglr.fine\_tune.results }\OtherTok{\textless{}{-}} 
  \FunctionTok{model.tune.param\_range}\NormalTok{(UMG\_effect.loop\_starter,}
\NormalTok{                         UMGE.rglr.fine\_tune.cache.path,}
\NormalTok{                         UMGE.rglr.fine\_tune.cache.base\_name,}
\NormalTok{                         regularize.test\_lambda.UMG\_effect.cv)}

\NormalTok{UMGE.rglr.fine\_tune.RMSE.best }\OtherTok{\textless{}{-}}\NormalTok{ UMGE.rglr.fine\_tune.results}\SpecialCharTok{$}\NormalTok{best\_result[}\StringTok{"best\_RMSE"}\NormalTok{]}
\end{Highlighting}
\end{Shaded}

\begin{noteblock}
The custom functions \hyperref[func.model.tune.param_range]{model.tune.param\_range} and \hyperref[func.regularize.test_lambda.UMG_effect.cv]{regularize.test\_lambda.UMG\_effect.cv} used in the code snippet above are described in the sections \hyperref[appndx_a.CHF.model_tuning_utils]{Model Tuning Utils} and \hyperref[umge-model-regularization-1]{UMGE Model: Regularization}, respectively, of \hyperref[appndx_a]{Appendix A}.

\end{noteblock}

\newpage

Below are the results of fine-tuning the \emph{UMGE Model}:

\begin{verbatim}
## *** Path to the cache directory for intermediate fine-tuning results ***
\end{verbatim}

\begin{Shaded}
\begin{Highlighting}[]
\NormalTok{UMGE.rglr.fine\_tune.cache.path}
\end{Highlighting}
\end{Shaded}

\begin{verbatim}
## [1] "data/regularization/2.UMG-effect/fine-tune"
\end{verbatim}

\begin{verbatim}
## *** Fine-tuning results object data structure ***
\end{verbatim}

\begin{Shaded}
\begin{Highlighting}[]
\FunctionTok{str}\NormalTok{(UMGE.rglr.fine\_tune.results)}
\end{Highlighting}
\end{Shaded}

\begin{verbatim}
## List of 3
##  $ best_result           : Named num [1:2] 0.0359 0.873
##   ..- attr(*, "names")= chr [1:2] "param.best_value" "best_RMSE"
##  $ param_values.endpoints: Named num [1:3] 0.035 0.0425 0.000937
##   ..- attr(*, "names")= chr [1:3] "" "" ""
##  $ tuned.result          :'data.frame':  9 obs. of  2 variables:
##   ..$ parameter.value: num [1:9] 0.035 0.0359 0.0369 0.0378 0.0388 ...
##   ..$ RMSE           : num [1:9] 0.873 0.873 0.873 0.873 0.873 ...
\end{verbatim}

\begin{verbatim}
## *** Fine-tuning: best results ***
\end{verbatim}

\begin{Shaded}
\begin{Highlighting}[]
\NormalTok{UMGE.rglr.fine\_tune.results}\SpecialCharTok{$}\NormalTok{best\_result}
\end{Highlighting}
\end{Shaded}

\begin{verbatim}
## param.best_value        best_RMSE 
##        0.0359375        0.8729730
\end{verbatim}

\begin{Shaded}
\begin{Highlighting}[]
\NormalTok{UMGE.rglr.fine\_tune.RMSE.best}
\end{Highlighting}
\end{Shaded}

\begin{verbatim}
## best_RMSE 
##  0.872973
\end{verbatim}

\newpage

The following \href{https://github.com/AzKurban-edX-DS/Capstone-MovieLens/blob/main/r/src/capstone-movielens.main.R\#L1048}{code snippet} provides a visual representation of the \emph{fine-tuning results} we have just computed:

\begin{Shaded}
\begin{Highlighting}[]
\NormalTok{UMGE.rglr.fine\_tune.results}\SpecialCharTok{$}\NormalTok{tuned.result }\SpecialCharTok{|\textgreater{}}
  \FunctionTok{data.plot}\NormalTok{(}\AttributeTok{title =} \StringTok{"UMGE Model Regularization: Fine{-}tuned result"}\NormalTok{,}
              \AttributeTok{xname =} \StringTok{"parameter.value"}\NormalTok{,}
              \AttributeTok{yname =} \StringTok{"RMSE"}\NormalTok{,}
              \AttributeTok{xlabel =} \FunctionTok{TeX}\NormalTok{(r}\StringTok{\textquotesingle{}[$\textbackslash{}lambda$]\textquotesingle{}}\NormalTok{),}
              \AttributeTok{ylabel =} \FunctionTok{str\_glue}\NormalTok{(}\StringTok{"Deviation from the best RMSE value ("}\NormalTok{,}
                                \FunctionTok{as.character}\NormalTok{(}\FunctionTok{round}\NormalTok{(UMGE.rglr.fine\_tune.RMSE.best, }\AttributeTok{digits =} \DecValTok{7}\NormalTok{)),}
                                \StringTok{")"}\NormalTok{),}
              \AttributeTok{normalize =} \ConstantTok{TRUE}\NormalTok{)}
\end{Highlighting}
\end{Shaded}

\pandocbounded{\includegraphics[keepaspectratio]{capstone-movielens.report_files/figure-latex/main-script@l1048-1.pdf}}

\begin{noteblock}
Note that in the code snippet above, we use the custom data visualization function \hyperref[func.data.plot]{data.plot} (described in section \hyperref[data-visualization-functions]{Data Visualization Functions} of \hyperref[appndx_a]{Appendix A}) with the argument \hyperref[func.data.plot.args.normalize]{normalize} set to \texttt{TRUE}, which means that deviations from the minimum \(y\) value are used to plot, rather than the \(y\) values themselves.

\end{noteblock}

\newpage

\paragraph{\texorpdfstring{UMGE Model Regularization: \emph{Retraining on the \texttt{edx} with the best} \(\lambda\)}{UMGE Model Regularization: Retraining on the edx with the best \textbackslash lambda}}\label{umge-model-regularization-retraining-on-the-edx-with-the-best-lambda}

\hfill\break

\begin{noteblock}
The complete version of the source code provided in this section are available in the \href{https://github.com/AzKurban-edX-DS/Capstone-MovieLens/blob/main/r/src/capstone-movielens.main.R\#L1062}{UMGE Model Regularization: Re-training with the best \texttt{lambda}} section of the \href{https://github.com/AzKurban-edX-DS/Capstone-MovieLens/blob/main/r/src/capstone-movielens.main.R\#L1062}{capstone-movielens.main.R} script on \emph{GitHub}.

\end{noteblock}

Now, we can refine our \emph{User+Movie+Genre Effect (UMGE) Model} by retraining on the entire \texttt{edx} dataset with the best value of the \(\lambda\) parameter we just figured out (let's call it \emph{Regularized User+Movie+Genre Effect Model} or \emph{Regularized UMGE Model} for short), for the definitive \emph{RMSE} calculation and use in subsequent models.

The following \href{https://github.com/AzKurban-edX-DS/Capstone-MovieLens/blob/main/r/src/capstone-movielens.main.R\#L1076}{code snippet} performs this operation:

\begin{Shaded}
\begin{Highlighting}[]
\NormalTok{best\_result }\OtherTok{\textless{}{-}}\NormalTok{ UMGE.rglr.fine\_tune.results}\SpecialCharTok{$}\NormalTok{best\_result}
\NormalTok{UMGE.rglr.best\_lambda }\OtherTok{\textless{}{-}}\NormalTok{ best\_result[}\StringTok{"param.best\_value"}\NormalTok{]}

\FunctionTok{put\_log1}\NormalTok{(}\StringTok{"Re{-}training Regularized User+Movie+Genre Effect Model for the best \textasciigrave{}lambda\textasciigrave{}: \%1..."}\NormalTok{,}
\NormalTok{         UMGE.rglr.best\_lambda)}
  
\NormalTok{rglr.UMG\_effect }\OtherTok{\textless{}{-}}\NormalTok{ edx.sgr }\SpecialCharTok{|\textgreater{}} \FunctionTok{train\_user\_movie\_genre\_effect}\NormalTok{(UMGE.rglr.best\_lambda)}
\end{Highlighting}
\end{Shaded}

\begin{noteblock}
In the code snippet above we use the \hyperref[func.train_user_movie_genre_effect]{train\_user\_movie\_genre\_effect} function described in section \hyperref[umge-model-utility-functions]{UMGE Model: Utility Functions} of \hyperref[appndx_a]{Appendix A}.

\end{noteblock}

\begin{verbatim}
## *** The Best UMG Effect Fine-tuning Results ***
\end{verbatim}

\begin{Shaded}
\begin{Highlighting}[]
\NormalTok{UMGE.rglr.fine\_tune.results}\SpecialCharTok{$}\NormalTok{best\_result}
\end{Highlighting}
\end{Shaded}

\begin{verbatim}
## param.best_value        best_RMSE 
##        0.0359375        0.8729730
\end{verbatim}

\begin{verbatim}
## *** Regularized UMG Effect Structure ***
\end{verbatim}

\begin{Shaded}
\begin{Highlighting}[]
\FunctionTok{str}\NormalTok{(rglr.UMG\_effect)}
\end{Highlighting}
\end{Shaded}

\begin{verbatim}
## tibble [10,677 x 2] (S3: tbl_df/tbl/data.frame)
##  $ movieId: int [1:10677] 1 2 3 4 5 6 7 8 9 10 ...
##  $ g      : num [1:10677] -7.48e-05 -7.24e-05 -8.12e-05 -6.48e-05 -1.14e-04 ...
\end{verbatim}

\begin{Shaded}
\begin{Highlighting}[]
\FunctionTok{print\_log1}\NormalTok{(}\StringTok{"Regularized UMGE Model has been re{-}trained for the best \textasciigrave{}lambda\textasciigrave{}: \%1."}\NormalTok{,}
\NormalTok{         UMGE.rglr.best\_lambda)}
\end{Highlighting}
\end{Shaded}

\begin{verbatim}
## Regularized UMGE Model has been re-trained for the best `lambda`: 0.0359375.
\end{verbatim}

\newpage

Now, we are ready to construct predictors and calculate the \emph{RMSE} score for the ultimately \emph{Regularized UMGE Model} using the following \href{https://github.com/AzKurban-edX-DS/Capstone-MovieLens/blob/main/r/src/capstone-movielens.main.R\#L1087}{line of code}:

\begin{Shaded}
\begin{Highlighting}[]
\NormalTok{  rglr.UMG\_effect.RMSE }\OtherTok{\textless{}{-}} \FunctionTok{calc\_user\_movie\_genre\_effect\_RMSE.cv}\NormalTok{(rglr.UMG\_effect)}
\end{Highlighting}
\end{Shaded}

\begin{Shaded}
\begin{Highlighting}[]
\FunctionTok{print\_log1}\NormalTok{(}\StringTok{"The best RMSE for the UMGE Model after being regularized: \%1"}\NormalTok{,}
\NormalTok{         rglr.UMG\_effect.RMSE)}
\end{Highlighting}
\end{Shaded}

\begin{verbatim}
## The best RMSE for the UMGE Model after being regularized: 0.872973004075585
\end{verbatim}

\begin{noteblock}
In the code snippet above, we use the \hyperref[func.calc_user_movie_genre_effect_RMSE.cv]{calc\_user\_movie\_genre\_effect\_RMSE.cv} function described in section \hyperref[umge-model-utility-functions]{UMGE Model: Utility Functions} of \hyperref[appndx_a]{Appendix A}.

\end{noteblock}

Finally, we add the definitive \emph{RMSE} value obtained above to our \emph{Result Table} and print the table using the following \href{https://github.com/AzKurban-edX-DS/Capstone-MovieLens/blob/main/r/src/capstone-movielens.main.R\#L1114}{code snippet}:

\begin{Shaded}
\begin{Highlighting}[]
\NormalTok{RMSEs.ResultTibble.rglr.UMGE }\OtherTok{\textless{}{-}}\NormalTok{ RMSEs.ResultTibble.UMGE }\SpecialCharTok{|\textgreater{}} 
  \FunctionTok{RMSEs.AddRow}\NormalTok{(}\StringTok{"Regularized User+Movie+Genre Effect Model"}\NormalTok{, }
\NormalTok{               rglr.UMG\_effect.RMSE,}
               \AttributeTok{comment =} \StringTok{"Computed for \textasciigrave{}lambda\textasciigrave{} = \%1"} \SpecialCharTok{|\textgreater{}}
                 \FunctionTok{msg.glue}\NormalTok{(UMGE.rglr.best\_lambda))}
\end{Highlighting}
\end{Shaded}

\begin{Shaded}
\begin{Highlighting}[]
\FunctionTok{RMSE\_kable}\NormalTok{(RMSEs.ResultTibble.rglr.UMGE)}
\end{Highlighting}
\end{Shaded}

\begin{tabular}{>{\raggedright\arraybackslash}p{15em}>{\centering\arraybackslash}p{5em}>{\raggedright\arraybackslash}p{30em}}
\toprule
\textbf{Method} & \textbf{RMSE} & \textbf{Comment}\\
\midrule
Project Objective & 0.8649000 & \\
OMR Model & 1.0603462 & Overall Mean Rating (OMR) Model\\
UE Model & 0.9697962 & User Effect (UE) Model\\
UME Model & 0.8732081 & User+Movie Effect (UME) Model\\
Regularized UME Model & 0.8729730 & Computed for `lambda` = 0.38745002746582\\
\addlinespace
UMGE Model & 0.8729730 & User+Movie+Genre Effect (UMGE) Model\\
Regularized UMGE Model & 0.8729730 & Computed for `lambda` = 0.0359375\\
\bottomrule
\end{tabular}

\begin{noteblock}
In the code snippet above, we use the \hyperref[func.RMSEs.AddRow]{RMSEs.AddRow} and \hyperref[func.RMSE_kable]{RMSE\_kable} functions described in section \hyperref[result-rmses-tibble-functions]{Result RMSEs Tibble Functions} of \hyperref[appndx_a]{Appendix A}.

\end{noteblock}

As we can see, the current model still doesn't show any significant improvement after \emph{Regularization}, even though the data analysis we made earlier in the \hyperref[movie-genres-effect-analysis]{Movie Genres Effect Analysis} section showed strong evidence of a genre effect.

Apparently, we need a better model to account for a genre effect more efficiently.

\newpage

\subsection{User+Movie+Genre+Year Effect (UMGYE) Model}\label{usermoviegenreyear-effect-umgye-model}

\hfill\break

\subsubsection{Year Effect Analysis}\label{year-effect-analysis}

\begin{noteblock}
The \emph{Year Effect} visualization and analysis code in \emph{this Section} are cited from the article \href{https://www.kaggle.com/code/amirmotefaker/movie-recommendation-system-using-r-best/notebook}{Movie Recommendation System using R - BEST} mentioned earlier in this report\autocite{MRS-R-BEST}.

\end{noteblock}

\paragraph{\texorpdfstring{\href{https://www.kaggle.com/code/amirmotefaker/movie-recommendation-system-using-r-best/notebook\#Yearly-rating-count}{Yearly rating count}\autocite{MRS-R-BEST}}{Yearly rating count{[}@MRS-R-BEST{]}}}\label{yearly-rating-countmrs-r-best}

\hfill\break

\begin{Shaded}
\begin{Highlighting}[]
\FunctionTok{print}\NormalTok{(edx }\SpecialCharTok{|\textgreater{}} 
  \FunctionTok{mutate}\NormalTok{(}\AttributeTok{year =} \FunctionTok{year}\NormalTok{(}\FunctionTok{as\_datetime}\NormalTok{(timestamp, }\AttributeTok{origin =} \StringTok{"1970{-}01{-}01"}\NormalTok{))) }\SpecialCharTok{|\textgreater{}}
  \FunctionTok{group\_by}\NormalTok{(year) }\SpecialCharTok{|\textgreater{}}
  \FunctionTok{summarize}\NormalTok{(}\AttributeTok{count =} \FunctionTok{n}\NormalTok{())}
\NormalTok{)}
\end{Highlighting}
\end{Shaded}

\begin{verbatim}
## # A tibble: 15 x 2
##     year   count
##    <dbl>   <int>
##  1  1995       2
##  2  1996  942772
##  3  1997  414101
##  4  1998  181634
##  5  1999  709893
##  6  2000 1144349
##  7  2001  683355
##  8  2002  524959
##  9  2003  619938
## 10  2004  691429
## 11  2005 1059277
## 12  2006  689315
## 13  2007  629168
## 14  2008  696740
## 15  2009   13123
\end{verbatim}

\newpage

\paragraph{\texorpdfstring{\href{https://www.kaggle.com/code/amirmotefaker/movie-recommendation-system-using-r-best/notebook\#Average-rating-per-year-plot}{Average rating per year plot}\autocite{MRS-R-BEST}}{Average rating per year plot{[}@MRS-R-BEST{]}}}\label{average-rating-per-year-plotmrs-r-best}

\hfill\break

\begin{Shaded}
\begin{Highlighting}[]
\NormalTok{edx }\SpecialCharTok{|\textgreater{}} 
  \FunctionTok{mutate}\NormalTok{(}\AttributeTok{year =} \FunctionTok{year}\NormalTok{(}\FunctionTok{as\_datetime}\NormalTok{(timestamp, }\AttributeTok{origin =} \StringTok{"1970{-}01{-}01"}\NormalTok{))) }\SpecialCharTok{|\textgreater{}}
  \FunctionTok{group\_by}\NormalTok{(year) }\SpecialCharTok{|\textgreater{}}
  \FunctionTok{summarize}\NormalTok{(}\AttributeTok{rating\_avg =} \FunctionTok{mean}\NormalTok{(rating)) }\SpecialCharTok{|\textgreater{}}
  \FunctionTok{ggplot}\NormalTok{(}\FunctionTok{aes}\NormalTok{(}\AttributeTok{x =}\NormalTok{ year, }\AttributeTok{y =}\NormalTok{ rating\_avg)) }\SpecialCharTok{+}
  \FunctionTok{geom\_bar}\NormalTok{(}\AttributeTok{stat =} \StringTok{"identity"}\NormalTok{, }\AttributeTok{fill =} \StringTok{"\#8888ff"}\NormalTok{) }\SpecialCharTok{+} 
  \FunctionTok{ggtitle}\NormalTok{(}\StringTok{"Average rating per year"}\NormalTok{) }\SpecialCharTok{+}
  \FunctionTok{xlab}\NormalTok{(}\StringTok{"Year"}\NormalTok{) }\SpecialCharTok{+}
  \FunctionTok{ylab}\NormalTok{(}\StringTok{"Average rating"}\NormalTok{) }\SpecialCharTok{+}
  \FunctionTok{scale\_y\_continuous}\NormalTok{(}\AttributeTok{labels =}\NormalTok{ comma) }\SpecialCharTok{+} 
  \FunctionTok{theme\_economist}\NormalTok{() }\SpecialCharTok{+}
  \FunctionTok{theme}\NormalTok{(}\AttributeTok{axis.title.x =} \FunctionTok{element\_text}\NormalTok{(}\AttributeTok{vjust =} \SpecialCharTok{{-}}\DecValTok{5}\NormalTok{, }\AttributeTok{face =} \StringTok{"bold"}\NormalTok{), }
        \AttributeTok{axis.title.y =} \FunctionTok{element\_text}\NormalTok{(}\AttributeTok{vjust =} \DecValTok{10}\NormalTok{, }\AttributeTok{face =} \StringTok{"bold"}\NormalTok{), }
        \AttributeTok{plot.margin =} \FunctionTok{margin}\NormalTok{(}\FloatTok{0.7}\NormalTok{, }\FloatTok{0.5}\NormalTok{, }\DecValTok{1}\NormalTok{, }\FloatTok{1.2}\NormalTok{, }\StringTok{"cm"}\NormalTok{))}
\end{Highlighting}
\end{Shaded}

\pandocbounded{\includegraphics[keepaspectratio]{capstone-movielens.report_files/figure-latex/unnamed-chunk-156-1.pdf}}

\newpage

\subsubsection{Mathematical Description of the UMGYE Model}\label{mathematical-description-of-the-umgye-model}

\hfill\break

If we define \(\gamma(v_{i,j})\) as the \emph{year effect} for the year (here denotes by \(v_{i,j}\)) of rating a movie \(j\) by a user \(i\), the formula \eqref{eq:UMGE-model} describing the \emph{UMGE Model}, for the current model, takes the form:

\begin{equation}
Y_{i,j} = \mu + \alpha_i + \beta_j + g_{i,j} + \gamma(v_{i,j}) + \varepsilon_{i,j}
\label{eq:UMGYE-model}
\end{equation}

Therefore, the formula \eqref{eq:genre-effect} for calculation the prediction of a \emph{genre effect} as a residual, for a \emph{year effect}
\[
\hat{\gamma}(v_{i,j}) = \gamma(v_{i,j}) + \varepsilon_{i,j}
\]

takes the following form:

\begin{equation}
\hat{\gamma}(v_{i,j}) = Y_{i,j} - (\mu + \alpha_i + \beta_j + g_{i,j})
\label{eq:year-effect}
\end{equation}

\newpage

\subsubsection{UMGYE Model Building}\label{umgye-model-building}

\hfill\break

\begin{noteblock}
The complete source code of building and training the current model is available in the \href{https://github.com/AzKurban-edX-DS/Capstone-MovieLens/blob/main/r/src/capstone-movielens.main.R\#L1163}{UMGYE Model Building} section of the \href{https://github.com/AzKurban-edX-DS/Capstone-MovieLens/blob/main/r/src/capstone-movielens.main.R\#L1163}{capstone-movielens.main.R} script on \emph{GitHub}.

\end{noteblock}

Below is the \href{https://github.com/AzKurban-edX-DS/Capstone-MovieLens/blob/main/r/src/capstone-movielens.main.R\#L1177}{line of code} that trains our model with the use of \emph{K-Fold Cross Validation} method, where the \(K\) is the length of the \hyperref[appndx_b.edx_CV]{\texttt{edx\_CV} Object} (described in detail in \hyperref[appndx_b]{Appendix B: Models Training Datasets}) to which the \emph{K-Fold Cross-Validation} is applied (in \emph{this Project}, we use \(K = 5\)):

\begin{Shaded}
\begin{Highlighting}[]
\NormalTok{  cv.UMGY\_effect }\OtherTok{\textless{}{-}} \FunctionTok{train\_UMGY\_effect.cv}\NormalTok{()}
\end{Highlighting}
\end{Shaded}

\begin{noteblock}
In the code snippet above we use the \hyperref[func.train_UMGY_effect.cv]{train\_UMGY\_effect.cv} user-defined function described in section \hyperref[umge-model-utility-functions]{UMGE Model: Utility Functions} of \hyperref[appndx_a]{Appendix A}.

\end{noteblock}

\begin{Shaded}
\begin{Highlighting}[]
\FunctionTok{str}\NormalTok{(cv.UMGY\_effect)}
\end{Highlighting}
\end{Shaded}

\begin{verbatim}
## tibble [15 x 2] (S3: tbl_df/tbl/data.frame)
##  $ year: num [1:15] 1995 1996 1997 1998 1999 ...
##  $ ye  : num [1:15] 0.6695 0.1152 0.0118 0.0261 0.0335 ...
\end{verbatim}

We can now construct predictors and calculate the \emph{RMSE} score for the \emph{UMGYE Model} using the following \href{https://github.com/AzKurban-edX-DS/Capstone-MovieLens/blob/main/r/src/capstone-movielens.main.R\#L1195}{line of code}:

\begin{Shaded}
\begin{Highlighting}[]
\NormalTok{cv.UMGY\_effect.RMSE }\OtherTok{\textless{}{-}} \FunctionTok{calc\_UMGY\_effect\_RMSE.cv}\NormalTok{(cv.UMGY\_effect)}
\end{Highlighting}
\end{Shaded}

\begin{Shaded}
\begin{Highlighting}[]
\NormalTok{cv.UMGY\_effect.RMSE}
\end{Highlighting}
\end{Shaded}

\begin{verbatim}
## [1] 0.8723975
\end{verbatim}

\begin{noteblock}
In the code snippets above, we use the \hyperref[func.calc_UMGY_effect_RMSE.cv]{calc\_UMGY\_effect\_RMSE.cv} user-defined function described in section \hyperref[umgye-model-utility-functions]{UMGYE Model: Utility Functions} of \hyperref[appndx_a]{Appendix A}.

\end{noteblock}

\newpage

Finally, we add the \emph{RMSE} value obtained above to our \emph{Result Table} and print the table using the following \href{https://github.com/AzKurban-edX-DS/Capstone-MovieLens/blob/main/r/src/capstone-movielens.main.R\#L1199}{code snippet}:

\begin{Shaded}
\begin{Highlighting}[]
\NormalTok{RMSEs.ResultTibble.UMGYE }\OtherTok{\textless{}{-}}\NormalTok{ RMSEs.ResultTibble.rglr.UMGE }\SpecialCharTok{|\textgreater{}} 
  \FunctionTok{RMSEs.AddRow}\NormalTok{(}\StringTok{"UMGYE Model"}\NormalTok{, }
\NormalTok{               cv.UMGY\_effect.RMSE,}
               \AttributeTok{comment =} \StringTok{"User+Movie+Genre+Year Effect (UMGYE) Model"}\NormalTok{)}
\end{Highlighting}
\end{Shaded}

\begin{Shaded}
\begin{Highlighting}[]
\FunctionTok{RMSE\_kable}\NormalTok{(RMSEs.ResultTibble.UMGYE)}
\end{Highlighting}
\end{Shaded}

\begin{tabular}{>{\raggedright\arraybackslash}p{15em}>{\centering\arraybackslash}p{5em}>{\raggedright\arraybackslash}p{30em}}
\toprule
\textbf{Method} & \textbf{RMSE} & \textbf{Comment}\\
\midrule
Project Objective & 0.8649000 & \\
OMR Model & 1.0603462 & Overall Mean Rating (OMR) Model\\
UE Model & 0.9697962 & User Effect (UE) Model\\
UME Model & 0.8732081 & User+Movie Effect (UME) Model\\
Regularized UME Model & 0.8729730 & Computed for `lambda` = 0.38745002746582\\
\addlinespace
UMGE Model & 0.8729730 & User+Movie+Genre Effect (UMGE) Model\\
Regularized UMGE Model & 0.8729730 & Computed for `lambda` = 0.0359375\\
UMGYE Model & 0.8723975 & User+Movie+Genre+Year Effect (UMGYE) Model\\
\bottomrule
\end{tabular}

\begin{noteblock}
In the code snippets above, we use the \hyperref[func.RMSEs.AddRow]{RMSEs.AddRow} and \hyperref[func.RMSE_kable]{RMSE\_kable} functions described in section \hyperref[result-rmses-tibble-functions]{Result RMSEs Tibble Functions} of \hyperref[appndx_a]{Appendix A}.

\end{noteblock}

Now, we have a bit better result than one for the previous (\emph{Regularized UMGE}) model.

\newpage

\subsubsection{UMGYE Model Regularization}\label{umgye-model-regularization}

\hfill\break

\begin{noteblock}
The complete version of the source code provided in this section can be found in the \href{https://github.com/AzKurban-edX-DS/Capstone-MovieLens/blob/main/r/src/capstone-movielens.main.R\#L1210}{UMGYE Model Regularization} section of the \href{https://github.com/AzKurban-edX-DS/Capstone-MovieLens/blob/main/r/src/capstone-movielens.main.R\#L1210}{capstone-movielens.main.R} script.

\end{noteblock}

We begin this section with the concept's mathematical description presented below in the subsection \hyperref[umgye-model-regularization-mathematical-description]{UMGYE Model Regularization: \emph{Mathematical Description}}.

Next, we will implement the \emph{UMGYE Model Regularization} in the following three steps:

\begin{enumerate}
\def\labelenumi{\arabic{enumi}.}
\item
  \textbf{Pre-configuration:} (Described in subsection \hyperref[umgye-model-regularization-pre-configuration]{UMGYE Model Regularization: \emph{Pre-configuration}}) Preliminary determination of the optimal range of \emph{regularization parameter} \(\lambda\) values for the \emph{K-Fold Cross-Validation} samples, where the \(K\) is the length of the \hyperref[appndx_b.edx_CV]{\texttt{edx\_CV} Object} (described in detail in \hyperref[appndx_b]{Appendix B: Models Training Datasets}) to which the \emph{K-Fold Cross-Validation} is applied (in \emph{this Project}, we use \(K = 5\));
\item
  \textbf{Fine-tuning:} (Described in the subsection \hyperref[umgye-model-regularization-fine-tuning]{UMGYE Model Regularization: \emph{Fine-tuning}}) Determining the best value of \(\lambda\) with the highest possible accuracy that minimizes the \emph{RMSE} score for the model.
\item
  \textbf{Retraining:} (Described in subsection \hyperref[umgye-model-regularization-retraining-on-the-edx-with-the-best-lambda]{UMGYE Model Regularization: \emph{Retraining on the \texttt{edx} with the best} \(\lambda\)}) Retraining the model on the entire \texttt{edx} dataset with the best value of \(\lambda\) determined in the previous step.
\end{enumerate}

\paragraph{\texorpdfstring{UMGYE Model Regularization: \emph{Mathematical Description}}{UMGYE Model Regularization: Mathematical Description}}\label{umgye-model-regularization-mathematical-description}

\hfill\break

We have already explained the idea of the \emph{Linear Model Regularization} earlier in section \hyperref[ume-model-regularization]{UME Model Regularization}. We have also seen how the formula \eqref{eq:ME-penalty} for adding a penalty to the \emph{UME Model} is transformed into the formula \eqref{eq:GE-penalty} for the \emph{UMGE Model}. For the current model, this formula takes the form:

\begin{equation}
\sum_{i,j} \left(y_{u,i} - \mu - \alpha_i - \beta_j - g_{i,j} - \gamma(v_{i,j})\right)^2 + \lambda \sum_{i,j} \gamma(v_{i,j})^2
\label{eq:YE-penalty}
\end{equation}

And the formula \eqref{eq:GE-regularized} for calculating the values of the \emph{treatment effect} that minimizes the equation, for the current model, takes the form:

\begin{equation}
\hat{\gamma}(v_{i,j}, \lambda) = \frac{1}{\lambda + n_v} \sum_{r=1}^{n_v} \left(Y_{i,j} - \mu - \alpha_i - \beta_j - g_{i,j}\right)
\label{eq:YE-regularized}
\end{equation}

where \(n_v\) is the number of ratings made in year \(v\).

\newpage

\paragraph{\texorpdfstring{UMGYE Model Regularization: \emph{Pre-configuration}}{UMGYE Model Regularization: Pre-configuration}}\label{umgye-model-regularization-pre-configuration}

\hfill\break

\begin{noteblock}
The complete version of the source code provided in this section can be found in the \href{https://github.com/AzKurban-edX-DS/Capstone-MovieLens/blob/main/r/src/capstone-movielens.main.R\#L1224}{UMGYE Model Regularization: Pre-configuration} section of the \href{https://github.com/AzKurban-edX-DS/Capstone-MovieLens/blob/main/r/src/capstone-movielens.main.R\#L1224}{capstone-movielens.main.R} script on \emph{GitHub}.

\end{noteblock}

As for the previous models, we will use the function \hyperref[func.tune.model_param]{tune.model\_param} passing in the \hyperref[func.tune.model_param.args.fn_tune.test.param_value]{fn\_tune.test.param\_value} argument the model-specific helper function (this time designed for the \emph{UMGYE Model}): \hyperref[func.regularize.test_lambda.UMGY_effect.cv]{regularize.test\_lambda.UMGY\_effect.cv}.

\begin{noteblock}
The functions \hyperref[func.tune.model_param]{tune.model\_param} and \hyperref[func.regularize.test_lambda.UMGY_effect.cv]{regularize.test\_lambda.UMGY\_effect.cv} are described in the sections \hyperref[appndx_a.CHF.model_tuning_utils]{Model Tuning Utils} and \hyperref[umgye-model-regularization-1]{UMGYE Model: Regularization}, respectively, of \hyperref[appndx_a]{Appendix A}.

\end{noteblock}

The following \href{https://github.com/AzKurban-edX-DS/Capstone-MovieLens/blob/main/r/src/capstone-movielens.main.R\#L1241}{code snippet} performs this operation:

\begin{Shaded}
\begin{Highlighting}[]
\NormalTok{  lambdas }\OtherTok{\textless{}{-}} \FunctionTok{seq}\NormalTok{(}\DecValTok{0}\NormalTok{, }\DecValTok{512}\NormalTok{, }\DecValTok{32}\NormalTok{)}
\NormalTok{  cv.UMGYE.preset.result }\OtherTok{\textless{}{-}} 
    \FunctionTok{tune.model\_param}\NormalTok{(lambdas, }
\NormalTok{                     regularize.test\_lambda.UMGY\_effect.cv,}
                     \AttributeTok{steps.beyond\_min =} \DecValTok{16}\NormalTok{)}
\end{Highlighting}
\end{Shaded}

\begin{Shaded}
\begin{Highlighting}[]
\FunctionTok{str}\NormalTok{(cv.UMGYE.preset.result)}
\end{Highlighting}
\end{Shaded}

\begin{verbatim}
## List of 2
##  $ tuned.result:'data.frame':    17 obs. of  2 variables:
##   ..$ RMSE           : num [1:17] 0.872 0.872 0.872 0.872 0.872 ...
##   ..$ parameter.value: num [1:17] 0 32 64 96 128 160 192 224 256 288 ...
##  $ best_result : Named num [1:2] 224 0.872
##   ..- attr(*, "names")= chr [1:2] "param.best_value" "best_RMSE"
\end{verbatim}

\begin{Shaded}
\begin{Highlighting}[]
\NormalTok{cv.UMGYE.preset.result}\SpecialCharTok{$}\NormalTok{best\_result}
\end{Highlighting}
\end{Shaded}

\begin{verbatim}
## param.best_value        best_RMSE 
##      224.0000000        0.8721855
\end{verbatim}

\newpage

Now, let's visualize the results of the \(\lambda\) range pre-configuration using the following \href{https://github.com/AzKurban-edX-DS/Capstone-MovieLens/blob/main/r/src/capstone-movielens.main.R\#L1270}{code snippet}:

\begin{Shaded}
\begin{Highlighting}[]
\NormalTok{cv.UMGYE.preset.result}\SpecialCharTok{$}\NormalTok{tuned.result }\SpecialCharTok{|\textgreater{}}
  \FunctionTok{data.plot}\NormalTok{(}\AttributeTok{title =} \FunctionTok{TeX}\NormalTok{(r}\StringTok{\textquotesingle{}[UMGYE Model Regularization: $\textbackslash{}lambda$ Range Pre{-}configuration]\textquotesingle{}}\NormalTok{),}
              \AttributeTok{xname =} \StringTok{"parameter.value"}\NormalTok{, }
              \AttributeTok{yname =} \StringTok{"RMSE"}\NormalTok{, }
              \AttributeTok{xlabel =} \FunctionTok{TeX}\NormalTok{(r}\StringTok{\textquotesingle{}[$\textbackslash{}lambda$]\textquotesingle{}}\NormalTok{), }
              \AttributeTok{ylabel =} \StringTok{"RMSE"}\NormalTok{)}
\end{Highlighting}
\end{Shaded}

\pandocbounded{\includegraphics[keepaspectratio]{capstone-movielens.report_files/figure-latex/main-script@l1270-1.pdf}}

\begin{noteblock}
In the code snippet above, we use the custom data visualization function \hyperref[func.data.plot]{data.plot} described in section \hyperref[data-visualization-functions]{Data Visualization Functions} of \hyperref[appndx_a]{Appendix A}.

\end{noteblock}

\newpage

\paragraph{\texorpdfstring{UMGYE Model Regularization: \emph{Fine-tuning}}{UMGYE Model Regularization: Fine-tuning}}\label{umgye-model-regularization-fine-tuning}

\hfill\break

\begin{noteblock}
The complete version of the source code provided in this section can be found in the \href{https://github.com/AzKurban-edX-DS/Capstone-MovieLens/blob/main/r/src/capstone-movielens.main.R\#L1281}{UMGYE Model Regularization: Fine-tuning} section of the \href{https://github.com/AzKurban-edX-DS/Capstone-MovieLens/blob/main/r/src/capstone-movielens.main.R\#L1281}{capstone-movielens.main.R} script on \emph{GitHub}.

\end{noteblock}

We are now ready to perform the fine-tuning step of our model \emph{regularization} process to determine the best value for the \(\lambda\) parameter.

The following \href{https://github.com/AzKurban-edX-DS/Capstone-MovieLens/blob/main/r/src/capstone-movielens.main.R\#L1282}{piece of code} prepares the interval of values for the \(\lambda\) parameter, over which the operation has to be done:

\begin{Shaded}
\begin{Highlighting}[]
\NormalTok{endpoints }\OtherTok{\textless{}{-}} 
  \FunctionTok{get\_fine\_tune.param.endpoints}\NormalTok{(cv.UMGYE.preset.result}\SpecialCharTok{$}\NormalTok{tuned.result)}

\NormalTok{UMGYE.loop\_starter }\OtherTok{\textless{}{-}} \FunctionTok{c}\NormalTok{(endpoints[}\StringTok{"start"}\NormalTok{], }
\NormalTok{                        endpoints[}\StringTok{"end"}\NormalTok{], }
                        \DecValTok{8}\NormalTok{)}
\end{Highlighting}
\end{Shaded}

\begin{verbatim}
## *** Values of the endpoints and the divisor for the interval of `lambda` values ***
\end{verbatim}

\begin{Shaded}
\begin{Highlighting}[]
\NormalTok{UMGYE.loop\_starter}
\end{Highlighting}
\end{Shaded}

\begin{verbatim}
## start   end       
##   192   256     8
\end{verbatim}

\begin{noteblock}
The helper function \hyperref[func.get_fine_tune.param.endpoints]{get\_fine\_tune.param.endpoints} used in code snippet above is described in the sections \hyperref[appndx_a.CHF.model_tuning_utils]{Model Tuning Utils} of \hyperref[appndx_a]{Appendix A}.

\end{noteblock}

And the next \href{https://github.com/AzKurban-edX-DS/Capstone-MovieLens/blob/main/r/src/capstone-movielens.main.R\#L1291}{code snippet} below accomplishes the task of \emph{fine-tuning} the model:

\begin{Shaded}
\begin{Highlighting}[]
\NormalTok{cache.base\_name }\OtherTok{\textless{}{-}} \StringTok{"UMGYE.rglr.fine{-}tuning"}

\NormalTok{UMGYE.rglr.fine\_tune.results }\OtherTok{\textless{}{-}} 
  \FunctionTok{model.tune.param\_range}\NormalTok{(UMGYE.loop\_starter,}
\NormalTok{                         UMGYE.rglr.fine\_tune.cache.path,}
\NormalTok{                         cache.base\_name,}
\NormalTok{                         regularize.test\_lambda.UMGY\_effect.cv)}

\NormalTok{UMGYE.rglr.fine\_tune.RMSE.best }\OtherTok{\textless{}{-}}\NormalTok{ UMGYE.rglr.fine\_tune.results}\SpecialCharTok{$}\NormalTok{best\_result[}\StringTok{"best\_RMSE"}\NormalTok{]}
\end{Highlighting}
\end{Shaded}

\begin{noteblock}
The custom functions \hyperref[func.model.tune.param_range]{model.tune.param\_range} and \hyperref[func.regularize.test_lambda.UMGY_effect.cv]{regularize.test\_lambda.UMGY\_effect.cv} used in the code snippet above are described in the sections \hyperref[appndx_a.CHF.model_tuning_utils]{Model Tuning Utils} and \hyperref[umgye-model-regularization-1]{UMGYE Model: Regularization}, respectively, of \hyperref[appndx_a]{Appendix A}.

\end{noteblock}

\newpage

Below are the results of fine-tuning the \emph{UMGYE Model}:

\begin{verbatim}
## *** Path to the cache directory for intermediate fine-tuning results ***
\end{verbatim}

\begin{Shaded}
\begin{Highlighting}[]
\NormalTok{UMGYE.rglr.fine\_tune.cache.path}
\end{Highlighting}
\end{Shaded}

\begin{verbatim}
## [1] "data/regularization/3.UMGY-effect/fine-tune"
\end{verbatim}

\begin{verbatim}
## *** Fine-tuning results object data structure ***
\end{verbatim}

\begin{Shaded}
\begin{Highlighting}[]
\FunctionTok{str}\NormalTok{(UMGYE.rglr.fine\_tune.results)}
\end{Highlighting}
\end{Shaded}

\begin{verbatim}
## List of 3
##  $ best_result           : Named num [1:2] 233.785 0.872
##   ..- attr(*, "names")= chr [1:2] "param.best_value" "best_RMSE"
##  $ param_values.endpoints: Named num [1:3] 2.34e+02 2.34e+02 7.63e-05
##   ..- attr(*, "names")= chr [1:3] "" "" ""
##  $ tuned.result          :'data.frame':  9 obs. of  2 variables:
##   ..$ parameter.value: num [1:9] 234 234 234 234 234 ...
##   ..$ RMSE           : num [1:9] 0.872 0.872 0.872 0.872 0.872 ...
\end{verbatim}

\begin{verbatim}
## *** Fine-tuning: best results ***
\end{verbatim}

\begin{Shaded}
\begin{Highlighting}[]
\NormalTok{UMGYE.rglr.fine\_tune.results}\SpecialCharTok{$}\NormalTok{best\_result}
\end{Highlighting}
\end{Shaded}

\begin{verbatim}
## param.best_value        best_RMSE 
##      233.7849426        0.8721854
\end{verbatim}

\begin{Shaded}
\begin{Highlighting}[]
\NormalTok{UMGYE.rglr.fine\_tune.RMSE.best}
\end{Highlighting}
\end{Shaded}

\begin{verbatim}
## best_RMSE 
## 0.8721854
\end{verbatim}

\newpage

The following \href{https://github.com/AzKurban-edX-DS/Capstone-MovieLens/blob/main/r/src/capstone-movielens.main.R\#L1306}{code snippet} provides a visual representation of the \emph{fine-tuning results} we have just computed:

\begin{Shaded}
\begin{Highlighting}[]
\NormalTok{UMGYE.rglr.fine\_tune.results}\SpecialCharTok{$}\NormalTok{tuned.result }\SpecialCharTok{|\textgreater{}}
  \FunctionTok{data.plot}\NormalTok{(}\AttributeTok{title =} \StringTok{"UMGYE Model Regularization: Fine{-}tuned result"}\NormalTok{,}
              \AttributeTok{xname =} \StringTok{"parameter.value"}\NormalTok{,}
              \AttributeTok{yname =} \StringTok{"RMSE"}\NormalTok{,}
              \AttributeTok{xlabel =} \FunctionTok{TeX}\NormalTok{(r}\StringTok{\textquotesingle{}[$\textbackslash{}lambda$]\textquotesingle{}}\NormalTok{),}
              \AttributeTok{ylabel =} \FunctionTok{str\_glue}\NormalTok{(}\StringTok{"Deviation from the best RMSE value ("}\NormalTok{,}
                                \FunctionTok{as.character}\NormalTok{(}\FunctionTok{round}\NormalTok{(UMGYE.rglr.fine\_tune.RMSE.best, }\AttributeTok{digits =} \DecValTok{7}\NormalTok{)),}
                                \StringTok{")"}\NormalTok{),}
              \AttributeTok{normalize =} \ConstantTok{TRUE}\NormalTok{)}
\end{Highlighting}
\end{Shaded}

\pandocbounded{\includegraphics[keepaspectratio]{capstone-movielens.report_files/figure-latex/main-script@l1306-1.pdf}}

\begin{noteblock}
Note that in the code snippet above, we use the custom data visualization function \hyperref[func.data.plot]{data.plot} (described in section \hyperref[data-visualization-functions]{Data Visualization Functions} of \hyperref[appndx_a]{Appendix A}) with the argument \hyperref[func.data.plot.args.normalize]{normalize} set to \texttt{TRUE}, which means that deviations from the minimum \(y\) value are used to plot, rather than the \(y\) values themselves.

\end{noteblock}

\newpage

\paragraph{\texorpdfstring{UMGYE Model Regularization: \emph{Retraining on the \texttt{edx} with the best} \(\lambda\)}{UMGYE Model Regularization: Retraining on the edx with the best \textbackslash lambda}}\label{umgye-model-regularization-retraining-on-the-edx-with-the-best-lambda}

\hfill\break

\begin{noteblock}
The complete version of the source code provided in this section are available in the \href{https://github.com/AzKurban-edX-DS/Capstone-MovieLens/blob/main/r/src/capstone-movielens.main.R\#L1320}{UMGYE Model Regularization: Re-training with the best \texttt{lambda}} section of the \href{https://github.com/AzKurban-edX-DS/Capstone-MovieLens/blob/main/r/src/capstone-movielens.main.R\#L1320}{capstone-movielens.main.R} script on \emph{GitHub}.

\end{noteblock}

Now, we can refine our \emph{User+Movie+Genre+Year Effect (UMGYE) Model} by retraining on the entire \texttt{edx} dataset with the best value of the \(\lambda\) parameter we just figured out (let's call it \emph{Regularized User+Movie+Genre+Year Effect Model} or \emph{Regularized UMGYE Model} for short), for the definitive \emph{RMSE} calculation and use in subsequent models.

The following \href{https://github.com/AzKurban-edX-DS/Capstone-MovieLens/blob/main/r/src/capstone-movielens.main.R\#L1333}{code snippet} performs this operation:

\begin{Shaded}
\begin{Highlighting}[]
\NormalTok{best\_result }\OtherTok{\textless{}{-}}\NormalTok{ UMGYE.rglr.fine\_tune.results}\SpecialCharTok{$}\NormalTok{best\_result}
\NormalTok{UMGYE.rglr.best\_lambda }\OtherTok{\textless{}{-}}\NormalTok{ best\_result[}\StringTok{"param.best\_value"}\NormalTok{]}

\FunctionTok{put\_log1}\NormalTok{(}\StringTok{"Re{-}training Regularized User+Movie+Genre+Year Effect Model for the best \textasciigrave{}lambda\textasciigrave{}: \%1..."}\NormalTok{,}
\NormalTok{         UMGYE.rglr.best\_lambda)}

\NormalTok{rglr.UMGY\_effect }\OtherTok{\textless{}{-}}\NormalTok{ edx }\SpecialCharTok{|\textgreater{}} \FunctionTok{train\_UMGY\_effect}\NormalTok{(UMGYE.rglr.best\_lambda)}
\end{Highlighting}
\end{Shaded}

\begin{noteblock}
In the code snippet above we use the \hyperref[func.train_UMGY_effect]{train\_UMGY\_effect} function described in section \hyperref[umgye-model-utility-functions]{UMGYE Model: Utility Functions} of \hyperref[appndx_a]{Appendix A}.

\end{noteblock}

\begin{verbatim}
## *** The Best UMGY Effect Fine-tuning Results ***
\end{verbatim}

\begin{Shaded}
\begin{Highlighting}[]
\NormalTok{UMGYE.rglr.fine\_tune.results}\SpecialCharTok{$}\NormalTok{best\_result}
\end{Highlighting}
\end{Shaded}

\begin{verbatim}
## param.best_value        best_RMSE 
##      233.7849426        0.8721854
\end{verbatim}

\begin{verbatim}
## *** Regularized UMGY Effect Structure ***
\end{verbatim}

\begin{Shaded}
\begin{Highlighting}[]
\FunctionTok{str}\NormalTok{(rglr.UMGY\_effect)}
\end{Highlighting}
\end{Shaded}

\begin{verbatim}
## tibble [15 x 2] (S3: tbl_df/tbl/data.frame)
##  $ year: num [1:15] 1995 1996 1997 1998 1999 ...
##  $ ye  : num [1:15] 0.0018 0.06806 0.01205 0.01016 0.00634 ...
\end{verbatim}

\begin{Shaded}
\begin{Highlighting}[]
\FunctionTok{print\_log1}\NormalTok{(}\StringTok{"Regularized UMGYE Model has been re{-}trained for the best \textasciigrave{}lambda\textasciigrave{}: \%1."}\NormalTok{,}
\NormalTok{         UMGYE.rglr.best\_lambda)}
\end{Highlighting}
\end{Shaded}

\begin{verbatim}
## Regularized UMGYE Model has been re-trained for the best `lambda`: 233.784942626953.
\end{verbatim}

\newpage

Now, we are ready to construct predictors and calculate the \emph{RMSE} score for the ultimately \emph{Regularized UMGYE Model} using the following \href{https://github.com/AzKurban-edX-DS/Capstone-MovieLens/blob/main/r/src/capstone-movielens.main.R\#L1341}{line of code}:

\begin{Shaded}
\begin{Highlighting}[]
\NormalTok{  rglr.UMGY\_effect.RMSE }\OtherTok{\textless{}{-}} \FunctionTok{calc\_UMGY\_effect\_RMSE.cv}\NormalTok{(rglr.UMGY\_effect)}
\end{Highlighting}
\end{Shaded}

\begin{Shaded}
\begin{Highlighting}[]
\FunctionTok{print\_log1}\NormalTok{(}\StringTok{"The best RMSE for the UMGYE Model after being regularized: \%1"}\NormalTok{,}
\NormalTok{         rglr.UMGY\_effect.RMSE)}
\end{Highlighting}
\end{Shaded}

\begin{verbatim}
## The best RMSE for the UMGYE Model after being regularized: 0.872185956006394
\end{verbatim}

\begin{noteblock}
In the code snippet above, we use the \hyperref[func.calc_UMGY_effect_RMSE.cv]{calc\_UMGY\_effect\_RMSE.cv} function described in section \hyperref[umgye-model-utility-functions]{UMGYE Model: Utility Functions} of \hyperref[appndx_a]{Appendix A}.

\end{noteblock}

Finally, we add the definitive \emph{RMSE} value obtained above to our \emph{Result Table} and print the table using the following \href{https://github.com/AzKurban-edX-DS/Capstone-MovieLens/blob/main/r/src/capstone-movielens.main.R\#L1366}{code snippet}:

\begin{Shaded}
\begin{Highlighting}[]
\NormalTok{RMSEs.ResultTibble.rglr.UMGYE }\OtherTok{\textless{}{-}}\NormalTok{ RMSEs.ResultTibble.UMGYE }\SpecialCharTok{|\textgreater{}} 
  \FunctionTok{RMSEs.AddRow}\NormalTok{(}\StringTok{"Regularized UMGYE Model"}\NormalTok{, }
\NormalTok{               rglr.UMGY\_effect.RMSE,}
               \AttributeTok{comment =} \StringTok{"Computed for \textasciigrave{}lambda\textasciigrave{} = \%1"} \SpecialCharTok{|\textgreater{}}
                 \FunctionTok{msg.glue}\NormalTok{(UMGYE.rglr.best\_lambda))}
\end{Highlighting}
\end{Shaded}

\begin{Shaded}
\begin{Highlighting}[]
\FunctionTok{RMSE\_kable}\NormalTok{(RMSEs.ResultTibble.rglr.UMGYE)}
\end{Highlighting}
\end{Shaded}

\begin{tabular}{>{\raggedright\arraybackslash}p{15em}>{\centering\arraybackslash}p{5em}>{\raggedright\arraybackslash}p{30em}}
\toprule
\textbf{Method} & \textbf{RMSE} & \textbf{Comment}\\
\midrule
Project Objective & 0.8649000 & \\
OMR Model & 1.0603462 & Overall Mean Rating (OMR) Model\\
UE Model & 0.9697962 & User Effect (UE) Model\\
UME Model & 0.8732081 & User+Movie Effect (UME) Model\\
Regularized UME Model & 0.8729730 & Computed for `lambda` = 0.38745002746582\\
\addlinespace
UMGE Model & 0.8729730 & User+Movie+Genre Effect (UMGE) Model\\
Regularized UMGE Model & 0.8729730 & Computed for `lambda` = 0.0359375\\
UMGYE Model & 0.8723975 & User+Movie+Genre+Year Effect (UMGYE) Model\\
Regularized UMGYE Model & 0.8721860 & Computed for `lambda` = 233.784942626953\\
\bottomrule
\end{tabular}

\begin{noteblock}
In the code snippet above, we use the \hyperref[func.RMSEs.AddRow]{RMSEs.AddRow} and \hyperref[func.RMSE_kable]{RMSE\_kable} functions described in section \hyperref[result-rmses-tibble-functions]{Result RMSEs Tibble Functions} of \hyperref[appndx_a]{Appendix A}.

\end{noteblock}

\newpage

\subsection{User+Movie+Genre+Year+SmoothedDay Effect (UMGYDE) Model}\label{usermoviegenreyearsmoothedday-effect-umgyde-model}

\hfill\break

\subsubsection{Mathematical Description of the UMGYDE Model}\label{mathematical-description-of-the-umgyde-model}

\hfill\break

If we define \(d_{i,j}\) as the \emph{number of days since the earliest record} in the \texttt{edx} dataset for movie \(j\) rated by user \(i\), then the formula \eqref{eq:UMGYE-model} describing the \emph{UMGYE Model}, for the current model, takes the form:

\begin{equation}
Y_{i,j} = \mu + \alpha_i + \beta_j + g_{i,j} + \gamma(v_{i,j}) + s(d_{i,j}) + \varepsilon_{i,j}
\label{eq:UMGYDE-model}
\end{equation}

with \(s\) a \emph{smoothed day} function of \(d_{i,j}\)

Therefore, the formula \eqref{eq:year-effect} for calculation the prediction of a \emph{year effect} as a residual, for a \emph{smoothed day effect}
\[
\hat{s}(d_{i,j}) = s(d_{i,j}) + \varepsilon_{i,j}
\]

takes the following form:

\begin{equation}
\hat{s}(d_{i,j}) = Y_{i,j} - (\mu + \alpha_i + \beta_j + g_{i,j} + \gamma(v_{i,j}))
\label{eq:smoothed-day-effect}
\end{equation}

\newpage

\subsubsection{\texorpdfstring{UMGYDE Model Building with \texttt{loess} Default Parameters}{UMGYDE Model Building with loess Default Parameters}}\label{umgyde-model-building-with-loess-default-parameters}

\hfill\break

\begin{noteblock}
The complete source code of building and training the current model is available in the \href{https://github.com/AzKurban-edX-DS/Capstone-MovieLens/blob/main/r/src/capstone-movielens.main.R\#L1477}{UMGYDEM Training with default parameters} section of the \href{https://github.com/AzKurban-edX-DS/Capstone-MovieLens/blob/main/r/src/capstone-movielens.main.R\#L1477}{capstone-movielens.main.R} script on \emph{GitHub}.

\end{noteblock}

We are going to use the custom helper function (\hyperref[func.train_UMGY_SmoothedDay_effect.cv]{train\_UMGY\_SmoothedDay\_effect.cv} (described in section \hyperref[umgyde-model-utility-functions]{UMGYDE Model: Utility Functions} of \hyperref[appndx_a]{Appendix A}) specifically designed to train the \emph{UMGYDE Model}, which under the hood implicitly calls the \href{https://www.rdocumentation.org/packages/stats/versions/3.6.2/topics/loess}{stats::loess} \texttt{R} function, passing its \texttt{degree} and \texttt{span} parameters values taken from the caller's \hyperref[func.train_UMGY_SmoothedDay_effect.cv.args.degree]{degree} and \hyperref[func.train_UMGY_SmoothedDay_effect.cv.args.span]{span} arguments, respectively.

We will start by calling the \hyperref[func.train_UMGY_SmoothedDay_effect.cv]{train\_UMGY\_SmoothedDay\_effect.cv} with default argument values, which, in addition to not use \emph{regularization technique} yet (which will be used in further training later for the current model), means that the \href{https://www.rdocumentation.org/packages/stats/versions/3.6.2/topics/loess}{loess} function will be internally called with \texttt{degree\ =\ 1} and \texttt{span\ =\ 0.75} argument values.

Below is the \href{https://github.com/AzKurban-edX-DS/Capstone-MovieLens/blob/main/r/src/capstone-movielens.main.R\#L1495}{line of code} that trains our model with the default \texttt{degree} and \texttt{span} parameter values with the use of \emph{K-Fold Cross Validation}, where the \(K\) is the length of the \hyperref[appndx_b.edx_CV]{\texttt{edx\_CV} Object} (described in detail in \hyperref[appndx_b]{Appendix B: Models Training Datasets}) to which the \emph{K-Fold Cross-Validation} is applied (in \emph{this Project}, we use \(K = 5\)):

\begin{Shaded}
\begin{Highlighting}[]
\NormalTok{  cv.UMGYDE.default\_params }\OtherTok{\textless{}{-}} \FunctionTok{train\_UMGY\_SmoothedDay\_effect.cv}\NormalTok{()}
\end{Highlighting}
\end{Shaded}

\begin{Shaded}
\begin{Highlighting}[]
\FunctionTok{str}\NormalTok{(cv.UMGYDE.default\_params)}
\end{Highlighting}
\end{Shaded}

\begin{verbatim}
## tibble [4,640 x 4] (S3: tbl_df/tbl/data.frame)
##  $ days       : int [1:4640] 0 385 388 389 392 393 394 396 397 399 ...
##  $ de         : num [1:4640] 0.6677 0.2939 0.1661 0.3206 -0.0871 ...
##  $ year       : num [1:4640] 1995 1996 1996 1996 1996 ...
##  $ de_smoothed: num [1:4640] 0.0425 0.0351 0.035 0.035 0.035 ...
\end{verbatim}

\newpage

The following \href{https://github.com/AzKurban-edX-DS/Capstone-MovieLens/blob/main/r/src/capstone-movielens.main.R\#L1521}{code snippet} provides a visual representation of the \emph{Smoothed Day Effect} data we have just computed:

\begin{Shaded}
\begin{Highlighting}[]
\NormalTok{cv.UMGYDE.default\_params }\SpecialCharTok{|\textgreater{}}
  \FunctionTok{ggplot}\NormalTok{(}\FunctionTok{aes}\NormalTok{(}\AttributeTok{x =}\NormalTok{ days)) }\SpecialCharTok{+}
  \FunctionTok{geom\_point}\NormalTok{(}\FunctionTok{aes}\NormalTok{(}\AttributeTok{y =}\NormalTok{ de), }\AttributeTok{size =} \DecValTok{3}\NormalTok{, }\AttributeTok{alpha =}\NormalTok{ .}\DecValTok{5}\NormalTok{, }\AttributeTok{color =} \StringTok{"grey"}\NormalTok{) }\SpecialCharTok{+} 
  \FunctionTok{geom\_line}\NormalTok{(}\FunctionTok{aes}\NormalTok{(}\AttributeTok{y =}\NormalTok{ de\_smoothed), }\AttributeTok{color =} \StringTok{"red"}\NormalTok{)}
\end{Highlighting}
\end{Shaded}

\pandocbounded{\includegraphics[keepaspectratio]{capstone-movielens.report_files/figure-latex/main-script@l1521-1.pdf}}

Now, let's construct predictors and calculate the \emph{RMSE} score for the \emph{current model} using the following \href{https://github.com/AzKurban-edX-DS/Capstone-MovieLens/blob/main/r/src/capstone-movielens.main.R\#L1531}{line of code}:

\begin{Shaded}
\begin{Highlighting}[]
\NormalTok{cv.UMGYDE.default\_params.RMSE }\OtherTok{\textless{}{-}}\NormalTok{ cv.UMGYDE.default\_params }\SpecialCharTok{|\textgreater{}}
  \FunctionTok{calc\_UMGY\_SmoothedDay\_effect.RMSE.cv}\NormalTok{()}
\end{Highlighting}
\end{Shaded}

\begin{Shaded}
\begin{Highlighting}[]
\NormalTok{cv.UMGYDE.default\_params.RMSE}
\end{Highlighting}
\end{Shaded}

\begin{verbatim}
## [1] 0.8722314
\end{verbatim}

\begin{noteblock}
In the code snippets above, we use the \hyperref[func.calc_UMGY_SmoothedDay_effect.RMSE.cv]{calc\_UMGY\_SmoothedDay\_effect.RMSE.cv} custom helper function described in section \hyperref[umgyde-model-utility-functions]{UMGYDE Model: Utility Functions} of \hyperref[appndx_a]{Appendix A}.

\end{noteblock}

\newpage

Finally, we add the \emph{RMSE} value obtained above to our \emph{Result Table} and print the table using the following \href{https://github.com/AzKurban-edX-DS/Capstone-MovieLens/blob/main/r/src/capstone-movielens.main.R\#L1542}{code snippet}:

\begin{Shaded}
\begin{Highlighting}[]
\NormalTok{RMSEs.ResultTibble.UMGYDE }\OtherTok{\textless{}{-}}\NormalTok{ RMSEs.ResultTibble.rglr.UMGYE }\SpecialCharTok{|\textgreater{}} 
  \FunctionTok{RMSEs.AddRow}\NormalTok{(}\StringTok{"UMGYDE (Default) Model"}\NormalTok{, }
\NormalTok{               cv.UMGYDE.default\_params.RMSE,}
               \AttributeTok{comment =} \StringTok{"User+Movie+Genre+Year+Day Effect (UMGYDE) Model }
\StringTok{computed using \textasciigrave{}stats::loess\textasciigrave{} function with \textasciigrave{}degree=1\textasciigrave{} \& \textasciigrave{}span=0.75\textasciigrave{} parameter values."}\NormalTok{)}
\end{Highlighting}
\end{Shaded}

\begin{Shaded}
\begin{Highlighting}[]
\FunctionTok{RMSE\_kable}\NormalTok{(RMSEs.ResultTibble.UMGYDE)}
\end{Highlighting}
\end{Shaded}

\begin{tabular}{>{\raggedright\arraybackslash}p{15em}>{\centering\arraybackslash}p{5em}>{\raggedright\arraybackslash}p{30em}}
\toprule
\textbf{Method} & \textbf{RMSE} & \textbf{Comment}\\
\midrule
Project Objective & 0.8649000 & \\
OMR Model & 1.0603462 & Overall Mean Rating (OMR) Model\\
UE Model & 0.9697962 & User Effect (UE) Model\\
UME Model & 0.8732081 & User+Movie Effect (UME) Model\\
Regularized UME Model & 0.8729730 & Computed for `lambda` = 0.38745002746582\\
\addlinespace
UMGE Model & 0.8729730 & User+Movie+Genre Effect (UMGE) Model\\
Regularized UMGE Model & 0.8729730 & Computed for `lambda` = 0.0359375\\
UMGYE Model & 0.8723975 & User+Movie+Genre+Year Effect (UMGYE) Model\\
Regularized UMGYE Model & 0.8721860 & Computed for `lambda` = 233.784942626953\\
UMGYDE (Default) Model & 0.8722314 & User+Movie+Genre+Year+Day Effect (UMGYDE) Model 
computed using `stats::loess` function with `degree=1` \& `span=0.75` parameter values.\\
\bottomrule
\end{tabular}

\begin{noteblock}
In the code snippet above we also use the \hyperref[func.RMSEs.AddRow]{RMSEs.AddRow} and \hyperref[func.RMSE_kable]{RMSE\_kable} functions described in section \hyperref[result-rmses-tibble-functions]{Result RMSEs Tibble Functions} of \hyperref[appndx_a]{Appendix A}.

\end{noteblock}

Unfortunately, with the default parameters, we obtained an even worse result than before.
Let's tune the current model by adjusting the \texttt{degree} and \texttt{span} parameters and see what we get.

\newpage

\subsubsection{\texorpdfstring{UMGYDE Model Tuning by \texttt{degree} and \texttt{span} Parameters}{UMGYDE Model Tuning by degree and span Parameters}}\label{umgyde-model-tuning-by-degree-and-span-parameters}

\hfill\break

\begin{noteblock}
The complete source code of the solution described in this section is available in the \href{https://github.com/AzKurban-edX-DS/Capstone-MovieLens/blob/main/r/src/capstone-movielens.main.R\#L1553}{UMGYDE Model Tuning by \texttt{span} \& \texttt{degree} parameters} section of the \href{https://github.com/AzKurban-edX-DS/Capstone-MovieLens/blob/main/r/src/capstone-movielens.main.R\#L1553}{capstone-movielens.main.R} script on \emph{GitHub}.

\end{noteblock}

Since the \texttt{degree} parameter can only take three discrete values, 0, 1, and 2, we will perform the tuning in three stages - one for each value.

In addition, we will divide each stage into the following two steps for tuning the model by the \texttt{span} parameter:

\begin{enumerate}
\def\labelenumi{\arabic{enumi}.}
\item
  \textbf{Pre-configuration:} Preliminary determination of the optimal range of the \texttt{span} parameter values for the \emph{K-Fold Cross-Validation} samples, where the \(K\) is the length of the \hyperref[appndx_b.edx_CV]{\texttt{edx\_CV} Object} (described in detail in \hyperref[appndx_b]{Appendix B: Models Training Datasets}) to which the \emph{K-Fold Cross-Validation} is applied (in \emph{this Project}, we use \(K = 5\));
\item
  \textbf{Fine-tuning:} Determining the best value of the \texttt{span} parameter with the highest possible accuracy that minimizes the \emph{RMSE} score for the model.
\end{enumerate}

Finally, we will retrain the model on the entire \texttt{edx} dataset with the best \texttt{degree} and \texttt{span} values determined by the tuning result.

\newpage

\paragraph{\texorpdfstring{UMGYDE Model Tuning: Stage 1 (\texttt{degree\ =\ 0})}{UMGYDE Model Tuning: Stage 1 (degree = 0)}}\label{umgyde-model-tuning-stage-1-degree-0}

\hfill\break

\subparagraph{\texorpdfstring{Pre-configuration: Optimal \texttt{span} Range Determination (\texttt{degree\ =\ 0})}{Pre-configuration: Optimal span Range Determination (degree = 0)}}\label{pre-configuration-optimal-span-range-determination-degree-0}

\hfill\break

\begin{noteblock}
The complete source code of the solution described in this section is available in the \href{https://github.com/AzKurban-edX-DS/Capstone-MovieLens/blob/main/r/src/capstone-movielens.main.R\#L1558}{UMGYDE Model Tuning: Pre-configuration (\texttt{degree\ =\ 0})} section of the \href{https://github.com/AzKurban-edX-DS/Capstone-MovieLens/blob/main/r/src/capstone-movielens.main.R\#L1558}{capstone-movielens.main.R} script on \emph{GitHub}.

\end{noteblock}

The following \href{https://github.com/AzKurban-edX-DS/Capstone-MovieLens/blob/main/r/src/capstone-movielens.main.R\#L1578}{code snippet} determines the optimal range of \texttt{span} for the further model tuning when \texttt{degree\ =\ 0}:

\begin{Shaded}
\begin{Highlighting}[]
\NormalTok{  spans }\OtherTok{\textless{}{-}} \FunctionTok{seq}\NormalTok{(}\FloatTok{0.0005}\NormalTok{, }\DecValTok{1}\NormalTok{, }\FloatTok{0.001}\NormalTok{)}
\NormalTok{  lss.UMGYDE.preset.degree0.result }\OtherTok{\textless{}{-}} 
    \FunctionTok{tune.model\_param}\NormalTok{(spans, train\_UMGY\_SmoothedDay\_effect.RMSE.cv.degree0)}
\end{Highlighting}
\end{Shaded}

\begin{noteblock}

In the code snippet above, we use the following functions described in \hyperref[appndx_a]{Appendix A}:

\begin{itemize}
\item
  \hyperref[func.tune.model_param]{tune.model\_param} (described in section \hyperref[appndx_a.CHF.model_tuning_utils]{Model Tuning Utils}) to select the best value of the \emph{tuning parameter} (\texttt{span} in our case) from the given sequence of the \emph{parameter values} passed in the \hyperref[func.tune.model_param.args.param_values]{param\_values} argument.
\item
  \hyperref[func.train_UMGY_SmoothedDay_effect.RMSE.cv.degree0]{train\_UMGY\_SmoothedDay\_effect.RMSE.cv.degree0} (described in section \hyperref[umgyde-model-tuning-loess-params]{UMGYDE Model: Tuning \texttt{loess} Params}) used as an auxiliary function passed (in the \hyperref[func.tune.model_param.args.fn_tune.test.param_value]{fn\_tune.test.param\_value} argument) to the \hyperref[func.tune.model_param]{tune.model\_param} to test the \emph{UMGYDE Model} with the \href{https://www.rdocumentation.org/packages/stats/versions/3.6.2/topics/loess}{loess} \texttt{degree} parameter set to \(0\).
\end{itemize}

\end{noteblock}

\begin{Shaded}
\begin{Highlighting}[]
\FunctionTok{str}\NormalTok{(lss.UMGYDE.preset.degree0.result)}
\end{Highlighting}
\end{Shaded}

\begin{verbatim}
## List of 2
##  $ tuned.result:'data.frame':    5 obs. of  2 variables:
##   ..$ RMSE           : num [1:5] 0.872 0.871 0.871 0.872 0.872
##   ..$ parameter.value: num [1:5] 0.0005 0.0015 0.0025 0.0035 0.0045
##  $ best_result : Named num [1:2] 0.0015 0.8712
##   ..- attr(*, "names")= chr [1:2] "param.best_value" "best_RMSE"
\end{verbatim}

\newpage

The \href{https://github.com/AzKurban-edX-DS/Capstone-MovieLens/blob/main/r/src/capstone-movielens.main.R\#L1609}{code snippet} below provides a visual representation of the dependence of the \emph{RMSE} value on the \href{https://www.rdocumentation.org/packages/stats/versions/3.6.2/topics/loess}{loess} \texttt{span} parameter value over the selected interval, with the parameter \texttt{degree\ =\ 0}, as we have figured out in this step of tuning:

\begin{Shaded}
\begin{Highlighting}[]
\NormalTok{plt.title }\OtherTok{=} \StringTok{"UMGYDE Model, \textasciigrave{}loess\textasciigrave{} function args: range of \textasciigrave{}span\textasciigrave{} values for \textasciigrave{}degree = 0\textasciigrave{}"}
\NormalTok{lss.UMGYDE.preset.degree0.result}\SpecialCharTok{$}\NormalTok{tuned.result }\SpecialCharTok{|\textgreater{}}
  \FunctionTok{data.plot}\NormalTok{(plt.title,}
              \AttributeTok{xname =} \StringTok{"parameter.value"}\NormalTok{, }
              \AttributeTok{yname =} \StringTok{"RMSE"}\NormalTok{, }
              \AttributeTok{xlabel =} \StringTok{"spans"}\NormalTok{, }
              \AttributeTok{ylabel =} \StringTok{"RMSE"}\NormalTok{)}
\end{Highlighting}
\end{Shaded}

\pandocbounded{\includegraphics[keepaspectratio]{capstone-movielens.report_files/figure-latex/main-script@l1609-1.pdf}}

\begin{Shaded}
\begin{Highlighting}[]
\FunctionTok{rm}\NormalTok{(plt.title)}
\end{Highlighting}
\end{Shaded}

\begin{noteblock}
In the code snippet above, we use the custom data visualization function \hyperref[func.data.plot]{data.plot} described in section \hyperref[data-visualization-functions]{Data Visualization Functions} of \hyperref[appndx_a]{Appendix A}.

\end{noteblock}

\newpage

\subparagraph{\texorpdfstring{UMGYDE Model Fine-tuning (\texttt{degree\ =\ 0})}{UMGYDE Model Fine-tuning (degree = 0)}}\label{umgyde-model-fine-tuning-degree-0}

\hfill\break

\begin{noteblock}
The complete source code of the solution described in this section is available in the \href{https://github.com/AzKurban-edX-DS/Capstone-MovieLens/blob/main/r/src/capstone-movielens.main.R\#L1624}{UMGYDE Model Fine-tuning (\texttt{degree\ =\ 0})} section of the \href{https://github.com/AzKurban-edX-DS/Capstone-MovieLens/blob/main/r/src/capstone-movielens.main.R\#L1624}{capstone-movielens.main.R} script on \emph{GitHub}.

\end{noteblock}

The following \href{https://github.com/AzKurban-edX-DS/Capstone-MovieLens/blob/main/r/src/capstone-movielens.main.R\#L1625}{code snippet} performs the \emph{fine-tuning} of the \emph{UMGYDE Model} by precise selection of the \href{https://www.rdocumentation.org/packages/stats/versions/3.6.2/topics/loess}{loess} \texttt{span} parameter with the \texttt{degree} parameter set to \(0\):

\begin{Shaded}
\begin{Highlighting}[]
\NormalTok{lss.fine\_tune.loop\_starter }\OtherTok{\textless{}{-}} 
  \FunctionTok{c}\NormalTok{(lss.UMGYDE.preset.degree0.result}\SpecialCharTok{$}\NormalTok{tuned.result}\SpecialCharTok{$}\NormalTok{parameter.value[}\DecValTok{1}\NormalTok{], }
\NormalTok{    lss.UMGYDE.preset.degree0.result}\SpecialCharTok{$}\NormalTok{tuned.result}\SpecialCharTok{$}\NormalTok{parameter.value[}\DecValTok{3}\NormalTok{], }
    \DecValTok{8}\NormalTok{)}

\NormalTok{cache\_file.base\_name }\OtherTok{\textless{}{-}} \StringTok{"UMGYDE.degree0.tuning{-}span"}

\NormalTok{lss.UMGYDE.fine\_tune.degree0.result }\OtherTok{\textless{}{-}} 
  \FunctionTok{model.tune.param\_range}\NormalTok{(lss.fine\_tune.loop\_starter,}
\NormalTok{                         UMGYDE.fine\_tune.degree0.data.path,}
\NormalTok{                         cache\_file.base\_name,}
\NormalTok{                         train\_UMGY\_SmoothedDay\_effect.RMSE.cv.degree0)}
\end{Highlighting}
\end{Shaded}

\begin{noteblock}

In the code snippet above, we use the following functions described in \hyperref[appndx_a]{Appendix A: Support Functions}:

\begin{itemize}
\item
  \hyperref[func.model.tune.param_range]{model.tune.param\_range} (described in section \hyperref[appndx_a.CHF.model_tuning_utils]{Model Tuning Utils}) to enhance the accuracy of the \emph{tuning parameter best value} using the \emph{best value neighborhood} information passed in the \hyperref[func.model.tune.param_range.args.loop_starter]{loop\_starter} argument.
\item
  \hyperref[func.train_UMGY_SmoothedDay_effect.RMSE.cv.degree0]{train\_UMGY\_SmoothedDay\_effect.RMSE.cv.degree0} (already mentioned in the previous section) used as an auxiliary function passed (in the \hyperref[func.model.tune.param_range.args.fn_tune.test.param_value]{fn\_tune.test.param\_value} argument) to the \hyperref[func.model.tune.param_range]{model.tune.param\_range} to test the \emph{UMGYDE Model} with the \href{https://www.rdocumentation.org/packages/stats/versions/3.6.2/topics/loess}{loess} \texttt{degree} parameter set to \(0\).
\end{itemize}

\end{noteblock}

\begin{Shaded}
\begin{Highlighting}[]
\NormalTok{lss.fine\_tune.loop\_starter}
\end{Highlighting}
\end{Shaded}

\begin{verbatim}
## [1] 0.0005 0.0025 8.0000
\end{verbatim}

\begin{Shaded}
\begin{Highlighting}[]
\FunctionTok{str}\NormalTok{(lss.UMGYDE.fine\_tune.degree0.result)}
\end{Highlighting}
\end{Shaded}

\begin{verbatim}
## List of 3
##  $ best_result           : Named num [1:2] 0.00109 0.87111
##   ..- attr(*, "names")= chr [1:2] "param.best_value" "best_RMSE"
##  $ param_values.endpoints: num [1:3] 1.06e-03 1.31e-03 3.13e-05
##  $ tuned.result          :'data.frame':  9 obs. of  2 variables:
##   ..$ parameter.value: num [1:9] 0.00106 0.00109 0.00113 0.00116 0.00119 ...
##   ..$ RMSE           : num [1:9] 0.871 0.871 0.871 0.871 0.871 ...
\end{verbatim}

\newpage

The \href{https://github.com/AzKurban-edX-DS/Capstone-MovieLens/blob/main/r/src/capstone-movielens.main.R\#L1651}{code snippet} below provides a visual representation of the dependence of the \emph{RMSE} value on the \href{https://www.rdocumentation.org/packages/stats/versions/3.6.2/topics/loess}{loess} \texttt{span} parameter value over the neighborhood of its \emph{best value} (when the parameter \texttt{degree\ =\ 0}), which we have determined in this (\emph{fine-tuning}) step:

\begin{Shaded}
\begin{Highlighting}[]
\NormalTok{plt.title }\OtherTok{=} \StringTok{"Fine{-}tuned UMGYDE Model with \textasciigrave{}loess\textasciigrave{} parameter: \textasciigrave{}degree\textasciigrave{} = 0"}

\NormalTok{lss.UMGYDE.fine\_tune.degree0.result}\SpecialCharTok{$}\NormalTok{tuned.result }\SpecialCharTok{|\textgreater{}}
  \FunctionTok{data.plot}\NormalTok{(plt.title, }
              \AttributeTok{xname =} \StringTok{"parameter.value"}\NormalTok{, }
              \AttributeTok{yname =} \StringTok{"RMSE"}\NormalTok{, }
              \AttributeTok{xlabel =} \StringTok{"spans"}\NormalTok{, }
              \AttributeTok{ylabel =} \StringTok{"RMSE"}\NormalTok{)}
\end{Highlighting}
\end{Shaded}

\pandocbounded{\includegraphics[keepaspectratio]{capstone-movielens.report_files/figure-latex/main-script@l1651-1.pdf}}

\begin{Shaded}
\begin{Highlighting}[]
\FunctionTok{rm}\NormalTok{(plt.title)}
\end{Highlighting}
\end{Shaded}

\begin{noteblock}
In the code snippet above, we use the custom data visualization function \hyperref[func.data.plot]{data.plot} described in section \hyperref[data-visualization-functions]{Data Visualization Functions} of \hyperref[appndx_a]{Appendix A}.

\end{noteblock}

\newpage

Now, we can add the best \emph{RMSE} figured out for the \emph{UMGYDE Fine-tuned Model} with the \href{https://www.rdocumentation.org/packages/stats/versions/3.6.2/topics/loess}{loess} \texttt{degree\ =\ 0} (let's call it \emph{Tuned UMGYDE.d0 Model}) to our \emph{Result Table} and print the table using the following \href{https://github.com/AzKurban-edX-DS/Capstone-MovieLens/blob/main/r/src/capstone-movielens.main.R\#L1662}{code snippet}:

\begin{Shaded}
\begin{Highlighting}[]
\NormalTok{RMSEs.ResultTibble.UMGYDE0 }\OtherTok{\textless{}{-}}\NormalTok{ RMSEs.ResultTibble.UMGYDE }\SpecialCharTok{|\textgreater{}} 
  \FunctionTok{RMSEs.AddRow}\NormalTok{(}\StringTok{"Tuned UMGYDE.d0 Model"}\NormalTok{, }
\NormalTok{               lss.UMGYDE.fine\_tune.degree0.result.best\_RMSE,}
               \AttributeTok{comment =} \StringTok{"UMGYDE Model computed using function call: \textasciigrave{}loess(degree = 0, span = \%1)\textasciigrave{}"} \SpecialCharTok{|\textgreater{}}
                 \FunctionTok{msg.glue}\NormalTok{(lss.UMGYDE.fine\_tune.degree0.result.best\_span))}
\end{Highlighting}
\end{Shaded}

\begin{Shaded}
\begin{Highlighting}[]
\FunctionTok{RMSE\_kable}\NormalTok{(RMSEs.ResultTibble.UMGYDE0)}
\end{Highlighting}
\end{Shaded}

\begin{tabular}{>{\raggedright\arraybackslash}p{15em}>{\centering\arraybackslash}p{5em}>{\raggedright\arraybackslash}p{30em}}
\toprule
\textbf{Method} & \textbf{RMSE} & \textbf{Comment}\\
\midrule
Project Objective & 0.8649000 & \\
OMR Model & 1.0603462 & Overall Mean Rating (OMR) Model\\
UE Model & 0.9697962 & User Effect (UE) Model\\
UME Model & 0.8732081 & User+Movie Effect (UME) Model\\
Regularized UME Model & 0.8729730 & Computed for `lambda` = 0.38745002746582\\
\addlinespace
UMGE Model & 0.8729730 & User+Movie+Genre Effect (UMGE) Model\\
Regularized UMGE Model & 0.8729730 & Computed for `lambda` = 0.0359375\\
UMGYE Model & 0.8723975 & User+Movie+Genre+Year Effect (UMGYE) Model\\
Regularized UMGYE Model & 0.8721860 & Computed for `lambda` = 233.784942626953\\
UMGYDE (Default) Model & 0.8722314 & User+Movie+Genre+Year+Day Effect (UMGYDE) Model 
computed using `stats::loess` function with `degree=1` \& `span=0.75` parameter values.\\
\addlinespace
Tuned UMGYDE.d0 Model & 0.8711135 & UMGYDE Model computed using function call: `loess(degree = 0, span = 0.00109375)`\\
\bottomrule
\end{tabular}

\begin{noteblock}
In the code snippet above we use the \hyperref[func.RMSEs.AddRow]{RMSEs.AddRow} and \hyperref[func.RMSE_kable]{RMSE\_kable} functions described in section \hyperref[result-rmses-tibble-functions]{Result RMSEs Tibble Functions} of \hyperref[appndx_a]{Appendix A}.

\end{noteblock}

\newpage

\paragraph{\texorpdfstring{UMGYDE Model Tuning: Stage 2 (\texttt{degree\ =\ 1})}{UMGYDE Model Tuning: Stage 2 (degree = 1)}}\label{umgyde-model-tuning-stage-2-degree-1}

\hfill\break

\subparagraph{\texorpdfstring{Pre-configuration: Optimal \texttt{span} Range Determination (\texttt{degree\ =\ 1})}{Pre-configuration: Optimal span Range Determination (degree = 1)}}\label{pre-configuration-optimal-span-range-determination-degree-1}

\hfill\break

\begin{noteblock}
The complete source code of the solution described in this section is available in the \href{https://github.com/AzKurban-edX-DS/Capstone-MovieLens/blob/main/r/src/capstone-movielens.main.R\#L1675}{UMGYDE Model Tuning: Pre-configuration (\texttt{degree\ =\ 1})} section of the \href{https://github.com/AzKurban-edX-DS/Capstone-MovieLens/blob/main/r/src/capstone-movielens.main.R\#L1675}{capstone-movielens.main.R} script on \emph{GitHub}.

\end{noteblock}

Below is the \href{https://github.com/AzKurban-edX-DS/Capstone-MovieLens/blob/main/r/src/capstone-movielens.main.R\#L1695}{code snippet} that determines the optimal range of \texttt{span} for the further model tuning when \texttt{degree\ =\ 1}:

\begin{Shaded}
\begin{Highlighting}[]
\NormalTok{  spans }\OtherTok{\textless{}{-}} \FunctionTok{seq}\NormalTok{(}\FloatTok{0.0005}\NormalTok{, }\DecValTok{1}\NormalTok{, }\FloatTok{0.001}\NormalTok{)}
\NormalTok{  lss.UMGYDE.preset.degree1.result }\OtherTok{\textless{}{-}} 
    \FunctionTok{tune.model\_param}\NormalTok{(spans, train\_UMGY\_SmoothedDay\_effect.RMSE.cv.degree1)}
\end{Highlighting}
\end{Shaded}

\begin{noteblock}

In the code snippet above, we use the following functions described in \hyperref[appndx_a]{Appendix A: Support Functions}:

\begin{itemize}
\item
  \hyperref[func.tune.model_param]{tune.model\_param} (described in section \hyperref[appndx_a.CHF.model_tuning_utils]{Model Tuning Utils}) to select the best value of the \emph{tuning parameter} (\texttt{span} in our case) from the given sequence of the \emph{parameter values} passed in the \hyperref[func.tune.model_param.args.param_values]{param\_values} argument.
\item
  \hyperref[func.train_UMGY_SmoothedDay_effect.RMSE.cv.degree1]{train\_UMGY\_SmoothedDay\_effect.RMSE.cv.degree1} (described in section \hyperref[umgyde-model-tuning-loess-params]{UMGYDE Model: Tuning \texttt{loess} Params}) used as an auxiliary function passed (in the \hyperref[func.tune.model_param.args.fn_tune.test.param_value]{fn\_tune.test.param\_value} argument) to the \hyperref[func.tune.model_param]{tune.model\_param} to test the \emph{UMGYDE Model} with the \href{https://www.rdocumentation.org/packages/stats/versions/3.6.2/topics/loess}{loess} \texttt{degree} parameter set to \(1\).
\end{itemize}

\end{noteblock}

\begin{Shaded}
\begin{Highlighting}[]
\FunctionTok{str}\NormalTok{(lss.UMGYDE.preset.degree1.result)}
\end{Highlighting}
\end{Shaded}

\begin{verbatim}
## List of 2
##  $ tuned.result:'data.frame':    5 obs. of  2 variables:
##   ..$ RMSE           : num [1:5] 0.872 0.871 0.871 0.872 0.872
##   ..$ parameter.value: num [1:5] 0.0005 0.0015 0.0025 0.0035 0.0045
##  $ best_result : Named num [1:2] 0.0015 0.8711
##   ..- attr(*, "names")= chr [1:2] "param.best_value" "best_RMSE"
\end{verbatim}

\newpage

The \href{https://github.com/AzKurban-edX-DS/Capstone-MovieLens/blob/main/r/src/capstone-movielens.main.R\#L1726}{code snippet} below provides a visual representation of the dependence of the \emph{RMSE} value on the \href{https://www.rdocumentation.org/packages/stats/versions/3.6.2/topics/loess}{loess} \texttt{span} parameter value over the selected interval, with the parameter \texttt{degree\ =\ 1}, as we have figured out in this step of tuning:

\begin{Shaded}
\begin{Highlighting}[]
\NormalTok{lss.UMGYDE.preset.degree1.result}\SpecialCharTok{$}\NormalTok{tuned.result }\SpecialCharTok{|\textgreater{}}
  \FunctionTok{data.plot}\NormalTok{(}\AttributeTok{title =} \StringTok{"UMGYDE Model, \textasciigrave{}loess\textasciigrave{} function args: range of \textasciigrave{}span\textasciigrave{} values for \textasciigrave{}degree = 1\textasciigrave{}"}\NormalTok{,}
              \AttributeTok{xname =} \StringTok{"parameter.value"}\NormalTok{, }
              \AttributeTok{yname =} \StringTok{"RMSE"}\NormalTok{, }
              \AttributeTok{xlabel =} \StringTok{"spans"}\NormalTok{, }
              \AttributeTok{ylabel =} \StringTok{"RMSE"}\NormalTok{)}
\end{Highlighting}
\end{Shaded}

\pandocbounded{\includegraphics[keepaspectratio]{capstone-movielens.report_files/figure-latex/main-script@l1726-1.pdf}}

\begin{noteblock}
In the code snippet above, we use the custom data visualization function \hyperref[func.data.plot]{data.plot} described in section \hyperref[data-visualization-functions]{Data Visualization Functions} of \hyperref[appndx_a]{Appendix A}.

\end{noteblock}

\newpage

\subparagraph{\texorpdfstring{UMGYDE Model Fine-tuning (\texttt{degree\ =\ 1})}{UMGYDE Model Fine-tuning (degree = 1)}}\label{umgyde-model-fine-tuning-degree-1}

\hfill\break

\begin{noteblock}
The complete source code of the solution described in this section is available in the \href{https://github.com/AzKurban-edX-DS/Capstone-MovieLens/blob/main/r/src/capstone-movielens.main.R\#L1738}{UMGYDE Model Fine-tuning (\texttt{degree\ =\ 1})} section of the \href{https://github.com/AzKurban-edX-DS/Capstone-MovieLens/blob/main/r/src/capstone-movielens.main.R\#L1738}{capstone-movielens.main.R} script on \emph{GitHub}.

\end{noteblock}

The following \href{https://github.com/AzKurban-edX-DS/Capstone-MovieLens/blob/main/r/src/capstone-movielens.main.R\#L1739}{code snippet} performs the \emph{fine-tuning} of the \emph{UMGYDE Model} by precise selection of the \href{https://www.rdocumentation.org/packages/stats/versions/3.6.2/topics/loess}{loess} \texttt{span} parameter with the \texttt{degree} parameter set to \(1\):

\begin{Shaded}
\begin{Highlighting}[]
\NormalTok{lss.fine\_tune.loop\_starter }\OtherTok{\textless{}{-}} 
  \FunctionTok{c}\NormalTok{(lss.UMGYDE.preset.degree1.result}\SpecialCharTok{$}\NormalTok{tuned.result}\SpecialCharTok{$}\NormalTok{parameter.value[}\DecValTok{1}\NormalTok{], }
\NormalTok{    lss.UMGYDE.preset.degree1.result}\SpecialCharTok{$}\NormalTok{tuned.result}\SpecialCharTok{$}\NormalTok{parameter.value[}\DecValTok{3}\NormalTok{], }
    \DecValTok{8}\NormalTok{)}

\NormalTok{cache\_file.base\_name }\OtherTok{\textless{}{-}} \StringTok{"UMGYDE.degree1.tuning{-}span"}

\NormalTok{lss.UMGYDE.fine\_tune.degree1.result }\OtherTok{\textless{}{-}} 
  \FunctionTok{model.tune.param\_range}\NormalTok{(lss.fine\_tune.loop\_starter,}
\NormalTok{                         UMGYDE.fine\_tune.degree1.data.path,}
\NormalTok{                         cache\_file.base\_name,}
\NormalTok{                         train\_UMGY\_SmoothedDay\_effect.RMSE.cv.degree1)}
\end{Highlighting}
\end{Shaded}

\begin{noteblock}

In the code snippet above, we use the following functions described in \hyperref[appndx_a]{Appendix A}:

\begin{itemize}
\item
  \hyperref[func.model.tune.param_range]{model.tune.param\_range} (described in section \hyperref[appndx_a.CHF.model_tuning_utils]{Model Tuning Utils}) to enhance the accuracy of the \emph{tuning parameter best value} using the \emph{best value neighborhood} information passed in the \hyperref[func.model.tune.param_range.args.loop_starter]{loop\_starter} argument.
\item
  \hyperref[func.train_UMGY_SmoothedDay_effect.RMSE.cv.degree1]{train\_UMGY\_SmoothedDay\_effect.RMSE.cv.degree1} (already mentioned in the previous section) used as an auxiliary function passed (in the \hyperref[func.model.tune.param_range.args.fn_tune.test.param_value]{fn\_tune.test.param\_value} argument) to the \hyperref[func.model.tune.param_range]{model.tune.param\_range} to test the \emph{UMGYDE Model} with the \href{https://www.rdocumentation.org/packages/stats/versions/3.6.2/topics/loess}{loess} \texttt{degree} parameter set to \(1\).
\end{itemize}

\end{noteblock}

\begin{Shaded}
\begin{Highlighting}[]
\NormalTok{lss.fine\_tune.loop\_starter}
\end{Highlighting}
\end{Shaded}

\begin{verbatim}
## [1] 0.0005 0.0025 8.0000
\end{verbatim}

\begin{Shaded}
\begin{Highlighting}[]
\FunctionTok{str}\NormalTok{(lss.UMGYDE.fine\_tune.degree1.result)}
\end{Highlighting}
\end{Shaded}

\begin{verbatim}
## List of 3
##  $ best_result           : Named num [1:2] 0.00109 0.87083
##   ..- attr(*, "names")= chr [1:2] "param.best_value" "best_RMSE"
##  $ param_values.endpoints: num [1:3] 1.06e-03 1.31e-03 3.13e-05
##  $ tuned.result          :'data.frame':  9 obs. of  2 variables:
##   ..$ parameter.value: num [1:9] 0.00106 0.00109 0.00113 0.00116 0.00119 ...
##   ..$ RMSE           : num [1:9] 0.871 0.871 0.871 0.871 0.871 ...
\end{verbatim}

\newpage

The \href{https://github.com/AzKurban-edX-DS/Capstone-MovieLens/blob/main/r/src/capstone-movielens.main.R\#L1765}{code snippet} below provides a visual representation of the dependence of the \emph{RMSE} value on the \href{https://www.rdocumentation.org/packages/stats/versions/3.6.2/topics/loess}{loess} \texttt{span} parameter value over the neighborhood of its \emph{best value} (when the parameter \texttt{degree\ =\ 1}), which we have determined in this (\emph{fine-tuning}) step:

\begin{Shaded}
\begin{Highlighting}[]
\NormalTok{lss.UMGYDE.fine\_tune.degree1.result}\SpecialCharTok{$}\NormalTok{tuned.result }\SpecialCharTok{|\textgreater{}}
  \FunctionTok{data.plot}\NormalTok{(}\AttributeTok{title =} \StringTok{"Fine{-}tuned UMGYDE Model with \textasciigrave{}loess\textasciigrave{} parameter: \textasciigrave{}degree = 1\textasciigrave{}"}\NormalTok{, }
                             \AttributeTok{xname =} \StringTok{"parameter.value"}\NormalTok{, }
                             \AttributeTok{yname =} \StringTok{"RMSE"}\NormalTok{, }
                             \AttributeTok{xlabel =} \StringTok{"spans"}\NormalTok{, }
                             \AttributeTok{ylabel =} \StringTok{"RMSE"}\NormalTok{)}
\end{Highlighting}
\end{Shaded}

\pandocbounded{\includegraphics[keepaspectratio]{capstone-movielens.report_files/figure-latex/main-script@l1765-1.pdf}}

\begin{noteblock}
In the code snippet above, we use the custom data visualization function \hyperref[func.data.plot]{data.plot} described in section \hyperref[data-visualization-functions]{Data Visualization Functions} of \hyperref[appndx_a]{Appendix A}.

\end{noteblock}

\newpage

Now, we can add the best \emph{RMSE} figured out for the \emph{UMGYDE Fine-tuned Model} with the \href{https://www.rdocumentation.org/packages/stats/versions/3.6.2/topics/loess}{loess} \texttt{degree\ =\ 1} (let's call it \emph{Tuned UMGYDE.d1 Model}) to our \emph{Result Table} and print the table using the following \href{https://github.com/AzKurban-edX-DS/Capstone-MovieLens/blob/main/r/src/capstone-movielens.main.R\#L1773}{code snippet}:

\begin{Shaded}
\begin{Highlighting}[]
\NormalTok{RMSEs.ResultTibble.UMGYDE1 }\OtherTok{\textless{}{-}}\NormalTok{ RMSEs.ResultTibble.UMGYDE0 }\SpecialCharTok{|\textgreater{}} 
  \FunctionTok{RMSEs.AddRow}\NormalTok{(}\StringTok{"Tuned UMGYDE.d1 Model"}\NormalTok{, }
\NormalTok{               lss.UMGYDE.fine\_tune.degree1.result.best\_RMSE,}
               \AttributeTok{comment =} \StringTok{"UMGYDE Model computed using function call: \textasciigrave{}loess(degree = 1, span = \%1)\textasciigrave{}"} \SpecialCharTok{|\textgreater{}}
                 \FunctionTok{msg.glue}\NormalTok{(lss.UMGYDE.fine\_tune.degree1.result.best\_span))}
\end{Highlighting}
\end{Shaded}

\begin{Shaded}
\begin{Highlighting}[]
\FunctionTok{RMSE\_kable}\NormalTok{(RMSEs.ResultTibble.UMGYDE1)}
\end{Highlighting}
\end{Shaded}

\begin{tabular}{>{\raggedright\arraybackslash}p{15em}>{\centering\arraybackslash}p{5em}>{\raggedright\arraybackslash}p{30em}}
\toprule
\textbf{Method} & \textbf{RMSE} & \textbf{Comment}\\
\midrule
Project Objective & 0.8649000 & \\
OMR Model & 1.0603462 & Overall Mean Rating (OMR) Model\\
UE Model & 0.9697962 & User Effect (UE) Model\\
UME Model & 0.8732081 & User+Movie Effect (UME) Model\\
Regularized UME Model & 0.8729730 & Computed for `lambda` = 0.38745002746582\\
\addlinespace
UMGE Model & 0.8729730 & User+Movie+Genre Effect (UMGE) Model\\
Regularized UMGE Model & 0.8729730 & Computed for `lambda` = 0.0359375\\
UMGYE Model & 0.8723975 & User+Movie+Genre+Year Effect (UMGYE) Model\\
Regularized UMGYE Model & 0.8721860 & Computed for `lambda` = 233.784942626953\\
UMGYDE (Default) Model & 0.8722314 & User+Movie+Genre+Year+Day Effect (UMGYDE) Model 
computed using `stats::loess` function with `degree=1` \& `span=0.75` parameter values.\\
\addlinespace
Tuned UMGYDE.d0 Model & 0.8711135 & UMGYDE Model computed using function call: `loess(degree = 0, span = 0.00109375)`\\
Tuned UMGYDE.d1 Model & 0.8708300 & UMGYDE Model computed using function call: `loess(degree = 1, span = 0.00109375)`\\
\bottomrule
\end{tabular}

\begin{noteblock}
In the code snippet above we use the \hyperref[func.RMSEs.AddRow]{RMSEs.AddRow} and \hyperref[func.RMSE_kable]{RMSE\_kable} functions described in section \hyperref[result-rmses-tibble-functions]{Result RMSEs Tibble Functions} of \hyperref[appndx_a]{Appendix A}.

\end{noteblock}

\newpage

\paragraph{\texorpdfstring{UMGYDE Model Tuning: Stage 3 (\texttt{degree\ =\ 2})}{UMGYDE Model Tuning: Stage 3 (degree = 2)}}\label{umgyde-model-tuning-stage-3-degree-2}

\hfill\break

\subparagraph{\texorpdfstring{Pre-configuration: Optimal \texttt{span} Range Determination (\texttt{degree\ =\ 2})}{Pre-configuration: Optimal span Range Determination (degree = 2)}}\label{pre-configuration-optimal-span-range-determination-degree-2}

\hfill\break

\begin{noteblock}
The complete source code of the solution described in this section is available in the \href{https://github.com/AzKurban-edX-DS/Capstone-MovieLens/blob/main/r/src/capstone-movielens.main.R\#L1787}{UMGYDE Model Tuning: Pre-configuration (\texttt{degree\ =\ 2})} section of the \href{https://github.com/AzKurban-edX-DS/Capstone-MovieLens/blob/main/r/src/capstone-movielens.main.R\#L1787}{capstone-movielens.main.R} script on \emph{GitHub}.

\end{noteblock}

Below is the \href{https://github.com/AzKurban-edX-DS/Capstone-MovieLens/blob/main/r/src/capstone-movielens.main.R\#L1807}{code snippet} that determines the optimal range of \texttt{span} for the further model tuning when \texttt{degree\ =\ 2}:

\begin{Shaded}
\begin{Highlighting}[]
\NormalTok{  spans }\OtherTok{\textless{}{-}} \FunctionTok{seq}\NormalTok{(}\FloatTok{0.0005}\NormalTok{, }\DecValTok{1}\NormalTok{, }\FloatTok{0.001}\NormalTok{)}
\NormalTok{  lss.UMGYDE.preset.degree2.result }\OtherTok{\textless{}{-}} 
    \FunctionTok{tune.model\_param}\NormalTok{(spans, train\_UMGY\_SmoothedDay\_effect.RMSE.cv.degree2)}
\end{Highlighting}
\end{Shaded}

\begin{noteblock}

In the code snippet above, we use the following functions described in \hyperref[appndx_a]{Appendix A}:

\begin{itemize}
\item
  \hyperref[func.tune.model_param]{tune.model\_param} (described in section \hyperref[appndx_a.CHF.model_tuning_utils]{Model Tuning Utils}) to select the best value of the \emph{tuning parameter} (\texttt{span} in our case) from the given sequence of the \emph{parameter values} passed in the \hyperref[func.tune.model_param.args.param_values]{param\_values} argument.
\item
  \hyperref[func.train_UMGY_SmoothedDay_effect.RMSE.cv.degree2]{train\_UMGY\_SmoothedDay\_effect.RMSE.cv.degree2} (described in section \hyperref[umgyde-model-tuning-loess-params]{UMGYDE Model: Tuning \texttt{loess} Params}) used as an auxiliary function passed (in the \hyperref[func.tune.model_param.args.fn_tune.test.param_value]{fn\_tune.test.param\_value} argument) to the \hyperref[func.tune.model_param]{tune.model\_param} to test the \emph{UMGYDE Model} with the \href{https://www.rdocumentation.org/packages/stats/versions/3.6.2/topics/loess}{loess} \texttt{degree} parameter set to \(2\).
\end{itemize}

\end{noteblock}

\begin{Shaded}
\begin{Highlighting}[]
\FunctionTok{str}\NormalTok{(lss.UMGYDE.preset.degree2.result)}
\end{Highlighting}
\end{Shaded}

\begin{verbatim}
## List of 2
##  $ tuned.result:'data.frame':    5 obs. of  2 variables:
##   ..$ RMSE           : num [1:5] 0.872 0.871 0.871 0.871 0.871
##   ..$ parameter.value: num [1:5] 0.0005 0.0015 0.0025 0.0035 0.0045
##  $ best_result : Named num [1:2] 0.0015 0.871
##   ..- attr(*, "names")= chr [1:2] "param.best_value" "best_RMSE"
\end{verbatim}

\newpage

The \href{https://github.com/AzKurban-edX-DS/Capstone-MovieLens/blob/main/r/src/capstone-movielens.main.R\#L1838}{code snippet} below provides a visual representation of the dependence of the \emph{RMSE} value on the \href{https://www.rdocumentation.org/packages/stats/versions/3.6.2/topics/loess}{loess} \texttt{span} parameter value over the selected interval, with the parameter \texttt{degree\ =\ 2}, as we have figured out in this step of tuning:

\begin{Shaded}
\begin{Highlighting}[]
\NormalTok{lss.UMGYDE.preset.degree2.result}\SpecialCharTok{$}\NormalTok{tuned.result }\SpecialCharTok{|\textgreater{}}
  \FunctionTok{data.plot}\NormalTok{(}\AttributeTok{title =} \StringTok{"UMGYDE Model, \textasciigrave{}loess\textasciigrave{} function args: range of \textasciigrave{}span\textasciigrave{} values for \textasciigrave{}degree = 2\textasciigrave{}"}\NormalTok{,}
              \AttributeTok{xname =} \StringTok{"parameter.value"}\NormalTok{, }
              \AttributeTok{yname =} \StringTok{"RMSE"}\NormalTok{, }
              \AttributeTok{xlabel =} \StringTok{"spans"}\NormalTok{, }
              \AttributeTok{ylabel =} \StringTok{"RMSE"}\NormalTok{)}
\end{Highlighting}
\end{Shaded}

\pandocbounded{\includegraphics[keepaspectratio]{capstone-movielens.report_files/figure-latex/main-script@l1838-1.pdf}}

\begin{noteblock}
In the code snippet above, we use the custom data visualization function \hyperref[func.data.plot]{data.plot} described in section \hyperref[data-visualization-functions]{Data Visualization Functions} of \hyperref[appndx_a]{Appendix A}.

\end{noteblock}

\newpage

\subparagraph{\texorpdfstring{UMGYDE Model Fine-tuning (\texttt{degree\ =\ 2})}{UMGYDE Model Fine-tuning (degree = 2)}}\label{umgyde-model-fine-tuning-degree-2}

\hfill\break

\begin{noteblock}
The complete source code of the solution described in this section is available in the \href{https://github.com/AzKurban-edX-DS/Capstone-MovieLens/blob/main/r/src/capstone-movielens.main.R\#L1850}{UMGYDE Model Fine-tuning (\texttt{degree\ =\ 2})} section of the \href{https://github.com/AzKurban-edX-DS/Capstone-MovieLens/blob/main/r/src/capstone-movielens.main.R\#L1850}{capstone-movielens.main.R} script on \emph{GitHub}.

\end{noteblock}

The following \href{https://github.com/AzKurban-edX-DS/Capstone-MovieLens/blob/main/r/src/capstone-movielens.main.R\#L1851}{code snippet} performs the \emph{fine-tuning} of the \emph{UMGYDE Model} by precise selection of the \href{https://www.rdocumentation.org/packages/stats/versions/3.6.2/topics/loess}{loess} \texttt{span} parameter with the \texttt{degree} parameter set to \(2\):

\begin{Shaded}
\begin{Highlighting}[]
\NormalTok{lss.fine\_tune.loop\_starter }\OtherTok{\textless{}{-}} 
  \FunctionTok{c}\NormalTok{(lss.UMGYDE.preset.degree2.result}\SpecialCharTok{$}\NormalTok{tuned.result}\SpecialCharTok{$}\NormalTok{parameter.value[}\DecValTok{1}\NormalTok{], }
\NormalTok{    lss.UMGYDE.preset.degree2.result}\SpecialCharTok{$}\NormalTok{tuned.result}\SpecialCharTok{$}\NormalTok{parameter.value[}\DecValTok{3}\NormalTok{], }
    \DecValTok{8}\NormalTok{)}

\NormalTok{cache\_file.base\_name }\OtherTok{\textless{}{-}} \StringTok{"UMGYDE.degree2.tuning{-}span"}

\NormalTok{lss.UMGYDE.fine\_tune.degree2.result }\OtherTok{\textless{}{-}} 
  \FunctionTok{model.tune.param\_range}\NormalTok{(lss.fine\_tune.loop\_starter,}
\NormalTok{                         UMGYDE.fine\_tune.degree2.data.path,}
\NormalTok{                         cache\_file.base\_name,}
\NormalTok{                         train\_UMGY\_SmoothedDay\_effect.RMSE.cv.degree2)}
\end{Highlighting}
\end{Shaded}

\begin{noteblock}

In the code snippet above, we use the following functions described in \hyperref[appndx_a]{Appendix A: Support Functions}:

\begin{itemize}
\item
  \hyperref[func.model.tune.param_range]{model.tune.param\_range} (described in section \hyperref[appndx_a.CHF.model_tuning_utils]{Model Tuning Utils}) to enhance the accuracy of the \emph{tuning parameter best value} using the \emph{best value neighborhood} information passed in the \hyperref[func.model.tune.param_range.args.loop_starter]{loop\_starter} argument.
\item
  \hyperref[func.train_UMGY_SmoothedDay_effect.RMSE.cv.degree2]{train\_UMGY\_SmoothedDay\_effect.RMSE.cv.degree2} (already mentioned in the previous section) used as an auxiliary function passed (in the \hyperref[func.model.tune.param_range.args.fn_tune.test.param_value]{fn\_tune.test.param\_value} argument) to the \hyperref[func.model.tune.param_range]{model.tune.param\_range} to test the \emph{UMGYDE Model} with the \href{https://www.rdocumentation.org/packages/stats/versions/3.6.2/topics/loess}{loess} \texttt{degree} parameter set to \(2\).
\end{itemize}

\end{noteblock}

\begin{Shaded}
\begin{Highlighting}[]
\NormalTok{lss.fine\_tune.loop\_starter}
\end{Highlighting}
\end{Shaded}

\begin{verbatim}
## [1] 0.0005 0.0025 8.0000
\end{verbatim}

\begin{Shaded}
\begin{Highlighting}[]
\FunctionTok{str}\NormalTok{(lss.UMGYDE.fine\_tune.degree2.result)}
\end{Highlighting}
\end{Shaded}

\begin{verbatim}
## List of 3
##  $ best_result           : Named num [1:2] 0.00153 0.87099
##   ..- attr(*, "names")= chr [1:2] "param.best_value" "best_RMSE"
##  $ param_values.endpoints: num [1:3] 1.50e-03 1.75e-03 3.13e-05
##  $ tuned.result          :'data.frame':  9 obs. of  2 variables:
##   ..$ parameter.value: num [1:9] 0.0015 0.00153 0.00156 0.00159 0.00163 ...
##   ..$ RMSE           : num [1:9] 0.871 0.871 0.871 0.871 0.871 ...
\end{verbatim}

\newpage

The \href{https://github.com/AzKurban-edX-DS/Capstone-MovieLens/blob/main/r/src/capstone-movielens.main.R\#L1877}{code snippet} below provides a visual representation of the dependence of the \emph{RMSE} value on the \href{https://www.rdocumentation.org/packages/stats/versions/3.6.2/topics/loess}{loess} \texttt{span} parameter value over the neighborhood of its \emph{best value} (when the parameter \texttt{degree\ =\ 2}), which we have determined in this (\emph{fine-tuning}) step.
Since the left endpoint value of the \emph{RMSE} range has much smaller deviation from the minimum than the value of the right endpoint, for visual clarity, we plot the left part of the graph once more on an \emph{enlarged scale} (below the main graph):

\begin{Shaded}
\begin{Highlighting}[]
\NormalTok{lss.UMGYDE.fine\_tune.degree2.result}\SpecialCharTok{$}\NormalTok{tuned.result }\SpecialCharTok{|\textgreater{}}
  \FunctionTok{data.plot.left\_detailed}\NormalTok{(}\AttributeTok{title =} \StringTok{"Fine{-}tuned UMGYDE Model with \textasciigrave{}loess\textasciigrave{} parameter: \textasciigrave{}degree = 2\textasciigrave{}"}\NormalTok{, }
                             \AttributeTok{title.left =} \StringTok{"Left Part of the Chart Above (Zoomed in)"}\NormalTok{,}
                             \AttributeTok{left.n =} \DecValTok{8}\NormalTok{,}
                             \AttributeTok{xname =} \StringTok{"parameter.value"}\NormalTok{, }
                             \AttributeTok{yname =} \StringTok{"RMSE"}\NormalTok{, }
                             \AttributeTok{xlabel1 =} \StringTok{"spans"}\NormalTok{, }
                             \AttributeTok{ylabel1 =} \StringTok{"RMSE"}\NormalTok{)}
\end{Highlighting}
\end{Shaded}

\pandocbounded{\includegraphics[keepaspectratio]{capstone-movielens.report_files/figure-latex/main-script@l1877-1.pdf}}

\begin{noteblock}
In the code snippet above, we use the custom data visualization function \hyperref[func.data.plot.left_detailed]{data.plot.left\_detailed} described in section \hyperref[data-visualization-functions]{Data Visualization Functions} of \hyperref[appndx_a]{Appendix A}.

\end{noteblock}

\newpage

Now, we can add the best \emph{RMSE} value figured out for the \emph{UMGYDE Fine-tuned Model} with the \href{https://www.rdocumentation.org/packages/stats/versions/3.6.2/topics/loess}{loess} \texttt{degree\ =\ 2} (let's call it \emph{Tuned UMGYDE.d2 Model}) to our \emph{Result Table} and print the table using the following \href{https://github.com/AzKurban-edX-DS/Capstone-MovieLens/blob/main/r/src/capstone-movielens.main.R\#L1887}{code snippet}:

\begin{Shaded}
\begin{Highlighting}[]
\NormalTok{RMSEs.ResultTibble.UMGYDE2 }\OtherTok{\textless{}{-}}\NormalTok{ RMSEs.ResultTibble.UMGYDE1 }\SpecialCharTok{|\textgreater{}} 
  \FunctionTok{RMSEs.AddRow}\NormalTok{(}\StringTok{"Tuned UMGYDE.d2 Model"}\NormalTok{, }
\NormalTok{               lss.UMGYDE.fine\_tune.degree2.result.best\_RMSE,}
               \AttributeTok{comment =} \StringTok{"UMGYDE Model computed using function call: \textasciigrave{}loess(degree = 2, span = \%1)\textasciigrave{}"} \SpecialCharTok{|\textgreater{}}
                 \FunctionTok{msg.glue}\NormalTok{(lss.UMGYDE.fine\_tune.degree2.result.best\_span))}
\end{Highlighting}
\end{Shaded}

\begin{Shaded}
\begin{Highlighting}[]
\FunctionTok{RMSE\_kable}\NormalTok{(RMSEs.ResultTibble.UMGYDE2)}
\end{Highlighting}
\end{Shaded}

\begin{tabular}{>{\raggedright\arraybackslash}p{15em}>{\centering\arraybackslash}p{5em}>{\raggedright\arraybackslash}p{30em}}
\toprule
\textbf{Method} & \textbf{RMSE} & \textbf{Comment}\\
\midrule
Project Objective & 0.8649000 & \\
OMR Model & 1.0603462 & Overall Mean Rating (OMR) Model\\
UE Model & 0.9697962 & User Effect (UE) Model\\
UME Model & 0.8732081 & User+Movie Effect (UME) Model\\
Regularized UME Model & 0.8729730 & Computed for `lambda` = 0.38745002746582\\
\addlinespace
UMGE Model & 0.8729730 & User+Movie+Genre Effect (UMGE) Model\\
Regularized UMGE Model & 0.8729730 & Computed for `lambda` = 0.0359375\\
UMGYE Model & 0.8723975 & User+Movie+Genre+Year Effect (UMGYE) Model\\
Regularized UMGYE Model & 0.8721860 & Computed for `lambda` = 233.784942626953\\
UMGYDE (Default) Model & 0.8722314 & User+Movie+Genre+Year+Day Effect (UMGYDE) Model 
computed using `stats::loess` function with `degree=1` \& `span=0.75` parameter values.\\
\addlinespace
Tuned UMGYDE.d0 Model & 0.8711135 & UMGYDE Model computed using function call: `loess(degree = 0, span = 0.00109375)`\\
Tuned UMGYDE.d1 Model & 0.8708300 & UMGYDE Model computed using function call: `loess(degree = 1, span = 0.00109375)`\\
Tuned UMGYDE.d2 Model & 0.8709891 & UMGYDE Model computed using function call: `loess(degree = 2, span = 0.00153125)`\\
\bottomrule
\end{tabular}

\begin{noteblock}
In the code snippet above we use the \hyperref[func.RMSEs.AddRow]{RMSEs.AddRow} and \hyperref[func.RMSE_kable]{RMSE\_kable} functions described in section \hyperref[result-rmses-tibble-functions]{Result RMSEs Tibble Functions} of \hyperref[appndx_a]{Appendix A}.

\end{noteblock}

\newpage

\paragraph{\texorpdfstring{UMGYDE Model Tuning: Re-training on the \texttt{edx} with the Best Params}{UMGYDE Model Tuning: Re-training on the edx with the Best Params}}\label{umgyde-model-tuning-re-training-on-the-edx-with-the-best-params}

\hfill\break

\begin{noteblock}
The complete version of the source code provided in this section are available in the \href{https://github.com/AzKurban-edX-DS/Capstone-MovieLens/blob/main/r/src/capstone-movielens.main.R\#L1900}{UMGYDE Tuned Model: Retraining with the best params} section of the \href{https://github.com/AzKurban-edX-DS/Capstone-MovieLens/blob/main/r/src/capstone-movielens.main.R\#L1900}{capstone-movielens.main.R} script on \emph{GitHub}.

\end{noteblock}

Now, we can refine the current model by retraining on the entire \texttt{edx} dataset with the best values of the \href{https://www.rdocumentation.org/packages/stats/versions/3.6.2/topics/loess}{loess} parameters \texttt{degree} and \texttt{span} we just figured out (let's call it \emph{Tuned UMGYDE Best Model}), for the definitive \emph{RMSE} calculation and use in subsequent models:

The following \href{https://github.com/AzKurban-edX-DS/Capstone-MovieLens/blob/main/r/src/capstone-movielens.main.R\#L1913}{code snippet} performs this operation:

\begin{Shaded}
\begin{Highlighting}[]
\CommentTok{\# The Best Parameters and RMSE Value }
\NormalTok{lss.best\_results }\OtherTok{\textless{}{-}} \FunctionTok{data.frame}\NormalTok{(}\AttributeTok{degree =}\NormalTok{ degree, }
                         \AttributeTok{span =} \FunctionTok{c}\NormalTok{(lss.UMGYDE.fine\_tune.degree0.result.best\_span,}
\NormalTok{                                  lss.UMGYDE.fine\_tune.degree1.result.best\_span,}
\NormalTok{                                  lss.UMGYDE.fine\_tune.degree2.result.best\_span),}
                         
                         \AttributeTok{RMSE =} \FunctionTok{c}\NormalTok{(lss.UMGYDE.fine\_tune.degree0.result.best\_RMSE, }
\NormalTok{                                  lss.UMGYDE.fine\_tune.degree1.result.best\_RMSE,}
\NormalTok{                                  lss.UMGYDE.fine\_tune.degree2.result.best\_RMSE))}

\NormalTok{lss.best\_RMSE.idx }\OtherTok{\textless{}{-}} \FunctionTok{which.min}\NormalTok{(lss.best\_results}\SpecialCharTok{$}\NormalTok{RMSE)}


\NormalTok{lss.UMGYDE.best\_params }\OtherTok{\textless{}{-}} 
  \FunctionTok{c}\NormalTok{(}\AttributeTok{degree =}\NormalTok{  lss.best\_results[lss.best\_RMSE.idx, }\StringTok{"degree"}\NormalTok{],  }\CommentTok{\# 1}
    \AttributeTok{span =}\NormalTok{ lss.best\_results[lss.best\_RMSE.idx, }\StringTok{"span"}\NormalTok{], }\CommentTok{\# 0.00087,}
    \AttributeTok{RMSE =}\NormalTok{ lss.best\_results[lss.best\_RMSE.idx, }\StringTok{"RMSE"}\NormalTok{]) }\CommentTok{\# 0.8568619}

\NormalTok{lss.best\_degree }\OtherTok{\textless{}{-}}\NormalTok{ lss.UMGYDE.best\_params[}\StringTok{"degree"}\NormalTok{]}
\NormalTok{lss.best\_span }\OtherTok{\textless{}{-}}\NormalTok{ lss.UMGYDE.best\_params[}\StringTok{"span"}\NormalTok{]}
\NormalTok{lss.best\_RMSE }\OtherTok{\textless{}{-}}\NormalTok{ lss.UMGYDE.best\_params[}\StringTok{"RMSE"}\NormalTok{]}

\FunctionTok{put\_log2}\NormalTok{(}\StringTok{"Re{-}training model using \textasciigrave{}loess\textasciigrave{} function with the best parameters: }
\StringTok{span = \%1, degree = \%2"}\NormalTok{, lss.best\_span, lss.best\_degree)}

\NormalTok{lss.UMGYD\_effect }\OtherTok{\textless{}{-}}\NormalTok{ edx }\SpecialCharTok{|\textgreater{}} 
  \FunctionTok{train\_UMGY\_SmoothedDay\_effect}\NormalTok{(lss.best\_degree, lss.best\_span)}
\end{Highlighting}
\end{Shaded}

\begin{noteblock}
In the code snippet above we use the \hyperref[func.train_UMGY_SmoothedDay_effect]{train\_UMGY\_SmoothedDay\_effect} custom helper function described in section \hyperref[umgyde-model-utility-functions]{UMGYDE Model: Utility Functions} of \hyperref[appndx_a]{Appendix A}.

\end{noteblock}

\newpage

\begin{Shaded}
\begin{Highlighting}[]
\NormalTok{lss.UMGYDE.best\_params}
\end{Highlighting}
\end{Shaded}

\begin{verbatim}
##     degree       span       RMSE 
## 1.00000000 0.00109375 0.87083000
\end{verbatim}

\begin{Shaded}
\begin{Highlighting}[]
\FunctionTok{str}\NormalTok{(lss.UMGYD\_effect)}
\end{Highlighting}
\end{Shaded}

\begin{verbatim}
## tibble [4,640 x 4] (S3: tbl_df/tbl/data.frame)
##  $ days       : int [1:4640] 0 385 388 389 392 393 394 396 397 399 ...
##  $ de         : num [1:4640] 0.2106 0.2506 0.1395 0.2797 -0.0392 ...
##  $ year       : num [1:4640] 1995 1996 1996 1996 1996 ...
##  $ de_smoothed: num [1:4640] 0.228 0.242 0.205 0.183 0.44 ...
\end{verbatim}

The following \href{https://github.com/AzKurban-edX-DS/Capstone-MovieLens/blob/main/r/src/capstone-movielens.main.R\#L1993}{code snippet} provides a visual representation of the final \emph{Smoothed Day Effect} data we have just obtained:

\begin{Shaded}
\begin{Highlighting}[]
\NormalTok{lss.UMGYD\_effect }\SpecialCharTok{|\textgreater{}}
  \FunctionTok{ggplot}\NormalTok{(}\FunctionTok{aes}\NormalTok{(}\AttributeTok{x =}\NormalTok{ days)) }\SpecialCharTok{+}
  \FunctionTok{geom\_point}\NormalTok{(}\FunctionTok{aes}\NormalTok{(}\AttributeTok{y =}\NormalTok{ de), }\AttributeTok{size =} \DecValTok{3}\NormalTok{, }\AttributeTok{alpha =}\NormalTok{ .}\DecValTok{5}\NormalTok{, }\AttributeTok{color =} \StringTok{"grey"}\NormalTok{) }\SpecialCharTok{+} 
  \FunctionTok{geom\_line}\NormalTok{(}\FunctionTok{aes}\NormalTok{(}\AttributeTok{y =}\NormalTok{ de\_smoothed), }\AttributeTok{color =} \StringTok{"red"}\NormalTok{)}
\end{Highlighting}
\end{Shaded}

\pandocbounded{\includegraphics[keepaspectratio]{capstone-movielens.report_files/figure-latex/main-script@l1993-1.pdf}}

\begin{noteblock}
It should be noted, however, that the graph above is not as smooth as when training the model with default {[}loess{]} parameters (\texttt{degree} and \texttt{span}), although we have now obtained a better (\emph{RMSE}) result.

\end{noteblock}

\newpage

Now, we are ready to construct predictors and calculate the \emph{RMSE} score for the ultimately \emph{Tuned UMGYDE Best Model} using the following \href{https://github.com/AzKurban-edX-DS/Capstone-MovieLens/blob/main/r/src/capstone-movielens.main.R\#L2007}{line of code}:

\begin{Shaded}
\begin{Highlighting}[]
\NormalTok{lss.UMGYD\_effect.RMSE }\OtherTok{\textless{}{-}} \FunctionTok{calc\_UMGY\_SmoothedDay\_effect.RMSE.cv}\NormalTok{(lss.UMGYD\_effect)}
\end{Highlighting}
\end{Shaded}

\begin{Shaded}
\begin{Highlighting}[]
\NormalTok{lss.UMGYD\_effect.RMSE}
\end{Highlighting}
\end{Shaded}

\begin{verbatim}
## [1] 0.8707855
\end{verbatim}

\begin{noteblock}
In the code snippets above, we use the \hyperref[func.calc_UMGY_SmoothedDay_effect.RMSE.cv]{calc\_UMGY\_SmoothedDay\_effect.RMSE.cv} custom helper function described in section \hyperref[umgyde-model-utility-functions]{UMGYDE Model: Utility Functions} of \hyperref[appndx_a]{Appendix A}.

\end{noteblock}

Finally, we add the \emph{RMSE} value obtained above for the fully tuned model (let's call it \emph{Tuned UMGYDE Best Model}) to our \emph{Result Table} and print the table using the following \href{https://github.com/AzKurban-edX-DS/Capstone-MovieLens/blob/main/r/src/capstone-movielens.main.R\#L2024}{code snippet}:

\begin{Shaded}
\begin{Highlighting}[]
\NormalTok{RMSEs.ResultTibble.UMGYDE.tuned }\OtherTok{\textless{}{-}}\NormalTok{ RMSEs.ResultTibble.UMGYDE2 }\SpecialCharTok{|\textgreater{}} 
  \FunctionTok{RMSEs.AddRow}\NormalTok{(}\StringTok{"Tuned UMGYDE Best Model"}\NormalTok{, }
\NormalTok{               lss.UMGYD\_effect.RMSE,}
               \AttributeTok{comment =} \StringTok{"UMGYDE Model computed using \textasciigrave{}loess\textasciigrave{} function call with the best degree \& span values."}\NormalTok{)}
\end{Highlighting}
\end{Shaded}

\begin{Shaded}
\begin{Highlighting}[]
\FunctionTok{RMSE\_kable}\NormalTok{(RMSEs.ResultTibble.UMGYDE.tuned)}
\end{Highlighting}
\end{Shaded}

\begin{tabular}{>{\raggedright\arraybackslash}p{15em}>{\centering\arraybackslash}p{5em}>{\raggedright\arraybackslash}p{30em}}
\toprule
\textbf{Method} & \textbf{RMSE} & \textbf{Comment}\\
\midrule
Project Objective & 0.8649000 & \\
OMR Model & 1.0603462 & Overall Mean Rating (OMR) Model\\
UE Model & 0.9697962 & User Effect (UE) Model\\
UME Model & 0.8732081 & User+Movie Effect (UME) Model\\
Regularized UME Model & 0.8729730 & Computed for `lambda` = 0.38745002746582\\
\addlinespace
UMGE Model & 0.8729730 & User+Movie+Genre Effect (UMGE) Model\\
Regularized UMGE Model & 0.8729730 & Computed for `lambda` = 0.0359375\\
UMGYE Model & 0.8723975 & User+Movie+Genre+Year Effect (UMGYE) Model\\
Regularized UMGYE Model & 0.8721860 & Computed for `lambda` = 233.784942626953\\
UMGYDE (Default) Model & 0.8722314 & User+Movie+Genre+Year+Day Effect (UMGYDE) Model 
computed using `stats::loess` function with `degree=1` \& `span=0.75` parameter values.\\
\addlinespace
Tuned UMGYDE.d0 Model & 0.8711135 & UMGYDE Model computed using function call: `loess(degree = 0, span = 0.00109375)`\\
Tuned UMGYDE.d1 Model & 0.8708300 & UMGYDE Model computed using function call: `loess(degree = 1, span = 0.00109375)`\\
Tuned UMGYDE.d2 Model & 0.8709891 & UMGYDE Model computed using function call: `loess(degree = 2, span = 0.00153125)`\\
Tuned UMGYDE Best Model & 0.8707855 & UMGYDE Model computed using `loess` function call with the best degree \& span values.\\
\bottomrule
\end{tabular}

\begin{noteblock}
In the code snippet above we use the \hyperref[func.RMSEs.AddRow]{RMSEs.AddRow} and \hyperref[func.RMSE_kable]{RMSE\_kable} functions described in section \hyperref[result-rmses-tibble-functions]{Result RMSEs Tibble Functions} of \hyperref[appndx_a]{Appendix A}.

\end{noteblock}

Now, as the final stage of the tuning process, let's \emph{regularize} our model (as we did with the previous models).

\newpage

\subsubsection{UMGYDE Model Regularization}\label{umgyde-model-regularization}

\hfill\break

\begin{noteblock}
The complete version of the source code provided in this section can be found in the \href{https://github.com/AzKurban-edX-DS/Capstone-MovieLens/blob/main/r/src/capstone-movielens.main.R\#L2039}{UMGYDE Model Regularization} section of the \href{https://github.com/AzKurban-edX-DS/Capstone-MovieLens/blob/main/r/src/capstone-movielens.main.R\#L2039}{capstone-movielens.main.R} script.

\end{noteblock}

We begin this section with the concept's mathematical description presented below in the subsection \hyperref[umgyde-model-regularization-mathematical-description]{UMGYDE Model Regularization: \emph{Mathematical Description}}.

Next, we will implement the \emph{UMGYDE Model Regularization} in the following three steps:

\begin{enumerate}
\def\labelenumi{\arabic{enumi}.}
\item
  \textbf{Pre-configuration:} (Described in subsection \hyperref[umgyde-model-regularization-pre-configuration]{UMGYDE Model Regularization: \emph{Pre-configuration}}) Preliminary determination of the optimal range of \emph{regularization parameter} \(\lambda\) values for the \emph{K-Fold Cross-Validation} samples, where the \(K\) is the length of the \hyperref[appndx_b.edx_CV]{\texttt{edx\_CV} Object} (described in detail in \hyperref[appndx_b]{Appendix B: Models Training Datasets}) to which the \emph{K-Fold Cross-Validation} is applied (in \emph{this Project}, we use \(K = 5\));
\item
  \textbf{Fine-tuning:} (Described in the subsection \hyperref[umgyde-model-regularization-fine-tuning]{UMGYDE Model Regularization: \emph{Fine-tuning}}) Determining the best value of \(\lambda\) with the highest possible accuracy that minimizes the \emph{RMSE} score for the model.
\item
  \textbf{Retraining:} (Described in subsection \hyperref[umgyde-model-regularization-retraining-on-the-edx-with-the-best-lambda]{UMGYDE Model Regularization: \emph{Retraining on the \texttt{edx} with the best} \(\lambda\)}) Retraining the model on the entire \texttt{edx} dataset with the best value of \(\lambda\) determined in the previous step.
\end{enumerate}

\paragraph{\texorpdfstring{UMGYDE Model Regularization: \emph{Mathematical Description}}{UMGYDE Model Regularization: Mathematical Description}}\label{umgyde-model-regularization-mathematical-description}

\hfill\break

We have already explained the idea of the \emph{Linear Model Regularization} earlier in section \hyperref[ume-model-regularization]{UME Model Regularization}. We have also seen how the formula \eqref{eq:ME-penalty} for adding a penalty to the \emph{UME Model} is transformed into the formula \eqref{eq:GE-penalty} for the \emph{UMGE Model} and then into the formula \eqref{eq:YE-penalty} for the \emph{UMGYE Model}. For the current model, this formula takes the form:

\begin{equation}
\sum_{i,j} \left(y_{u,i} - \mu - \alpha_i - \beta_j - g_{i,j} - \gamma(v_{i,j}) - s(d_{i,j})\right)^2 + \lambda \sum_{i,j} s(d_{i,j})^2
\label{eq:DE-penalty}
\end{equation}

And the formula \eqref{eq:YE-regularized} for calculating the values of the \emph{treatment effect} that minimizes the equation will take the form:

\begin{equation}
\hat{s}(d_{i,j}, \lambda) = \frac{1}{\lambda + n_d} \sum_{r=1}^{n_{d}} \left(Y_{i,j} - \mu - \alpha_i - \beta_j - g_{i,j} - \gamma(v_{i,j})\right)
\label{eq:DE-regularized}
\end{equation}

where \(n_d\) is the number of ratings made on the day \(d\).

\newpage

\paragraph{\texorpdfstring{UMGYDE Model Regularization: \emph{Pre-configuration}}{UMGYDE Model Regularization: Pre-configuration}}\label{umgyde-model-regularization-pre-configuration}

\hfill\break

\begin{noteblock}
The complete version of the source code shown in this section is available in the \href{https://github.com/AzKurban-edX-DS/Capstone-MovieLens/blob/main/r/src/capstone-movielens.main.R\#L2053}{UMGYDE Model Regularization: Pre-configuration} section of the \href{https://github.com/AzKurban-edX-DS/Capstone-MovieLens/blob/main/r/src/capstone-movielens.main.R\#L2053}{capstone-movielens.main.R} script on \emph{GitHub}.

\end{noteblock}

As for previous models, we will use the function \hyperref[func.tune.model_param]{tune.model\_param} passing in the \hyperref[func.tune.model_param.args.fn_tune.test.param_value]{fn\_tune.test.param\_value} argument the model-specific helper function (this time designed for the \emph{UMGYDE Model}): \hyperref[func.regularize.test_lambda.UMGYD_effect.cv]{regularize.test\_lambda.UMGYD\_effect.cv}.

\begin{noteblock}
The functions \hyperref[func.tune.model_param]{tune.model\_param} and \hyperref[func.regularize.test_lambda.UMGYD_effect.cv]{regularize.test\_lambda.UMGYD\_effect.cv} are described in the sections \hyperref[appndx_a.CHF.model_tuning_utils]{Model Tuning Utils} and \hyperref[umgye-model-regularization-1]{UMGYE Model: Regularization}, respectively, of \hyperref[appndx_a]{Appendix A}.

\end{noteblock}

The following \href{https://github.com/AzKurban-edX-DS/Capstone-MovieLens/blob/main/r/src/capstone-movielens.main.R\#L2070}{piece of code} performs this operation:

\begin{Shaded}
\begin{Highlighting}[]
\NormalTok{  lambdas }\OtherTok{\textless{}{-}} \FunctionTok{seq}\NormalTok{(}\DecValTok{0}\NormalTok{, }\DecValTok{256}\NormalTok{, }\DecValTok{16}\NormalTok{)}
\NormalTok{  cv.UMGYDE.preset.result }\OtherTok{\textless{}{-}} 
    \FunctionTok{tune.model\_param}\NormalTok{(lambdas, regularize.test\_lambda.UMGYD\_effect.cv)}
\end{Highlighting}
\end{Shaded}

\begin{Shaded}
\begin{Highlighting}[]
\FunctionTok{str}\NormalTok{(cv.UMGYDE.preset.result)}
\end{Highlighting}
\end{Shaded}

\begin{verbatim}
## List of 2
##  $ tuned.result:'data.frame':    5 obs. of  2 variables:
##   ..$ RMSE           : num [1:5] 0.871 0.871 0.871 0.871 0.871
##   ..$ parameter.value: num [1:5] 0 16 32 48 64
##  $ best_result : Named num [1:2] 16 0.871
##   ..- attr(*, "names")= chr [1:2] "param.best_value" "best_RMSE"
\end{verbatim}

\begin{Shaded}
\begin{Highlighting}[]
\NormalTok{cv.UMGYDE.preset.result}\SpecialCharTok{$}\NormalTok{best\_result}
\end{Highlighting}
\end{Shaded}

\begin{verbatim}
## param.best_value        best_RMSE 
##       16.0000000        0.8708202
\end{verbatim}

\newpage

Now, let's visualize the results of the \(\lambda\) range pre-configuration using the following \href{https://github.com/AzKurban-edX-DS/Capstone-MovieLens/blob/main/r/src/capstone-movielens.main.R\#L2098}{code snippet}:

\begin{Shaded}
\begin{Highlighting}[]
\NormalTok{cv.UMGYDE.preset.result}\SpecialCharTok{$}\NormalTok{tuned.result }\SpecialCharTok{|\textgreater{}}
  \FunctionTok{data.plot}\NormalTok{(}\AttributeTok{title =} \FunctionTok{TeX}\NormalTok{(r}\StringTok{\textquotesingle{}[UMGYDE Model Regularization: $\textbackslash{}lambda$ Range Pre{-}configuration]\textquotesingle{}}\NormalTok{),}
              \AttributeTok{xname =} \StringTok{"parameter.value"}\NormalTok{, }
              \AttributeTok{yname =} \StringTok{"RMSE"}\NormalTok{, }
              \AttributeTok{xlabel =} \FunctionTok{TeX}\NormalTok{(r}\StringTok{\textquotesingle{}[$\textbackslash{}lambda$]\textquotesingle{}}\NormalTok{), }
              \AttributeTok{ylabel =} \StringTok{"RMSE"}\NormalTok{)}
\end{Highlighting}
\end{Shaded}

\pandocbounded{\includegraphics[keepaspectratio]{capstone-movielens.report_files/figure-latex/main-script@l2098-1.pdf}}

\begin{noteblock}
In the code snippet above, we use the custom data visualization function \hyperref[func.data.plot]{data.plot} described in section \hyperref[data-visualization-functions]{Data Visualization Functions} of \hyperref[appndx_a]{Appendix A}.

\end{noteblock}

\newpage

\paragraph{\texorpdfstring{UMGYDE Model Regularization: \emph{Fine-tuning}}{UMGYDE Model Regularization: Fine-tuning}}\label{umgyde-model-regularization-fine-tuning}

\hfill\break

\begin{noteblock}
The complete version of the source code shown in this section can be found in the \href{https://github.com/AzKurban-edX-DS/Capstone-MovieLens/blob/main/r/src/capstone-movielens.main.R\#L2109}{UMGYDE Model Regularization: Fine-tuning} section of the \href{https://github.com/AzKurban-edX-DS/Capstone-MovieLens/blob/main/r/src/capstone-movielens.main.R\#L2109}{capstone-movielens.main.R} script on \emph{GitHub}.

\end{noteblock}

We are now ready to perform the fine-tuning step of our model \emph{regularization} process to determine the best value for the \(\lambda\) parameter.

The following \href{https://github.com/AzKurban-edX-DS/Capstone-MovieLens/blob/main/r/src/capstone-movielens.main.R\#L2110}{piece of code} prepares the interval of values for the \(\lambda\) parameter, over which the operation has to be done:

\begin{Shaded}
\begin{Highlighting}[]
\NormalTok{endpoints }\OtherTok{\textless{}{-}} 
  \FunctionTok{get\_fine\_tune.param.endpoints}\NormalTok{(cv.UMGYDE.preset.result}\SpecialCharTok{$}\NormalTok{tuned.result)}

\NormalTok{UMGYDE.loop\_starter }\OtherTok{\textless{}{-}} \FunctionTok{c}\NormalTok{(endpoints[}\StringTok{"start"}\NormalTok{], }
\NormalTok{                        endpoints[}\StringTok{"end"}\NormalTok{], }
                        \DecValTok{8}\NormalTok{)}
\end{Highlighting}
\end{Shaded}

\begin{noteblock}
The helper function \hyperref[func.get_fine_tune.param.endpoints]{get\_fine\_tune.param.endpoints} used in code snippet above is described in the sections \hyperref[appndx_a.CHF.model_tuning_utils]{Model Tuning Utils} of \hyperref[appndx_a]{Appendix A}.

\end{noteblock}

\begin{verbatim}
## *** Values of the endpoints and the divisor for the interval of `lambda` values ***
\end{verbatim}

\begin{Shaded}
\begin{Highlighting}[]
\NormalTok{UMGYDE.loop\_starter}
\end{Highlighting}
\end{Shaded}

\begin{verbatim}
## start   end       
##     0    32     8
\end{verbatim}

And the next \href{https://github.com/AzKurban-edX-DS/Capstone-MovieLens/blob/main/r/src/capstone-movielens.main.R\#L2120}{code snippet} below accomplishes the task of \emph{fine-tuning} the model:

\begin{Shaded}
\begin{Highlighting}[]
\NormalTok{cache.base\_name }\OtherTok{\textless{}{-}} \StringTok{"UMGYDE.rglr.fine{-}tuning"}

\NormalTok{UMGYDE.rglr.fine\_tune.results }\OtherTok{\textless{}{-}} 
  \FunctionTok{model.tune.param\_range}\NormalTok{(UMGYDE.loop\_starter,}
\NormalTok{                         UMGYDE.rglr.fine\_tune.cache.path,}
\NormalTok{                         cache.base\_name,}
\NormalTok{                         regularize.test\_lambda.UMGYD\_effect.cv)}

\NormalTok{UMGYDE.rglr.fine\_tune.RMSE.best }\OtherTok{\textless{}{-}}\NormalTok{ UMGYDE.rglr.fine\_tune.results}\SpecialCharTok{$}\NormalTok{best\_result[}\StringTok{"best\_RMSE"}\NormalTok{]}
\end{Highlighting}
\end{Shaded}

\begin{noteblock}
The custom functions \hyperref[func.model.tune.param_range]{model.tune.param\_range} and \hyperref[func.regularize.test_lambda.UMGYD_effect.cv]{regularize.test\_lambda.UMGYD\_effect.cv} used in the code snippet above are described in the sections \hyperref[appndx_a.CHF.model_tuning_utils]{Model Tuning Utils} and \hyperref[umgyde-model-regularization-1]{UMGYDE Model: Regularization}, respectively, of \hyperref[appndx_a]{Appendix A}.

\end{noteblock}

\newpage

Below are the results of fine-tuning the \emph{UMGYDE Model}:

\begin{verbatim}
## *** Path to the cache directory for intermediate fine-tuning results ***
\end{verbatim}

\begin{Shaded}
\begin{Highlighting}[]
\NormalTok{UMGYDE.rglr.fine\_tune.cache.path}
\end{Highlighting}
\end{Shaded}

\begin{verbatim}
## [1] "data/regularization/4.UMGYD-effect/fine-tune"
\end{verbatim}

\begin{verbatim}
## *** Fine-tuning results object data structure ***
\end{verbatim}

\begin{Shaded}
\begin{Highlighting}[]
\FunctionTok{str}\NormalTok{(UMGYDE.rglr.fine\_tune.results)}
\end{Highlighting}
\end{Shaded}

\begin{verbatim}
## List of 3
##  $ best_result           : Named num [1:2] 10.716 0.871
##   ..- attr(*, "names")= chr [1:2] "param.best_value" "best_RMSE"
##  $ param_values.endpoints: Named num [1:3] 1.07e+01 1.07e+01 2.29e-05
##   ..- attr(*, "names")= chr [1:3] "" "" ""
##  $ tuned.result          :'data.frame':  9 obs. of  2 variables:
##   ..$ parameter.value: num [1:9] 10.7 10.7 10.7 10.7 10.7 ...
##   ..$ RMSE           : num [1:9] 0.871 0.871 0.871 0.871 0.871 ...
\end{verbatim}

\begin{verbatim}
## *** Fine-tuning: best results ***
\end{verbatim}

\begin{Shaded}
\begin{Highlighting}[]
\NormalTok{UMGYDE.rglr.fine\_tune.results}\SpecialCharTok{$}\NormalTok{best\_result}
\end{Highlighting}
\end{Shaded}

\begin{verbatim}
## param.best_value        best_RMSE 
##       10.7158432        0.8708198
\end{verbatim}

\begin{Shaded}
\begin{Highlighting}[]
\NormalTok{UMGYDE.rglr.fine\_tune.RMSE.best}
\end{Highlighting}
\end{Shaded}

\begin{verbatim}
## best_RMSE 
## 0.8708198
\end{verbatim}

\newpage

The following \href{https://github.com/AzKurban-edX-DS/Capstone-MovieLens/blob/main/r/src/capstone-movielens.main.R\#L2135}{code snippet} provides a visual representation of the \emph{fine-tuning results} we have just computed:

\begin{Shaded}
\begin{Highlighting}[]
\NormalTok{UMGYDE.rglr.fine\_tune.results}\SpecialCharTok{$}\NormalTok{tuned.result }\SpecialCharTok{|\textgreater{}}
  \FunctionTok{data.plot}\NormalTok{(}\AttributeTok{title =} \StringTok{"UMGYDE Model Regularization: Fine{-}tuned result"}\NormalTok{,}
              \AttributeTok{xname =} \StringTok{"parameter.value"}\NormalTok{,}
              \AttributeTok{yname =} \StringTok{"RMSE"}\NormalTok{,}
              \AttributeTok{xlabel =} \FunctionTok{TeX}\NormalTok{(r}\StringTok{\textquotesingle{}[$\textbackslash{}lambda$]\textquotesingle{}}\NormalTok{),}
              \AttributeTok{ylabel =} \FunctionTok{str\_glue}\NormalTok{(}\StringTok{"Deviation from the best RMSE value ("}\NormalTok{,}
                                \FunctionTok{as.character}\NormalTok{(}\FunctionTok{round}\NormalTok{(UMGYDE.rglr.fine\_tune.RMSE.best, }\AttributeTok{digits =} \DecValTok{7}\NormalTok{)),}
                                \StringTok{")"}\NormalTok{),}
              \AttributeTok{normalize =} \ConstantTok{TRUE}\NormalTok{)}
\end{Highlighting}
\end{Shaded}

\pandocbounded{\includegraphics[keepaspectratio]{capstone-movielens.report_files/figure-latex/main-script@l2135-1.pdf}}

\begin{noteblock}
Note that in the code snippet above, we use the custom data visualization function \hyperref[func.data.plot]{data.plot} (described in section \hyperref[data-visualization-functions]{Data Visualization Functions} of \hyperref[appndx_a]{Appendix A}) with the argument \hyperref[func.data.plot.args.normalize]{normalize} set to \texttt{TRUE}, which means that deviations from the minimum \(y\) value are used to plot, rather than the \(y\) values themselves.

\end{noteblock}

\newpage

\paragraph{\texorpdfstring{UMGYDE Model Regularization: \emph{Retraining on the \texttt{edx} with the best} \(\lambda\)}{UMGYDE Model Regularization: Retraining on the edx with the best \textbackslash lambda}}\label{umgyde-model-regularization-retraining-on-the-edx-with-the-best-lambda}

\hfill\break

\begin{noteblock}
The complete version of the source code shown in this section is available in the \href{https://github.com/AzKurban-edX-DS/Capstone-MovieLens/blob/main/r/src/capstone-movielens.main.R\#L2148}{UMGYDE Model Regularization: Retraining with the best params} section of the \href{https://github.com/AzKurban-edX-DS/Capstone-MovieLens/blob/main/r/src/capstone-movielens.main.R\#L2148}{capstone-movielens.main.R} script on \emph{GitHub}.

\end{noteblock}

Now, we can refine the \emph{UMGYDE Model} by retraining on the entire \texttt{edx} dataset with the best value of the \(\lambda\) parameter we just figured out (let's call it \emph{Regularized UMGYDE Model}), for the definitive \emph{RMSE} calculation and use in subsequent models:

The following \href{https://github.com/AzKurban-edX-DS/Capstone-MovieLens/blob/main/r/src/capstone-movielens.main.R\#L2161}{code snippet} performs this operation:

\begin{Shaded}
\begin{Highlighting}[]
\NormalTok{best\_result }\OtherTok{\textless{}{-}}\NormalTok{ UMGYDE.rglr.fine\_tune.results}\SpecialCharTok{$}\NormalTok{best\_result}
\NormalTok{UMGYDE.rglr.best\_lambda }\OtherTok{\textless{}{-}}\NormalTok{ best\_result[}\StringTok{"param.best\_value"}\NormalTok{]}

\FunctionTok{put\_log1}\NormalTok{(}\StringTok{"Re{-}training Regularized User+Movie+Genre+Year+(Smoothed)Day Effect Model for the best \textasciigrave{}lambda\textasciigrave{}: \%1..."}\NormalTok{,}
\NormalTok{         UMGYDE.rglr.best\_lambda)}

\NormalTok{rglr.UMGYD\_effect }\OtherTok{\textless{}{-}}\NormalTok{ edx }\SpecialCharTok{|\textgreater{}} 
  \FunctionTok{regularize.train\_UMGYD\_effect}\NormalTok{(UMGYDE.rglr.best\_lambda)}
\end{Highlighting}
\end{Shaded}

\begin{noteblock}
In the code snippet above we use the \hyperref[func.regularize.train_UMGYD_effect]{regularize.train\_UMGYD\_effect} function described in section \hyperref[umgyde-model-regularization-1]{UMGYDE Model: Regularization} of \hyperref[appndx_a]{Appendix A}.

\end{noteblock}

\begin{verbatim}
## *** The Best UMGYD Effect Fine-tuning Results ***
\end{verbatim}

\begin{Shaded}
\begin{Highlighting}[]
\NormalTok{UMGYDE.rglr.fine\_tune.results}\SpecialCharTok{$}\NormalTok{best\_result}
\end{Highlighting}
\end{Shaded}

\begin{verbatim}
## param.best_value        best_RMSE 
##       10.7158432        0.8708198
\end{verbatim}

\begin{verbatim}
## *** Regularized UMGYD Effect Structure ***
\end{verbatim}

\begin{Shaded}
\begin{Highlighting}[]
  \FunctionTok{str}\NormalTok{(rglr.UMGYD\_effect)}
\end{Highlighting}
\end{Shaded}

\begin{verbatim}
## tibble [4,640 x 4] (S3: tbl_df/tbl/data.frame)
##  $ days       : int [1:4640] 0 385 388 389 392 393 394 396 397 399 ...
##  $ de         : num [1:4640] 0.0331 0.1537 0.1174 0.239 -0.0277 ...
##  $ year       : num [1:4640] 1995 1996 1996 1996 1996 ...
##  $ de_smoothed: num [1:4640] 0.025 0.1492 0.1638 0.1565 0.0627 ...
\end{verbatim}

\begin{Shaded}
\begin{Highlighting}[]
\FunctionTok{print\_log1}\NormalTok{(}\StringTok{"Regularized UMGYDE Model has been re{-}trained for the best \textasciigrave{}lambda\textasciigrave{}: \%1."}\NormalTok{,}
\NormalTok{         UMGYDE.rglr.best\_lambda)}
\end{Highlighting}
\end{Shaded}

\begin{verbatim}
## Regularized UMGYDE Model has been re-trained for the best `lambda`: 10.7158432006836.
\end{verbatim}

\newpage

Now, we are ready to construct predictors and calculate the \emph{RMSE} score for the ultimately \emph{Regularized UMGYE Model} using the following \href{https://github.com/AzKurban-edX-DS/Capstone-MovieLens/blob/main/r/src/capstone-movielens.main.R\#L2172}{line of code}:

\begin{Shaded}
\begin{Highlighting}[]
\NormalTok{  rglr.UMGYD\_effect.RMSE }\OtherTok{\textless{}{-}} \FunctionTok{calc\_UMGY\_SmoothedDay\_effect.RMSE.cv}\NormalTok{(rglr.UMGYD\_effect)}
\end{Highlighting}
\end{Shaded}

\begin{noteblock}
In the code snippets above, we use the \hyperref[func.calc_UMGY_SmoothedDay_effect.RMSE.cv]{calc\_UMGY\_SmoothedDay\_effect.RMSE.cv} user-defined described in section \hyperref[umgyde-model-utility-functions]{UMGYDE Model: Utility Functions} of \hyperref[appndx_a]{Appendix A}.

\end{noteblock}

Finally, we add the definitive \emph{RMSE} value obtained above to our \emph{Result Table} and print the table using the following \href{https://github.com/AzKurban-edX-DS/Capstone-MovieLens/blob/main/r/src/capstone-movielens.main.R\#L2218}{code snippet}:

\begin{Shaded}
\begin{Highlighting}[]
\NormalTok{RMSEs.ResultTibble.UMGYDE.rglr.tuned }\OtherTok{\textless{}{-}}\NormalTok{ RMSEs.ResultTibble.UMGYDE.tuned }\SpecialCharTok{|\textgreater{}} 
  \FunctionTok{RMSEs.AddRow}\NormalTok{(}\StringTok{"Regularized UMGYDE Model"}\NormalTok{, }
\NormalTok{               rglr.UMGYD\_effect.RMSE,}
               \AttributeTok{comment =} \StringTok{"The best tuned and regularized UMGYDE Model."}\NormalTok{)}
\end{Highlighting}
\end{Shaded}

\begin{Shaded}
\begin{Highlighting}[]
\FunctionTok{RMSE\_kable}\NormalTok{(RMSEs.ResultTibble.UMGYDE.rglr.tuned)}
\end{Highlighting}
\end{Shaded}

\begin{tabular}{>{\raggedright\arraybackslash}p{15em}>{\centering\arraybackslash}p{5em}>{\raggedright\arraybackslash}p{30em}}
\toprule
\textbf{Method} & \textbf{RMSE} & \textbf{Comment}\\
\midrule
Project Objective & 0.8649000 & \\
OMR Model & 1.0603462 & Overall Mean Rating (OMR) Model\\
UE Model & 0.9697962 & User Effect (UE) Model\\
UME Model & 0.8732081 & User+Movie Effect (UME) Model\\
Regularized UME Model & 0.8729730 & Computed for `lambda` = 0.38745002746582\\
\addlinespace
UMGE Model & 0.8729730 & User+Movie+Genre Effect (UMGE) Model\\
Regularized UMGE Model & 0.8729730 & Computed for `lambda` = 0.0359375\\
UMGYE Model & 0.8723975 & User+Movie+Genre+Year Effect (UMGYE) Model\\
Regularized UMGYE Model & 0.8721860 & Computed for `lambda` = 233.784942626953\\
UMGYDE (Default) Model & 0.8722314 & User+Movie+Genre+Year+Day Effect (UMGYDE) Model 
computed using `stats::loess` function with `degree=1` \& `span=0.75` parameter values.\\
\addlinespace
Tuned UMGYDE.d0 Model & 0.8711135 & UMGYDE Model computed using function call: `loess(degree = 0, span = 0.00109375)`\\
Tuned UMGYDE.d1 Model & 0.8708300 & UMGYDE Model computed using function call: `loess(degree = 1, span = 0.00109375)`\\
Tuned UMGYDE.d2 Model & 0.8709891 & UMGYDE Model computed using function call: `loess(degree = 2, span = 0.00153125)`\\
Tuned UMGYDE Best Model & 0.8707855 & UMGYDE Model computed using `loess` function call with the best degree \& span values.\\
Regularized UMGYDE Model & 0.8707755 & The best tuned and regularized UMGYDE Model.\\
\bottomrule
\end{tabular}

\begin{noteblock}
In the code snippet above, we use the \hyperref[func.RMSEs.AddRow]{RMSEs.AddRow} and \hyperref[func.RMSE_kable]{RMSE\_kable} functions described in section \hyperref[result-rmses-tibble-functions]{Result RMSEs Tibble Functions} of \hyperref[appndx_a]{Appendix A}.

\end{noteblock}

\newpage

Now, we have a better result than the ones for the previous models, but still insufficient to meet the Project Objective. Let's see what else we can do to achieve our ultimate goal.

As explained in \href{https://rafalab.dfci.harvard.edu/dsbook/large-datasets.html\#matrix-factorization}{Section 33.11 \texttt{Matrix\ Factorization}} of the \emph{Course Textbook (First Edition)}, so far our models ignore \emph{``an important source of variation related to the fact that groups of movies have similar rating patterns and groups of users have similar rating patterns as well\ldots{}''} \autocite{IDS_E1_33-11}

The author shows that in this case, the \emph{Matrix Factorization} method can significantly improve our results. In the next section, we will apply this method to our latest model and see what results we achieve.

\newpage

\subsection{UMGYDE Model: Matrix Factorization (MF)}\label{umgyde-model-matrix-factorization-mf}

\hfill\break

\begin{noteblock}
The complete source code shown in this section is available in the \href{https://github.com/AzKurban-edX-DS/Capstone-MovieLens/blob/main/r/src/capstone-movielens.main.R\#L2230}{Matrix Factorization (MF)} section of the \href{https://github.com/AzKurban-edX-DS/Capstone-MovieLens/blob/main/r/src/capstone-movielens.main.R\#L2230}{capstone-movielens.main.R} script on \emph{GitHub}.

\end{noteblock}

For our next solution, we will use the \href{https://cran.r-project.org/web/packages/recosystem/index.html}{\texttt{recosystem} package} to perform the \emph{Parallel Matrix Factorization}.

\subsubsection{MF: Matematical Description}\label{mf-matematical-description}

\hfill\break

As outlined in \href{https://cran.r-project.org/web/packages/recosystem/vignettes/introduction.html}{this article}\autocite{yqiu_recosystem230505}, the idea of the \emph{Matrix Factorization} method is to approximate the whole rating matrix \(R_{m \times n}\) by the product of two matrices of lower dimensions, \(P_{n \times k}\) and \(Q_{n\times k}\), such that

\begin{equation}
\mathbb{R} \approx \mathbb{PQ^T}
\label{eq:R-PQ-MF}
\end{equation}

In relation to our model, the expression \eqref{eq:R-PQ-MF} will take the form:

\begin{equation}
\mathbf{R} \thicksim \mathbf{\hat{R}} + \mathbf{PQ^T} +  \varepsilon
\label{eq:UMGYDE-MF}
\end{equation}

where:

\begin{itemize}
\tightlist
\item
  \(\mathbf{R}\) is the \(U_\mathrm{m} \times M_n\) rating matrix with \(U_m\) users and \(M_n\) movies;
\item
  \(\mathbf{\hat{R}}\) represents the predictions from our best model: \emph{Regularized UMGYDE Model};
\item
  \(\mathbf{P}\) and \(\mathbf{Q}\) are \(P_{m \times k}\) and \(Q_{n \times k}\) matrices, respectively, where \(k\) is the number of \emph{latent features} to be found.
\end{itemize}

If we denote the \(u\)-th row of \(\mathbf{P}\) as \(\mathbf{p}_u\) and the \(v\)-th row of \(\mathbf{Q}\) as \(\mathbf{q}_v\), then the unknown rating \(\mathbf{r}_{u,v}\) given by user \(u\) on movie item \(v\) for our model can be estimated as \(\hat{\mathbf{r}}_{u,v} + \mathbf{p}_{u}\mathbf{q}^T_{v}\), where the \(\hat{\mathbf{r}}_{u,v}\) is the prediction given by our last \emph{Regularized UMGYDE Model}.

A typical solution for \(\mathbf{P}\) and \(\mathbf{Q}\) is given by the following optimization problem \autocite{chin_FPSGM_MF_SMS2015a,chin_LRS_SGM_MF2015b}:

\[
\begin{array}{c} { min} \\
{\scriptstyle P,Q} \end{array} \sum_{(u,v)\in R}[f(\mathbf{p}_u, \mathbf{q}_v; \mathbf{r}_{u,v}) + \mu_P\parallel\mathbf{p}_u\parallel_1 + \mu_Q\parallel\mathbf{q}_v\parallel_1 + \frac{\lambda_P}2\parallel\mathbf{p}_u\parallel^2_2 + \frac{\lambda_Q}2\parallel\mathbf{q}_v\parallel^2_2]
\]

where \((u, v)\) are locations of observed entries in \(\mathbf{R}, \mathbf{r}_{u,v}\) is the observed ratings, \(f\) is the loss function, and \(\mu_P\), \(\mu_Q\), \(\lambda_P\), \(\lambda_Q\) are penalty parameters to avoid overfitting.\autocite{yqiu_recosystem230505}

\begin{quote}
\emph{The process of solving the matrices \(P\) and \(Q\) is referred to as model training, and the selection of penalty parameters is called parameter tuning. In \texttt{recosystem}, we provide convenient functions for these two tasks, and additionally have functions for model exporting (outputting \(P\) and \(Q\) matrices) and prediction.}\autocite{yqiu_recosystem230505}
\end{quote}

\newpage

\subsubsection{MF: Model Building}\label{mf-model-building}

\hfill\break

\begin{noteblock}
The complete source code shown in this section is available in the \href{https://github.com/AzKurban-edX-DS/Capstone-MovieLens/blob/main/r/src/capstone-movielens.main.R\#L2253}{Perform the Matrix Factorization \& Final Test} section of the \href{https://github.com/AzKurban-edX-DS/Capstone-MovieLens/blob/main/r/src/capstone-movielens.main.R\#L2253}{capstone-movielens.main.R} script on \emph{GitHub}.

\end{noteblock}

We assume that the \emph{Matrix Factorization} method will be the last one to provide sufficient results, so we will use the entire \texttt{edx} dataset as the \emph{Training Set} and \texttt{final\_holdout\_test} as the \emph{Test Set} for our \emph{final model training and testing}, skipping the intermediate \emph{training/validation split operations} that applied for the previous models.

\paragraph{MF: Getting Residuals From the UMGYDE Model Prediction Values}\label{mf-getting-residuals-from-the-umgyde-model-prediction-values}

\hfill\break

First, we will obtain residuals from the \emph{UMGYDE Model} prediction values using the following \href{https://github.com/AzKurban-edX-DS/Capstone-MovieLens/blob/main/r/src/capstone-movielens.main.R\#L2268}{line of code}:

\begin{Shaded}
\begin{Highlighting}[]
\NormalTok{  mf.edx.residual }\OtherTok{\textless{}{-}} \FunctionTok{mf.residual.dataframe}\NormalTok{(edx)}
\end{Highlighting}
\end{Shaded}

\begin{noteblock}
In the code snippets above, we use the \hyperref[func.mf.residual.dataframe]{mf.residual.dataframe} user-defined helper function described in section \hyperref[umgyde-model-matrix-factorization]{UMGYDE Model: Matrix Factorization} of \hyperref[appndx_a]{Appendix A}.

\end{noteblock}

\begin{Shaded}
\begin{Highlighting}[]
  \FunctionTok{str}\NormalTok{(mf.edx.residual)}
\end{Highlighting}
\end{Shaded}

\begin{verbatim}
## 'data.frame':    9000055 obs. of  3 variables:
##  $ userId : int  1 1 1 1 1 1 1 1 1 1 ...
##  $ movieId: int  122 185 292 316 329 355 356 362 364 370 ...
##  $ rsdl   : num  0.4867 0.3223 0.0386 0.1137 0.1269 ...
\end{verbatim}

\paragraph{\texorpdfstring{MF: Transforming Input Data to Be Compatible With the \texttt{recosystem} Package}{MF: Transforming Input Data to Be Compatible With the recosystem Package}}\label{mf-transforming-input-data-to-be-compatible-with-the-recosystem-package}

\hfill\break

Next, we convert the residuals data and \texttt{final\_holdout\_test} dataset to the input data objects compatible with the \href{https://cran.r-project.org/web/packages/recosystem/index.html}{recosystem} package using the following \href{https://github.com/AzKurban-edX-DS/Capstone-MovieLens/blob/main/r/src/capstone-movielens.main.R\#L2273}{code snippet}:

\begin{Shaded}
\begin{Highlighting}[]
  \FunctionTok{set.seed}\NormalTok{(}\DecValTok{5430}\NormalTok{)}
\NormalTok{  mf.edx.residual.reco }\OtherTok{\textless{}{-}} \FunctionTok{with}\NormalTok{(mf.edx.residual, }
                               \FunctionTok{data\_memory}\NormalTok{(}\AttributeTok{user\_index =}\NormalTok{ userId, }
                                           \AttributeTok{item\_index =}\NormalTok{ movieId,}
                                           \AttributeTok{rating =}\NormalTok{ rsdl))}

\NormalTok{  final\_holdout\_test.reco }\OtherTok{\textless{}{-}} \FunctionTok{with}\NormalTok{(final\_holdout\_test, }
                          \FunctionTok{data\_memory}\NormalTok{(}\AttributeTok{user\_index =}\NormalTok{ userId, }
                                      \AttributeTok{item\_index =}\NormalTok{ movieId, }
                                      \AttributeTok{rating =}\NormalTok{ rating))}
\end{Highlighting}
\end{Shaded}

\newpage

\begin{verbatim}
## *** `Reco` input data objects structure ***
\end{verbatim}

\begin{Shaded}
\begin{Highlighting}[]
\FunctionTok{str}\NormalTok{(mf.edx.residual.reco)}
\end{Highlighting}
\end{Shaded}

\begin{verbatim}
## Formal class 'DataSource' [package "recosystem"] with 3 slots
##   ..@ source:List of 3
##   .. ..$ : int [1:9000055] 1 1 1 1 1 1 1 1 1 1 ...
##   .. ..$ : int [1:9000055] 122 185 292 316 329 355 356 362 364 370 ...
##   .. ..$ : num [1:9000055] 0.4867 0.3223 0.0386 0.1137 0.1269 ...
##   ..@ index1: logi FALSE
##   ..@ type  : chr "memory"
\end{verbatim}

\begin{Shaded}
\begin{Highlighting}[]
\FunctionTok{str}\NormalTok{(final\_holdout\_test.reco)}
\end{Highlighting}
\end{Shaded}

\begin{verbatim}
## Formal class 'DataSource' [package "recosystem"] with 3 slots
##   ..@ source:List of 3
##   .. ..$ : int [1:999999] 1 1 1 2 2 2 3 3 4 4 ...
##   .. ..$ : int [1:999999] 231 480 586 151 858 1544 590 4995 34 432 ...
##   .. ..$ : num [1:999999] 5 5 5 3 2 3 3.5 4.5 5 3 ...
##   ..@ index1: logi FALSE
##   ..@ type  : chr "memory"
\end{verbatim}

\paragraph{\texorpdfstring{MF: Creating and Tuning the \texttt{Reco} Object}{MF: Creating and Tuning the Reco Object}}\label{mf-creating-and-tuning-the-reco-object}

\hfill\break

Now we can create a \texttt{reco} object of the \texttt{Reco} class and tune it using the \texttt{reco\$tune} method, passing the residual data prepared in the previous step in the \texttt{train\_data} argument using the following \href{https://github.com/AzKurban-edX-DS/Capstone-MovieLens/blob/main/r/src/capstone-movielens.main.R\#L2284}{code snippet}:

\begin{Shaded}
\begin{Highlighting}[]
\NormalTok{  reco }\OtherTok{\textless{}{-}} \FunctionTok{Reco}\NormalTok{()}
  
\NormalTok{  reco.tuned }\OtherTok{\textless{}{-}}\NormalTok{ reco}\SpecialCharTok{$}\FunctionTok{tune}\NormalTok{(mf.edx.residual.reco, }\AttributeTok{opts =} \FunctionTok{list}\NormalTok{(}\AttributeTok{dim =} \FunctionTok{c}\NormalTok{(}\DecValTok{10}\NormalTok{, }\DecValTok{20}\NormalTok{, }\DecValTok{30}\NormalTok{),}
                                                \AttributeTok{lrate    =} \FunctionTok{c}\NormalTok{(}\FloatTok{0.1}\NormalTok{, }\FloatTok{0.2}\NormalTok{),}
                                                \AttributeTok{nthread  =} \DecValTok{4}\NormalTok{,}
                                                \AttributeTok{niter    =} \DecValTok{10}\NormalTok{,}
                                                \AttributeTok{verbose  =} \ConstantTok{TRUE}\NormalTok{))}
\end{Highlighting}
\end{Shaded}

\newpage

\begin{verbatim}
## *** `Reco` object structure ***
\end{verbatim}

\begin{Shaded}
\begin{Highlighting}[]
\FunctionTok{str}\NormalTok{(reco)}
\end{Highlighting}
\end{Shaded}

\begin{verbatim}
## Reference class 'RecoSys' [package "recosystem"] with 2 fields
##  $ model     :Reference class 'RecoModel' [package "recosystem"] with 5 fields
##   ..$ path    : chr ""
##   ..$ nuser   : int 71568
##   ..$ nitem   : int 65134
##   ..$ nfac    : int 30
##   ..$ matrices:List of 3
##   .. ..$ P:Formal class 'float32' [package "float"] with 1 slot
## ..@ Data: int [1:30, 1:71568] NaN NaN NaN NaN NaN NaN NaN NaN NaN NaN ...
##   .. ..$ Q:Formal class 'float32' [package "float"] with 1 slot
## ..@ Data: int [1:30, 1:65134] NaN NaN NaN NaN NaN NaN NaN NaN NaN NaN ...
##   .. ..$ b:Formal class 'float32' [package "float"] with 1 slot
## ..@ Data: int [1:1] 0.00114
##   ..and 16 methods, of which 2 are  possibly relevant:
##   ..  initialize, show#envRefClass
##  $ train_pars:List of 12
##   ..$ loss    : int 0
##   ..$ dim     : int 30
##   ..$ costp_l1: num 0
##   ..$ costp_l2: num 0.01
##   ..$ costq_l1: num 0
##   ..$ costq_l2: num 0.1
##   ..$ lrate   : num 0.1
##   ..$ niter   : num 20
##   ..$ nthread : num 4
##   ..$ nbin    : int 20
##   ..$ nmf     : logi FALSE
##   ..$ verbose : logi TRUE
##  and 19 methods, of which 5 are  possibly relevant:
##    output, predict, show#envRefClass, train, tune
\end{verbatim}

\newpage

\begin{verbatim}
## *** Tuned `Reco` object structure ***
\end{verbatim}

\begin{Shaded}
\begin{Highlighting}[]
\FunctionTok{str}\NormalTok{(reco.tuned)}
\end{Highlighting}
\end{Shaded}

\begin{verbatim}
## List of 2
##  $ min:List of 7
##   ..$ dim     : int 30
##   ..$ costp_l1: num 0
##   ..$ costp_l2: num 0.01
##   ..$ costq_l1: num 0
##   ..$ costq_l2: num 0.1
##   ..$ lrate   : num 0.1
##   ..$ loss_fun: num 0.795
##  $ res:'data.frame': 96 obs. of  7 variables:
##   ..$ dim     : int [1:96] 10 20 30 10 20 30 10 20 30 10 ...
##   ..$ costp_l1: num [1:96] 0 0 0 0.1 0.1 0.1 0 0 0 0.1 ...
##   ..$ costp_l2: num [1:96] 0.01 0.01 0.01 0.01 0.01 0.01 0.1 0.1 0.1 0.1 ...
##   ..$ costq_l1: num [1:96] 0 0 0 0 0 0 0 0 0 0 ...
##   ..$ costq_l2: num [1:96] 0.01 0.01 0.01 0.01 0.01 0.01 0.01 0.01 0.01 0.01 ...
##   ..$ lrate   : num [1:96] 0.1 0.1 0.1 0.1 0.1 0.1 0.1 0.1 0.1 0.1 ...
##   ..$ loss_fun: num [1:96] 0.807 0.813 0.822 0.854 0.87 ...
##   ..- attr(*, "out.attrs")=List of 2
##   .. ..$ dim     : Named int [1:6] 3 2 2 2 2 2
##   .. .. ..- attr(*, "names")= chr [1:6] "dim" "costp_l1" "costp_l2" "costq_l1" ...
##   .. ..$ dimnames:List of 6
##   .. .. ..$ dim     : chr [1:3] "dim=10" "dim=20" "dim=30"
##   .. .. ..$ costp_l1: chr [1:2] "costp_l1=0.0" "costp_l1=0.1"
##   .. .. ..$ costp_l2: chr [1:2] "costp_l2=0.01" "costp_l2=0.10"
##   .. .. ..$ costq_l1: chr [1:2] "costq_l1=0.0" "costq_l1=0.1"
##   .. .. ..$ costq_l2: chr [1:2] "costq_l2=0.01" "costq_l2=0.10"
##   .. .. ..$ lrate   : chr [1:2] "lrate=0.1" "lrate=0.2"
\end{verbatim}

\newpage

\paragraph{MF: Final Training}\label{mf-final-training}

\hfill\break

Finally, we train the model for the last time using the \texttt{reco\$train} method, passing the same residual data we used for the tuning, in the \texttt{train\_data} argument, using the following \href{https://github.com/AzKurban-edX-DS/Capstone-MovieLens/blob/main/r/src/capstone-movielens.main.R\#L2296}{code snippet}:

\begin{Shaded}
\begin{Highlighting}[]
\NormalTok{  reco}\SpecialCharTok{$}\FunctionTok{train}\NormalTok{(mf.edx.residual.reco, }\AttributeTok{opts =} \FunctionTok{c}\NormalTok{(reco.tuned}\SpecialCharTok{$}\NormalTok{min,}
                                       \AttributeTok{niter =} \DecValTok{20}\NormalTok{, }
                                       \AttributeTok{nthread =} \DecValTok{4}\NormalTok{)) }
\end{Highlighting}
\end{Shaded}

\subsubsection{MF: Final Holdout Test}\label{mf-final-holdout-test}

\hfill\break

From now on, we will work with the \texttt{final\_holdout\_test} dataset for the final testing of our model, including obtaining final predictions and computing the final \emph{RMSE} score.

First, let's compute predictions for the best model we built above, \emph{Regularized UMGYDE Model} (as described in section \hyperref[umgyde-model-regularization-retraining-on-the-edx-with-the-best-lambda]{UMGYDE Model Regularization: \emph{Retraining on the \texttt{edx} with the best} \(\lambda\)}), using the following \href{https://github.com/AzKurban-edX-DS/Capstone-MovieLens/blob/main/r/src/capstone-movielens.main.R\#L2300}{code snippet}:

\begin{Shaded}
\begin{Highlighting}[]
\CommentTok{\# Compute predictions from Regularized UMGYDE Model}
\NormalTok{final.UMGYDE.predicted }\OtherTok{\textless{}{-}}\NormalTok{ final\_holdout\_test }\SpecialCharTok{|\textgreater{}}
  \FunctionTok{UMGY\_SmoothedDay\_effect.predict}\NormalTok{(rglr.UMGYD\_effect)}
\end{Highlighting}
\end{Shaded}

\begin{Shaded}
\begin{Highlighting}[]
\FunctionTok{str}\NormalTok{(final.UMGYDE.predicted)}
\end{Highlighting}
\end{Shaded}

\begin{verbatim}
## 'data.frame':    999999 obs. of  5 variables:
##  $ userId   : int  1 1 1 2 2 2 3 3 4 4 ...
##  $ movieId  : int  231 480 586 151 858 1544 590 4995 34 432 ...
##  $ timestamp: int  838983392 838983653 838984068 868246450 868245645 868245920 1136075494 1133571200 844416936 844417070 ...
##  $ rating   : num  5 5 5 3 2 3 3.5 4.5 5 3 ...
##  $ predicted: num  4.49 5 4.63 3.3 4.07 ...
\end{verbatim}

\begin{noteblock}
In the code snippet above we use the \hyperref[func.UMGY_SmoothedDay_effect.predict]{UMGY\_SmoothedDay\_effect.predict} function described in section \hyperref[umgyde-model-utility-functions]{UMGYDE Model: Utility Functions} of \hyperref[appndx_a]{Appendix A}.

\end{noteblock}

Next, we get predicted residuals from the \texttt{reco} object, using the following \href{https://github.com/AzKurban-edX-DS/Capstone-MovieLens/blob/main/r/src/capstone-movielens.main.R\#L2323}{code snippet}:

\begin{Shaded}
\begin{Highlighting}[]
\CommentTok{\# Compute residuals from trained Reco object}
\NormalTok{mf.reco.residual }\OtherTok{\textless{}{-}}\NormalTok{ reco}\SpecialCharTok{$}\FunctionTok{predict}\NormalTok{(final\_holdout\_test.reco, }\FunctionTok{out\_memory}\NormalTok{())}
\end{Highlighting}
\end{Shaded}

\begin{Shaded}
\begin{Highlighting}[]
\FunctionTok{str}\NormalTok{(mf.reco.residual)}
\end{Highlighting}
\end{Shaded}

\begin{verbatim}
##  num [1:999999] 0.198 -0.265 0.106 0.377 0.566 ...
\end{verbatim}

\newpage

Finally, we compute the ultimate predictions combining the predicted rating values we obtained above from the \emph{Regularized UMGYDE Model} and the residuals obtained from the \texttt{reco} object, using the following \href{https://github.com/AzKurban-edX-DS/Capstone-MovieLens/blob/main/r/src/capstone-movielens.main.R\#L2328}{code snippet}:

\begin{Shaded}
\begin{Highlighting}[]
\CommentTok{\# Compute Final Holdout Predictions}
\NormalTok{mf.predicted\_ratings }\OtherTok{\textless{}{-}} 
  \FunctionTok{clamp}\NormalTok{(final.UMGYDE.predicted}\SpecialCharTok{$}\NormalTok{predicted }\SpecialCharTok{+}\NormalTok{ mf.reco.residual)}
\end{Highlighting}
\end{Shaded}

\begin{Shaded}
\begin{Highlighting}[]
\FunctionTok{str}\NormalTok{(mf.predicted\_ratings)}
\end{Highlighting}
\end{Shaded}

\begin{verbatim}
##  num [1:999999] 4.68 4.74 4.74 3.68 4.64 ...
\end{verbatim}

\begin{noteblock}
In the code snippet above, we use the \hyperref[func.clamp]{clamp} user-defined helper function described in section \hyperref[appndx_a.CHF.utility_funcs]{Utility Functions} of \hyperref[appndx_a]{Appendix A}.

\end{noteblock}

And ultimately, we are ready to compute the final \emph{RMSE} score using the following \href{https://github.com/AzKurban-edX-DS/Capstone-MovieLens/blob/main/r/src/capstone-movielens.main.R\#L2353}{code snippet}:

\begin{Shaded}
\begin{Highlighting}[]
\NormalTok{final\_holdout\_test.RMSE }\OtherTok{\textless{}{-}} \FunctionTok{rmse2}\NormalTok{(final\_holdout\_test}\SpecialCharTok{$}\NormalTok{rating,}
\NormalTok{                                     mf.predicted\_ratings)}
\end{Highlighting}
\end{Shaded}

\begin{Shaded}
\begin{Highlighting}[]
\NormalTok{final\_holdout\_test.RMSE}
\end{Highlighting}
\end{Shaded}

\begin{verbatim}
## [1] 0.7876658
\end{verbatim}

\begin{noteblock}
In the code snippet above, we use the \hyperref[func.rmse2]{rmse2} user-defined helper function described in section \hyperref[root-mean-squared-error-calculation]{(Root) Mean Squared Error Calculation} of \hyperref[appndx_a]{Appendix A}.

\end{noteblock}

\newpage

Finally, we add the \emph{RMSE} value obtained above to our \emph{Result Table} and print the table using the following \href{https://github.com/AzKurban-edX-DS/Capstone-MovieLens/blob/main/r/src/capstone-movielens.main.R\#L2359}{code snippet}:

\begin{Shaded}
\begin{Highlighting}[]
\NormalTok{final.MF.RMSEs.ResultTibble }\OtherTok{\textless{}{-}}\NormalTok{ RMSEs.ResultTibble.UMGYDE.rglr.tuned }\SpecialCharTok{|\textgreater{}} 
  \FunctionTok{RMSEs.AddRow}\NormalTok{(}\StringTok{"MF (Final Test)"}\NormalTok{, }
\NormalTok{               final\_holdout\_test.RMSE,}
               \AttributeTok{comment =} \StringTok{"Matrix Factorization of the Best Model Residuals, Final Holdout Test"}\NormalTok{)}
\end{Highlighting}
\end{Shaded}

\begin{Shaded}
\begin{Highlighting}[]
\FunctionTok{RMSE\_kable}\NormalTok{(final.MF.RMSEs.ResultTibble)}
\end{Highlighting}
\end{Shaded}

\begin{tabular}{>{\raggedright\arraybackslash}p{15em}>{\centering\arraybackslash}p{5em}>{\raggedright\arraybackslash}p{30em}}
\toprule
\textbf{Method} & \textbf{RMSE} & \textbf{Comment}\\
\midrule
Project Objective & 0.8649000 & \\
OMR Model & 1.0603462 & Overall Mean Rating (OMR) Model\\
UE Model & 0.9697962 & User Effect (UE) Model\\
UME Model & 0.8732081 & User+Movie Effect (UME) Model\\
Regularized UME Model & 0.8729730 & Computed for `lambda` = 0.38745002746582\\
\addlinespace
UMGE Model & 0.8729730 & User+Movie+Genre Effect (UMGE) Model\\
Regularized UMGE Model & 0.8729730 & Computed for `lambda` = 0.0359375\\
UMGYE Model & 0.8723975 & User+Movie+Genre+Year Effect (UMGYE) Model\\
Regularized UMGYE Model & 0.8721860 & Computed for `lambda` = 233.784942626953\\
UMGYDE (Default) Model & 0.8722314 & User+Movie+Genre+Year+Day Effect (UMGYDE) Model 
computed using `stats::loess` function with `degree=1` \& `span=0.75` parameter values.\\
\addlinespace
Tuned UMGYDE.d0 Model & 0.8711135 & UMGYDE Model computed using function call: `loess(degree = 0, span = 0.00109375)`\\
Tuned UMGYDE.d1 Model & 0.8708300 & UMGYDE Model computed using function call: `loess(degree = 1, span = 0.00109375)`\\
Tuned UMGYDE.d2 Model & 0.8709891 & UMGYDE Model computed using function call: `loess(degree = 2, span = 0.00153125)`\\
Tuned UMGYDE Best Model & 0.8707855 & UMGYDE Model computed using `loess` function call with the best degree \& span values.\\
Regularized UMGYDE Model & 0.8707755 & The best tuned and regularized UMGYDE Model.\\
\addlinespace
MF (Final Test) & 0.7876658 & Matrix Factorization of the Best Model Residuals, Final Holdout Test\\
\bottomrule
\end{tabular}

\begin{noteblock}
In the code snippet above, we use the \hyperref[func.RMSEs.AddRow]{RMSEs.AddRow} and \hyperref[func.RMSE_kable]{RMSE\_kable} functions described in section \hyperref[result-rmses-tibble-functions]{Result RMSEs Tibble Functions} of \hyperref[appndx_a]{Appendix A}.

\end{noteblock}

Eventually, we have reached our ultimate goal, as the \emph{RMSE} we obtained has achieved the required Project Objective.

\newpage

\section{Results}\label{results}

\hfill\break

The \emph{Result Summary Table} below, constructed using the following \href{https://github.com/AzKurban-edX-DS/Capstone-MovieLens/blob/main/r/src/capstone-movielens.main.R\#L2368}{code snippet}, shows in detail the progress of our \emph{Recommender System} improvement:

\begin{Shaded}
\begin{Highlighting}[]
\NormalTok{total.RMSEs.ResultTibble }\OtherTok{\textless{}{-}} \FunctionTok{RMSEs.AddDiffColumn}\NormalTok{(final.MF.RMSEs.ResultTibble)}
\end{Highlighting}
\end{Shaded}

\begin{Shaded}
\begin{Highlighting}[]
\FunctionTok{RMSE.Total\_kable}\NormalTok{(total.RMSEs.ResultTibble)}
\end{Highlighting}
\end{Shaded}

\begin{tabular}{>{\raggedright\arraybackslash}p{15em}>{\centering\arraybackslash}p{7em}>{\raggedright\arraybackslash}p{5em}>{\raggedright\arraybackslash}p{25em}}
\toprule
\textbf{Method} & \textbf{RMSE} & \textbf{Diff} & \textbf{Comment}\\
\midrule
Project Objective & 0.8649000 & 0.0000000 & \\
OMR Model & 1.0603462 & -0.1954462 & Overall Mean Rating (OMR) Model\\
UE Model & 0.9697962 & 0.0905500 & User Effect (UE) Model\\
UME Model & 0.8732081 & 0.0965881 & User+Movie Effect (UME) Model\\
Regularized UME Model & 0.8729730 & 0.0002351 & Computed for `lambda` = 0.38745002746582\\
\addlinespace
UMGE Model & 0.8729730 & 0.0000000 & User+Movie+Genre Effect (UMGE) Model\\
Regularized UMGE Model & 0.8729730 & 0.0000000 & Computed for `lambda` = 0.0359375\\
UMGYE Model & 0.8723975 & 0.0005755 & User+Movie+Genre+Year Effect (UMGYE) Model\\
Regularized UMGYE Model & 0.8721860 & 0.0002116 & Computed for `lambda` = 233.784942626953\\
UMGYDE (Default) Model & 0.8722314 & -0.0000454 & User+Movie+Genre+Year+Day Effect (UMGYDE) Model 
computed using `stats::loess` function with `degree=1` \& `span=0.75` parameter values.\\
\addlinespace
Tuned UMGYDE.d0 Model & 0.8711135 & 0.0011179 & UMGYDE Model computed using function call: `loess(degree = 0, span = 0.00109375)`\\
Tuned UMGYDE.d1 Model & 0.8708300 & 0.0002835 & UMGYDE Model computed using function call: `loess(degree = 1, span = 0.00109375)`\\
Tuned UMGYDE.d2 Model & 0.8709891 & -0.0001591 & UMGYDE Model computed using function call: `loess(degree = 2, span = 0.00153125)`\\
Tuned UMGYDE Best Model & 0.8707855 & 0.0002036 & UMGYDE Model computed using `loess` function call with the best degree \& span values.\\
Regularized UMGYDE Model & 0.8707755 & 0.0000100 & The best tuned and regularized UMGYDE Model.\\
\addlinespace
MF (Final Test) & 0.7876658 & 0.0831097 & Matrix Factorization of the Best Model Residuals, Final Holdout Test\\
\bottomrule
\end{tabular}

\begin{noteblock}
In the code snippet above, we use the \hyperref[func.RMSEs.AddDiffColumn]{RMSEs.AddDiffColumn} and \hyperref[func.RMSE.Total_kable]{RMSE.Total\_kable} functions described in section \hyperref[result-rmses-tibble-functions]{Result RMSEs Tibble Functions} of \hyperref[appndx_a]{Appendix A}.

\end{noteblock}

\newpage

\section{Conclusion}\label{conclusion}

\hfill\break

In this project, a \emph{Machine Learning Algorithm} was successfully developed for a \emph{Recommender System} (hereafter referred to as \emph{RS MLA}) to predict movie ratings using the \href{http://grouplens.org/datasets/movielens/10m/}{10M version of the MovieLens} dataset.

Since it was needed to convert the original dataset into a \emph{Data Source}, suitable for processing by the \emph{RS MLA}, a package \href{https://github.com/AzKurban-edX-DS/edx.capstone.movielens.data/blob/main/README.md}{edx.capstone.movielens.data}, specifically designed for this purpose, was developed and used, providing the \emph{RS MLA} with the necessary datasets.

Following the Netflix Grand Prize Contest requirements, the \emph{Root Mean Squared Error (RMSE)} metric was used to evaluate the accuracy of our models.

By incorporating essential factors such as overall average rating, movie, user, and some other observable effects, as well as efficient techniques such as \emph{Regularization} and \emph{Matrix Factorization} methods, the accuracy was significantly improved compared to the baseline model.

The baseline model was the \emph{Overall Mean Rating} model described in section \hyperref[overall-mean-rating-omr-model]{Overall Mean Rating (OMR) Model}.

The following effects were taken into account:

\begin{itemize}
\tightlist
\item
  \textbf{\emph{User:}} based on which the \emph{User Effect (UE) Model} described in section \hyperref[user-effect-ue-model]{User Effect (UE) Model} was built (significantly improved the \emph{RMSE} compared to the baseline model);
\item
  \textbf{\emph{Movie:}} based on which (and the \emph{User Effect}) the \emph{Regularized User+Movie Effect (UME) Model} described in section \hyperref[usermovie-effect-ume-model]{User+Movie Effect (UME) Model} was eventually built (even better improved the \emph{RMSE} compared to the previous (\emph{UE}) model);
\item
  \textbf{\emph{Genre:}} based on which (and the previous effects) the \emph{Regularized User+Movie+Genre Effect (UMGE) Model} described in section \hyperref[usermoviegenre-effect-umge-model]{User+Movie+Genre Effect (UMGE) Model} was built (even after \emph{regularization} it showed very little improvement in \emph{RMSE} compared to the previous (\emph{UME}) model; before regularization, for some reason, there was no improvement at all);
\item
  \textbf{\emph{Year:}} based on which (and the previous effects) the \emph{Regularized User+Movie+Genre+Year Effect (UMGYE) Model} described in section \hyperref[usermoviegenreyear-effect-umgye-model]{User+Movie+Genre+Year Effect (UMGYE) Model} was built (provided a much better improvement in \emph{RMSE} compared to the previous model but still two orders of magnitude less than the \emph{UE} and \emph{UME} models showed);
\item
  \textbf{\emph{Smoothed Day}} based on which (and the previous effects) the \emph{Regularized (and Tuned) User+Movie+Genre+Year+SmoothedDay Effect (UMGYDE) Model} described in section \hyperref[usermoviegenreyearsmoothedday-effect-umgyde-model]{User+Movie+Genre+Year+SmoothedDay Effect (UMGYDE) Model} was built (showed almost twice better improvement than the previous (\emph{UMGYE}) model but still not good enough to meet the \emph{Project Objective});.
\item
  \textbf{\emph{Smoothed Day (Matrix Factorization):}} (described in section \hyperref[umgyde-model-matrix-factorization-mf]{UMGYDE Model: Matrix Factorization (MF)}) \emph{Matrix Factorization} method performed on the \emph{Regularized UMGYDE Model} provided substantial improvement comparable to that shown by the \emph{UE} and \emph{UME} models, and ultimately allowed us to reach the \emph{Project Objective}.
\end{itemize}

The \emph{Result Summary Table} provided above in section \hyperref[results]{Results} shows in detail the progress of our \emph{Recommender System} improvement.

Although the \emph{Project Objective} was achieved, further research could explore the use of other methods, such as advanced algorithms described in \emph{Part V Advanced Algorithms} of the \emph{Recommender Systems Handbook} (written by \emph{Francesco Ricci, Lior Rokach, Bracha Shapira, and Paul B. Kantor})\autocite{RRSK_RS_HB}, or \emph{Deep Learning Models} as described in article \href{https://www.sciencedirect.com/science/article/pii/S2772485925000134\#:~:text=In\%20this\%20work\%2C\%20the\%20authors,capturing\%20user\%20preferences\%20over\%20time.}{Advanced Deep Learning Models for Improving Movie Rating Predictions: A Benchmarking Study}\autocite{MVRM_DL}.

\newpage

\section{Appendix A: Support Functions}\label{appndx_a}

\subsection{Common Helper Functions}\label{common-helper-functions}

\subsubsection{Logging Functions}\label{appndx_a.CHF.logging-funcs}

\begin{noteblock}
The full source code of the functions described in this section is available in the \href{https://github.com/AzKurban-edX-DS/Capstone-MovieLens/blob/main/r/src/support-functions/logging-functions.R}{logging-functions.R} script on \emph{GitHub}.

\end{noteblock}

\paragraph{\texorpdfstring{\href{https://github.com/AzKurban-edX-DS/Capstone-MovieLens/blob/main/r/src/support-functions/logging-functions.R\#L3}{open\_logfile} Function}{open\_logfile Function}}\label{func.open_logfile}

\hfill\break

A function to initialize the log file.

\phantomsection\label{func.open_logfile.usage}
\subparagraph{\texorpdfstring{\href{https://github.com/AzKurban-edX-DS/Capstone-MovieLens/blob/main/r/src/capstone-movielens.main.R\#L5}{Usage}}{Usage}}\label{usage}

\hfill\break

\begin{Shaded}
\begin{Highlighting}[]
\FunctionTok{open\_logfile}\NormalTok{(}\StringTok{".data{-}analysis"}\NormalTok{)}
\end{Highlighting}
\end{Shaded}

\phantomsection\label{func.open_logfile.args}
\subparagraph{Arguments}\label{arguments}

\hfill\break

\phantomsection\label{func.open_logfile.args.file_name}
\begin{itemize}
\tightlist
\item
  \textbf{file\_name:} A name used as a base to generate a unique log file name;
\end{itemize}

\phantomsection\label{func.open_logfile.details}
\subparagraph{Details}\label{details}

\hfill\break

This function is a wrapper for the \href{https://www.rdocumentation.org/packages/logr/versions/1.3.9/topics/log_open}{logr::log\_open} function. It also calls internally the following functions to generate a unique file name based on the \hyperref[func.open_logfile.args.file_name]{file\_name} argument and \emph{system time}:

\begin{itemize}
\item
  \href{https://www.rdocumentation.org/packages/base/versions/3.6.2/topics/Sys.time}{Sys.time};
\item
  \href{https://www.rdocumentation.org/packages/stringr/versions/1.5.0/topics/str_replace}{stringr::str\_replace};
\item
  \href{https://www.rdocumentation.org/packages/stringr/versions/1.5.0/topics/str_replace}{stringr::str\_replace\_all}
\end{itemize}

For more information on the \emph{inner functions} listed above, see the \href{https://www.rdocumentation.org/}{R Documentation}.

\phantomsection\label{func.open_logfile.value}
\subparagraph{Value}\label{value}

\hfill\break

The path of the log.

\newpage

\phantomsection\label{func.open_logfile.code}
\subparagraph{Source Code}\label{source-code}

\hfill\break

The source code of the \href{https://github.com/AzKurban-edX-DS/Capstone-MovieLens/blob/main/r/src/support-functions/logging-functions.R\#L3}{open\_logfile} function is shown below:

\begin{Shaded}
\begin{Highlighting}[]
\FunctionTok{library}\NormalTok{(logr)}
\FunctionTok{library}\NormalTok{(stringr)}

\NormalTok{open\_logfile }\OtherTok{\textless{}{-}} \ControlFlowTok{function}\NormalTok{(file\_name)\{}
\NormalTok{  log\_file\_name }\OtherTok{\textless{}{-}} \FunctionTok{as.character}\NormalTok{(}\FunctionTok{Sys.time}\NormalTok{()) }\SpecialCharTok{|\textgreater{}} 
    \FunctionTok{str\_replace\_all}\NormalTok{(}\StringTok{\textquotesingle{}:\textquotesingle{}}\NormalTok{, }\StringTok{\textquotesingle{}\_\textquotesingle{}}\NormalTok{) }\SpecialCharTok{|\textgreater{}} 
    \FunctionTok{str\_replace}\NormalTok{(}\StringTok{\textquotesingle{} \textquotesingle{}}\NormalTok{, }\StringTok{\textquotesingle{}T\textquotesingle{}}\NormalTok{) }\SpecialCharTok{|\textgreater{}}
    \FunctionTok{str\_c}\NormalTok{(file\_name)}
  
  \FunctionTok{log\_open}\NormalTok{(}\AttributeTok{file\_name =}\NormalTok{ log\_file\_name)}
\NormalTok{\}}
\end{Highlighting}
\end{Shaded}

\begin{noteblock}
The source code of the \href{https://github.com/AzKurban-edX-DS/Capstone-MovieLens/blob/main/r/src/support-functions/logging-functions.R\#L3}{open\_logfile} function is defined in the \href{https://github.com/AzKurban-edX-DS/Capstone-MovieLens/blob/main/r/src/support-functions/logging-functions.R\#L3}{logging-functions.R} script on \emph{GitHub}.

\end{noteblock}

\paragraph{\texorpdfstring{\href{https://github.com/AzKurban-edX-DS/Capstone-MovieLens/blob/main/r/src/support-functions/logging-functions.R\#L11}{print\_start\_date} Function}{print\_start\_date Function}}\label{func.print_start_date}

\hfill\break

Prints the system date and time at the time the function is called and returns the system's idea of the current date and time.

\phantomsection\label{func.print_start_date.usage}
\subparagraph{Usage}\label{usage-1}

\hfill\break

\begin{Shaded}
\begin{Highlighting}[]
\NormalTok{start\_date }\OtherTok{\textless{}{-}} \FunctionTok{print\_start\_date}\NormalTok{()}
\end{Highlighting}
\end{Shaded}

\begin{verbatim}
## [1] "Thu Feb 26 08:59:18 2026"
\end{verbatim}

\begin{Shaded}
\begin{Highlighting}[]
\NormalTok{start\_date}
\end{Highlighting}
\end{Shaded}

\begin{verbatim}
## [1] "2026-02-26 08:59:18 MSK"
\end{verbatim}

\phantomsection\label{func.print_start_date.details}
\subparagraph{Details}\label{details-1}

\hfill\break

This function internally calls \href{https://www.rdocumentation.org/packages/base/versions/3.6.2/topics/print}{base::print}, \href{https://www.rdocumentation.org/packages/base/versions/3.6.2/topics/date}{base::date}, and \href{https://www.rdocumentation.org/link/Sys.time?package=base&version=3.6.2}{base::Sys.time} \texttt{R} functions.

It first prints the value returned by the \texttt{date} function, then returns the object returned by the \texttt{Sys.time} function.

\phantomsection\label{func.print_start_date.value}
\subparagraph{Value}\label{value-1}

\hfill\break

An object of class \texttt{POSIXct} (see \href{https://www.rdocumentation.org/link/DateTimeClasses?package=base&version=3.6.2}{DateTimeClasses}).

\phantomsection\label{func.print_start_date.code}
\subparagraph{Source Code}\label{source-code-1}

\hfill\break

The source code of the \href{https://github.com/AzKurban-edX-DS/Capstone-MovieLens/blob/main/r/src/support-functions/logging-functions.R\#L11}{print\_start\_date} function is shown below:

\begin{Shaded}
\begin{Highlighting}[]
\NormalTok{print\_start\_date }\OtherTok{\textless{}{-}} \ControlFlowTok{function}\NormalTok{()\{}
  \FunctionTok{print}\NormalTok{(}\FunctionTok{date}\NormalTok{())}
  \FunctionTok{Sys.time}\NormalTok{()}
\NormalTok{\}}
\end{Highlighting}
\end{Shaded}

\begin{noteblock}
The source code of the \href{https://github.com/AzKurban-edX-DS/Capstone-MovieLens/blob/main/r/src/support-functions/logging-functions.R\#L11}{print\_start\_date} function is defined in the \href{https://github.com/AzKurban-edX-DS/Capstone-MovieLens/blob/main/r/src/support-functions/logging-functions.R\#L11}{logging-functions.R} script on \emph{GitHub}.

\end{noteblock}

\paragraph{\texorpdfstring{\href{https://github.com/AzKurban-edX-DS/Capstone-MovieLens/blob/main/r/src/support-functions/logging-functions.R\#L15}{put\_start\_date} Function}{put\_start\_date Function}}\label{func.put_start_date}

\hfill\break

Outputs the system date and time to the currently opened log (using the \hyperref[func.open_logfile]{open\_logfile} function described above) at the time the function is called and returns the system's idea of the current date and time.

\phantomsection\label{func.put_start_date.usage}
\subparagraph{\texorpdfstring{\href{https://github.com/AzKurban-edX-DS/Capstone-MovieLens/blob/main/r/src/capstone-movielens.main.R\#L99}{Usage}}{Usage}}\label{usage-2}

\hfill\break

\begin{Shaded}
\begin{Highlighting}[]
\NormalTok{  start }\OtherTok{\textless{}{-}} \FunctionTok{put\_start\_date}\NormalTok{()}
\end{Highlighting}
\end{Shaded}

\phantomsection\label{func.put_start_date.details}
\subparagraph{Details}\label{details-2}

\hfill\break

This function internally calls \href{https://www.rdocumentation.org/packages/logr/versions/1.3.4/topics/log_print}{logr::put}, \href{https://www.rdocumentation.org/packages/base/versions/3.6.2/topics/date}{base::date}, and \href{https://www.rdocumentation.org/link/Sys.time?package=base&version=3.6.2}{base::Sys.time} \texttt{R} functions.

It first outputs the value returned by the \texttt{date} function to the currently opened log, then returns the object returned by the \texttt{Sys.time} function.

\phantomsection\label{func.put_start_date.value}
\subparagraph{Value}\label{value-2}

\hfill\break

An object of class \texttt{POSIXct} (see \href{https://www.rdocumentation.org/link/DateTimeClasses?package=base&version=3.6.2}{DateTimeClasses}).

\phantomsection\label{func.put_start_date.code}
\subparagraph{Source Code}\label{source-code-2}

\hfill\break

The source code of the \href{https://github.com/AzKurban-edX-DS/Capstone-MovieLens/blob/main/r/src/support-functions/logging-functions.R\#L15}{put\_start\_date} function is shown below:

\begin{Shaded}
\begin{Highlighting}[]
\NormalTok{put\_start\_date }\OtherTok{\textless{}{-}} \ControlFlowTok{function}\NormalTok{()\{}
  \FunctionTok{put}\NormalTok{(}\FunctionTok{date}\NormalTok{())}
  \FunctionTok{Sys.time}\NormalTok{()}
\NormalTok{\}}
\end{Highlighting}
\end{Shaded}

\begin{noteblock}
The source code of the \href{https://github.com/AzKurban-edX-DS/Capstone-MovieLens/blob/main/r/src/support-functions/logging-functions.R\#L15}{put\_start\_date} function is defined in the \href{https://github.com/AzKurban-edX-DS/Capstone-MovieLens/blob/main/r/src/support-functions/logging-functions.R\#L15}{logging-functions.R} script on \emph{GitHub}.

\end{noteblock}

\paragraph{\texorpdfstring{\href{https://github.com/AzKurban-edX-DS/Capstone-MovieLens/blob/main/r/src/support-functions/logging-functions.R\#L19}{print\_end\_date} Function}{print\_end\_date Function}}\label{func.print_end_date}

\hfill\break

Prints the system date and time at the time the function is called, and then the time elapsed since the date and time specified by the \hyperref[func.print_end_date.args.start]{start} argument.

\phantomsection\label{func.print_end_date.usage}
\subparagraph{Usage}\label{usage-3}

\hfill\break

\begin{Shaded}
\begin{Highlighting}[]
\NormalTok{start\_date }\OtherTok{\textless{}{-}} \FunctionTok{print\_start\_date}\NormalTok{()}
\end{Highlighting}
\end{Shaded}

\begin{verbatim}
## [1] "Thu Feb 26 08:59:18 2026"
\end{verbatim}

\begin{Shaded}
\begin{Highlighting}[]
\FunctionTok{Sys.sleep}\NormalTok{(}\DecValTok{3}\NormalTok{)}
\FunctionTok{print\_end\_date}\NormalTok{(start\_date)}
\end{Highlighting}
\end{Shaded}

\begin{verbatim}
## [1] "Thu Feb 26 08:59:21 2026"
## Time difference of 3.038683 secs
\end{verbatim}

\phantomsection\label{func.print_end_date.args}
\subparagraph{Arguments}\label{arguments-1}

\hfill\break

\phantomsection\label{func.print_end_date.args.start}
\begin{itemize}
\tightlist
\item
  \textbf{start:} An object of class \texttt{POSIXct} representing the value of the moment from which time is measured.
\end{itemize}

\phantomsection\label{func.print_end_date.details}
\subparagraph{Details}\label{details-3}

\hfill\break

This function internally calls \href{https://www.rdocumentation.org/packages/base/versions/3.6.2/topics/print}{base::print}, \href{https://www.rdocumentation.org/packages/base/versions/3.6.2/topics/date}{base::date}, and \href{https://www.rdocumentation.org/link/Sys.time?package=base&version=3.6.2}{base::Sys.time} \texttt{R} functions.

It first prints the value returned by the \texttt{date} function, then builds an object of class \texttt{difftime} representing the time elapsed since the date and time specified by the \hyperref[func.print_end_date.args.start]{start} argument, prints the object, and, finally, returns the object to the client code.

\phantomsection\label{func.print_end_date.value}
\subparagraph{Value}\label{value-3}

\hfill\break

An object of class \texttt{difftime} representing the time elapsed since the date and time specified by the \hyperref[func.print_end_date.args.start]{start} argument.

\phantomsection\label{func.print_end_date.code}
\subparagraph{Source Code}\label{source-code-3}

\hfill\break

The source code of the \href{https://github.com/AzKurban-edX-DS/Capstone-MovieLens/blob/main/r/src/support-functions/logging-functions.R\#L19}{print\_end\_date} function is shown below:

\begin{Shaded}
\begin{Highlighting}[]
\NormalTok{print\_end\_date }\OtherTok{\textless{}{-}} \ControlFlowTok{function}\NormalTok{(start)\{}
  \FunctionTok{print}\NormalTok{(}\FunctionTok{date}\NormalTok{())}
  \FunctionTok{print}\NormalTok{(}\FunctionTok{Sys.time}\NormalTok{() }\SpecialCharTok{{-}}\NormalTok{ start)}
\NormalTok{\}}
\end{Highlighting}
\end{Shaded}

\begin{noteblock}
The source code of the \href{https://github.com/AzKurban-edX-DS/Capstone-MovieLens/blob/main/r/src/support-functions/logging-functions.R\#L19}{print\_end\_date} function is defined in the \href{https://github.com/AzKurban-edX-DS/Capstone-MovieLens/blob/main/r/src/support-functions/logging-functions.R\#L19}{logging-functions.R} script on \emph{GitHub}.

\end{noteblock}

\paragraph{\texorpdfstring{\href{https://github.com/AzKurban-edX-DS/Capstone-MovieLens/blob/main/r/src/support-functions/logging-functions.R\#L23}{put\_end\_date} Function}{put\_end\_date Function}}\label{func.put_end_date}

\hfill\break

Outputs the system date and time to the currently open log (using the \hyperref[func.open_logfile]{open\_logfile} function described above).

The first value in the output is the system date and time at the time the function is called, and the second value is the time elapsed since the date and time specified by the \hyperref[func.put_end_date.args.start]{start} argument.

\phantomsection\label{func.put_end_date.usage}
\subparagraph{\texorpdfstring{\href{https://github.com/AzKurban-edX-DS/Capstone-MovieLens/blob/main/r/src/capstone-movielens.main.R\#L101}{Usage}}{Usage}}\label{usage-4}

\hfill\break

\begin{Shaded}
\begin{Highlighting}[]
\FunctionTok{put\_end\_date}\NormalTok{(start)}
\end{Highlighting}
\end{Shaded}

\phantomsection\label{func.put_end_date.args}
\subparagraph{Arguments}\label{arguments-2}

\hfill\break

\phantomsection\label{func.put_end_date.args.start}
\begin{itemize}
\tightlist
\item
  \textbf{start:} An object of class \texttt{POSIXct} representing the value of the moment from which time is measured.
\end{itemize}

\phantomsection\label{func.put_end_date.details}
\subparagraph{Details}\label{details-4}

\hfill\break

This function works similarly to the \hyperref[func.print_end_date]{print\_end\_date} function described above, but unlike the latter, it outputs the information to the currently open log file.

\phantomsection\label{func.put_end_date.value}
\subparagraph{Value}\label{value-4}

\hfill\break

An object of class \texttt{difftime} representing the time elapsed since the date and time specified by the \hyperref[func.put_end_date.args.start]{start} argument.

\phantomsection\label{func.put_end_date.code}
\subparagraph{Source Code}\label{source-code-4}

\hfill\break

The source code of the \href{https://github.com/AzKurban-edX-DS/Capstone-MovieLens/blob/main/r/src/support-functions/logging-functions.R\#L23}{put\_end\_date} function is shown below:

\begin{Shaded}
\begin{Highlighting}[]
\NormalTok{put\_end\_date }\OtherTok{\textless{}{-}} \ControlFlowTok{function}\NormalTok{(start)\{}
  \FunctionTok{put}\NormalTok{(}\FunctionTok{date}\NormalTok{())}
  \FunctionTok{put}\NormalTok{(}\FunctionTok{Sys.time}\NormalTok{() }\SpecialCharTok{{-}}\NormalTok{ start)}
\NormalTok{\}}
\end{Highlighting}
\end{Shaded}

\begin{noteblock}
The source code of the \href{https://github.com/AzKurban-edX-DS/Capstone-MovieLens/blob/main/r/src/support-functions/logging-functions.R\#L23}{put\_end\_date} function is defined in the \href{https://github.com/AzKurban-edX-DS/Capstone-MovieLens/blob/main/r/src/support-functions/logging-functions.R\#L23}{logging-functions.R} script on \emph{GitHub}.

\end{noteblock}

\newpage

\paragraph{\texorpdfstring{\href{https://github.com/AzKurban-edX-DS/Capstone-MovieLens/blob/main/r/src/support-functions/logging-functions.R\#L28}{print\_log} Function}{print\_log Function}}\label{func.print_log}

\hfill\break

Prints a multiline string specified by the \hyperref[func.print_log.args.msg]{msg} argument.

\phantomsection\label{func.print_log.usage}
\subparagraph{Usage}\label{usage-5}

\hfill\break

\begin{Shaded}
\begin{Highlighting}[]
\FunctionTok{print\_log}\NormalTok{(}\StringTok{"Hello World!"}\NormalTok{)}
\end{Highlighting}
\end{Shaded}

\begin{verbatim}
## Hello World!
\end{verbatim}

\phantomsection\label{func.print_log.args}
\subparagraph{Arguments}\label{arguments-3}

\hfill\break

\phantomsection\label{func.print_log.args.msg}
\begin{itemize}
\tightlist
\item
  \textbf{msg:} A string to print (can be multiline).
\end{itemize}

\phantomsection\label{func.print_log.details}
\subparagraph{Details}\label{details-5}

\hfill\break

This function internally calls \href{https://www.rdocumentation.org/packages/stringr/versions/1.5.0/topics/str_glue}{stringr::str\_glue} and \href{https://www.rdocumentation.org/packages/base/versions/3.6.2/topics/print}{base::print} \texttt{R} functions to print the (possibly multiline) string specified by the \hyperref[func.print_log.args.msg]{msg} argument.

\phantomsection\label{func.print_log.value}
\subparagraph{Value}\label{value-5}

\hfill\break

A \href{https://glue.tidyverse.org/reference/glue.html}{glue object}, as created by \href{https://glue.tidyverse.org/reference/as_glue.html}{glue::as\_glue()}.

\phantomsection\label{func.print_log.code}
\subparagraph{Source Code}\label{source-code-5}

\hfill\break

The source code of the \href{https://github.com/AzKurban-edX-DS/Capstone-MovieLens/blob/main/r/src/support-functions/logging-functions.R\#L28}{print\_log} function is shown below:

\begin{Shaded}
\begin{Highlighting}[]
\NormalTok{print\_log }\OtherTok{\textless{}{-}} \ControlFlowTok{function}\NormalTok{(msg)\{}
  \FunctionTok{print}\NormalTok{(}\FunctionTok{str\_glue}\NormalTok{(msg))}
\NormalTok{\}}
\end{Highlighting}
\end{Shaded}

\begin{noteblock}
The source code of the \href{https://github.com/AzKurban-edX-DS/Capstone-MovieLens/blob/main/r/src/support-functions/logging-functions.R\#L28}{print\_log} function is defined in the \href{https://github.com/AzKurban-edX-DS/Capstone-MovieLens/blob/main/r/src/support-functions/logging-functions.R\#L28}{logging-functions.R} script on \emph{GitHub}.

\end{noteblock}

\newpage

\paragraph{\texorpdfstring{\href{https://github.com/AzKurban-edX-DS/Capstone-MovieLens/blob/main/r/src/support-functions/logging-functions.R\#L31}{put\_log} Function}{put\_log Function}}\label{func.put_log}

\hfill\break

Outputs a multiline string specified by the \hyperref[func.put_log.args.msg]{msg} argument to the currently opened log (using the \hyperref[func.open_logfile]{open\_logfile} function described above).

\phantomsection\label{func.put_log.usage}
\subparagraph{\texorpdfstring{\href{https://github.com/AzKurban-edX-DS/Capstone-MovieLens/blob/main/r/src/capstone-movielens.main.R\#L231}{Usage}}{Usage}}\label{usage-6}

\hfill\break

\begin{Shaded}
\begin{Highlighting}[]
\FunctionTok{put\_log}\NormalTok{(}\StringTok{"Computing Average Ratings per User (User Mean Ratings)..."}\NormalTok{)}
\end{Highlighting}
\end{Shaded}

\begin{verbatim}
## Computing Average Ratings per User (User Mean Ratings)...
\end{verbatim}

\phantomsection\label{func.put_log.args}
\subparagraph{Arguments}\label{arguments-4}

\hfill\break

\phantomsection\label{func.put_log.args.msg}
\begin{itemize}
\tightlist
\item
  \textbf{msg:} A string to print (can be multiline).
\end{itemize}

\phantomsection\label{func.put_log.details}
\subparagraph{Details}\label{details-6}

\hfill\break

This function internally calls \href{https://www.rdocumentation.org/packages/stringr/versions/1.5.0/topics/str_glue}{stringr::str\_glue} and \href{https://www.rdocumentation.org/packages/logr/versions/1.3.4/topics/log_print}{logr::put} \texttt{R} functions to output the (possibly multiline) string specified by the \hyperref[func.put_log.args.msg]{msg} argument.

The function works similarly to the \hyperref[func.print_log]{print\_log} function described above, but unlike the latter, it outputs the information to the currently open log file.

\phantomsection\label{func.put_log.value}
\subparagraph{Value}\label{value-6}

\hfill\break

A \href{https://glue.tidyverse.org/reference/glue.html}{glue object}, as created by \href{https://glue.tidyverse.org/reference/as_glue.html}{glue::as\_glue()}.

\phantomsection\label{func.put_log.code}
\subparagraph{Source Code}\label{source-code-6}

\hfill\break

The source code of the \href{https://github.com/AzKurban-edX-DS/Capstone-MovieLens/blob/main/r/src/support-functions/logging-functions.R\#L31}{put\_log} function is shown below:

\begin{Shaded}
\begin{Highlighting}[]
\NormalTok{put\_log }\OtherTok{\textless{}{-}} \ControlFlowTok{function}\NormalTok{(msg)\{}
  \FunctionTok{put}\NormalTok{(}\FunctionTok{str\_glue}\NormalTok{(msg))}
\NormalTok{\}}
\end{Highlighting}
\end{Shaded}

\begin{noteblock}
The source code of the \href{https://github.com/AzKurban-edX-DS/Capstone-MovieLens/blob/main/r/src/support-functions/logging-functions.R\#L31}{put\_log} function is defined in the \href{https://github.com/AzKurban-edX-DS/Capstone-MovieLens/blob/main/r/src/support-functions/logging-functions.R\#L31}{logging-functions.R} script on \emph{GitHub}.

\end{noteblock}

\newpage

\paragraph{\texorpdfstring{\href{https://github.com/AzKurban-edX-DS/Capstone-MovieLens/blob/main/r/src/support-functions/logging-functions.R\#L35}{get\_log1} Function}{get\_log1 Function}}\label{func.get_log1}

\hfill\break

Creates a character vector from the template passed in the \hyperref[func.get_log1.args.msg_template]{msg\_template} argument, using the value specified by the \hyperref[func.get_log1.args.arg1]{arg1} argument to replace all the occurrences of the pattern \emph{``\%1''} found in the template.

\phantomsection\label{func.get_log1.usage}
\subparagraph{Usage}\label{usage-7}

\hfill\break

\begin{Shaded}
\begin{Highlighting}[]
  \FunctionTok{put}\NormalTok{(}\FunctionTok{get\_log1}\NormalTok{(msg\_template, arg1))}
\end{Highlighting}
\end{Shaded}

\begin{noteblock}
In the \href{https://github.com/AzKurban-edX-DS/Capstone-MovieLens/blob/main/r/src/support-functions/logging-functions.R\#L42}{code snippet} above, the function \texttt{get\_log1} is called internally from the \hyperref[func.put_log1]{put\_log1} function described below in \emph{this Section}.

\end{noteblock}

\phantomsection\label{func.get_log1.args}
\subparagraph{Arguments}\label{arguments-5}

\hfill\break

\phantomsection\label{func.get_log1.args.msg_template}
\begin{itemize}
\tightlist
\item
  \textbf{msg\_template:} Any object coercible to a character vector containing at least one message template element, i.e., a string (possibly multiline) containing at least one occurrence of the pattern \emph{``\%1''} to be replaced by the value specified by the \hyperref[func.get_log1.args.arg1]{arg1} argument.
\end{itemize}

\phantomsection\label{func.get_log1.args.arg1}
\begin{itemize}
\tightlist
\item
  \textbf{arg1:} Value to replace all occurrences of the pattern \emph{``\%1''} in the message template passed in the \hyperref[func.get_log1.args.msg_template]{msg\_template} argument.
\end{itemize}

\phantomsection\label{func.get_log1.details}
\subparagraph{Details}\label{details-7}

\hfill\break

This function internally calls \href{https://www.rdocumentation.org/packages/stringr/versions/1.5.0/topics/str_glue}{stringr::str\_glue} and \href{https://www.rdocumentation.org/packages/stringr/versions/1.5.0/topics/str_replace}{stringr::str\_replace\_all} \texttt{R} functions to form a character vector from the template passed in the \hyperref[func.get_log1.args.msg_template]{msg\_template} argument, using the value specified by the \hyperref[func.get_log1.args.arg1]{arg1} argument to replace all the occurrences of the pattern \emph{``\%1''} found in the template.

\begin{noteblock}
This is an auxiliary function intended for use in other custom logging functions, such as the \hyperref[func.put_log1]{put\_log1} (as shown in the \hyperref[func.get_log1.usage]{Usage} section of \emph{this Description}) and \hyperref[func.print_log1]{print\_log1} described below in \emph{this Section}.

\end{noteblock}

\phantomsection\label{func.get_log1.value}
\subparagraph{Value}\label{value-7}

\hfill\break

A glue object, returned by the internally called \href{https://www.rdocumentation.org/packages/stringr/versions/1.5.0/topics/str_glue}{stringr::str\_glue} function, created from the character vector returned by the (also internally called) function \href{https://www.rdocumentation.org/packages/stringr/versions/1.5.0/topics/str_replace}{stringr::str\_replace\_all}.

\newpage

\phantomsection\label{func.get_log1.code}
\subparagraph{Source Code}\label{source-code-7}

\hfill\break

The source code of the \href{https://github.com/AzKurban-edX-DS/Capstone-MovieLens/blob/main/r/src/support-functions/logging-functions.R\#L35}{get\_log1} function is shown below:

\begin{Shaded}
\begin{Highlighting}[]
\NormalTok{get\_log1 }\OtherTok{\textless{}{-}} \ControlFlowTok{function}\NormalTok{(msg\_template, arg1) \{}
  \FunctionTok{str\_glue}\NormalTok{(}\FunctionTok{str\_replace\_all}\NormalTok{(msg\_template, }\StringTok{"\%1"}\NormalTok{, }\FunctionTok{as.character}\NormalTok{(arg1)))}
\NormalTok{\}}
\end{Highlighting}
\end{Shaded}

\begin{noteblock}
The source code of the \href{https://github.com/AzKurban-edX-DS/Capstone-MovieLens/blob/main/r/src/support-functions/logging-functions.R\#L35}{get\_log1} function is defined in the \href{https://github.com/AzKurban-edX-DS/Capstone-MovieLens/blob/main/r/src/support-functions/logging-functions.R\#L35}{logging-functions.R} script on \emph{GitHub}.

\end{noteblock}

\newpage

\paragraph{\texorpdfstring{\href{https://github.com/AzKurban-edX-DS/Capstone-MovieLens/blob/main/r/src/support-functions/logging-functions.R\#L38}{print\_log1} Function}{print\_log1 Function}}\label{func.print_log1}

\hfill\break

Prints a log created from the template passed in the \hyperref[func.print_log1.args.msg_template]{msg\_template} argument, using the value specified by the \hyperref[func.print_log1.args.arg1]{arg1} argument to replace all the occurrences of the pattern \emph{``\%1''} found in the template.

\phantomsection\label{func.print_log1.usage}
\subparagraph{\texorpdfstring{\hyperref[chunk.main_script.l177m]{Usage}}{Usage}}\label{usage-8}

\hfill\break

\begin{Shaded}
\begin{Highlighting}[]
\NormalTok{which\_min\_deviation }\OtherTok{\textless{}{-}}\NormalTok{ deviation[}\FunctionTok{which.min}\NormalTok{(deviation.RMSE)]}
\NormalTok{min\_rmse }\OtherTok{=} \FunctionTok{min}\NormalTok{(deviation.RMSE)}

\FunctionTok{print\_log1}\NormalTok{(}\StringTok{"Minimum RMSE is achieved when the deviation from the mean is: \%1"}\NormalTok{,}
\NormalTok{         which\_min\_deviation)}
\end{Highlighting}
\end{Shaded}

\begin{verbatim}
## Minimum RMSE is achieved when the deviation from the mean is: 0
\end{verbatim}

\begin{Shaded}
\begin{Highlighting}[]
\FunctionTok{print\_log1}\NormalTok{(}\StringTok{"Is the previously computed RMSE the best for the current model: \%1"}\NormalTok{,}
\NormalTok{         mu.RMSE }\SpecialCharTok{==}\NormalTok{ min\_rmse)}
\end{Highlighting}
\end{Shaded}

\begin{verbatim}
## Is the previously computed RMSE the best for the current model: TRUE
\end{verbatim}

\phantomsection\label{func.print_log1.args}
\subparagraph{Arguments}\label{arguments-6}

\hfill\break

\phantomsection\label{func.print_log1.args.msg_template}
\begin{itemize}
\tightlist
\item
  \textbf{msg\_template:} Any object coercible to a character vector containing at least one message template element, i.e., a string (possibly multiline) containing at least one occurrence of the pattern \emph{``\%1''} to be replaced by the value specified by the \hyperref[func.print_log1.args.arg1]{arg1} argument.
\end{itemize}

\phantomsection\label{func.print_log1.args.arg1}
\begin{itemize}
\tightlist
\item
  \textbf{arg1:} Value to replace all occurrences of the pattern \emph{``\%1''} in the message template passed in the \hyperref[func.print_log1.args.msg_template]{msg\_template} argument.
\end{itemize}

\phantomsection\label{func.print_log1.details}
\subparagraph{Details}\label{details-8}

\hfill\break

This function internally calls the \href{https://www.rdocumentation.org/packages/base/versions/3.6.2/topics/print}{base::print} \texttt{R} function to print the object returned by the \hyperref[func.get_log1]{get\_log1} function (also called internally) described above.

\phantomsection\label{func.print_log1.value}
\subparagraph{Value}\label{value-8}

\hfill\break

A glue object, returned by the internally called function \hyperref[func.get_log1]{get\_log1} described above.

\newpage

\phantomsection\label{func.print_log1.code}
\subparagraph{Source Code}\label{source-code-8}

\hfill\break

The source code of the \href{https://github.com/AzKurban-edX-DS/Capstone-MovieLens/blob/main/r/src/support-functions/logging-functions.R\#L38}{print\_log1} function is shown below:

\begin{Shaded}
\begin{Highlighting}[]
\NormalTok{print\_log1 }\OtherTok{\textless{}{-}} \ControlFlowTok{function}\NormalTok{(msg\_template, arg1)\{}
  \FunctionTok{print}\NormalTok{(}\FunctionTok{get\_log1}\NormalTok{(msg\_template, arg1))}
\NormalTok{\}}
\end{Highlighting}
\end{Shaded}

\begin{noteblock}
The source code of the \href{https://github.com/AzKurban-edX-DS/Capstone-MovieLens/blob/main/r/src/support-functions/logging-functions.R\#L38}{print\_log1} function is defined in the \href{https://github.com/AzKurban-edX-DS/Capstone-MovieLens/blob/main/r/src/support-functions/logging-functions.R\#L38}{logging-functions.R} script on \emph{GitHub}.

\end{noteblock}

\newpage

\paragraph{\texorpdfstring{\href{https://github.com/AzKurban-edX-DS/Capstone-MovieLens/blob/main/r/src/support-functions/logging-functions.R\#L41}{put\_log1} Function}{put\_log1 Function}}\label{func.put_log1}

\hfill\break

Outputs a log to the currently open log file (using the \hyperref[func.open_logfile]{open\_logfile} function described above), created from the template passed in the \hyperref[func.put_log1.args.msg_template]{msg\_template} argument, using the value specified by the \hyperref[func.put_log1.args.arg1]{arg1} argument to replace all the occurrences of the pattern \emph{``\%1''} found in the template.

\phantomsection\label{func.put_log1.usage}
\subparagraph{\texorpdfstring{\href{https://github.com/AzKurban-edX-DS/Capstone-MovieLens/blob/main/r/src/capstone-movielens.main.R\#L177}{Usage}}{Usage}}\label{usage-9}

\hfill\break

\begin{Shaded}
\begin{Highlighting}[]
\NormalTok{which\_min\_deviation }\OtherTok{\textless{}{-}}\NormalTok{ deviation[}\FunctionTok{which.min}\NormalTok{(deviation.RMSE)]}
\NormalTok{min\_rmse }\OtherTok{=} \FunctionTok{min}\NormalTok{(deviation.RMSE)}
\end{Highlighting}
\end{Shaded}

\begin{Shaded}
\begin{Highlighting}[]
\FunctionTok{put\_log1}\NormalTok{(}\StringTok{"Minimum RMSE is achieved when the deviation from the mean is: \%1"}\NormalTok{,}
\NormalTok{         which\_min\_deviation)}

\FunctionTok{put\_log1}\NormalTok{(}\StringTok{"Is the previously computed RMSE the best for the current model: \%1"}\NormalTok{,}
\NormalTok{         mu.RMSE }\SpecialCharTok{==}\NormalTok{ min\_rmse)}
\end{Highlighting}
\end{Shaded}

\begin{verbatim}
## Minimum RMSE is achieved when the deviation from the mean is: 0
\end{verbatim}

\begin{verbatim}
## Is the previously computed RMSE the best for the current model: TRUE
\end{verbatim}

\phantomsection\label{func.put_log1.args}
\subparagraph{Arguments}\label{arguments-7}

\hfill\break

\phantomsection\label{func.put_log1.args.msg_template}
\begin{itemize}
\tightlist
\item
  \textbf{msg\_template:} Any object coercible to a character vector containing at least one message template element, i.e., a string (possibly multiline) containing at least one occurrence of the pattern \emph{``\%1''} to be replaced by the value specified by the \hyperref[func.put_log1.args.arg1]{arg1} argument.
\end{itemize}

\phantomsection\label{func.put_log1.args.arg1}
\begin{itemize}
\tightlist
\item
  \textbf{arg1:} Value to replace all occurrences of the pattern \emph{``\%1''} in the message template passed in the \hyperref[func.put_log1.args.msg_template]{msg\_template} argument.
\end{itemize}

\phantomsection\label{func.put_log1.details}
\subparagraph{Details}\label{details-9}

\hfill\break

This function internally calls the \href{https://www.rdocumentation.org/packages/logr/versions/1.3.4/topics/log_print}{logr::put} \texttt{R} function to output the log object returned by the (also called internally) \hyperref[func.get_log1]{get\_log1} function described above.

The function works similarly to the \hyperref[func.print_log1]{print\_log1} function described above, but unlike the latter, it outputs the information to the currently open log file.

\phantomsection\label{func.put_log1.value}
\subparagraph{Value}\label{value-9}

\hfill\break

A glue object, returned by the internally called function \hyperref[func.get_log1]{get\_log1} described above.

\newpage

\phantomsection\label{func.put_log1.code}
\subparagraph{Source Code}\label{source-code-9}

\hfill\break

The source code of the \href{https://github.com/AzKurban-edX-DS/Capstone-MovieLens/blob/main/r/src/support-functions/logging-functions.R\#L41}{put\_log1} function is shown below:

\begin{Shaded}
\begin{Highlighting}[]
\NormalTok{put\_log1 }\OtherTok{\textless{}{-}} \ControlFlowTok{function}\NormalTok{(msg\_template, arg1)\{}
  \FunctionTok{put}\NormalTok{(}\FunctionTok{get\_log1}\NormalTok{(msg\_template, arg1))}
\NormalTok{\}}
\end{Highlighting}
\end{Shaded}

\begin{noteblock}
The source code of the \href{https://github.com/AzKurban-edX-DS/Capstone-MovieLens/blob/main/r/src/support-functions/logging-functions.R\#L41}{put\_log1} function is defined in the \href{https://github.com/AzKurban-edX-DS/Capstone-MovieLens/blob/main/r/src/support-functions/logging-functions.R\#L41}{logging-functions.R} script on \emph{GitHub}.

\end{noteblock}

\newpage

\paragraph{\texorpdfstring{\href{https://github.com/AzKurban-edX-DS/Capstone-MovieLens/blob/main/r/src/support-functions/logging-functions.R\#L45}{get\_log2} Function}{get\_log2 Function}}\label{func.get_log2}

\hfill\break

Creates a character vector from the template passed in the \hyperref[func.get_log2.args.msg_template]{msg\_template} argument, using the values specified by the \hyperref[func.get_log2.args.arg1]{arg1} and \hyperref[func.get_log2.args.arg2]{arg2} arguments to replace all the occurrences of the placeholders \emph{``\%1''} and \emph{``\%2''}, respectively, found in the template.

\phantomsection\label{func.get_log2.usage}
\subparagraph{Usage}\label{usage-10}

\hfill\break

\begin{Shaded}
\begin{Highlighting}[]
  \FunctionTok{put}\NormalTok{(}\FunctionTok{get\_log2}\NormalTok{(msg\_template, arg1, arg2))}
\end{Highlighting}
\end{Shaded}

\begin{noteblock}
In the \href{https://github.com/AzKurban-edX-DS/Capstone-MovieLens/blob/main/r/src/support-functions/logging-functions.R\#L55}{code snippet} above, the function \texttt{get\_log2} is called internally from the \hyperref[func.put_log2]{put\_log2} function described below in \emph{this Section}.

\end{noteblock}

\phantomsection\label{func.get_log2.args}
\subparagraph{Arguments}\label{arguments-8}

\hfill\break

\phantomsection\label{func.get_log2.args.msg_template}
\begin{itemize}
\tightlist
\item
  \textbf{msg\_template:} Any object coercible to a character vector containing at least one message template element, i.e., a string (possibly multiline) containing at least one occurrence of each of the placeholders \emph{``\%1''} and \emph{``\%2''}, to be replaced by the values specified by the \hyperref[func.get_log2.args.arg1]{arg1} and \hyperref[func.get_log2.args.arg2]{arg2} arguments, respectively.
\end{itemize}

\phantomsection\label{func.get_log2.args.arg1}
\begin{itemize}
\tightlist
\item
  \textbf{arg1:} Value to replace all occurrences of the placeholder \emph{``\%1''} in the message template passed in the \hyperref[func.get_log2.args.msg_template]{msg\_template} argument.
\end{itemize}

\phantomsection\label{func.get_log2.args.arg2}
\begin{itemize}
\tightlist
\item
  \textbf{arg2:} Value to replace all occurrences of the placeholder \emph{``\%2''} in the message template passed in the \hyperref[func.get_log2.args.msg_template]{msg\_template} argument.
\end{itemize}

\phantomsection\label{func.get_log2.details}
\subparagraph{Details}\label{details-10}

\hfill\break

This function internally calls \href{https://www.rdocumentation.org/packages/stringr/versions/1.5.0/topics/str_glue}{stringr::str\_glue} and \href{https://www.rdocumentation.org/packages/stringr/versions/1.5.0/topics/str_replace}{stringr::str\_replace\_all} \texttt{R} functions to form a character vector from the template passed in the \hyperref[func.get_log2.args.msg_template]{msg\_template} argument, using the values specified by the \hyperref[func.get_log2.args.arg1]{arg1} and \hyperref[func.get_log2.args.arg2]{arg2} arguments to replace all the occurrences of the placeholders \emph{``\%1''} and \emph{``\%2''}, respectively, found in the template.

\begin{noteblock}
This is an auxiliary function intended for use in other custom logging functions, such as the \hyperref[func.put_log2]{put\_log2} (as shown in the \hyperref[func.get_log2.usage]{Usage} section of \emph{this Description}) and \hyperref[func.print_log2]{print\_log2} described below in \emph{this Section}.

\end{noteblock}

\phantomsection\label{func.get_log2.value}
\subparagraph{Value}\label{value-10}

\hfill\break

A glue object, returned by the internally called \href{https://www.rdocumentation.org/packages/stringr/versions/1.5.0/topics/str_glue}{stringr::str\_glue} function, created from the character vector returned by the (also internally called) function \href{https://www.rdocumentation.org/packages/stringr/versions/1.5.0/topics/str_replace}{stringr::str\_replace\_all}.

\phantomsection\label{func.get_log2.code}
\subparagraph{Source Code}\label{source-code-10}

\hfill\break

The source code of the \href{https://github.com/AzKurban-edX-DS/Capstone-MovieLens/blob/main/r/src/support-functions/logging-functions.R\#L45}{get\_log2} function is shown below:

\begin{Shaded}
\begin{Highlighting}[]
\NormalTok{get\_log2 }\OtherTok{\textless{}{-}} \ControlFlowTok{function}\NormalTok{(msg\_template, arg1, arg2) \{}
\NormalTok{  msg\_template }\SpecialCharTok{|\textgreater{}} 
    \FunctionTok{str\_replace\_all}\NormalTok{(}\StringTok{"\%1"}\NormalTok{, }\FunctionTok{as.character}\NormalTok{(arg1)) }\SpecialCharTok{|\textgreater{}}
    \FunctionTok{str\_replace\_all}\NormalTok{(}\StringTok{"\%2"}\NormalTok{, }\FunctionTok{as.character}\NormalTok{(arg2)) }\SpecialCharTok{|\textgreater{}}
    \FunctionTok{str\_glue}\NormalTok{()}
\NormalTok{\}}
\end{Highlighting}
\end{Shaded}

\begin{noteblock}
The source code of the \href{https://github.com/AzKurban-edX-DS/Capstone-MovieLens/blob/main/r/src/support-functions/logging-functions.R\#L45}{get\_log2} function is defined in the \href{https://github.com/AzKurban-edX-DS/Capstone-MovieLens/blob/main/r/src/support-functions/logging-functions.R\#L45}{logging-functions.R} script on \emph{GitHub}.

\end{noteblock}

\newpage

\paragraph{\texorpdfstring{\href{https://github.com/AzKurban-edX-DS/Capstone-MovieLens/blob/main/r/src/support-functions/logging-functions.R\#L51}{print\_log2} Function}{print\_log2 Function}}\label{func.print_log2}

\hfill\break

Prints a log created from the template passed in the \hyperref[func.print_log2.args.msg_template]{msg\_template} argument, using the values specified by the \hyperref[func.print_log2.args.arg1]{arg1} and \hyperref[func.print_log2.args.arg2]{arg2} arguments to replace all the occurrences of the placeholders \emph{``\%1''} and \emph{``\%2''}, respectively, found in the template.

\phantomsection\label{func.print_log2.usage}
\subparagraph{\texorpdfstring{\href{https://github.com/AzKurban-edX-DS/Capstone-MovieLens/blob/main/r/src/capstone-movielens.main.R\#L119}{Usage}}{Usage}}\label{usage-11}

\hfill\break

\begin{Shaded}
\begin{Highlighting}[]
  \FunctionTok{print\_log2}\NormalTok{(}\StringTok{"\%1{-}Fold Cross Validation ultimate RMSE: \%2"}\NormalTok{, CVFolds\_N, mu.RMSE)}
\end{Highlighting}
\end{Shaded}

\begin{verbatim}
## 5-Fold Cross Validation ultimate RMSE: 1.06034618796622
\end{verbatim}

\phantomsection\label{func.print_log2.args}
\subparagraph{Arguments}\label{arguments-9}

\hfill\break

\phantomsection\label{func.print_log2.args.msg_template}
\begin{itemize}
\tightlist
\item
  \textbf{msg\_template:} Any object coercible to a character vector containing at least one message template element, i.e., a string (possibly multiline) containing at least one occurrence of each of the placeholders \emph{``\%1''} and \emph{``\%2''}, to be replaced by the values specified by the \hyperref[func.print_log2.args.arg1]{arg1} and \hyperref[func.print_log2.args.arg2]{arg2} arguments, respectively.
\end{itemize}

\phantomsection\label{func.print_log2.args.arg1}
\begin{itemize}
\tightlist
\item
  \textbf{arg1:} Value to replace all occurrences of the placeholder \emph{``\%1''} in the message template passed in the \hyperref[func.print_log2.args.msg_template]{msg\_template} argument.
\end{itemize}

\phantomsection\label{func.print_log2.args.arg2}
\begin{itemize}
\tightlist
\item
  \textbf{arg2:} Value to replace all occurrences of the placeholder \emph{``\%2''} in the message template passed in the \hyperref[func.print_log2.args.msg_template]{msg\_template} argument.
\end{itemize}

\phantomsection\label{func.print_log2.details}
\subparagraph{Details}\label{details-11}

\hfill\break

This function internally calls the \href{https://www.rdocumentation.org/packages/base/versions/3.6.2/topics/print}{base::print} \texttt{R} function to print the object returned by the \hyperref[func.get_log2]{get\_log2} function (also called internally) described above.

\phantomsection\label{func.print_log2.value}
\subparagraph{Value}\label{value-11}

\hfill\break

A glue object, returned by the internally called function \hyperref[func.get_log2]{get\_log2} described above.

\phantomsection\label{func.print_log2.code}
\subparagraph{Source Code}\label{source-code-11}

\hfill\break

The source code of the \href{https://github.com/AzKurban-edX-DS/Capstone-MovieLens/blob/main/r/src/support-functions/logging-functions.R\#L51}{print\_log2} function is shown below:

\begin{Shaded}
\begin{Highlighting}[]
\NormalTok{print\_log2 }\OtherTok{\textless{}{-}} \ControlFlowTok{function}\NormalTok{(msg\_template, arg1, arg2)\{}
  \FunctionTok{print}\NormalTok{(}\FunctionTok{get\_log2}\NormalTok{(msg\_template, arg1, arg2))}
\NormalTok{\}}
\end{Highlighting}
\end{Shaded}

\begin{noteblock}
The source code of the \href{https://github.com/AzKurban-edX-DS/Capstone-MovieLens/blob/main/r/src/support-functions/logging-functions.R\#L51}{print\_log2} function is defined in the \href{https://github.com/AzKurban-edX-DS/Capstone-MovieLens/blob/main/r/src/support-functions/logging-functions.R\#L51}{logging-functions.R} script on \emph{GitHub}.

\end{noteblock}

\newpage

\paragraph{\texorpdfstring{\href{https://github.com/AzKurban-edX-DS/Capstone-MovieLens/blob/main/r/src/support-functions/logging-functions.R\#L54}{put\_log2} Function}{put\_log2 Function}}\label{func.put_log2}

\hfill\break

Outputs a log to the currently open log file (using the \hyperref[func.open_logfile]{open\_logfile} function described above), created from the template passed in the \hyperref[func.put_log2.args.msg_template]{msg\_template} argument, using the values specified by the \hyperref[func.put_log2.args.arg1]{arg1} and \hyperref[func.put_log2.args.arg2]{arg2} arguments to replace all the occurrences of the placeholders \emph{``\%1''} and \emph{``\%2''}, respectively, found in the template.

\phantomsection\label{func.put_log2.usage}
\subparagraph{\texorpdfstring{\href{https://github.com/AzKurban-edX-DS/Capstone-MovieLens/blob/main/r/src/capstone-movielens.main.R\#L119}{Usage}}{Usage}}\label{usage-12}

\hfill\break

\begin{Shaded}
\begin{Highlighting}[]
  \FunctionTok{put\_log2}\NormalTok{(}\StringTok{"\%1{-}Fold Cross Validation ultimate RMSE: \%2"}\NormalTok{, CVFolds\_N, mu.RMSE)}
\end{Highlighting}
\end{Shaded}

\begin{verbatim}
## 5-Fold Cross Validation ultimate RMSE: 1.06034618796622
\end{verbatim}

\phantomsection\label{func.put_log2.args}
\subparagraph{Arguments}\label{arguments-10}

\hfill\break

\phantomsection\label{func.put_log2.args.msg_template}
\begin{itemize}
\tightlist
\item
  \textbf{msg\_template:} Any object coercible to a character vector containing at least one message template element, i.e., a string (possibly multiline) containing at least one occurrence of each of the placeholders \emph{``\%1''} and \emph{``\%2''}, to be replaced by the values specified by the \hyperref[func.put_log2.args.arg1]{arg1} and \hyperref[func.put_log2.args.arg2]{arg2} arguments, respectively.
\end{itemize}

\phantomsection\label{func.put_log2.args.arg1}
\begin{itemize}
\tightlist
\item
  \textbf{arg1:} Value to replace all occurrences of the placeholder \emph{``\%1''} in the message template passed in the \hyperref[func.put_log2.args.msg_template]{msg\_template} argument.
\end{itemize}

\phantomsection\label{func.put_log2.args.arg2}
\begin{itemize}
\tightlist
\item
  \textbf{arg2:} Value to replace all occurrences of the placeholder \emph{``\%2''} in the message template passed in the \hyperref[func.put_log2.args.msg_template]{msg\_template} argument.
\end{itemize}

\phantomsection\label{func.put_log2.details}
\subparagraph{Details}\label{details-12}

\hfill\break

This function internally calls the \href{https://www.rdocumentation.org/packages/logr/versions/1.3.4/topics/log_print}{logr::put} \texttt{R} function to output the log object returned by the (also called internally) \hyperref[func.get_log2]{get\_log2} function described above.

The function works similarly to the \hyperref[func.print_log2]{print\_log2} function described above, but unlike the latter, it outputs the information to the currently open log file.

\phantomsection\label{func.put_log2.value}
\subparagraph{Value}\label{value-12}

\hfill\break

A glue object, returned by the internally called function \hyperref[func.get_log2]{get\_log2} described above.

\newpage

\phantomsection\label{func.put_log2.code}
\subparagraph{Source Code}\label{source-code-12}

\hfill\break

The source code of the \href{https://github.com/AzKurban-edX-DS/Capstone-MovieLens/blob/main/r/src/support-functions/logging-functions.R\#L54}{put\_log2} function is shown below:

\begin{Shaded}
\begin{Highlighting}[]
\NormalTok{put\_log2 }\OtherTok{\textless{}{-}} \ControlFlowTok{function}\NormalTok{(msg\_template, arg1, arg2)\{}
  \FunctionTok{put}\NormalTok{(}\FunctionTok{get\_log2}\NormalTok{(msg\_template, arg1, arg2))}
\NormalTok{\}}
\end{Highlighting}
\end{Shaded}

\begin{noteblock}
The source code of the \href{https://github.com/AzKurban-edX-DS/Capstone-MovieLens/blob/main/r/src/support-functions/logging-functions.R\#L54}{put\_log2} function is defined in the \href{https://github.com/AzKurban-edX-DS/Capstone-MovieLens/blob/main/r/src/support-functions/logging-functions.R\#L54}{logging-functions.R} script on \emph{GitHub}.

\end{noteblock}

\newpage

\paragraph{\texorpdfstring{\href{https://github.com/AzKurban-edX-DS/Capstone-MovieLens/blob/main/r/src/support-functions/logging-functions.R\#L58}{get\_log3} Function}{get\_log3 Function}}\label{func.get_log3}

\hfill\break

Creates a character vector from the template passed in the \hyperref[func.get_log3.args.msg_template]{msg\_template} argument, using the values specified by the \hyperref[func.get_log3.args.arg1]{arg1}, \hyperref[func.get_log3.args.arg2]{arg2}, and \hyperref[func.get_log3.args.arg3]{arg3} arguments to replace all the occurrences of the placeholders \emph{``\%1''}, \emph{``\%2''}, and \emph{``\%3''}, respectively, found in the template.

\phantomsection\label{func.get_log3.usage}
\subparagraph{Usage}\label{usage-13}

\hfill\break

\begin{Shaded}
\begin{Highlighting}[]
  \FunctionTok{put}\NormalTok{(}\FunctionTok{get\_log3}\NormalTok{(msg\_template, arg1, arg2, arg3))}
\end{Highlighting}
\end{Shaded}

\begin{noteblock}
In the \href{https://github.com/AzKurban-edX-DS/Capstone-MovieLens/blob/main/r/src/support-functions/logging-functions.R\#L69}{code snippet} above, the function \texttt{get\_log3} is called internally from the \hyperref[func.put_log3]{put\_log3} function described below in \emph{this Section}.

\end{noteblock}

\phantomsection\label{func.get_log3.args}
\subparagraph{Arguments}\label{arguments-11}

\hfill\break

\phantomsection\label{func.get_log3.args.msg_template}
\begin{itemize}
\tightlist
\item
  \textbf{msg\_template:} Any object coercible to a character vector containing at least one message template element, i.e., a string (possibly multiline) containing at least one occurrence of each of the placeholders \emph{``\%1''}, \emph{``\%2''}, and \emph{``\%3''} to be replaced by the values specified by the \hyperref[func.get_log3.args.arg1]{arg1}, \hyperref[func.get_log3.args.arg2]{arg2}, and \hyperref[func.get_log3.args.arg3]{arg3} arguments, respectively.
\end{itemize}

\phantomsection\label{func.get_log3.args.arg1}
\begin{itemize}
\tightlist
\item
  \textbf{arg1:} Value to replace all occurrences of the placeholder \emph{``\%1''} in the message template passed in the \hyperref[func.get_log3.args.msg_template]{msg\_template} argument.
\end{itemize}

\phantomsection\label{func.get_log3.args.arg2}
\begin{itemize}
\tightlist
\item
  \textbf{arg2:} Value to replace all occurrences of the placeholder \emph{``\%2''} in the message template passed in the \hyperref[func.get_log3.args.msg_template]{msg\_template} argument.
\end{itemize}

\phantomsection\label{func.get_log3.args.arg3}
\begin{itemize}
\tightlist
\item
  \textbf{arg3:} Value to replace all occurrences of the placeholder \emph{``\%3''} in the message template passed in the \hyperref[func.get_log3.args.msg_template]{msg\_template} argument.
\end{itemize}

\phantomsection\label{func.get_log3.details}
\subparagraph{Details}\label{details-13}

\hfill\break

This function internally calls \href{https://www.rdocumentation.org/packages/stringr/versions/1.5.0/topics/str_glue}{stringr::str\_glue} and \href{https://www.rdocumentation.org/packages/stringr/versions/1.5.0/topics/str_replace}{stringr::str\_replace\_all} \texttt{R} functions to form a character vector from the template passed in the \hyperref[func.get_log3.args.msg_template]{msg\_template} argument, using the values specified by the \hyperref[func.get_log3.args.arg1]{arg1}, \hyperref[func.get_log3.args.arg2]{arg2}, and \hyperref[func.get_log3.args.arg3]{arg3} arguments to replace all the occurrences of the placeholders \emph{``\%1''}, \emph{``\%2''}, and \emph{``\%3''}, respectively, found in the template.

\begin{noteblock}
This is an auxiliary function intended for use in other custom logging functions, such as the \hyperref[func.put_log3]{put\_log3} (as shown in the \hyperref[func.get_log3.usage]{Usage} section of \emph{this Description}) and \hyperref[func.print_log3]{print\_log3} described below in \emph{this Section}.

\end{noteblock}

\phantomsection\label{func.get_log3.value}
\subparagraph{Value}\label{value-13}

\hfill\break

A glue object, returned by the internally called \href{https://www.rdocumentation.org/packages/stringr/versions/1.5.0/topics/str_glue}{stringr::str\_glue} function, created from the character vector returned by the (also internally called) function \href{https://www.rdocumentation.org/packages/stringr/versions/1.5.0/topics/str_replace}{stringr::str\_replace\_all}.

\phantomsection\label{func.get_log3.code}
\subparagraph{Source Code}\label{source-code-13}

\hfill\break

The source code of the \href{https://github.com/AzKurban-edX-DS/Capstone-MovieLens/blob/main/r/src/support-functions/logging-functions.R\#L58}{get\_log3} function is shown below:

\begin{Shaded}
\begin{Highlighting}[]
\NormalTok{get\_log3 }\OtherTok{\textless{}{-}} \ControlFlowTok{function}\NormalTok{(msg\_template, arg1, arg2, arg3) \{}
\NormalTok{  msg\_template }\SpecialCharTok{|\textgreater{}} 
    \FunctionTok{str\_replace\_all}\NormalTok{(}\StringTok{"\%1"}\NormalTok{, }\FunctionTok{as.character}\NormalTok{(arg1)) }\SpecialCharTok{|\textgreater{}}
    \FunctionTok{str\_replace\_all}\NormalTok{(}\StringTok{"\%2"}\NormalTok{, }\FunctionTok{as.character}\NormalTok{(arg2)) }\SpecialCharTok{|\textgreater{}}
    \FunctionTok{str\_replace\_all}\NormalTok{(}\StringTok{"\%3"}\NormalTok{, }\FunctionTok{as.character}\NormalTok{(arg3)) }\SpecialCharTok{|\textgreater{}}
    \FunctionTok{str\_glue}\NormalTok{()}
\NormalTok{\}}
\end{Highlighting}
\end{Shaded}

\begin{noteblock}
The source code of the \href{https://github.com/AzKurban-edX-DS/Capstone-MovieLens/blob/main/r/src/support-functions/logging-functions.R\#L58}{get\_log3} function is defined in the \href{https://github.com/AzKurban-edX-DS/Capstone-MovieLens/blob/main/r/src/support-functions/logging-functions.R\#L58}{logging-functions.R} script on \emph{GitHub}.

\end{noteblock}

\newpage

\paragraph{\texorpdfstring{\href{https://github.com/AzKurban-edX-DS/Capstone-MovieLens/blob/main/r/src/support-functions/logging-functions.R\#L65}{print\_log3} Function}{print\_log3 Function}}\label{func.print_log3}

\hfill\break

Prints a log created from the template passed in the \hyperref[func.print_log3.args.msg_template]{msg\_template} argument, using the values specified by the \hyperref[func.print_log3.args.arg1]{arg1}, \hyperref[func.print_log3.args.arg2]{arg2}, and \hyperref[func.print_log3.args.arg3]{arg3} arguments to replace all the occurrences of the placeholders \emph{``\%1''}, \emph{``\%2''}, and \emph{``\%3''}, respectively, found in the template.

\phantomsection\label{func.print_log3.usage}
\subparagraph{\texorpdfstring{\href{https://github.com/AzKurban-edX-DS/Capstone-MovieLens/blob/main/r/src/capstone-movielens.main.R\#L2009}{Usage}}{Usage}}\label{usage-14}

\hfill\break

\begin{Shaded}
\begin{Highlighting}[]
\FunctionTok{print\_log3}\NormalTok{(}\StringTok{"RMSE value has been computed using \textasciigrave{}loess\textasciigrave{} function }
\StringTok{with the best parameters for the \%1{-}Fold Cross Validation samples:}
\StringTok{degree = \%2;}
\StringTok{span = \%3."}\NormalTok{,}
\NormalTok{        CVFolds\_N,}
\NormalTok{        lss.best\_degree,}
\NormalTok{        lss.best\_span)}
\end{Highlighting}
\end{Shaded}

\begin{verbatim}
## RMSE value has been computed using `loess` function 
## with the best parameters for the 5-Fold Cross Validation samples:
## degree = 1;
## span = 0.00109375.
\end{verbatim}

\phantomsection\label{func.print_log3.args}
\subparagraph{Arguments}\label{arguments-12}

\hfill\break

\phantomsection\label{func.print_log3.args.msg_template}
\begin{itemize}
\tightlist
\item
  \textbf{msg\_template:} Any object coercible to a character vector containing at least one message template element, i.e., a string (possibly multiline) containing at least one occurrence of each of the placeholders \emph{``\%1''}, \emph{``\%2''}, and \emph{``\%3''} to be replaced by the values specified by the \hyperref[func.print_log3.args.arg1]{arg1}, \hyperref[func.print_log3.args.arg2]{arg2}, and \hyperref[func.print_log3.args.arg3]{arg3} arguments, respectively.
\end{itemize}

\phantomsection\label{func.print_log3.args.arg1}
\begin{itemize}
\tightlist
\item
  \textbf{arg1:} Value to replace all occurrences of the placeholder \emph{``\%1''} in the message template passed in the \hyperref[func.print_log3.args.msg_template]{msg\_template} argument.
\end{itemize}

\phantomsection\label{func.print_log3.args.arg2}
\begin{itemize}
\tightlist
\item
  \textbf{arg2:} Value to replace all occurrences of the placeholder \emph{``\%2''} in the message template passed in the \hyperref[func.print_log3.args.msg_template]{msg\_template} argument.
\end{itemize}

\phantomsection\label{func.print_log3.args.arg3}
\begin{itemize}
\tightlist
\item
  \textbf{arg3:} Value to replace all occurrences of the placeholder \emph{``\%3''} in the message template passed in the \hyperref[func.print_log3.args.msg_template]{msg\_template} argument.
\end{itemize}

\phantomsection\label{func.print_log3.details}
\subparagraph{Details}\label{details-14}

\hfill\break

This function internally calls the \href{https://www.rdocumentation.org/packages/base/versions/3.6.2/topics/print}{base::print} \texttt{R} function to print the object returned by the \hyperref[func.get_log3]{get\_log3} function (also called internally) described above.

\phantomsection\label{func.print_log3.value}
\subparagraph{Value}\label{value-14}

\hfill\break

A glue object, returned by the internally called function \hyperref[func.get_log3]{get\_log3} described above.

\newpage

\phantomsection\label{func.print_log3.code}
\subparagraph{Source Code}\label{source-code-14}

\hfill\break

The source code of the \href{https://github.com/AzKurban-edX-DS/Capstone-MovieLens/blob/main/r/src/support-functions/logging-functions.R\#L65}{print\_log3} function is shown below:

\begin{Shaded}
\begin{Highlighting}[]
\NormalTok{print\_log3 }\OtherTok{\textless{}{-}} \ControlFlowTok{function}\NormalTok{(msg\_template, arg1, arg2, arg3)\{}
  \FunctionTok{print}\NormalTok{(}\FunctionTok{get\_log3}\NormalTok{(msg\_template, arg1, arg2, arg3))}
\NormalTok{\}}
\end{Highlighting}
\end{Shaded}

\begin{noteblock}
The source code of the \href{https://github.com/AzKurban-edX-DS/Capstone-MovieLens/blob/main/r/src/support-functions/logging-functions.R\#L65}{print\_log3} function is defined in the \href{https://github.com/AzKurban-edX-DS/Capstone-MovieLens/blob/main/r/src/support-functions/logging-functions.R\#L65}{logging-functions.R} script on \emph{GitHub}.

\end{noteblock}

\newpage

\paragraph{\texorpdfstring{\href{https://github.com/AzKurban-edX-DS/Capstone-MovieLens/blob/main/r/src/support-functions/logging-functions.R\#L68}{put\_log3} Function}{put\_log3 Function}}\label{func.put_log3}

\hfill\break

Outputs a log to the currently open log file (using the \hyperref[func.open_logfile]{open\_logfile} function described above), created from the template passed in the \hyperref[func.put_log3.args.msg_template]{msg\_template} argument, using the values specified by the \hyperref[func.put_log3.args.arg1]{arg1}, \hyperref[func.put_log3.args.arg2]{arg2}, and \hyperref[func.put_log3.args.arg3]{arg3} arguments to replace all the occurrences of the placeholders \emph{``\%1''}, \emph{``\%2''}, and \emph{``\%3''}, respectively, found in the template.

\phantomsection\label{func.put_log3.usage}
\subparagraph{\texorpdfstring{\href{https://github.com/AzKurban-edX-DS/Capstone-MovieLens/blob/main/r/src/capstone-movielens.main.R\#2009}{Usage}}{Usage}}\label{usage-15}

\hfill\break

\begin{Shaded}
\begin{Highlighting}[]
\FunctionTok{put\_log3}\NormalTok{(}\StringTok{"RMSE value has been computed using \textasciigrave{}loess\textasciigrave{} function }
\StringTok{with the best parameters for the \%1{-}Fold Cross Validation samples:}
\StringTok{degree = \%2;}
\StringTok{span = \%3."}\NormalTok{,}
\NormalTok{        CVFolds\_N,}
\NormalTok{        lss.best\_degree,}
\NormalTok{        lss.best\_span)}
\end{Highlighting}
\end{Shaded}

\begin{verbatim}
## RMSE value has been computed using `loess` function 
## with the best parameters for the 5-Fold Cross Validation samples:
## degree = 1;
## span = 0.00109375.
\end{verbatim}

\phantomsection\label{func.put_log3.args}
\subparagraph{Arguments}\label{arguments-13}

\hfill\break

\phantomsection\label{func.put_log3.args.msg_template}
\begin{itemize}
\tightlist
\item
  \textbf{msg\_template:} Any object coercible to a character vector containing at least one message template element, i.e., a string (possibly multiline) containing at least one occurrence of each of the placeholders \emph{``\%1''}, \emph{``\%2''}, and \emph{``\%3''} to be replaced by the values specified by the \hyperref[func.put_log3.args.arg1]{arg1}, \hyperref[func.put_log3.args.arg2]{arg2}, and \hyperref[func.put_log3.args.arg3]{arg3} arguments, respectively.
\end{itemize}

\phantomsection\label{func.put_log3.args.arg1}
\begin{itemize}
\tightlist
\item
  \textbf{arg1:} Value to replace all occurrences of the placeholder \emph{``\%1''} in the message template passed in the \hyperref[func.put_log3.args.msg_template]{msg\_template} argument.
\end{itemize}

\phantomsection\label{func.put_log3.args.arg2}
\begin{itemize}
\tightlist
\item
  \textbf{arg2:} Value to replace all occurrences of the placeholder \emph{``\%2''} in the message template passed in the \hyperref[func.put_log3.args.msg_template]{msg\_template} argument.
\end{itemize}

\phantomsection\label{func.put_log3.args.arg3}
\begin{itemize}
\tightlist
\item
  \textbf{arg3:} Value to replace all occurrences of the placeholder \emph{``\%3''} in the message template passed in the \hyperref[func.put_log3.args.msg_template]{msg\_template} argument.
\end{itemize}

\phantomsection\label{func.put_log3.details}
\subparagraph{Details}\label{details-15}

\hfill\break

This function internally calls the \href{https://www.rdocumentation.org/packages/logr/versions/1.3.4/topics/log_print}{logr::put} \texttt{R} function to output the log object returned by the (also called internally) \hyperref[func.get_log3]{get\_log3} function described above.

The function works similarly to the \hyperref[func.print_log3]{print\_log3} function described above, but unlike the latter, it outputs the information to the currently open log file.

\phantomsection\label{func.put_log3.value}
\subparagraph{Value}\label{value-15}

\hfill\break

A glue object, returned by the internally called function \hyperref[func.get_log3]{get\_log3} described above.

\phantomsection\label{func.put_log3.code}
\subparagraph{Source Code}\label{source-code-15}

\hfill\break

The source code of the \href{https://github.com/AzKurban-edX-DS/Capstone-MovieLens/blob/main/r/src/support-functions/logging-functions.R\#L68}{put\_log3} function is shown below:

\begin{Shaded}
\begin{Highlighting}[]
\NormalTok{put\_log3 }\OtherTok{\textless{}{-}} \ControlFlowTok{function}\NormalTok{(msg\_template, arg1, arg2, arg3)\{}
  \FunctionTok{put}\NormalTok{(}\FunctionTok{get\_log3}\NormalTok{(msg\_template, arg1, arg2, arg3))}
\NormalTok{\}}
\end{Highlighting}
\end{Shaded}

\begin{noteblock}
The source code of the \href{https://github.com/AzKurban-edX-DS/Capstone-MovieLens/blob/main/r/src/support-functions/logging-functions.R\#L68}{put\_log3} function is defined in the \href{https://github.com/AzKurban-edX-DS/Capstone-MovieLens/blob/main/r/src/support-functions/logging-functions.R\#L68}{logging-functions.R} script on \emph{GitHub}.

\end{noteblock}

\newpage

\paragraph{\texorpdfstring{\href{https://github.com/AzKurban-edX-DS/Capstone-MovieLens/blob/main/r/src/support-functions/logging-functions.R\#L72}{get\_log4} Function}{get\_log4 Function}}\label{func.get_log4}

\hfill\break

Creates a character vector from the template passed in the \hyperref[func.get_log4.args.msg_template]{msg\_template} argument, using the values specified by the \hyperref[func.get_log4.args.arg1]{arg1}, \hyperref[func.get_log4.args.arg2]{arg2}, \hyperref[func.get_log4.args.arg3]{arg3}, and \hyperref[func.get_log4.args.arg4]{arg4} arguments to replace all the occurrences of the placeholders \emph{``\%1''}, \emph{``\%2''}, \emph{``\%3''}, and \emph{``\%4''}, respectively, found in the template.

\phantomsection\label{func.get_log4.usage}
\subparagraph{Usage}\label{usage-16}

\hfill\break

\begin{Shaded}
\begin{Highlighting}[]
  \FunctionTok{put}\NormalTok{(}\FunctionTok{get\_log4}\NormalTok{(msg\_template, arg1, arg2, arg3, arg4))}
\end{Highlighting}
\end{Shaded}

\begin{noteblock}
In the \href{https://github.com/AzKurban-edX-DS/Capstone-MovieLens/blob/main/r/src/support-functions/logging-functions.R\#L84}{code snippet} above, the function \texttt{get\_log4} is called internally from the \hyperref[func.put_log4]{put\_log4} function described below in \emph{this Section}.

\end{noteblock}

\phantomsection\label{func.get_log4.args}
\subparagraph{Arguments}\label{arguments-14}

\hfill\break

\phantomsection\label{func.get_log4.args.msg_template}
\begin{itemize}
\tightlist
\item
  \textbf{msg\_template:} Any object coercible to a character vector containing at least one message template element, i.e., a string (possibly multiline) containing at least one occurrence of each of the placeholders \emph{``\%1''}, \emph{``\%2''}, \emph{``\%3''}, and \emph{``\%4''} to be replaced by the values specified by the \hyperref[func.get_log4.args.arg1]{arg1}, \hyperref[func.get_log4.args.arg2]{arg2}, \hyperref[func.get_log4.args.arg3]{arg3}, and \hyperref[func.get_log4.args.arg4]{arg4} arguments, respectively.
\end{itemize}

\phantomsection\label{func.get_log4.args.arg1}
\begin{itemize}
\tightlist
\item
  \textbf{arg1:} Value to replace all occurrences of the placeholder \emph{``\%1''} in the message template passed in the \hyperref[func.get_log4.args.msg_template]{msg\_template} argument.
\end{itemize}

\phantomsection\label{func.get_log4.args.arg2}
\begin{itemize}
\tightlist
\item
  \textbf{arg2:} Value to replace all occurrences of the placeholder \emph{``\%2''} in the message template passed in the \hyperref[func.get_log4.args.msg_template]{msg\_template} argument.
\end{itemize}

\phantomsection\label{func.get_log4.args.arg3}
\begin{itemize}
\tightlist
\item
  \textbf{arg3:} Value to replace all occurrences of the placeholder \emph{``\%3''} in the message template passed in the \hyperref[func.get_log4.args.msg_template]{msg\_template} argument.
\end{itemize}

\phantomsection\label{func.get_log4.args.arg4}
\begin{itemize}
\tightlist
\item
  \textbf{arg4:} Value to replace all occurrences of the placeholder \emph{``\%4''} in the message template passed in the \hyperref[func.get_log4.args.msg_template]{msg\_template} argument.
\end{itemize}

\phantomsection\label{func.get_log4.details}
\subparagraph{Details}\label{details-16}

\hfill\break

This function internally calls \href{https://www.rdocumentation.org/packages/stringr/versions/1.5.0/topics/str_glue}{stringr::str\_glue} and \href{https://www.rdocumentation.org/packages/stringr/versions/1.5.0/topics/str_replace}{stringr::str\_replace\_all} \texttt{R} functions to form a character vector from the template passed in the \hyperref[func.get_log4.args.msg_template]{msg\_template} argument, using the values specified by the \hyperref[func.get_log4.args.arg1]{arg1}, \hyperref[func.get_log4.args.arg2]{arg2}, \hyperref[func.get_log4.args.arg3]{arg3}, and \hyperref[func.get_log4.args.arg4]{arg4} arguments to replace all the occurrences of the placeholders \emph{``\%1''}, \emph{``\%2''}, \emph{``\%3''}, and \emph{``\%4''}, respectively, found in the template.

\begin{noteblock}
This is an auxiliary function intended for use in other custom logging functions, such as the \hyperref[func.put_log4]{put\_log4} (as shown in the \hyperref[func.get_log4.usage]{Usage} section of \emph{this Description}) and \hyperref[func.print_log4]{print\_log4} described below in \emph{this Section}.

\end{noteblock}

\phantomsection\label{func.get_log4.value}
\subparagraph{Value}\label{value-16}

\hfill\break

A glue object, returned by the internally called \href{https://www.rdocumentation.org/packages/stringr/versions/1.5.0/topics/str_glue}{stringr::str\_glue} function, created from the character vector returned by the (also internally called) function \href{https://www.rdocumentation.org/packages/stringr/versions/1.5.0/topics/str_replace}{stringr::str\_replace\_all}.

\phantomsection\label{func.get_log4.code}
\subparagraph{Source Code}\label{source-code-16}

\hfill\break

The source code of the \href{https://github.com/AzKurban-edX-DS/Capstone-MovieLens/blob/main/r/src/support-functions/logging-functions.R\#L72}{get\_log4} function is shown below:

\begin{Shaded}
\begin{Highlighting}[]
\NormalTok{get\_log4 }\OtherTok{\textless{}{-}} \ControlFlowTok{function}\NormalTok{(msg\_template, arg1, arg2, arg3, arg4) \{}
\NormalTok{  msg\_template }\SpecialCharTok{|\textgreater{}} 
    \FunctionTok{str\_replace\_all}\NormalTok{(}\StringTok{"\%1"}\NormalTok{, }\FunctionTok{as.character}\NormalTok{(arg1)) }\SpecialCharTok{|\textgreater{}}
    \FunctionTok{str\_replace\_all}\NormalTok{(}\StringTok{"\%2"}\NormalTok{, }\FunctionTok{as.character}\NormalTok{(arg2)) }\SpecialCharTok{|\textgreater{}}
    \FunctionTok{str\_replace\_all}\NormalTok{(}\StringTok{"\%3"}\NormalTok{, }\FunctionTok{as.character}\NormalTok{(arg3)) }\SpecialCharTok{|\textgreater{}}
    \FunctionTok{str\_replace\_all}\NormalTok{(}\StringTok{"\%4"}\NormalTok{, }\FunctionTok{as.character}\NormalTok{(arg4)) }\SpecialCharTok{|\textgreater{}}
    \FunctionTok{str\_glue}\NormalTok{()}
\NormalTok{\}}
\end{Highlighting}
\end{Shaded}

\begin{noteblock}
The source code of the \href{https://github.com/AzKurban-edX-DS/Capstone-MovieLens/blob/main/r/src/support-functions/logging-functions.R\#L72}{get\_log4} function is defined in the \href{https://github.com/AzKurban-edX-DS/Capstone-MovieLens/blob/main/r/src/support-functions/logging-functions.R\#L72}{logging-functions.R} script on \emph{GitHub}.

\end{noteblock}

\newpage

\paragraph{\texorpdfstring{\href{https://github.com/AzKurban-edX-DS/Capstone-MovieLens/blob/main/r/src/support-functions/logging-functions.R\#L80}{print\_log4} Function}{print\_log4 Function}}\label{func.print_log4}

\hfill\break

Prints a log created from the template passed in the \hyperref[func.print_log4.args.msg_template]{msg\_template} argument, using the values specified by the \hyperref[func.print_log4.args.arg1]{arg1}, \hyperref[func.print_log4.args.arg2]{arg2}, \hyperref[func.print_log4.args.arg3]{arg3}, and \hyperref[func.print_log4.args.arg4]{arg4} arguments to replace all the occurrences of the placeholders \emph{``\%1''}, \emph{``\%2''}, \emph{``\%3''}, and \emph{``\%4''}, respectively, found in the template.

\phantomsection\label{func.print_log4.usage}
\subparagraph{\texorpdfstring{\href{https://www.pragmaticinstitute.com/resources/articles/data/4-important-aspects-of-data-science/}{Usage}\autocite{IADS_4}}{Usage{[}@IADS\_4{]}}}\label{usageiads_4}

\hfill\break

\begin{Shaded}
\begin{Highlighting}[]
\NormalTok{arg1 }\OtherTok{\textless{}{-}} \StringTok{"Data Collection"}
\NormalTok{arg2 }\OtherTok{\textless{}{-}} \StringTok{"Data Cleaning and Transformation"}
\NormalTok{arg3 }\OtherTok{\textless{}{-}} \StringTok{"Statistical Analysis"}
\NormalTok{arg4 }\OtherTok{\textless{}{-}} \StringTok{"Data Visualization"}

\NormalTok{msg\_template }\OtherTok{\textless{}{-}} \StringTok{"The 4 Important Aspects of Data Science are as follows:}
\StringTok{  1. \%1;}
\StringTok{  2. \%2;}
\StringTok{  3. \%3;}
\StringTok{  4. \%4."}

\FunctionTok{print\_log4}\NormalTok{(msg\_template, arg1, arg2, arg3, arg4)}
\end{Highlighting}
\end{Shaded}

\begin{verbatim}
## The 4 Important Aspects of Data Science are as follows:
## 1. Data Collection;
## 2. Data Cleaning and Transformation;
## 3. Statistical Analysis;
## 4. Data Visualization.
\end{verbatim}

\phantomsection\label{func.print_log4.args}
\subparagraph{Arguments}\label{arguments-15}

\hfill\break

\phantomsection\label{func.print_log4.args.msg_template}
\begin{itemize}
\tightlist
\item
  \textbf{msg\_template:} Any object coercible to a character vector containing at least one message template element, i.e., a string (possibly multiline) containing at least one occurrence of each of the placeholders \emph{``\%1''}, \emph{``\%2''}, \emph{``\%3''}, and \emph{``\%4''} to be replaced by the values specified by the \hyperref[func.print_log4.args.arg1]{arg1}, \hyperref[func.print_log4.args.arg2]{arg2}, \hyperref[func.print_log4.args.arg3]{arg3}, and \hyperref[func.print_log4.args.arg4]{arg4} arguments, respectively.
\end{itemize}

\phantomsection\label{func.print_log4.args.arg1}
\begin{itemize}
\tightlist
\item
  \textbf{arg1:} Value to replace all occurrences of the placeholder \emph{``\%1''} in the message template passed in the \hyperref[func.print_log4.args.msg_template]{msg\_template} argument.
\end{itemize}

\phantomsection\label{func.print_log4.args.arg2}
\begin{itemize}
\tightlist
\item
  \textbf{arg2:} Value to replace all occurrences of the placeholder \emph{``\%2''} in the message template passed in the \hyperref[func.print_log4.args.msg_template]{msg\_template} argument.
\end{itemize}

\phantomsection\label{func.print_log4.args.arg3}
\begin{itemize}
\tightlist
\item
  \textbf{arg3:} Value to replace all occurrences of the placeholder \emph{``\%3''} in the message template passed in the \hyperref[func.print_log4.args.msg_template]{msg\_template} argument.
\end{itemize}

\phantomsection\label{func.print_log4.args.arg4}
\begin{itemize}
\tightlist
\item
  \textbf{arg4:} Value to replace all occurrences of the placeholder \emph{``\%4''} in the message template passed in the \hyperref[func.print_log4.args.msg_template]{msg\_template} argument.
\end{itemize}

\phantomsection\label{func.print_log4.details}
\subparagraph{Details}\label{details-17}

\hfill\break

This function internally calls the \href{https://www.rdocumentation.org/packages/base/versions/3.6.2/topics/print}{base::print} \texttt{R} function to print the object returned by the \hyperref[func.get_log4]{get\_log4} function (also called internally) described above.

\phantomsection\label{func.print_log4.value}
\subparagraph{Value}\label{value-17}

\hfill\break

A glue object, returned by the internally called function \hyperref[func.get_log4]{get\_log4} described above.

\phantomsection\label{func.print_log4.code}
\subparagraph{Source Code}\label{source-code-17}

\hfill\break

The source code of the \href{https://github.com/AzKurban-edX-DS/Capstone-MovieLens/blob/main/r/src/support-functions/logging-functions.R\#L80}{print\_log4} function is shown below:

\begin{Shaded}
\begin{Highlighting}[]
\NormalTok{print\_log4 }\OtherTok{\textless{}{-}} \ControlFlowTok{function}\NormalTok{(msg\_template, arg1, arg2, arg3, arg4)\{}
  \FunctionTok{print}\NormalTok{(}\FunctionTok{get\_log4}\NormalTok{(msg\_template, arg1, arg2, arg3, arg4))}
\NormalTok{\}}
\end{Highlighting}
\end{Shaded}

\begin{noteblock}
The source code of the \href{https://github.com/AzKurban-edX-DS/Capstone-MovieLens/blob/main/r/src/support-functions/logging-functions.R\#L80}{print\_log4} function is defined in the \href{https://github.com/AzKurban-edX-DS/Capstone-MovieLens/blob/main/r/src/support-functions/logging-functions.R\#L80}{logging-functions.R} script on \emph{GitHub}.

\end{noteblock}

\newpage

\paragraph{\texorpdfstring{\href{https://github.com/AzKurban-edX-DS/Capstone-MovieLens/blob/main/r/src/support-functions/logging-functions.R\#L83}{put\_log4} Function}{put\_log4 Function}}\label{func.put_log4}

\hfill\break

Outputs a log to the currently open log file (using the \hyperref[func.open_logfile]{open\_logfile} function described above), created from the template passed in the \hyperref[func.put_log4.args.msg_template]{msg\_template} argument, using the values specified by the \hyperref[func.put_log4.args.arg1]{arg1}, \hyperref[func.put_log4.args.arg2]{arg2}, \hyperref[func.put_log4.args.arg3]{arg3}, and \hyperref[func.put_log4.args.arg4]{arg4} arguments to replace all the occurrences of the placeholders \emph{``\%1''}, \emph{``\%2''}, \emph{``\%3''}, and \emph{``\%4''}, respectively, found in the template.

\phantomsection\label{func.put_log4.usage}
\subparagraph{\texorpdfstring{\href{https://www.pragmaticinstitute.com/resources/articles/data/4-important-aspects-of-data-science/}{Usage}\autocite{IADS_4}}{Usage{[}@IADS\_4{]}}}\label{usageiads_4-1}

\hfill\break

\begin{Shaded}
\begin{Highlighting}[]
\NormalTok{arg1 }\OtherTok{\textless{}{-}} \StringTok{"Data Collection"}
\NormalTok{arg2 }\OtherTok{\textless{}{-}} \StringTok{"Data Cleaning and Transformation"}
\NormalTok{arg3 }\OtherTok{\textless{}{-}} \StringTok{"Statistical Analysis"}
\NormalTok{arg4 }\OtherTok{\textless{}{-}} \StringTok{"Data Visualization"}

\NormalTok{msg\_template }\OtherTok{\textless{}{-}} \StringTok{"The 4 Important Aspects of Data Science are as follows:}
\StringTok{  1. \%1;}
\StringTok{  2. \%2;}
\StringTok{  3. \%3;}
\StringTok{  4. \%4."}
\end{Highlighting}
\end{Shaded}

\begin{Shaded}
\begin{Highlighting}[]
\FunctionTok{put\_log4}\NormalTok{(msg\_template, arg1, arg2, arg3, arg4)}
\end{Highlighting}
\end{Shaded}

\begin{verbatim}
## The 4 Important Aspects of Data Science are as follows:
## 1. Data Collection;
## 2. Data Cleaning and Transformation;
## 3. Statistical Analysis;
## 4. Data Visualization.
\end{verbatim}

\phantomsection\label{func.put_log4.args}
\subparagraph{Arguments}\label{arguments-16}

\hfill\break

\phantomsection\label{func.put_log4.args.msg_template}
\begin{itemize}
\tightlist
\item
  \textbf{msg\_template:} Any object coercible to a character vector containing at least one message template element, i.e., a string (possibly multiline) containing at least one occurrence of each of the placeholders \emph{``\%1''}, \emph{``\%2''}, \emph{``\%3''}, and \emph{``\%4''} to be replaced by the values specified by the \hyperref[func.put_log4.args.arg1]{arg1}, \hyperref[func.put_log4.args.arg2]{arg2}, \hyperref[func.put_log4.args.arg3]{arg3}, and \hyperref[func.put_log4.args.arg4]{arg4} arguments, respectively.
\end{itemize}

\phantomsection\label{func.put_log4.args.arg1}
\begin{itemize}
\tightlist
\item
  \textbf{arg1:} Value to replace all occurrences of the placeholder \emph{``\%1''} in the message template passed in the \hyperref[func.put_log4.args.msg_template]{msg\_template} argument.
\end{itemize}

\phantomsection\label{func.put_log4.args.arg2}
\begin{itemize}
\tightlist
\item
  \textbf{arg2:} Value to replace all occurrences of the placeholder \emph{``\%2''} in the message template passed in the \hyperref[func.put_log4.args.msg_template]{msg\_template} argument.
\end{itemize}

\phantomsection\label{func.put_log4.args.arg3}
\begin{itemize}
\tightlist
\item
  \textbf{arg3:} Value to replace all occurrences of the placeholder \emph{``\%3''} in the message template passed in the \hyperref[func.put_log4.args.msg_template]{msg\_template} argument.
\end{itemize}

\phantomsection\label{func.put_log4.args.arg4}
\begin{itemize}
\tightlist
\item
  \textbf{arg4:} Value to replace all occurrences of the placeholder \emph{``\%4''} in the message template passed in the \hyperref[func.put_log4.args.msg_template]{msg\_template} argument.
\end{itemize}

\phantomsection\label{func.put_log4.details}
\subparagraph{Details}\label{details-18}

\hfill\break

This function internally calls the \href{https://www.rdocumentation.org/packages/logr/versions/1.3.4/topics/log_print}{logr::put} \texttt{R} function to output the log object returned by the (also called internally) \hyperref[func.get_log4]{get\_log4} function described above.

The function works similarly to the \hyperref[func.print_log4]{print\_log4} function described above, but unlike the latter, it outputs the information to the currently open log file.

\phantomsection\label{func.put_log4.value}
\subparagraph{Value}\label{value-18}

\hfill\break

A glue object, returned by the internally called function \hyperref[func.get_log4]{get\_log4} described above.

\phantomsection\label{func.put_log4.code}
\subparagraph{Source Code}\label{source-code-18}

\hfill\break

The source code of the \href{https://github.com/AzKurban-edX-DS/Capstone-MovieLens/blob/main/r/src/support-functions/logging-functions.R\#L83}{put\_log4} function is shown below:

\begin{Shaded}
\begin{Highlighting}[]
\NormalTok{put\_log4 }\OtherTok{\textless{}{-}} \ControlFlowTok{function}\NormalTok{(msg\_template, arg1, arg2, arg3, arg4)\{}
  \FunctionTok{put}\NormalTok{(}\FunctionTok{get\_log4}\NormalTok{(msg\_template, arg1, arg2, arg3, arg4))}
\NormalTok{\}}
\end{Highlighting}
\end{Shaded}

\begin{noteblock}
The source code of the \href{https://github.com/AzKurban-edX-DS/Capstone-MovieLens/blob/main/r/src/support-functions/logging-functions.R\#L83}{put\_log4} function is defined in the \href{https://github.com/AzKurban-edX-DS/Capstone-MovieLens/blob/main/r/src/support-functions/logging-functions.R\#L83}{logging-functions.R} script on \emph{GitHub}.

\end{noteblock}

\newpage

\subsubsection{Utility Functions}\label{appndx_a.CHF.utility_funcs}

\begin{noteblock}
The complete source code of the functions described in this section are available in the \href{https://github.com/AzKurban-edX-DS/Capstone-MovieLens/blob/main/r/src/support-functions/common-helper.functions.R\#L445}{Utility Functions} section of the \href{https://github.com/AzKurban-edX-DS/Capstone-MovieLens/blob/main/r/src/support-functions/common-helper.functions.R\#L445}{common-helper.functions.R} script on \emph{GitHub}.

\end{noteblock}

\paragraph{\texorpdfstring{\href{https://github.com/AzKurban-edX-DS/Capstone-MovieLens/blob/main/r/src/support-functions/common-helper.functions.R\#L448}{clamp} Function}{clamp Function}}\label{func.clamp}

\hfill\break

Clamps the input value passed in \hyperref[func.clamp.args.r]{r} argument between the values passed in \hyperref[func.clamp.args.min]{min} and \hyperref[func.clamp.args.max]{max} arguments.

\phantomsection\label{func.clamp.usage}
\subparagraph{\texorpdfstring{\href{https://github.com/AzKurban-edX-DS/Capstone-MovieLens/blob/main/r/src/capstone-movielens.main.R\#L347}{Usage}}{Usage}}\label{usage-17}

\hfill\break

\begin{Shaded}
\begin{Highlighting}[]
\NormalTok{edx.user\_effect.MSEs }\OtherTok{\textless{}{-}} \FunctionTok{sapply}\NormalTok{(edx\_CV, }\ControlFlowTok{function}\NormalTok{(cv\_fold\_dat)\{}
\NormalTok{  cv\_fold\_dat}\SpecialCharTok{$}\NormalTok{validation\_set }\SpecialCharTok{|\textgreater{}}
    \FunctionTok{left\_join}\NormalTok{(edx.user\_effect, }\AttributeTok{by =} \StringTok{"userId"}\NormalTok{) }\SpecialCharTok{|\textgreater{}}
    \FunctionTok{mutate}\NormalTok{(}\AttributeTok{resid =}\NormalTok{ rating }\SpecialCharTok{{-}} \FunctionTok{clamp}\NormalTok{(mu }\SpecialCharTok{+}\NormalTok{ a)) }\SpecialCharTok{|\textgreater{}} 
    \FunctionTok{pull}\NormalTok{(resid) }\SpecialCharTok{|\textgreater{}} \FunctionTok{mse}\NormalTok{()}
\NormalTok{\})}
\end{Highlighting}
\end{Shaded}

\begin{Shaded}
\begin{Highlighting}[]
\FunctionTok{data.frame}\NormalTok{(}\AttributeTok{fold\_No =} \DecValTok{1}\SpecialCharTok{:}\DecValTok{5}\NormalTok{, }\AttributeTok{MSE =}\NormalTok{ edx.user\_effect.MSEs) }\SpecialCharTok{|\textgreater{}}
  \FunctionTok{data.plot}\NormalTok{(}\AttributeTok{title =} \StringTok{"MSE resuls of the 5{-}fold CV method applied to the User Effect Model"}\NormalTok{,}
            \AttributeTok{xname =} \StringTok{"fold\_No"}\NormalTok{, }
            \AttributeTok{yname =} \StringTok{"MSE"}\NormalTok{)}
\end{Highlighting}
\end{Shaded}

\pandocbounded{\includegraphics[keepaspectratio]{capstone-movielens.report_files/figure-latex/main-script@l355u_clamp-1.pdf}}

\begin{noteblock}
In the \href{https://github.com/AzKurban-edX-DS/Capstone-MovieLens/blob/main/r/src/capstone-movielens.main.R\#L355}{code snippet} above, we use the custom data visualization function \hyperref[func.data.plot]{data.plot} described in section \hyperref[data-visualization-functions]{Data Visualization Functions} of \emph{this Appendix}.

\end{noteblock}

\phantomsection\label{func.clamp.args}
\subparagraph{Arguments}\label{arguments-17}

\hfill\break

\phantomsection\label{func.clamp.args.r}
\begin{itemize}
\tightlist
\item
  \textbf{r:} Value to clamp;
\end{itemize}

\phantomsection\label{func.clamp.args.min}
\begin{itemize}
\tightlist
\item
  \textbf{min:} (Optional, 0.5 by default) Minimum allowed value;
\end{itemize}

\phantomsection\label{func.clamp.args.max}
\begin{itemize}
\tightlist
\item
  \textbf{max:} (Optional, 5 by default) Maximum allowed value;
\end{itemize}

:::

\phantomsection\label{func.clamp.details}
\subparagraph{Details}\label{details-19}

\hfill\break
As explained in \href{https://rafalab.dfci.harvard.edu/dsbook-part-2/highdim/regularization.html\#user-effects}{Section 24.1 \emph{Case study: recommendation systems / User effects}} of the \emph{Course Textbook} we know ratings cannot be below 0.5 or above 5. For this reason, we need a function described in that section\autocite{IDS2_24-1}:

This function is a wrapper for the \href{https://www.rdocumentation.org/packages/base/versions/3.6.2/topics/Extremes}{base::pmin} and \href{https://www.rdocumentation.org/packages/base/versions/3.6.2/topics/Extremes}{base::pmax} \texttt{R} functions, which are called internally to constrain the input value passed in the \hyperref[func.clamp.args.r]{r} argument to be within the values passed in the \hyperref[func.clamp.args.min]{min} and \hyperref[func.clamp.args.max]{max} arguments, respectively.

in \emph{this Project} this function is used to clamp the computed movie rating values between 0.5 and 5.

\phantomsection\label{func.clamp.value}
\subparagraph{Value}\label{value-19}

\hfill\break

The value clamped between \hyperref[func.clamp.args.min]{min} and \hyperref[func.clamp.args.max]{max} argument values.

\phantomsection\label{func.clamp.code}
\subparagraph{Source Code}\label{source-code-19}

\hfill\break

The source code of the \href{https://github.com/AzKurban-edX-DS/Capstone-MovieLens/blob/main/r/src/support-functions/common-helper.functions.R\#L448}{clamp} function is shown below:

\begin{Shaded}
\begin{Highlighting}[]
\NormalTok{clamp }\OtherTok{\textless{}{-}} \ControlFlowTok{function}\NormalTok{(x, }\AttributeTok{min =} \FloatTok{0.5}\NormalTok{, }\AttributeTok{max =} \DecValTok{5}\NormalTok{) }\FunctionTok{pmax}\NormalTok{(}\FunctionTok{pmin}\NormalTok{(x, max), min)}
\end{Highlighting}
\end{Shaded}

\begin{noteblock}
The source code of the \href{https://github.com/AzKurban-edX-DS/Capstone-MovieLens/blob/main/r/src/support-functions/common-helper.functions.R\#L448}{clamp} function is also available in the \href{https://github.com/AzKurban-edX-DS/Capstone-MovieLens/blob/main/r/src/support-functions/common-helper.functions.R\#L445}{Utility Functions} section of the \href{https://github.com/AzKurban-edX-DS/Capstone-MovieLens/blob/main/r/src/support-functions/common-helper.functions.R\#L448}{common-helper.functions.R} script on \emph{GitHub}.

\end{noteblock}

\newpage

\paragraph{\texorpdfstring{\href{https://github.com/AzKurban-edX-DS/Capstone-MovieLens/blob/main/r/src/support-functions/common-helper.functions.R\#L453}{msg.set\_arg} Function}{msg.set\_arg Function}}\label{func.msg.set_arg}

\hfill\break

Creates a character vector from the template passed in the \hyperref[func.msg.set_arg.args.msg_template]{msg\_template} argument, using the value specified by the \hyperref[func.msg.set_arg.args.arg]{arg} argument to replace all the occurrences of the placeholder passed in the \hyperref[func.msg.set_arg.args.arg.name]{arg.name} argument found in the template.

\phantomsection\label{func.msg.set_arg.usage}
\subparagraph{Usage}\label{usage-18}

\hfill\break

\begin{Shaded}
\begin{Highlighting}[]
\NormalTok{  msg\_template }\SpecialCharTok{|\textgreater{}}
    \FunctionTok{msg.set\_arg}\NormalTok{(arg, arg.name) }\SpecialCharTok{|\textgreater{}}
    \FunctionTok{str\_glue}\NormalTok{()}
\end{Highlighting}
\end{Shaded}

\begin{noteblock}
In the above \href{https://github.com/AzKurban-edX-DS/Capstone-MovieLens/blob/main/r/src/support-functions/common-helper.functions.R\#L458}{code snippet}, the \texttt{msg.set\_arg} function is called internally from the \hyperref[func.msg.glue]{msg.glue} function described below in \emph{this Section}.

\end{noteblock}

\phantomsection\label{func.msg.set_arg.args}
\subparagraph{Arguments}\label{arguments-18}

\hfill\break

\phantomsection\label{func.msg.set_arg.args.msg_template}
\begin{itemize}
\tightlist
\item
  \textbf{msg\_template:} Any object coercible to a character vector containing at least one message template element, i.e., a string (possibly multiline) containing at least one occurrence of a pattern specified by the \hyperref[func.msg.set_arg.args.arg.name]{arg.name} argument.
\end{itemize}

\phantomsection\label{func.msg.set_arg.args.arg.name}
\begin{itemize}
\tightlist
\item
  \textbf{arg.name:} (Optional, \emph{``\%1''} by default) Substring (pattern or \emph{placeholder}) to replace all occurrences found in the message template passed in the \hyperref[func.msg.set_arg.args.msg_template]{msg\_template} argument with the value specified in the \hyperref[func.msg.set_arg.args.arg]{arg} argument.
\end{itemize}

\phantomsection\label{func.msg.set_arg.args.arg}
\begin{itemize}
\tightlist
\item
  \textbf{arg:} Value to replace all occurrences of the pattern specified by the \hyperref[func.msg.set_arg.args.arg.name]{arg.name} argument in the message template passed in the \hyperref[func.msg.set_arg.args.msg_template]{msg\_template} argument.
\end{itemize}

\phantomsection\label{func.msg.set_arg.details}
\subparagraph{Details}\label{details-20}

\hfill\break

This function is a wrapper for the \href{https://www.rdocumentation.org/packages/stringr/versions/1.5.0/topics/str_replace}{stringr::str\_replace\_all} \texttt{R} function, which is called internally to replace all occurrences of the pattern (placeholder) specified by the \hyperref[func.msg.set_arg.args.arg.name]{arg\_name} argument (\emph{``\%1''} by default), in the message template passed in the \hyperref[func.msg.set_arg.args.msg_template]{msg\_template} argument with the value specified in the \hyperref[func.msg.set_arg.args.arg]{arg} argument.

\begin{noteblock}
This is an auxiliary function intended to be called internally from the \hyperref[func.msg.glue]{msg.glue} function (described below in \emph{this Section}) as shown above in the \hyperref[func.msg.set_arg.usage]{Usage} section of \emph{this Description}.

\end{noteblock}

\phantomsection\label{func.msg.set_arg.value}
\subparagraph{Value}\label{value-20}

\hfill\break

The character vector returned by the internally called \href{https://www.rdocumentation.org/packages/stringr/versions/1.5.0/topics/str_replace}{stringr::str\_replace\_all()}.

\phantomsection\label{func.msg.set_arg.code}
\subparagraph{Source Code}\label{source-code-20}

\hfill\break

The source code of the \href{https://github.com/AzKurban-edX-DS/Capstone-MovieLens/blob/main/r/src/support-functions/common-helper.functions.R\#L453}{msg.set\_arg} function is shown below:

\begin{Shaded}
\begin{Highlighting}[]
\NormalTok{msg.set\_arg }\OtherTok{\textless{}{-}} \ControlFlowTok{function}\NormalTok{(msg\_template, arg, }\AttributeTok{arg.name =} \StringTok{"\%1"}\NormalTok{) \{}
\NormalTok{  msg\_template }\SpecialCharTok{|\textgreater{}} 
    \FunctionTok{str\_replace\_all}\NormalTok{(arg.name, }\FunctionTok{as.character}\NormalTok{(arg))}
\NormalTok{\}}
\end{Highlighting}
\end{Shaded}

\begin{noteblock}
The source code of the \href{https://github.com/AzKurban-edX-DS/Capstone-MovieLens/blob/main/r/src/support-functions/common-helper.functions.R\#L453}{msg.set\_arg} function is also available in the \href{https://github.com/AzKurban-edX-DS/Capstone-MovieLens/blob/main/r/src/support-functions/common-helper.functions.R\#L449}{Utility Functions} section of the \href{https://github.com/AzKurban-edX-DS/Capstone-MovieLens/blob/main/r/src/support-functions/common-helper.functions.R\#L453}{common-helper.functions.R} script on \emph{GitHub}.

\end{noteblock}

\newpage

\paragraph{\texorpdfstring{\href{https://github.com/AzKurban-edX-DS/Capstone-MovieLens/blob/main/r/src/support-functions/common-helper.functions.R\#L457}{msg.glue} Function}{msg.glue Function}}\label{func.msg.glue}

\hfill\break

Creates a \href{https://glue.tidyverse.org/reference/glue.html}{glue::glue} object from the template passed in the \hyperref[func.msg.glue.args.msg_template]{msg\_template} argument, using the value specified by the \hyperref[func.msg.glue.args.arg]{arg} argument to replace all the occurrences of the placeholder passed in the \hyperref[func.msg.glue.args.arg.name]{arg.name} argument found in the template.

\phantomsection\label{func.msg.glue.usage}
\subparagraph{\texorpdfstring{\href{https://github.com/AzKurban-edX-DS/Capstone-MovieLens/blob/main/r/src/capstone-movielens.main.R\#L719}{Usage}}{Usage}}\label{usage-19}

\hfill\break

\begin{Shaded}
\begin{Highlighting}[]
\DocumentationTok{\#\#\#\# Add a row to the RMSE Result Table for the Regularized User+Movie Effect Model {-}{-}{-}{-}{-}{-}{-}{-}}
\NormalTok{RMSEs.ResultTibble.rglr.UME }\OtherTok{\textless{}{-}}\NormalTok{ RMSEs.ResultTibble.UME }\SpecialCharTok{|\textgreater{}} 
  \FunctionTok{RMSEs.AddRow}\NormalTok{(}\StringTok{"Regularized UME Model"}\NormalTok{, }
\NormalTok{               UME.rglr.retrain.RMSE,}
               \AttributeTok{comment =} \StringTok{"Computed for \textasciigrave{}lambda\textasciigrave{} = \%1"} \SpecialCharTok{|\textgreater{}}
                 \FunctionTok{msg.glue}\NormalTok{(UME.rglr.best\_lambda))}
\end{Highlighting}
\end{Shaded}

\begin{Shaded}
\begin{Highlighting}[]
\FunctionTok{RMSE\_kable}\NormalTok{(RMSEs.ResultTibble.rglr.UME)}
\end{Highlighting}
\end{Shaded}

\begin{tabular}{>{\raggedright\arraybackslash}p{15em}>{\centering\arraybackslash}p{5em}>{\raggedright\arraybackslash}p{30em}}
\toprule
\textbf{Method} & \textbf{RMSE} & \textbf{Comment}\\
\midrule
Project Objective & 0.8649000 & \\
OMR Model & 1.0603462 & Overall Mean Rating (OMR) Model\\
UE Model & 0.9697962 & User Effect (UE) Model\\
UME Model & 0.8732081 & User+Movie Effect (UME) Model\\
Regularized UME Model & 0.8729730 & Computed for `lambda` = 0.38745002746582\\
\bottomrule
\end{tabular}

\begin{noteblock}
In the \emph{code snippet} above, we use the \hyperref[func.RMSEs.AddRow]{RMSEs.AddRow} and \hyperref[func.RMSE_kable]{RMSE\_kable} functions described in section \hyperref[result-rmses-tibble-functions]{Result RMSEs Tibble Functions} of \emph{this Appendix}.

\end{noteblock}

\phantomsection\label{func.msg.glue.args}
\subparagraph{Arguments}\label{arguments-19}

\hfill\break

\phantomsection\label{func.msg.glue.args.msg_template}
\begin{itemize}
\tightlist
\item
  \textbf{msg\_template:} Any object coercible to a character vector containing at least one message template element, i.e., a string (possibly multiline) containing at least one occurrence of a pattern specified by the \hyperref[func.msg.glue.args.arg.name]{arg.name} argument.
\end{itemize}

\phantomsection\label{func.msg.glue.args.arg.name}
\begin{itemize}
\tightlist
\item
  \textbf{arg.name:} (Optional, \emph{``\%1''} by default) Substring (pattern or \emph{placeholder}) to replace all occurrences found in the message template passed in the \hyperref[func.msg.glue.args.msg_template]{msg\_template} argument with the value specified in the \hyperref[func.msg.glue.args.arg]{arg} argument.
\end{itemize}

\phantomsection\label{func.msg.glue.args.arg}
\begin{itemize}
\tightlist
\item
  \textbf{arg:} Value to replace all occurrences of the pattern specified by the \hyperref[func.msg.glue.args.arg.name]{arg.name} argument in the message template passed in the \hyperref[func.msg.glue.args.msg_template]{msg\_template} argument.
\end{itemize}

\phantomsection\label{func.msg.glue.details}
\subparagraph{Details}\label{details-21}

\hfill\break

This function internally calls the \href{https://www.rdocumentation.org/packages/stringr/versions/1.5.0/topics/str_glue}{stringr::str\_glue} \texttt{R} function to format the character vector returned by the (also called internally) \hyperref[func.msg.set_arg]{msg.set\_arg} function described above in \emph{this Section}.

\phantomsection\label{func.msg.glue.value}
\subparagraph{Value}\label{value-21}

\hfill\break

A \href{https://glue.tidyverse.org/reference/glue.html}{glue::glue} object created from the character vector returned by the internally called \hyperref[func.msg.set_arg]{msg.set\_arg()}.

\phantomsection\label{func.msg.glue.code}
\subparagraph{Source Code}\label{source-code-21}

\hfill\break

The source code of the \href{https://github.com/AzKurban-edX-DS/Capstone-MovieLens/blob/main/r/src/support-functions/common-helper.functions.R\#L457}{msg.glue} function is shown below:

\begin{Shaded}
\begin{Highlighting}[]
\NormalTok{msg.glue }\OtherTok{\textless{}{-}} \ControlFlowTok{function}\NormalTok{(msg\_template, arg, }\AttributeTok{arg.name =} \StringTok{"\%1"}\NormalTok{)\{}
\NormalTok{  msg\_template }\SpecialCharTok{|\textgreater{}}
    \FunctionTok{msg.set\_arg}\NormalTok{(arg, arg.name) }\SpecialCharTok{|\textgreater{}}
    \FunctionTok{str\_glue}\NormalTok{()}
\NormalTok{\}}
\end{Highlighting}
\end{Shaded}

\begin{noteblock}
The source code of the \href{https://github.com/AzKurban-edX-DS/Capstone-MovieLens/blob/main/r/src/support-functions/common-helper.functions.R\#L457}{msg.glue} function is also available in the \href{https://github.com/AzKurban-edX-DS/Capstone-MovieLens/blob/main/r/src/support-functions/common-helper.functions.R\#L449}{Utility Functions} section of the \href{https://github.com/AzKurban-edX-DS/Capstone-MovieLens/blob/main/r/src/support-functions/common-helper.functions.R\#L457}{common-helper.functions.R} script on \emph{GitHub}.

\end{noteblock}

\newpage

\subsubsection{(Root) Mean Squared Error Calculation}\label{root-mean-squared-error-calculation}

\begin{noteblock}
The complete source code of the functions described in this section are available in the \href{https://github.com/AzKurban-edX-DS/Capstone-MovieLens/blob/main/r/src/support-functions/common-helper.functions.R\#L3}{(R)MSE-related functions} section of the \href{https://github.com/AzKurban-edX-DS/Capstone-MovieLens/blob/main/r/src/support-functions/common-helper.functions.R\#L3}{common-helper.functions.R} script on \emph{GitHub}.

\end{noteblock}

\paragraph{\texorpdfstring{\href{https://github.com/AzKurban-edX-DS/Capstone-MovieLens/blob/main/r/src/support-functions/common-helper.functions.R\#L5}{mse} Function}{mse Function}}\label{func.mse}

\hfill\break

Calculates \emph{Mean Squared Error} of the set of \emph{residuals} (passed in the \hyperref[func.mse.args.r]{r} argument) as described in \href{https://rafalab.dfci.harvard.edu/dsbook-part-2/highdim/regularization.html\#sec-netflix-loss-function}{Section 24.1 \emph{Case study: recommendation systems / Loss function}} of the \emph{Course Textbook (New Edition)}\autocite{IDS2_24-1_LF}.

\phantomsection\label{func.mse.usage}
\subparagraph{\texorpdfstring{\href{https://github.com/AzKurban-edX-DS/Capstone-MovieLens/blob/main/r/src/capstone-movielens.main.R\#L347}{Usage}}{Usage}}\label{usage-20}

\hfill\break

\begin{Shaded}
\begin{Highlighting}[]
\NormalTok{edx.user\_effect.MSEs }\OtherTok{\textless{}{-}} \FunctionTok{sapply}\NormalTok{(edx\_CV, }\ControlFlowTok{function}\NormalTok{(cv\_fold\_dat)\{}
\NormalTok{  cv\_fold\_dat}\SpecialCharTok{$}\NormalTok{validation\_set }\SpecialCharTok{|\textgreater{}}
    \FunctionTok{left\_join}\NormalTok{(edx.user\_effect, }\AttributeTok{by =} \StringTok{"userId"}\NormalTok{) }\SpecialCharTok{|\textgreater{}}
    \FunctionTok{mutate}\NormalTok{(}\AttributeTok{resid =}\NormalTok{ rating }\SpecialCharTok{{-}} \FunctionTok{clamp}\NormalTok{(mu }\SpecialCharTok{+}\NormalTok{ a)) }\SpecialCharTok{|\textgreater{}} 
    \FunctionTok{pull}\NormalTok{(resid) }\SpecialCharTok{|\textgreater{}} \FunctionTok{mse}\NormalTok{()}
\NormalTok{\})}
\end{Highlighting}
\end{Shaded}

\begin{Shaded}
\begin{Highlighting}[]
\FunctionTok{data.frame}\NormalTok{(}\AttributeTok{fold\_No =} \DecValTok{1}\SpecialCharTok{:}\DecValTok{5}\NormalTok{, }\AttributeTok{MSE =}\NormalTok{ edx.user\_effect.MSEs) }\SpecialCharTok{|\textgreater{}}
  \FunctionTok{data.plot}\NormalTok{(}\AttributeTok{title =} \StringTok{"MSE resuls of the 5{-}fold CV method applied to the User Effect Model"}\NormalTok{,}
            \AttributeTok{xname =} \StringTok{"fold\_No"}\NormalTok{, }
            \AttributeTok{yname =} \StringTok{"MSE"}\NormalTok{)}
\end{Highlighting}
\end{Shaded}

\pandocbounded{\includegraphics[keepaspectratio]{capstone-movielens.report_files/figure-latex/unnamed-chunk-332-1.pdf}}

\begin{noteblock}
In the \href{https://github.com/AzKurban-edX-DS/Capstone-MovieLens/blob/main/r/src/capstone-movielens.main.R\#L355}{code snippet} above, we use the custom data visualization function \hyperref[func.data.plot]{data.plot} described in section \hyperref[data-visualization-functions]{Data Visualization Functions} of \emph{this Appendix}.

\end{noteblock}

\phantomsection\label{func.mse.args}
\subparagraph{Arguments}\label{arguments-20}

\hfill\break

\phantomsection\label{func.mse.args.r}
\begin{itemize}
\tightlist
\item
  \textbf{r:} Numeric vector representing a set of \emph{residuals} (for example of computing a \emph{residuals} vector, see section \hyperref[func.mse.usage]{Usage} of \emph{this Description}.
\end{itemize}

\phantomsection\label{func.mse.code}
\subparagraph{Source Code}\label{source-code-22}

\hfill\break

The source code of the \href{https://github.com/AzKurban-edX-DS/Capstone-MovieLens/blob/main/r/src/support-functions/common-helper.functions.R\#L5}{mse} function is shown below:

\begin{Shaded}
\begin{Highlighting}[]
\NormalTok{mse }\OtherTok{\textless{}{-}} \ControlFlowTok{function}\NormalTok{(r) }\FunctionTok{mean}\NormalTok{(r}\SpecialCharTok{\^{}}\DecValTok{2}\NormalTok{)}
\end{Highlighting}
\end{Shaded}

\begin{noteblock}
The source code of the \href{https://github.com/AzKurban-edX-DS/Capstone-MovieLens/blob/main/r/src/support-functions/common-helper.functions.R\#L5}{mse} function is also available in the \href{https://github.com/AzKurban-edX-DS/Capstone-MovieLens/blob/main/r/src/support-functions/common-helper.functions.R\#L3}{(R)MSE-related functions} section of the \href{https://github.com/AzKurban-edX-DS/Capstone-MovieLens/blob/main/r/src/support-functions/common-helper.functions.R\#L5}{common-helper.functions.R} script on \emph{GitHub}.

\end{noteblock}

\newpage

\paragraph{\texorpdfstring{\href{https://github.com/AzKurban-edX-DS/Capstone-MovieLens/blob/main/r/src/support-functions/common-helper.functions.R\#L6}{mse\_cv} Function}{mse\_cv Function}}\label{func.mse_cv}

\hfill\break

Calculates the average value of the \emph{Mean Squared Errors} calculated with the use of \emph{K-Fold Cross-Validation} as described in \href{https://rafalab.dfci.harvard.edu/dsbook-part-2/ml/resampling-methods.html\#k-fold-cross-validation}{Section 29.6 \emph{Cross validation / K-fold cross validation}} of the \emph{Course Textbook (New Edition)}\autocite{IDS2_29-6_KFCV}.

\begin{noteblock}
The \(K\) value for the \emph{K-Fold Cross-Validation} is determined by the length of the \hyperref[appndx_b.edx_CV]{\texttt{edx\_CV} Object} described below in \hyperref[appndx_b]{Appendix B: Models Training Datasets} (in \emph{this Project}, we use \(K = 5\)).

\end{noteblock}

\phantomsection\label{func.mse_cv.usage}
\subparagraph{Usage}\label{usage-21}

\hfill\break

\begin{Shaded}
\begin{Highlighting}[]
\FunctionTok{mse\_cv}\NormalTok{(r\_list)}
\end{Highlighting}
\end{Shaded}

\phantomsection\label{func.mse_cv.args}
\subparagraph{Arguments}\label{arguments-21}

\hfill\break

\phantomsection\label{func.mse_cv.args.r_list}
\begin{itemize}
\tightlist
\item
  \textbf{r\_list:} A list of numeric vectors, each representing a set of residuals as described in \href{https://rafalab.dfci.harvard.edu/dsbook-part-2/highdim/regularization.html\#sec-netflix-loss-function}{Section 24.1 \emph{Case Study: Recommender Systems / Loss Function}} of the \emph{Course Textbook (New Edition)}\autocite{IDS2_24-1_LF}, computed in the corresponding iteration of \emph{K-Fold Cross-Validation} (the \(K\) is \(5\) for \emph{this Project}).
\end{itemize}

\phantomsection\label{funcc.mse_cv.code}
\subparagraph{Source Code}\label{source-code-23}

\hfill\break

The source code of the \href{https://github.com/AzKurban-edX-DS/Capstone-MovieLens/blob/main/r/src/support-functions/common-helper.functions.R\#L6}{mse\_cv} function is shown below:

\begin{Shaded}
\begin{Highlighting}[]
\NormalTok{mse\_cv }\OtherTok{\textless{}{-}} \ControlFlowTok{function}\NormalTok{(r\_list) \{}
\NormalTok{  mses }\OtherTok{\textless{}{-}} \FunctionTok{sapply}\NormalTok{(r\_list, }\FunctionTok{mse}\NormalTok{(r))}
  \FunctionTok{mean}\NormalTok{(mses)}
\NormalTok{\}}
\end{Highlighting}
\end{Shaded}

\begin{noteblock}
The source code of the \href{https://github.com/AzKurban-edX-DS/Capstone-MovieLens/blob/main/r/src/support-functions/common-helper.functions.R\#L6}{mse\_cv} function is also available in the \href{https://github.com/AzKurban-edX-DS/Capstone-MovieLens/blob/main/r/src/support-functions/common-helper.functions.R\#L3}{(R)MSE-related functions} section of the \href{https://github.com/AzKurban-edX-DS/Capstone-MovieLens/blob/main/r/src/support-functions/common-helper.functions.R\#L6}{common-helper.functions.R} script on \emph{GitHub}.

\end{noteblock}

\newpage

\paragraph{\texorpdfstring{\href{https://github.com/AzKurban-edX-DS/Capstone-MovieLens/blob/main/r/src/support-functions/common-helper.functions.R\#L10}{rmse} Function}{rmse Function}}\label{func.rmse}

\hfill\break

Calculates \emph{Root Mean Squared Error} as described in \href{https://rafalab.dfci.harvard.edu/dsbook-part-2/highdim/regularization.html\#sec-netflix-loss-function}{Section 24.1 \emph{Case study: recommendation systems / Loss function}} of the \emph{Course Textbook (New Edition)}\autocite{IDS2_24-1_LF}.

\phantomsection\label{func.rmse.usage}
\subparagraph{Usage}\label{usage-22}

\hfill\break

\begin{Shaded}
\begin{Highlighting}[]
\FunctionTok{rmse}\NormalTok{(r)}
\end{Highlighting}
\end{Shaded}

\phantomsection\label{func.rmse.args}
\subparagraph{Arguments}\label{arguments-22}

\hfill\break

\phantomsection\label{func.rmse.args.r}
\begin{itemize}
\tightlist
\item
  \textbf{r:} Numeric vector representing a set of residuals as described in \href{https://rafalab.dfci.harvard.edu/dsbook-part-2/highdim/regularization.html\#sec-netflix-loss-function}{Section 24.1 \emph{Case study: recommendation systems / Loss function}} of the \emph{Course Textbook (New Edition)}\autocite{IDS2_24-1_LF}.
\end{itemize}

\phantomsection\label{func.rmse.code}
\subparagraph{Source Code}\label{source-code-24}

\hfill\break

The source code of the \href{https://github.com/AzKurban-edX-DS/Capstone-MovieLens/blob/main/r/src/support-functions/common-helper.functions.R\#L10}{rmse} function is shown below:

\begin{Shaded}
\begin{Highlighting}[]
\NormalTok{rmse }\OtherTok{\textless{}{-}} \ControlFlowTok{function}\NormalTok{(r) }\FunctionTok{sqrt}\NormalTok{(}\FunctionTok{mse}\NormalTok{(r))}
\end{Highlighting}
\end{Shaded}

\begin{noteblock}
The source code of the \href{https://github.com/AzKurban-edX-DS/Capstone-MovieLens/blob/main/r/src/support-functions/common-helper.functions.R\#L10}{rmse} function is also available in the \href{https://github.com/AzKurban-edX-DS/Capstone-MovieLens/blob/main/r/src/support-functions/common-helper.functions.R\#L3}{(R)MSE-related functions} section of the \href{https://github.com/AzKurban-edX-DS/Capstone-MovieLens/blob/main/r/src/support-functions/common-helper.functions.R\#L10}{common-helper.functions.R} script on \emph{GitHub}.

\end{noteblock}

\newpage

\paragraph{\texorpdfstring{\href{https://github.com/AzKurban-edX-DS/Capstone-MovieLens/blob/main/r/src/support-functions/common-helper.functions.R\#L11}{rmse2} Function}{rmse2 Function}}\label{func.rmse2}

\hfill\break

Calculates \emph{Root Mean Squared Error} as described in \href{https://rafalab.dfci.harvard.edu/dsbook-part-2/highdim/regularization.html\#sec-netflix-loss-function}{Section 24.1 \emph{Case study: recommendation systems / Loss function}} of the \emph{Course Textbook (New Edition)}\autocite{IDS2_24-1_LF}.

\phantomsection\label{func.rmse2.usage}
\subparagraph{Usage}\label{usage-23}

\hfill\break

\begin{Shaded}
\begin{Highlighting}[]
\FunctionTok{rmse2}\NormalTok{(true\_ratings, predicted\_ratings)}
\end{Highlighting}
\end{Shaded}

\phantomsection\label{func.rmse2.args}
\subparagraph{Arguments}\label{arguments-23}

\hfill\break

\phantomsection\label{func.rmse2.args.true_ratings}
\begin{itemize}
\tightlist
\item
  \textbf{true\_ratings:} Numeric vector representing a set of \emph{actual rating} values stored in the validation dataset.
\end{itemize}

\phantomsection\label{func.rmse2.args.predicted_ratings}
\begin{itemize}
\tightlist
\item
  \textbf{predicted\_ratings:} A numeric vector representing a set of \emph{predicted rating} values computed from the data stored in the test dataset.
\end{itemize}

\phantomsection\label{func.rmse2.code}
\subparagraph{Source Code}\label{source-code-25}

\hfill\break

The source code of the \href{https://github.com/AzKurban-edX-DS/Capstone-MovieLens/blob/main/r/src/support-functions/common-helper.functions.R\#L11}{rmse2} function is shown below:

\begin{Shaded}
\begin{Highlighting}[]
\NormalTok{rmse2 }\OtherTok{\textless{}{-}} \ControlFlowTok{function}\NormalTok{(true\_ratings, predicted\_ratings) \{}
  \FunctionTok{rmse}\NormalTok{(true\_ratings }\SpecialCharTok{{-}}\NormalTok{ predicted\_ratings)}
\NormalTok{\}}
\end{Highlighting}
\end{Shaded}

\begin{noteblock}
The source code of the \href{https://github.com/AzKurban-edX-DS/Capstone-MovieLens/blob/main/r/src/support-functions/common-helper.functions.R\#L11}{rmse2} function is also available in the \href{https://github.com/AzKurban-edX-DS/Capstone-MovieLens/blob/main/r/src/support-functions/common-helper.functions.R\#L3}{(R)MSE-related functions} section of the \href{https://github.com/AzKurban-edX-DS/Capstone-MovieLens/blob/main/r/src/support-functions/common-helper.functions.R\#L11}{common-helper.functions.R} script on \emph{GitHub}.

\end{noteblock}

\newpage

\subsubsection{Result RMSEs Tibble Functions}\label{result-rmses-tibble-functions}

\hfill\break

\begin{noteblock}
The complete source code of the functions described in this section are available in the \href{https://github.com/AzKurban-edX-DS/Capstone-MovieLens/blob/main/r/src/support-functions/common-helper.functions.R\#L15}{RMSEs Result Tibble} section of the \href{https://github.com/AzKurban-edX-DS/Capstone-MovieLens/blob/main/r/src/support-functions/common-helper.functions.R\#L15}{common-helper.functions.R} script on \emph{GitHub}.

\end{noteblock}

\paragraph{\texorpdfstring{\href{https://github.com/AzKurban-edX-DS/Capstone-MovieLens/blob/main/r/src/support-functions/common-helper.functions.R\#L16}{CreateRMSEs\_ResultTibble} Function}{CreateRMSEs\_ResultTibble Function}}\label{func.CreateRMSEs_ResultTibble}

\hfill\break

Creates a \emph{RMSEs Result Tibble} of type \emph{data frame}.

\phantomsection\label{func.CreateRMSEs_ResultTibble.usage}
\subparagraph{\texorpdfstring{\href{https://github.com/AzKurban-edX-DS/Capstone-MovieLens/blob/main/r/src/capstone-movielens.main.R\#L87}{Usage}}{Usage}}\label{usage-24}

\hfill\break

\begin{Shaded}
\begin{Highlighting}[]
\CommentTok{\# Create the table and add a first row for the Project Objective}
\NormalTok{RMSEs.ResultTibble }\OtherTok{\textless{}{-}} \FunctionTok{CreateRMSEs\_ResultTibble}\NormalTok{()}
\end{Highlighting}
\end{Shaded}

\phantomsection\label{func.CreateRMSEs_ResultTibble.details}
\subparagraph{Details}\label{details-22}

\hfill\break

This function is a wrapper for the \href{https://www.rdocumentation.org/packages/tibble/versions/3.2.1/topics/tibble}{tibble::tibble} function, which is called internally to construct a data frame object representing the \emph{RMSEs Result Tibble} to store the \emph{RMSE} values for every model used in the project to predict movie ratings.

The result data frame, which is returned to the client code, has the following columns:

\begin{itemize}
\tightlist
\item
  \emph{Method} (chr);
\item
  \emph{RMSE} (num);
\item
  \emph{Comment} (chr)
\end{itemize}

\phantomsection\label{func.CreateRMSEs_ResultTibble.value}
\subparagraph{Value}\label{value-22}

\hfill\break

A tibble returned by the internally called \href{https://www.rdocumentation.org/packages/tibble/versions/3.2.1/topics/tibble}{tibble::tibble()}

\phantomsection\label{func.CreateRMSEs_ResultTibble.code}
\subparagraph{Source Code}\label{source-code-26}

\hfill\break

The source code of the \href{https://github.com/AzKurban-edX-DS/Capstone-MovieLens/blob/main/r/src/support-functions/common-helper.functions.R\#L16}{CreateRMSEs\_ResultTibble} function is shown below:

\begin{Shaded}
\begin{Highlighting}[]
\NormalTok{CreateRMSEs\_ResultTibble }\OtherTok{\textless{}{-}} \ControlFlowTok{function}\NormalTok{()\{}
  \FunctionTok{tibble}\NormalTok{(}\AttributeTok{Method =} \FunctionTok{c}\NormalTok{(}\StringTok{"Project Objective"}\NormalTok{),}
         \AttributeTok{RMSE =}\NormalTok{ project\_objective,}
         \AttributeTok{Comment =} \StringTok{" "}\NormalTok{)}
\NormalTok{\}}
\end{Highlighting}
\end{Shaded}

\begin{noteblock}
The source code of the \href{https://github.com/AzKurban-edX-DS/Capstone-MovieLens/blob/main/r/src/support-functions/common-helper.functions.R\#L16}{CreateRMSEs\_ResultTibble} function is also available in the \href{https://github.com/AzKurban-edX-DS/Capstone-MovieLens/blob/main/r/src/support-functions/common-helper.functions.R\#L15}{RMSEs Result Tibble} section of the \href{https://github.com/AzKurban-edX-DS/Capstone-MovieLens/blob/main/r/src/support-functions/common-helper.functions.R\#L16}{common-helper.functions.R} script on \emph{GitHub}.

\end{noteblock}

\newpage

\paragraph{\texorpdfstring{\href{https://github.com/AzKurban-edX-DS/Capstone-MovieLens/blob/main/r/src/support-functions/common-helper.functions.R\#L21}{RMSEs.AddRow} Function}{RMSEs.AddRow Function}}\label{func.RMSEs.AddRow}

\hfill\break

Adds a row to the original \emph{RMSEs Result Tibble} passed in the \hyperref[func.RMSEs.AddRow.args.RMSEs]{RMSEs} argument.

\phantomsection\label{func.RMSEs.AddRow.usage}
\subparagraph{\texorpdfstring{\href{https://github.com/AzKurban-edX-DS/Capstone-MovieLens/blob/main/r/src/capstone-movielens.main.R\#L199}{Usage}}{Usage}}\label{usage-25}

\hfill\break

\begin{Shaded}
\begin{Highlighting}[]
\DocumentationTok{\#\#\# Add a row to the RMSE Result Tibble {-}{-}{-}{-}{-}{-}{-}{-}{-}{-}{-}{-}{-}{-}{-}{-}{-}{-}{-}{-}{-}{-}{-}{-}{-}{-}{-}{-}{-}{-}{-}{-}{-}{-}{-}{-}{-}{-}{-}{-} }
\NormalTok{RMSEs.ResultTibble.OMR }\OtherTok{\textless{}{-}}\NormalTok{ RMSEs.ResultTibble }\SpecialCharTok{|\textgreater{}} 
  \FunctionTok{RMSEs.AddRow}\NormalTok{(}\StringTok{"OMR Model"}\NormalTok{, }
\NormalTok{               mu.RMSE,}
               \AttributeTok{comment =} \StringTok{"Overall Mean Rating (OMR) Model"}\NormalTok{)}
\end{Highlighting}
\end{Shaded}

\begin{Shaded}
\begin{Highlighting}[]
\FunctionTok{RMSE\_kable}\NormalTok{(RMSEs.ResultTibble.OMR)}
\end{Highlighting}
\end{Shaded}

\begin{tabular}{>{\raggedright\arraybackslash}p{15em}>{\centering\arraybackslash}p{5em}>{\raggedright\arraybackslash}p{30em}}
\toprule
\textbf{Method} & \textbf{RMSE} & \textbf{Comment}\\
\midrule
Project Objective & 0.864900 & \\
OMR Model & 1.060346 & Overall Mean Rating (OMR) Model\\
\bottomrule
\end{tabular}

\begin{noteblock}
In the \href{https://github.com/AzKurban-edX-DS/Capstone-MovieLens/blob/main/r/src/capstone-movielens.main.R\#L205}{code snippet} above, we use the \hyperref[func.RMSE_kable]{RMSE\_kable} functions described below in \emph{this Section}.

\end{noteblock}

\phantomsection\label{func.RMSEs.AddRow.args}
\subparagraph{Arguments}\label{arguments-24}

\hfill\break

\phantomsection\label{func.RMSEs.AddRow.args.RMSEs}
\begin{itemize}
\tightlist
\item
  \textbf{\emph{RMSEs:}} Original \emph{RMSEs Result Tibble} (of type \emph{data frame}) to modify;
\end{itemize}

\phantomsection\label{func.RMSEs.AddRow.args.method}
\begin{itemize}
\tightlist
\item
  \textbf{\emph{method:}} The \emph{Method or Model} being evaluated;
\end{itemize}

\phantomsection\label{func.RMSEs.AddRow.args.value}
\begin{itemize}
\tightlist
\item
  \textbf{\emph{value:}} The \emph{Root Mean Squared Error (RMSE)} value;
\end{itemize}

\phantomsection\label{func.RMSEs.AddRow.args.comment}
\begin{itemize}
\tightlist
\item
  \textbf{\emph{comment:}} (Optional, \emph{Empty string} by default) Detailed information about the record or any other comment (if needed);
\end{itemize}

\phantomsection\label{func.RMSEs.AddRow.args.before}
\begin{itemize}
\tightlist
\item
  \textbf{\emph{before:}} (Optional, \emph{Integer}, \texttt{NULL} by default) One-based row index where to add the new rows, if NULL (default): after last row;
\end{itemize}

\phantomsection\label{func.RMSEs.AddRow.code.details}
\subparagraph{Details}\label{details-23}

\hfill\break

This function is a wrapper for the \href{https://www.rdocumentation.org/packages/tibble/versions/3.2.1/topics/add_row}{tibble::add\_row} function, which is called internally to modify a data frame object representing the original \emph{RMSEs Result Tibble}, which stores the \emph{RMSE} values for the models used in the project to predict movie ratings.

For more information about the \emph{RMSEs Result Tibble}, see the \hyperref[func.CreateRMSEs_ResultTibble.details]{Details} section of the \hyperref[func.CreateRMSEs_ResultTibble]{CreateRMSEs\_ResultTibble} function description above.

\phantomsection\label{func.RMSEs.AddRow.value}
\subparagraph{Value}\label{value-23}

\hfill\break

The modified tibble returned by the internally called \href{https://www.rdocumentation.org/packages/tibble/versions/3.2.1/topics/add_row}{tibble::add\_row()}.

\phantomsection\label{func.RMSEs.AddRow.code}
\subparagraph{Source Code}\label{source-code-27}

\hfill\break

The source code of the \href{https://github.com/AzKurban-edX-DS/Capstone-MovieLens/blob/main/r/src/support-functions/common-helper.functions.R\#L21}{RMSEs.AddRow} function is shown below:

\begin{Shaded}
\begin{Highlighting}[]
\NormalTok{RMSEs.AddRow }\OtherTok{\textless{}{-}} \ControlFlowTok{function}\NormalTok{(RMSEs, }
\NormalTok{                         method, }
\NormalTok{                         value, }
                         \AttributeTok{comment =} \StringTok{""}\NormalTok{, }
                         \AttributeTok{before =} \ConstantTok{NULL}\NormalTok{)\{}
\NormalTok{  RMSEs }\SpecialCharTok{|\textgreater{}}
    \FunctionTok{add\_row}\NormalTok{(}\AttributeTok{Method =}\NormalTok{ method,}
            \AttributeTok{RMSE =}\NormalTok{ value,}
            \AttributeTok{Comment =}\NormalTok{ comment,}
            \AttributeTok{.before =}\NormalTok{ before)}
\NormalTok{\}}
\end{Highlighting}
\end{Shaded}

\begin{noteblock}
The source code of the \href{https://github.com/AzKurban-edX-DS/Capstone-MovieLens/blob/main/r/src/support-functions/common-helper.functions.R\#L21}{RMSEs.AddRow} function is also available in the \href{https://github.com/AzKurban-edX-DS/Capstone-MovieLens/blob/main/r/src/support-functions/common-helper.functions.R\#L15}{RMSEs Result Tibble} section of the \href{https://github.com/AzKurban-edX-DS/Capstone-MovieLens/blob/main/r/src/support-functions/common-helper.functions.R\#L21}{common-helper.functions.R} script on \emph{GitHub}.

\end{noteblock}

\newpage

\paragraph{\texorpdfstring{\href{https://github.com/AzKurban-edX-DS/Capstone-MovieLens/blob/main/r/src/support-functions/common-helper.functions.R\#L32}{RMSEs.AddDiffColumn} Function}{RMSEs.AddDiffColumn Function}}\label{func.RMSEs.AddDiffColumn}

\hfill\break

Adds a calculated column named \emph{Diff} to the \emph{RMSEs Result Tibble} (passed in the \hyperref[func.RMSEs.AddDiffColumn.args.RMSEs]{RMSEs} argument) that contains, for each row, the difference between the value of the \emph{RMSE} column for the previous row and the value of the same column for the current row. This is true for each row except the first, for which the value is always zero.

\phantomsection\label{func.RMSEs.AddDiffColumn.usage}
\subparagraph{\texorpdfstring{\href{https://github.com/AzKurban-edX-DS/Capstone-MovieLens/blob/main/r/src/capstone-movielens.main.R\#L2389}{Usage}}{Usage}}\label{usage-26}

\hfill\break

\begin{Shaded}
\begin{Highlighting}[]
\NormalTok{total.RMSEs.ResultTibble }\OtherTok{\textless{}{-}} \FunctionTok{RMSEs.AddDiffColumn}\NormalTok{(final.MF.RMSEs.ResultTibble)}
\end{Highlighting}
\end{Shaded}

\begin{Shaded}
\begin{Highlighting}[]
\FunctionTok{RMSE.Total\_kable}\NormalTok{(total.RMSEs.ResultTibble)}
\end{Highlighting}
\end{Shaded}

\begin{tabular}{>{\raggedright\arraybackslash}p{15em}>{\centering\arraybackslash}p{7em}>{\raggedright\arraybackslash}p{5em}>{\raggedright\arraybackslash}p{25em}}
\toprule
\textbf{Method} & \textbf{RMSE} & \textbf{Diff} & \textbf{Comment}\\
\midrule
Project Objective & 0.8649000 & 0.0000000 & \\
OMR Model & 1.0603462 & -0.1954462 & Overall Mean Rating (OMR) Model\\
UE Model & 0.9697962 & 0.0905500 & User Effect (UE) Model\\
UME Model & 0.8732081 & 0.0965881 & User+Movie Effect (UME) Model\\
Regularized UME Model & 0.8729730 & 0.0002351 & Computed for `lambda` = 0.38745002746582\\
\addlinespace
UMGE Model & 0.8729730 & 0.0000000 & User+Movie+Genre Effect (UMGE) Model\\
Regularized UMGE Model & 0.8729730 & 0.0000000 & Computed for `lambda` = 0.0359375\\
UMGYE Model & 0.8723975 & 0.0005755 & User+Movie+Genre+Year Effect (UMGYE) Model\\
Regularized UMGYE Model & 0.8721860 & 0.0002116 & Computed for `lambda` = 233.784942626953\\
UMGYDE (Default) Model & 0.8722314 & -0.0000454 & User+Movie+Genre+Year+Day Effect (UMGYDE) Model 
computed using `stats::loess` function with `degree=1` \& `span=0.75` parameter values.\\
\addlinespace
Tuned UMGYDE.d0 Model & 0.8711135 & 0.0011179 & UMGYDE Model computed using function call: `loess(degree = 0, span = 0.00109375)`\\
Tuned UMGYDE.d1 Model & 0.8708300 & 0.0002835 & UMGYDE Model computed using function call: `loess(degree = 1, span = 0.00109375)`\\
Tuned UMGYDE.d2 Model & 0.8709891 & -0.0001591 & UMGYDE Model computed using function call: `loess(degree = 2, span = 0.00153125)`\\
Tuned UMGYDE Best Model & 0.8707855 & 0.0002036 & UMGYDE Model computed using `loess` function call with the best degree \& span values.\\
Regularized UMGYDE Model & 0.8707755 & 0.0000100 & The best tuned and regularized UMGYDE Model.\\
\addlinespace
MF (Final Test) & 0.7876658 & 0.0831097 & Matrix Factorization of the Best Model Residuals, Final Holdout Test\\
\bottomrule
\end{tabular}

\begin{noteblock}
In the \href{https://github.com/AzKurban-edX-DS/Capstone-MovieLens/blob/main/r/src/capstone-movielens.main.R\#L2390}{code snippet} above, we use the \hyperref[func.RMSE.Total_kable]{RMSE.Total\_kable} functions described below in \emph{this Section}.

\end{noteblock}

\phantomsection\label{func.RMSEs.AddDiffColumn.args}
\subparagraph{Arguments}\label{arguments-25}

\hfill\break

\phantomsection\label{func.RMSEs.AddDiffColumn.args.RMSEs}
\begin{itemize}
\tightlist
\item
  \textbf{\emph{RMSEs:}} \emph{RMSEs Result Tibble} to modify;
\end{itemize}

\phantomsection\label{func.RMSEs.AddDiffColumn.details}
\subparagraph{Details}\label{details-24}

\hfill\break

This function is a wrapper for the \href{https://www.rdocumentation.org/packages/tibble/versions/3.2.1/topics/add_column}{tibble::add\_column} function, which is called internally to modify a data frame object representing the original \emph{RMSEs Result Tibble}. It also internally uses the \hyperref[func.RMSEs.AddRow]{RMSEs.AddRow} function described above to perform the necessary intermediate calculations.

For more information about the \emph{RMSEs Result Tibble}, see the \hyperref[func.CreateRMSEs_ResultTibble.details]{Details} section of the \hyperref[func.CreateRMSEs_ResultTibble]{CreateRMSEs\_ResultTibble} function description above.

\phantomsection\label{func.RMSEs.AddRow.value}
\subparagraph{Value}\label{value-24}

\hfill\break

The modified tibble returned by the internally called \href{https://www.rdocumentation.org/packages/tibble/versions/3.2.1/topics/add_column}{tibble::add\_column()}.

\phantomsection\label{func.RMSEs.AddDiffColumn.code}
\subparagraph{Source Code}\label{source-code-28}

\hfill\break

The source code of the \href{https://github.com/AzKurban-edX-DS/Capstone-MovieLens/blob/main/r/src/support-functions/common-helper.functions.R\#L32}{RMSEs.AddDiffColumn} function is shown below:

\begin{Shaded}
\begin{Highlighting}[]
\NormalTok{RMSEs.AddDiffColumn }\OtherTok{\textless{}{-}} \ControlFlowTok{function}\NormalTok{(RMSEs)\{}
\NormalTok{  RMSEs.Diff }\OtherTok{\textless{}{-}}\NormalTok{ RMSEs }\SpecialCharTok{|\textgreater{}}
    \FunctionTok{RMSEs.AddRow}\NormalTok{(}\ConstantTok{NULL}\NormalTok{, project\_objective, }\AttributeTok{before =} \DecValTok{1}\NormalTok{) }\CommentTok{\#|\textgreater{}}

\NormalTok{  RMSEs.Diff }\OtherTok{\textless{}{-}}\NormalTok{ RMSEs.Diff[}\SpecialCharTok{{-}}\FunctionTok{nrow}\NormalTok{(RMSEs.Diff),]}
    
\NormalTok{  RMSEs }\SpecialCharTok{|\textgreater{}}
  \FunctionTok{add\_column}\NormalTok{(}\AttributeTok{Diff =}\NormalTok{ RMSEs.Diff}\SpecialCharTok{$}\NormalTok{RMSE }\SpecialCharTok{{-}}\NormalTok{ RMSEs}\SpecialCharTok{$}\NormalTok{RMSE, }
             \AttributeTok{.before =} \StringTok{"Comment"}\NormalTok{)}
  
\NormalTok{\}}
\end{Highlighting}
\end{Shaded}

\begin{noteblock}
The source code of the \href{https://github.com/AzKurban-edX-DS/Capstone-MovieLens/blob/main/r/src/support-functions/common-helper.functions.R\#L32}{RMSEs.AddDiffColumn} function is also available in the \href{https://github.com/AzKurban-edX-DS/Capstone-MovieLens/blob/main/r/src/support-functions/common-helper.functions.R\#L15}{RMSEs Result Tibble} section of the \href{https://github.com/AzKurban-edX-DS/Capstone-MovieLens/blob/main/r/src/support-functions/common-helper.functions.R\#L32}{common-helper.functions.R} script on \emph{GitHub}.

\end{noteblock}

\paragraph{\texorpdfstring{\href{https://github.com/AzKurban-edX-DS/Capstone-MovieLens/blob/main/r/src/support-functions/common-helper.functions.R\#L66}{RSME.tibble.col\_width} Function}{RSME.tibble.col\_width Function}}\label{func.RSME.tibble.col_width}

\hfill\break

Converts the \emph{RMSEs Result Tibble} column width value passed in argument \hyperref[func.RSME.tibble.col_width.args.x]{x} to a character string compatible for use as the \texttt{width} argument value to pass to the \href{https://www.rdocumentation.org/packages/kableExtra/versions/1.3.4/topics/column_spec}{kableExtra::column\_spec} function.

\phantomsection\label{func.RSME.tibble.col_width.usage}
\subparagraph{Usage}\label{usage-27}

\hfill\break

\begin{Shaded}
\begin{Highlighting}[]
\NormalTok{  RMSEs }\SpecialCharTok{|\textgreater{}}
    \FunctionTok{kable}\NormalTok{(}\AttributeTok{align=}\StringTok{\textquotesingle{}lcl\textquotesingle{}}\NormalTok{, }\AttributeTok{booktabs =}\NormalTok{ T, }\AttributeTok{padding =} \DecValTok{5}\NormalTok{) }\SpecialCharTok{|\textgreater{}} 
    \FunctionTok{row\_spec}\NormalTok{(}\DecValTok{0}\NormalTok{, }\AttributeTok{bold =}\NormalTok{ T) }\SpecialCharTok{|\textgreater{}}
    \FunctionTok{column\_spec}\NormalTok{(}\AttributeTok{column =} \DecValTok{1}\NormalTok{, }\AttributeTok{width =} \FunctionTok{RSME.tibble.col\_width}\NormalTok{(col1width)) }\SpecialCharTok{|\textgreater{}}
    \FunctionTok{column\_spec}\NormalTok{(}\AttributeTok{column =} \DecValTok{2}\NormalTok{, }\AttributeTok{width =} \FunctionTok{RSME.tibble.col\_width}\NormalTok{(col2width)) }\SpecialCharTok{|\textgreater{}}
    \FunctionTok{column\_spec}\NormalTok{(}\AttributeTok{column =} \DecValTok{3}\NormalTok{, }\AttributeTok{width =} \FunctionTok{RSME.tibble.col\_width}\NormalTok{(col3width)) }
\end{Highlighting}
\end{Shaded}

\begin{noteblock}
In the \href{https://github.com/AzKurban-edX-DS/Capstone-MovieLens/blob/main/r/src/support-functions/common-helper.functions.R\#L47}{code snippet} above, the function \texttt{RSME.tibble.col\_width} is called internally from the \hyperref[func.RMSE_kable]{RMSE\_kable} function described below in \emph{this Section}.

\end{noteblock}

\phantomsection\label{func.RSME.tibble.col_width.args}
\subparagraph{Arguments}\label{arguments-26}

\hfill\break

\phantomsection\label{func.RSME.tibble.col_width.args.x}
\begin{itemize}
\tightlist
\item
  \textbf{\emph{x:}} (Numeric) Column width value to be formatted;
\end{itemize}

\phantomsection\label{func.RSME.tibble.col_width.details}
\subparagraph{Details}\label{details-25}

\hfill\break

This is a utility function, a wrapper for the \hyperref[func.msg.glue]{msg.glue} function described in section \hyperref[appndx_a.CHF.utility_funcs]{Utility Functions} of \emph{this Appendix}, which is called internally to format the value passed in argument \hyperref[func.RSME.tibble.col_width.args.x]{x} for subsequent passing to the function \href{https://www.rdocumentation.org/packages/kableExtra/versions/1.3.4/topics/column_spec}{kableExtra::column\_spec} as a value of the \texttt{width} argument.

\begin{noteblock}
As a utility, this function is intended for use in other user-defined functions, such as the \hyperref[func.RMSE_kable]{RMSE\_kable} (as shown above in the \hyperref[func.RSME.tibble.col_width.usage]{Usage} section of \emph{this Description}) and \hyperref[func.RMSE.Total_kable]{RMSE.Total\_kable} described below in \emph{this Section}.

\end{noteblock}

\phantomsection\label{func.RSME.tibble.col_width.value}
\subparagraph{Value}\label{value-25}

\hfill\break

A character string representing the formatted column width value.

\phantomsection\label{func.RSME.tibble.col_width.code}
\subparagraph{Source Code}\label{source-code-29}

\hfill\break

The source code of the \href{https://github.com/AzKurban-edX-DS/Capstone-MovieLens/blob/main/r/src/support-functions/common-helper.functions.R\#L66}{RSME.tibble.col\_width} function is shown below:

\begin{Shaded}
\begin{Highlighting}[]
\NormalTok{RSME.tibble.col\_width }\OtherTok{\textless{}{-}} \ControlFlowTok{function}\NormalTok{(x)\{}
  \StringTok{"\%1em"} \SpecialCharTok{|\textgreater{}} \FunctionTok{msg.glue}\NormalTok{(x)}
\NormalTok{\}}
\end{Highlighting}
\end{Shaded}

\begin{noteblock}
The source code of the \href{https://github.com/AzKurban-edX-DS/Capstone-MovieLens/blob/main/r/src/support-functions/common-helper.functions.R\#L66}{RSME.tibble.col\_width} function is also available in the \href{https://github.com/AzKurban-edX-DS/Capstone-MovieLens/blob/main/r/src/support-functions/common-helper.functions.R\#L15}{RMSEs Result Tibble} section of the \href{https://github.com/AzKurban-edX-DS/Capstone-MovieLens/blob/main/r/src/support-functions/common-helper.functions.R\#L66}{common-helper.functions.R} script on \emph{GitHub}.

\end{noteblock}

\newpage

\paragraph{\texorpdfstring{\href{https://github.com/AzKurban-edX-DS/Capstone-MovieLens/blob/main/r/src/support-functions/common-helper.functions.R\#L43}{RMSE\_kable} Function}{RMSE\_kable Function}}\label{func.RMSE_kable}

\hfill\break

Prints the \emph{RMSEs Result Tibble}.

\begin{noteblock}

It is assumed that the tibble has the following three columns:

\begin{enumerate}
\def\labelenumi{\arabic{enumi}.}
\tightlist
\item
  \emph{Method};
\item
  \emph{RMSE};
\item
  \emph{Comment};
\end{enumerate}

\end{noteblock}

\phantomsection\label{func.RMSE_kable.usage}
\subparagraph{\texorpdfstring{\href{https://github.com/AzKurban-edX-DS/Capstone-MovieLens/blob/main/r/src/capstone-movielens.main.R\#L205}{Usage}}{Usage}}\label{usage-28}

\hfill\break

\begin{Shaded}
\begin{Highlighting}[]
\FunctionTok{RMSE\_kable}\NormalTok{(RMSEs.ResultTibble.OMR)}
\end{Highlighting}
\end{Shaded}

\begin{tabular}{>{\raggedright\arraybackslash}p{15em}>{\centering\arraybackslash}p{5em}>{\raggedright\arraybackslash}p{30em}}
\toprule
\textbf{Method} & \textbf{RMSE} & \textbf{Comment}\\
\midrule
Project Objective & 0.864900 & \\
OMR Model & 1.060346 & Overall Mean Rating (OMR) Model\\
\bottomrule
\end{tabular}

\phantomsection\label{func.RMSE_kable.args}
\subparagraph{Arguments}\label{arguments-27}

\hfill\break

\phantomsection\label{func.RMSE_kable.args.RMSEs}
\begin{itemize}
\tightlist
\item
  \textbf{\emph{RMSEs:}} \emph{RMSEs Result Tibble} to print;
\end{itemize}

\phantomsection\label{func.RMSE_kable.args.col1width}
\begin{itemize}
\tightlist
\item
  \textbf{\emph{col1width:}} Width of the first column (Optional, 15 by default);
\end{itemize}

\phantomsection\label{func.RMSE_kable.args.col2width}
\begin{itemize}
\tightlist
\item
  \textbf{\emph{col2width:}} Width of the second column (Optional, 5 by default);
\end{itemize}

\phantomsection\label{func.RMSE_kable.args.col3width}
\begin{itemize}
\tightlist
\item
  \textbf{\emph{col3width:}} Width of the third column (Optional, 30 by default).
\end{itemize}

\phantomsection\label{func.RMSE_kable.details}
\subparagraph{Details}\label{details-26}

\hfill\break

This function is a wrapper for the \href{https://www.rdocumentation.org/packages/knitr/versions/1.43/topics/kable}{knitr::kable} function, which is called internally to create a table (based on the \emph{RMSEs Result Tibble} data), formatted for output in the project report.

The \href{https://www.rdocumentation.org/packages/kableExtra/versions/1.3.4/topics/column_spec}{kableExtra::column\_spec} function, along with the \hyperref[func.RSME.tibble.col_width]{RSME.tibble.col\_width} function described above, are also used internally to specify the width of the table columns.

For more information about the \emph{RMSEs Result Tibble}, see the \hyperref[func.CreateRMSEs_ResultTibble.details]{Details} section of the \hyperref[func.CreateRMSEs_ResultTibble]{CreateRMSEs\_ResultTibble} function description above.

\phantomsection\label{func.RMSE_kable.value}
\subparagraph{Value}\label{value-26}

\hfill\break

The table created by the internally called \href{https://www.rdocumentation.org/packages/knitr/versions/1.43/topics/kable}{knitr::kable()}.

\phantomsection\label{func.RMSE_kable.code}
\subparagraph{Source Code}\label{source-code-30}

\hfill\break

The source code of the \href{https://github.com/AzKurban-edX-DS/Capstone-MovieLens/blob/main/r/src/support-functions/common-helper.functions.R\#L43}{RMSE\_kable} function is shown below:

\begin{Shaded}
\begin{Highlighting}[]
\NormalTok{RMSE\_kable }\OtherTok{\textless{}{-}} \ControlFlowTok{function}\NormalTok{(RMSEs,}
                       \AttributeTok{col1width =} \DecValTok{15}\NormalTok{,}
                       \AttributeTok{col2width =} \DecValTok{5}\NormalTok{,}
                       \AttributeTok{col3width =} \DecValTok{30}\NormalTok{)\{}
\NormalTok{  RMSEs }\SpecialCharTok{|\textgreater{}}
    \FunctionTok{kable}\NormalTok{(}\AttributeTok{align=}\StringTok{\textquotesingle{}lcl\textquotesingle{}}\NormalTok{, }\AttributeTok{booktabs =}\NormalTok{ T, }\AttributeTok{padding =} \DecValTok{5}\NormalTok{) }\SpecialCharTok{|\textgreater{}} 
    \FunctionTok{row\_spec}\NormalTok{(}\DecValTok{0}\NormalTok{, }\AttributeTok{bold =}\NormalTok{ T) }\SpecialCharTok{|\textgreater{}}
    \FunctionTok{column\_spec}\NormalTok{(}\AttributeTok{column =} \DecValTok{1}\NormalTok{, }\AttributeTok{width =} \FunctionTok{RSME.tibble.col\_width}\NormalTok{(col1width)) }\SpecialCharTok{|\textgreater{}}
    \FunctionTok{column\_spec}\NormalTok{(}\AttributeTok{column =} \DecValTok{2}\NormalTok{, }\AttributeTok{width =} \FunctionTok{RSME.tibble.col\_width}\NormalTok{(col2width)) }\SpecialCharTok{|\textgreater{}}
    \FunctionTok{column\_spec}\NormalTok{(}\AttributeTok{column =} \DecValTok{3}\NormalTok{, }\AttributeTok{width =} \FunctionTok{RSME.tibble.col\_width}\NormalTok{(col3width)) }
\NormalTok{\}}
\end{Highlighting}
\end{Shaded}

\begin{noteblock}
The source code of the \href{https://github.com/AzKurban-edX-DS/Capstone-MovieLens/blob/main/r/src/support-functions/common-helper.functions.R\#L43}{RMSE\_kable} function is also available in the \href{https://github.com/AzKurban-edX-DS/Capstone-MovieLens/blob/main/r/src/support-functions/common-helper.functions.R\#L15}{RMSEs Result Tibble} section of the \href{https://github.com/AzKurban-edX-DS/Capstone-MovieLens/blob/main/r/src/support-functions/common-helper.functions.R\#L43}{common-helper.functions.R} script on \emph{GitHub}.

\end{noteblock}

\newpage

\paragraph{\texorpdfstring{\href{https://github.com/AzKurban-edX-DS/Capstone-MovieLens/blob/main/r/src/support-functions/common-helper.functions.R\#L54}{RMSE.Total\_kable} Function}{RMSE.Total\_kable Function}}\label{func.RMSE.Total_kable}

\hfill\break

Prints the \emph{Final RMSEs Result Tibble}.

\begin{noteblock}

It is assumed that the tibble has the following four columns:

\begin{enumerate}
\def\labelenumi{\arabic{enumi}.}
\tightlist
\item
  \emph{Method};
\item
  \emph{RMSE};
\item
  \emph{Diff};
\item
  \emph{Comment};
\end{enumerate}

\end{noteblock}

\phantomsection\label{func.RMSE.Total_kableusage}
\subparagraph{\texorpdfstring{\href{https://github.com/AzKurban-edX-DS/Capstone-MovieLens/blob/main/r/src/capstone-movielens.main.R\#L2390}{Usage}}{Usage}}\label{usage-29}

\hfill\break

\begin{Shaded}
\begin{Highlighting}[]
\FunctionTok{RMSE.Total\_kable}\NormalTok{(total.RMSEs.ResultTibble)}
\end{Highlighting}
\end{Shaded}

\begin{tabular}{>{\raggedright\arraybackslash}p{15em}>{\centering\arraybackslash}p{7em}>{\raggedright\arraybackslash}p{5em}>{\raggedright\arraybackslash}p{25em}}
\toprule
\textbf{Method} & \textbf{RMSE} & \textbf{Diff} & \textbf{Comment}\\
\midrule
Project Objective & 0.8649000 & 0.0000000 & \\
OMR Model & 1.0603462 & -0.1954462 & Overall Mean Rating (OMR) Model\\
UE Model & 0.9697962 & 0.0905500 & User Effect (UE) Model\\
UME Model & 0.8732081 & 0.0965881 & User+Movie Effect (UME) Model\\
Regularized UME Model & 0.8729730 & 0.0002351 & Computed for `lambda` = 0.38745002746582\\
\addlinespace
UMGE Model & 0.8729730 & 0.0000000 & User+Movie+Genre Effect (UMGE) Model\\
Regularized UMGE Model & 0.8729730 & 0.0000000 & Computed for `lambda` = 0.0359375\\
UMGYE Model & 0.8723975 & 0.0005755 & User+Movie+Genre+Year Effect (UMGYE) Model\\
Regularized UMGYE Model & 0.8721860 & 0.0002116 & Computed for `lambda` = 233.784942626953\\
UMGYDE (Default) Model & 0.8722314 & -0.0000454 & User+Movie+Genre+Year+Day Effect (UMGYDE) Model 
computed using `stats::loess` function with `degree=1` \& `span=0.75` parameter values.\\
\addlinespace
Tuned UMGYDE.d0 Model & 0.8711135 & 0.0011179 & UMGYDE Model computed using function call: `loess(degree = 0, span = 0.00109375)`\\
Tuned UMGYDE.d1 Model & 0.8708300 & 0.0002835 & UMGYDE Model computed using function call: `loess(degree = 1, span = 0.00109375)`\\
Tuned UMGYDE.d2 Model & 0.8709891 & -0.0001591 & UMGYDE Model computed using function call: `loess(degree = 2, span = 0.00153125)`\\
Tuned UMGYDE Best Model & 0.8707855 & 0.0002036 & UMGYDE Model computed using `loess` function call with the best degree \& span values.\\
Regularized UMGYDE Model & 0.8707755 & 0.0000100 & The best tuned and regularized UMGYDE Model.\\
\addlinespace
MF (Final Test) & 0.7876658 & 0.0831097 & Matrix Factorization of the Best Model Residuals, Final Holdout Test\\
\bottomrule
\end{tabular}

\newpage

\phantomsection\label{func.RMSE.Total_kable.args}
\subparagraph{Arguments}\label{arguments-28}

\hfill\break

\phantomsection\label{func.RMSE.Total_kable.args.RMSEs}
\begin{itemize}
\tightlist
\item
  \textbf{\emph{RMSEs:}} The \emph{Final RMSEs Result Tibble} to print;
\end{itemize}

\phantomsection\label{func.RMSE.Total_kable.args.}
\begin{itemize}
\tightlist
\item
  \textbf{\emph{col1width:}} (Optional, 15 by default) Width of the first column;
\end{itemize}

\phantomsection\label{func.RMSE.Total_kable.args.}
\begin{itemize}
\tightlist
\item
  \textbf{\emph{col2width:}} (Optional, 7 by default) Width of the second column;
\end{itemize}

\phantomsection\label{func.RMSE.Total_kable.args.}
\begin{itemize}
\tightlist
\item
  \textbf{\emph{col3width:}} (Optional, 5 by default) Width of the third column;
\end{itemize}

\phantomsection\label{func.RMSE.Total_kable.args.}
\begin{itemize}
\tightlist
\item
  \textbf{\emph{col4width:}} (Optional, 25 by default) Width of the fourth column;
\end{itemize}

\phantomsection\label{func.RMSE.Total_kable.details}
\subparagraph{Details}\label{details-27}

\hfill\break

This function is a wrapper for the \hyperref[func.RMSE_kable]{RMSE\_kable} function described above, which is called internally to print the table, created from the final \emph{RMSEs Result Tibble} data.

It is assumed that the \emph{Final RMSEs Result Tibble} is previously constructed from the original \emph{RMSEs Result Tibble} using the \hyperref[func.RMSEs.AddDiffColumn]{RMSEs.AddDiffColumn} function described above.

The \href{https://www.rdocumentation.org/packages/kableExtra/versions/1.3.4/topics/column_spec}{kableExtra::column\_spec} function, along with the \hyperref[func.RSME.tibble.col_width]{RSME.tibble.col\_width} function described above, are also used internally to set the width of the last column of the table.

For more information about the \emph{RMSEs Result Tibble}, see the \hyperref[func.CreateRMSEs_ResultTibble.details]{Details} section of the \hyperref[func.CreateRMSEs_ResultTibble]{CreateRMSEs\_ResultTibble} function description above.

\phantomsection\label{func.RMSE.Total_kable.value}
\subparagraph{Value}\label{value-27}

\hfill\break

The table returned by the internally called \hyperref[func.RMSE_kable]{RMSE\_kable()}.

\phantomsection\label{func.RMSE.Total_kable.code}
\subparagraph{Source Code}\label{source-code-31}

\hfill\break

The source code of the \href{https://github.com/AzKurban-edX-DS/Capstone-MovieLens/blob/main/r/src/support-functions/common-helper.functions.R\#L54}{RMSE.Total\_kable} function is shown below:

\begin{Shaded}
\begin{Highlighting}[]
\NormalTok{RMSE.Total\_kable }\OtherTok{\textless{}{-}} \ControlFlowTok{function}\NormalTok{(RMSEs,}
                       \AttributeTok{col1width =} \DecValTok{15}\NormalTok{,}
                       \AttributeTok{col2width =} \DecValTok{7}\NormalTok{,}
                       \AttributeTok{col3width =} \DecValTok{5}\NormalTok{, }
                       \AttributeTok{col4width =} \DecValTok{25}\NormalTok{)\{}
\NormalTok{  RMSEs }\SpecialCharTok{|\textgreater{}} 
    \FunctionTok{RMSE.Total\_kable}\NormalTok{(col1width,}
\NormalTok{               col2width,}
\NormalTok{               col3width) }\SpecialCharTok{|\textgreater{}}
    \FunctionTok{column\_spec}\NormalTok{(}\AttributeTok{column =} \DecValTok{4}\NormalTok{, }
                \AttributeTok{width =} \FunctionTok{RSME.tibble.col\_width}\NormalTok{(col4width)) }
\NormalTok{\}}
\end{Highlighting}
\end{Shaded}

\begin{noteblock}
The source code of the \href{https://github.com/AzKurban-edX-DS/Capstone-MovieLens/blob/main/r/src/support-functions/common-helper.functions.R\#L54}{RMSE.Total\_kable} function is also available in the \href{https://github.com/AzKurban-edX-DS/Capstone-MovieLens/blob/main/r/src/support-functions/common-helper.functions.R\#L15}{RMSEs Result Tibble} section of the \href{https://github.com/AzKurban-edX-DS/Capstone-MovieLens/blob/main/r/src/support-functions/common-helper.functions.R\#L54}{common-helper.functions.R} script on \emph{GitHub}.

\end{noteblock}

\newpage

\subsubsection{Model Tuning Utils}\label{appndx_a.CHF.model_tuning_utils}

\hfill\break

\begin{noteblock}
The complete source code of the functions described in this section are available in the \href{https://github.com/AzKurban-edX-DS/Capstone-MovieLens/blob/main/r/src/support-functions/common-helper.functions.R\#L69}{Regularization} section of the \href{https://github.com/AzKurban-edX-DS/Capstone-MovieLens/blob/main/r/src/support-functions/common-helper.functions.R\#L69}{common-helper.functions.R} script on \emph{GitHub}.

\end{noteblock}

\paragraph{\texorpdfstring{\href{https://github.com/AzKurban-edX-DS/Capstone-MovieLens/blob/main/r/src/support-functions/common-helper.functions.R\#L70}{mean\_reg} Function}{mean\_reg Function}}\label{func.mean_reg}

\hfill\break

Calculates the arithmetic mean taking into account the \emph{regularization parameter} \(\lambda\) passed in the \hyperref[func.mean_reg.args.lambda]{lambda} argument. We will call it the \emph{penalized mean}.

\phantomsection\label{func.mean_reg.usage}
\subparagraph{Usage}\label{usage-30}

\hfill\break

\begin{Shaded}
\begin{Highlighting}[]
\NormalTok{  UM.effect }\OtherTok{\textless{}{-}}\NormalTok{ train\_set }\SpecialCharTok{|\textgreater{}}
    \FunctionTok{left\_join}\NormalTok{(edx.user\_effect, }\AttributeTok{by =} \StringTok{"userId"}\NormalTok{) }\SpecialCharTok{|\textgreater{}}
    \FunctionTok{mutate}\NormalTok{(}\AttributeTok{resid =}\NormalTok{ rating }\SpecialCharTok{{-}}\NormalTok{ (mu }\SpecialCharTok{+}\NormalTok{ a)) }\SpecialCharTok{|\textgreater{}} 
    \FunctionTok{group\_by}\NormalTok{(movieId) }\SpecialCharTok{|\textgreater{}}
    \FunctionTok{summarise}\NormalTok{(}\AttributeTok{b =} \FunctionTok{mean\_reg}\NormalTok{(resid, lambda), }\AttributeTok{n =} \FunctionTok{n}\NormalTok{())}
\end{Highlighting}
\end{Shaded}

\begin{noteblock}
In the \href{https://github.com/AzKurban-edX-DS/Capstone-MovieLens/blob/main/r/src/support-functions/UM-effect.functions.R\#L10}{sample code snippet} above, the \texttt{mean\_reg} function is called internally from the \hyperref[func.train_user_movie_effect]{train\_user\_movie\_effect} function described below in section \hyperref[ume-model-utility-functions]{UME Model: Utility Functions} of \emph{this Appendix}.

\end{noteblock}

\phantomsection\label{func.mean_reg.args}
\subparagraph{Arguments}\label{arguments-29}

\hfill\break

\phantomsection\label{func.mean_reg.args.vals}
\begin{itemize}
\tightlist
\item
  \textbf{vals:} Values (usually a numeric vector) to calculate the \emph{penalized mean};
\end{itemize}

\phantomsection\label{func.mean_reg.args.lambda}
\begin{itemize}
\tightlist
\item
  \textbf{lambda:} (Optional, \emph{Numeric}, 0 by default) \emph{Regularization parameter} \(\lambda\).
\end{itemize}

\phantomsection\label{func.mean_reg.args.na.rm}
\begin{itemize}
\tightlist
\item
  \textbf{na.rm:} (Optional, \emph{Boolean}, \texttt{TRUE} by default) Logical value indicating whether \texttt{NA} values should be stripped before the computation proceeds.
\end{itemize}

\newpage

\phantomsection\label{func.mean_reg.details}
\subparagraph{Details}\label{details-28}

\hfill\break

This function is used in the \emph{regularization technique} (described in \href{https://rafalab.dfci.harvard.edu/dsbook-part-2/highdim/regularization.html\#penalized-least-squares}{Section 24.2 \emph{Penalized least squares}} of the \emph{Course Textbook (New Edition)}\autocite{IDS2_24-2}) to calculate the \emph{penalized mean} of the values passed in the \hyperref[func.mean_reg.args.vals]{vals} argument as follows:
\[
\mu(\lambda) = \frac{1}{\lambda + N} \sum_{i=1}^{N} x_i
\]

\begin{noteblock}
This is a helper function intended for use in other helper functions (such as, for instance, \hyperref[func.train_user_movie_effect]{train\_user\_movie\_effect} as shown above in the \hyperref[func.mean_reg.usage]{Usage} section of \emph{this Description}) that compute \emph{penalized estimates} during the \emph{regularization} process of the models being trained\autocite{IDS2_24-2}. All these functions are described below in section \hyperref[models-training-support-functions]{Models Training: Support Functions} of \emph{this Appendix}.

\end{noteblock}

When the \hyperref[func.mean_reg.args.lambda]{lambda} parameter is \texttt{0} (meaning the \(\lambda = 0\)), the function calculates the simple arithmetic mean:
\[
\mu = \frac{1}{N} \sum_{i=1}^{N} x_i
\]

\begin{noteblock}
If the \hyperref[func.mean_reg.args.lambda]{lambda} parameter is \texttt{0} (the default), the function is equivalent to the standard R function \href{https://www.rdocumentation.org/packages/base/versions/3.6.2/topics/mean}{base::mean}, calling with the parameter \texttt{trim\ =\ 0} (the default for the \texttt{base::mean} function).

\end{noteblock}

\phantomsection\label{func.mean_reg.value}
\subparagraph{Value}\label{value-28}

\hfill\break

(Numeric) \emph{Penalized mean} of the values passed in the \hyperref[func.mean_reg.args.vals]{vals} argument.

\phantomsection\label{func.mean_reg.code}
\subparagraph{Source Code}\label{source-code-32}

\hfill\break

The source code of the \href{https://github.com/AzKurban-edX-DS/Capstone-MovieLens/blob/main/r/src/support-functions/common-helper.functions.R\#L70}{mean\_reg} function is shown below:

\begin{Shaded}
\begin{Highlighting}[]
\NormalTok{mean\_reg }\OtherTok{\textless{}{-}} \ControlFlowTok{function}\NormalTok{(vals, }\AttributeTok{lambda =} \DecValTok{0}\NormalTok{, }\AttributeTok{na.rm =} \ConstantTok{TRUE}\NormalTok{)\{}
  \ControlFlowTok{if}\NormalTok{ (}\FunctionTok{is.na}\NormalTok{(lambda)) \{}
    \FunctionTok{stop}\NormalTok{(}\StringTok{"Function: mean\_reg}
\StringTok{\textasciigrave{}lambda\textasciigrave{} is \textasciigrave{}NA\textasciigrave{}"}\NormalTok{)}
\NormalTok{  \}}
  
  \FunctionTok{names}\NormalTok{(lambda) }\OtherTok{\textless{}{-}} \ConstantTok{NULL}
\NormalTok{  sums }\OtherTok{\textless{}{-}} \FunctionTok{sum}\NormalTok{(vals, }\AttributeTok{na.rm =}\NormalTok{ na.rm)}
\NormalTok{  N }\OtherTok{\textless{}{-}} \FunctionTok{ifelse}\NormalTok{(na.rm, }\FunctionTok{sum}\NormalTok{(}\SpecialCharTok{!}\FunctionTok{is.na}\NormalTok{(vals)), }\FunctionTok{length}\NormalTok{(vals))}
\NormalTok{  sums}\SpecialCharTok{/}\NormalTok{(N }\SpecialCharTok{+}\NormalTok{ lambda)}
\NormalTok{\}}
\end{Highlighting}
\end{Shaded}

\begin{noteblock}
The source code of the \href{https://github.com/AzKurban-edX-DS/Capstone-MovieLens/blob/main/r/src/support-functions/common-helper.functions.R\#L70}{mean\_reg} is also available in the \href{https://github.com/AzKurban-edX-DS/Capstone-MovieLens/blob/main/r/src/support-functions/common-helper.functions.R\#L69}{Regularization} section of the \href{https://github.com/AzKurban-edX-DS/Capstone-MovieLens/blob/main/r/src/support-functions/common-helper.functions.R\#L70}{common-helper.functions.R} script on \emph{GitHub}.

\end{noteblock}

\newpage

\paragraph{\texorpdfstring{\href{https://github.com/AzKurban-edX-DS/Capstone-MovieLens/blob/main/r/src/support-functions/common-helper.functions.R\#L81}{get\_fine\_tune.param.endpoints.idx} Function}{get\_fine\_tune.param.endpoints.idx Function}}\label{func.get_fine_tune.param.endpoints.idx}

\hfill\break

Implements an algorithm that finds in the \emph{RMSE} column of the data frame passed in the \hyperref[func.get_fine_tune.param.endpoints.idx.args.preset.result]{preset.result} argument the indices of the pair of the nearest neighboring elements with values greater than the minimum to the left and right of the element (or group of elements) with the minimum value.

\begin{noteblock}
The algorithm assumes that the set of values in the \emph{RMSE} column is a monotonically non-increasing function of the index until the minimum is reached and monotonically non-decreasing thereafter. That is, it is assumed that the function has only one minimum on the given set.

\end{noteblock}

\phantomsection\label{func.get_fine_tune.param.endpoints.idx.usage}
\subparagraph{Usage}\label{usage-31}

\hfill\break

The following \href{https://github.com/AzKurban-edX-DS/Capstone-MovieLens/blob/main/r/src/capstone-movielens.main.R\#L571}{code snippet} creates the object \texttt{cv.UME.preset.result} as a result of the \hyperref[func.tune.model_param]{tune.model\_param} function call, which we can subsequently use to extract a value to pass in the \hyperref[func.get_fine_tune.param.endpoints.idx.args.preset.result]{preset.result} argument to the function \emph{being described here}:

\begin{Shaded}
\begin{Highlighting}[]
\NormalTok{lambdas }\OtherTok{\textless{}{-}} \FunctionTok{seq}\NormalTok{(}\DecValTok{0}\NormalTok{, }\DecValTok{1}\NormalTok{, }\FloatTok{0.1}\NormalTok{)}
\NormalTok{cv.UME.preset.result }\OtherTok{\textless{}{-}} 
  \FunctionTok{tune.model\_param}\NormalTok{(lambdas, regularize.test\_lambda.UM\_effect.cv)}
\end{Highlighting}
\end{Shaded}

\begin{Shaded}
\begin{Highlighting}[]
\FunctionTok{str}\NormalTok{(cv.UME.preset.result)}
\end{Highlighting}
\end{Shaded}

\begin{verbatim}
## List of 2
##  $ tuned.result:'data.frame':    8 obs. of  2 variables:
##   ..$ RMSE           : num [1:8] 0.873 0.873 0.873 0.873 0.873 ...
##   ..$ parameter.value: num [1:8] 0 0.1 0.2 0.3 0.4 0.5 0.6 0.7
##  $ best_result : Named num [1:2] 0.4 0.873
##   ..- attr(*, "names")= chr [1:2] "param.best_value" "best_RMSE"
\end{verbatim}

\begin{noteblock}
The \emph{custom helper function} \hyperref[func.tune.model_param]{tune.model\_param} used in the \emph{code snippet above} is described below in \emph{this Section}.

\end{noteblock}

The following code snippet provides an example of passing the data contained in the object created in the \emph{code snippet above} to the \hyperref[func.get_fine_tune.param.endpoints.idx.args.preset.result]{preset.result} argument of the function \emph{being described}:

\begin{Shaded}
\begin{Highlighting}[]
\NormalTok{  preset.result }\OtherTok{\textless{}{-}}\NormalTok{ cv.UME.preset.result}\SpecialCharTok{$}\NormalTok{tuned.result}
  \FunctionTok{print}\NormalTok{(preset.result)}
\end{Highlighting}
\end{Shaded}

\begin{verbatim}
##        RMSE parameter.value
## 1 0.8732081             0.0
## 2 0.8732069             0.1
## 3 0.8732062             0.2
## 4 0.8732058             0.3
## 5 0.8732057             0.4
## 6 0.8732058             0.5
## 7 0.8732062             0.6
## 8 0.8732067             0.7
\end{verbatim}

\begin{Shaded}
\begin{Highlighting}[]
  \CommentTok{\# Internal call from the user{-}defined function \textasciigrave{}get\_fine\_tune.param.endpoints\textasciigrave{} }
\NormalTok{  preset.result.idx }\OtherTok{\textless{}{-}} \FunctionTok{get\_fine\_tune.param.endpoints.idx}\NormalTok{(preset.result)}
\NormalTok{  preset.result.idx}
\end{Highlighting}
\end{Shaded}

\begin{verbatim}
## start   end  best 
##     4     6     5
\end{verbatim}

\begin{noteblock}
In the above \href{https://github.com/AzKurban-edX-DS/Capstone-MovieLens/blob/main/r/src/support-functions/common-helper.functions.R\#L111}{code snippet}, the \texttt{get\_fine\_tune.param.endpoints.idx} function is called internally from the \hyperref[func.get_fine_tune.param.endpoints]{get\_fine\_tune.param.endpoints} function described below in \emph{this Section}.

\end{noteblock}

\phantomsection\label{func.get_fine_tune.param.endpoints.idx.args}
\subparagraph{Arguments}\label{arguments-30}

\hfill\break

\phantomsection\label{func.get_fine_tune.param.endpoints.idx.args.preset.result}
\begin{itemize}
\tightlist
\item
  \textbf{preset.result:} Data frame that has an \emph{RMSE} column containing the \texttt{RMSE} values corresponding to the \emph{regularization parameter} \(\lambda\) values for the model being regularized, as described in section \hyperref[func.tune.model_param.details]{Details} of the \hyperref[func.tune.model_param]{tune.model\_param} function description below.
\end{itemize}

\phantomsection\label{func.get_fine_tune.param.endpoints.idx.details}
\subparagraph{Details}\label{details-29}

\hfill\break

This is an auxiliary function intended to be called internally from the other user-defined functions, such as the \hyperref[func.get_fine_tune.param.endpoints]{get\_fine\_tune.param.endpoints} (as shown above in the \hyperref[func.get_fine_tune.param.endpoints.idx.usage]{Usage} section of \emph{this Description}) and \hyperref[func.model.tune.param_range]{model.tune.param\_range} (both described below in \emph{this Section}).

\begin{noteblock}
The element stored in the \texttt{\$tuned.result} variable of the \emph{list} object returned by the \hyperref[func.tune.model_param]{tune.model\_param} function (described below \emph{in this Section}), is a \emph{data frame} suitable for use as the value of the \hyperref[func.get_fine_tune.param.endpoints.idx.args.preset.result]{preset.result} argument.

\end{noteblock}

\phantomsection\label{func.get_fine_tune.param.endpoints.idx.value}
\subparagraph{Value}\label{value-29}

\hfill\break

Vector containing the index of the first element corresponding to the minimum \texttt{RMSE} value in the \emph{RMSE} column of the data frame passed in the \hyperref[func.get_fine_tune.param.endpoints.idx.args.preset.result]{preset.result} argument, along with the indices of the pair of the nearest neighboring elements with values greater than the minimum to the left and right of the element (or group of elements) with the minimum value:

\begin{Shaded}
\begin{Highlighting}[]
  \FunctionTok{c}\NormalTok{(start, }\CommentTok{\# interval start index}
\NormalTok{    end,   }\CommentTok{\# interval end index}
\NormalTok{    best)  }\CommentTok{\# index of the best \textasciigrave{}RMSE\textasciigrave{}}
\end{Highlighting}
\end{Shaded}

The \hyperref[func.get_fine_tune.param.endpoints.idx.usage]{Usage} section of \emph{this Description} above provides an example of the value returned by the function \emph{being described}.

\newpage

\phantomsection\label{func.get_fine_tune.param.endpoints.idx.code}
\subparagraph{Source Code}\label{source-code-33}

\hfill\break

The source code of the \href{https://github.com/AzKurban-edX-DS/Capstone-MovieLens/blob/main/r/src/support-functions/common-helper.functions.R\#L81}{get\_fine\_tune.param.endpoints.idx} function is shown below:

\begin{Shaded}
\begin{Highlighting}[]
\NormalTok{get\_fine\_tune.param.endpoints.idx }\OtherTok{\textless{}{-}} \ControlFlowTok{function}\NormalTok{(preset.result) \{}
\NormalTok{  best\_RMSE }\OtherTok{\textless{}{-}} \FunctionTok{min}\NormalTok{(preset.result}\SpecialCharTok{$}\NormalTok{RMSE)}
\NormalTok{  best\_RMSE.idx }\OtherTok{\textless{}{-}} \FunctionTok{which.min}\NormalTok{(preset.result}\SpecialCharTok{$}\NormalTok{RMSE)}
  \CommentTok{\# best\_lambda \textless{}{-} preset.result$parameter.value[best\_RMSE.idx]}
  
\NormalTok{  preset.result.N }\OtherTok{\textless{}{-}} \FunctionTok{length}\NormalTok{(preset.result}\SpecialCharTok{$}\NormalTok{RMSE)}
\NormalTok{  i }\OtherTok{\textless{}{-}}\NormalTok{ best\_RMSE.idx}
\NormalTok{  j }\OtherTok{\textless{}{-}}\NormalTok{ i}
  
  \ControlFlowTok{while}\NormalTok{ (i }\SpecialCharTok{\textgreater{}} \DecValTok{1}\NormalTok{) \{}
\NormalTok{    i }\OtherTok{\textless{}{-}}\NormalTok{ i }\SpecialCharTok{{-}} \DecValTok{1}
    
    \ControlFlowTok{if}\NormalTok{ (preset.result}\SpecialCharTok{$}\NormalTok{RMSE[i] }\SpecialCharTok{\textgreater{}}\NormalTok{ best\_RMSE) \{}
      \ControlFlowTok{break}
\NormalTok{    \}}
\NormalTok{  \}}
  
  \ControlFlowTok{while}\NormalTok{ (j }\SpecialCharTok{\textless{}}\NormalTok{ preset.result.N) \{}
\NormalTok{    j }\OtherTok{\textless{}{-}}\NormalTok{ j }\SpecialCharTok{+} \DecValTok{1}
    
    \ControlFlowTok{if}\NormalTok{ (preset.result}\SpecialCharTok{$}\NormalTok{RMSE[j] }\SpecialCharTok{\textgreater{}}\NormalTok{ best\_RMSE) \{}
      \ControlFlowTok{break}
\NormalTok{    \}}
\NormalTok{  \}}
  
  \FunctionTok{c}\NormalTok{(}\AttributeTok{start =}\NormalTok{ i, }
    \AttributeTok{end =}\NormalTok{ j,}
    \AttributeTok{best =}\NormalTok{ best\_RMSE.idx)}
\NormalTok{\}}
\end{Highlighting}
\end{Shaded}

\begin{noteblock}
The source code of the \href{https://github.com/AzKurban-edX-DS/Capstone-MovieLens/blob/main/r/src/support-functions/common-helper.functions.R\#L81}{get\_fine\_tune.param.endpoints.idx} function is also available in the \href{https://github.com/AzKurban-edX-DS/Capstone-MovieLens/blob/main/r/src/support-functions/common-helper.functions.R\#L69}{Regularization} section of the \href{https://github.com/AzKurban-edX-DS/Capstone-MovieLens/blob/main/r/src/support-functions/common-helper.functions.R\#L81}{common-helper.functions.R} script on \emph{GitHub}.

\end{noteblock}

\newpage

\paragraph{\texorpdfstring{\href{https://github.com/AzKurban-edX-DS/Capstone-MovieLens/blob/main/r/src/support-functions/common-helper.functions.R\#L109}{get\_fine\_tune.param.endpoints} Function}{get\_fine\_tune.param.endpoints Function}}\label{func.get_fine_tune.param.endpoints}

\hfill\break

Implements an algorithm that finds in the \emph{RMSE} column of the data frame passed in the \hyperref[func.get_fine_tune.param.endpoints.args.preset.result]{preset.result} argument the pair of the nearest neighboring elements with values greater than the minimum to the left and right of the element (or group of elements) with the minimum value.

\begin{noteblock}
The algorithm assumes that the set of values in the \emph{RMSE} column is a monotonically non-increasing function of the index until the minimum is reached and monotonically non-decreasing thereafter. That is, it is assumed that the function has only one minimum on the given set.

\end{noteblock}

\phantomsection\label{func.get_fine_tune.param.endpoints.usage}
\subparagraph{Usage}\label{usage-32}

\hfill\break

The following \href{https://github.com/AzKurban-edX-DS/Capstone-MovieLens/blob/main/r/src/capstone-movielens.main.R\#L571}{code snippet} creates the object \texttt{cv.UME.preset.result} as a result of the \hyperref[func.tune.model_param]{tune.model\_param} function call, which we can subsequently use to extract a value to pass in the \hyperref[func.get_fine_tune.param.endpoints.args.preset.result]{preset.result} argument to the function \emph{being described here}:

\begin{Shaded}
\begin{Highlighting}[]
\NormalTok{lambdas }\OtherTok{\textless{}{-}} \FunctionTok{seq}\NormalTok{(}\DecValTok{0}\NormalTok{, }\DecValTok{1}\NormalTok{, }\FloatTok{0.1}\NormalTok{)}
\NormalTok{cv.UME.preset.result }\OtherTok{\textless{}{-}} 
  \FunctionTok{tune.model\_param}\NormalTok{(lambdas, regularize.test\_lambda.UM\_effect.cv)}
\end{Highlighting}
\end{Shaded}

\begin{Shaded}
\begin{Highlighting}[]
\FunctionTok{str}\NormalTok{(cv.UME.preset.result)}
\end{Highlighting}
\end{Shaded}

\begin{verbatim}
## List of 2
##  $ tuned.result:'data.frame':    8 obs. of  2 variables:
##   ..$ RMSE           : num [1:8] 0.873 0.873 0.873 0.873 0.873 ...
##   ..$ parameter.value: num [1:8] 0 0.1 0.2 0.3 0.4 0.5 0.6 0.7
##  $ best_result : Named num [1:2] 0.4 0.873
##   ..- attr(*, "names")= chr [1:2] "param.best_value" "best_RMSE"
\end{verbatim}

\begin{Shaded}
\begin{Highlighting}[]
\CommentTok{\# Object to be passed to the \textasciigrave{}preset.result\textasciigrave{} argument of the \textasciigrave{}get\_fine\_tune.param.endpoints\textasciigrave{} function:}
\FunctionTok{print}\NormalTok{(cv.UME.preset.result}\SpecialCharTok{$}\NormalTok{tuned.result)}
\end{Highlighting}
\end{Shaded}

\begin{verbatim}
##        RMSE parameter.value
## 1 0.8732081             0.0
## 2 0.8732069             0.1
## 3 0.8732062             0.2
## 4 0.8732058             0.3
## 5 0.8732057             0.4
## 6 0.8732058             0.5
## 7 0.8732062             0.6
## 8 0.8732067             0.7
\end{verbatim}

\begin{noteblock}
The \emph{custom helper function} \hyperref[func.tune.model_param]{tune.model\_param} used in the \emph{code snippet above} is described below in \emph{this Section}.

\end{noteblock}

\newpage

In the following \href{https://github.com/AzKurban-edX-DS/Capstone-MovieLens/blob/main/r/src/capstone-movielens.main.R\#L608}{code snippet}, the data stored in the \texttt{\$tuned.result} variable of the \emph{list} object returned by the \hyperref[func.tune.model_param]{tune.model\_param} function in the \emph{code snippet above}, is passed in the \hyperref[func.get_fine_tune.param.endpoints.args.preset.result]{preset.result} argument to the function \emph{being described}:

\begin{Shaded}
\begin{Highlighting}[]
\NormalTok{endpoints }\OtherTok{\textless{}{-}} 
  \FunctionTok{get\_fine\_tune.param.endpoints}\NormalTok{(cv.UME.preset.result}\SpecialCharTok{$}\NormalTok{tuned.result)}

\NormalTok{endpoints}
\end{Highlighting}
\end{Shaded}

\begin{verbatim}
## start   end  best 
##   0.3   0.5   0.4
\end{verbatim}

\phantomsection\label{func.get_fine_tune.param.endpoints.args}
\subparagraph{Arguments}\label{arguments-31}

\hfill\break

\phantomsection\label{func.get_fine_tune.param.endpoints.args.preset.result}
\begin{itemize}
\tightlist
\item
  \textbf{preset.result:} Data frame that has an \emph{RMSE} column containing the \texttt{RMSE} values corresponding to the \emph{regularization parameter} \(\lambda\) values for the model being regularized, as described in section \hyperref[func.tune.model_param.details]{Details} of the \hyperref[func.tune.model_param]{tune.model\_param} function description below.
\end{itemize}

\phantomsection\label{func.get_fine_tune.param.endpoints.details}
\subparagraph{Details}\label{details-30}

\hfill\break

This function internally calls \hyperref[func.get_fine_tune.param.endpoints.idx]{get\_fine\_tune.param.endpoints.idx} function described above in \emph{this Section} to figure out the indices of the elements, the values of which have to be determined.

\begin{noteblock}
The element stored in the \texttt{\$tuned.result} variable of the \emph{list} object returned by the \hyperref[func.tune.model_param]{tune.model\_param} function (described below \emph{in this Section}), is a \emph{data frame} suitable for use as the value of the \hyperref[func.get_fine_tune.param.endpoints.args.preset.result]{preset.result} argument.

\end{noteblock}

\phantomsection\label{func.get_fine_tune.param.endpoints.value}
\subparagraph{Value}\label{value-30}

\hfill\break

Vector containing the \emph{regularization parameter} \(\lambda\) value corresponding to the first element with the minimum \texttt{RMSE} value in the \emph{RMSE} column of the data frame passed in the \hyperref[func.get_fine_tune.param.endpoints.args.preset.result]{preset.result} argument, along with the pair of the \(\lambda\) values corresponding to the nearest neighboring elements with values greater than the minimum \texttt{RMSE} to the left and right of the element (or group of elements) with the minimum \texttt{RMSE} value:

The \hyperref[func.get_fine_tune.param.endpoints.usage]{Usage} section of \emph{this Description} above provides an example of the value returned by the function \emph{being described}.

\newpage

\phantomsection\label{func.get_fine_tune.param.endpoints.code}
\subparagraph{Source Code}\label{source-code-34}

\hfill\break

The source code of the \href{https://github.com/AzKurban-edX-DS/Capstone-MovieLens/blob/main/r/src/support-functions/common-helper.functions.R\#L109}{get\_fine\_tune.param.endpoints} function is shown below:

\begin{Shaded}
\begin{Highlighting}[]
\NormalTok{get\_fine\_tune.param.endpoints }\OtherTok{\textless{}{-}} \ControlFlowTok{function}\NormalTok{(preset.result) \{}

\NormalTok{  preset.result.idx }\OtherTok{\textless{}{-}} \FunctionTok{get\_fine\_tune.param.endpoints}\NormalTok{(preset.result)}
  
\NormalTok{  i }\OtherTok{\textless{}{-}}\NormalTok{ preset.result.idx[}\StringTok{"start"}\NormalTok{]}
\NormalTok{  j }\OtherTok{\textless{}{-}}\NormalTok{ preset.result.idx[}\StringTok{"end"}\NormalTok{]}
\NormalTok{  best.idx }\OtherTok{\textless{}{-}}\NormalTok{ preset.result.idx[}\StringTok{"best"}\NormalTok{]}
  
  \FunctionTok{c}\NormalTok{(}\AttributeTok{start =}\NormalTok{ preset.result}\SpecialCharTok{$}\NormalTok{parameter.value[i], }
    \AttributeTok{end =}\NormalTok{ preset.result}\SpecialCharTok{$}\NormalTok{parameter.value[j],}
    \AttributeTok{best =}\NormalTok{ preset.result}\SpecialCharTok{$}\NormalTok{parameter.value[best.idx])}
\NormalTok{\}}
\end{Highlighting}
\end{Shaded}

\begin{noteblock}
The source code of the \href{https://github.com/AzKurban-edX-DS/Capstone-MovieLens/blob/main/r/src/support-functions/common-helper.functions.R\#L109}{get\_fine\_tune.param.endpoints} function is also available in the \href{https://github.com/AzKurban-edX-DS/Capstone-MovieLens/blob/main/r/src/support-functions/common-helper.functions.R\#L69}{Regularization} section of the \href{https://github.com/AzKurban-edX-DS/Capstone-MovieLens/blob/main/r/src/support-functions/common-helper.functions.R\#L109}{common-helper.functions.R} script on \emph{GitHub}.

\end{noteblock}

\newpage

\paragraph{\texorpdfstring{\href{https://github.com/AzKurban-edX-DS/Capstone-MovieLens/blob/main/r/src/support-functions/common-helper.functions.R\#L121}{get\_best\_param.result} Function}{get\_best\_param.result Function}}\label{func.get_best_param.result}

\hfill\break

Selects the best values from a pair of matched vectors: a vector of sorted \emph{tuning parameter} values and a vector of corresponding \emph{Root Mean Squared Errors (RMSE)} values passed in the \hyperref[func.get_best_param.result.args.param_values]{param\_values} and \hyperref[func.get_best_param.result.args.RMSEs]{RMSEs} arguments, respectively.

\phantomsection\label{func.get_best_param.result.usage}
\subparagraph{Usage}\label{usage-33}

\hfill\break

In the following \href{https://github.com/AzKurban-edX-DS/Capstone-MovieLens/blob/main/r/src/support-functions/common-helper.functions.R\#L413}{code snippet}, the \texttt{get\_best\_param.result} function is called internally from the \hyperref[func.model.tune.param_range]{model.tune.param\_range} function described below in \emph{this Section}:

\begin{Shaded}
\begin{Highlighting}[]
\NormalTok{    param\_values.best\_result }\OtherTok{\textless{}{-}} 
      \FunctionTok{get\_best\_param.result}\NormalTok{(tuned.result}\SpecialCharTok{$}\NormalTok{parameter.value, }
\NormalTok{                          tuned.result}\SpecialCharTok{$}\NormalTok{RMSE)}
    
    \FunctionTok{put\_log2}\NormalTok{(}\StringTok{"Function \textasciigrave{}model.tune.param\_range\textasciigrave{}:}
\StringTok{Currently reached best RMSE for \textasciigrave{}parameter value = \%1\textasciigrave{}: \%2"}\NormalTok{,}
\NormalTok{             param\_values.best\_result[}\StringTok{"param.best\_value"}\NormalTok{],}
\NormalTok{             param\_values.best\_result[}\StringTok{"best\_RMSE"}\NormalTok{])}
\end{Highlighting}
\end{Shaded}

\begin{noteblock}
In the code snippet above, we use the custom function \hyperref[func.put_log2]{put\_log2} described in section \hyperref[appndx_a.CHF.logging-funcs]{Logging Functions} of \emph{this Appendix}.

\end{noteblock}

\phantomsection\label{func.get_best_param.result.args}
\subparagraph{Arguments}\label{arguments-32}

\hfill\break

\phantomsection\label{func.get_best_param.result.args.param_values}
\begin{itemize}
\tightlist
\item
  \textbf{param\_values:} \emph{Numeric vector} containing a sorted list of the \emph{tuning parameter} values to search for the best one during the \emph{Model Regularization} process;
\end{itemize}

\phantomsection\label{func.get_best_param.result.args.RMSEs}
\begin{itemize}
\tightlist
\item
  \textbf{RMSEs:} \emph{Numeric vector} containing a list of \emph{RMSE} values computed during the \emph{Model Regularization} process.
\end{itemize}

\phantomsection\label{func.get_best_param.result.details}
\subparagraph{Details}\label{details-31}

\hfill\break

This is a \emph{auxiliary function} used to select the \emph{best values} during the \emph{Model Parameter Tuning} process (including the \emph{Model Regularization} procedure that utilizes the \emph{penalized approach} to regularize predictions as explained in \href{https://rafalab.dfci.harvard.edu/dsbook-part-2/highdim/regularization.html\#penalized-least-squares}{Section 24.2 \emph{Penalized least squares}} of the \emph{Course Textbook (New Edition)}\autocite{IDS2_24-2}).

With this approach, the best value of the \emph{tuning parameter} is considered to be the value corresponding to the minimum value of the \emph{RMSE} for the model being tested.

\begin{noteblock}
In \emph{This Project}, this function is used internally in the \hyperref[func.model.tune.param_range]{model.tune.param\_range} custom helper function (as shown above in the \hyperref[func.get_best_param.result.usage]{Usage} section of \emph{this Description}) described below in \emph{this Section}.

\end{noteblock}

\phantomsection\label{func.get_best_param.result.value}
\subparagraph{Value}\label{value-31}

\hfill\break

\emph{Numeric vector} containing a pair of values: the \emph{best tuning parameter} value and corresponding \emph{minimal RMSE}.

\phantomsection\label{func.get_best_param.result.code}
\subparagraph{Source Code}\label{source-code-35}

\hfill\break

The source code of the \href{https://github.com/AzKurban-edX-DS/Capstone-MovieLens/blob/main/r/src/support-functions/common-helper.functions.R\#L121}{get\_best\_param.result} function is shown below:

\begin{Shaded}
\begin{Highlighting}[]
\NormalTok{get\_best\_param.result }\OtherTok{\textless{}{-}} \ControlFlowTok{function}\NormalTok{(param\_values, RMSEs)\{}
\NormalTok{  best\_pvalue\_idx }\OtherTok{\textless{}{-}} \FunctionTok{which.min}\NormalTok{(RMSEs)}
  \FunctionTok{c}\NormalTok{(}\AttributeTok{param.best\_value =}\NormalTok{ param\_values[best\_pvalue\_idx], }
    \AttributeTok{best\_RMSE =}\NormalTok{ RMSEs[best\_pvalue\_idx])}
\NormalTok{\}}
\end{Highlighting}
\end{Shaded}

\begin{noteblock}
The source code of the \href{https://github.com/AzKurban-edX-DS/Capstone-MovieLens/blob/main/r/src/support-functions/common-helper.functions.R\#L121}{get\_best\_param.result} function is also available in the \href{https://github.com/AzKurban-edX-DS/Capstone-MovieLens/blob/main/r/src/support-functions/common-helper.functions.R\#L69}{Regularization} section of the \href{https://github.com/AzKurban-edX-DS/Capstone-MovieLens/blob/main/r/src/support-functions/common-helper.functions.R\#L121}{common-helper.functions.R} script on \emph{GitHub}.

\end{noteblock}

\newpage

\paragraph{\texorpdfstring{\href{https://github.com/AzKurban-edX-DS/Capstone-MovieLens/blob/main/r/src/support-functions/common-helper.functions.R\#L126}{tune.model\_param} Function}{tune.model\_param Function}}\label{func.tune.model_param}

\hfill\break

Implements an algorithm for selecting the best value of the \emph{tuning parameter} from the sorted \emph{parameter vector} passed in the \hyperref[func.tune.model_param.args.param_values]{param\_values} argument.

\phantomsection\label{func.tune.model_param.usage}
\subparagraph{\texorpdfstring{\href{https://github.com/AzKurban-edX-DS/Capstone-MovieLens/blob/main/r/src/capstone-movielens.main.R\#L571}{Usage}}{Usage}}\label{usage-34}

\hfill\break

\begin{Shaded}
\begin{Highlighting}[]
\NormalTok{  lambdas }\OtherTok{\textless{}{-}} \FunctionTok{seq}\NormalTok{(}\DecValTok{0}\NormalTok{, }\DecValTok{1}\NormalTok{, }\FloatTok{0.1}\NormalTok{)}
\NormalTok{  cv.UME.preset.result }\OtherTok{\textless{}{-}} 
    \FunctionTok{tune.model\_param}\NormalTok{(lambdas, regularize.test\_lambda.UM\_effect.cv)}
\end{Highlighting}
\end{Shaded}

\begin{Shaded}
\begin{Highlighting}[]
\FunctionTok{print}\NormalTok{(cv.UME.preset.result)}
\end{Highlighting}
\end{Shaded}

\begin{verbatim}
## $tuned.result
##        RMSE parameter.value
## 1 0.8732081             0.0
## 2 0.8732069             0.1
## 3 0.8732062             0.2
## 4 0.8732058             0.3
## 5 0.8732057             0.4
## 6 0.8732058             0.5
## 7 0.8732062             0.6
## 8 0.8732067             0.7
## 
## $best_result
## param.best_value        best_RMSE 
##        0.4000000        0.8732057
\end{verbatim}

\begin{noteblock}
In the \emph{code snippet} above, the \texttt{tune.model\_param} function accepts a pointer to the auxiliary function \hyperref[func.regularize.test_lambda.UM_effect.cv]{regularize.test\_lambda.UM\_effect.cv} (described below in section \hyperref[ume-model-regularization-1]{UME Model: Regularization} of \emph{this Appendix}), as a value of an argument \hyperref[func.tune.model_param.args.fn_tune.test.param_value]{fn\_tune.test.param\_value}.

\end{noteblock}

\phantomsection\label{func.tune.model_param.args}
\subparagraph{Arguments}\label{arguments-33}

\hfill\break

\phantomsection\label{func.tune.model_param.args.param_values}
\begin{itemize}
\tightlist
\item
  \textbf{param\_values:} Numeric vector containing a sorted list of parameter values to search for the best one during the model regularization process;
\end{itemize}

\phantomsection\label{func.tune.model_param.args.fn_tune.test.param_value}
\begin{itemize}
\tightlist
\item
  \textbf{fn\_tune.test.param\_value:} Model-specific helper function that validates the model by computing the model's \emph{RMSE} value for a given parameter value;
\end{itemize}

\phantomsection\label{func.tune.model_param.args.break.if_min}
\begin{itemize}
\tightlist
\item
  \textbf{break.if\_min:} (Optional, \texttt{TRUE} by default) Boolean value that determines whether the algorithm should terminate after completing the number of steps specified by the \hyperref[func.tune.model_param.args.steps.beyond_min]{steps.beyond\_min} argument from the moment the minimum \texttt{RMSE} value is found;
\end{itemize}

\phantomsection\label{func.tune.model_param.args.steps.beyond_min}
\begin{itemize}
\tightlist
\item
  \textbf{steps.beyond\_min:} (Optional, 2 by default; only takes effect if \hyperref[func.tune.model_param.args.break.if_min]{break.if\_min} is \texttt{TRUE}) Specifies the number of steps after finding the minimum \emph{RMSE} value after which the algorithm should terminate.
\end{itemize}

\phantomsection\label{func.tune.model_param.details}
\subparagraph{Details}\label{details-32}

\hfill\break

With the approach used in this algorithm, the best value of the \emph{tuning parameter} is considered to be the value corresponding to the minimum value of the \emph{Root Mean Square Error (RMSE)} for the model being tested.

To find the best value, the algorithm iterates over the values from the list passed in the \hyperref[func.tune.model_param.args.param_values]{param\_values} argument, computing the \emph{RMSE} of the model for each of them, using the helper function specified in the \hyperref[func.tune.model_param.args.fn_tune.test.param_value]{fn\_tune.test.param\_value} argument.

The algorithm continues iterating until the end of the parameter values list if \hyperref[func.tune.model_param.args.break.if_min]{break.if\_min} is \texttt{FALSE}.

Otherwise, the algorithm terminates after the number of cycles specified by the argument \hyperref[func.tune.model_param.args.steps.beyond_min]{steps.beyond\_min} has completed since the minimum \emph{RMSE} value was reached.

\begin{noteblock}
The algorithm assumes that the set of the \emph{RMSE} values is a monotonically non-increasing function of the \emph{tuning parameter} until the minimum is reached and monotonically non-decreasing thereafter. That is, it is assumed that the function can have only one minimum on the given set.

\end{noteblock}

\phantomsection\label{func.tune.model_param.value}
\subparagraph{Value}\label{value-32}

\hfill\break

This function returns a list of two objects:

\begin{Shaded}
\begin{Highlighting}[]
  \FunctionTok{list}\NormalTok{(}\AttributeTok{tuned.result =} \FunctionTok{data.frame}\NormalTok{(}\AttributeTok{RMSE =}\NormalTok{ RMSEs\_tmp,}
                                \AttributeTok{parameter.value =}\NormalTok{ param\_vals\_tmp),}
       \AttributeTok{best\_result =}\NormalTok{ param\_values.best\_result)}
\end{Highlighting}
\end{Shaded}

The objects comprising the return list are as follows:

\begin{itemize}
\item
  \texttt{best\_result}: Vector containing a \emph{pair of values}: the found \emph{best value of the tuning parameter} for which the \emph{RMSE} value is minimal, along with the minimal \emph{RMSE} value itself;
\item
  \texttt{tuned.result}: Data frame containing two variables:

  \begin{itemize}
  \item
    \emph{parameter.value}: A vector containing all processed values of the \emph{tuning parameter};
  \item
    \emph{RMSE}: A vector containing the \emph{RMSE} values corresponding to the processed values of the \emph{tuning parameter}.
  \end{itemize}
\end{itemize}

\newpage

\subparagraph{Source Code}\label{source-code-36}

\hfill\break
Below is the simplified version of the source code of the \href{https://github.com/AzKurban-edX-DS/Capstone-MovieLens/blob/main/r/src/support-functions/common-helper.functions.R\#L126}{tune.model\_param} function:

\begin{Shaded}
\begin{Highlighting}[]
\NormalTok{tune.model\_param }\OtherTok{\textless{}{-}} \ControlFlowTok{function}\NormalTok{(param\_values, }
\NormalTok{                             fn\_tune.test.param\_value, }
                             \AttributeTok{break.if\_min =} \ConstantTok{TRUE}\NormalTok{,}
                             \AttributeTok{steps.beyond\_min =} \DecValTok{2}\NormalTok{)\{}
\NormalTok{  n }\OtherTok{\textless{}{-}} \FunctionTok{length}\NormalTok{(param\_values)}
\NormalTok{  param\_vals\_tmp }\OtherTok{\textless{}{-}} \FunctionTok{numeric}\NormalTok{()}
\NormalTok{  RMSEs\_tmp }\OtherTok{\textless{}{-}} \FunctionTok{numeric}\NormalTok{()}
\NormalTok{  RMSE\_min }\OtherTok{\textless{}{-}} \ConstantTok{Inf}
\NormalTok{  i\_max.beyond\_RMSE\_min }\OtherTok{\textless{}{-}} \ConstantTok{Inf}
\NormalTok{  prm\_val.best }\OtherTok{\textless{}{-}} \ConstantTok{NA}
  
  \CommentTok{\# ...}
  
  \ControlFlowTok{for}\NormalTok{ (i }\ControlFlowTok{in} \DecValTok{1}\SpecialCharTok{:}\NormalTok{n) \{}
    \FunctionTok{put\_log1}\NormalTok{(}\StringTok{"Function: \textasciigrave{}tune.model\_param\textasciigrave{}: Iteration \%1"}\NormalTok{, i)}
\NormalTok{    prm\_val }\OtherTok{\textless{}{-}}\NormalTok{ param\_values[i]}
\NormalTok{    param\_vals\_tmp[i] }\OtherTok{\textless{}{-}}\NormalTok{ prm\_val}
    
\NormalTok{    RMSE\_tmp }\OtherTok{\textless{}{-}} \FunctionTok{fn\_tune.test.param\_value}\NormalTok{(prm\_val)}
\NormalTok{    RMSEs\_tmp[i] }\OtherTok{\textless{}{-}}\NormalTok{ RMSE\_tmp}

    \FunctionTok{plot}\NormalTok{(param\_vals\_tmp[RMSEs\_tmp], RMSEs\_tmp[RMSEs\_tmp])}

    \ControlFlowTok{if}\NormalTok{(RMSE\_tmp }\SpecialCharTok{\textgreater{}}\NormalTok{ RMSE\_min)\{}
      \FunctionTok{warning}\NormalTok{(}\StringTok{"Function: \textasciigrave{}tune.model\_param\textasciigrave{}: \textasciigrave{}RSME\textasciigrave{} reached its minimum: "}\NormalTok{, }
\NormalTok{              RMSE\_min, }\StringTok{"for parameter value: "}\NormalTok{, prm\_val)}

      \ControlFlowTok{if}\NormalTok{ (i }\SpecialCharTok{\textgreater{}}\NormalTok{ i\_max.beyond\_RMSE\_min) \{}
        \FunctionTok{warning}\NormalTok{(}\StringTok{"Function: \textasciigrave{}tune.model\_param\textasciigrave{}:}
\StringTok{Operation is breaked (after \textasciigrave{}RSME\textasciigrave{} reached its minimum) on the following step: "}\NormalTok{, i)}
        \ControlFlowTok{break}
\NormalTok{      \}}
      \ControlFlowTok{next}
\NormalTok{    \}}

\NormalTok{    RMSE\_min }\OtherTok{\textless{}{-}}\NormalTok{ RMSE\_tmp}
\NormalTok{    prm\_val.best }\OtherTok{\textless{}{-}}\NormalTok{ prm\_val}
    
    \ControlFlowTok{if}\NormalTok{ (break.if\_min) \{}
\NormalTok{      i\_max.beyond\_RMSE\_min }\OtherTok{\textless{}{-}}\NormalTok{ i }\SpecialCharTok{+}\NormalTok{ steps.beyond\_min}
\NormalTok{    \}}
\NormalTok{  \}}
  
\NormalTok{  param\_values.best\_result }\OtherTok{\textless{}{-}} \FunctionTok{c}\NormalTok{(}\AttributeTok{param.best\_value =}\NormalTok{ prm\_val.best, }
                                \AttributeTok{best\_RMSE =}\NormalTok{ RMSE\_min)}

  \FunctionTok{list}\NormalTok{(}\AttributeTok{tuned.result =} \FunctionTok{data.frame}\NormalTok{(}\AttributeTok{RMSE =}\NormalTok{ RMSEs\_tmp,}
                                \AttributeTok{parameter.value =}\NormalTok{ param\_vals\_tmp),}
       \AttributeTok{best\_result =}\NormalTok{ param\_values.best\_result)}
\NormalTok{\}}
\end{Highlighting}
\end{Shaded}

\begin{noteblock}
The complete source code of the \href{https://github.com/AzKurban-edX-DS/Capstone-MovieLens/blob/main/r/src/support-functions/common-helper.functions.R\#L126}{tune.model\_param} function is available in the \href{https://github.com/AzKurban-edX-DS/Capstone-MovieLens/blob/main/r/src/support-functions/common-helper.functions.R\#L69}{Regularization} section of the \href{https://github.com/AzKurban-edX-DS/Capstone-MovieLens/blob/main/r/src/support-functions/common-helper.functions.R\#L126}{common-helper.functions.R} script on \emph{GitHub}.

\end{noteblock}

\paragraph{\texorpdfstring{\href{https://github.com/AzKurban-edX-DS/Capstone-MovieLens/blob/main/r/src/support-functions/common-helper.functions.R\#L208}{model.tune.param\_range} Function}{model.tune.param\_range Function}}\label{func.model.tune.param_range}

\hfill\break

Implements an algorithm for fine-tuning the model under test by figuring out the most precise value possible of the \emph{tuning parameter} over a given interval determined by the \(start\) and \(end\) values of the interval passed in the \hyperref[func.model.tune.param_range.args.loop_starter]{loop\_starter} argument.

\phantomsection\label{func.model.tune.param_range.usage}
\subparagraph{\texorpdfstring{\href{https://github.com/AzKurban-edX-DS/Capstone-MovieLens/blob/main/r/src/capstone-movielens.main.R\#L608}{Usage}}{Usage}}\label{usage-35}

\hfill\break

\begin{Shaded}
\begin{Highlighting}[]
\NormalTok{endpoints }\OtherTok{\textless{}{-}} 
  \FunctionTok{get\_fine\_tune.param.endpoints}\NormalTok{(cv.UME.preset.result}\SpecialCharTok{$}\NormalTok{tuned.result)}

\NormalTok{UM\_effect.loop\_starter }\OtherTok{\textless{}{-}} \FunctionTok{c}\NormalTok{(endpoints[}\StringTok{"start"}\NormalTok{], }
\NormalTok{                            endpoints[}\StringTok{"end"}\NormalTok{], }
                            \DecValTok{8}\NormalTok{)}
\end{Highlighting}
\end{Shaded}

\begin{Shaded}
\begin{Highlighting}[]
\NormalTok{UM\_effect.loop\_starter}
\end{Highlighting}
\end{Shaded}

\begin{verbatim}
## start   end       
##   0.3   0.5   8.0
\end{verbatim}

\begin{noteblock}
In the code snippet above, we use the user-defined auxiliary function \hyperref[func.get_fine_tune.param.endpoints]{get\_fine\_tune.param.endpoints} described above in \emph{this Section}.

\end{noteblock}

\begin{Shaded}
\begin{Highlighting}[]
\NormalTok{UME.rglr.fine\_tune.cache.base\_name }\OtherTok{\textless{}{-}} \StringTok{"UME.rglr.fine{-}tune"}

\NormalTok{UME.rglr.fine\_tune.results }\OtherTok{\textless{}{-}} 
  \FunctionTok{model.tune.param\_range}\NormalTok{(UM\_effect.loop\_starter,}
\NormalTok{                         UME.rglr.fine\_tune.cache.path,}
\NormalTok{                         UME.rglr.fine\_tune.cache.base\_name,}
\NormalTok{                         regularize.test\_lambda.UM\_effect.cv)}

\NormalTok{UME.rglr.fine\_tune.RMSE.best }\OtherTok{\textless{}{-}}\NormalTok{ UME.rglr.fine\_tune.results}\SpecialCharTok{$}\NormalTok{best\_result[}\StringTok{"best\_RMSE"}\NormalTok{]}
\end{Highlighting}
\end{Shaded}

\begin{Shaded}
\begin{Highlighting}[]
\FunctionTok{str}\NormalTok{(UME.rglr.fine\_tune.results)}
\end{Highlighting}
\end{Shaded}

\begin{verbatim}
## List of 3
##  $ best_result           : Named num [1:2] 0.387 0.873
##   ..- attr(*, "names")= chr [1:2] "param.best_value" "best_RMSE"
##  $ param_values.endpoints: Named num [1:3] 3.87e-01 3.87e-01 9.54e-07
##   ..- attr(*, "names")= chr [1:3] "" "" ""
##  $ tuned.result          :'data.frame':  9 obs. of  2 variables:
##   ..$ parameter.value: num [1:9] 0.387 0.387 0.387 0.387 0.387 ...
##   ..$ RMSE           : num [1:9] 0.873 0.873 0.873 0.873 0.873 ...
\end{verbatim}

\newpage

\begin{Shaded}
\begin{Highlighting}[]
\FunctionTok{print}\NormalTok{(UME.rglr.fine\_tune.results)}
\end{Highlighting}
\end{Shaded}

\begin{verbatim}
## $best_result
## param.best_value        best_RMSE 
##        0.3874500        0.8732057 
## 
## $param_values.endpoints
##                                        
## 3.874491e-01 3.874557e-01 9.536743e-07 
## 
## $tuned.result
##   parameter.value      RMSE
## 1       0.3874481 0.8732057
## 2       0.3874491 0.8732057
## 3       0.3874500 0.8732057
## 4       0.3874510 0.8732057
## 5       0.3874519 0.8732057
## 6       0.3874529 0.8732057
## 7       0.3874538 0.8732057
## 8       0.3874548 0.8732057
## 9       0.3874557 0.8732057
\end{verbatim}

\phantomsection\label{func.model.tune.param_range.args}
\subparagraph{Arguments}\label{arguments-34}

\hfill\break

\phantomsection\label{func.model.tune.param_range.args.loop_starter}
\textbf{\emph{loop\_starter}}

\hfill\break
A numeric vector of the form \texttt{c(start,\ end,\ dvs)}, where \texttt{start} and \texttt{end} are the endpoints of the interval over which the value of the \emph{tuning parameter} that minimizes \emph{RMSE} has to be found.

The \texttt{dvs} is a divisor used to calculate the \emph{step size} for partitioning the interval into a sequence of parameter values, among which the value that minimizes \emph{RMSE} has to be found.

So the \emph{step size} of the sequence is calculated as follows:
\[
step = \frac{end - start}{dvs}
\]

The sequence obtained as a result of the transformation is equivalent to the one generated by the function \href{https://www.rdocumentation.org/packages/base/versions/3.6.2/topics/seq}{base::seq} as follows:

\begin{Shaded}
\begin{Highlighting}[]
\FunctionTok{seq}\NormalTok{(start, end, step)}
\end{Highlighting}
\end{Shaded}

\begin{noteblock}
In the function body, the \emph{step size} is stored in the local variable \texttt{seq\_increment}, and the values of \(start\), \(end\), and \(dvs\) are stored in the local variables \texttt{seq\_start}, \texttt{seq\_end}, and \texttt{interval\_divisor}, respectively.

\end{noteblock}

\newpage

In fact, the \href{https://www.rdocumentation.org/packages/base/versions/3.6.2/topics/seq}{seq} function is called internally to generate the sequence during the algorithm's execution by the following \href{https://github.com/AzKurban-edX-DS/Capstone-MovieLens/blob/main/r/src/support-functions/common-helper.functions.R\#L256}{line of code}:

\begin{Shaded}
\begin{Highlighting}[]
\NormalTok{    test\_param\_vals }\OtherTok{\textless{}{-}} \FunctionTok{seq}\NormalTok{(seq\_start, seq\_end, seq\_increment)}
\end{Highlighting}
\end{Shaded}

\phantomsection\label{func.model.tune.param_range.args.tune_dir_path}
\textbf{\emph{tune\_dir\_path}}

\hfill\break

To improve performance, the algorithm caches intermediate results in the file system. The value passed in \emph{this argument} specifies the path to the directory where the files are cached.

\phantomsection\label{func.model.tune.param_range.args.cache_file_base_name}
\textbf{\emph{cache\_file\_base\_name}}

\hfill\break

The algorithm generates unique names for cache files based on this and the \hyperref[func.model.tune.param_range.args.loop_starter]{loop\_starter} arguments' values, as well as a few other intermediate values calculated at runtime.

\phantomsection\label{func.model.tune.param_range.args.fn_tune.test.param_value}
\textbf{\emph{fn\_tune.test.param\_value}}

\hfill\break

Name of a model-specific helper function to pass to the \hyperref[func.tune.model_param.args.fn_tune.test.param_value]{same-named argument} of the \hyperref[func.tune.model_param]{tune.model\_param} function (see description above) that is called internally during the execution of the algorithm.

\phantomsection\label{func.model.tune.param_range.args.max.identical.min_RMSE.count}
\textbf{\emph{max.identical.min\_RMSE.count}}

\hfill\break
(Optional, 4 by default) If more than one identical minimum \emph{RMSE} value is calculated during execution, the number of identical minimums is limited by the value of \emph{this argument}. When it is reached, the algorithm considers the task execution to be complete.

\phantomsection\label{func.model.tune.param_range.args.endpoint.min_diff}
\textbf{\emph{endpoint.min\_diff}}

\hfill\break
(Optional, 0 by default) Defines the sensitivity threshold for determining the neighborhood boundaries of the minimum \texttt{RMSE} value (see the \hyperref[func.model.tune.param_range.details]{Details} section below for more details).

\phantomsection\label{func.model.tune.param_range.args.break.if_min}
\textbf{\emph{break.if\_min}}

\hfill\break
(Optional, \texttt{TRUE} by default) Boolean value to pass to the \hyperref[func.tune.model_param.args.break.if_min]{same-named argument} of the \hyperref[func.tune.model_param]{tune.model\_param} function (see description above) that is called internally during the execution of the algorithm.

\phantomsection\label{func.model.tune.param_range.args.steps.beyond_min}
\textbf{\emph{steps.beyond\_min}}

\hfill\break

(Optional, 2 by default) Numeric value to pass to the \hyperref[func.tune.model_param.args.steps.beyond_min]{same-named argument} of the \hyperref[func.tune.model_param]{tune.model\_param} function (see description above) that is called internally during the execution of the algorithm.

\newpage

\phantomsection\label{func.model.tune.param_range.details}
\subparagraph{\texorpdfstring{\textbf{Details}}{Details}}\label{details-33}

\hfill\break

With the approach used in this algorithm, the best value of the \emph{tuning parameter} (we call this the \emph{best value of the tuning parameter}) is considered to be the value corresponding to the minimum value of the \emph{Root Mean Square Error (RMSE)} for the model being tested.

During the execution, the algorithm organizes a cycle consisting of the following steps:

\phantomsection\label{func.model.tune.param_range.details.step1}
\begin{enumerate}
\def\labelenumi{\arabic{enumi}.}
\tightlist
\item
  Calculates the step size for partitioning the interval into a sequence of values and save it to the local variable \texttt{seq\_increment} as described in section \hyperref[func.model.tune.param_range.args.loop_starter]{loop\_starter} argument description above using the following \href{https://github.com/AzKurban-edX-DS/Capstone-MovieLens/blob/main/r/src/support-functions/common-helper.functions.R\#L240}{line of code}:
\end{enumerate}

\begin{Shaded}
\begin{Highlighting}[]
\NormalTok{    seq\_increment }\OtherTok{\textless{}{-}}\NormalTok{ (seq\_end }\SpecialCharTok{{-}}\NormalTok{ seq\_start)}\SpecialCharTok{/}\NormalTok{interval\_divisor }
\end{Highlighting}
\end{Shaded}

\phantomsection\label{func.model.tune.param_range.details.step2}
\begin{enumerate}
\def\labelenumi{\arabic{enumi}.}
\setcounter{enumi}{1}
\tightlist
\item
  As shown in the following \href{https://github.com/AzKurban-edX-DS/Capstone-MovieLens/blob/main/r/src/support-functions/common-helper.functions.R\#L242}{code snippet}, checks if the \emph{step size} of the generated sequence is less than the minimum allowed value (in the current implementation, we use \texttt{1e-13}):
\end{enumerate}

\begin{Shaded}
\begin{Highlighting}[]
    \ControlFlowTok{if}\NormalTok{ (seq\_increment }\SpecialCharTok{\textless{}} \FloatTok{0.0000000000001}\NormalTok{) \{}
      \FunctionTok{warning}\NormalTok{(}\StringTok{"Function \textasciigrave{}model.tune.param\_range\textasciigrave{}:}
\StringTok{parameter value increment is too small."}\NormalTok{)}
      
      \FunctionTok{put\_log2}\NormalTok{(}\StringTok{"Function \textasciigrave{}model.tune.param\_range\textasciigrave{}:}
\StringTok{Final best RMSE for \textasciigrave{}parameter value = \%1\textasciigrave{}: \%2"}\NormalTok{,}
\NormalTok{               param\_values.best\_result[}\StringTok{"param.best\_value"}\NormalTok{],}
\NormalTok{               param\_values.best\_result[}\StringTok{"best\_RMSE"}\NormalTok{])}
      
      \FunctionTok{put}\NormalTok{(param\_values.best\_result)}
      \ControlFlowTok{break}
\NormalTok{    \}}
\end{Highlighting}
\end{Shaded}

If the check condition is met, the loop is terminated, and the execution proceeds to the \href{https://github.com/AzKurban-edX-DS/Capstone-MovieLens/blob/main/r/src/support-functions/common-helper.functions.R\#L426}{Finalizing Execution} section of the code described in \hyperref[func.model.tune.param_range.details.finalizing]{Finalizing Execution} subsection below:

Otherwise, proceeds to the \hyperref[func.model.tune.param_range.details.step3]{next step}:

\begin{noteblock}
In the code snippet above, we use the custom function \hyperref[func.put_log2]{put\_log2} described in section \hyperref[appndx_a.CHF.logging-funcs]{Logging Functions} of \emph{this Appendix}.

\end{noteblock}

\phantomsection\label{func.model.tune.param_range.details.step3}
\begin{enumerate}
\def\labelenumi{\arabic{enumi}.}
\setcounter{enumi}{2}
\tightlist
\item
  (The \emph{step size} is no less than the minimum allowed value: \texttt{seq\_increment\ \textgreater{}=\ 1e-13}) Creates a vector containing a sequence of parameter values (as described in section \hyperref[func.model.tune.param_range.args.loop_starter]{loop\_starter} argument description above) using the following \href{https://github.com/AzKurban-edX-DS/Capstone-MovieLens/blob/main/r/src/support-functions/common-helper.functions.R\#L256}{line of code}:
\end{enumerate}

\begin{Shaded}
\begin{Highlighting}[]
\NormalTok{    test\_param\_vals }\OtherTok{\textless{}{-}} \FunctionTok{seq}\NormalTok{(seq\_start, seq\_end, seq\_increment)}
\end{Highlighting}
\end{Shaded}

\phantomsection\label{func.model.tune.param_range.details.step4}
\begin{enumerate}
\def\labelenumi{\arabic{enumi}.}
\setcounter{enumi}{3}
\tightlist
\item
  Generates a unique name for the current iteration's cache file based on the values passed in the \hyperref[func.model.tune.param_range.args.cache_file_base_name]{cache\_file\_base\_name} and \hyperref[func.model.tune.param_range.args.loop_starter]{loop\_starter} arguments, along with the values stored in the local variables \texttt{seq\_start}, \texttt{seq\_end}, and \texttt{interval\_divisor}, concatenates the name with the path passed in the \hyperref[func.model.tune.param_range.args.tune_dir_path]{tune\_dir\_path} argument, and stores the value in the local variable \texttt{file\_path\_tmp}.
\end{enumerate}

\newpage

The following \href{https://github.com/AzKurban-edX-DS/Capstone-MovieLens/blob/main/r/src/support-functions/common-helper.functions.R\#L258}{code snippet} performs this step of execution:

\begin{Shaded}
\begin{Highlighting}[]
\NormalTok{    file\_name\_tmp }\OtherTok{\textless{}{-}}\NormalTok{ cache\_file\_base\_name }\SpecialCharTok{|\textgreater{}}
      \FunctionTok{str\_c}\NormalTok{(}\StringTok{"\_"}\NormalTok{) }\SpecialCharTok{|\textgreater{}}
      \FunctionTok{str\_c}\NormalTok{(}\FunctionTok{as.character}\NormalTok{(loop\_starter[}\DecValTok{1}\NormalTok{])) }\SpecialCharTok{|\textgreater{}}
      \FunctionTok{str\_c}\NormalTok{(}\StringTok{"\_"}\NormalTok{) }\SpecialCharTok{|\textgreater{}}
      \FunctionTok{str\_c}\NormalTok{(}\FunctionTok{as.character}\NormalTok{(loop\_starter[}\DecValTok{3}\NormalTok{])) }\SpecialCharTok{|\textgreater{}}
      \FunctionTok{str\_c}\NormalTok{(}\StringTok{"\_"}\NormalTok{) }\SpecialCharTok{|\textgreater{}}
      \FunctionTok{str\_c}\NormalTok{(}\FunctionTok{as.character}\NormalTok{(interval\_divisor)) }\SpecialCharTok{|\textgreater{}}
      \FunctionTok{str\_c}\NormalTok{(}\StringTok{"."}\NormalTok{) }\SpecialCharTok{|\textgreater{}}
      \FunctionTok{str\_c}\NormalTok{(}\FunctionTok{as.character}\NormalTok{(seq\_start)) }\SpecialCharTok{|\textgreater{}}
      \FunctionTok{str\_c}\NormalTok{(}\StringTok{"{-}"}\NormalTok{) }\SpecialCharTok{|\textgreater{}}
      \FunctionTok{str\_c}\NormalTok{(}\FunctionTok{as.character}\NormalTok{(seq\_end)) }\SpecialCharTok{|\textgreater{}}
      \FunctionTok{str\_c}\NormalTok{(}\StringTok{".RData"}\NormalTok{)}
    
\NormalTok{    file\_path\_tmp }\OtherTok{\textless{}{-}} \FunctionTok{file.path}\NormalTok{(tune\_dir\_path, file\_name\_tmp)}
    
    \FunctionTok{put\_log1}\NormalTok{(}\StringTok{"Function \textasciigrave{}model.tune.param\_range\textasciigrave{}:}
\StringTok{File path generated: \%1"}\NormalTok{, file\_path\_tmp)}
\end{Highlighting}
\end{Shaded}

\begin{noteblock}
In the code snippet above, we use the custom logging function \hyperref[func.put_log1]{put\_log1} described above in section \hyperref[appndx_a.CHF.logging-funcs]{Logging Functions} of \emph{this Appendix}.

\end{noteblock}

\begin{enumerate}
\def\labelenumi{\arabic{enumi}.}
\setcounter{enumi}{4}
\tightlist
\item
  Checks whether a file with the full name stored in the \texttt{file\_path\_tmp} variable exists, and if so, loads all the necessary information for the current iteration from it and goes to \hyperref[func.model.tune.param_range.details.step8]{step 8}. Otherwise, proceeds to the \hyperref[func.model.tune.param_range.details.step6]{next step}.
\end{enumerate}

The following \href{https://github.com/AzKurban-edX-DS/Capstone-MovieLens/blob/main/r/src/support-functions/common-helper.functions.R\#L276}{code snippet} performs this step of execution:

\begin{Shaded}
\begin{Highlighting}[]
    \ControlFlowTok{if}\NormalTok{ (}\FunctionTok{file.exists}\NormalTok{(file\_path\_tmp)) \{}
      \FunctionTok{put\_log1}\NormalTok{(}\StringTok{"Function \textasciigrave{}model.tune.param\_range\textasciigrave{}:}
\StringTok{Loading tuning data from file: \%1..."}\NormalTok{, file\_path\_tmp)}
      
      \FunctionTok{load}\NormalTok{(file\_path\_tmp)}
      \FunctionTok{put\_log1}\NormalTok{(}\StringTok{"Function \textasciigrave{}model.tune.param\_range\textasciigrave{}:}
\StringTok{Tuning data has been loaded from file: \%1"}\NormalTok{, file\_path\_tmp)}
      
\NormalTok{      tuned.result }\OtherTok{\textless{}{-}}\NormalTok{ tuned\_result}\SpecialCharTok{$}\NormalTok{tuned.result}
\NormalTok{    \} }\ControlFlowTok{else}\NormalTok{ \{}
      \CommentTok{\# ...}
\NormalTok{    \}}
\end{Highlighting}
\end{Shaded}

\phantomsection\label{func.model.tune.param_range.details.step6}
\begin{enumerate}
\def\labelenumi{\arabic{enumi}.}
\setcounter{enumi}{5}
\tightlist
\item
  Calls the helper function \hyperref[func.tune.model_param]{tune.model\_param} and stores the returned result in the local variable \texttt{tuned\_result}, passing the vector created earlier in \hyperref[func.model.tune.param_range.details.step3]{step 3} to the argument \hyperref[func.tune.model_param.args.param_values]{param\_values} along with the values of the remaining arguments passed in the following arguments of the same name to \emph{this function}:
\end{enumerate}

\begin{itemize}
\tightlist
\item
  \hyperref[func.model.tune.param_range.args.fn_tune.test.param_value]{fn\_tune.test.param\_value};
\item
  \hyperref[func.model.tune.param_range.args.break.if_min]{break.if\_min};
\item
  \hyperref[func.model.tune.param_range.args.steps.beyond_min]{steps.beyond\_min}.
\end{itemize}

The following \href{https://github.com/AzKurban-edX-DS/Capstone-MovieLens/blob/main/r/src/support-functions/common-helper.functions.R\#L292}{code snippet} performs this step of execution:

\begin{Shaded}
\begin{Highlighting}[]
\NormalTok{      tuned\_result }\OtherTok{\textless{}{-}} \FunctionTok{tune.model\_param}\NormalTok{(test\_param\_vals, }
\NormalTok{                                        fn\_tune.test.param\_value,}
\NormalTok{                                        break.if\_min,}
\NormalTok{                                        steps.beyond\_min)}
\end{Highlighting}
\end{Shaded}

\phantomsection\label{func.model.tune.param_range.details.step7}
\begin{enumerate}
\def\labelenumi{\arabic{enumi}.}
\setcounter{enumi}{6}
\tightlist
\item
  Extracts the data frame element stored in the variable \texttt{tuned\_result\$tuned.result} from the \emph{list object} saved in the local variable \texttt{tuned\_result} in the \hyperref[func.model.tune.param_range.details.step6]{previous step} and stores it in the local variable \texttt{tuned.result}. Then stores the variable \texttt{tuned\_result} (along with the necessary auxiliary information) in the cache file with the full name created earlier in \hyperref[func.model.tune.param_range.details.step4]{step 4}.
\end{enumerate}

The following \href{https://github.com/AzKurban-edX-DS/Capstone-MovieLens/blob/main/r/src/support-functions/common-helper.functions.R\#L297}{code snippet} performs this step of execution:

\begin{Shaded}
\begin{Highlighting}[]
\NormalTok{      tuned.result }\OtherTok{\textless{}{-}}\NormalTok{ tuned\_result}\SpecialCharTok{$}\NormalTok{tuned.result}
      
      \FunctionTok{save}\NormalTok{(tuned\_result,}
\NormalTok{           param.best\_value,}
\NormalTok{           seq\_increment,}
\NormalTok{           interval\_divisor,}
           \AttributeTok{file =}\NormalTok{ file\_path\_tmp)}

      \FunctionTok{put\_log1}\NormalTok{(}\StringTok{"Function \textasciigrave{}model.tune.param\_range\textasciigrave{}:}
\StringTok{File saved: \%1"}\NormalTok{, file\_path\_tmp)}
\end{Highlighting}
\end{Shaded}

\phantomsection\label{func.model.tune.param_range.details.step8}
\begin{enumerate}
\def\labelenumi{\arabic{enumi}.}
\setcounter{enumi}{7}
\tightlist
\item
  Determines endpoints for a new interval of the \emph{tuning parameter} values in the neighborhood of the current minimum \emph{RMSE} value (figured out so far) with use of the helper function \hyperref[func.get_fine_tune.param.endpoints.idx]{get\_fine\_tune.param.endpoints.idx} (described above in \emph{this Section}), to improve the accuracy of the \emph{RMSE} calculation in the next iteration.
\end{enumerate}

The following \href{https://github.com/AzKurban-edX-DS/Capstone-MovieLens/blob/main/r/src/support-functions/common-helper.functions.R\#L310}{code snippet} performs this step of execution:

\begin{Shaded}
\begin{Highlighting}[]
\NormalTok{    bound.idx }\OtherTok{\textless{}{-}} \FunctionTok{get\_fine\_tune.param.endpoints.idx}\NormalTok{(tuned.result)}
\NormalTok{    start.idx }\OtherTok{\textless{}{-}}\NormalTok{ bound.idx[}\StringTok{"start"}\NormalTok{]}
\NormalTok{    end.idx }\OtherTok{\textless{}{-}}\NormalTok{ bound.idx[}\StringTok{"end"}\NormalTok{]}

\NormalTok{    prm\_val.leftmost.tmp }\OtherTok{\textless{}{-}}\NormalTok{ tuned.result}\SpecialCharTok{$}\NormalTok{parameter.value[start.idx]}
\NormalTok{    RMSE.leftmost.tmp }\OtherTok{\textless{}{-}}\NormalTok{ tuned.result}\SpecialCharTok{$}\NormalTok{RMSE[start.idx]}

\NormalTok{    prm\_val.rightmost.tmp }\OtherTok{\textless{}{-}}\NormalTok{ tuned.result}\SpecialCharTok{$}\NormalTok{parameter.value[end.idx]}
\NormalTok{    RMSE.rightmost.tmp }\OtherTok{\textless{}{-}}\NormalTok{ tuned.result}\SpecialCharTok{$}\NormalTok{RMSE[end.idx]}
    
\NormalTok{    min\_RMSE }\OtherTok{\textless{}{-}}\NormalTok{ tuned.result}\SpecialCharTok{$}\NormalTok{RMSE[best\_RMSE.idx]}
\NormalTok{    min\_RMSE.prm\_val }\OtherTok{\textless{}{-}}\NormalTok{ tuned.result}\SpecialCharTok{$}\NormalTok{parameter.value[best\_RMSE.idx]}

\NormalTok{    seq\_start }\OtherTok{\textless{}{-}}\NormalTok{ prm\_val.leftmost.tmp}
\NormalTok{    seq\_end }\OtherTok{\textless{}{-}}\NormalTok{ prm\_val.rightmost.tmp}
\end{Highlighting}
\end{Shaded}

\phantomsection\label{func.model.tune.param_range.details.step9}
\begin{enumerate}
\def\labelenumi{\arabic{enumi}.}
\setcounter{enumi}{8}
\tightlist
\item
  Checks whether the local variable \texttt{best\_RMSE} is already initialized, and if not, initializes it to the minimum \emph{RMSE} value for the current iteration, figured out in the \hyperref[func.model.tune.param_range.details.step8]{previous step}, along with local variables intended to store the endpoint values.
\end{enumerate}

The following \href{https://github.com/AzKurban-edX-DS/Capstone-MovieLens/blob/main/r/src/support-functions/common-helper.functions.R\#L327}{code snippet} performs this step of execution:

\begin{Shaded}
\begin{Highlighting}[]
    \ControlFlowTok{if}\NormalTok{ (}\FunctionTok{is.na}\NormalTok{(best\_RMSE)) \{}
\NormalTok{      prm\_val.leftmost }\OtherTok{\textless{}{-}}\NormalTok{ prm\_val.leftmost.tmp}
\NormalTok{      RMSE.leftmost }\OtherTok{\textless{}{-}}\NormalTok{ RMSE.leftmost.tmp}
      
\NormalTok{      prm\_val.rightmost }\OtherTok{\textless{}{-}}\NormalTok{ prm\_val.rightmost.tmp}
\NormalTok{      RMSE.rightmost }\OtherTok{\textless{}{-}}\NormalTok{ RMSE.rightmost.tmp}
      
\NormalTok{      param.best\_value }\OtherTok{\textless{}{-}}\NormalTok{ min\_RMSE.prm\_val}
\NormalTok{      best\_RMSE }\OtherTok{\textless{}{-}}\NormalTok{ min\_RMSE}
\NormalTok{    \}}
\end{Highlighting}
\end{Shaded}

\phantomsection\label{func.model.tune.param_range.details.step10}
\begin{enumerate}
\def\labelenumi{\arabic{enumi}.}
\setcounter{enumi}{9}
\tightlist
\item
  Checks the deviations from the minimum \emph{RMSE} value (stored in \texttt{min\_RMSE} local variable) for the values, corresponding to the endpoints of the new interval of the \emph{tuning parameter} values (figured out earlier in \hyperref[func.model.tune.param_range.details.step8]{step 8}), and updates the corresponding local variables related to the endpoints, if the deviation reached the minimum threshold specified by the \hyperref[func.model.tune.param_range.args.endpoint.min_diff]{endpoint.min\_diff} argument.
\end{enumerate}

The following \href{https://github.com/AzKurban-edX-DS/Capstone-MovieLens/blob/main/r/src/support-functions/common-helper.functions.R\#L340}{code snippet} performs this step of execution:

\begin{Shaded}
\begin{Highlighting}[]
    \ControlFlowTok{if}\NormalTok{ (RMSE.leftmost.tmp }\SpecialCharTok{{-}}\NormalTok{ min\_RMSE }\SpecialCharTok{\textgreater{}=}\NormalTok{ endpoint.min\_diff) \{}
\NormalTok{      prm\_val.leftmost }\OtherTok{\textless{}{-}}\NormalTok{ prm\_val.leftmost.tmp}
\NormalTok{      RMSE.leftmost }\OtherTok{\textless{}{-}}\NormalTok{ RMSE.leftmost.tmp}
\NormalTok{    \} }
    
    \ControlFlowTok{if}\NormalTok{ (RMSE.rightmost.tmp }\SpecialCharTok{{-}}\NormalTok{ min\_RMSE }\SpecialCharTok{\textgreater{}=}\NormalTok{ endpoint.min\_diff) \{}
\NormalTok{      prm\_val.rightmost }\OtherTok{\textless{}{-}}\NormalTok{ prm\_val.rightmost.tmp}
\NormalTok{      RMSE.rightmost }\OtherTok{\textless{}{-}}\NormalTok{ RMSE.rightmost.tmp}
\NormalTok{    \} }
\end{Highlighting}
\end{Shaded}

\phantomsection\label{func.model.tune.param_range.details.step11}
\begin{enumerate}
\def\labelenumi{\arabic{enumi}.}
\setcounter{enumi}{10}
\tightlist
\item
  Makes sure that the index of the start endpoint of the parameter interval figured out in the \hyperref[func.model.tune.param_range.details.step8]{step8} is less than the index of the end endpoint and, if not, the loop is terminated with a corresponding \emph{warning message} and execution proceeds to the \href{https://github.com/AzKurban-edX-DS/Capstone-MovieLens/blob/main/r/src/support-functions/common-helper.functions.R\#L426}{Finalizing Execution} section of the code described below in section \hyperref[func.model.tune.param_range.details.finalizing]{Finalizing Execution} subsection of \emph{this Description}.
\end{enumerate}

Otherwise, proceeds to the \hyperref[func.model.tune.param_range.details.step11]{next step}.

The following \href{https://github.com/AzKurban-edX-DS/Capstone-MovieLens/blob/main/r/src/support-functions/common-helper.functions.R\#L352}{code snippet} performs this step of execution:

\begin{Shaded}
\begin{Highlighting}[]
    \ControlFlowTok{if}\NormalTok{ (end.idx }\SpecialCharTok{{-}}\NormalTok{ start.idx }\SpecialCharTok{\textless{}=} \DecValTok{0}\NormalTok{) \{}
      \FunctionTok{warning}\NormalTok{(}\StringTok{"\textasciigrave{}tuned.result$parameter.value\textasciigrave{} sequential start index are the same or greater than end one."}\NormalTok{)}
      \FunctionTok{put\_log1}\NormalTok{(}\StringTok{"Function \textasciigrave{}model.tune.param\_range\textasciigrave{}:}
\StringTok{Current minimal RMSE: \%1"}\NormalTok{, rmse\_min)}
      
      \FunctionTok{put\_log2}\NormalTok{(}\StringTok{"Function \textasciigrave{}model.tune.param\_range\textasciigrave{}:}
\StringTok{Reached minimal RMSE for the test parameter value = \%1: \%2"}\NormalTok{,}
\NormalTok{               param\_values.best\_result[}\StringTok{"param.best\_value"}\NormalTok{],}
\NormalTok{               param\_values.best\_result[}\StringTok{"best\_RMSE"}\NormalTok{])}
      
      \FunctionTok{put}\NormalTok{(param\_values.best\_result)}
      \ControlFlowTok{break}
\NormalTok{    \}}
\end{Highlighting}
\end{Shaded}

\begin{noteblock}
In the code snippet above, we use the custom logging functions \hyperref[func.put_log1]{put\_log1} and \hyperref[func.put_log2]{put\_log2} described above in section \hyperref[appndx_a.CHF.logging-funcs]{Logging Functions} of \emph{this Appendix}.

\end{noteblock}

\phantomsection\label{func.model.tune.param_range.details.step11}
\begin{enumerate}
\def\labelenumi{\arabic{enumi}.}
\setcounter{enumi}{11}
\tightlist
\item
  Makes sure that the minimum \emph{RMSE} value achieved so far (and stored in the local variable \texttt{best\_RMSE}) is no less than the minimum \emph{RMSE} value calculated for the current iteration and stored in the local variable \texttt{min\_RMSE} (that is \texttt{best\_RMSE\ \textgreater{}=\ min\_RMSE}) and, if not, \textbf{\emph{terminates the execution}} with an error message (this case is considered a \textbf{fatal error!})
\end{enumerate}

Otherwise, the further execution depends on the following condition:

\begin{itemize}
\tightlist
\item
  If \texttt{best\_RMSE\ ==\ min\_RMSE}, proceeds to the \hyperref[func.model.tune.param_range.details.step13]{next step} with a corresponding \emph{warning message};
\item
  Otherwise, proceeds to \hyperref[func.model.tune.param_range.details.step14]{step 14}.
\end{itemize}

The following \href{https://github.com/AzKurban-edX-DS/Capstone-MovieLens/blob/main/r/src/support-functions/common-helper.functions.R\#L367}{code snippet} performs this step of execution:

\begin{Shaded}
\begin{Highlighting}[]
    \ControlFlowTok{if}\NormalTok{ (best\_RMSE }\SpecialCharTok{==}\NormalTok{ min\_RMSE) \{}
      
            \FunctionTok{warning}\NormalTok{(}\StringTok{"Currently computed minimal RMSE equals the previously reached best one: "}\NormalTok{,}
\NormalTok{              best\_RMSE, }\StringTok{"}
\StringTok{Currently computed minial value is: "}\NormalTok{, min\_RMSE)}
      
      \FunctionTok{put\_log2}\NormalTok{(}\StringTok{"Function \textasciigrave{}model.tune.param\_range\textasciigrave{}:}
\StringTok{Current minimal RMSE for \textasciigrave{}parameter value = \%1\textasciigrave{}: \%2"}\NormalTok{,}
\NormalTok{               tuned.result}\SpecialCharTok{$}\NormalTok{parameter.value[}\FunctionTok{which.min}\NormalTok{(tuned.result}\SpecialCharTok{$}\NormalTok{RMSE)],}
\NormalTok{               min\_RMSE)}
      
      \FunctionTok{put\_log2}\NormalTok{(}\StringTok{"Function \textasciigrave{}model.tune.param\_range\textasciigrave{}:}
\StringTok{So far reached best RMSE for \textasciigrave{}parameter value = \%1\textasciigrave{}: \%2"}\NormalTok{,}
\NormalTok{               param\_values.best\_result[}\StringTok{"param.best\_value"}\NormalTok{],}
\NormalTok{               param\_values.best\_result[}\StringTok{"best\_RMSE"}\NormalTok{])}
      
      \FunctionTok{put}\NormalTok{(param\_values.best\_result)}
      

      \CommentTok{\# the next step of the current iteration}
      \CommentTok{\# ...}
      
\NormalTok{    \} }\ControlFlowTok{else} \ControlFlowTok{if}\NormalTok{ (best\_RMSE }\SpecialCharTok{\textless{}}\NormalTok{ min\_RMSE) \{}
      \CommentTok{\# Fatal Error! (stops the execution)}
      
      \FunctionTok{warning}\NormalTok{(}\StringTok{"Current minimal RMSE is greater than previously computed best value: "}\NormalTok{,}
\NormalTok{              best\_RMSE, }\StringTok{"}
\StringTok{Currently computed minial value is: "}\NormalTok{, min\_RMSE)}
      \FunctionTok{stop}\NormalTok{(}\StringTok{"Current minimal RMSE is greater than previously computed best value: "}\NormalTok{,}
\NormalTok{           best\_RMSE, }\StringTok{"}
\StringTok{Currently computed minial value is: "}\NormalTok{, min\_RMSE)}
\NormalTok{    \}}

    \CommentTok{\# step 14 of the current iteration}
\end{Highlighting}
\end{Shaded}

\newpage

\phantomsection\label{func.model.tune.param_range.details.step13}
\begin{enumerate}
\def\labelenumi{\arabic{enumi}.}
\setcounter{enumi}{12}
\tightlist
\item
  (\texttt{best\_RMSE\ ==\ min\_RMSE}) Checks if the number of equal values of the \emph{best RMSE} in the \emph{tuned result} for the current iteration (stored in the local variable \texttt{tuned.result\$RMSE}) reached the value specified by the value passed in the argument \hyperref[func.model.tune.param_range.args.max.identical.min_RMSE.count]{max.identical.min\_RMSE.count}, and, if so, saves the best values of the \emph{tuning parameter} and corresponding \emph{RMSE} in the local variable \texttt{param\_values.best\_result}, terminates the loop with a corresponding warning message, and proceeds to the \href{https://github.com/AzKurban-edX-DS/Capstone-MovieLens/blob/main/r/src/support-functions/common-helper.functions.R\#L426}{Finalizing Execution} section of the code described below in section \hyperref[func.model.tune.param_range.details.finalizing]{Finalizing Execution} subsection of \emph{this Description}.
\end{enumerate}

Otherwise, proceeds to the \hyperref[func.model.tune.param_range.details.step14]{next step}.

The following \href{https://github.com/AzKurban-edX-DS/Capstone-MovieLens/blob/main/r/src/support-functions/common-helper.functions.R\#L384}{code snippet} performs this step of execution:

\begin{Shaded}
\begin{Highlighting}[]
      \ControlFlowTok{if}\NormalTok{ (}\FunctionTok{sum}\NormalTok{(tuned.result}\SpecialCharTok{$}\NormalTok{RMSE[tuned.result}\SpecialCharTok{$}\NormalTok{RMSE }\SpecialCharTok{==}\NormalTok{ min\_RMSE]) }\SpecialCharTok{\textgreater{}=}\NormalTok{ max.identical.min\_RMSE.count) \{}
        \FunctionTok{warning}\NormalTok{(}\StringTok{"Minimal \textasciigrave{}RMSE\textasciigrave{}identical values count reached it maximum allowed value: "}\NormalTok{,}
\NormalTok{                max.identical.min\_RMSE.count)}

        \FunctionTok{put}\NormalTok{(tuned.result}\SpecialCharTok{$}\NormalTok{RMSE)}
        
\NormalTok{        param\_values.best\_result }\OtherTok{\textless{}{-}}
          \FunctionTok{get\_best\_param.result}\NormalTok{(tuned.result}\SpecialCharTok{$}\NormalTok{parameter.value,}
\NormalTok{                                tuned.result}\SpecialCharTok{$}\NormalTok{RMSE)}

        \FunctionTok{put\_log2}\NormalTok{(}\StringTok{"Function \textasciigrave{}model.tune.param\_range\textasciigrave{}:}
\StringTok{      Reached the best RMSE for \textasciigrave{}parameter value = \%1\textasciigrave{}: \%2"}\NormalTok{,}
\NormalTok{                 param\_values.best\_result[}\StringTok{"param.best\_value"}\NormalTok{],}
\NormalTok{                 param\_values.best\_result[}\StringTok{"best\_RMSE"}\NormalTok{])}
        \ControlFlowTok{break}
\NormalTok{      \}}

      \CommentTok{\# proceed to the next step of the current iteration}
      \CommentTok{\# ...}
\end{Highlighting}
\end{Shaded}

\begin{noteblock}
To retrieve the best results for the current iteration from the data stored in the local variable \texttt{tuned.result}, the helper function \hyperref[func.get_best_param.result]{get\_best\_param.result} described above is used.

\end{noteblock}

\phantomsection\label{func.model.tune.param_range.details.step14}
\begin{enumerate}
\def\labelenumi{\arabic{enumi}.}
\setcounter{enumi}{13}
\tightlist
\item
  (\texttt{best\_RMSE\ \textgreater{}=\ min\_RMSE}) Updates the following local variables:
\end{enumerate}

\begin{itemize}
\tightlist
\item
  \texttt{best\_RMSE} with the value stored in the \texttt{min\_RMSE}(meaning the best \emph{RMSE} has been possibly improved in the current iteration);
\item
  \texttt{param.best\_value} with the corresponding \emph{best parameter value} (stored in the \texttt{min\_RMSE.prm\_val})
\item
  \texttt{param\_values.best\_result} with the best results retrieved from the \texttt{tuned.result}.
\end{itemize}

\newpage

The following \href{https://github.com/AzKurban-edX-DS/Capstone-MovieLens/blob/main/r/src/support-functions/common-helper.functions.R\#L410}{code snippet} performs this step of execution:

\begin{Shaded}
\begin{Highlighting}[]
\NormalTok{    best\_RMSE }\OtherTok{\textless{}{-}}\NormalTok{ min\_RMSE}
\NormalTok{    param.best\_value }\OtherTok{\textless{}{-}}\NormalTok{ min\_RMSE.prm\_val}

\NormalTok{    param\_values.best\_result }\OtherTok{\textless{}{-}} 
      \FunctionTok{get\_best\_param.result}\NormalTok{(tuned.result}\SpecialCharTok{$}\NormalTok{parameter.value, }
\NormalTok{                          tuned.result}\SpecialCharTok{$}\NormalTok{RMSE)}
    
    \FunctionTok{put\_log2}\NormalTok{(}\StringTok{"Function \textasciigrave{}model.tune.param\_range\textasciigrave{}:}
\StringTok{Currently reached best RMSE for \textasciigrave{}parameter value = \%1\textasciigrave{}: \%2"}\NormalTok{,}
\NormalTok{             param\_values.best\_result[}\StringTok{"param.best\_value"}\NormalTok{],}
\NormalTok{             param\_values.best\_result[}\StringTok{"best\_RMSE"}\NormalTok{])}
    
    \FunctionTok{put}\NormalTok{(param\_values.best\_result)}
  \ErrorTok{\}}
  \CommentTok{\# End repeat loop}
\end{Highlighting}
\end{Shaded}

\begin{noteblock}
In the code snippet above, we use the user-defined auxiliary function \hyperref[func.get_best_param.result]{get\_best\_param.result} described above in \emph{this Section}.

\end{noteblock}

\phantomsection\label{func.model.tune.param_range.details.step15}
\begin{enumerate}
\def\labelenumi{\arabic{enumi}.}
\setcounter{enumi}{14}
\tightlist
\item
  Proceeds to the next iteration, starting from \hyperref[func.model.tune.param_range.details.step1]{step 1}
\end{enumerate}

\phantomsection\label{func.model.tune.param_range.details.finalizing}
\textbf{\emph{Finalizing Execution}}

\hfill\break

Finally, the algorithm initializes the objects to be included in the return list with the results obtained during execution and returns the \emph{list object} to the client code.

The following \href{https://github.com/AzKurban-edX-DS/Capstone-MovieLens/blob/main/r/src/support-functions/common-helper.functions.R\#L427}{code snippet} performs this step of execution:

\begin{Shaded}
\begin{Highlighting}[]
  \CommentTok{\# Finalizing execution:}
\NormalTok{  n }\OtherTok{\textless{}{-}} \FunctionTok{length}\NormalTok{(tuned.result}\SpecialCharTok{$}\NormalTok{parameter.value)}
  
\NormalTok{  parameter.value }\OtherTok{\textless{}{-}}\NormalTok{ tuned.result}\SpecialCharTok{$}\NormalTok{parameter.value}
\NormalTok{  result.RMSE }\OtherTok{\textless{}{-}}\NormalTok{ tuned.result}\SpecialCharTok{$}\NormalTok{RMSE}
  
  \ControlFlowTok{if}\NormalTok{ (result.RMSE[}\DecValTok{1}\NormalTok{] }\SpecialCharTok{==}\NormalTok{ best\_RMSE) \{}
\NormalTok{    parameter.value[}\DecValTok{1}\NormalTok{] }\OtherTok{\textless{}{-}}\NormalTok{ prm\_val.leftmost}
\NormalTok{    result.RMSE[}\DecValTok{1}\NormalTok{] }\OtherTok{\textless{}{-}}\NormalTok{ RMSE.leftmost}
\NormalTok{  \}}
  
  \ControlFlowTok{if}\NormalTok{ (result.RMSE[n] }\SpecialCharTok{==}\NormalTok{ best\_RMSE) \{}
\NormalTok{    parameter.value[n}\SpecialCharTok{+}\DecValTok{1}\NormalTok{] }\OtherTok{\textless{}{-}}\NormalTok{ prm\_val.rightmost}
\NormalTok{    result.RMSE[n}\SpecialCharTok{+}\DecValTok{1}\NormalTok{] }\OtherTok{\textless{}{-}}\NormalTok{ RMSE.rightmost}
\NormalTok{  \}}
  
  \FunctionTok{list}\NormalTok{(}\AttributeTok{best\_result =}\NormalTok{ param\_values.best\_result,}
       \AttributeTok{param\_values.endpoints =} \FunctionTok{c}\NormalTok{(prm\_val.leftmost, prm\_val.rightmost, seq\_increment),}
       \AttributeTok{tuned.result =} \FunctionTok{data.frame}\NormalTok{(}\AttributeTok{parameter.value =}\NormalTok{ parameter.value,}
                                 \AttributeTok{RMSE =}\NormalTok{ result.RMSE))}
\end{Highlighting}
\end{Shaded}

\newpage

\phantomsection\label{func.model.tune.param_range.value}
\subparagraph{\texorpdfstring{\textbf{Value}}{Value}}\label{value-33}

\hfill\break

From the \href{https://github.com/AzKurban-edX-DS/Capstone-MovieLens/blob/main/r/src/support-functions/common-helper.functions.R\#L443}{source code} of the function \emph{being described here}, we can see that it returns a list of three objects:

\begin{Shaded}
\begin{Highlighting}[]
  \FunctionTok{list}\NormalTok{(}\AttributeTok{best\_result =}\NormalTok{ param\_values.best\_result,}
       \AttributeTok{param\_values.endpoints =} \FunctionTok{c}\NormalTok{(prm\_val.leftmost, prm\_val.rightmost, seq\_increment),}
       \AttributeTok{tuned.result =} \FunctionTok{data.frame}\NormalTok{(}\AttributeTok{parameter.value =}\NormalTok{ parameter.value,}
                                 \AttributeTok{RMSE =}\NormalTok{ result.RMSE))}
\end{Highlighting}
\end{Shaded}

The objects comprising the return list are as follows:

\begin{itemize}
\item
  \texttt{best\_result}: A vector containing a \emph{pair of values}: the found \emph{best value of the tuning parameter} for which the \emph{RMSE} value is minimal, along with the minimal \emph{RMSE} value itself;
\item
  \texttt{param\_values.endpoints}: Information related to the final interval of the \emph{tuning parameter} values;
\item
  \texttt{tuned.result}: A data frame containing two variables:

  \begin{itemize}
  \item
    \emph{parameter.value}: A vector containing all processed values of the \emph{tuning parameter};
  \item
    \emph{RMSE}: A vector containing the \emph{RMSE} values corresponding to the processed \emph{tuning parameter} values.
  \end{itemize}
\end{itemize}

\phantomsection\label{func.model.tune.param_range.code}
\subparagraph{Source Code}\label{source-code-37}

\hfill\break

The complete source code of the function \href{https://github.com/AzKurban-edX-DS/Capstone-MovieLens/blob/main/r/src/support-functions/common-helper.functions.R\#L208}{model.tune.param\_range} is available in the \href{https://github.com/AzKurban-edX-DS/Capstone-MovieLens/blob/main/r/src/support-functions/common-helper.functions.R\#L69}{Regularization} section of the \href{https://github.com/AzKurban-edX-DS/Capstone-MovieLens/blob/main/r/src/support-functions/common-helper.functions.R\#L208}{common-helper.functions.R} script on \emph{GitHub}.

\begin{noteblock}
The essential parts of the source code are also provided and explained in the \hyperref[func.model.tune.param_range.details]{Details} section of \emph{this Description}.

\end{noteblock}

\newpage

\subsection{Data Helper Functions}\label{appndx_a.data_helper.functions}

\hfill\break

\subsubsection{Initializing Input Datasets}\label{initializing-input-datasets}

\begin{noteblock}
The complete source code of the functions described in this section is available in the the \href{https://github.com/AzKurban-edX-DS/Capstone-MovieLens/blob/main/r/src/support-functions/data.helper.functions.R\#L88}{Initializing input datasets} section of the \href{https://github.com/AzKurban-edX-DS/Capstone-MovieLens/blob/main/r/src/support-functions/data.helper.functions.R\#L88}{data.helper.functions.R} script on \emph{GitHub}.

\end{noteblock}

\paragraph{\texorpdfstring{\href{https://github.com/AzKurban-edX-DS/Capstone-MovieLens/blob/main/r/src/support-functions/data.helper.functions.R\#L89}{make\_source\_datasets} Function}{make\_source\_datasets Function}}\label{func.make_source_datasets}

\hfill\break

Creates the major input datasets (\emph{Movielens Datasets}) used throughout the project.

\phantomsection\label{func.make_source_datasets.usage}
\subparagraph{Usage}\label{usage-36}

\hfill\break

\begin{Shaded}
\begin{Highlighting}[]
\NormalTok{movielens\_datasets }\OtherTok{\textless{}{-}} \FunctionTok{make\_source\_datasets}\NormalTok{()}
\end{Highlighting}
\end{Shaded}

\begin{noteblock}
The above \href{https://github.com/AzKurban-edX-DS/Capstone-MovieLens/blob/main/r/src/support-functions/data.helper.functions.R\#L240}{line of code} creates the \emph{Movielens Datasets} using the function \texttt{make\_source\_datasets}, which is called internally from the custom helper function \hyperref[func.init_source_datasets]{init\_source\_datasets} described below in \emph{this Section}.

\end{noteblock}

\phantomsection\label{func.make_source_datasets.details}
\subparagraph{Details}\label{details-34}

\hfill\break

To produce the \emph{Movielens Datasets} (listed in the \hyperref[func.make_source_datasets.value]{Value} section of \emph{this Appendix} below), the function uses the \texttt{edx} as the original dataset, and, during the execution, it calls internally the following \emph{user-defined data helper functions} described below in section \hyperref[data-processing-functions]{Data Processing Functions} of \emph{this Appendix}:

\begin{itemize}
\tightlist
\item
  \hyperref[func.splitGenreRows]{splitGenreRows} to create the \hyperref[appndx_b.edx.sgr]{\texttt{edx.sgr} Object};
\item
  \hyperref[func.sample_train_validation_sets]{sample\_train\_validation\_sets} to create \emph{Train} and \emph{Validation} sets of the \emph{inner datasets} of the \hyperref[appndx_b.edx_CV]{\texttt{edx\_CV} Object}.
\end{itemize}

Additionally, it also calls the following \emph{custom logging functions} described above in section \hyperref[appndx_a.CHF.logging-funcs]{Logging Functions} of \emph{this Appendix}:

\begin{itemize}
\tightlist
\item
  \hyperref[func.put_start_date]{put\_start\_date};
\item
  \hyperref[func.put_end_date]{put\_end\_date};
\item
  \hyperref[func.put_log]{put\_log};
\item
  \hyperref[func.put_log1]{put\_log1}.
\end{itemize}

\begin{noteblock}
The function being described is a utility and is used internally in the \hyperref[func.init_source_datasets]{init\_source\_datasets} function as shown above in the \hyperref[func.make_source_datasets.usage]{Usage} section of \emph{this Description}.

\end{noteblock}

\newpage

\phantomsection\label{func.make_source_datasets.value}
\subparagraph{Value}\label{value-34}

\hfill\break

The list of the following datasets (described in detail in \hyperref[appndx_b]{Appendix B: Models Training Datasets}) are returned to the client code:

\begin{itemize}
\item
  \hyperref[appndx_b.edx.mx]{\texttt{edx.mx} Matrix Object};
\item
  \hyperref[appndx_b.edx.sgr]{\texttt{edx.sgr} Object};
\item
  \hyperref[appndx_b.movie_map]{\texttt{movie\_map} Object};
\item
  \hyperref[appndx_b.date_days_map]{\texttt{date\_days\_map} Object};
\item
  \hyperref[appndx_b.edx_CV]{\texttt{edx\_CV} Object}.
\end{itemize}

\phantomsection\label{func.make_source_datasets.code}
\subparagraph{Source Code}\label{source-code-38}

\hfill\break

The source code of the \href{https://github.com/AzKurban-edX-DS/Capstone-MovieLens/blob/main/r/src/support-functions/data.helper.functions.R\#L89}{make\_source\_datasets} function is shown below:

\begin{Shaded}
\begin{Highlighting}[]
\NormalTok{make\_source\_datasets }\OtherTok{\textless{}{-}} \ControlFlowTok{function}\NormalTok{()\{}
  \FunctionTok{put\_log}\NormalTok{(}\StringTok{"Function: \textasciigrave{}make\_source\_datasets\textasciigrave{}: Creating source datasets..."}\NormalTok{)}
  
  \FunctionTok{put\_log}\NormalTok{(}\StringTok{"Function: \textasciigrave{}make\_source\_datasets\textasciigrave{}: Creating Rating Matrix from \textasciigrave{}edx\textasciigrave{} dataset..."}\NormalTok{)}
\NormalTok{  edx.mx }\OtherTok{\textless{}{-}}\NormalTok{ edx }\SpecialCharTok{|\textgreater{}} 
    \FunctionTok{mutate}\NormalTok{(}\AttributeTok{userId =} \FunctionTok{factor}\NormalTok{(userId),}
           \AttributeTok{movieId =} \FunctionTok{factor}\NormalTok{(movieId)) }\SpecialCharTok{|\textgreater{}}
    \FunctionTok{select}\NormalTok{(movieId, userId, rating) }\SpecialCharTok{|\textgreater{}}
    \FunctionTok{pivot\_wider}\NormalTok{(}\AttributeTok{names\_from =}\NormalTok{ movieId, }\AttributeTok{values\_from =}\NormalTok{ rating) }\SpecialCharTok{|\textgreater{}}
    \FunctionTok{column\_to\_rownames}\NormalTok{(}\StringTok{"userId"}\NormalTok{) }\SpecialCharTok{|\textgreater{}}
    \FunctionTok{as.matrix}\NormalTok{()}
  
  \FunctionTok{put\_log}\NormalTok{(}\StringTok{"Function: \textasciigrave{}make\_source\_datasets\textasciigrave{}:}
\StringTok{Matrix created: \textasciigrave{}edx.mx\textasciigrave{} of the following dimentions:"}\NormalTok{)}
  \FunctionTok{put}\NormalTok{(}\FunctionTok{dim}\NormalTok{(edx.mx))}

  \CommentTok{\#\textgreater{} To be able to map movie IDs to titles we create the following lookup table:}
\NormalTok{  movie\_map }\OtherTok{\textless{}{-}}\NormalTok{ edx }\SpecialCharTok{|\textgreater{}} \FunctionTok{select}\NormalTok{(movieId, title, genres) }\SpecialCharTok{|\textgreater{}} 
    \FunctionTok{distinct}\NormalTok{(movieId, }\AttributeTok{.keep\_all =} \ConstantTok{TRUE}\NormalTok{)}
  
  \FunctionTok{put\_log}\NormalTok{(}\StringTok{"Function: \textasciigrave{}make\_source\_datasets\textasciigrave{}: Dataset created: movie\_map"}\NormalTok{)}
  \FunctionTok{put}\NormalTok{(}\FunctionTok{summary}\NormalTok{(movie\_map))}
  
  \FunctionTok{put\_log}\NormalTok{(}\StringTok{"Function: \textasciigrave{}make\_source\_datasets\textasciigrave{}: Creating Date{-}Days Map dataset..."}\NormalTok{)}
\NormalTok{  date\_days\_map }\OtherTok{\textless{}{-}}\NormalTok{ edx }\SpecialCharTok{|\textgreater{}}
    \FunctionTok{mutate}\NormalTok{(}\AttributeTok{date\_time =} \FunctionTok{as\_datetime}\NormalTok{(timestamp)) }\SpecialCharTok{|\textgreater{}}
    \FunctionTok{mutate}\NormalTok{(}\AttributeTok{date =} \FunctionTok{as\_date}\NormalTok{(date\_time)) }\SpecialCharTok{|\textgreater{}}
    \FunctionTok{mutate}\NormalTok{(}\AttributeTok{year =} \FunctionTok{year}\NormalTok{(date\_time)) }\SpecialCharTok{|\textgreater{}}
    \FunctionTok{mutate}\NormalTok{(}\AttributeTok{days =} \FunctionTok{as.integer}\NormalTok{(date }\SpecialCharTok{{-}} \FunctionTok{min}\NormalTok{(date))) }\SpecialCharTok{|\textgreater{}}
    \FunctionTok{select}\NormalTok{(timestamp, date\_time, date, year, days) }\SpecialCharTok{|\textgreater{}}
    \FunctionTok{distinct}\NormalTok{(timestamp, }\AttributeTok{.keep\_all =} \ConstantTok{TRUE}\NormalTok{)}
  
  \FunctionTok{put\_log}\NormalTok{(}\StringTok{"Function: \textasciigrave{}make\_source\_datasets\textasciigrave{}: Dataset created: date\_days\_map"}\NormalTok{)}
  \FunctionTok{put}\NormalTok{(}\FunctionTok{summary}\NormalTok{(date\_days\_map))}
  
  \FunctionTok{put\_log}\NormalTok{(}\StringTok{"Function: \textasciigrave{}make\_source\_datasets\textasciigrave{}: }
\StringTok{To account for the Movie Genre Effect, we need a dataset with split rows }
\StringTok{for movies belonging to multiple genres."}\NormalTok{)}
\NormalTok{  edx.sgr }\OtherTok{\textless{}{-}} \FunctionTok{splitGenreRows}\NormalTok{(edx)}
  
  
  
  \CommentTok{\#\textgreater{} We will use K{-}fold cross validation as explained in }
  \CommentTok{\#\textgreater{} Section 29.6.1: "K{-}fold validation" of the Cource Textbook:}
  \CommentTok{\#\textgreater{} https://rafalab.dfci.harvard.edu/dsbook{-}part{-}2/ml/resampling{-}methods.html\#k{-}fold{-}cross{-}validation}
  \CommentTok{\#\textgreater{} We are going to compute the following version of the MSE introducing in that section:}
  
  \CommentTok{\# $$}
  \CommentTok{\#   \textbackslash{}mbox\{MSE\}(\textbackslash{}lambda) \textbackslash{}approx\textbackslash{}frac\{1\}\{B\} \textbackslash{}sum\_\{b = 1\}\^{}B \textbackslash{}frac\{1\}\{N\}\textbackslash{}sum\_\{i = 1\}\^{}N \textbackslash{}left(\textbackslash{}hat\{y\}\_i\^{}b(\textbackslash{}lambda) {-} y\_i\^{}b\textbackslash{}right)\^{}2 }
  \CommentTok{\# $$}
  
\NormalTok{  start }\OtherTok{\textless{}{-}} \FunctionTok{put\_start\_date}\NormalTok{()}
\NormalTok{  edx\_CV }\OtherTok{\textless{}{-}} \FunctionTok{lapply}\NormalTok{(kfold\_index,  }\ControlFlowTok{function}\NormalTok{(fold\_i)\{}
    
    \FunctionTok{put\_log1}\NormalTok{(}\StringTok{"Method \textasciigrave{}make\_source\_datasets\textasciigrave{}: }
\StringTok{Creating K{-}Fold Cross{-}Validation Datasets, Fold \%1"}\NormalTok{, fold\_i)}
    
    \CommentTok{\#\textgreater{} We split the initial datasets into training sets, which we will use to build }
    \CommentTok{\#\textgreater{} and train our models, and validation sets in which we will compute the accuracy }
    \CommentTok{\#\textgreater{} of our predictions, the way described in the \textasciigrave{}Section 23.1.1 Movielens data\textasciigrave{}}
    \CommentTok{\#\textgreater{} (https://rafalab.dfci.harvard.edu/dsbook{-}part{-}2/highdim/regularization.html\#movielens{-}data) }
    \CommentTok{\#\textgreater{} of the Course Textbook.}
    
\NormalTok{    split\_sets }\OtherTok{\textless{}{-}}\NormalTok{ edx }\SpecialCharTok{|\textgreater{}}
      \FunctionTok{sample\_train\_validation\_sets}\NormalTok{(fold\_i}\SpecialCharTok{*}\DecValTok{1000}\NormalTok{)}
    
\NormalTok{    train\_set }\OtherTok{\textless{}{-}}\NormalTok{ split\_sets}\SpecialCharTok{$}\NormalTok{train\_set}
\NormalTok{    validation\_set }\OtherTok{\textless{}{-}}\NormalTok{ split\_sets}\SpecialCharTok{$}\NormalTok{validation\_set}
    
    \FunctionTok{put\_log}\NormalTok{(}\StringTok{"Function: \textasciigrave{}make\_source\_datasets\textasciigrave{}: }
\StringTok{Sampling 20\% from the split{-}row version of the \textasciigrave{}edx\textasciigrave{} dataset..."}\NormalTok{)}
\NormalTok{    split\_sets.gs }\OtherTok{\textless{}{-}}\NormalTok{ edx.sgr }\SpecialCharTok{|\textgreater{}}
      \FunctionTok{sample\_train\_validation\_sets}\NormalTok{(fold\_i}\SpecialCharTok{*}\DecValTok{2000}\NormalTok{)}

\NormalTok{    train.sgr }\OtherTok{\textless{}{-}}\NormalTok{ split\_sets.gs}\SpecialCharTok{$}\NormalTok{train\_set}
\NormalTok{    validation.sgr }\OtherTok{\textless{}{-}}\NormalTok{ split\_sets.gs}\SpecialCharTok{$}\NormalTok{validation\_set}
    
    \CommentTok{\# put\_log("Function: \textasciigrave{}make\_source\_datasets\textasciigrave{}: Dataset created: validation.sgr")}
    \CommentTok{\# put(summary(validation.sgr))}
    
    \CommentTok{\#\textgreater{} We will use the array representation described in \textasciigrave{}Section 17.5 of the Textbook\textasciigrave{}}
    \CommentTok{\#\textgreater{} (https://rafalab.dfci.harvard.edu/dsbook{-}part{-}2/linear{-}models/treatment{-}effect{-}models.html\#sec{-}anova), }
    \CommentTok{\#\textgreater{} for the training data. }
    \CommentTok{\#\textgreater{} To create this matrix, we use \textasciigrave{}tidyr::pivot\_wider\textasciigrave{} function:}
    
    \CommentTok{\# train\_set \textless{}{-} mutate(train\_set, userId = factor(userId), movieId = factor(movieId))}
    \CommentTok{\# train.sgr \textless{}{-} mutate(train.sgr, userId = factor(userId), movieId = factor(movieId))}
    
    \FunctionTok{put\_log}\NormalTok{(}\StringTok{"Function: \textasciigrave{}make\_source\_datasets\textasciigrave{}: Creating Rating Matrix from Train Set..."}\NormalTok{)}
\NormalTok{    train\_mx }\OtherTok{\textless{}{-}}\NormalTok{ train\_set }\SpecialCharTok{|\textgreater{}} 
      \FunctionTok{mutate}\NormalTok{(}\AttributeTok{userId =} \FunctionTok{factor}\NormalTok{(userId),}
             \AttributeTok{movieId =} \FunctionTok{factor}\NormalTok{(movieId)) }\SpecialCharTok{|\textgreater{}}
      \FunctionTok{select}\NormalTok{(movieId, userId, rating) }\SpecialCharTok{|\textgreater{}}
      \FunctionTok{pivot\_wider}\NormalTok{(}\AttributeTok{names\_from =}\NormalTok{ movieId, }\AttributeTok{values\_from =}\NormalTok{ rating) }\SpecialCharTok{|\textgreater{}}
      \FunctionTok{column\_to\_rownames}\NormalTok{(}\StringTok{"userId"}\NormalTok{) }\SpecialCharTok{|\textgreater{}}
      \FunctionTok{as.matrix}\NormalTok{()}
    
    \FunctionTok{put\_log}\NormalTok{(}\StringTok{"Function: \textasciigrave{}make\_source\_datasets\textasciigrave{}:}
\StringTok{Matrix created: \textasciigrave{}train\_mx\textasciigrave{} of the following dimentions:"}\NormalTok{)}
    \FunctionTok{put}\NormalTok{(}\FunctionTok{dim}\NormalTok{(train\_mx))}

    \FunctionTok{list}\NormalTok{(}\AttributeTok{train\_set =}\NormalTok{ train\_set,}
         \AttributeTok{train\_mx =}\NormalTok{ train\_mx,}
         \AttributeTok{train.sgr =}\NormalTok{ train.sgr,}
         \AttributeTok{validation\_set =}\NormalTok{ validation\_set)}
\NormalTok{  \})}
  \FunctionTok{put\_end\_date}\NormalTok{(start)}
  \FunctionTok{put\_log}\NormalTok{(}\StringTok{"Function: \textasciigrave{}make\_source\_datasets\textasciigrave{}: }
\StringTok{Set of K{-}Fold Cross{-}Validation datasets created: edx\_CV"}\NormalTok{)}
  
\NormalTok{  tuning\_sets }\OtherTok{\textless{}{-}}\NormalTok{ edx }\SpecialCharTok{|\textgreater{}}
    \FunctionTok{sample\_train\_validation\_sets}\NormalTok{(}\DecValTok{5423}\NormalTok{)}
  
  \FunctionTok{list}\NormalTok{(}\AttributeTok{edx\_CV =}\NormalTok{ edx\_CV,}
       \AttributeTok{edx.mx =}\NormalTok{ edx.mx,}
       \AttributeTok{edx.sgr =}\NormalTok{ edx.sgr,}
       \AttributeTok{tuning\_sets =}\NormalTok{ tuning\_sets,}
       \AttributeTok{movie\_map =}\NormalTok{ movie\_map,}
       \AttributeTok{date\_days\_map =}\NormalTok{ date\_days\_map)}
\NormalTok{\}}
\end{Highlighting}
\end{Shaded}

\begin{noteblock}
The source code of the \href{https://github.com/AzKurban-edX-DS/Capstone-MovieLens/blob/main/r/src/support-functions/data.helper.functions.R\#L89}{make\_source\_datasets} function is also available in the \href{https://github.com/AzKurban-edX-DS/Capstone-MovieLens/blob/main/r/src/support-functions/data.helper.functions.R\#L88}{Initializing input datasets} section of the \href{https://github.com/AzKurban-edX-DS/Capstone-MovieLens/blob/main/r/src/support-functions/data.helper.functions.R\#L89}{data.helper.functions.R} script on \emph{GitHub}.

\end{noteblock}

\newpage

\paragraph{\texorpdfstring{\href{https://github.com/AzKurban-edX-DS/Capstone-MovieLens/blob/main/r/src/support-functions/data.helper.functions.R\#L211}{init\_source\_datasets} Function}{init\_source\_datasets Function}}\label{func.init_source_datasets}

\hfill\break

Creates or loads from the \emph{Local Cache} file the major input datasets used throughout the project.

\phantomsection\label{func.init_source_datasets.usage}
\subparagraph{\texorpdfstring{\href{https://github.com/AzKurban-edX-DS/Capstone-MovieLens/blob/main/r/_startup/1.init.R\#L132}{Usage}}{Usage}}\label{usage-37}

\hfill\break

\begin{Shaded}
\begin{Highlighting}[]
\NormalTok{movielens\_datasets }\OtherTok{\textless{}{-}} \FunctionTok{init\_source\_datasets}\NormalTok{()}
\end{Highlighting}
\end{Shaded}

\subparagraph{Details}\label{details-35}

\hfill\break

The function checks whether the \emph{Movielens Datasets} are cached on the hard disk, calling the \href{https://www.rdocumentation.org/packages/base/versions/3.3.0/topics/files}{base::file.exists} \texttt{R} function, and, if so, loads them from the \emph{Local Cache} file calling the \hyperref[func.load_movielens_data_from_file]{load\_movielens\_data\_from\_file} user-defined auxiliary function described below in section \hyperref[data-processing-functions]{Data Processing Functions} of \emph{this Appendix}.

Otherwise, it calls the \hyperref[func.make_source_datasets]{make\_source\_datasets} function described above in \emph{this Section}, which creates all the necessary datasets using the \texttt{edx} as the original dataset, saves the result datasets in a dedicated cache file using the \href{https://www.rdocumentation.org/packages/base/versions/3.3.0/topics/save}{base::save} `R' function, and returns the datasets to the client code.

\phantomsection\label{func.init_source_datasets.value}
\subparagraph{Value}\label{value-35}

\hfill\break

The list of the following datasets (described in detail in \hyperref[appndx_b]{Appendix B: Models Training Datasets}) is returned to the client code:

\begin{itemize}
\tightlist
\item
  \hyperref[appndx_b.edx.mx]{\texttt{edx.mx} Matrix Object};
\item
  \hyperref[appndx_b.edx.sgr]{\texttt{edx.sgr} Object};
\item
  \hyperref[appndx_b.movie_map]{\texttt{movie\_map} Object};
\item
  \hyperref[appndx_b.date_days_map]{\texttt{date\_days\_map} Object};
\item
  \hyperref[appndx_b.edx_CV]{\texttt{edx\_CV} Object}.
\end{itemize}

\newpage

\phantomsection\label{func.init_source_datasets.code}
\subparagraph{Source Code}\label{source-code-39}

\hfill\break

The simplified version of the source code of the \href{https://github.com/AzKurban-edX-DS/Capstone-MovieLens/blob/main/r/src/support-functions/data.helper.functions.R\#L211}{init\_source\_datasets} function is shown below:

\begin{Shaded}
\begin{Highlighting}[]
\NormalTok{init\_source\_datasets }\OtherTok{\textless{}{-}} \ControlFlowTok{function}\NormalTok{()\{}
  \ControlFlowTok{if}\NormalTok{(}\FunctionTok{file.exists}\NormalTok{(movielens\_datasets\_file\_path))\{}
\NormalTok{    movielens\_datasets }\OtherTok{\textless{}{-}} \FunctionTok{load\_movielens\_data\_from\_file}\NormalTok{(movielens\_datasets\_file\_path)}
\NormalTok{  \} }\ControlFlowTok{else} \ControlFlowTok{if}\NormalTok{(}\FunctionTok{file.exists}\NormalTok{(movielens\_datasets\_zip)) \{}
    \FunctionTok{unzip}\NormalTok{(movielens\_datasets\_zip, movielens\_datasets\_file\_path)}

    \ControlFlowTok{if}\NormalTok{(}\SpecialCharTok{!}\FunctionTok{file.exists}\NormalTok{(movielens\_datasets\_file\_path)) \{}
      \FunctionTok{put\_log}\NormalTok{(}\StringTok{"Method \textasciigrave{}init\_source\_datasets\textasciigrave{}: }
\StringTok{File does not exists: \{movielens\_datasets\_file\}."}\NormalTok{)}
      \FunctionTok{stop}\NormalTok{(}\StringTok{"Failed to unzip MovieLens data zip{-}archive."}\NormalTok{)}
\NormalTok{    \}}
\NormalTok{    movielens\_datasets }\OtherTok{\textless{}{-}} \FunctionTok{load\_movielens\_data\_from\_file}\NormalTok{(movielens\_datasets\_file\_path)}
\NormalTok{  \} }\ControlFlowTok{else}\NormalTok{ \{}
    \FunctionTok{library}\NormalTok{(edx.capstone.movielens.data)}
\NormalTok{    movielens\_datasets }\OtherTok{\textless{}{-}} \FunctionTok{make\_source\_datasets}\NormalTok{()}
    \FunctionTok{put}\NormalTok{(}\StringTok{"Method \textasciigrave{}init\_source\_datasets\textasciigrave{}: }
\StringTok{All required datasets have been created."}\NormalTok{)}
    
    \FunctionTok{save}\NormalTok{(movielens\_datasets, }\AttributeTok{file =}\NormalTok{  movielens\_datasets\_file\_path)}

    \ControlFlowTok{if}\NormalTok{(}\SpecialCharTok{!}\FunctionTok{file.exists}\NormalTok{(movielens\_datasets\_file\_path)) \{}
      \FunctionTok{warning}\NormalTok{(}\StringTok{"MovieLens data was not saved to file."}\NormalTok{)}
\NormalTok{    \} }\ControlFlowTok{else}\NormalTok{ \{}
      \FunctionTok{put\_log}\NormalTok{(}\StringTok{"Method \textasciigrave{}init\_source\_datasets\textasciigrave{}: }
\StringTok{Datasets have been saved to file: \{movielens\_datasets\_file\_path\}."}\NormalTok{) }

      \FunctionTok{zip}\NormalTok{(movielens\_datasets\_zip, movielens\_datasets\_file\_path)}

      \ControlFlowTok{if}\NormalTok{(}\SpecialCharTok{!}\FunctionTok{file.exists}\NormalTok{(movielens\_datasets\_zip))\{}
        \FunctionTok{warning}\NormalTok{(}\StringTok{"Failed to zip MovieLens data file."}\NormalTok{)}
\NormalTok{      \} }\ControlFlowTok{else}\NormalTok{ \{}
        \FunctionTok{put\_log}\NormalTok{(}\StringTok{"Method \textasciigrave{}init\_source\_datasets\textasciigrave{}: }
\StringTok{Zip{-}archive created: \{movielens\_datasets\_zip\}."}\NormalTok{)}
\NormalTok{      \}}
\NormalTok{    \}}
\NormalTok{  \}}
\NormalTok{  movielens\_datasets}
\NormalTok{\}}
\end{Highlighting}
\end{Shaded}

\begin{noteblock}
The complete source code of the \href{https://github.com/AzKurban-edX-DS/Capstone-MovieLens/blob/main/r/src/support-functions/data.helper.functions.R\#L211}{init\_source\_datasets} function is available in the \href{https://github.com/AzKurban-edX-DS/Capstone-MovieLens/blob/main/r/src/support-functions/data.helper.functions.R\#L88}{Initializing input datasets} section of the \href{https://github.com/AzKurban-edX-DS/Capstone-MovieLens/blob/main/r/src/support-functions/data.helper.functions.R\#L211}{data.helper.functions.R} script on \emph{GitHub}.

\end{noteblock}

\newpage

\subsubsection{Data Processing Functions}\label{data-processing-functions}

\begin{noteblock}
The complete source code of the functions described in this section is available in the the \href{https://github.com/AzKurban-edX-DS/Capstone-MovieLens/blob/main/r/src/support-functions/data.helper.functions.R\#L3}{Data processing functions} section of the \href{https://github.com/AzKurban-edX-DS/Capstone-MovieLens/blob/main/r/src/support-functions/data.helper.functions.R\#L3}{data.helper.functions.R} script on \emph{GitHub}.

\end{noteblock}

\paragraph{\texorpdfstring{\href{https://github.com/AzKurban-edX-DS/Capstone-MovieLens/blob/main/r/src/support-functions/data.helper.functions.R\#L4}{load\_movielens\_data\_from\_file} Function}{load\_movielens\_data\_from\_file Function}}\label{func.load_movielens_data_from_file}

\hfill\break

Reloads \emph{Movielens} datasets from the cache file (previously created by the user-defined helper function \hyperref[func.init_source_datasets]{init\_source\_datasets} described above in section \hyperref[initializing-input-datasets]{Initializing Input Datasets} to \emph{this Appendix}) using the path passed in the \hyperref[func.load_movielens_data_from_file.args.file_path]{file\_path} argument.

\phantomsection\label{func.load_movielens_data_from_file.usage}
\subparagraph{Usage}\label{usage-38}

\hfill\break

\begin{Shaded}
\begin{Highlighting}[]
\NormalTok{    movielens\_datasets }\OtherTok{\textless{}{-}} \FunctionTok{load\_movielens\_data\_from\_file}\NormalTok{(movielens\_datasets\_file\_path)}
\end{Highlighting}
\end{Shaded}

\begin{noteblock}
The above \href{https://github.com/AzKurban-edX-DS/Capstone-MovieLens/blob/main/r/src/support-functions/data.helper.functions.R\#L216}{line of code} loads the \emph{Movielens Datasets} using the function \texttt{load\_movielens\_data\_from\_file}, which is called internally from the function \hyperref[func.init_source_datasets]{init\_source\_datasets} mentioned above in the \emph{current Description}.

\end{noteblock}

\phantomsection\label{func.load_movielens_data_from_file.args}
\subparagraph{Arguments}\label{arguments-35}

\hfill\break

\phantomsection\label{func.load_movielens_data_from_file.args.file_path}
\begin{itemize}
\tightlist
\item
  \textbf{file\_path:} Full path to the \emph{cache file}.
\end{itemize}

\phantomsection\label{func.load_movielens_data_from_file.details}
\subparagraph{Details}\label{details-36}

\hfill\break

This function is a wrapper for the \texttt{R} function \href{https://www.rdocumentation.org/packages/base/versions/3.3.0/topics/load}{base::load}. It also internally calls the auxiliary \texttt{R} function \href{https://www.rdocumentation.org/packages/base/versions/3.3.0/topics/files}{base::file.exists} to check for the existence of the \emph{cache file} at the path specified by the \hyperref[func.load_movielens_data_from_file.args.file_path]{file\_path} argument, as well as the following custom logging functions described in section \hyperref[appndx_a.CHF.logging-funcs]{Logging Functions} of \emph{this Appendix}:

\begin{itemize}
\tightlist
\item
  \hyperref[func.put_start_date]{put\_start\_date};
\item
  \hyperref[func.put_end_date]{put\_end\_date}.
\item
  \hyperref[func.put_log1]{put\_log1};
\end{itemize}

\begin{noteblock}
This is an auxiliary function that is used internally in the \hyperref[func.init_source_datasets]{init\_source\_datasets} function as shown above in the \hyperref[func.load_movielens_data_from_file.usage]{Usage} section of \emph{this Description}.

\end{noteblock}

\phantomsection\label{func.load_movielens_data_from_file.value}
\subparagraph{Value}\label{value-36}

\hfill\break

The list of the following datasets (described in detail in \hyperref[appndx_b]{Appendix B: Models Training Datasets}) is returned to the client code:

\begin{itemize}
\tightlist
\item
  \hyperref[appndx_b.edx.mx]{\texttt{edx.mx} Matrix Object};
\item
  \hyperref[appndx_b.edx.sgr]{\texttt{edx.sgr} Object};
\item
  \hyperref[appndx_b.movie_map]{\texttt{movie\_map} Object};
\item
  \hyperref[appndx_b.date_days_map]{\texttt{date\_days\_map} Object};
\item
  \hyperref[appndx_b.edx_CV]{\texttt{edx\_CV} Object}.
\end{itemize}

\phantomsection\label{func.load_movielens_data_from_file.code}
\subparagraph{Source Code}\label{source-code-40}

\hfill\break

The source code of the \href{https://github.com/AzKurban-edX-DS/Capstone-MovieLens/blob/main/r/src/support-functions/data.helper.functions.R\#L4}{load\_movielens\_data\_from\_file} function is shown below:

\begin{Shaded}
\begin{Highlighting}[]
\NormalTok{load\_movielens\_data\_from\_file }\OtherTok{\textless{}{-}} \ControlFlowTok{function}\NormalTok{(file\_path)\{}
  \FunctionTok{put\_log1}\NormalTok{(}\StringTok{"Loading MovieLens datasets from file: \%1..."}\NormalTok{, }
\NormalTok{           file\_path)}
\NormalTok{  start }\OtherTok{\textless{}{-}} \FunctionTok{put\_start\_date}\NormalTok{()}
  \FunctionTok{load}\NormalTok{(file\_path)}
  \FunctionTok{put\_end\_date}\NormalTok{(start)}
  \FunctionTok{put\_log1}\NormalTok{(}\StringTok{"MoviLens datasets have been loaded from file: \%s."}\NormalTok{, }
\NormalTok{           file\_path)}
\NormalTok{  movielens\_datasets}
\NormalTok{\}}
\end{Highlighting}
\end{Shaded}

\begin{noteblock}
The source code of the \href{https://github.com/AzKurban-edX-DS/Capstone-MovieLens/blob/main/r/src/support-functions/data.helper.functions.R\#L4}{load\_movielens\_data\_from\_file} function is also available in the \href{https://github.com/AzKurban-edX-DS/Capstone-MovieLens/blob/main/r/src/support-functions/data.helper.functions.R\#L3}{Data processing functions} section of the \href{https://github.com/AzKurban-edX-DS/Capstone-MovieLens/blob/main/r/src/support-functions/data.helper.functions.R\#L4}{data.helper.functions.R} script on \emph{GitHub}.

\end{noteblock}

\newpage

\paragraph{\texorpdfstring{\href{https://github.com/AzKurban-edX-DS/Capstone-MovieLens/blob/main/r/src/support-functions/data.helper.functions.R\#L14}{splitByUser} Function}{splitByUser Function}}\label{func.splitByUser}

\hfill\break

Divides the set of \emph{row indices} of the dataset passed in the \hyperref[func.splitByUser.args.data]{data} argument into the groups defined by the \texttt{.\$userId} variable in that dataset.

\phantomsection\label{func.splitByUser.usage}
\subparagraph{Usage}\label{usage-39}

\hfill\break

\begin{Shaded}
\begin{Highlighting}[]
    \FunctionTok{sapply}\NormalTok{(}\FunctionTok{splitByUser}\NormalTok{(data),}
           \ControlFlowTok{function}\NormalTok{(i) }\FunctionTok{sample}\NormalTok{(i, }\FunctionTok{ceiling}\NormalTok{(}\FunctionTok{length}\NormalTok{(i)}\SpecialCharTok{*}\NormalTok{.}\DecValTok{2}\NormalTok{))) }\SpecialCharTok{|\textgreater{}} 
    \FunctionTok{unlist}\NormalTok{() }\SpecialCharTok{|\textgreater{}} 
    \FunctionTok{sort}\NormalTok{()}
\end{Highlighting}
\end{Shaded}

\begin{noteblock}
In the \href{https://github.com/AzKurban-edX-DS/Capstone-MovieLens/blob/main/r/src/support-functions/data.helper.functions.R\#L21}{code snippet} above, the function \texttt{splitByUser} is called internally from the \hyperref[func.sample_train_validation_sets]{sample\_train\_validation\_sets} function described below in \emph{this Section}.

\end{noteblock}

\phantomsection\label{func.splitByUser.args}
\subparagraph{Arguments}\label{arguments-36}

\hfill\break

\phantomsection\label{func.splitByUser.args.data}
\begin{itemize}
\tightlist
\item
  \textbf{data:} Dataset to \emph{split by user};
\end{itemize}

\phantomsection\label{func.splitByUser.details}
\subparagraph{Details}\label{details-37}

\hfill\break

This function is a wrapper for the \href{https://www.rdocumentation.org/packages/base/versions/3.3/topics/split}{base::split} \texttt{R} function, which is called internally to divide the \emph{set of row indices} of the \emph{dataset} passed in the \hyperref[func.splitByUser.args.data]{data} argument into the groups defined by the variable \texttt{.\$userId}.

\begin{noteblock}
This is an auxiliary function intended for use in the \hyperref[func.sample_train_validation_sets]{sample\_train\_validation\_sets} function as shown in the \hyperref[func.splitByUser.usage]{Usage} section above of the \emph{current Description}.

\end{noteblock}

\phantomsection\label{func.splitByUser.value}
\subparagraph{Value}\label{value-37}

\hfill\break

List of vectors containing the values (corresponding \emph{row indices} of the dataset passed in the \hyperref[]{data}) for the \texttt{.\$userId} groups. The components of the list are named by the values of the \texttt{.\$userId} (after converting to a factor). For more details, please refer to the documentation of the \href{https://www.rdocumentation.org/packages/base/versions/3.3/topics/split}{base::split} function, which, in fact, forms the return value.

\phantomsection\label{func.splitByUser.code}
\subparagraph{Source Code}\label{source-code-41}

\hfill\break

The source code of the \href{https://github.com/AzKurban-edX-DS/Capstone-MovieLens/blob/main/r/src/support-functions/data.helper.functions.R\#L14}{splitByUser} function is shown below:

\begin{Shaded}
\begin{Highlighting}[]
\NormalTok{splitByUser }\OtherTok{\textless{}{-}} \ControlFlowTok{function}\NormalTok{(data)\{}
  \FunctionTok{split}\NormalTok{(}\DecValTok{1}\SpecialCharTok{:}\FunctionTok{nrow}\NormalTok{(data), data}\SpecialCharTok{$}\NormalTok{userId)}
\NormalTok{\}}
\end{Highlighting}
\end{Shaded}

\begin{noteblock}
The source code of the \href{https://github.com/AzKurban-edX-DS/Capstone-MovieLens/blob/main/r/src/support-functions/data.helper.functions.R\#L14}{splitByUser} function is also available in the \href{https://github.com/AzKurban-edX-DS/Capstone-MovieLens/blob/main/r/src/support-functions/data.helper.functions.R\#L3}{Data processing functions} section of the \href{https://github.com/AzKurban-edX-DS/Capstone-MovieLens/blob/main/r/src/support-functions/data.helper.functions.R\#L14}{data.helper.functions.R} script on \emph{GitHub}.

\end{noteblock}

\paragraph{\texorpdfstring{\href{https://github.com/AzKurban-edX-DS/Capstone-MovieLens/blob/main/r/src/support-functions/data.helper.functions.R\#L17}{sample\_train\_validation\_sets} Function}{sample\_train\_validation\_sets Function}}\label{func.sample_train_validation_sets}

\hfill\break

Implements an algorithm to split the input dataset passed in the \hyperref[func.sample_train_validation_sets.args.data]{data} argument into \emph{Train} and \emph{Validation} sets in the way described in \href{https://rafalab.dfci.harvard.edu/dsbook-part-2/ml/resampling-methods.html\#cross-validation}{Section 29.6 \emph{Cross validation}} of the \emph{Course Textbook (New Edition)}\autocite{IDS2_29-6}.

\phantomsection\label{func.sample_train_validation_sets.usage}
\subparagraph{Usage}\label{usage-40}

\hfill\break

\begin{Shaded}
\begin{Highlighting}[]
\NormalTok{split\_sets }\OtherTok{\textless{}{-}}\NormalTok{ edx }\SpecialCharTok{|\textgreater{}}
  \FunctionTok{sample\_train\_validation\_sets}\NormalTok{(fold\_i}\SpecialCharTok{*}\DecValTok{1000}\NormalTok{)}
\end{Highlighting}
\end{Shaded}

\begin{noteblock}
In the \href{https://github.com/AzKurban-edX-DS/Capstone-MovieLens/blob/main/r/src/support-functions/data.helper.functions.R\#L153}{code snippet} above, the function \texttt{sample\_train\_validation\_sets} is called internally from the \hyperref[func.make_source_datasets]{make\_source\_datasets} function described above in section \hyperref[initializing-input-datasets]{Initializing Input Datasets} of \emph{this Appendix}.

\end{noteblock}

\phantomsection\label{func.sample_train_validation_sets.args}
\subparagraph{Arguments}\label{arguments-37}

\hfill\break

\phantomsection\label{func.sample_train_validation_sets.args.data}
\begin{itemize}
\tightlist
\item
  \textbf{data:} Input dataset to split;
\end{itemize}

\phantomsection\label{func.sample_train_validation_sets.args.seed}
\begin{itemize}
\tightlist
\item
  \textbf{seed:} Seed value;
\end{itemize}

\phantomsection\label{func.sample_train_validation_sets.details}
\subparagraph{Details}\label{details-38}

\hfill\break

\begin{noteblock}
This is an auxiliary function intended for use in the \hyperref[func.make_source_datasets]{make\_source\_datasets} function as shown in the \hyperref[func.sample_train_validation_sets.usage]{Usage} section above of the \emph{current Description}.

\end{noteblock}

To split the dataset passed in the \hyperref[func.sample_train_validation_sets.args.data]{data} argument, the algorithm first takes a sample of 20\% of the size of the set of its \emph{row indices} for each user and sorts them by their corresponding \texttt{.\$userId} value.

Next, the algorithm creates a \emph{Train Set} from the rows with indices that \emph{IS NOT} in the \emph{set of samples} taken in the previous step.

Finally, it creates a \emph{Validation Set} from the remaining data, including \emph{ONLY THOSE ROWS} that have \texttt{.\$userId} and \texttt{.\$movieId} contained in the \emph{Train Set}.

\begin{noteblock}
During its execution, the algorithm internally calls the user-defined helper function \hyperref[func.splitByUser]{splitByUser} described above in \emph{this Section}, as well as the custom logging function \hyperref[func.put_log]{put\_log} described above in section \hyperref[appndx_a.CHF.logging-funcs]{Logging Functions} of \emph{this Appendix}.

\end{noteblock}

\phantomsection\label{func.sample_train_validation_sets.value}
\subparagraph{Value}\label{value-38}

\hfill\break

List of size 2, consisting of the \emph{Training} and \emph{Testing} datasets created during the algorithm execution.

\phantomsection\label{func.sample_train_validation_sets.code}
\subparagraph{Source Code}\label{source-code-42}

\hfill\break

Below is a simplified version of the source code of the \href{https://github.com/AzKurban-edX-DS/Capstone-MovieLens/blob/main/r/src/support-functions/data.helper.functions.R\#L17}{sample\_train\_validation\_sets} function:

\begin{Shaded}
\begin{Highlighting}[]
\NormalTok{sample\_train\_validation\_sets }\OtherTok{\textless{}{-}} \ControlFlowTok{function}\NormalTok{(data, seed)\{}
  \FunctionTok{put\_log}\NormalTok{(}\StringTok{"Function: \textasciigrave{}sample\_train\_validation\_sets\textasciigrave{}: Sampling 20\% of the \textasciigrave{}data\textasciigrave{} data..."}\NormalTok{)}
  \FunctionTok{set.seed}\NormalTok{(seed)}
\NormalTok{  validation\_ind }\OtherTok{\textless{}{-}} 
    \FunctionTok{sapply}\NormalTok{(}\FunctionTok{splitByUser}\NormalTok{(data),}
           \ControlFlowTok{function}\NormalTok{(i) }\FunctionTok{sample}\NormalTok{(i, }\FunctionTok{ceiling}\NormalTok{(}\FunctionTok{length}\NormalTok{(i)}\SpecialCharTok{*}\NormalTok{.}\DecValTok{2}\NormalTok{))) }\SpecialCharTok{|\textgreater{}} 
    \FunctionTok{unlist}\NormalTok{() }\SpecialCharTok{|\textgreater{}} 
    \FunctionTok{sort}\NormalTok{()}
  
  \FunctionTok{put\_log}\NormalTok{(}\StringTok{"Function: \textasciigrave{}sample\_train\_validation\_sets\textasciigrave{}: }
\StringTok{Extracting 80\% of the original \textasciigrave{}data\textasciigrave{} not used for the Validation Set, }
\StringTok{excluding data for users who provided no more than a specified number of ratings: \{min\_nratings\}."}\NormalTok{)}
\NormalTok{  train\_set }\OtherTok{\textless{}{-}}\NormalTok{ data[}\SpecialCharTok{{-}}\NormalTok{validation\_ind,]}

  \FunctionTok{put\_log}\NormalTok{(}\StringTok{"Function: \textasciigrave{}sample\_train\_validation\_sets\textasciigrave{}: }
\StringTok{To make sure we don’t include movies in the Training Set that should not be there, }
\StringTok{we exclude entries using the semi\_join function from the Validation Set."}\NormalTok{)}
\NormalTok{  tmp.data }\OtherTok{\textless{}{-}}\NormalTok{ data[validation\_ind,]}
  
\NormalTok{  validation\_set }\OtherTok{\textless{}{-}}\NormalTok{ tmp.data }\SpecialCharTok{|\textgreater{}} 
    \FunctionTok{semi\_join}\NormalTok{(train\_set, }\AttributeTok{by =} \StringTok{"movieId"}\NormalTok{) }\SpecialCharTok{|\textgreater{}} 
    \FunctionTok{semi\_join}\NormalTok{(train\_set, }\AttributeTok{by =} \StringTok{"userId"}\NormalTok{) }\SpecialCharTok{|\textgreater{}}
    \FunctionTok{as.data.frame}\NormalTok{()}
  
  \CommentTok{\# Add rows excluded from \textasciigrave{}validation\_set\textasciigrave{} into \textasciigrave{}train\_set\textasciigrave{}}
\NormalTok{  tmp.excluded }\OtherTok{\textless{}{-}} \FunctionTok{anti\_join}\NormalTok{(tmp.data, validation\_set)}
\NormalTok{  train\_set }\OtherTok{\textless{}{-}} \FunctionTok{rbind}\NormalTok{(train\_set, tmp.excluded)}

  \CommentTok{\# Return result datassets    }
  \FunctionTok{list}\NormalTok{(}\AttributeTok{train\_set =}\NormalTok{ train\_set, }
       \AttributeTok{validation\_set =}\NormalTok{ validation\_set)}
\NormalTok{\}}
\end{Highlighting}
\end{Shaded}

\begin{noteblock}
The complete source code of the \href{https://github.com/AzKurban-edX-DS/Capstone-MovieLens/blob/main/r/src/support-functions/data.helper.functions.R\#L17}{sample\_train\_validation\_sets} function is available in the \href{https://github.com/AzKurban-edX-DS/Capstone-MovieLens/blob/main/r/src/support-functions/data.helper.functions.R\#L3}{Data processing functions} section of the \href{https://github.com/AzKurban-edX-DS/Capstone-MovieLens/blob/main/r/src/support-functions/data.helper.functions.R\#L17}{data.helper.functions.R} script on \emph{GitHub}.

\end{noteblock}

\newpage

\paragraph{\texorpdfstring{\href{https://github.com/AzKurban-edX-DS/Capstone-MovieLens/blob/main/r/src/support-functions/data.helper.functions.R\#L73}{splitGenreRows} Function}{splitGenreRows Function}}\label{func.splitGenreRows}

\hfill\break
Splits rows of the original dataset passed in the \hyperref[func.splitGenreRows.args.data]{data} argument for movies that belong to multiple genres.

\phantomsection\label{func.splitGenreRows.usage}
\subparagraph{Usage}\label{usage-41}

\hfill\break

\begin{Shaded}
\begin{Highlighting}[]
\NormalTok{edx.sgr }\OtherTok{\textless{}{-}} \FunctionTok{splitGenreRows}\NormalTok{(edx)}
\end{Highlighting}
\end{Shaded}

\begin{noteblock}
The above \href{https://github.com/AzKurban-edX-DS/Capstone-MovieLens/blob/main/r/src/support-functions/data.helper.functions.R\#L128}{line of code} splits rows of the \texttt{edx} dataset using the function \texttt{splitGenreRows}, which is called internally from the function \hyperref[func.make_source_datasets]{make\_source\_datasets} described above in section \hyperref[initializing-input-datasets]{Initializing Input Datasets} of \emph{this Appendix}.

\end{noteblock}

\phantomsection\label{func.splitGenreRows.args}
\subparagraph{Arguments}\label{arguments-38}

\hfill\break

\phantomsection\label{func.splitGenreRows.args.data}
\begin{itemize}
\tightlist
\item
  \textbf{data:} A dataset containing rows corresponding to movies that belong to multiple genres..
\end{itemize}

\phantomsection\label{func.splitGenreRows.details}
\subparagraph{Details}\label{details-39}

\hfill\break
The function is a wrapper for the \href{https://www.rdocumentation.org/packages/tidyr/versions/1.3.0/topics/separate_rows}{tidyr::separate\_rows} \texttt{R} function, which is called internally to operate.

\begin{noteblock}
This is an auxiliary function intended for use in the \hyperref[func.make_source_datasets]{make\_source\_datasets} function as shown in the \hyperref[func.splitGenreRows.usage]{Usage} section above of \emph{this Description}.

\end{noteblock}

\phantomsection\label{func.splitGenreRows.value}
\subparagraph{Value}\label{value-39}

\hfill\break

Dataset object with split rows (those belong to multiple genres) described in detail below in the \hyperref[appndx_b.edx.sgr]{\texttt{edx.sgr} Object} of the \hyperref[appndx_b]{Appendix B: Models Training Datasets}.

\phantomsection\label{func.splitGenreRows.code}
\subparagraph{Source Code}\label{source-code-43}

\hfill\break

The source code of the \href{https://github.com/AzKurban-edX-DS/Capstone-MovieLens/blob/main/r/src/support-functions/data.helper.functions.R\#L73}{splitGenreRows} function is shown below:

\begin{Shaded}
\begin{Highlighting}[]
\NormalTok{splitGenreRows }\OtherTok{\textless{}{-}} \ControlFlowTok{function}\NormalTok{(data)\{}
  \FunctionTok{put}\NormalTok{(}\StringTok{"Splitting dataset rows related to multiple genres..."}\NormalTok{)}
\NormalTok{  start }\OtherTok{\textless{}{-}} \FunctionTok{put\_start\_date}\NormalTok{()}
\NormalTok{  gs\_splitted }\OtherTok{\textless{}{-}}\NormalTok{ data }\SpecialCharTok{|\textgreater{}}
    \FunctionTok{separate\_rows}\NormalTok{(genres, }\AttributeTok{sep =} \StringTok{"}\SpecialCharTok{\textbackslash{}\textbackslash{}}\StringTok{|"}\NormalTok{)}
  \FunctionTok{put}\NormalTok{(}\StringTok{"Dataset rows related to multiple genres have been splitted to have single genre per row."}\NormalTok{)}
  \FunctionTok{put\_end\_date}\NormalTok{(start)}
\NormalTok{  gs\_splitted}
\NormalTok{\}}
\end{Highlighting}
\end{Shaded}

\begin{noteblock}
The source code of the \href{https://github.com/AzKurban-edX-DS/Capstone-MovieLens/blob/main/r/src/support-functions/data.helper.functions.R\#L73}{splitGenreRows} function is also available in the \href{https://github.com/AzKurban-edX-DS/Capstone-MovieLens/blob/main/r/src/support-functions/data.helper.functions.R\#L3}{Data processing functions} section of the \href{https://github.com/AzKurban-edX-DS/Capstone-MovieLens/blob/main/r/src/support-functions/data.helper.functions.R\#L73}{data.helper.functions.R} script on \emph{GitHub}.

\end{noteblock}

\newpage

\subsubsection{Models Training Functions}\label{models-training-functions}

\paragraph{\texorpdfstring{\href{https://github.com/AzKurban-edX-DS/Capstone-MovieLens/blob/main/r/src/support-functions/data.helper.functions.R\#L425}{union\_cv\_results} Function}{union\_cv\_results Function}}\label{func.union_cv_results}

\hfill\break

Aggregates the input data frames passed in the \hyperref[func.union_cv_results.args.data_list]{data\_list} argument into a single data frame.

\phantomsection\label{func.union_cv_results.usage}
\subparagraph{Usage}\label{usage-42}

\hfill\break

\begin{Shaded}
\begin{Highlighting}[]
\NormalTok{  user\_movie\_effects\_united }\OtherTok{\textless{}{-}} \FunctionTok{union\_cv\_results}\NormalTok{(user\_movie\_effects\_ls)}
\end{Highlighting}
\end{Shaded}

\begin{noteblock}
In the above \href{https://github.com/AzKurban-edX-DS/Capstone-MovieLens/blob/main/r/src/support-functions/UM-effect.functions.R\#L39}{code snippet}, the \texttt{union\_cv\_results} function is called internally from the \hyperref[func.train_user_movie_effect.cv]{train\_user\_movie\_effect.cv} function described in section \hyperref[ume-model-utility-functions]{UME Model: Utility Functions} of \emph{this Appendix}.

\end{noteblock}

\phantomsection\label{func.union_cv_results.args}
\subparagraph{Arguments}\label{arguments-39}

\hfill\break

\phantomsection\label{func.union_cv_results.args.data_list}
\begin{itemize}
\tightlist
\item
  \textbf{data\_list:} List of data frames representing the results of the \emph{K-Fold Cross-Validation} execution (as described in \href{https://rafalab.dfci.harvard.edu/dsbook-part-2/ml/resampling-methods.html\#k-fold-cross-validation}{Section 29.6 \emph{Cross validation / K-fold cross validation}} of the \emph{Course Textbook (New Edition)}\autocite{IDS2_29-6_KFCV}).
\end{itemize}

\phantomsection\label{func.union_cv_results.details}
\subparagraph{Details}\label{details-40}

\hfill\break

This function internally uses the \href{https://www.rdocumentation.org/packages/base/versions/3.6.2/topics/sets}{base::union} \texttt{R} function to combine the data from the list of data frames passed in the \hyperref[func.union_cv_results.args.data_list]{data\_list} argument into a single data frame for further processing.

The function is used to aggregate the results of the \emph{K-Fold Cross-Validation}, where the \(K\) value is determined by the length of the \hyperref[appndx_b.edx_CV]{\texttt{edx\_CV} Object} (described in \hyperref[appndx_b]{Appendix B: Models Training Datasets}). In \emph{this Project}, we use \(K = 5\)).

\begin{noteblock}
This is a utility function intended to be called internally from the other user-defined function, such as, for instance, the \hyperref[func.train_user_movie_effect.cv]{train\_user\_movie\_effect.cv}, as shown above in the \hyperref[func.union_cv_results.usage]{Usage} section of \emph{this Description}.

\end{noteblock}

\phantomsection\label{func.union_cv_results.value}
\subparagraph{Value}\label{value-40}

\hfill\break

A data frame that is a union of the input list of data frames.

\newpage

\phantomsection\label{func.union_cv_results.code}
\subparagraph{Source Code}\label{source-code-44}

\hfill\break

The source code of the \href{https://github.com/AzKurban-edX-DS/Capstone-MovieLens/blob/main/r/src/support-functions/data.helper.functions.R\#L425}{union\_cv\_results} function is shown below:

\begin{Shaded}
\begin{Highlighting}[]
\NormalTok{union\_cv\_results }\OtherTok{\textless{}{-}} \ControlFlowTok{function}\NormalTok{(data\_list) \{}
\NormalTok{  out\_dat }\OtherTok{\textless{}{-}}\NormalTok{ data\_list[[}\DecValTok{1}\NormalTok{]]}
  
  \ControlFlowTok{for}\NormalTok{ (i }\ControlFlowTok{in} \DecValTok{2}\SpecialCharTok{:}\NormalTok{CVFolds\_N)\{}
\NormalTok{    out\_dat }\OtherTok{\textless{}{-}} \FunctionTok{union}\NormalTok{(out\_dat, }
\NormalTok{                     data\_list[[i]])}
\NormalTok{  \}}
  
\NormalTok{  out\_dat}
\NormalTok{\}}
\end{Highlighting}
\end{Shaded}

\begin{noteblock}
The source code of the \href{https://github.com/AzKurban-edX-DS/Capstone-MovieLens/blob/main/r/src/support-functions/data.helper.functions.R\#L425}{union\_cv\_results} is also available in the \href{https://github.com/AzKurban-edX-DS/Capstone-MovieLens/blob/main/r/src/support-functions/data.helper.functions.R\#L424}{Model training} section of the \href{https://github.com/AzKurban-edX-DS/Capstone-MovieLens/blob/main/r/src/support-functions/data.helper.functions.R\#L425}{data.helper.functions.R} script on \emph{GitHub}.

\end{noteblock}

\newpage

\subsubsection{Data Visualization Functions}\label{data-visualization-functions}

\paragraph{\texorpdfstring{\href{https://github.com/AzKurban-edX-DS/Capstone-MovieLens/blob/main/r/src/support-functions/data.helper.functions.R\#L580}{data.plot} Function}{data.plot Function}}\label{func.data.plot}

\hfill\break

Plots a graph based on the dataset passed in the \hyperref[func.data.plot.args.data]{data} parameter, using both points and a line.

\phantomsection\label{func.data.plot.usage}
\subparagraph{\texorpdfstring{\href{https://github.com/AzKurban-edX-DS/Capstone-MovieLens/blob/main/r/src/capstone-movielens.main.R\#L1838}{Usage}}{Usage}}\label{usage-43}

\hfill\break

\begin{Shaded}
\begin{Highlighting}[]
\NormalTok{lss.UMGYDE.preset.degree2.result}\SpecialCharTok{$}\NormalTok{tuned.result }\SpecialCharTok{|\textgreater{}}
  \FunctionTok{data.plot}\NormalTok{(}\AttributeTok{title =} \StringTok{"UMGYDE Model, \textasciigrave{}loess\textasciigrave{} function args: range of \textasciigrave{}span\textasciigrave{} values for \textasciigrave{}degree = 2\textasciigrave{}"}\NormalTok{,}
              \AttributeTok{xname =} \StringTok{"parameter.value"}\NormalTok{, }
              \AttributeTok{yname =} \StringTok{"RMSE"}\NormalTok{, }
              \AttributeTok{xlabel =} \StringTok{"spans"}\NormalTok{, }
              \AttributeTok{ylabel =} \StringTok{"RMSE"}\NormalTok{)}
\end{Highlighting}
\end{Shaded}

\pandocbounded{\includegraphics[keepaspectratio]{capstone-movielens.report_files/figure-latex/main-script@l1838u-1.pdf}}

\newpage

\phantomsection\label{func.data.plot.args}
\subparagraph{Arguments}\label{arguments-40}

\hfill\break

\phantomsection\label{func.data.plot.args.data}
\begin{itemize}
\tightlist
\item
  \textbf{data:} Dataset to use for the plot;
\end{itemize}

\phantomsection\label{func.data.plot.args.title}
\begin{itemize}
\tightlist
\item
  \textbf{title:} Title of the plot;
\end{itemize}

\phantomsection\label{func.data.plot.args.xname}
\begin{itemize}
\tightlist
\item
  \textbf{xname:} The name of the dataset column used as the source of the \(x\) variable for the plot;
\end{itemize}

\phantomsection\label{func.data.plot.args.yname}
\begin{itemize}
\tightlist
\item
  \textbf{yname:} The name of the dataset column used as the source of the \(y\) variable for the plot;
\end{itemize}

\phantomsection\label{func.data.plot.args.xlabel}
\begin{itemize}
\tightlist
\item
  \textbf{xlabel:} (Optional, \emph{Character string}, \texttt{NULL} by default) The \(x\) axis label. If \texttt{NULL}, the value of the \texttt{xname} parameter is used for the label;
\end{itemize}

\phantomsection\label{func.data.plot.args.ylabel}
\begin{itemize}
\tightlist
\item
  \textbf{ylabel:} (Optional, \emph{Character string}, \texttt{NULL} by default) The \(y\) axis label. If \texttt{NULL}, the value of the \texttt{yname} parameter is used for the label.;
\end{itemize}

\phantomsection\label{func.data.plot.args.line_col}
\begin{itemize}
\tightlist
\item
  \textbf{line\_col:} (Optional, \emph{Character string}, \textbf{\emph{\textcolor{blue}{blue}}} by default) Line color.
\end{itemize}

\phantomsection\label{func.data.plot.args.normalize}
\begin{itemize}
\tightlist
\item
  \textbf{normalize:} (Optional, \emph{Boolean}, \texttt{FALSE} by default) If \texttt{TRUE}, deviations from the minimum \(y\) value are used to plot, rather than the \(y\) values themselves. Otherwise, the \(y\) values are used.
\end{itemize}

\phantomsection\label{func.data.plot.details}
\subparagraph{Details}\label{details-41}

\hfill\break

The function is a wrapper for the \href{https://ggplot2.tidyverse.org/reference/ggplot.html}{ggplot2::ggplot} function and is a simple and convenient tool for visualizing the data of \emph{this Project}.

\newpage

\phantomsection\label{func.data.plot.code}
\subparagraph{Source Code}\label{source-code-45}

\hfill\break

The soure code of the \href{https://github.com/AzKurban-edX-DS/Capstone-MovieLens/blob/main/r/src/support-functions/data.helper.functions.R\#L580}{data.plot} function is shown below:

\begin{Shaded}
\begin{Highlighting}[]
\NormalTok{data.plot }\OtherTok{\textless{}{-}} \ControlFlowTok{function}\NormalTok{(data, }
\NormalTok{                      title, }
\NormalTok{                      xname, }
\NormalTok{                      yname, }
                      \AttributeTok{xlabel =} \ConstantTok{NULL}\NormalTok{, }
                      \AttributeTok{ylabel =} \ConstantTok{NULL}\NormalTok{,}
                      \AttributeTok{line\_col =} \StringTok{"blue"}\NormalTok{,}
                      \AttributeTok{normalize =} \ConstantTok{FALSE}\NormalTok{) \{}
\NormalTok{  y }\OtherTok{\textless{}{-}}\NormalTok{ data[, yname]}
  
  \ControlFlowTok{if}\NormalTok{ (normalize) \{}
\NormalTok{    y }\OtherTok{\textless{}{-}}\NormalTok{ y }\SpecialCharTok{{-}} \FunctionTok{min}\NormalTok{(y)}
\NormalTok{  \}}
  
  \ControlFlowTok{if}\NormalTok{ (}\FunctionTok{is.null}\NormalTok{(xlabel)) \{}
\NormalTok{    xlabel }\OtherTok{=}\NormalTok{ xname}
\NormalTok{  \}}
  \ControlFlowTok{if}\NormalTok{ (}\FunctionTok{is.null}\NormalTok{(ylabel)) \{}
\NormalTok{    ylabel }\OtherTok{=}\NormalTok{ yname}
\NormalTok{  \}}
  
\NormalTok{  aes\_mapping }\OtherTok{\textless{}{-}} \FunctionTok{aes}\NormalTok{(}\AttributeTok{x =}\NormalTok{ data[, xname], }\AttributeTok{y =}\NormalTok{ y)}
  
\NormalTok{  data }\SpecialCharTok{|\textgreater{}} 
    \FunctionTok{ggplot}\NormalTok{(}\AttributeTok{mapping =}\NormalTok{ aes\_mapping) }\SpecialCharTok{+}
    \FunctionTok{ggtitle}\NormalTok{(title) }\SpecialCharTok{+}
    \FunctionTok{xlab}\NormalTok{(xlabel) }\SpecialCharTok{+}
    \FunctionTok{ylab}\NormalTok{(ylabel) }\SpecialCharTok{+}
    \FunctionTok{geom\_point}\NormalTok{() }\SpecialCharTok{+} 
    \FunctionTok{geom\_line}\NormalTok{(}\AttributeTok{color=}\NormalTok{line\_col)}
\NormalTok{\}}
\end{Highlighting}
\end{Shaded}

\begin{noteblock}
The source code of the \href{https://github.com/AzKurban-edX-DS/Capstone-MovieLens/blob/main/r/src/support-functions/data.helper.functions.R\#L580}{data.plot} is also available in the \href{https://github.com/AzKurban-edX-DS/Capstone-MovieLens/blob/main/r/src/support-functions/data.helper.functions.R\#L435}{Data Visualization} section of the \href{https://github.com/AzKurban-edX-DS/Capstone-MovieLens/blob/main/r/src/support-functions/data.helper.functions.R\#L580}{data.helper.functions.R} script on \emph{GitHub}.

\end{noteblock}

\newpage

\paragraph{\texorpdfstring{\href{https://github.com/AzKurban-edX-DS/Capstone-MovieLens/blob/main/r/src/support-functions/data.helper.functions.R\#L481}{data.plot.left.n} Function}{data.plot.left.n Function}}\label{func.data.plot.left.n}

\hfill\break

Plots a graph based on the first \(n\) rows (the \(n\) is specified by the \hyperref[func.data.plot.left.n.args.left.n]{left.n} parameter) of the dataset passed in the \hyperref[func.data.plot.left.n.args.data]{data} parameter, using both points and a line.

\begin{noteblock}
The dataset passed in the \hyperref[func.data.plot.left.n.args.data]{data} parameter is assumed to be sorted by the column used for the \(x\) axis (specified by the column name passed in the \hyperref[func.data.plot.left.n.args.xname]{xname} parameter).

\end{noteblock}

\phantomsection\label{func.data.plot.left.n.usage}
\subparagraph{Usage}\label{usage-44}

\hfill\break

\begin{Shaded}
\begin{Highlighting}[]
\NormalTok{data.plot.left\_detailed }\OtherTok{\textless{}{-}} \ControlFlowTok{function}\NormalTok{(data, }
                                    \AttributeTok{left.n =} \DecValTok{0}\NormalTok{,}
                                    \AttributeTok{title =} \ConstantTok{NULL}\NormalTok{,}
                                    \AttributeTok{title.left =} \ConstantTok{NULL}\NormalTok{,}
\NormalTok{                                    xname, }
\NormalTok{                                    yname, }
                                    \AttributeTok{xlabel1 =} \ConstantTok{NULL}\NormalTok{, }
                                    \AttributeTok{xlabel2 =} \ConstantTok{NULL}\NormalTok{, }
                                    \AttributeTok{ylabel1 =} \ConstantTok{NULL}\NormalTok{,}
                                    \AttributeTok{ylabel2 =} \ConstantTok{NULL}\NormalTok{,}
                                    \AttributeTok{line\_col1 =} \StringTok{"blue"}\NormalTok{,}
                                    \AttributeTok{line\_col2 =} \StringTok{"red"}\NormalTok{,}
                                    \AttributeTok{normalize =} \ConstantTok{FALSE}\NormalTok{) \{}
\CommentTok{\# ...}

\NormalTok{  p2 }\OtherTok{\textless{}{-}}\NormalTok{ data }\SpecialCharTok{|\textgreater{}}
    \FunctionTok{data.plot.left.n}\NormalTok{(}\AttributeTok{left.n =}\NormalTok{ left.n,}
                     \AttributeTok{title =}\NormalTok{ title.left,}
                     \AttributeTok{xname =}\NormalTok{ xname, }
                     \AttributeTok{yname =}\NormalTok{ yname, }
                     \AttributeTok{xlabel =}\NormalTok{ xlabel2, }
                     \AttributeTok{ylabel =}\NormalTok{ ylabel2,}
                     \AttributeTok{line\_col =}\NormalTok{ line\_col2,}
                     \AttributeTok{normalize =}\NormalTok{ normalize)}
  \FunctionTok{grid.arrange}\NormalTok{(p1, p2)}
\NormalTok{\}}
\end{Highlighting}
\end{Shaded}

\begin{noteblock}
In the \href{https://github.com/AzKurban-edX-DS/Capstone-MovieLens/blob/main/r/src/support-functions/data.helper.functions.R\#L470}{code snippet} above, the function \texttt{data.plot.left.n} is called internally from the \hyperref[func.data.plot.left_detailed]{data.plot.left\_detailed} function described below in this section.

\end{noteblock}

\newpage

\phantomsection\label{func.data.plot.left.n.args}
\subparagraph{Arguments}\label{arguments-41}

\hfill\break

\phantomsection\label{func.data.plot.left.n.args.data}
\begin{itemize}
\tightlist
\item
  \textbf{data:} Dataset to use for the plot.
\end{itemize}

\phantomsection\label{func.data.plot.left.n.args.left.n}
\begin{itemize}
\tightlist
\item
  \textbf{left.n:} (Optional, \emph{Integer}, \(0\) by default)If a positive numeric value, specifies the number of the first rows of the dataset to plot. Otherwise, the entire dataset data is set to plot.
\end{itemize}

\phantomsection\label{func.data.plot.left.n.args.title}
\begin{itemize}
\tightlist
\item
  \textbf{title:} Title of the plot.
\end{itemize}

\phantomsection\label{func.data.plot.left.n.args.xname}
\begin{itemize}
\tightlist
\item
  \textbf{xname:} The name of the dataset column used as the source of the \(x\) variable for the plot.
\end{itemize}

\phantomsection\label{func.data.plot.left.n.args.yname}
\begin{itemize}
\tightlist
\item
  \textbf{yname:} The name of the dataset column used as the source of the \(y\) variable for the plot.
\end{itemize}

\phantomsection\label{func.data.plot.left.n.args.xlabel}
\begin{itemize}
\tightlist
\item
  \emph{xlabel}:** The \(x\) axis label. If \texttt{NULL}, the value of the \texttt{xname} parameter is used for the label.
\end{itemize}

\phantomsection\label{func.data.plot.left.n.args.ylabel}
\begin{itemize}
\tightlist
\item
  \textbf{ylabel:} The \(y\) axis label. If \texttt{NULL}, the value of the \texttt{yname} parameter is used for the label.
\end{itemize}

\phantomsection\label{func.data.plot.left.n.args.line_col}
\begin{itemize}
\tightlist
\item
  \textbf{line\_col:} (Optional, \emph{Character string}, \textbf{\emph{\textcolor{red}{red}}} by default) Line color.
\end{itemize}

\phantomsection\label{func.data.plot.left.n.args.normalize}
\begin{itemize}
\tightlist
\item
  \textbf{normalize:} (Optional, \emph{Boolean}, \texttt{FALSE} by default) If \texttt{TRUE}, deviations from the minimum \(y\) value are used to plot, rather than the \(y\) values themselves. Otherwise, the \(y\) values are used.
\end{itemize}

\phantomsection\label{func.data.plot.left.n.details}
\subparagraph{Details}\label{details-42}

\hfill\break

The function is a wrapper for the \hyperref[func.data.plot]{data.plot} function described above and is a simple and convenient tool for more detailed visualization of the left part of the graph, initially built based on the entire data set passed in the \hyperref[func.data.plot.left.n.args.data]{data} parameter.

\begin{noteblock}
This is an auxiliary function that is used internally in the \hyperref[func.data.plot.left_detailed]{data.plot.left\_detailed} function (described below in this section) as shown above in the \hyperref[func.data.plot.left.n.usage]{Usage} section of \emph{this Description}.

\end{noteblock}

\newpage

\phantomsection\label{func.data.plot.left.n.code}
\subparagraph{Source Code}\label{source-code-46}

\hfill\break

The source code of the \href{https://github.com/AzKurban-edX-DS/Capstone-MovieLens/blob/main/r/src/support-functions/data.helper.functions.R\#L481}{data.plot.left.n} function is shown below:

\begin{Shaded}
\begin{Highlighting}[]
\NormalTok{data.plot.left.n }\OtherTok{\textless{}{-}} \ControlFlowTok{function}\NormalTok{(data, }
                             \AttributeTok{left.n =} \DecValTok{0}\NormalTok{,}
\NormalTok{                             title, }
\NormalTok{                             xname, }
\NormalTok{                             yname, }
\NormalTok{                             xlabel, }
\NormalTok{                             ylabel,}
                             \AttributeTok{line\_col =} \StringTok{"red"}\NormalTok{,}
                             \AttributeTok{normalize =} \ConstantTok{FALSE}\NormalTok{) \{}
\NormalTok{  x\_col }\OtherTok{\textless{}{-}}\NormalTok{ data[, xname] }
\NormalTok{  y\_col }\OtherTok{\textless{}{-}}\NormalTok{ data[, yname]}
  
  
\NormalTok{  data.left }\OtherTok{\textless{}{-}}\NormalTok{ data}
  
  \ControlFlowTok{if}\NormalTok{ (left.n }\SpecialCharTok{\textgreater{}} \DecValTok{0}\NormalTok{) \{}
\NormalTok{    data.left }\OtherTok{\textless{}{-}}\NormalTok{ data }\SpecialCharTok{|\textgreater{}} 
      \FunctionTok{head}\NormalTok{(left.n)}
\NormalTok{  \}}
  
\NormalTok{  data.left }\SpecialCharTok{|\textgreater{}}
    \FunctionTok{data.plot}\NormalTok{(}\AttributeTok{title =}\NormalTok{ title,}
              \AttributeTok{xname =}\NormalTok{ xname,}
              \AttributeTok{yname =}\NormalTok{ yname,}
              \AttributeTok{xlabel =}\NormalTok{ xlabel,}
              \AttributeTok{ylabel =}\NormalTok{ ylabel,}
              \AttributeTok{line\_col =}\NormalTok{ line\_col,}
              \AttributeTok{normalize =}\NormalTok{ normalize)}
\NormalTok{\}}
\end{Highlighting}
\end{Shaded}

\begin{noteblock}
The source code of the \href{https://github.com/AzKurban-edX-DS/Capstone-MovieLens/blob/main/r/src/support-functions/data.helper.functions.R\#L481}{data.plot.left.n} is also available in the \href{https://github.com/AzKurban-edX-DS/Capstone-MovieLens/blob/main/r/src/support-functions/data.helper.functions.R\#L435}{Data Visualization} section of the \href{https://github.com/AzKurban-edX-DS/Capstone-MovieLens/blob/main/r/src/support-functions/data.helper.functions.R\#L481}{data.helper.functions.R} script on \emph{GitHub}.

\end{noteblock}

\newpage

\paragraph{\texorpdfstring{\href{https://github.com/AzKurban-edX-DS/Capstone-MovieLens/blob/main/r/src/support-functions/data.helper.functions.R\#L536}{data.plot.left\_detailed} Function}{data.plot.left\_detailed Function}}\label{func.data.plot.left_detailed}

\hfill\break

Plots two graphs, placing them vertically one above the other: the first one at the top is the \emph{main graph} plotted by the internally called auxiliary function \hyperref[func.data.plot]{data.plot} (described above in this section) based on the data passed in the \hyperref[func.data.plot.left_detailed.args.data]{data} argument, and the second one is the left part of the \emph{main graph}, based on the first \(n\) rows of the data used for the \emph{main graph}, and plotted on an \emph{enlarged scale} (the \(n\) value is passed in the \hyperref[func.data.plot.left.n.args.left.n]{left.n} argument).

\begin{noteblock}
The dataset passed in the \hyperref[func.data.plot.left.n.args.data]{data} parameter is assumed to be sorted by the column used for the \(x\) axis (specified by the column name passed in the \hyperref[func.data.plot.left.n.args.xname]{xname} parameter).

\end{noteblock}

\phantomsection\label{func.data.plot.left_detailed.usage}
\subparagraph{\texorpdfstring{\href{https://github.com/AzKurban-edX-DS/Capstone-MovieLens/blob/main/r/src/capstone-movielens.main.R\#L1877}{Usage}}{Usage}}\label{usage-45}

\hfill\break

\begin{Shaded}
\begin{Highlighting}[]
\NormalTok{lss.UMGYDE.fine\_tune.degree2.result}\SpecialCharTok{$}\NormalTok{tuned.result }\SpecialCharTok{|\textgreater{}}
  \FunctionTok{data.plot.left\_detailed}\NormalTok{(}\AttributeTok{title =} \StringTok{"Fine{-}tuned UMGYDE Model with \textasciigrave{}loess\textasciigrave{} parameter: \textasciigrave{}degree = 2\textasciigrave{}"}\NormalTok{, }
                             \AttributeTok{title.left =} \StringTok{"Left Part of the Chart Above (Zoomed in)"}\NormalTok{,}
                             \AttributeTok{left.n =} \DecValTok{8}\NormalTok{,}
                             \AttributeTok{xname =} \StringTok{"parameter.value"}\NormalTok{, }
                             \AttributeTok{yname =} \StringTok{"RMSE"}\NormalTok{, }
                             \AttributeTok{xlabel1 =} \StringTok{"spans"}\NormalTok{, }
                             \AttributeTok{ylabel1 =} \StringTok{"RMSE"}\NormalTok{)}
\end{Highlighting}
\end{Shaded}

\pandocbounded{\includegraphics[keepaspectratio]{capstone-movielens.report_files/figure-latex/main-script@l1877u-1.pdf}}

\newpage

\phantomsection\label{func.data.plot.left_detailed.args}
\subparagraph{Arguments}\label{arguments-42}

\hfill\break

\phantomsection\label{func.data.plot.left_detailed.args.data}
\begin{itemize}
\tightlist
\item
  \textbf{data:} Dataset to use for the plot.
\end{itemize}

\phantomsection\label{func.data.plot.left.n.args.left.n}
\begin{itemize}
\tightlist
\item
  \textbf{left.n:} If a positive numeric value, specifies the number of the first rows of the dataset to plot. Otherwise, the entire dataset data is set to plot.
\end{itemize}

\phantomsection\label{func.data.plot.left_detailed.args.title}
\begin{itemize}
\tightlist
\item
  \textbf{title:} Title of the plot.
\end{itemize}

\phantomsection\label{func.data.plot.left_detailed.args.xname}
\begin{itemize}
\tightlist
\item
  \textbf{xname:} The name of the dataset column used as the source of the \(x\) variable for the plot.
\end{itemize}

\phantomsection\label{func.data.plot.left_detailed.args.yname}
\begin{itemize}
\tightlist
\item
  \textbf{yname:} The name of the dataset column used as the source of the \(y\) variable for the plot.
\end{itemize}

\phantomsection\label{func.data.plot.left_detailed.args.xlabel1}
\begin{itemize}
\tightlist
\item
  \textbf{xlabel1:} (Optional, \emph{Character string}, \texttt{NULL} by default) The \(x\) axis label for the first plot. If \texttt{NULL}, the value of the \texttt{xname} parameter is used for the label.
\end{itemize}

\phantomsection\label{func.data.plot.left_detailed.args.xlabel2}
\begin{itemize}
\tightlist
\item
  \textbf{xlabel2:} (Optional, \emph{Character string}, \texttt{NULL} by default) The \(x\) axis label for the second plot. If \texttt{NULL}, the value of the \texttt{xname} parameter is used for the label.
\end{itemize}

\phantomsection\label{func.data.plot.left_detailed.args.ylabel1}
\begin{itemize}
\tightlist
\item
  \textbf{ylabel1:} (Optional, \emph{Character string}, \texttt{NULL} by default) The \(y\) axis label for the first plot. If \texttt{NULL}, the value of the \texttt{yname} parameter is used for the label.
\end{itemize}

\phantomsection\label{func.data.plot.left_detailed.args.ylabel2}
\begin{itemize}
\tightlist
\item
  \textbf{ylabel2:} (Optional, \emph{Character string}, \texttt{NULL} by default) The \(y\) axis label for the second plot. If \texttt{NULL}, the value of the \texttt{yname} parameter is used for the label.
\end{itemize}

\phantomsection\label{func.data.plot.left_detailed.args.line_col}
\begin{itemize}
\tightlist
\item
  \textbf{line\_col1} (Optional, \emph{Character string}, \textbf{\emph{\textcolor{blue}{blue}}} by default) The line color (main graqh).
\end{itemize}

\phantomsection\label{func.data.plot.left_detailed.args.line_col}
\begin{itemize}
\tightlist
\item
  \textbf{line\_col2} (Optional, \emph{Character string}, \textbf{\emph{\textcolor{red}{red}}} by default) The line color (second graph: the left part of the main graph, which is shown on an enlarged scale).
\end{itemize}

\phantomsection\label{func.data.plot.left_detailed.args.normalize}
\begin{itemize}
\tightlist
\item
  \textbf{normalize:} (Optional, \emph{Boolean}, \texttt{FALSE} by default) If \texttt{TRUE}, deviations from the minimum \(y\) value are used to plot, rather than the \(y\) values themselves. Otherwise, the \(y\) values are used.
\end{itemize}

\phantomsection\label{func.data.plot.left_detailed.details}
\subparagraph{Details}\label{details-43}

\hfill\break

Internally, this function calls the \hyperref[func.data.plot]{data.plot} and \hyperref[func.data.plot.left.n]{data.plot.left.n} functions (the both described above in this section) to build two graphs, which it then combines into a grid using \href{https://www.rdocumentation.org/packages/gridExtra/versions/2.3/topics/arrangeGrob}{gridExtra::grid.arrange} \texttt{R} function.

\phantomsection\label{func.data.plot.left_detailed.code}
\subparagraph{Source Code}\label{source-code-47}

\hfill\break

The soure code of the \href{https://github.com/AzKurban-edX-DS/Capstone-MovieLens/blob/main/r/src/support-functions/data.helper.functions.R\#L536}{data.plot.left\_detailed} function is shown below:

\begin{Shaded}
\begin{Highlighting}[]
\NormalTok{data.plot.left\_detailed }\OtherTok{\textless{}{-}} \ControlFlowTok{function}\NormalTok{(data, }
                                    \AttributeTok{left.n =} \DecValTok{0}\NormalTok{,}
                                    \AttributeTok{title =} \ConstantTok{NULL}\NormalTok{,}
                                    \AttributeTok{title.left =} \ConstantTok{NULL}\NormalTok{,}
\NormalTok{                                    xname, }
\NormalTok{                                    yname, }
                                    \AttributeTok{xlabel1 =} \ConstantTok{NULL}\NormalTok{, }
                                    \AttributeTok{xlabel2 =} \ConstantTok{NULL}\NormalTok{, }
                                    \AttributeTok{ylabel1 =} \ConstantTok{NULL}\NormalTok{,}
                                    \AttributeTok{ylabel2 =} \ConstantTok{NULL}\NormalTok{,}
                                    \AttributeTok{line\_col1 =} \StringTok{"blue"}\NormalTok{,}
                                    \AttributeTok{line\_col2 =} \StringTok{"red"}\NormalTok{,}
                                    \AttributeTok{normalize =} \ConstantTok{FALSE}\NormalTok{) \{}
  \ControlFlowTok{if}\NormalTok{(}\FunctionTok{is.null}\NormalTok{(xlabel1)) \{}
\NormalTok{    xlabel1 }\OtherTok{\textless{}{-}}\NormalTok{ xname}
\NormalTok{  \}}
  \ControlFlowTok{if}\NormalTok{(}\FunctionTok{is.null}\NormalTok{(xlabel2)) \{}
\NormalTok{    xlabel2 }\OtherTok{\textless{}{-}} \FunctionTok{str\_glue}\NormalTok{(xlabel1, }\StringTok{" (left part)"}\NormalTok{)}
\NormalTok{  \}}
  \ControlFlowTok{if}\NormalTok{(}\FunctionTok{is.null}\NormalTok{(ylabel1)) \{}
\NormalTok{    ylabel1 }\OtherTok{\textless{}{-}}\NormalTok{ yname}
\NormalTok{  \}}
  \ControlFlowTok{if}\NormalTok{(}\FunctionTok{is.null}\NormalTok{(ylabel2)) \{}
\NormalTok{    ylabel2 }\OtherTok{\textless{}{-}}\NormalTok{ ylabel1}
\NormalTok{  \}}
  
\NormalTok{  p1 }\OtherTok{\textless{}{-}}\NormalTok{ data }\SpecialCharTok{|\textgreater{}}
    \FunctionTok{data.plot}\NormalTok{(}\AttributeTok{title =}\NormalTok{ title,}
              \AttributeTok{xname =}\NormalTok{ xname, }
              \AttributeTok{yname =}\NormalTok{ yname, }
              \AttributeTok{xlabel =}\NormalTok{ xlabel1, }
              \AttributeTok{ylabel =}\NormalTok{ ylabel1,}
              \AttributeTok{line\_col =}\NormalTok{ line\_col1)}
  
\NormalTok{  p2 }\OtherTok{\textless{}{-}}\NormalTok{ data }\SpecialCharTok{|\textgreater{}}
    \FunctionTok{data.plot.left.n}\NormalTok{(}\AttributeTok{left.n =}\NormalTok{ left.n,}
                     \AttributeTok{title =}\NormalTok{ title.left,}
                     \AttributeTok{xname =}\NormalTok{ xname, }
                     \AttributeTok{yname =}\NormalTok{ yname, }
                     \AttributeTok{xlabel =}\NormalTok{ xlabel2, }
                     \AttributeTok{ylabel =}\NormalTok{ ylabel2,}
                     \AttributeTok{line\_col =}\NormalTok{ line\_col2,}
                     \AttributeTok{normalize =}\NormalTok{ normalize)}
  \FunctionTok{grid.arrange}\NormalTok{(p1, p2)}
\NormalTok{\}}
\end{Highlighting}
\end{Shaded}

\begin{noteblock}
The source code of the \href{https://github.com/AzKurban-edX-DS/Capstone-MovieLens/blob/main/r/src/support-functions/data.helper.functions.R\#L536}{data.plot.left\_detailed} is also available in the \href{https://github.com/AzKurban-edX-DS/Capstone-MovieLens/blob/main/r/src/support-functions/data.helper.functions.R\#L435}{Data Visualization} section of the \href{https://github.com/AzKurban-edX-DS/Capstone-MovieLens/blob/main/r/src/support-functions/data.helper.functions.R\#L536}{data.helper.functions.R} script on \emph{GitHub}.

\end{noteblock}

\newpage

\subsection{Models Training: Support Functions}\label{models-training-support-functions}

\hfill\break

\subsubsection{OMR Model: Helper Functions}\label{omr-model-helper-functions}

\hfill\break

\begin{noteblock}
The complete source code of the functions described in this section is available in the \href{https://github.com/AzKurban-edX-DS/Capstone-MovieLens/blob/main/r/src/support-functions/OMR-model.functions.R}{OMR-model.functions.R} script on \emph{GitHub}.

\end{noteblock}

\paragraph{\texorpdfstring{\href{https://github.com/AzKurban-edX-DS/Capstone-MovieLens/blob/main/r/src/support-functions/OMR-model.functions.R\#L2}{naive\_model\_MSEs} Function}{naive\_model\_MSEs Function}}\label{func.naive_model_MSEs}

\hfill\break

Validates the \emph{Overall Mean Rating (OMR) Model} with use of \emph{K-Fold Cross-Validation} method as described in \href{https://rafalab.dfci.harvard.edu/dsbook-part-2/ml/resampling-methods.html\#k-fold-cross-validation}{Section 29.6 \emph{Cross validation / K-fold cross validation}} of the \emph{Course Textbook (New Edition)}\autocite{IDS2_29-6_KFCV}.

\begin{noteblock}
The \(K\) value for the \emph{K-Fold Cross-Validation} is determined by the length of the \hyperref[appndx_b.edx_CV]{\texttt{edx\_CV} Object} described below in \hyperref[appndx_b]{Appendix B: Models Training Datasets} (in \emph{this Project}, we use \(K = 5\)).

\end{noteblock}

\phantomsection\label{func.naive_model_MSEs.usage}
\subparagraph{\texorpdfstring{\href{https://github.com/AzKurban-edX-DS/Capstone-MovieLens/blob/main/r/src/capstone-movielens.main.R\#L106}{Usage}}{Usage}}\label{usage-46}

\hfill\break

\begin{Shaded}
\begin{Highlighting}[]
\NormalTok{mu }\OtherTok{\textless{}{-}} \FunctionTok{mean}\NormalTok{(edx}\SpecialCharTok{$}\NormalTok{rating)}
\NormalTok{mu.MSEs }\OtherTok{\textless{}{-}} \FunctionTok{naive\_model\_MSEs}\NormalTok{(mu)}
\end{Highlighting}
\end{Shaded}

\phantomsection\label{func.naive_model_MSEs.args}
\subparagraph{Arguments}\label{arguments-43}

\hfill\break

\phantomsection\label{func.naive_model_MSEs.args.}
\begin{itemize}
\tightlist
\item
  \textbf{mu:} The \emph{Overall Mean Rating (OMR)} value.
\end{itemize}

\phantomsection\label{func.naive_model_MSEs.details}
\subparagraph{Details}\label{details-44}

\hfill\break

The function iterates through the \hyperref[appndx_b.edx_CV]{\texttt{edx\_CV} Object} items internally calling the user-defined helper function \hyperref[func.mse]{mse} (described above in section \hyperref[root-mean-squared-error-calculation]{(Root) Mean Squared Error Calculation} of \emph{this Appendix}) for each iteration.

\phantomsection\label{func.naive_model_MSEs.value}
\subparagraph{Value}\label{value-41}

\hfill\break

Numeric vector of size \(K\) containing the Mean Squared Error (MSE) values obtained from applying the \(K\)\emph{-Fold Cross-Validation} to the \emph{Overall Average Rating (OMR) Model}, where the \(K\) value is determined by the length of the \hyperref[appndx_b.edx_CV]{\texttt{edx\_CV} Object} (in \emph{this Project}, we use \(K = 5\)).

\phantomsection\label{func.naive_model_MSEs.code}
\subparagraph{Source Code}\label{source-code-48}

\hfill\break

The source code of the \href{https://github.com/AzKurban-edX-DS/Capstone-MovieLens/blob/main/r/src/support-functions/OMR-model.functions.R\#L2}{naive\_model\_MSEs} function is shown below:

\begin{Shaded}
\begin{Highlighting}[]
\NormalTok{naive\_model\_MSEs }\OtherTok{\textless{}{-}} \ControlFlowTok{function}\NormalTok{(mu) \{}
  \FunctionTok{sapply}\NormalTok{(edx\_CV, }\ControlFlowTok{function}\NormalTok{(cv\_item)\{}
    \FunctionTok{mse}\NormalTok{(cv\_item}\SpecialCharTok{$}\NormalTok{validation\_set}\SpecialCharTok{$}\NormalTok{rating }\SpecialCharTok{{-}}\NormalTok{ mu)}
\NormalTok{  \})}
\NormalTok{\}}
\end{Highlighting}
\end{Shaded}

\begin{noteblock}
The source code of the \href{https://github.com/AzKurban-edX-DS/Capstone-MovieLens/blob/main/r/src/support-functions/OMR-model.functions.R\#L2}{naive\_model\_MSEs} function is defined in the \href{https://github.com/AzKurban-edX-DS/Capstone-MovieLens/blob/main/r/src/support-functions/OMR-model.functions.R\#L2}{OMR-model.functions.R} script on \emph{GitHub}.

\end{noteblock}

\newpage

\paragraph{\texorpdfstring{\href{https://github.com/AzKurban-edX-DS/Capstone-MovieLens/blob/main/r/src/support-functions/OMR-model.functions.R\#L7}{naive\_model\_RMSE} Function}{naive\_model\_RMSE Function}}\label{func.naive_model_RMSE}

\hfill\break
Validates the \emph{Overall Mean Rating (OMR) Model} with use of \emph{K-Fold Cross-Validation} as described in \href{https://rafalab.dfci.harvard.edu/dsbook-part-2/ml/resampling-methods.html\#k-fold-cross-validation}{Section 29.6 \emph{Cross validation / K-fold cross validation}} of the \emph{Course Textbook (New Edition)}\autocite{IDS2_29-6_KFCV}.

\phantomsection\label{func.naive_model_RMSE.usage}
\subparagraph{\texorpdfstring{\href{https://github.com/AzKurban-edX-DS/Capstone-MovieLens/blob/main/r/src/capstone-movielens.main.R\#L155}{Usage}}{Usage}}\label{usage-47}

\hfill\break

\begin{Shaded}
\begin{Highlighting}[]
\NormalTok{  deviation }\OtherTok{\textless{}{-}} \FunctionTok{seq}\NormalTok{(}\DecValTok{0}\NormalTok{, }\DecValTok{6}\NormalTok{, }\FloatTok{0.1}\NormalTok{) }\SpecialCharTok{{-}} \DecValTok{3}

\NormalTok{  deviation.RMSE }\OtherTok{\textless{}{-}} \FunctionTok{sapply}\NormalTok{(deviation, }\ControlFlowTok{function}\NormalTok{(delta)\{}
    \FunctionTok{naive\_model\_RMSE}\NormalTok{(mu }\SpecialCharTok{+}\NormalTok{ delta)}
\NormalTok{  \})}
\end{Highlighting}
\end{Shaded}

\phantomsection\label{func.naive_model_RMSE.args}
\subparagraph{Arguments}\label{arguments-44}

\hfill\break

\phantomsection\label{func.naive_model_RMSE.args.}
\begin{itemize}
\tightlist
\item
  \textbf{mu:} The overall mean rating value.
\end{itemize}

\phantomsection\label{func.naive_model_RMSE.details}
\subparagraph{Details}\label{details-45}

\hfill\break

The \(K\) value for the \emph{K-Fold Cross-Validation} is determined by the length of the \hyperref[appndx_b.edx_CV]{\texttt{edx\_CV} Object} described below in \hyperref[appndx_b]{Appendix B: Models Training Datasets} (in \emph{this Project}, we use \(K = 5\)).

\begin{noteblock}
This function internally calls the \hyperref[func.naive_model_MSEs]{naive\_model\_MSEs} function described above and then calculates the square root of the mean of the obtained results.

\end{noteblock}

\phantomsection\label{func.naive_model_RMSE.value}
\subparagraph{Value}\label{value-42}

\hfill\break

The Root Mean Squared Errors (RMSE) value obtained from applying the \emph{K-Fold Cross-Validation} method to the \emph{Overall Average Rating (OMR) Model}, where the \(K\) value is determined by the length of the \hyperref[appndx_b.edx_CV]{\texttt{edx\_CV} Object} (in \emph{this Project}, we use \(K = 5\)).

\phantomsection\label{func.naive_model_RMSE.code}
\subparagraph{Source Code}\label{source-code-49}

\hfill\break

The source code of the \href{https://github.com/AzKurban-edX-DS/Capstone-MovieLens/blob/main/r/src/support-functions/OMR-model.functions.R\#L7}{naive\_model\_RMSE} function is shown below:

\begin{Shaded}
\begin{Highlighting}[]
\NormalTok{naive\_model\_RMSE }\OtherTok{\textless{}{-}} \ControlFlowTok{function}\NormalTok{(mu)\{}
  \FunctionTok{sqrt}\NormalTok{(}\FunctionTok{mean}\NormalTok{(}\FunctionTok{naive\_model\_MSEs}\NormalTok{(mu)))}
\NormalTok{\}}
\end{Highlighting}
\end{Shaded}

\begin{noteblock}
The source code of the \href{https://github.com/AzKurban-edX-DS/Capstone-MovieLens/blob/main/r/src/support-functions/OMR-model.functions.R\#L7}{naive\_model\_RMSE} function is defined in the \href{https://github.com/AzKurban-edX-DS/Capstone-MovieLens/blob/main/r/src/support-functions/OMR-model.functions.R\#L7}{OMR-model.functions.R} script on \emph{GitHub}.

\end{noteblock}

\newpage

\subsubsection{UME Model: Utility Functions}\label{ume-model-utility-functions}

\hfill\break

\begin{noteblock}
The complete source code of the functions described in this section is available in the \href{https://github.com/AzKurban-edX-DS/Capstone-MovieLens/blob/main/r/src/support-functions/UM-effect.functions.R\#L3}{Utility Functions} section of the \href{https://github.com/AzKurban-edX-DS/Capstone-MovieLens/blob/main/r/src/support-functions/UM-effect.functions.R\#L3}{UM-effect.functions.R} script on \emph{GitHub}.

\end{noteblock}

\paragraph{\texorpdfstring{\href{https://github.com/AzKurban-edX-DS/Capstone-MovieLens/blob/main/r/src/support-functions/UM-effect.functions.R\#L4}{train\_user\_movie\_effect} Function}{train\_user\_movie\_effect Function}}\label{func.train_user_movie_effect}

\hfill\break

Trains (and optionally \emph{regularizes} if the \hyperref[func.train_user_movie_effect.args.lambda]{lambda} argument value is non-default) the \emph{User+Movie Effect (UME) Model} using the training dataset passed in the \hyperref[func.train_user_movie_effect.args.train_set]{train\_set} argument.

\phantomsection\label{func.train_user_movie_effect.usage}
\subparagraph{\texorpdfstring{\href{https://github.com/AzKurban-edX-DS/Capstone-MovieLens/blob/main/r/src/capstone-movielens.main.R\#L676}{Usage}}{Usage}}\label{usage-48}

\hfill\break

\begin{Shaded}
\begin{Highlighting}[]
\NormalTok{  rglr.UM\_effect }\OtherTok{\textless{}{-}} \FunctionTok{train\_user\_movie\_effect}\NormalTok{(edx, UME.rglr.best\_lambda)}
\end{Highlighting}
\end{Shaded}

\begin{Shaded}
\begin{Highlighting}[]
\FunctionTok{str}\NormalTok{(rglr.UM\_effect)}
\end{Highlighting}
\end{Shaded}

\begin{verbatim}
## tibble [10,677 x 3] (S3: tbl_df/tbl/data.frame)
##  $ movieId: int [1:10677] 1 2 3 4 5 6 7 8 9 10 ...
##  $ b      : num [1:10677] 0.331 -0.305 -0.364 -0.599 -0.443 ...
##  $ n      : int [1:10677] 23790 10779 7028 1577 6400 12346 7259 821 2278 15187 ...
\end{verbatim}

\phantomsection\label{func.train_user_movie_effect.args}
\subparagraph{Arguments}\label{arguments-45}

\hfill\break

\phantomsection\label{func.train_user_movie_effect.args.train_set}
\begin{itemize}
\tightlist
\item
  \textbf{train\_set:} Training dataset;
\end{itemize}

\phantomsection\label{func.train_user_movie_effect.args.lambda}
\begin{itemize}
\tightlist
\item
  \textbf{lambda:} (Optional, \emph{Numeric}, \(0\) by default) Regularization parameter \(\lambda\).
\end{itemize}

\phantomsection\label{func.train_user_movie_effect.details}
\subparagraph{Details}\label{details-46}

\hfill\break

The \emph{Regularization technique} described in \href{https://rafalab.dfci.harvard.edu/dsbook-part-2/highdim/regularization.html\#penalized-least-squares}{Section 24.2 \emph{Penalized least squares}} of the \emph{Course Textbook (New Edition)}\autocite{IDS2_24-2} is applied to the model being trained only when the \emph{regularization parameter} \(\lambda\) passed in the \hyperref[func.train_user_movie_effect.args.lambda]{lambda} argument, has a value other than the default (to be more exact, greater than zero).

\begin{noteblock}
\emph{Regularization} is committed through the \emph{penalized regression} by computing the \emph{penalized mean} performed by the internally called \hyperref[func.mean_reg]{mean\_reg} function described above in section \hyperref[appndx_a.CHF.model_tuning_utils]{Model Tuning Utils} of \emph{this Appendix}.

\end{noteblock}

\phantomsection\label{func.train_user_movie_effect.value}
\subparagraph{Value}\label{value-43}

\hfill\break

Data frame object representing the trained (and also \emph{regularized} if \hyperref[func.train_user_movie_effect.args.lambda]{lambda} arg value other than default) \emph{UM Effect Model}.

\newpage

\phantomsection\label{func.train_user_movie_effect.code}
\subparagraph{Source Code}\label{source-code-50}

\hfill\break

The source code of the \href{https://github.com/AzKurban-edX-DS/Capstone-MovieLens/blob/main/r/src/support-functions/UM-effect.functions.R\#L4}{train\_user\_movie\_effect} function is shown below:

\begin{Shaded}
\begin{Highlighting}[]
\NormalTok{train\_user\_movie\_effect }\OtherTok{\textless{}{-}} \ControlFlowTok{function}\NormalTok{(train\_set, }\AttributeTok{lambda =} \DecValTok{0}\NormalTok{)\{}
  \ControlFlowTok{if}\NormalTok{ (}\FunctionTok{is.na}\NormalTok{(lambda)) \{}
    \FunctionTok{stop}\NormalTok{(}\StringTok{"Function: train\_user\_movie\_effect}
\StringTok{\textasciigrave{}lambda\textasciigrave{} is \textasciigrave{}NA\textasciigrave{}"}\NormalTok{)}
\NormalTok{  \}}

\NormalTok{  UM.effect }\OtherTok{\textless{}{-}}\NormalTok{ train\_set }\SpecialCharTok{|\textgreater{}}
    \FunctionTok{left\_join}\NormalTok{(edx.user\_effect, }\AttributeTok{by =} \StringTok{"userId"}\NormalTok{) }\SpecialCharTok{|\textgreater{}}
    \FunctionTok{mutate}\NormalTok{(}\AttributeTok{resid =}\NormalTok{ rating }\SpecialCharTok{{-}}\NormalTok{ (mu }\SpecialCharTok{+}\NormalTok{ a)) }\SpecialCharTok{|\textgreater{}} 
    \FunctionTok{group\_by}\NormalTok{(movieId) }\SpecialCharTok{|\textgreater{}}
    \FunctionTok{summarise}\NormalTok{(}\AttributeTok{b =} \FunctionTok{mean\_reg}\NormalTok{(resid, lambda), }\AttributeTok{n =} \FunctionTok{n}\NormalTok{())}
  
  \FunctionTok{stopifnot}\NormalTok{(}\SpecialCharTok{!}\FunctionTok{is.na}\NormalTok{(}\FunctionTok{mean}\NormalTok{(UM.effect}\SpecialCharTok{$}\NormalTok{b)))}
\NormalTok{  UM.effect}
\NormalTok{\}}
\end{Highlighting}
\end{Shaded}

\begin{noteblock}
The source code of the \href{https://github.com/AzKurban-edX-DS/Capstone-MovieLens/blob/main/r/src/support-functions/UM-effect.functions.R\#L4}{train\_user\_movie\_effect} function is also available in the \href{https://github.com/AzKurban-edX-DS/Capstone-MovieLens/blob/main/r/src/support-functions/UM-effect.functions.R\#L3}{Utility Functions} section of the \href{https://github.com/AzKurban-edX-DS/Capstone-MovieLens/blob/main/r/src/support-functions/UM-effect.functions.R\#L4}{UM-effect.functions.R} script on \emph{GitHub}.

\end{noteblock}

\newpage

\paragraph{\texorpdfstring{\href{https://github.com/AzKurban-edX-DS/Capstone-MovieLens/blob/main/r/src/support-functions/UM-effect.functions.R\#L19}{train\_user\_movie\_effect.cv} Function}{train\_user\_movie\_effect.cv Function}}\label{func.train_user_movie_effect.cv}

\hfill\break

Trains (and optionally \emph{regularizes} if the \hyperref[func.train_user_movie_effect.args.cv.lambda]{lambda} argument value is non-default) the \emph{User+Movie Effect (UME) Model} with use of \emph{K-Fold Cross-Validation} as described in \href{https://rafalab.dfci.harvard.edu/dsbook-part-2/ml/resampling-methods.html\#k-fold-cross-validation}{Section 29.6 \emph{Cross validation / K-fold cross validation}} of the \emph{Course Textbook (New Edition)}\autocite{IDS2_29-6_KFCV}.

\phantomsection\label{func.train_user_movie_effect.cv.usage}
\subparagraph{\texorpdfstring{\href{https://github.com/AzKurban-edX-DS/Capstone-MovieLens/blob/main/r/src/capstone-movielens.main.R\#L472}{Usage}}{Usage}}\label{usage-49}

\hfill\break

\begin{Shaded}
\begin{Highlighting}[]
\NormalTok{  cv.UM\_effect }\OtherTok{\textless{}{-}} \FunctionTok{train\_user\_movie\_effect.cv}\NormalTok{()}
\end{Highlighting}
\end{Shaded}

\begin{Shaded}
\begin{Highlighting}[]
\FunctionTok{str}\NormalTok{(cv.UM\_effect)}
\end{Highlighting}
\end{Shaded}

\begin{verbatim}
## tibble [10,677 x 3] (S3: tbl_df/tbl/data.frame)
##  $ movieId: int [1:10677] 1 2 3 4 5 6 7 8 9 10 ...
##  $ b      : num [1:10677] 0.335 -0.306 -0.365 -0.598 -0.444 ...
##  $ n      : num [1:10677] 18907 8593 5574 1253 5065 ...
\end{verbatim}

\phantomsection\label{func.train_user_movie_effect.cv.args}
\subparagraph{Arguments}\label{arguments-46}

\hfill\break

\phantomsection\label{func.train_user_movie_effect.cv.args.lambda}
\begin{itemize}
\tightlist
\item
  \textbf{lambda:} (Optional, \emph{Numeric}, \(0\) by default) Regularization parameter \(\lambda\).
\end{itemize}

\phantomsection\label{func.train_user_movie_effect.cv.details}
\subparagraph{Details}\label{details-47}

\hfill\break

The \emph{Regularization technique} described in \href{https://rafalab.dfci.harvard.edu/dsbook-part-2/highdim/regularization.html\#penalized-least-squares}{Section 24.2 \emph{Penalized least squares}} of the \emph{Course Textbook (New Edition)}\autocite{IDS2_24-2} is applied to the model being trained only when the \emph{regularization parameter} \(\lambda\) passed in the \hyperref[func.train_user_movie_effect.cv.args.lambda]{lambda} argument, has a value other than the default (to be more exact, greater than zero).

Internally, this function calls the helper function \hyperref[func.train_user_movie_effect]{train\_user\_movie\_effect} described above for each iteration of the \emph{K-Fold Cross-Validation} applied to the \hyperref[appndx_b.edx_CV]{\texttt{edx\_CV} Object} described below in \hyperref[appndx_b]{Appendix B: Models Training Datasets}. It also calls internally the \hyperref[func.union_cv_results]{union\_cv\_results} function described above in section \hyperref[models-training-functions]{Models Training Functions} of this \emph{Appendix} to aggregate the cross-validation results.

\begin{noteblock}
The \(K\) value for the \emph{K-Fold Cross-Validation} is determined by the length of the \hyperref[appndx_b.edx_CV]{\texttt{edx\_CV} Object} (in \emph{this Project}, we use \(K = 5\)).

\end{noteblock}

\phantomsection\label{func.train_user_movie_effect.cv.value}
\subparagraph{Value}\label{value-44}

\hfill\break

Data frame object representing the \emph{UM Effect Model} trained (and also \emph{regularized} if the \hyperref[func.train_user_movie_effect.cv.args.lambda]{lambda} argument value is other than the default) with use of \emph{K-Fold Cross-Validation}, where the \(K\) is the length of the \hyperref[appndx_b.edx_CV]{\texttt{edx\_CV} Object} (in \emph{this Project}, we use \(K = 5\)).

\newpage

\phantomsection\label{func.train_user_movie_effect.cv.code}
\subparagraph{Source Code}\label{source-code-51}

\hfill\break

The simplified version of the source code of the \href{https://github.com/AzKurban-edX-DS/Capstone-MovieLens/blob/main/r/src/support-functions/UM-effect.functions.R\#L19}{train\_user\_movie\_effect.cv} function is shown below:

\begin{Shaded}
\begin{Highlighting}[]
\NormalTok{train\_user\_movie\_effect.cv }\OtherTok{\textless{}{-}} \ControlFlowTok{function}\NormalTok{(}\AttributeTok{lambda =} \DecValTok{0}\NormalTok{)\{}
\CommentTok{\# ...}
\NormalTok{  start }\OtherTok{\textless{}{-}} \FunctionTok{put\_start\_date}\NormalTok{()}
\NormalTok{  user\_movie\_effects\_ls }\OtherTok{\textless{}{-}} \FunctionTok{lapply}\NormalTok{(edx\_CV, }\ControlFlowTok{function}\NormalTok{(cv\_fold\_dat)\{}
\NormalTok{    cv\_fold\_dat}\SpecialCharTok{$}\NormalTok{train\_set }\SpecialCharTok{|\textgreater{}} \FunctionTok{train\_user\_movie\_effect}\NormalTok{(lambda)}
\NormalTok{  \})}
  \FunctionTok{put\_end\_date}\NormalTok{(start)}
  \FunctionTok{put\_log}\NormalTok{(}\StringTok{"Function: train\_user\_movie\_effect.cv:}
\StringTok{User+Movie Effect list have been computed"}\NormalTok{)}
  
\NormalTok{  user\_movie\_effects\_united }\OtherTok{\textless{}{-}} \FunctionTok{union\_cv\_results}\NormalTok{(user\_movie\_effects\_ls)}

\NormalTok{  user\_movie\_effect }\OtherTok{\textless{}{-}}\NormalTok{ user\_movie\_effects\_united }\SpecialCharTok{|\textgreater{}}
    \FunctionTok{group\_by}\NormalTok{(movieId) }\SpecialCharTok{|\textgreater{}}
    \FunctionTok{summarise}\NormalTok{(}\AttributeTok{b =} \FunctionTok{mean}\NormalTok{(b), }\AttributeTok{n =} \FunctionTok{mean}\NormalTok{(n))}
  \CommentTok{\# ...}
\NormalTok{  user\_movie\_effect}
\NormalTok{\}}
\end{Highlighting}
\end{Shaded}

\begin{noteblock}
The complete source code of the \href{https://github.com/AzKurban-edX-DS/Capstone-MovieLens/blob/main/r/src/support-functions/UM-effect.functions.R\#L19}{train\_user\_movie\_effect.cv} function is available in the \href{https://github.com/AzKurban-edX-DS/Capstone-MovieLens/blob/main/r/src/support-functions/UM-effect.functions.R\#L3}{Utility Functions} section of the \href{https://github.com/AzKurban-edX-DS/Capstone-MovieLens/blob/main/r/src/support-functions/UM-effect.functions.R\#L19}{UM-effect.functions.R} script on \emph{GitHub}.

\end{noteblock}

\newpage

\paragraph{\texorpdfstring{\href{https://github.com/AzKurban-edX-DS/Capstone-MovieLens/blob/main/r/src/support-functions/UM-effect.functions.R\#L53}{calc\_user\_movie\_effect\_MSE} Function}{calc\_user\_movie\_effect\_MSE Function}}\label{func.calc_user_movie_effect_MSE}

\hfill\break

Validates the \emph{UM Effect (UME) Model} built and trained by the \hyperref[func.train_user_movie_effect]{train\_user\_movie\_effect} or \hyperref[func.train_user_movie_effect.cv]{train\_user\_movie\_effect.cv} function call (both described above in \emph{this Section}).

\phantomsection\label{func.calc_user_movie_effect_MSE.usage}
\subparagraph{Usage}\label{usage-50}

\hfill\break

\begin{Shaded}
\begin{Highlighting}[]
\NormalTok{  user\_movie\_effects\_MSEs }\OtherTok{\textless{}{-}} \FunctionTok{sapply}\NormalTok{(edx\_CV, }\ControlFlowTok{function}\NormalTok{(cv\_fold\_dat)\{}
\NormalTok{    cv\_fold\_dat}\SpecialCharTok{$}\NormalTok{validation\_set }\SpecialCharTok{|\textgreater{}} \FunctionTok{calc\_user\_movie\_effect\_MSE}\NormalTok{(um\_effect)}
\NormalTok{  \})}
\end{Highlighting}
\end{Shaded}

\begin{noteblock}
In the \href{https://github.com/AzKurban-edX-DS/Capstone-MovieLens/blob/main/r/src/support-functions/UM-effect.functions.R\#L71}{code snippet} above, the function \texttt{calc\_user\_movie\_effect\_MSE} is called internally from the \hyperref[func.calc_user_movie_effect_MSE.cv]{calc\_user\_movie\_effect\_MSE.cv} function (described below in \emph{this Section}) for each iteration of the \emph{K-Fold Cross-Validation}, to calculate the \emph{MSE} for the corresponding validation dataset of the \hyperref[appndx_b.edx_CV]{\texttt{edx\_CV} Object} (described in \hyperref[appndx_b]{Appendix B: Models Training Datasets}).

\end{noteblock}

\phantomsection\label{func.calc_user_movie_effect_MSE.args}
\subparagraph{Arguments}\label{arguments-47}

\hfill\break

\phantomsection\label{func.calc_user_movie_effect_MSE.args.test_set}
\begin{itemize}
\tightlist
\item
  \textbf{test\_set:} Validation dataset;
\end{itemize}

\phantomsection\label{func.calc_user_movie_effect_MSE.args.um_effect}
\begin{itemize}
\tightlist
\item
  \textbf{um\_effect:} Data frame object representing the \emph{UM Effect Model}.
\end{itemize}

\phantomsection\label{func.calc_user_movie_effect_MSE.details}
\subparagraph{Details}\label{details-48}

\hfill\break

The function calculates \emph{Mean Squared Error (MSE)} for the \emph{UME Model} passed in the \hyperref[func.calc_user_movie_effect_MSE.args.um_effect]{um\_effect} argument using the validation dataset passed in the \hyperref[func.calc_user_movie_effect_MSE.args.test_set]{test\_set} arg.

\begin{noteblock}
This is an auxiliary function intended to be called internally from the \hyperref[func.calc_user_movie_effect_MSE.cv]{calc\_user\_movie\_effect\_MSE.cv} (described below in \emph{this Section}) as shown in the \hyperref[func.calc_user_movie_effect_MSE.usage]{Usage} section above of \emph{this Description}.

\end{noteblock}

\phantomsection\label{func.calc_user_movie_effect_MSE.value}
\subparagraph{Value}\label{value-45}

\hfill\break

\emph{MSE} value calculated for the \emph{UME Model}.

\newpage

\phantomsection\label{func.calc_user_movie_effect_MSE.code}
\subparagraph{Source Code}\label{source-code-52}

\hfill\break

The source code of the \href{https://github.com/AzKurban-edX-DS/Capstone-MovieLens/blob/main/r/src/support-functions/UM-effect.functions.R\#L53}{calc\_user\_movie\_effect\_MSE} function is shown below:

\begin{Shaded}
\begin{Highlighting}[]
\NormalTok{calc\_user\_movie\_effect\_MSE }\OtherTok{\textless{}{-}} \ControlFlowTok{function}\NormalTok{(test\_set, um\_effect)\{}
\NormalTok{  mse.result }\OtherTok{\textless{}{-}}\NormalTok{ test\_set }\SpecialCharTok{|\textgreater{}}
    \FunctionTok{left\_join}\NormalTok{(edx.user\_effect, }\AttributeTok{by =} \StringTok{"userId"}\NormalTok{) }\SpecialCharTok{|\textgreater{}}
    \FunctionTok{left\_join}\NormalTok{(um\_effect, }\AttributeTok{by =} \StringTok{"movieId"}\NormalTok{) }\SpecialCharTok{|\textgreater{}}
    \FunctionTok{mutate}\NormalTok{(}\AttributeTok{resid =}\NormalTok{ rating }\SpecialCharTok{{-}} \FunctionTok{clamp}\NormalTok{(mu }\SpecialCharTok{+}\NormalTok{ a }\SpecialCharTok{+}\NormalTok{ b)) }\SpecialCharTok{|\textgreater{}} 
    \FunctionTok{pull}\NormalTok{(resid) }\SpecialCharTok{|\textgreater{}} \FunctionTok{mse}\NormalTok{()}
  
  \FunctionTok{stopifnot}\NormalTok{(}\SpecialCharTok{!}\FunctionTok{is.na}\NormalTok{(mse.result))}
\NormalTok{  mse.result}
\NormalTok{\}}
\end{Highlighting}
\end{Shaded}

\begin{noteblock}
The source code of the \href{https://github.com/AzKurban-edX-DS/Capstone-MovieLens/blob/main/r/src/support-functions/UM-effect.functions.R\#L53}{calc\_user\_movie\_effect\_MSE} function is also available in the \href{https://github.com/AzKurban-edX-DS/Capstone-MovieLens/blob/main/r/src/support-functions/UM-effect.functions.R\#L3}{Utility Functions} section of the \href{https://github.com/AzKurban-edX-DS/Capstone-MovieLens/blob/main/r/src/support-functions/UM-effect.functions.R\#L53}{UM-effect.functions.R} script on \emph{GitHub}.

\end{noteblock}

\newpage

\paragraph{\texorpdfstring{\href{https://github.com/AzKurban-edX-DS/Capstone-MovieLens/blob/main/r/src/support-functions/UM-effect.functions.R\#L67}{calc\_user\_movie\_effect\_MSE.cv} Function}{calc\_user\_movie\_effect\_MSE.cv Function}}\label{func.calc_user_movie_effect_MSE.cv}

\hfill\break

Validates the \emph{UM Effect (UME) Model} built and trained by the \hyperref[func.train_user_movie_effect]{train\_user\_movie\_effect} or \hyperref[func.train_user_movie_effect.cv]{train\_user\_movie\_effect.cv} function call (both described above in \emph{this Section}) with use of \emph{K-Fold Cross-Validation} (as described in \href{https://rafalab.dfci.harvard.edu/dsbook-part-2/ml/resampling-methods.html\#k-fold-cross-validation}{Section 29.6 \emph{Cross validation / K-fold cross validation}} of the \emph{Course Textbook (New Edition)}\autocite{IDS2_29-6_KFCV}).

\phantomsection\label{func.calc_user_movie_effect_MSE.cv.usage}
\subparagraph{Usage}\label{usage-51}

\hfill\break

\begin{Shaded}
\begin{Highlighting}[]
\NormalTok{user\_movie\_effects\_MSE }\OtherTok{\textless{}{-}} \FunctionTok{calc\_user\_movie\_effect\_MSE.cv}\NormalTok{(um\_effect)}
\end{Highlighting}
\end{Shaded}

\begin{noteblock}
In the \href{https://github.com/AzKurban-edX-DS/Capstone-MovieLens/blob/main/r/src/support-functions/UM-effect.functions.R\#L83}{code snippet} above, the function \emph{being described here} is called internally from the \hyperref[func.calc_user_movie_effect_RMSE.cv]{calc\_user\_movie\_effect\_RMSE.cv} function described below in \emph{this Section}.

\end{noteblock}

\phantomsection\label{func.calc_user_movie_effect_MSE.cv.args}
\subparagraph{Arguments}\label{arguments-48}

\hfill\break

\phantomsection\label{func.calc_user_movie_effect_MSE.cv.args.um_effect}
\begin{itemize}
\tightlist
\item
  \textbf{um\_effect:} Data frame object representing the \emph{UM Effect Model}.
\end{itemize}

\phantomsection\label{func.calc_user_movie_effect_MSE.cv.details}
\subparagraph{Details}\label{details-49}

\hfill\break

This function calculates \emph{Mean Squared Error (MSE)} for the \emph{UME Model} passed in the \hyperref[func.calc_user_movie_effect_MSE.cv.args.um_effect]{um\_effect} argument using the validation datasets contained in the \hyperref[appndx_b.edx_CV]{\texttt{edx\_CV} Object} described below in \hyperref[appndx_b]{Appendix B: Models Training Datasets}.

It calls internally the \hyperref[func.calc_user_movie_effect_MSE]{calc\_user\_movie\_effect\_MSE} function for each iteration of the \emph{K-Fold Cross-Validation}, calculating the \emph{MSE} for the corresponding validation dataset of the \hyperref[appndx_b.edx_CV]{\texttt{edx\_CV} Object}, and then calculates the mean of the obtained results.

The \(K\) value for the \emph{K-Fold Cross-Validation} is determined by the length of the \hyperref[appndx_b.edx_CV]{\texttt{edx\_CV} Object} (in \emph{this Project}, we use \(K = 5\)).

\begin{noteblock}
This is an auxiliary function intended to be called internally from the \hyperref[func.calc_user_movie_effect_RMSE.cv]{calc\_user\_movie\_effect\_RMSE.cv} as shown in the \hyperref[func.calc_user_movie_effect_MSE.cv.usage]{Usage} section above of \emph{this Description}.

\end{noteblock}

\phantomsection\label{func.calc_user_movie_effect_MSE.cv.value}
\subparagraph{Value}\label{value-46}

\hfill\break

The \emph{MSE} value calculated for the \emph{UME Model} using \emph{K-Fold Cross-Validation} (see the \hyperref[func.calc_user_movie_effect_MSE.cv.details]{Details} section of \emph{this Description} above for more details).

\newpage

\phantomsection\label{func.calc_user_movie_effect_MSE.cv.code}
\subparagraph{Source Code}\label{source-code-53}

\hfill\break

The source code of the \href{https://github.com/AzKurban-edX-DS/Capstone-MovieLens/blob/main/r/src/support-functions/UM-effect.functions.R\#L67}{calc\_user\_movie\_effect\_MSE.cv} function is shown below:

\begin{Shaded}
\begin{Highlighting}[]
\NormalTok{calc\_user\_movie\_effect\_MSE.cv }\OtherTok{\textless{}{-}} \ControlFlowTok{function}\NormalTok{(um\_effect)\{}
  \FunctionTok{put\_log}\NormalTok{(}\StringTok{"Function: user\_movie\_effects\_MSE.cv:}
\StringTok{Computing the RMSE taking into account User+Movie Effects..."}\NormalTok{)}
\NormalTok{  start }\OtherTok{\textless{}{-}} \FunctionTok{put\_start\_date}\NormalTok{()}
\NormalTok{  user\_movie\_effects\_MSEs }\OtherTok{\textless{}{-}} \FunctionTok{sapply}\NormalTok{(edx\_CV, }\ControlFlowTok{function}\NormalTok{(cv\_fold\_dat)\{}
\NormalTok{    cv\_fold\_dat}\SpecialCharTok{$}\NormalTok{validation\_set }\SpecialCharTok{|\textgreater{}} \FunctionTok{calc\_user\_movie\_effect\_MSE}\NormalTok{(um\_effect)}
\NormalTok{  \})}
  \FunctionTok{put\_end\_date}\NormalTok{(start)}
  
  \FunctionTok{put\_log1}\NormalTok{(}\StringTok{"Function: user\_movie\_effects\_MSE.cv:}
\StringTok{MSE values have been plotted for the \%1{-}Fold Cross{-}Validation samples."}\NormalTok{, }
\NormalTok{           CVFolds\_N)}
  
  \FunctionTok{mean}\NormalTok{(user\_movie\_effects\_MSEs)}
\NormalTok{\}}
\end{Highlighting}
\end{Shaded}

\begin{noteblock}
The source code of the \href{https://github.com/AzKurban-edX-DS/Capstone-MovieLens/blob/main/r/src/support-functions/UM-effect.functions.R\#L67}{calc\_user\_movie\_effect\_MSE.cv} function is also available in the \href{https://github.com/AzKurban-edX-DS/Capstone-MovieLens/blob/main/r/src/support-functions/UM-effect.functions.R\#L3}{Utility Functions} section of the \href{https://github.com/AzKurban-edX-DS/Capstone-MovieLens/blob/main/r/src/support-functions/UM-effect.functions.R\#L67}{UM-effect.functions.R} script on \emph{GitHub}.

\end{noteblock}

\newpage

\paragraph{\texorpdfstring{\href{https://github.com/AzKurban-edX-DS/Capstone-MovieLens/blob/main/r/src/support-functions/UM-effect.functions.R\#L82}{calc\_user\_movie\_effect\_RMSE.cv} Function}{calc\_user\_movie\_effect\_RMSE.cv Function}}\label{func.calc_user_movie_effect_RMSE.cv}

\hfill\break

Validates the \emph{UM Effect (UME) Model} (built and trained by the \hyperref[func.train_user_movie_effect]{train\_user\_movie\_effect} or \hyperref[func.train_user_movie_effect.cv]{train\_user\_movie\_effect.cv} function call) with use of \emph{K-Fold Cross-Validation} as described in \href{https://rafalab.dfci.harvard.edu/dsbook-part-2/ml/resampling-methods.html\#k-fold-cross-validation}{Section 29.6 \emph{Cross validation / K-fold cross validation}} of the \emph{Course Textbook (New Edition)}\autocite{IDS2_29-6_KFCV}.

\phantomsection\label{func.calc_user_movie_effect_RMSE.cv.usage}
\subparagraph{\texorpdfstring{\href{https://github.com/AzKurban-edX-DS/Capstone-MovieLens/blob/main/r/src/capstone-movielens.main.R\#L513}{Usage}}{Usage}}\label{usage-52}

\hfill\break

\begin{Shaded}
\begin{Highlighting}[]
\NormalTok{cv.UM\_effect.RMSE }\OtherTok{\textless{}{-}} \FunctionTok{calc\_user\_movie\_effect\_RMSE.cv}\NormalTok{(cv.UM\_effect)}
\end{Highlighting}
\end{Shaded}

\begin{Shaded}
\begin{Highlighting}[]
\NormalTok{cv.UM\_effect.RMSE}
\end{Highlighting}
\end{Shaded}

\begin{verbatim}
## [1] 0.8732081
\end{verbatim}

\phantomsection\label{func.calc_user_movie_effect_RMSE.cv.args}
\subparagraph{Arguments}\label{arguments-49}

\hfill\break

\phantomsection\label{func.calc_user_movie_effect_RMSE.cv.args.um_effect}
\begin{itemize}
\tightlist
\item
  \textbf{um\_effect:} Data frame object representing the \emph{UM Effect Model}.
\end{itemize}

\phantomsection\label{func.calc_user_movie_effect_RMSE.cv.details}
\subparagraph{Details}\label{details-50}

\hfill\break

The function calculates \emph{Root Squared Error (RMSE)} for the \emph{UME Model} passed in the \hyperref[func.calc_user_movie_effect_RMSE.cv.args.um_effect]{um\_effect} argument using the validation datasets contained in the \hyperref[appndx_b.edx_CV]{\texttt{edx\_CV} Object} described in \hyperref[appndx_b]{Appendix B: Models Training Datasets}.

\begin{noteblock}
This function internally calls the \hyperref[func.calc_user_movie_effect_MSE.cv]{calc\_user\_movie\_effect\_MSE.cv} function described above, to calculate the \emph{MSE} for the \emph{UME Model} with use of \emph{K-Fold Cross-Validation}, and then calculates the square root of the result.

The \(K\) value for the \emph{K-Fold Cross-Validation} is determined by the length of the \hyperref[appndx_b.edx_CV]{\texttt{edx\_CV} Object} (in \emph{this Project}, we use \(K = 5\)).

\end{noteblock}

\phantomsection\label{func.calc_user_movie_effect_RMSE.cv.value}
\subparagraph{Value}\label{value-47}

\hfill\break

The \emph{RMSE} value calculated for the \emph{UME Model} using \emph{K-Fold Cross-Validation} (see the \hyperref[func.calc_user_movie_effect_RMSE.cv.details]{Details} section of \emph{this Description} above for more details).

\newpage

\phantomsection\label{func.calc_user_movie_effect_RMSE.cv.code}
\subparagraph{Source Code}\label{source-code-54}

\hfill\break

The source code of the \href{https://github.com/AzKurban-edX-DS/Capstone-MovieLens/blob/main/r/src/support-functions/UM-effect.functions.R\#L82}{calc\_user\_movie\_effect\_RMSE.cv} function is shown below:

\begin{Shaded}
\begin{Highlighting}[]
\NormalTok{calc\_user\_movie\_effect\_RMSE.cv }\OtherTok{\textless{}{-}} \ControlFlowTok{function}\NormalTok{(um\_effect)\{}
\NormalTok{  user\_movie\_effects\_MSE }\OtherTok{\textless{}{-}} \FunctionTok{calc\_user\_movie\_effect\_MSE.cv}\NormalTok{(um\_effect)}
\NormalTok{  um\_effect\_RMSE }\OtherTok{\textless{}{-}} \FunctionTok{sqrt}\NormalTok{(user\_movie\_effects\_MSE)}
  \FunctionTok{put\_log2}\NormalTok{(}\StringTok{"Function: user\_movie\_effects\_RMSE.cv:}
\StringTok{\%1{-}Fold Cross{-}Validation ultimate RMSE: \%2"}\NormalTok{, }\FunctionTok{length}\NormalTok{(edx\_CV), um\_effect\_RMSE)}
\NormalTok{  um\_effect\_RMSE}
\NormalTok{\}}
\end{Highlighting}
\end{Shaded}

\begin{noteblock}
The source code of the \href{https://github.com/AzKurban-edX-DS/Capstone-MovieLens/blob/main/r/src/support-functions/UM-effect.functions.R\#L82}{calc\_user\_movie\_effect\_RMSE.cv} function is also available in the \href{https://github.com/AzKurban-edX-DS/Capstone-MovieLens/blob/main/r/src/support-functions/UM-effect.functions.R\#L3}{Utility Functions} section of the \href{https://github.com/AzKurban-edX-DS/Capstone-MovieLens/blob/main/r/src/support-functions/UM-effect.functions.R\#L82}{UM-effect.functions.R} script on \emph{GitHub}.

\end{noteblock}

\newpage

\subsubsection{UME Model: Regularization}\label{ume-model-regularization-1}

\hfill\break

\paragraph{\texorpdfstring{\href{https://github.com/AzKurban-edX-DS/Capstone-MovieLens/blob/main/r/src/support-functions/UM-effect.functions.R\#L91}{regularize.test\_lambda.UM\_effect.cv} Function}{regularize.test\_lambda.UM\_effect.cv Function}}\label{func.regularize.test_lambda.UM_effect.cv}

\hfill\break

Validates the \emph{regularization parameter} \(\lambda\) passed in the \hyperref[func.regularize.test_lambda.UM_effect.cv.args.lambda]{lambda} argument for the \emph{User+Movie Effect (UME) Model} in the \emph{regularization process} using \emph{penalized approach} with use of \emph{K-Fold Cross-Validation}.

\phantomsection\label{func.regularize.test_lambda.UM_effect.cv.usage}
\subparagraph{Usage}\label{usage-53}

\hfill\break

\begin{Shaded}
\begin{Highlighting}[]
\NormalTok{  lambdas }\OtherTok{\textless{}{-}} \FunctionTok{seq}\NormalTok{(}\DecValTok{0}\NormalTok{, }\DecValTok{1}\NormalTok{, }\FloatTok{0.1}\NormalTok{)}
\NormalTok{  cv.UME.preset.result }\OtherTok{\textless{}{-}} 
    \FunctionTok{tune.model\_param}\NormalTok{(lambdas, regularize.test\_lambda.UM\_effect.cv)}
\end{Highlighting}
\end{Shaded}

\begin{Shaded}
\begin{Highlighting}[]
\FunctionTok{str}\NormalTok{(cv.UME.preset.result)}
\end{Highlighting}
\end{Shaded}

\begin{verbatim}
## List of 2
##  $ tuned.result:'data.frame':    8 obs. of  2 variables:
##   ..$ RMSE           : num [1:8] 0.873 0.873 0.873 0.873 0.873 ...
##   ..$ parameter.value: num [1:8] 0 0.1 0.2 0.3 0.4 0.5 0.6 0.7
##  $ best_result : Named num [1:2] 0.4 0.873
##   ..- attr(*, "names")= chr [1:2] "param.best_value" "best_RMSE"
\end{verbatim}

\begin{noteblock}
In the \href{https://github.com/AzKurban-edX-DS/Capstone-MovieLens/blob/main/r/src/capstone-movielens.main.R\#L571}{code snippet} above the function \emph{being described here} is passed in the \hyperref[func.tune.model_param.args.fn_tune.test.param_value]{fn\_tune.test.param\_value} argument to the \hyperref[func.tune.model_param]{tune.model\_param} function described above in section \hyperref[appndx_a.CHF.model_tuning_utils]{Model Tuning Utils} of \emph{this Appendix}.

Such a use of this function is also noted in the \hyperref[func.regularize.test_lambda.UM_effect.cv.details]{Details} section of \emph{this Description} below.

\end{noteblock}

\phantomsection\label{func.regularize.test_lambda.UM_effect.cv.args}
\subparagraph{Arguments}\label{arguments-50}

\hfill\break

\phantomsection\label{func.regularize.test_lambda.UM_effect.cv.args.lambda}
\begin{itemize}
\tightlist
\item
  \textbf{lambda:} \emph{Regularization parameter} \(\lambda\) used to validate the model in the \emph{regularization process}.
\end{itemize}

\phantomsection\label{func.regularize.test_lambda.UM_effect.cv.details}
\subparagraph{Details}\label{details-51}

\hfill\break

The function is used in the \emph{regularization process} of the \emph{UME Model} using the \emph{penalized approach} to regularize predictions for the given \emph{regularization parameter} \(\lambda\) (passed in the \hyperref[func.regularize.test_lambda.UM_effect.cv.args.lambda]{lambda} argument) as explained in \href{https://rafalab.dfci.harvard.edu/dsbook-part-2/highdim/regularization.html\#penalized-least-squares}{Section 24.2 \emph{Penalized least squares}} of the \emph{Course Textbook (New Edition)}\autocite{IDS2_24-2}.

Internally it uses the \hyperref[func.train_user_movie_effect.cv]{train\_user\_movie\_effect.cv} and \hyperref[func.calc_user_movie_effect_RMSE.cv]{calc\_user\_movie\_effect\_RMSE.cv} functions described above to validate the \emph{UME Model} for the given \(\lambda\) with use of the \emph{K-Fold Cross-Validation} method, as described in \href{https://rafalab.dfci.harvard.edu/dsbook-part-2/ml/resampling-methods.html\#k-fold-cross-validation}{Section 29.6 \emph{Cross validation / K-fold cross validation}} of the \emph{Course Textbook (New Edition)}\autocite{IDS2_29-6_KFCV}, where the value of \(K\) is determined by the length of the \hyperref[appndx_b.edx_CV]{\texttt{edx\_CV} Object} described below in \hyperref[appndx_b]{Appendix B: Models Training Datasets} (in \emph{this Project}, we use \(K = 5\)).

\begin{noteblock}
This function is an auxiliary function specifically designed to be passed in the \texttt{fn\_tune.test.param\_value} argument to the \hyperref[func.tune.model_param]{tune.model\_param} (as shown above in the \hyperref[func.regularize.test_lambda.UM_effect.cv.usage]{Usage} section of \emph{this Description}) and the \hyperref[func.model.tune.param_range]{model.tune.param\_range} functions described above in section \hyperref[appndx_a.CHF.model_tuning_utils]{Model Tuning Utils} of \emph{this Appendix}.

\end{noteblock}

\phantomsection\label{func.regularize.test_lambda.UM_effect.cv.value}
\subparagraph{Value}\label{value-48}

\hfill\break

The \emph{Root Mean Squared Errors (RMSE)} value computed for the UME Model applying the \emph{regularization parameter} \(\lambda\) passed in the \hyperref[func.regularize.test_lambda.UM_effect.cv.args.lambda]{lambda} argument.

\phantomsection\label{func.regularize.test_lambda.UM_effect.cv.code}
\subparagraph{Source Code}\label{source-code-55}

\hfill\break

The source code of the \href{https://github.com/AzKurban-edX-DS/Capstone-MovieLens/blob/main/r/src/support-functions/UM-effect.functions.R\#L91}{regularize.test\_lambda.UM\_effect.cv} function is shown below:

\begin{Shaded}
\begin{Highlighting}[]
\NormalTok{regularize.test\_lambda.UM\_effect.cv }\OtherTok{\textless{}{-}} \ControlFlowTok{function}\NormalTok{(lambda)\{}
  \ControlFlowTok{if}\NormalTok{ (}\FunctionTok{is.na}\NormalTok{(lambda)) \{}
    \FunctionTok{stop}\NormalTok{(}\StringTok{"Function: regularize.test\_lambda.UM\_effect.cv}
\StringTok{\textasciigrave{}lambda\textasciigrave{} is \textasciigrave{}NA\textasciigrave{}"}\NormalTok{)}
\NormalTok{  \}}
\NormalTok{  um\_effect }\OtherTok{\textless{}{-}} \FunctionTok{train\_user\_movie\_effect.cv}\NormalTok{(lambda)}
  \FunctionTok{calc\_user\_movie\_effect\_RMSE.cv}\NormalTok{(um\_effect)}
\NormalTok{\}}
\end{Highlighting}
\end{Shaded}

\begin{noteblock}
The source code of the \href{https://github.com/AzKurban-edX-DS/Capstone-MovieLens/blob/main/r/src/support-functions/UM-effect.functions.R\#L91}{regularize.test\_lambda.UM\_effect.cv} function is also available in the \href{https://github.com/AzKurban-edX-DS/Capstone-MovieLens/blob/main/r/src/support-functions/UM-effect.functions.R\#L90}{Regularization} section of the \href{https://github.com/AzKurban-edX-DS/Capstone-MovieLens/blob/main/r/src/support-functions/UM-effect.functions.R\#L91}{UM-effect.functions.R} script on \emph{GitHub}.

\end{noteblock}

\newpage

\subsubsection{UMGE Model: Utility Functions}\label{umge-model-utility-functions}

\hfill\break

\begin{noteblock}
The complete source code of the functions described in this section is available in the \href{https://github.com/AzKurban-edX-DS/Capstone-MovieLens/blob/main/r/src/support-functions/UMG-effect.functions.R\#L3}{Utility Functions} section of the \href{https://github.com/AzKurban-edX-DS/Capstone-MovieLens/blob/main/r/src/support-functions/UMG-effect.functions.R\#L3}{UMG-effect.functions.R} script on \emph{GitHub}.

\end{noteblock}

\paragraph{\texorpdfstring{\href{https://github.com/AzKurban-edX-DS/Capstone-MovieLens/blob/main/r/src/support-functions/UMG-effect.functions.R\#L42}{train\_user\_movie\_genre\_effect} Function}{train\_user\_movie\_genre\_effect Function}}\label{func.train_user_movie_genre_effect}

\hfill\break

Trains (and optionally \emph{regularizes} if the \hyperref[func.train_user_movie_genre_effect.args.lambda]{lambda} argument value is non-default) the \emph{User+Movie+Genre Effect (UMGE) Model} using the training sample of the \texttt{Movielens} dataset (passed in the \hyperref[func.train_user_movie_genre_effect.args.train.sgr]{train.sgr} argument), specially prepared to train the \emph{UMGE Model}.

\phantomsection\label{func.train_user_movie_genre_effect.usage}
\subparagraph{\texorpdfstring{\href{https://github.com/AzKurban-edX-DS/Capstone-MovieLens/blob/main/r/src/capstone-movielens.main.R\#L1086}{Usage}}{Usage}}\label{usage-54}

\hfill\break

\begin{Shaded}
\begin{Highlighting}[]
\NormalTok{  rglr.UMG\_effect }\OtherTok{\textless{}{-}}\NormalTok{ edx.sgr }\SpecialCharTok{|\textgreater{}} \FunctionTok{train\_user\_movie\_genre\_effect}\NormalTok{(UMGE.rglr.best\_lambda)}
\end{Highlighting}
\end{Shaded}

\begin{Shaded}
\begin{Highlighting}[]
\FunctionTok{str}\NormalTok{(rglr.UMG\_effect)}
\end{Highlighting}
\end{Shaded}

\begin{verbatim}
## tibble [10,677 x 2] (S3: tbl_df/tbl/data.frame)
##  $ movieId: int [1:10677] 1 2 3 4 5 6 7 8 9 10 ...
##  $ g      : num [1:10677] -7.48e-05 -7.24e-05 -8.12e-05 -6.48e-05 -1.14e-04 ...
\end{verbatim}

\phantomsection\label{func.train_user_movie_genre_effect.args}
\subparagraph{Arguments}\label{arguments-51}

\hfill\break

\phantomsection\label{func.train_user_movie_genre_effect.args.train.sgr}
\begin{itemize}
\tightlist
\item
  \textbf{train.sgr:} Training sample of the \texttt{Movielens} dataset with split rows belonging to multiple genres;
\end{itemize}

\phantomsection\label{func.train_user_movie_genre_effect.args.lambda}
\begin{itemize}
\tightlist
\item
  \textbf{lambda:} (Optional, \emph{Numeric}, \(0\) by default) Regularization parameter \(\lambda\).
\end{itemize}

\phantomsection\label{func.train_user_movie_genre_effect.details}
\subparagraph{Details}\label{details-52}

\hfill\break
The dataset passed in the \hyperref[func.train_user_movie_genre_effect.args.train.sgr]{train.sgr} argument is assumed to be the result returned by the \hyperref[func.splitGenreRows]{splitGenreRows} helper function described above in section \hyperref[data-processing-functions]{Data Processing Functions} of \emph{this Appendix}.

For example, the \hyperref[appndx_b.edx.sgr]{\texttt{edx.sgr} Object} is an appropriate dataset as the value of the \hyperref[func.train_user_movie_genre_effect.args.train.sgr]{train.sgr}.

The \emph{Regularization technique} described in \href{https://rafalab.dfci.harvard.edu/dsbook-part-2/highdim/regularization.html\#penalized-least-squares}{Section 24.2 \emph{Penalized least squares}} of the \emph{Course Textbook (New Edition)}\autocite{IDS2_24-2} is applied to the model being trained only when the \emph{regularization parameter} \(\lambda\) passed in the \hyperref[func.train_user_movie_genre_effect.args.lambda]{lambda} argument, has a value other than the default (to be more exact, greater than zero).

\begin{noteblock}
\emph{Regularization} is committed through the \emph{penalized regression} by computing the \emph{penalized mean} performed by the internally called \hyperref[func.mean_reg]{mean\_reg} function described above in section \hyperref[appndx_a.CHF.model_tuning_utils]{Model Tuning Utils} of \emph{this Appendix}.

\end{noteblock}

\phantomsection\label{func.train_user_movie_genre_effect.value}
\subparagraph{Value}\label{value-49}

\hfill\break

Data frame object representing the trained (and also \emph{regularized} if \hyperref[func.train_user_movie_genre_effect.args.lambda]{lambda} arg value other than default) \emph{UMG Effect Model}.

\phantomsection\label{func.train_user_movie_genre_effect.code}
\subparagraph{Source Code}\label{source-code-56}

\hfill\break

The source code of the \href{https://github.com/AzKurban-edX-DS/Capstone-MovieLens/blob/main/r/src/support-functions/UMG-effect.functions.R\#L42}{train\_user\_movie\_genre\_effect} function is shown below:

\begin{Shaded}
\begin{Highlighting}[]
\NormalTok{train\_user\_movie\_genre\_effect }\OtherTok{\textless{}{-}} \ControlFlowTok{function}\NormalTok{(train.sgr, }\AttributeTok{lambda =} \DecValTok{0}\NormalTok{)\{}
  \ControlFlowTok{if}\NormalTok{ (}\FunctionTok{is.na}\NormalTok{(lambda)) \{}
    \FunctionTok{stop}\NormalTok{(}\StringTok{"Function: train\_user\_movie\_genre\_effect}
\StringTok{\textasciigrave{}lambda\textasciigrave{} is \textasciigrave{}NA\textasciigrave{}"}\NormalTok{)}
\NormalTok{  \}}

\NormalTok{  genre\_bias }\OtherTok{\textless{}{-}}\NormalTok{ train.sgr }\SpecialCharTok{|\textgreater{}}
      \FunctionTok{left\_join}\NormalTok{(edx.user\_effect, }\AttributeTok{by =} \StringTok{"userId"}\NormalTok{) }\SpecialCharTok{|\textgreater{}}
      \FunctionTok{left\_join}\NormalTok{(rglr.UM\_effect, }\AttributeTok{by =} \StringTok{"movieId"}\NormalTok{) }\SpecialCharTok{|\textgreater{}}
      \FunctionTok{mutate}\NormalTok{(}\AttributeTok{resid =}\NormalTok{ rating }\SpecialCharTok{{-}}\NormalTok{ (mu }\SpecialCharTok{+}\NormalTok{ a }\SpecialCharTok{+}\NormalTok{ b)) }\SpecialCharTok{|\textgreater{}}
      \FunctionTok{group\_by}\NormalTok{(genres) }\SpecialCharTok{|\textgreater{}}
      \FunctionTok{summarise}\NormalTok{(}\AttributeTok{g =} \FunctionTok{mean\_reg}\NormalTok{(resid, lambda), }\AttributeTok{n =} \FunctionTok{n}\NormalTok{())}

\NormalTok{    train.sgr }\SpecialCharTok{|\textgreater{}}
      \FunctionTok{left\_join}\NormalTok{(genre\_bias, }\AttributeTok{by =} \StringTok{"genres"}\NormalTok{) }\SpecialCharTok{|\textgreater{}}
      \FunctionTok{left\_join}\NormalTok{(rglr.UM\_effect, }\AttributeTok{by =} \StringTok{"movieId"}\NormalTok{) }\SpecialCharTok{|\textgreater{}}
      \FunctionTok{group\_by}\NormalTok{(movieId) }\SpecialCharTok{|\textgreater{}}
      \FunctionTok{summarise}\NormalTok{(}\AttributeTok{g =} \FunctionTok{mean}\NormalTok{(g))}
\NormalTok{\}}
\end{Highlighting}
\end{Shaded}

\begin{noteblock}
The source code of the \href{https://github.com/AzKurban-edX-DS/Capstone-MovieLens/blob/main/r/src/support-functions/UMG-effect.functions.R\#L42}{train\_user\_movie\_genre\_effect} function is also available in the \href{https://github.com/AzKurban-edX-DS/Capstone-MovieLens/blob/main/r/src/support-functions/UMG-effect.functions.R\#L3}{Utility Functions} section of the \href{https://github.com/AzKurban-edX-DS/Capstone-MovieLens/blob/main/r/src/support-functions/UMG-effect.functions.R\#L42}{UMG-effect.functions.R} script on \emph{GitHub}.

\end{noteblock}

\newpage

\paragraph{\texorpdfstring{\href{https://github.com/AzKurban-edX-DS/Capstone-MovieLens/blob/main/r/src/support-functions/UMG-effect.functions.R\#L61}{train\_user\_movie\_genre\_effect.cv} Function}{train\_user\_movie\_genre\_effect.cv Function}}\label{func.train_user_movie_genre_effect.cv}

\hfill\break

Trains (and optionally \emph{regularizes} if the \hyperref[func.train_user_movie_genre_effect.cv.args.lambda]{lambda} argument value is non-default) the \emph{User+Movie+Genre Effect (UMGE) Model} with use of \emph{K-Fold Cross-Validation} as described in \href{https://rafalab.dfci.harvard.edu/dsbook-part-2/ml/resampling-methods.html\#k-fold-cross-validation}{Section 29.6 \emph{Cross validation / K-fold cross validation}} of the \emph{Course Textbook (New Edition)}\autocite{IDS2_29-6_KFCV}.

\phantomsection\label{func.train_user_movie_genre_effect.cv.usage}
\subparagraph{Usage}\label{usage-55}

\hfill\break

In the following \href{https://github.com/AzKurban-edX-DS/Capstone-MovieLens/blob/main/r/src/capstone-movielens.main.R\#L902}{code snippet}, the function \emph{being described here} is called with the default value of the \hyperref[func.train_user_movie_genre_effect.cv.args.lambda]{lambda} argument (therefore, the \emph{Regularization technique} is not used in this call):

\begin{Shaded}
\begin{Highlighting}[]
\CommentTok{\# \textasciigrave{}lambda = 0\textasciigrave{} by default (the regularization technique is not applied)}
\NormalTok{cv.UMG\_effect }\OtherTok{\textless{}{-}} \FunctionTok{train\_user\_movie\_genre\_effect.cv}\NormalTok{()}
\end{Highlighting}
\end{Shaded}

\begin{Shaded}
\begin{Highlighting}[]
 \FunctionTok{str}\NormalTok{(cv.UMG\_effect)}
\end{Highlighting}
\end{Shaded}

\begin{verbatim}
## tibble [10,677 x 2] (S3: tbl_df/tbl/data.frame)
##  $ movieId: int [1:10677] 1 2 3 4 5 6 7 8 9 10 ...
##  $ g      : num [1:10677] -9.28e-06 -7.23e-05 -1.07e-04 -6.46e-05 4.56e-05 ...
\end{verbatim}

In the following \href{https://github.com/AzKurban-edX-DS/Capstone-MovieLens/blob/main/r/src/support-functions/UMG-effect.functions.R\#L154}{code snippet}, \emph{this function} is called internally from the other custom helper function \hyperref[func.regularize.test_lambda.UMG_effect.cv]{regularize.test\_lambda.UMG\_effect.cv} described below in section \hyperref[umge-model-regularization-1]{UMGE Model: Regularization} of \emph{this Appendix}.\newline
In this case, when the function is passed a non-default value in the \hyperref[func.train_user_movie_genre_effect.cv.args.lambda]{lambda} argument, the \emph{Regularization technique} is applied:

\begin{Shaded}
\begin{Highlighting}[]
\NormalTok{umg\_effect }\OtherTok{\textless{}{-}} \FunctionTok{train\_user\_movie\_genre\_effect.cv}\NormalTok{(lambda)}
\end{Highlighting}
\end{Shaded}

\phantomsection\label{func.train_user_movie_genre_effect.cv.args}
\subparagraph{Arguments}\label{arguments-52}

\hfill\break

\phantomsection\label{func.train_user_movie_genre_effect.cv.args.lambda}
\begin{itemize}
\tightlist
\item
  \textbf{lambda:} (Optional, \emph{Numeric}, \(0\) by default) Regularization parameter \(\lambda\).
\end{itemize}

\phantomsection\label{func.train_user_movie_genre_effect.cv.details}
\subparagraph{Details}\label{details-53}

\hfill\break

The \emph{Regularization technique} described in \href{https://rafalab.dfci.harvard.edu/dsbook-part-2/highdim/regularization.html\#penalized-least-squares}{Section 24.2 \emph{Penalized least squares}} of the \emph{Course Textbook (New Edition)}\autocite{IDS2_24-2} is applied to the model being trained only when the \emph{regularization parameter} \(\lambda\) passed in the \hyperref[func.train_user_movie_genre_effect.cv.args.lambda]{lambda} argument, has a value other than the default (to be more exact, greater than zero).

\begin{noteblock}
To use \emph{regularization}, in \emph{this Project}, the function is called internally from the other \emph{custom helper function} \hyperref[func.regularize.test_lambda.UMG_effect.cv]{regularize.test\_lambda.UMG\_effect.cv} (described below in section \hyperref[umge-model-regularization-1]{UMGE Model: Regularization} of \emph{this Appendix}) as shown above in the \hyperref[func.train_user_movie_genre_effect.cv.usage]{Usage} section of \emph{this Description}.

\end{noteblock}

Internally, this function calls the helper function \hyperref[func.train_user_movie_genre_effect]{train\_user\_movie\_genre\_effect} (described above in \emph{this Section}) for each iteration of the \emph{K-Fold Cross-Validation} applied to the \hyperref[appndx_b.edx_CV]{\texttt{edx\_CV} Object} described below in \hyperref[appndx_b]{Appendix B: Models Training Datasets}. It also calls internally the \hyperref[func.union_cv_results]{union\_cv\_results} function described above in section \hyperref[models-training-functions]{Models Training Functions} of this \emph{Appendix} to aggregate the cross-validation results.

\begin{noteblock}
The \(K\) value for the \emph{K-Fold Cross-Validation} is determined by the length of the \hyperref[appndx_b.edx_CV]{\texttt{edx\_CV} Object} (in \emph{this Project}, we use \(K = 5\)).

\end{noteblock}

\phantomsection\label{func.train_user_movie_genre_effect.cv.value}
\subparagraph{Value}\label{value-50}

\hfill\break

Data frame object representing the \emph{UMG Effect Model} trained (and also \emph{regularized} if the \hyperref[func.train_user_movie_genre_effect.cv.args.lambda]{lambda} argument value is other than the default) with use of \emph{K-Fold Cross-Validation}, where the \(K\) is the length of the \hyperref[appndx_b.edx_CV]{\texttt{edx\_CV} Object} (in \emph{this Project}, we use \(K = 5\)).

\phantomsection\label{func.train_user_movie_genre_effect.cv.code}
\subparagraph{Source Code}\label{source-code-57}

\hfill\break

The simplified version of the source code of the \href{https://github.com/AzKurban-edX-DS/Capstone-MovieLens/blob/main/r/src/support-functions/UMG-effect.functions.R\#L61}{train\_user\_movie\_genre\_effect.cv} function is shown below:

\begin{Shaded}
\begin{Highlighting}[]
\NormalTok{train\_user\_movie\_genre\_effect.cv }\OtherTok{\textless{}{-}} \ControlFlowTok{function}\NormalTok{(}\AttributeTok{lambda =} \DecValTok{0}\NormalTok{)\{}
\NormalTok{  user\_movie\_genre\_effects\_ls }\OtherTok{\textless{}{-}} \FunctionTok{lapply}\NormalTok{(kfold\_index, }\ControlFlowTok{function}\NormalTok{(fold\_i)\{}
\NormalTok{    cv\_fold\_dat }\OtherTok{\textless{}{-}}\NormalTok{ edx\_CV[[fold\_i]]}
\NormalTok{    umg\_effect }\OtherTok{\textless{}{-}}\NormalTok{ cv\_fold\_dat}\SpecialCharTok{$}\NormalTok{train.sgr }\SpecialCharTok{|\textgreater{}} \FunctionTok{train\_user\_movie\_genre\_effect}\NormalTok{(lambda)}
    
    \FunctionTok{put\_log2}\NormalTok{(}\StringTok{"User+Movie+Genre Effects have been computed for the Fold \%1 }
\StringTok{of the \%2{-}Fold Cross Validation samples."}\NormalTok{,}
\NormalTok{             fold\_i,}
\NormalTok{             CVFolds\_N)}
    
\NormalTok{    umg\_effect}
\NormalTok{  \})}

\NormalTok{  user\_movie\_genre\_effects\_united }\OtherTok{\textless{}{-}} \FunctionTok{union\_cv\_results}\NormalTok{(user\_movie\_genre\_effects\_ls)}

\NormalTok{  user\_movie\_genre\_effect }\OtherTok{\textless{}{-}}\NormalTok{ user\_movie\_genre\_effects\_united }\SpecialCharTok{|\textgreater{}}
    \FunctionTok{group\_by}\NormalTok{(movieId) }\SpecialCharTok{|\textgreater{}}
    \FunctionTok{summarise}\NormalTok{(}\AttributeTok{g =} \FunctionTok{mean}\NormalTok{(g))}
  
  \ControlFlowTok{if}\NormalTok{(lambda }\SpecialCharTok{==} \DecValTok{0}\NormalTok{) }\FunctionTok{put\_log}\NormalTok{(}\StringTok{"Function \textasciigrave{}train\_user\_movie\_genre\_effect.cv\textasciigrave{}:}
\StringTok{Training completed: User+Movie+Genre Effects model."}\NormalTok{)}
  \ControlFlowTok{else} \FunctionTok{put\_log1}\NormalTok{(}\StringTok{"Function \textasciigrave{}train\_user\_movie\_genre\_effect.cv\textasciigrave{}:}
\StringTok{Training completed: User+Movie+Genre Effects model for lambda: \%1..."}\NormalTok{,}
\NormalTok{                lambda)}
  
\NormalTok{  user\_movie\_genre\_effect}
\NormalTok{\}}
\end{Highlighting}
\end{Shaded}

\begin{noteblock}
The complete source code of the \href{https://github.com/AzKurban-edX-DS/Capstone-MovieLens/blob/main/r/src/support-functions/UMG-effect.functions.R\#L61}{train\_user\_movie\_genre\_effect.cv} function is available in the \href{https://github.com/AzKurban-edX-DS/Capstone-MovieLens/blob/main/r/src/support-functions/UMG-effect.functions.R\#L3}{Utility Functions} section of the \href{https://github.com/AzKurban-edX-DS/Capstone-MovieLens/blob/main/r/src/support-functions/UMG-effect.functions.R\#L61}{UMG-effect.functions.R} script on \emph{GitHub}.

\end{noteblock}

\newpage

\paragraph{\texorpdfstring{\href{https://github.com/AzKurban-edX-DS/Capstone-MovieLens/blob/main/r/src/support-functions/UMG-effect.functions.R\#L112}{calc\_user\_movie\_genre\_effect\_MSE} Function}{calc\_user\_movie\_genre\_effect\_MSE Function}}\label{func.calc_user_movie_genre_effect_MSE}

\hfill\break

Validates the \emph{UMG Effect (UMGE) Model} built and trained by the \hyperref[func.train_user_movie_genre_effect]{train\_user\_movie\_genre\_effect} or \hyperref[func.train_user_movie_genre_effect.cv]{train\_user\_movie\_genre\_effect.cv} function call (both described above in \emph{this Section}).

\phantomsection\label{func.calc_user_movie_genre_effect_MSE.usage}
\subparagraph{Usage}\label{usage-56}

\hfill\break

\begin{Shaded}
\begin{Highlighting}[]
\NormalTok{  user\_movie\_genre\_effects\_MSEs }\OtherTok{\textless{}{-}} \FunctionTok{sapply}\NormalTok{(edx\_CV, }\ControlFlowTok{function}\NormalTok{(cv\_dat)\{}
\NormalTok{    cv\_dat}\SpecialCharTok{$}\NormalTok{validation\_set }\SpecialCharTok{|\textgreater{}} \FunctionTok{calc\_user\_movie\_genre\_effect\_MSE}\NormalTok{(umg\_effect)}
\NormalTok{  \})}
\end{Highlighting}
\end{Shaded}

\begin{noteblock}
In the \href{https://github.com/AzKurban-edX-DS/Capstone-MovieLens/blob/main/r/src/support-functions/UMG-effect.functions.R\#L127}{code snippet} above, the function \emph{being described here} is called internally from the \hyperref[func.calc_user_movie_genre_effect_MSE.cv]{calc\_user\_movie\_genre\_effect\_MSE.cv} function (described below in \emph{this Section}) for each iteration of the \emph{K-Fold Cross-Validation}, to calculate the \emph{MSE} for the corresponding validation dataset of the \hyperref[appndx_b.edx_CV]{\texttt{edx\_CV} Object} (described in \hyperref[appndx_b]{Appendix B: Models Training Datasets}).

\end{noteblock}

\phantomsection\label{func.calc_user_movie_genre_effect_MSE.args}
\subparagraph{Arguments}\label{arguments-53}

\hfill\break

\phantomsection\label{func.calc_user_movie_genre_effect_MSE.args.test_set}
\begin{itemize}
\tightlist
\item
  \textbf{test\_set:} Validation dataset;
\end{itemize}

\phantomsection\label{func.calc_user_movie_genre_effect_MSE.args.umg_effect}
\begin{itemize}
\tightlist
\item
  \textbf{umg\_effect:} Data frame object representing the \emph{UMG Effect Model}.
\end{itemize}

\phantomsection\label{func.calc_user_movie_genre_effect_MSE.details}
\subparagraph{Details}\label{details-54}

\hfill\break

The function calculates \emph{Mean Squared Error (MSE)} for the \emph{UMGE Model} passed in the \hyperref[func.calc_user_movie_genre_effect_MSE.args.umg_effect]{umg\_effect} argument using the validation dataset passed in the \hyperref[func.calc_user_movie_genre_effect_MSE.args.test_set]{test\_set} arg.

\begin{noteblock}
This is an auxiliary function intended to be called internally from the \hyperref[func.calc_user_movie_genre_effect_MSE.cv]{calc\_user\_movie\_genre\_effect\_MSE.cv} (described below in \emph{this Section}) as shown in the \hyperref[func.calc_user_movie_genre_effect_MSE.usage]{Usage} section above of \emph{this Description}.

\end{noteblock}

\phantomsection\label{func.calc_user_movie_genre_effect_MSE.value}
\subparagraph{Value}\label{value-51}

\hfill\break

\emph{MSE} value calculated for the \emph{UMGE Model}.

\newpage

\phantomsection\label{func.calc_user_movie_genre_effect_MSE.code}
\subparagraph{Source Code}\label{source-code-58}

\hfill\break

The source code of the \href{https://github.com/AzKurban-edX-DS/Capstone-MovieLens/blob/main/r/src/support-functions/UMG-effect.functions.R\#L112}{calc\_user\_movie\_genre\_effect\_MSE} function is shown below:

\begin{Shaded}
\begin{Highlighting}[]
\NormalTok{calc\_user\_movie\_genre\_effect\_MSE }\OtherTok{\textless{}{-}} \ControlFlowTok{function}\NormalTok{(test\_set, umg\_effect)\{}
\NormalTok{  test\_set }\SpecialCharTok{|\textgreater{}}
    \FunctionTok{left\_join}\NormalTok{(edx.user\_effect, }\AttributeTok{by =} \StringTok{"userId"}\NormalTok{) }\SpecialCharTok{|\textgreater{}}
    \FunctionTok{left\_join}\NormalTok{(rglr.UM\_effect, }\AttributeTok{by =} \StringTok{"movieId"}\NormalTok{) }\SpecialCharTok{|\textgreater{}}
    \FunctionTok{left\_join}\NormalTok{(umg\_effect, }\AttributeTok{by =} \StringTok{"movieId"}\NormalTok{) }\SpecialCharTok{|\textgreater{}}
    \FunctionTok{mutate}\NormalTok{(}\AttributeTok{resid =}\NormalTok{ rating }\SpecialCharTok{{-}} \FunctionTok{clamp}\NormalTok{(mu }\SpecialCharTok{+}\NormalTok{ a }\SpecialCharTok{+}\NormalTok{ b }\SpecialCharTok{+}\NormalTok{ g)) }\SpecialCharTok{|\textgreater{}} 
    \FunctionTok{pull}\NormalTok{(resid) }\SpecialCharTok{|\textgreater{}} \FunctionTok{mse}\NormalTok{()}
\NormalTok{\}}
\end{Highlighting}
\end{Shaded}

\begin{noteblock}
The source code of the \href{https://github.com/AzKurban-edX-DS/Capstone-MovieLens/blob/main/r/src/support-functions/UMG-effect.functions.R\#L112}{calc\_user\_movie\_genre\_effect\_MSE} function is also available in the \href{https://github.com/AzKurban-edX-DS/Capstone-MovieLens/blob/main/r/src/support-functions/UMG-effect.functions.R\#L3}{Utility Functions} section of the \href{https://github.com/AzKurban-edX-DS/Capstone-MovieLens/blob/main/r/src/support-functions/UMG-effect.functions.R\#L112}{UMG-effect.functions.R} script on \emph{GitHub}.

\end{noteblock}

\newpage

\paragraph{\texorpdfstring{\href{https://github.com/AzKurban-edX-DS/Capstone-MovieLens/blob/main/r/src/support-functions/UMG-effect.functions.R\#L124}{calc\_user\_movie\_genre\_effect\_MSE.cv} Function}{calc\_user\_movie\_genre\_effect\_MSE.cv Function}}\label{func.calc_user_movie_genre_effect_MSE.cv}

\hfill\break

Validates the \emph{UMG Effect (UMGE) Model} built and trained by the \hyperref[func.train_user_movie_genre_effect]{train\_user\_movie\_genre\_effect} or \hyperref[func.train_user_movie_genre_effect.cv]{train\_user\_movie\_genre\_effect.cv} function call (both described above in \emph{this Section}) with use of \emph{K-Fold Cross-Validation} (as described in \href{https://rafalab.dfci.harvard.edu/dsbook-part-2/ml/resampling-methods.html\#k-fold-cross-validation}{Section 29.6 \emph{Cross validation / K-fold cross validation}} of the \emph{Course Textbook (New Edition)}\autocite{IDS2_29-6_KFCV}).

\phantomsection\label{func.calc_user_movie_genre_effect_MSE.cv.usage}
\subparagraph{Usage}\label{usage-57}

\hfill\break

\begin{Shaded}
\begin{Highlighting}[]
\NormalTok{  umg\_effect\_RMSE }\OtherTok{\textless{}{-}} \FunctionTok{sqrt}\NormalTok{(}\FunctionTok{calc\_user\_movie\_genre\_effect\_MSE.cv}\NormalTok{(umg\_effect))}
\end{Highlighting}
\end{Shaded}

\begin{noteblock}
In the \href{https://github.com/AzKurban-edX-DS/Capstone-MovieLens/blob/main/r/src/support-functions/UMG-effect.functions.R\#L139}{code snippet} above, the function \emph{being described here} is called internally from the \hyperref[func.calc_user_movie_genre_effect_RMSE.cv]{calc\_user\_movie\_genre\_effect\_RMSE.cv} function described below in \emph{this Section}.

\end{noteblock}

\phantomsection\label{func.calc_user_movie_genre_effect_MSE.cv.args}
\subparagraph{Arguments}\label{arguments-54}

\hfill\break

\phantomsection\label{func.calc_user_movie_genre_effect_MSE.cv.args.umg_effect}
\begin{itemize}
\tightlist
\item
  \textbf{umg\_effect:} Data frame object representing the \emph{UMG Effect Model}.
\end{itemize}

\phantomsection\label{func.calc_user_movie_genre_effect_MSE.cv.details}
\subparagraph{Details}\label{details-55}

\hfill\break

This function calculates \emph{MSE} for the \emph{UMGE Model} passed in the \hyperref[func.calc_user_movie_genre_effect_MSE.cv.args.umg_effect]{umg\_effect} argument using the validation datasets contained in the \hyperref[appndx_b.edx_CV]{\texttt{edx\_CV} Object} described below in \hyperref[appndx_b]{Appendix B: Models Training Datasets}.

It calls internally the \hyperref[func.calc_user_movie_genre_effect_MSE]{calc\_user\_movie\_genre\_effect\_MSE} function for each iteration of the \emph{K-Fold Cross-Validation}, calculating the \emph{MSE} for the corresponding validation dataset of the \hyperref[appndx_b.edx_CV]{\texttt{edx\_CV} Object}, and then calculates the mean of the obtained results.

The \(K\) value for the \emph{K-Fold Cross-Validation} is determined by the length of the \hyperref[appndx_b.edx_CV]{\texttt{edx\_CV} Object} (in \emph{this Project}, we use \(K = 5\)).

\begin{noteblock}
This is an auxiliary function intended to be called internally from the \hyperref[func.calc_user_movie_genre_effect_RMSE.cv]{calc\_user\_movie\_genre\_effect\_RMSE.cv} as shown in the \hyperref[func.calc_user_movie_genre_effect_MSE.cv.usage]{Usage} section above of \emph{this Description}.

\end{noteblock}

\phantomsection\label{func.calc_user_movie_genre_effect_MSE.cv.value}
\subparagraph{Value}\label{value-52}

\hfill\break

The \emph{MSE} value calculated for the \emph{UMGE Model} using \emph{K-Fold Cross-Validation} (see the \hyperref[func.calc_user_movie_genre_effect_MSE.cv.details]{Details} section of \emph{this Description} above for more details).

\phantomsection\label{func.calc_user_movie_genre_effect_MSE.cv.code}
\subparagraph{Source Code}\label{source-code-59}

\hfill\break

The simplified version of the source code of the \href{https://github.com/AzKurban-edX-DS/Capstone-MovieLens/blob/main/r/src/support-functions/UMG-effect.functions.R\#L124}{calc\_user\_movie\_genre\_effect\_MSE.cv} function is shown below:

\begin{Shaded}
\begin{Highlighting}[]
\NormalTok{calc\_user\_movie\_genre\_effect\_MSE.cv }\OtherTok{\textless{}{-}} \ControlFlowTok{function}\NormalTok{(umg\_effect)\{}
\NormalTok{  user\_movie\_genre\_effects\_MSEs }\OtherTok{\textless{}{-}} \FunctionTok{sapply}\NormalTok{(edx\_CV, }\ControlFlowTok{function}\NormalTok{(cv\_dat)\{}
\NormalTok{    cv\_dat}\SpecialCharTok{$}\NormalTok{validation\_set }\SpecialCharTok{|\textgreater{}} \FunctionTok{calc\_user\_movie\_genre\_effect\_MSE}\NormalTok{(umg\_effect)}
\NormalTok{  \})}

  \FunctionTok{mean}\NormalTok{(user\_movie\_genre\_effects\_MSEs)}
\NormalTok{\}}
\end{Highlighting}
\end{Shaded}

\begin{noteblock}
The completed source code of the \href{https://github.com/AzKurban-edX-DS/Capstone-MovieLens/blob/main/r/src/support-functions/UMG-effect.functions.R\#L124}{calc\_user\_movie\_genre\_effect\_MSE.cv} function is available in the \href{https://github.com/AzKurban-edX-DS/Capstone-MovieLens/blob/main/r/src/support-functions/UMG-effect.functions.R\#L3}{Utility Functions} section of the \href{https://github.com/AzKurban-edX-DS/Capstone-MovieLens/blob/main/r/src/support-functions/UMG-effect.functions.R\#L124}{UMG-effect.functions.R} script on \emph{GitHub}.

\end{noteblock}

\newpage

\paragraph{\texorpdfstring{\href{https://github.com/AzKurban-edX-DS/Capstone-MovieLens/blob/main/r/src/support-functions/UMG-effect.functions.R\#L138}{calc\_user\_movie\_genre\_effect\_RMSE.cv} Function}{calc\_user\_movie\_genre\_effect\_RMSE.cv Function}}\label{func.calc_user_movie_genre_effect_RMSE.cv}

\hfill\break

Validates the \emph{UMG Effect (UMGE) Model} (built and trained by the \hyperref[func.train_user_movie_genre_effect]{train\_user\_movie\_genre\_effect} or \hyperref[func.train_user_movie_genre_effect.cv]{train\_user\_movie\_genre\_effect.cv} function call) with use of \emph{K-Fold Cross-Validation} as described in \href{https://rafalab.dfci.harvard.edu/dsbook-part-2/ml/resampling-methods.html\#k-fold-cross-validation}{Section 29.6 \emph{Cross validation / K-fold cross validation}} of the \emph{Course Textbook (New Edition)}\autocite{IDS2_29-6_KFCV}.

\phantomsection\label{func.calc_user_movie_genre_effect_RMSE.cv.usage}
\subparagraph{\texorpdfstring{\href{https://github.com/AzKurban-edX-DS/Capstone-MovieLens/blob/main/r/src/capstone-movielens.main.R\#L931}{Usage}}{Usage}}\label{usage-58}

\hfill\break

\begin{Shaded}
\begin{Highlighting}[]
\DocumentationTok{\#\#\#\# Compute UMGE Model RMSE {-}{-}{-}{-}{-}{-}{-}{-}{-}{-}{-}{-}{-}{-}{-}{-}{-}{-}{-}{-}{-}{-}{-}{-}{-}{-}{-}{-}{-}{-}{-}{-}{-}{-}{-}{-}{-}{-}{-}{-}{-}{-}{-}{-}{-}{-}{-}{-}{-}{-}{-}}
\NormalTok{cv.UMG\_effect.RMSE }\OtherTok{\textless{}{-}} \FunctionTok{calc\_user\_movie\_genre\_effect\_RMSE.cv}\NormalTok{(cv.UMG\_effect)}
\end{Highlighting}
\end{Shaded}

\begin{Shaded}
\begin{Highlighting}[]
\NormalTok{cv.UMG\_effect.RMSE}
\end{Highlighting}
\end{Shaded}

\begin{verbatim}
## [1] 0.872973
\end{verbatim}

\phantomsection\label{func.calc_user_movie_genre_effect_RMSE.cv.args}
\subparagraph{Arguments}\label{arguments-55}

\hfill\break

\phantomsection\label{func.calc_user_movie_genre_effect_RMSE.cv.args.umg_effect}
\begin{itemize}
\tightlist
\item
  \textbf{umg\_effect:} Data frame object representing the \emph{UMG Effect Model}.
\end{itemize}

\phantomsection\label{func.calc_user_movie_genre_effect_RMSE.cv.details}
\subparagraph{Details}\label{details-56}

\hfill\break

The function calculates \emph{Root Squared Error (RMSE)} for the \emph{UMGE Model} passed in the \hyperref[func.calc_user_movie_genre_effect_RMSE.cv.args.umg_effect]{umg\_effect} argument using the validation datasets contained in the \hyperref[appndx_b.edx_CV]{\texttt{edx\_CV} Object} described below in \hyperref[appndx_b]{Appendix B: Models Training Datasets}.\newline
During its execution, it internally calls the \hyperref[func.calc_user_movie_genre_effect_MSE.cv]{calc\_user\_movie\_genre\_effect\_MSE.cv} function (described above in \emph{this Section}) to calculate the \emph{MSE} for the \emph{UMGE Model} (with use of \emph{K-Fold Cross-Validation}), and then calculates the square root of the result.

\begin{noteblock}
The \(K\) value for the \emph{K-Fold Cross-Validation} is determined by the length of the \hyperref[appndx_b.edx_CV]{\texttt{edx\_CV} Object} (in \emph{this Project}, we use \(K = 5\)).

\end{noteblock}

\phantomsection\label{func.calc_user_movie_genre_effect_RMSE.cv.value}
\subparagraph{Value}\label{value-53}

\hfill\break

The \emph{RMSE} value calculated for the \emph{UMGE Model} using \emph{K-Fold Cross-Validation} (see the \hyperref[func.calc_user_movie_genre_effect_RMSE.cv.details]{Details} section of \emph{this Description} above for more details).

\newpage

\phantomsection\label{func.calc_user_movie_genre_effect_RMSE.cv.code}
\subparagraph{Source Code}\label{source-code-60}

\hfill\break

The simplified version of the source code of the \href{https://github.com/AzKurban-edX-DS/Capstone-MovieLens/blob/main/r/src/support-functions/UMG-effect.functions.R\#L138}{calc\_user\_movie\_genre\_effect\_RMSE.cv} function is shown below:

\begin{Shaded}
\begin{Highlighting}[]
\NormalTok{calc\_user\_movie\_genre\_effect\_RMSE.cv }\OtherTok{\textless{}{-}} \ControlFlowTok{function}\NormalTok{(umg\_effect)\{}
  \FunctionTok{sqrt}\NormalTok{(}\FunctionTok{calc\_user\_movie\_genre\_effect\_MSE.cv}\NormalTok{(umg\_effect))}
\NormalTok{\}}
\end{Highlighting}
\end{Shaded}

\begin{noteblock}
The complete source code of the \href{https://github.com/AzKurban-edX-DS/Capstone-MovieLens/blob/main/r/src/support-functions/UMG-effect.functions.R\#L138}{calc\_user\_movie\_genre\_effect\_RMSE.cv} function is available in the \href{https://github.com/AzKurban-edX-DS/Capstone-MovieLens/blob/main/r/src/support-functions/UMG-effect.functions.R\#L3}{Utility Functions} section of the \href{https://github.com/AzKurban-edX-DS/Capstone-MovieLens/blob/main/r/src/support-functions/UMG-effect.functions.R\#L138}{UMG-effect.functions.R} script on \emph{GitHub}.

\end{noteblock}

\newpage

\subsubsection{UMGE Model: Regularization}\label{umge-model-regularization-1}

\hfill\break

\paragraph{\texorpdfstring{\href{https://github.com/AzKurban-edX-DS/Capstone-MovieLens/blob/main/r/src/support-functions/UMG-effect.functions.R\#L148}{regularize.test\_lambda.UMG\_effect.cv} Function}{regularize.test\_lambda.UMG\_effect.cv Function}}\label{func.regularize.test_lambda.UMG_effect.cv}

\hfill\break

Validates the \emph{regularization parameter} \(\lambda\) passed in the \hyperref[func.regularize.test_lambda.UMG_effect.cv.args.lambda]{lambda} argument for the \emph{User+Movie+Genre Effect (UMGE) Model} in the \emph{regularization process} using \emph{penalized approach} with use of \emph{K-Fold Cross-Validation}.

\phantomsection\label{func.regularize.test_lambda.UMG_effect.cv.usage}
\subparagraph{Usage}\label{usage-59}

\hfill\break

\begin{Shaded}
\begin{Highlighting}[]
\NormalTok{lambdas }\OtherTok{\textless{}{-}} \FunctionTok{seq}\NormalTok{(}\DecValTok{0}\NormalTok{, }\FloatTok{0.2}\NormalTok{, }\FloatTok{0.01}\NormalTok{)}
\NormalTok{cv.UMGE.preset.result }\OtherTok{\textless{}{-}} 
  \FunctionTok{tune.model\_param}\NormalTok{(lambdas, regularize.test\_lambda.UMG\_effect.cv)}
\end{Highlighting}
\end{Shaded}

\begin{Shaded}
\begin{Highlighting}[]
\FunctionTok{str}\NormalTok{(cv.UMGE.preset.result)}
\end{Highlighting}
\end{Shaded}

\begin{verbatim}
## List of 2
##  $ tuned.result:'data.frame':    8 obs. of  2 variables:
##   ..$ RMSE           : num [1:8] 0.873 0.873 0.873 0.873 0.873 ...
##   ..$ parameter.value: num [1:8] 0 0.01 0.02 0.03 0.04 0.05 0.06 0.07
##  $ best_result : Named num [1:2] 0.04 0.873
##   ..- attr(*, "names")= chr [1:2] "param.best_value" "best_RMSE"
\end{verbatim}

\begin{noteblock}
In the \href{https://github.com/AzKurban-edX-DS/Capstone-MovieLens/blob/main/r/src/capstone-movielens.main.R\#L978}{code snippet} above the function \emph{being described here} is passed in the \hyperref[func.tune.model_param.args.fn_tune.test.param_value]{fn\_tune.test.param\_value} argument to the \hyperref[func.tune.model_param]{tune.model\_param} function described above in section \hyperref[appndx_a.CHF.model_tuning_utils]{Model Tuning Utils} of \emph{this Appendix}.

Such a use of this function is also noted in the \hyperref[func.regularize.test_lambda.UMG_effect.cv.details]{Details} section of \emph{this Description} below.

\end{noteblock}

\phantomsection\label{func.regularize.test_lambda.UMG_effect.cv.args}
\subparagraph{Arguments}\label{arguments-56}

\hfill\break

\phantomsection\label{func.regularize.test_lambda.UMG_effect.cv.args.lambda}
\begin{itemize}
\tightlist
\item
  \textbf{lambda:} \emph{Regularization parameter} \(\lambda\) used to validate the model in the \emph{regularization process}.
\end{itemize}

\phantomsection\label{func.regularize.test_lambda.UMG_effect.cv.details}
\subparagraph{Details}\label{details-57}

\hfill\break

The function is used in the \emph{regularization process} of the \emph{UMGE Model} using the \emph{penalized approach} to regularize predictions for the given \emph{regularization parameter} \(\lambda\) (passed in the \hyperref[func.regularize.test_lambda.UMG_effect.cv.args.lambda]{lambda} argument) as explained in \href{https://rafalab.dfci.harvard.edu/dsbook-part-2/highdim/regularization.html\#penalized-least-squares}{Section 24.2 \emph{Penalized least squares}} of the \emph{Course Textbook (New Edition)}\autocite{IDS2_24-2}.

Internally it uses the \hyperref[func.train_user_movie_genre_effect.cv]{train\_user\_movie\_genre\_effect.cv} and \hyperref[func.calc_user_movie_genre_effect_RMSE.cv]{calc\_user\_movie\_genre\_effect\_RMSE.cv} functions (described above in \hyperref[umge-model-utility-functions]{UMGE Model: Utility Functions} section) to validate the \emph{UMGE Model} for the given \(\lambda\) with use of the \emph{K-Fold Cross-Validation} method, as described in \href{https://rafalab.dfci.harvard.edu/dsbook-part-2/ml/resampling-methods.html\#k-fold-cross-validation}{Section 29.6 \emph{Cross validation / K-fold cross validation}} of the \emph{Course Textbook (New Edition)}\autocite{IDS2_29-6_KFCV}, where the value of \(K\) is determined by the length of the \hyperref[appndx_b.edx_CV]{\texttt{edx\_CV} Object} described below in \hyperref[appndx_b]{Appendix B: Models Training Datasets} (in \emph{this Project}, we use \(K = 5\)).

\begin{noteblock}
This function is an auxiliary function specifically designed to be passed in the \texttt{fn\_tune.test.param\_value} argument to the \hyperref[func.tune.model_param]{tune.model\_param} (as shown above in the \hyperref[func.regularize.test_lambda.UMG_effect.cv.usage]{Usage} section of \emph{this Description}) and the \hyperref[func.model.tune.param_range]{model.tune.param\_range} functions described above in section \hyperref[appndx_a.CHF.model_tuning_utils]{Model Tuning Utils} of \emph{this Appendix}.

\end{noteblock}

\phantomsection\label{func.regularize.test_lambda.UMG_effect.cv.value}
\subparagraph{Value}\label{value-54}

\hfill\break

The \emph{Root Mean Squared Errors (RMSE)} value computed for the UMGE Model applying the \emph{regularization parameter} \(\lambda\) passed in the \hyperref[func.regularize.test_lambda.UMG_effect.cv.args.lambda]{lambda} argument.

\phantomsection\label{func.regularize.test_lambda.UMG_effect.cv.code}
\subparagraph{Source Code}\label{source-code-61}

\hfill\break

The simplified version of the source code of the \href{https://github.com/AzKurban-edX-DS/Capstone-MovieLens/blob/main/r/src/support-functions/UMG-effect.functions.R\#L148}{regularize.test\_lambda.UMG\_effect.cv} function is shown below:

\begin{Shaded}
\begin{Highlighting}[]
\NormalTok{regularize.test\_lambda.UMG\_effect.cv }\OtherTok{\textless{}{-}} \ControlFlowTok{function}\NormalTok{(lambda)\{}
\NormalTok{  umg\_effect }\OtherTok{\textless{}{-}} \FunctionTok{train\_user\_movie\_genre\_effect.cv}\NormalTok{(lambda)}
  \FunctionTok{calc\_user\_movie\_genre\_effect\_RMSE.cv}\NormalTok{(umg\_effect)}
\NormalTok{\}}
\end{Highlighting}
\end{Shaded}

\begin{noteblock}
The complete source code of the \href{https://github.com/AzKurban-edX-DS/Capstone-MovieLens/blob/main/r/src/support-functions/UMG-effect.functions.R\#L148}{regularize.test\_lambda.UMG\_effect.cv} function is available in the \href{https://github.com/AzKurban-edX-DS/Capstone-MovieLens/blob/main/r/src/support-functions/UMG-effect.functions.R\#L147}{Regularization} section of the \href{https://github.com/AzKurban-edX-DS/Capstone-MovieLens/blob/main/r/src/support-functions/UMG-effect.functions.R\#L148}{UMG-effect.functions.R} script on \emph{GitHub}.

\end{noteblock}

\newpage

\subsubsection{UMGYE Model: Utility Functions}\label{umgye-model-utility-functions}

\hfill\break

\begin{noteblock}
The complete source code of the functions described in this section is available in the \href{https://github.com/AzKurban-edX-DS/Capstone-MovieLens/blob/main/r/src/support-functions/UMGY-effect.functions.R\#L3}{Utility Functions} section of the \href{https://github.com/AzKurban-edX-DS/Capstone-MovieLens/blob/main/r/src/support-functions/UMGY-effect.functions.R\#L3}{UMGY-effect.functions.R} script on \emph{GitHub}.

\end{noteblock}

\paragraph{\texorpdfstring{\href{https://github.com/AzKurban-edX-DS/Capstone-MovieLens/blob/main/r/src/support-functions/UMGY-effect.functions.R\#L4}{calc\_date\_general\_effect} Function}{calc\_date\_general\_effect Function}}\label{func.calc_date_general_effect}

\hfill\break

This is an auxiliary function that calculates the \emph{Date General Effect} (optionally \emph{regularizes} if the \hyperref[func.calc_date_general_effect.args.lambda]{lambda} argument value is non-default) of the user ratings, using a \emph{training dataset} passed in the \hyperref[func.calc_date_general_effect.args.train_set]{train\_set} argument, and is intended for use in other custom helper functions (such as \hyperref[func.train_UMGY_effect]{train\_UMGY\_effect} described below in \emph{this Section}), as shown below in section \hyperref[func.calc_date_general_effect.usage]{Usage} of \emph{this Description}.

\phantomsection\label{func.calc_date_general_effect.usage}
\subparagraph{Usage}\label{usage-60}

\hfill\break

\begin{Shaded}
\begin{Highlighting}[]
\NormalTok{  train\_set }\SpecialCharTok{|\textgreater{}}
    \FunctionTok{calc\_date\_general\_effect}\NormalTok{(lambda) }\SpecialCharTok{|\textgreater{}}
    \FunctionTok{calc\_UMGY\_effect}\NormalTok{()}
\end{Highlighting}
\end{Shaded}

\begin{noteblock}
In the \href{https://github.com/AzKurban-edX-DS/Capstone-MovieLens/blob/main/r/src/support-functions/UMGY-effect.functions.R\#L82}{code snippet} above, the function \emph{being described here} is called, alongside the \hyperref[func.calc_UMGY_effect]{calc\_UMGY\_effect}, within the \hyperref[func.train_UMGY_effect]{train\_UMGY\_effect} (both described below in \emph{this Section}).

\end{noteblock}

\phantomsection\label{func.calc_date_general_effect.args}
\subparagraph{Arguments}\label{arguments-57}

\hfill\break

\phantomsection\label{func.calc_date_general_effect.args.train_set}
\begin{itemize}
\tightlist
\item
  \textbf{train\_set:} Training dataset;
\end{itemize}

\phantomsection\label{func.calc_date_general_effect.args.lambda}
\begin{itemize}
\tightlist
\item
  \textbf{lambda:} (Optional, \emph{Numeric}, \(0\) by default) Regularization parameter \(\lambda\).
\end{itemize}

\phantomsection\label{func.calc_date_general_effect.details}
\subparagraph{Details}\label{details-58}

\hfill\break

The \emph{Regularization technique} (described in \href{https://rafalab.dfci.harvard.edu/dsbook-part-2/highdim/regularization.html\#penalized-least-squares}{Section 24.2 \emph{Penalized least squares}} of the \emph{Course Textbook (New Edition)}\autocite{IDS2_24-2}) is applied to the model being trained only when the \emph{regularization parameter} \(\lambda\) passed in the \hyperref[func.calc_date_general_effect.args.lambda]{lambda} argument, has non-default value (to be more exact, greater than zero).

\begin{noteblock}
\emph{Regularization} is performed through \emph{penalized regression} by computing the \emph{penalized mean} using the internally called \hyperref[func.mean_reg]{mean\_reg} function (described above in section \hyperref[appndx_a.CHF.model_tuning_utils]{Model Tuning Utils} of \emph{this Appendix}).

\end{noteblock}

\phantomsection\label{func.calc_date_general_effect.value}
\subparagraph{Value}\label{value-55}

\hfill\break

\emph{Data frame} object, representing (optionally, \emph{Regularized}) \emph{Day General Effect}.

\phantomsection\label{func.calc_date_general_effect.code}
\subparagraph{Source Code}\label{source-code-62}

\hfill\break

Below is a slightly simplified version of the source code of the \href{https://github.com/AzKurban-edX-DS/Capstone-MovieLens/blob/main/r/src/support-functions/UMGY-effect.functions.R\#L4}{calc\_date\_general\_effect} function:

\begin{Shaded}
\begin{Highlighting}[]
\NormalTok{calc\_date\_general\_effect }\OtherTok{\textless{}{-}} \ControlFlowTok{function}\NormalTok{(train\_set, }\AttributeTok{lambda =} \DecValTok{0}\NormalTok{)\{}
\NormalTok{  dg\_effect }\OtherTok{\textless{}{-}}\NormalTok{ train\_set }\SpecialCharTok{|\textgreater{}} 
    \FunctionTok{left\_join}\NormalTok{(edx.user\_effect, }\AttributeTok{by =} \StringTok{"userId"}\NormalTok{) }\SpecialCharTok{|\textgreater{}}
    \FunctionTok{left\_join}\NormalTok{(rglr.UM\_effect, }\AttributeTok{by =} \StringTok{"movieId"}\NormalTok{) }\SpecialCharTok{|\textgreater{}}
    \FunctionTok{left\_join}\NormalTok{(rglr.UMG\_effect, }\AttributeTok{by =} \StringTok{"movieId"}\NormalTok{) }\SpecialCharTok{|\textgreater{}}
    \FunctionTok{left\_join}\NormalTok{(date\_days\_map, }\AttributeTok{by =} \StringTok{"timestamp"}\NormalTok{) }\SpecialCharTok{|\textgreater{}}
    \FunctionTok{mutate}\NormalTok{(}\AttributeTok{resid =}\NormalTok{ rating }\SpecialCharTok{{-}}\NormalTok{ (mu }\SpecialCharTok{+}\NormalTok{ a }\SpecialCharTok{+}\NormalTok{ b }\SpecialCharTok{+}\NormalTok{ g)) }\SpecialCharTok{|\textgreater{}}
  \FunctionTok{group\_by}\NormalTok{(days) }\SpecialCharTok{|\textgreater{}}
    \FunctionTok{summarise}\NormalTok{(}\AttributeTok{de =} \FunctionTok{mean\_reg}\NormalTok{(resid, lambda), }
              \AttributeTok{year =} \FunctionTok{mean}\NormalTok{(year))}
  
  \ControlFlowTok{if}\NormalTok{(lambda }\SpecialCharTok{==} \DecValTok{0}\NormalTok{) }\FunctionTok{put\_log}\NormalTok{(}\StringTok{"Function \textasciigrave{}calc\_date\_general\_effect\textasciigrave{}:}
\StringTok{Date Global Effect has been computed."}\NormalTok{)}
  \ControlFlowTok{else} \FunctionTok{put\_log1}\NormalTok{(}\StringTok{"Function \textasciigrave{}calc\_date\_general\_effect\textasciigrave{}:}
\StringTok{Date Global Effect has been computed for lambda: \%1..."}\NormalTok{,}
\NormalTok{                lambda)}
\NormalTok{  dg\_effect}
\NormalTok{\}}
\end{Highlighting}
\end{Shaded}

\begin{noteblock}
The complete source code of the \href{https://github.com/AzKurban-edX-DS/Capstone-MovieLens/blob/main/r/src/support-functions/UMGY-effect.functions.R\#L4}{calc\_date\_general\_effect} function is available in the \href{https://github.com/AzKurban-edX-DS/Capstone-MovieLens/blob/main/r/src/support-functions/UMGY-effect.functions.R\#L3}{Utility Functions} section of the \href{https://github.com/AzKurban-edX-DS/Capstone-MovieLens/blob/main/r/src/support-functions/UMGY-effect.functions.R\#L4}{UMGY-effect.functions.R} script on \emph{GitHub}.

\end{noteblock}

\newpage

\paragraph{\texorpdfstring{\href{https://github.com/AzKurban-edX-DS/Capstone-MovieLens/blob/main/r/src/support-functions/UMGY-effect.functions.R\#L32}{calc\_date\_general\_effect.cv} Function}{calc\_date\_general\_effect.cv Function}}\label{func.calc_date_general_effect.cv}

\hfill\break

This is an auxiliary function that calculates the \emph{Date General Effect} (optionally \emph{regularizes} if the \hyperref[func.calc_date_general_effect.cv.args.lambda]{lambda} argument value is non-default) of the user ratings, using the \emph{K-Fold Cross Validation} as described in \href{https://rafalab.dfci.harvard.edu/dsbook-part-2/ml/resampling-methods.html\#k-fold-cross-validation}{Section 29.6 \emph{Cross validation / K-fold cross validation}} of the \emph{Course Textbook (New Edition)}\autocite{IDS2_29-6_KFCV}, and is intended to be called within another custom helper function, the \hyperref[func.train_UMGY_effect.cv]{train\_UMGY\_effect.cv}) (described below in \emph{this Section}) as shown below in the \hyperref[func.calc_date_general_effect.cv.usage]{Usage} section of \emph{this Description}.

\phantomsection\label{func.calc_date_general_effect.cv.usage}
\subparagraph{Usage}\label{usage-61}

\hfill\break

\begin{Shaded}
\begin{Highlighting}[]
\FunctionTok{calc\_date\_general\_effect.cv}\NormalTok{(lambda) }\SpecialCharTok{|\textgreater{}} \FunctionTok{calc\_UMGY\_effect}\NormalTok{()  }
\end{Highlighting}
\end{Shaded}

\begin{noteblock}
In the above \href{https://github.com/AzKurban-edX-DS/Capstone-MovieLens/blob/main/r/src/support-functions/UMGY-effect.functions.R\#L92}{code snippet}, the function \emph{being described here} is called, alongside the \hyperref[func.calc_UMGY_effect]{calc\_UMGY\_effect}, within the \hyperref[func.train_UMGY_effect.cv]{train\_UMGY\_effect.cv} (both described below in \emph{this Section}).

\end{noteblock}

\phantomsection\label{func.calc_date_general_effect.cv.args}
\subparagraph{Arguments}\label{arguments-58}

\hfill\break

\phantomsection\label{func.calc_date_general_effect.cv.args.lambda}
\begin{itemize}
\tightlist
\item
  \textbf{lambda:} (Optional, \emph{Numeric}, \(0\) by default) Regularization parameter \(\lambda\).
\end{itemize}

\phantomsection\label{func.calc_date_general_effect.cv.details}
\subparagraph{Details}\label{details-59}

\hfill\break

The \emph{Regularization technique}, described in \href{https://rafalab.dfci.harvard.edu/dsbook-part-2/highdim/regularization.html\#penalized-least-squares}{Section 24.2 \emph{Penalized least squares}} of the \emph{Course Textbook (New Edition)}\autocite{IDS2_24-2}, is applied to the model being trained only when the \emph{regularization parameter} \(\lambda\), passed in the \hyperref[func.calc_date_general_effect.cv.args.lambda]{lambda} argument, has a non-default value (to be more exact, is a positive number).

Internally, this function calls the custom helper function \hyperref[func.calc_date_general_effect]{calc\_date\_general\_effect} (described above in \emph{this Section}) for each iteration of the \emph{K-Fold Cross-Validation} applied to the \hyperref[appndx_b.edx_CV]{\texttt{edx\_CV} Object}, described below in \hyperref[appndx_b]{Appendix B: Models Training Datasets}.
It also internally calls the user-defined auxiliary function \hyperref[func.union_cv_results]{union\_cv\_results}, described above in section \hyperref[models-training-functions]{Models Training Functions} of \emph{this Appendix}, to aggregate the cross-validation results.

\begin{noteblock}
The \(K\) value for the \emph{K-Fold Cross-Validation} method is determined by the length of the \hyperref[appndx_b.edx_CV]{\texttt{edx\_CV} Object} (in \emph{this Project}, we use \(K = 5\)).

\end{noteblock}

\phantomsection\label{func.calc_date_general_effect.cv.value}
\subparagraph{Value}\label{value-56}

\hfill\break

\emph{Data frame} object, representing (optionally, \emph{Regularized}) \emph{Date General Effect}.

\newpage

\phantomsection\label{func.calc_date_general_effect.cv.code}
\subparagraph{Source Code}\label{source-code-63}

\hfill\break

Below is a simplified version of the source code of the \href{https://github.com/AzKurban-edX-DS/Capstone-MovieLens/blob/main/r/src/support-functions/UMGY-effect.functions.R\#L32}{calc\_date\_general\_effect.cv} function:

\begin{Shaded}
\begin{Highlighting}[]
\NormalTok{calc\_date\_general\_effect.cv }\OtherTok{\textless{}{-}} \ControlFlowTok{function}\NormalTok{(}\AttributeTok{lambda =} \DecValTok{0}\NormalTok{)\{}

\NormalTok{  date\_general\_effect\_ls }\OtherTok{\textless{}{-}} \FunctionTok{lapply}\NormalTok{(edx\_CV,  }\ControlFlowTok{function}\NormalTok{(cv\_fold\_dat)\{}
\NormalTok{    cv\_fold\_dat}\SpecialCharTok{$}\NormalTok{train\_set }\SpecialCharTok{|\textgreater{}} \FunctionTok{calc\_date\_general\_effect}\NormalTok{(lambda)}
\NormalTok{  \})}
  \FunctionTok{put\_log1}\NormalTok{(}\StringTok{"Function \textasciigrave{}calc\_date\_general\_effect.cv\textasciigrave{}:}
\StringTok{Date Global Effect list has been computed for \%1{-}Fold Cross Validation samples."}\NormalTok{, }
\NormalTok{           CVFolds\_N)}
  
\NormalTok{  date\_general\_effect\_united }\OtherTok{\textless{}{-}} \FunctionTok{union\_cv\_results}\NormalTok{(date\_general\_effect\_ls)}

\NormalTok{  date\_general\_effect }\OtherTok{\textless{}{-}}\NormalTok{ date\_general\_effect\_united }\SpecialCharTok{|\textgreater{}}
    \FunctionTok{group\_by}\NormalTok{(days) }\SpecialCharTok{|\textgreater{}}
    \FunctionTok{summarise}\NormalTok{(}\AttributeTok{de =} \FunctionTok{mean}\NormalTok{(de, }\AttributeTok{na.rm =} \ConstantTok{TRUE}\NormalTok{), }\AttributeTok{year =} \FunctionTok{mean}\NormalTok{(year, }\AttributeTok{na.rm =} \ConstantTok{TRUE}\NormalTok{))}
  
  \ControlFlowTok{if}\NormalTok{(lambda }\SpecialCharTok{==} \DecValTok{0}\NormalTok{) }\FunctionTok{put\_log}\NormalTok{(}\StringTok{"Function \textasciigrave{}calc\_date\_general\_effect.cv\textasciigrave{}:}
\StringTok{Training completed: Date Global Effects model."}\NormalTok{)}
  \ControlFlowTok{else} \FunctionTok{put\_log1}\NormalTok{(}\StringTok{"Function \textasciigrave{}calc\_date\_general\_effect.cv\textasciigrave{}:}
\StringTok{Training completed: Date Global Effects model for lambda: \%1..."}\NormalTok{,}
\NormalTok{                lambda)}
  
\NormalTok{  date\_general\_effect}
\NormalTok{\}}
\end{Highlighting}
\end{Shaded}

\begin{noteblock}
The complete source code of the \href{https://github.com/AzKurban-edX-DS/Capstone-MovieLens/blob/main/r/src/support-functions/UMGY-effect.functions.R\#L32}{calc\_date\_general\_effect.cv} function is available in the \href{https://github.com/AzKurban-edX-DS/Capstone-MovieLens/blob/main/r/src/support-functions/UMGY-effect.functions.R\#L3}{Utility Functions} section of the \href{https://github.com/AzKurban-edX-DS/Capstone-MovieLens/blob/main/r/src/support-functions/UMGY-effect.functions.R\#L32}{UMGY-effect.functions.R} script on \emph{GitHub}.

\end{noteblock}

\newpage

\paragraph{\texorpdfstring{\href{https://github.com/AzKurban-edX-DS/Capstone-MovieLens/blob/main/r/src/support-functions/UMGY-effect.functions.R\#L71}{calc\_UMGY\_effect} Function}{calc\_UMGY\_effect Function}}\label{func.calc_UMGY_effect}

\hfill\break

Builds the \emph{User+Movie+Genre+Year Effect (UMGYE) Model} based on the \emph{Date General Effect} data structure returned by the \hyperref[func.calc_date_general_effect]{calc\_date\_general\_effect} or \hyperref[func.calc_date_general_effect.cv]{calc\_date\_general\_effect.cv} function (both described above in \emph{this Section}) and passed in the \hyperref[func.calc_UMGY_effect.args.date_general_effect]{date\_general\_effect} argument as shown in the example provided below in the \hyperref[func.calc_UMGY_effect.usage]{Usage} section of \emph{this Description}.

\phantomsection\label{func.calc_UMGY_effect.usage}
\subparagraph{Usage}\label{usage-62}

\hfill\break

\begin{Shaded}
\begin{Highlighting}[]
\NormalTok{  train\_set }\SpecialCharTok{|\textgreater{}}
    \FunctionTok{calc\_date\_general\_effect}\NormalTok{(lambda) }\SpecialCharTok{|\textgreater{}}
    \FunctionTok{calc\_UMGY\_effect}\NormalTok{()}
\end{Highlighting}
\end{Shaded}

\begin{noteblock}
In the \href{https://github.com/AzKurban-edX-DS/Capstone-MovieLens/blob/main/r/src/support-functions/UMGY-effect.functions.R\#L82}{code snippet} above, the function \emph{being described here} is called, alongside the \hyperref[func.calc_date_general_effect]{calc\_date\_general\_effect} (described above in \emph{this Section}), within the \hyperref[func.train_UMGY_effect]{train\_UMGY\_effect} (described below in \emph{this Section}).

\end{noteblock}

\phantomsection\label{func.calc_UMGY_effect.args}
\subparagraph{Arguments}\label{arguments-59}

\hfill\break

\phantomsection\label{func.calc_UMGY_effect.args.date_general_effect}
\begin{itemize}
\tightlist
\item
  \textbf{date\_general\_effect:} \emph{Data frame} object representing the \emph{Date General Effect};
\end{itemize}

\phantomsection\label{func.calc_UMGY_effect.details}
\subparagraph{Details}\label{details-60}

\hfill\break

This is an auxiliary function intended to be called internally from the following functions (both described below in \emph{this Section}):

\begin{itemize}
\tightlist
\item
  \hyperref[func.train_UMGY_effect]{train\_UMGY\_effect} (an example of calling the function \emph{being described} from \emph{this function} is shown above in the \hyperref[func.calc_UMGY_effect.usage]{Usage} section of \emph{this Description});
\item
  \hyperref[func.train_UMGY_effect.cv]{train\_UMGY\_effect.cv}.
\end{itemize}

\phantomsection\label{func.calc_UMGY_effect.value}
\subparagraph{Value}\label{value-57}

\hfill\break

Data frame object representing the \emph{UMGYE Model}.

\newpage

\phantomsection\label{func.calc_UMGY_effect.code}
\subparagraph{Source Code}\label{source-code-64}

\hfill\break

The source code of the \href{https://github.com/AzKurban-edX-DS/Capstone-MovieLens/blob/main/r/src/support-functions/UMGY-effect.functions.R\#L71}{calc\_UMGY\_effect} function is shown below:

\begin{Shaded}
\begin{Highlighting}[]
\NormalTok{calc\_UMGY\_effect }\OtherTok{\textless{}{-}} \ControlFlowTok{function}\NormalTok{(date\_general\_effect)\{}
\NormalTok{  date\_general\_effect }\SpecialCharTok{|\textgreater{}}
    \FunctionTok{group\_by}\NormalTok{(year) }\SpecialCharTok{|\textgreater{}}
    \FunctionTok{summarise}\NormalTok{(}\AttributeTok{ye =} \FunctionTok{mean}\NormalTok{(de, }\AttributeTok{na.rm =} \ConstantTok{TRUE}\NormalTok{))}
\NormalTok{\}}
\end{Highlighting}
\end{Shaded}

\begin{noteblock}
The source code of the \href{https://github.com/AzKurban-edX-DS/Capstone-MovieLens/blob/main/r/src/support-functions/UMGY-effect.functions.R\#L71}{calc\_UMGY\_effect} function is also available in the \href{https://github.com/AzKurban-edX-DS/Capstone-MovieLens/blob/main/r/src/support-functions/UMGY-effect.functions.R\#L3}{Utility Functions} section of the \href{https://github.com/AzKurban-edX-DS/Capstone-MovieLens/blob/main/r/src/support-functions/UMGY-effect.functions.R\#L71}{UMGY-effect.functions.R} script on \emph{GitHub}.

\end{noteblock}

\newpage

\paragraph{\texorpdfstring{\href{https://github.com/AzKurban-edX-DS/Capstone-MovieLens/blob/main/r/src/support-functions/UMGY-effect.functions.R\#L76}{train\_UMGY\_effect} Function}{train\_UMGY\_effect Function}}\label{func.train_UMGY_effect}

\hfill\break

Trains (and optionally \emph{regularizes} if the \hyperref[func.train_UMGY_effect.args.lambda]{lambda} argument value is non-default) the \emph{User+Movie+Genre+Year Effect} (hereafter \emph{UMGYE} for short) \emph{Model} using the training sample of the \texttt{Movielens} dataset passed in the \hyperref[func.train_UMGY_effect.args.train_set]{train\_set} argument.

\phantomsection\label{func.train_UMGY_effect.usage}
\subparagraph{\texorpdfstring{\href{https://github.com/AzKurban-edX-DS/Capstone-MovieLens/blob/main/r/src/capstone-movielens.main.R\#L1340}{Usage}}{Usage}}\label{usage-63}

\hfill\break

\begin{Shaded}
\begin{Highlighting}[]
\NormalTok{  rglr.UMGY\_effect }\OtherTok{\textless{}{-}}\NormalTok{ edx }\SpecialCharTok{|\textgreater{}} \FunctionTok{train\_UMGY\_effect}\NormalTok{(UMGYE.rglr.best\_lambda)}
\end{Highlighting}
\end{Shaded}

\begin{Shaded}
\begin{Highlighting}[]
\FunctionTok{str}\NormalTok{(rglr.UMGY\_effect)}
\end{Highlighting}
\end{Shaded}

\begin{verbatim}
## tibble [15 x 2] (S3: tbl_df/tbl/data.frame)
##  $ year: num [1:15] 1995 1996 1997 1998 1999 ...
##  $ ye  : num [1:15] 0.0018 0.06806 0.01205 0.01016 0.00634 ...
\end{verbatim}

\phantomsection\label{func.train_UMGY_effect.args}
\subparagraph{Arguments}\label{arguments-60}

\hfill\break

\phantomsection\label{func.train_UMGY_effect.args.train_set}
\begin{itemize}
\tightlist
\item
  \textbf{train\_set:} Training sample of the \texttt{Movielens} dataset;
\end{itemize}

\phantomsection\label{func.train_UMGY_effect.args.lambda}
\begin{itemize}
\tightlist
\item
  \textbf{lambda:} (Optional, \emph{Numeric}, \(0\) by default) Regularization parameter \(\lambda\).
\end{itemize}

\phantomsection\label{func.train_UMGY_effect.details}
\subparagraph{Details}\label{details-61}

\hfill\break

This function calls internally the following \emph{user-defined auxiliary functions} (both described above in \emph{this Section}):

\begin{itemize}
\tightlist
\item
  \hyperref[func.calc_date_general_effect]{calc\_date\_general\_effect};
\item
  \hyperref[func.calc_UMGY_effect]{calc\_UMGY\_effect}.
\end{itemize}

The \emph{Regularization technique} described in \href{https://rafalab.dfci.harvard.edu/dsbook-part-2/highdim/regularization.html\#penalized-least-squares}{Section 24.2 \emph{Penalized least squares}} of the \emph{Course Textbook (New Edition)}\autocite{IDS2_24-2} is applied to the model being trained only when the \emph{regularization parameter} \(\lambda\) passed in the \hyperref[func.train_UMGY_effect.args.lambda]{lambda} argument, has a value other than the default (to be more exact, greater than zero).

\phantomsection\label{func.train_UMGY_effect.value}
\subparagraph{Value}\label{value-58}

\hfill\break

Data frame object representing the trained (and optionally \emph{regularized}) \emph{UMGY Effect Model}.

\newpage

\phantomsection\label{func.train_UMGY_effect.code}
\subparagraph{Source Code}\label{source-code-65}

\hfill\break

The source code of the \href{https://github.com/AzKurban-edX-DS/Capstone-MovieLens/blob/main/r/src/support-functions/UMGY-effect.functions.R\#L76}{train\_UMGY\_effect} function is shown below:

\begin{Shaded}
\begin{Highlighting}[]
\NormalTok{train\_UMGY\_effect }\OtherTok{\textless{}{-}} \ControlFlowTok{function}\NormalTok{(train\_set, }\AttributeTok{lambda =} \DecValTok{0}\NormalTok{)\{}
  \ControlFlowTok{if}\NormalTok{ (}\FunctionTok{is.na}\NormalTok{(lambda)) \{}
    \FunctionTok{stop}\NormalTok{(}\StringTok{"Function: train\_UMGY\_effect}
\StringTok{\textasciigrave{}lambda\textasciigrave{} is \textasciigrave{}NA\textasciigrave{}"}\NormalTok{)}
\NormalTok{  \}}
  
\NormalTok{  train\_set }\SpecialCharTok{|\textgreater{}}
    \FunctionTok{calc\_date\_general\_effect}\NormalTok{(lambda) }\SpecialCharTok{|\textgreater{}}
    \FunctionTok{calc\_UMGY\_effect}\NormalTok{()}
\NormalTok{\}}
\end{Highlighting}
\end{Shaded}

\begin{noteblock}
The source code of the \href{https://github.com/AzKurban-edX-DS/Capstone-MovieLens/blob/main/r/src/support-functions/UMGY-effect.functions.R\#L76}{train\_UMGY\_effect} function is also available in the \href{https://github.com/AzKurban-edX-DS/Capstone-MovieLens/blob/main/r/src/support-functions/UMGY-effect.functions.R\#L3}{Utility Functions} section of the \href{https://github.com/AzKurban-edX-DS/Capstone-MovieLens/blob/main/r/src/support-functions/UMGY-effect.functions.R\#L76}{UMGY-effect.functions.R} script on \emph{GitHub}.

\end{noteblock}

\newpage

\paragraph{\texorpdfstring{\href{https://github.com/AzKurban-edX-DS/Capstone-MovieLens/blob/main/r/src/support-functions/UMGY-effect.functions.R\#L86}{train\_UMGY\_effect.cv} Function}{train\_UMGY\_effect.cv Function}}\label{func.train_UMGY_effect.cv}

\hfill\break

Trains (and optionally \emph{regularizes} if the \hyperref[func.train_UMGY_effect.cv.args.lambda]{lambda} argument value is non-default) the \emph{User+Movie+Genre+Year Effect (UMGYE) Model} with use of \emph{K-Fold Cross-Validation} as described in \href{https://rafalab.dfci.harvard.edu/dsbook-part-2/ml/resampling-methods.html\#k-fold-cross-validation}{Section 29.6 \emph{Cross validation / K-fold cross validation}} of the \emph{Course Textbook (New Edition)}\autocite{IDS2_29-6_KFCV}.

\begin{noteblock}
The \(K\) value for the \emph{K-Fold Cross-Validation} is determined by the length of the \hyperref[appndx_b.edx_CV]{\texttt{edx\_CV} Object} (in \emph{this Project}, we use \(K = 5\)).

\end{noteblock}

\phantomsection\label{func.train_UMGY_effect.cv.usage}
\subparagraph{Usage}\label{usage-64}

\hfill\break

\begin{Shaded}
\begin{Highlighting}[]
  \FunctionTok{train\_UMGY\_effect.cv}\NormalTok{(lambda) }\SpecialCharTok{|\textgreater{}}
    \FunctionTok{calc\_UMGY\_effect\_RMSE.cv}\NormalTok{()}
\end{Highlighting}
\end{Shaded}

\begin{noteblock}
In the \href{https://github.com/AzKurban-edX-DS/Capstone-MovieLens/blob/main/r/src/support-functions/UMGY-effect.functions.R\#L136}{code snippet} above, the function \emph{being described here} is called alongside the \hyperref[func.calc_UMGY_effect_RMSE.cv]{calc\_UMGY\_effect\_RMSE.cv} (described below in \emph{this Section}) within the custom helper function \hyperref[func.regularize.test_lambda.UMGY_effect.cv]{regularize.test\_lambda.UMGY\_effect.cv} (described below in section \hyperref[umgye-model-regularization-1]{UMGYE Model: Regularization} of \emph{this Appendix}).

\end{noteblock}

\phantomsection\label{func.train_UMGY_effect.cv.args}
\subparagraph{Arguments}\label{arguments-61}

\hfill\break

\phantomsection\label{func.train_UMGY_effect.cv.args.lambda}
\begin{itemize}
\tightlist
\item
  \textbf{lambda:} (Optional, \emph{Numeric}, \(0\) by default) Regularization parameter \(\lambda\).
\end{itemize}

\phantomsection\label{func.train_UMGY_effect.cv.details}
\subparagraph{Details}\label{details-62}

\hfill\break

This function calls internally the following \emph{user-defined auxiliary functions} (both described above in \emph{this Section}):

\begin{itemize}
\tightlist
\item
  \hyperref[func.calc_date_general_effect.cv]{calc\_date\_general\_effect.cv};
\item
  \hyperref[func.calc_UMGY_effect]{calc\_UMGY\_effect}.
\end{itemize}

The \emph{Regularization technique} described in \href{https://rafalab.dfci.harvard.edu/dsbook-part-2/highdim/regularization.html\#penalized-least-squares}{Section 24.2 \emph{Penalized least squares}} of the \emph{Course Textbook (New Edition)}\autocite{IDS2_24-2} is applied to the model being trained only when the \emph{regularization parameter} \(\lambda\) passed in the \hyperref[func.train_UMGY_effect.cv.args.lambda]{lambda} argument, has a value other than the default (to be more exact, greater than zero).

\begin{noteblock}
To use \emph{regularization}, in \emph{this Project}, this function is called within the \hyperref[func.regularize.test_lambda.UMGY_effect.cv]{regularize.test\_lambda.UMGY\_effect.cv} custom helper function (described below in section \hyperref[umgye-model-regularization-1]{UMGYE Model: Regularization} of \emph{this Appendix}) as shown above in the \hyperref[func.train_UMGY_effect.cv.usage]{Usage} section of \emph{this Description}.

\end{noteblock}

\phantomsection\label{func.train_UMGY_effect.cv.value}
\subparagraph{Value}\label{value-59}

\hfill\break

Data frame object representing the trained (and optionally \emph{regularized}) \emph{UMGY Effect Model}.

\newpage

\phantomsection\label{func.train_UMGY_effect.cv.code}
\subparagraph{Source Code}\label{source-code-66}

\hfill\break

The source code of the \href{https://github.com/AzKurban-edX-DS/Capstone-MovieLens/blob/main/r/src/support-functions/UMGY-effect.functions.R\#L86}{train\_UMGY\_effect.cv} function is shown below:

\begin{Shaded}
\begin{Highlighting}[]
\NormalTok{train\_UMGY\_effect.cv }\OtherTok{\textless{}{-}} \ControlFlowTok{function}\NormalTok{(}\AttributeTok{lambda =} \DecValTok{0}\NormalTok{)\{}
  \ControlFlowTok{if}\NormalTok{ (}\FunctionTok{is.na}\NormalTok{(lambda)) \{}
    \FunctionTok{stop}\NormalTok{(}\StringTok{"Function: train\_UMGY\_effect.cv}
\StringTok{\textasciigrave{}lambda\textasciigrave{} is \textasciigrave{}NA\textasciigrave{}"}\NormalTok{)}
\NormalTok{  \}}
  
  \FunctionTok{calc\_date\_general\_effect.cv}\NormalTok{(lambda) }\SpecialCharTok{|\textgreater{}} \FunctionTok{calc\_UMGY\_effect}\NormalTok{()  }
\NormalTok{\}}
\end{Highlighting}
\end{Shaded}

\begin{noteblock}
The source code of the \href{https://github.com/AzKurban-edX-DS/Capstone-MovieLens/blob/main/r/src/support-functions/UMGY-effect.functions.R\#L86}{train\_UMGY\_effect.cv} function is also available in the \href{https://github.com/AzKurban-edX-DS/Capstone-MovieLens/blob/main/r/src/support-functions/UMGY-effect.functions.R\#L3}{Utility Functions} section of the \href{https://github.com/AzKurban-edX-DS/Capstone-MovieLens/blob/main/r/src/support-functions/UMGY-effect.functions.R\#L86}{UMGY-effect.functions.R} script on \emph{GitHub}.

\end{noteblock}

\newpage

\paragraph{\texorpdfstring{\href{https://github.com/AzKurban-edX-DS/Capstone-MovieLens/blob/main/r/src/support-functions/UMGY-effect.functions.R\#L94}{calc\_UMGY\_effect\_MSE} Function}{calc\_UMGY\_effect\_MSE Function}}\label{func.calc_UMGY_effect_MSE}

\hfill\break

Validates the \emph{UMGY Effect (UMGYE) Model} built and trained by the \hyperref[func.train_UMGY_effect]{train\_UMGY\_effect} or \hyperref[func.train_UMGY_effect.cv]{train\_UMGY\_effect.cv} function call (both described above in \emph{this Section}).

\phantomsection\label{func.calc_UMGY_effect_MSE.usage}
\subparagraph{Usage}\label{usage-65}

\hfill\break

In the following \href{https://github.com/AzKurban-edX-DS/Capstone-MovieLens/blob/main/r/src/support-functions/UMGY-effect.functions.R\#L106}{code snippet}, the function \emph{being described here} is called internally from the \hyperref[func.calc_UMGY_effect_MSE.cv]{calc\_UMGY\_effect\_MSE.cv} function (described below in \emph{this Section}) for each iteration of the \emph{K-Fold Cross-Validation}, to calculate the \emph{MSE} for the corresponding validation dataset of the \hyperref[appndx_b.edx_CV]{\texttt{edx\_CV} Object} (described in \hyperref[appndx_b]{Appendix B: Models Training Datasets}).

\begin{Shaded}
\begin{Highlighting}[]
\NormalTok{  UMGY\_effect\_MSEs }\OtherTok{\textless{}{-}} \FunctionTok{sapply}\NormalTok{(edx\_CV, }\ControlFlowTok{function}\NormalTok{(cv\_fold\_dat)\{}
\NormalTok{    cv\_fold\_dat}\SpecialCharTok{$}\NormalTok{validation\_set }\SpecialCharTok{|\textgreater{}} \FunctionTok{calc\_UMGY\_effect\_MSE}\NormalTok{(UMGY\_effect)}
\NormalTok{  \})}
\end{Highlighting}
\end{Shaded}

\begin{noteblock}
The \(K\) value for the \emph{K-Fold Cross-Validation} is determined by the length of the \hyperref[appndx_b.edx_CV]{\texttt{edx\_CV} Object} (in \emph{this Project}, we use \(K = 5\)).

\end{noteblock}

\phantomsection\label{func.calc_UMGY_effect_MSE.args}
\subparagraph{Arguments}\label{arguments-62}

\hfill\break

\phantomsection\label{func.calc_UMGY_effect_MSE.args.test_set}
\begin{itemize}
\tightlist
\item
  \textbf{test\_set:} \emph{Validation} dataset;
\end{itemize}

\phantomsection\label{func.calc_UMGY_effect_MSE.args.UMGY_effect}
\begin{itemize}
\tightlist
\item
  \textbf{UMGY\_effect:} \emph{Data frame} object representing the \emph{UMGY Effect Model}.
\end{itemize}

\phantomsection\label{func.calc_UMGY_effect_MSE.details}
\subparagraph{Details}\label{details-63}

\hfill\break

The function calculates \emph{Mean Squared Error (MSE)} for the \emph{UMGYE Model} passed in the \hyperref[func.calc_UMGY_effect_MSE.args.UMGY_effect]{UMGY\_effect} argument using the validation dataset passed in the \hyperref[func.calc_UMGY_effect_MSE.args.test_set]{test\_set} arg.

\begin{noteblock}
This is an auxiliary function intended to be called within the \hyperref[func.calc_UMGY_effect_MSE.cv]{calc\_UMGY\_effect\_MSE.cv} (described below in \emph{this Section}) as shown in the \hyperref[func.calc_UMGY_effect_MSE.usage]{Usage} section above of \emph{this Description}.

\end{noteblock}

\phantomsection\label{func.calc_UMGY_effect_MSE.value}
\subparagraph{Value}\label{value-60}

\hfill\break

\emph{MSE} value calculated for the \emph{UMGYE Model}.

\newpage

\phantomsection\label{func.calc_UMGY_effect_MSE.code}
\subparagraph{Source Code}\label{source-code-67}

\hfill\break

The source code of the \href{https://github.com/AzKurban-edX-DS/Capstone-MovieLens/blob/main/r/src/support-functions/UMGY-effect.functions.R\#L94}{calc\_UMGY\_effect\_MSE} function is shown below:

\begin{Shaded}
\begin{Highlighting}[]
\NormalTok{calc\_UMGY\_effect\_MSE }\OtherTok{\textless{}{-}} \ControlFlowTok{function}\NormalTok{(test\_set, UMGY\_effect)\{}
\NormalTok{  test\_set }\SpecialCharTok{|\textgreater{}}
    \FunctionTok{left\_join}\NormalTok{(edx.user\_effect, }\AttributeTok{by =} \StringTok{"userId"}\NormalTok{) }\SpecialCharTok{|\textgreater{}}
    \FunctionTok{left\_join}\NormalTok{(rglr.UM\_effect, }\AttributeTok{by =} \StringTok{"movieId"}\NormalTok{) }\SpecialCharTok{|\textgreater{}}
    \FunctionTok{left\_join}\NormalTok{(rglr.UMG\_effect, }\AttributeTok{by =} \StringTok{"movieId"}\NormalTok{) }\SpecialCharTok{|\textgreater{}}
    \FunctionTok{left\_join}\NormalTok{(date\_days\_map, }\AttributeTok{by =} \StringTok{"timestamp"}\NormalTok{) }\SpecialCharTok{|\textgreater{}}
    \FunctionTok{left\_join}\NormalTok{(UMGY\_effect, }\AttributeTok{by=}\StringTok{\textquotesingle{}year\textquotesingle{}}\NormalTok{) }\SpecialCharTok{|\textgreater{}}
    \FunctionTok{mutate}\NormalTok{(}\AttributeTok{resid =}\NormalTok{ rating }\SpecialCharTok{{-}} \FunctionTok{clamp}\NormalTok{(mu }\SpecialCharTok{+}\NormalTok{ a }\SpecialCharTok{+}\NormalTok{ b }\SpecialCharTok{+}\NormalTok{ g }\SpecialCharTok{+}\NormalTok{ ye)) }\SpecialCharTok{|\textgreater{}} 
    \FunctionTok{pull}\NormalTok{(resid) }\SpecialCharTok{|\textgreater{}} \FunctionTok{mse}\NormalTok{()}
\NormalTok{\}}
\end{Highlighting}
\end{Shaded}

\begin{noteblock}
The source code of the \href{https://github.com/AzKurban-edX-DS/Capstone-MovieLens/blob/main/r/src/support-functions/UMGY-effect.functions.R\#L94}{calc\_UMGY\_effect\_MSE} function is also available in the \href{https://github.com/AzKurban-edX-DS/Capstone-MovieLens/blob/main/r/src/support-functions/UMGY-effect.functions.R\#L3}{Utility Functions} section of the \href{https://github.com/AzKurban-edX-DS/Capstone-MovieLens/blob/main/r/src/support-functions/UMGY-effect.functions.R\#L94}{UMGY-effect.functions.R} script on \emph{GitHub}.

\end{noteblock}

\newpage

\paragraph{\texorpdfstring{\href{https://github.com/AzKurban-edX-DS/Capstone-MovieLens/blob/main/r/src/support-functions/UMGY-effect.functions.R\#L104}{calc\_UMGY\_effect\_MSE.cv} Function}{calc\_UMGY\_effect\_MSE.cv Function}}\label{func.calc_UMGY_effect_MSE.cv}

\hfill\break

Validates the \emph{UMGY Effect (UMGYE) Model} built and trained by the \hyperref[func.train_UMGY_effect]{train\_UMGY\_effect} or \hyperref[func.train_UMGY_effect.cv]{train\_UMGY\_effect.cv} function call (both described above in \emph{this Section}) with use of \emph{K-Fold Cross-Validation} (as described in \href{https://rafalab.dfci.harvard.edu/dsbook-part-2/ml/resampling-methods.html\#k-fold-cross-validation}{Section 29.6 \emph{Cross validation / K-fold cross validation}} of the \emph{Course Textbook (New Edition)}\autocite{IDS2_29-6_KFCV}).

\begin{noteblock}
The \(K\) value for the \emph{K-Fold Cross-Validation} is determined by the length of the \hyperref[appndx_b.edx_CV]{\texttt{edx\_CV} Object} (in \emph{this Project}, we use \(K = 5\)).

\end{noteblock}

\phantomsection\label{func.calc_UMGY_effect_MSE.cv.usage}
\subparagraph{Usage}\label{usage-66}

\hfill\break

\begin{Shaded}
\begin{Highlighting}[]
\NormalTok{  UMGY\_effect\_RMSE }\OtherTok{\textless{}{-}} \FunctionTok{sqrt}\NormalTok{(}\FunctionTok{calc\_UMGY\_effect\_MSE.cv}\NormalTok{(UMGY\_effect))}
\end{Highlighting}
\end{Shaded}

\begin{noteblock}
In the \href{https://github.com/AzKurban-edX-DS/Capstone-MovieLens/blob/main/r/src/support-functions/UMGY-effect.functions.R\#L121}{code snippet} above, the function \emph{being described here} is called internally from the \hyperref[func.calc_UMGY_effect_RMSE.cv]{calc\_UMGY\_effect\_RMSE.cv} function described below in \emph{this Section}.

\end{noteblock}

\phantomsection\label{func.calc_UMGY_effect_MSE.cv.args}
\subparagraph{Arguments}\label{arguments-63}

\hfill\break

\phantomsection\label{func.calc_UMGY_effect_MSE.cv.args.UMGY_effect}
\begin{itemize}
\tightlist
\item
  \textbf{UMGY\_effect:} \emph{Data frame} object representing the \emph{UMGY Effect Model}.
\end{itemize}

\phantomsection\label{func.calc_UMGY_effect_MSE.cv.details}
\subparagraph{Details}\label{details-64}

\hfill\break

The function calculates the \emph{MSE} metric for the \emph{UMGYE Model} passed in the \hyperref[func.calc_UMGY_effect_MSE.cv.args.UMGY_effect]{UMGY\_effect} argument using the validation datasets contained in the \hyperref[appndx_b.edx_CV]{\texttt{edx\_CV} Object} described below in \hyperref[appndx_b]{Appendix B: Models Training Datasets}.

It internally calls the \hyperref[func.calc_UMGY_effect_MSE]{calc\_UMGY\_effect\_MSE} function for each iteration of the \emph{K-Fold Cross-Validation}, calculating the \emph{MSE} for the corresponding validation dataset of the \hyperref[appndx_b.edx_CV]{\texttt{edx\_CV} Object}, and then calculates the mean of the obtained results.

\begin{noteblock}
This is an auxiliary function used internally in the \hyperref[func.calc_UMGY_effect_RMSE.cv]{calc\_UMGY\_effect\_RMSE.cv} function (described below in \emph{this Section}) as shown in the \hyperref[func.calc_UMGY_effect_MSE.cv.usage]{Usage} section above of \emph{This Description}.

\end{noteblock}

\phantomsection\label{func.calc_UMGY_effect_MSE.cv.value}
\subparagraph{Value}\label{value-61}

\hfill\break

The \emph{MSE} value calculated for the \emph{UMGYE Model} using \emph{K-Fold Cross-Validation} (see the \hyperref[func.calc_UMGY_effect_MSE.cv.details]{Details} section of \emph{this Description} above for more details).

\newpage

\phantomsection\label{func.calc_UMGY_effect_MSE.cv.code}
\subparagraph{Source Code}\label{source-code-68}

\hfill\break

The simplified version of the source code of the \href{https://github.com/AzKurban-edX-DS/Capstone-MovieLens/blob/main/r/src/support-functions/UMGY-effect.functions.R\#L104}{calc\_UMGY\_effect\_MSE.cv} function is shown below:

\begin{Shaded}
\begin{Highlighting}[]
\NormalTok{calc\_UMGY\_effect\_MSE.cv }\OtherTok{\textless{}{-}} \ControlFlowTok{function}\NormalTok{(UMGY\_effect)\{}
\NormalTok{  start }\OtherTok{\textless{}{-}} \FunctionTok{put\_start\_date}\NormalTok{()}
\NormalTok{  UMGY\_effect\_MSEs }\OtherTok{\textless{}{-}} \FunctionTok{sapply}\NormalTok{(edx\_CV, }\ControlFlowTok{function}\NormalTok{(cv\_fold\_dat)\{}
\NormalTok{    cv\_fold\_dat}\SpecialCharTok{$}\NormalTok{validation\_set }\SpecialCharTok{|\textgreater{}} \FunctionTok{calc\_UMGY\_effect\_MSE}\NormalTok{(UMGY\_effect)}
\NormalTok{  \})}
  \FunctionTok{put\_end\_date}\NormalTok{(start)}
  \FunctionTok{put\_log1}\NormalTok{(}\StringTok{"Function: calc\_UMGY\_effect\_MSE.cv}
\StringTok{Date (Year) Effect MSE values have been computed for the \%1{-}Fold Cross Validation samples."}\NormalTok{, }
\NormalTok{           CVFolds\_N)}
  
  \FunctionTok{mean}\NormalTok{(UMGY\_effect\_MSEs)}
\NormalTok{\}}
\end{Highlighting}
\end{Shaded}

\begin{noteblock}
The completed source code of the \href{https://github.com/AzKurban-edX-DS/Capstone-MovieLens/blob/main/r/src/support-functions/UMGY-effect.functions.R\#L104}{calc\_UMGY\_effect\_MSE.cv} function is available in the \href{https://github.com/AzKurban-edX-DS/Capstone-MovieLens/blob/main/r/src/support-functions/UMGY-effect.functions.R\#L3}{Utility Functions} section of the \href{https://github.com/AzKurban-edX-DS/Capstone-MovieLens/blob/main/r/src/support-functions/UMGY-effect.functions.R\#L104}{UMGY-effect.functions.R} script on \emph{GitHub}.

\end{noteblock}

\newpage

\paragraph{\texorpdfstring{\href{https://github.com/AzKurban-edX-DS/Capstone-MovieLens/blob/main/r/src/support-functions/UMGY-effect.functions.R\#L120}{calc\_UMGY\_effect\_RMSE.cv} Function}{calc\_UMGY\_effect\_RMSE.cv Function}}\label{func.calc_UMGY_effect_RMSE.cv}

\hfill\break

Validates the \emph{UMGY Effect (UMGYE) Model} (built and trained by the \hyperref[func.train_UMGY_effect]{train\_UMGY\_effect} or \hyperref[func.train_UMGY_effect.cv]{train\_UMGY\_effect.cv} function call) with use of \emph{K-Fold Cross-Validation} as described in \href{https://rafalab.dfci.harvard.edu/dsbook-part-2/ml/resampling-methods.html\#k-fold-cross-validation}{Section 29.6 \emph{Cross validation / K-fold cross validation}} of the \emph{Course Textbook (New Edition)}\autocite{IDS2_29-6_KFCV}.

\begin{noteblock}
The \(K\) value for the \emph{K-Fold Cross-Validation} is determined by the length of the \hyperref[appndx_b.edx_CV]{\texttt{edx\_CV} Object} (in \emph{this Project}, we use \(K = 5\)).

\end{noteblock}

\phantomsection\label{func.calc_UMGY_effect_RMSE.cv.usage}
\subparagraph{\texorpdfstring{\href{https://github.com/AzKurban-edX-DS/Capstone-MovieLens/blob/main/r/src/capstone-movielens.main.R\#L1194}{Usage}}{Usage}}\label{usage-67}

\hfill\break

\begin{Shaded}
\begin{Highlighting}[]
\DocumentationTok{\#\#\#\# Compute User+Movie+Genre+Year Effect Model RMSE {-}{-}{-}{-}{-}{-}{-}{-}{-}{-}{-}{-}{-}{-}{-}{-}{-}{-}{-}{-}{-}{-}{-}{-}{-}{-}{-}}
\NormalTok{cv.UMGY\_effect.RMSE }\OtherTok{\textless{}{-}} \FunctionTok{calc\_UMGY\_effect\_RMSE.cv}\NormalTok{(cv.UMGY\_effect)}
\end{Highlighting}
\end{Shaded}

\begin{Shaded}
\begin{Highlighting}[]
\NormalTok{cv.UMGY\_effect.RMSE}
\end{Highlighting}
\end{Shaded}

\begin{verbatim}
## [1] 0.8723975
\end{verbatim}

\phantomsection\label{func.calc_UMGY_effect_RMSE.cv.args}
\subparagraph{Arguments}\label{arguments-64}

\hfill\break

\phantomsection\label{func.calc_UMGY_effect_RMSE.cv.args.UMGY_effect}
\begin{itemize}
\tightlist
\item
  \textbf{UMGY\_effect:} \emph{Data frame} object representing the \emph{UMGY Effect Model}.
\end{itemize}

\phantomsection\label{func.calc_UMGY_effect_RMSE.cv.details}
\subparagraph{Details}\label{details-65}

\hfill\break

The function calculates \emph{Root Squared Error (RMSE)} for the \emph{UMGYE Model} object passed in the \hyperref[func.calc_UMGY_effect_RMSE.cv.args.UMGY_effect]{UMGY\_effect} argument using the validation datasets contained in the \hyperref[appndx_b.edx_CV]{\texttt{edx\_CV} Object} described below in \hyperref[appndx_b]{Appendix B: Models Training Datasets}.\newline
During its execution, it internally calls the \hyperref[func.calc_UMGY_effect_MSE.cv]{calc\_UMGY\_effect\_MSE.cv} function (described above in \emph{this Section}) to calculate the \emph{MSE} for the \emph{UMGE Model} (with use of \emph{K-Fold Cross-Validation}), and then calculates the square root of the result.

\phantomsection\label{func.calc_UMGY_effect_RMSE.cv.value}
\subparagraph{Value}\label{value-62}

\hfill\break

The \emph{RMSE} value calculated for the \emph{UMGYE Model} using \emph{K-Fold Cross-Validation}.

\newpage

\phantomsection\label{func.calc_UMGY_effect_RMSE.cv.code}
\subparagraph{Source Code}\label{source-code-69}

\hfill\break

The source code of the \href{https://github.com/AzKurban-edX-DS/Capstone-MovieLens/blob/main/r/src/support-functions/UMGY-effect.functions.R\#L120}{calc\_UMGY\_effect\_RMSE.cv} function is shown below:

\begin{Shaded}
\begin{Highlighting}[]
\NormalTok{calc\_UMGY\_effect\_RMSE.cv }\OtherTok{\textless{}{-}} \ControlFlowTok{function}\NormalTok{(UMGY\_effect)\{}
\NormalTok{  UMGY\_effect\_RMSE }\OtherTok{\textless{}{-}} \FunctionTok{sqrt}\NormalTok{(}\FunctionTok{calc\_UMGY\_effect\_MSE.cv}\NormalTok{(UMGY\_effect))}
  \FunctionTok{put\_log2}\NormalTok{(}\StringTok{"\%1{-}Fold Cross Validation ultimate RMSE: \%2"}\NormalTok{, }
\NormalTok{           CVFolds\_N, }
\NormalTok{           UMGY\_effect\_RMSE)}
  
\NormalTok{  UMGY\_effect\_RMSE}
\NormalTok{\}}
\end{Highlighting}
\end{Shaded}

\begin{noteblock}
The source code of the \href{https://github.com/AzKurban-edX-DS/Capstone-MovieLens/blob/main/r/src/support-functions/UMGY-effect.functions.R\#L120}{calc\_UMGY\_effect\_RMSE.cv} function is also available in the \href{https://github.com/AzKurban-edX-DS/Capstone-MovieLens/blob/main/r/src/support-functions/UMGY-effect.functions.R\#L3}{Utility Functions} section of the \href{https://github.com/AzKurban-edX-DS/Capstone-MovieLens/blob/main/r/src/support-functions/UMGY-effect.functions.R\#L120}{UMGY-effect.functions.R} script on \emph{GitHub}.

\end{noteblock}

\newpage

\subsubsection{UMGYE Model: Regularization}\label{umgye-model-regularization-1}

\hfill\break

\paragraph{\texorpdfstring{\href{https://github.com/AzKurban-edX-DS/Capstone-MovieLens/blob/main/r/src/support-functions/UMGY-effect.functions.R\#L130}{regularize.test\_lambda.UMGY\_effect.cv} Function}{regularize.test\_lambda.UMGY\_effect.cv Function}}\label{func.regularize.test_lambda.UMGY_effect.cv}

\hfill\break

Validates the \emph{regularization parameter} \(\lambda\) passed in the \hyperref[func.regularize.test_lambda.UMGY_effect.cv.args.lambda]{lambda} argument for the \emph{User+Movie+Genre+Year Effect (UMGYE) Model} in the \emph{regularization process} using \emph{penalized approach} with use of \emph{K-Fold Cross-Validation}.

\phantomsection\label{func.regularize.test_lambda.UMGY_effect.cv.usage}
\subparagraph{Usage}\label{usage-68}

\hfill\break

\begin{Shaded}
\begin{Highlighting}[]
\NormalTok{  lambdas }\OtherTok{\textless{}{-}} \FunctionTok{seq}\NormalTok{(}\DecValTok{0}\NormalTok{, }\DecValTok{512}\NormalTok{, }\DecValTok{32}\NormalTok{)}
\NormalTok{  cv.UMGYE.preset.result }\OtherTok{\textless{}{-}} 
    \FunctionTok{tune.model\_param}\NormalTok{(lambdas, }
\NormalTok{                     regularize.test\_lambda.UMGY\_effect.cv,}
                     \AttributeTok{steps.beyond\_min =} \DecValTok{16}\NormalTok{)}
\end{Highlighting}
\end{Shaded}

\begin{Shaded}
\begin{Highlighting}[]
\FunctionTok{str}\NormalTok{(cv.UMGYE.preset.result)}
\end{Highlighting}
\end{Shaded}

\begin{verbatim}
## List of 2
##  $ tuned.result:'data.frame':    17 obs. of  2 variables:
##   ..$ RMSE           : num [1:17] 0.872 0.872 0.872 0.872 0.872 ...
##   ..$ parameter.value: num [1:17] 0 32 64 96 128 160 192 224 256 288 ...
##  $ best_result : Named num [1:2] 224 0.872
##   ..- attr(*, "names")= chr [1:2] "param.best_value" "best_RMSE"
\end{verbatim}

\begin{noteblock}
In the \href{https://github.com/AzKurban-edX-DS/Capstone-MovieLens/blob/main/r/src/capstone-movielens.main.R\#L1241}{code snippet} above the function \emph{being described here} is passed in the \hyperref[func.tune.model_param.args.fn_tune.test.param_value]{fn\_tune.test.param\_value} argument to the \hyperref[func.tune.model_param]{tune.model\_param} function described above in section \hyperref[appndx_a.CHF.model_tuning_utils]{Model Tuning Utils} of \emph{this Appendix}.

Such a use of this function is also noted in the \hyperref[func.regularize.test_lambda.UMGY_effect.cv.details]{Details} section of \emph{this Description} below.

\end{noteblock}

\phantomsection\label{func.regularize.test_lambda.UMGY_effect.cv.args}
\subparagraph{Arguments}\label{arguments-65}

\hfill\break

\phantomsection\label{func.regularize.test_lambda.UMGY_effect.cv.args.lambda}
\begin{itemize}
\tightlist
\item
  \textbf{lambda:} \emph{Regularization parameter} \(\lambda\) used to validate the model in the \emph{regularization process}.
\end{itemize}

\newpage

\phantomsection\label{func.regularize.test_lambda.UMGY_effect.cv.details}
\subparagraph{Details}\label{details-66}

\hfill\break

The function is used in the \emph{regularization process} of the \emph{UMGYE Model} using the \emph{penalized approach} to regularize predictions for the given \emph{regularization parameter} \(\lambda\) (passed in the \hyperref[func.regularize.test_lambda.UMGY_effect.cv.args.lambda]{lambda} argument) as explained in \href{https://rafalab.dfci.harvard.edu/dsbook-part-2/highdim/regularization.html\#penalized-least-squares}{Section 24.2 \emph{Penalized least squares}} of the \emph{Course Textbook (New Edition)}\autocite{IDS2_24-2}.

Internally it uses the \hyperref[func.train_UMGY_effect.cv]{train\_UMGY\_effect.cv} and \hyperref[func.calc_UMGY_effect_RMSE.cv]{calc\_UMGY\_effect\_RMSE.cv} functions (described above in \hyperref[umgye-model-utility-functions]{UMGYE Model: Utility Functions} section) to validate the \emph{UMGYE Model} for the given \(\lambda\) with use of the \emph{K-Fold Cross-Validation} method, as described in \href{https://rafalab.dfci.harvard.edu/dsbook-part-2/ml/resampling-methods.html\#k-fold-cross-validation}{Section 29.6 \emph{Cross validation / K-fold cross validation}} of the \emph{Course Textbook (New Edition)}\autocite{IDS2_29-6_KFCV}, where the value of \(K\) is determined by the length of the \hyperref[appndx_b.edx_CV]{\texttt{edx\_CV} Object} described below in \hyperref[appndx_b]{Appendix B: Models Training Datasets} (in \emph{this Project}, we use \(K = 5\)).

\begin{noteblock}
This function is an auxiliary function specifically designed to be passed in the \texttt{fn\_tune.test.param\_value} argument to the \hyperref[func.tune.model_param]{tune.model\_param} (as shown above in the \hyperref[func.regularize.test_lambda.UMGY_effect.cv.usage]{Usage} section of \emph{this Description}) and the \hyperref[func.model.tune.param_range]{model.tune.param\_range} functions described above in section \hyperref[appndx_a.CHF.model_tuning_utils]{Model Tuning Utils} of \emph{this Appendix}.

\end{noteblock}

\phantomsection\label{func.regularize.test_lambda.UMGY_effect.cv.value}
\subparagraph{Value}\label{value-63}

\hfill\break

The \emph{Root Mean Squared Errors (RMSE)} value computed for the UMGYE Model applying the \emph{regularization parameter} \(\lambda\) passed in the \hyperref[func.regularize.test_lambda.UMGY_effect.cv.args.lambda]{lambda} argument.

\phantomsection\label{func.regularize.test_lambda.UMGY_effect.cv.code}
\subparagraph{Source Code}\label{source-code-70}

\hfill\break

The source code of the \href{https://github.com/AzKurban-edX-DS/Capstone-MovieLens/blob/main/r/src/support-functions/UMGY-effect.functions.R\#L130}{regularize.test\_lambda.UMGY\_effect.cv} function is shown below:

\begin{Shaded}
\begin{Highlighting}[]
\NormalTok{regularize.test\_lambda.UMGY\_effect.cv }\OtherTok{\textless{}{-}} \ControlFlowTok{function}\NormalTok{(lambda)\{}
  \ControlFlowTok{if}\NormalTok{ (}\FunctionTok{is.na}\NormalTok{(lambda)) \{}
    \FunctionTok{stop}\NormalTok{(}\StringTok{"Function: regularize.test\_lambda.UMGY\_effect.cv}
\StringTok{\textasciigrave{}lambda\textasciigrave{} is \textasciigrave{}NA\textasciigrave{}"}\NormalTok{)}
\NormalTok{  \}}
  
  \FunctionTok{train\_UMGY\_effect.cv}\NormalTok{(lambda) }\SpecialCharTok{|\textgreater{}}
    \FunctionTok{calc\_UMGY\_effect\_RMSE.cv}\NormalTok{()}
\NormalTok{\}}
\end{Highlighting}
\end{Shaded}

\begin{noteblock}
The source code of the \href{https://github.com/AzKurban-edX-DS/Capstone-MovieLens/blob/main/r/src/support-functions/UMGY-effect.functions.R\#L130}{regularize.test\_lambda.UMGY\_effect.cv} function is also available in the \href{https://github.com/AzKurban-edX-DS/Capstone-MovieLens/blob/main/r/src/support-functions/UMGY-effect.functions.R\#L129}{Regularization} section of the \href{https://github.com/AzKurban-edX-DS/Capstone-MovieLens/blob/main/r/src/support-functions/UMGY-effect.functions.R\#L130}{UMGY-effect.functions.R} script on \emph{GitHub}.

\end{noteblock}

\newpage

\subsubsection{UMGYDE Model: Utility Functions}\label{umgyde-model-utility-functions}

\hfill\break

\begin{noteblock}
The complete source code of the functions described in this section is available in the \href{https://github.com/AzKurban-edX-DS/Capstone-MovieLens/blob/main/r/src/support-functions/UMGYD-effect.functions.R\#L3}{Utility Functions} section of the \href{https://github.com/AzKurban-edX-DS/Capstone-MovieLens/blob/main/r/src/support-functions/UMGYD-effect.functions.R\#L3}{UMGYD-effect.functions.R} script on \emph{GitHub}.

\end{noteblock}

\paragraph{\texorpdfstring{\href{https://github.com/AzKurban-edX-DS/Capstone-MovieLens/blob/main/r/src/support-functions/UMGYD-effect.functions.R\#L4}{calc\_day\_general\_effect} Function}{calc\_day\_general\_effect Function}}\label{func.calc_day_general_effect}

\hfill\break

Computes the \emph{Day General Effect} (hereafter \emph{DG Effect} or \emph{DGE} for short), optionally \emph{regularized} if the \hyperref[func.calc_day_general_effect.args.lambda]{lambda} argument value is non-default, of user ratings on a \emph{given day since the earliest record} in the \texttt{edx} dataset, using a \emph{training dataset} passed in the \hyperref[func.calc_day_general_effect.args.train_set]{train\_set} argument.

\phantomsection\label{func.calc_day_general_effect.usage}
\subparagraph{Usage}\label{usage-69}

\hfill\break

\begin{Shaded}
\begin{Highlighting}[]
\NormalTok{  train\_set }\SpecialCharTok{|\textgreater{}} 
    \FunctionTok{calc\_day\_general\_effect}\NormalTok{(lambda) }\SpecialCharTok{|\textgreater{}}
    \FunctionTok{calc\_UMGY\_SmoothedDay\_effect}\NormalTok{(degree, span)}
\end{Highlighting}
\end{Shaded}

\begin{noteblock}
In the \href{https://github.com/AzKurban-edX-DS/Capstone-MovieLens/blob/main/r/src/support-functions/UMGYD-effect.functions.R\#L108}{code snippet} above, the function \emph{being described here} is called, alongside the \hyperref[func.calc_UMGY_SmoothedDay_effect]{calc\_UMGY\_SmoothedDay\_effect}, within the \hyperref[func.train_UMGY_SmoothedDay_effect]{train\_UMGY\_SmoothedDay\_effect} (both described below in this section).

\end{noteblock}

\phantomsection\label{func.calc_day_general_effect.args}
\subparagraph{Arguments}\label{arguments-66}

\hfill\break

\phantomsection\label{func.calc_day_general_effect.args.train_set}
\begin{itemize}
\tightlist
\item
  \textbf{train\_set:} Training dataset containing data to compute the \emph{DG Effect};
\end{itemize}

\phantomsection\label{func.calc_day_general_effect.args.lambda}
\begin{itemize}
\tightlist
\item
  \textbf{lambda:} (Optional, \emph{Numeric}, \(0\) by default) Regularization parameter \(\lambda\).
\end{itemize}

\phantomsection\label{func.calc_day_general_effect.details}
\subparagraph{Details}\label{details-67}

\hfill\break

This is an auxiliary function intended to be called within the following custom helper functions: the \hyperref[func.train_UMGY_SmoothedDay_effect]{train\_UMGY\_SmoothedDay\_effect} (as shown above in the \hyperref[func.calc_day_general_effect.usage]{Usage} section of \emph{this Description}) and the \hyperref[func.calc_day_general_effect.cv]{calc\_day\_general\_effect.cv} (both described below in \emph{this Section}).

The \emph{Regularization technique} described in \href{https://rafalab.dfci.harvard.edu/dsbook-part-2/highdim/regularization.html\#penalized-least-squares}{Section 24.2 \emph{Penalized least squares}} of the \emph{Course Textbook (New Edition)}\autocite{IDS2_24-2} is applied to the model being trained only when the \emph{regularization parameter} \(\lambda\) passed in the \hyperref[func.calc_day_general_effect.args.lambda]{lambda} argument, has a value other than the default (to be more exact, greater than zero).

\begin{noteblock}
\emph{Regularization} is performed through \emph{penalized regression} by computing the \emph{penalized mean} using the internally called \hyperref[func.mean_reg]{mean\_reg} function (described above in section \hyperref[appndx_a.CHF.model_tuning_utils]{Model Tuning Utils} of \emph{this Appendix}).

\end{noteblock}

\phantomsection\label{func.calc_day_general_effect.value}
\subparagraph{Value}\label{value-64}

\hfill\break

\emph{Data frame} object, representing (optionally, \emph{Regularized}) \emph{Day General Effect}.

\phantomsection\label{func.calc_day_general_effect.code}
\subparagraph{Source Code}\label{source-code-71}

\hfill\break

Below is a slightly simplified version of the source code of the \href{https://github.com/AzKurban-edX-DS/Capstone-MovieLens/blob/main/r/src/support-functions/UMGYD-effect.functions.R\#L4}{calc\_day\_general\_effect} function:

\begin{Shaded}
\begin{Highlighting}[]
\NormalTok{calc\_day\_general\_effect }\OtherTok{\textless{}{-}} \ControlFlowTok{function}\NormalTok{(train\_set, }\AttributeTok{lambda =} \DecValTok{0}\NormalTok{)\{}
\NormalTok{  gday\_effect }\OtherTok{\textless{}{-}}\NormalTok{ train\_set }\SpecialCharTok{|\textgreater{}} 
    \FunctionTok{left\_join}\NormalTok{(edx.user\_effect, }\AttributeTok{by =} \StringTok{"userId"}\NormalTok{) }\SpecialCharTok{|\textgreater{}}
    \FunctionTok{left\_join}\NormalTok{(rglr.UM\_effect, }\AttributeTok{by =} \StringTok{"movieId"}\NormalTok{) }\SpecialCharTok{|\textgreater{}}
    \FunctionTok{left\_join}\NormalTok{(rglr.UMG\_effect, }\AttributeTok{by =} \StringTok{"movieId"}\NormalTok{) }\SpecialCharTok{|\textgreater{}}
    \FunctionTok{left\_join}\NormalTok{(date\_days\_map, }\AttributeTok{by =} \StringTok{"timestamp"}\NormalTok{) }\SpecialCharTok{|\textgreater{}}
    \FunctionTok{left\_join}\NormalTok{(rglr.UMGY\_effect, }\AttributeTok{by=}\StringTok{\textquotesingle{}year\textquotesingle{}}\NormalTok{) }\SpecialCharTok{|\textgreater{}}
    \FunctionTok{mutate}\NormalTok{(}\AttributeTok{resid =}\NormalTok{ rating }\SpecialCharTok{{-}}\NormalTok{ (mu }\SpecialCharTok{+}\NormalTok{ a }\SpecialCharTok{+}\NormalTok{ b }\SpecialCharTok{+}\NormalTok{ g }\SpecialCharTok{+}\NormalTok{ ye)) }\SpecialCharTok{|\textgreater{}}
    \FunctionTok{group\_by}\NormalTok{(days) }\SpecialCharTok{|\textgreater{}}
    \FunctionTok{summarise}\NormalTok{(}\AttributeTok{de =} \FunctionTok{mean\_reg}\NormalTok{(resid, lambda), }
              \AttributeTok{year =} \FunctionTok{mean}\NormalTok{(year))}
  
  \ControlFlowTok{if}\NormalTok{(lambda }\SpecialCharTok{==} \DecValTok{0}\NormalTok{) }\FunctionTok{put\_log}\NormalTok{(}\StringTok{"Function \textasciigrave{}calc\_day\_general\_effect\textasciigrave{}:}
\StringTok{Day General Effect has been computed."}\NormalTok{)}
  \ControlFlowTok{else} \FunctionTok{put\_log1}\NormalTok{(}\StringTok{"Function \textasciigrave{}calc\_day\_general\_effect\textasciigrave{}:}
\StringTok{Day General Effect has been computed for lambda: \%1..."}\NormalTok{,}
\NormalTok{                lambda)}
\NormalTok{  gday\_effect}
\NormalTok{\}}
\end{Highlighting}
\end{Shaded}

\begin{noteblock}
The complete source code of the \href{https://github.com/AzKurban-edX-DS/Capstone-MovieLens/blob/main/r/src/support-functions/UMGYD-effect.functions.R\#L4}{calc\_day\_general\_effect} function is available in the \href{https://github.com/AzKurban-edX-DS/Capstone-MovieLens/blob/main/r/src/support-functions/UMGYD-effect.functions.R\#L3}{Utility Functions} section of the \href{https://github.com/AzKurban-edX-DS/Capstone-MovieLens/blob/main/r/src/support-functions/UMGYD-effect.functions.R\#L4}{UMGYD-effect.functions.R} script on \emph{GitHub}.

\end{noteblock}

\newpage

\paragraph{\texorpdfstring{\href{https://github.com/AzKurban-edX-DS/Capstone-MovieLens/blob/main/r/src/support-functions/UMGYD-effect.functions.R\#L33}{calc\_day\_general\_effect.cv} Function}{calc\_day\_general\_effect.cv Function}}\label{func.calc_day_general_effect.cv}

\hfill\break

This is an auxiliary function that calculates the \emph{Day General Effect} (optionally \emph{regularizes} if the \hyperref[func.calc_day_general_effect.cv.args.lambda]{lambda} argument value is non-default) of the user ratings, using the \emph{K-Fold Cross Validation} as described in \href{https://rafalab.dfci.harvard.edu/dsbook-part-2/ml/resampling-methods.html\#k-fold-cross-validation}{Section 29.6 \emph{Cross validation / K-fold cross validation}} of the \emph{Course Textbook (New Edition)}\autocite{IDS2_29-6_KFCV}, and is intended to be called within another custom helper function, the \hyperref[func.train_UMGY_SmoothedDay_effect.cv]{train\_UMGY\_SmoothedDay\_effect.cv}) (described below in \emph{this Section}) as shown below in the \hyperref[func.calc_day_general_effect.cv.usage]{Usage} section of \emph{this Description}.

\phantomsection\label{func.calc_day_general_effect.cv.usage}
\subparagraph{Usage}\label{usage-70}

\hfill\break

\begin{Shaded}
\begin{Highlighting}[]
  \FunctionTok{calc\_day\_general\_effect.cv}\NormalTok{(lambda) }\SpecialCharTok{|\textgreater{}}
    \FunctionTok{calc\_UMGY\_SmoothedDay\_effect}\NormalTok{(degree, span)}
\end{Highlighting}
\end{Shaded}

\begin{noteblock}
In the above \href{https://github.com/AzKurban-edX-DS/Capstone-MovieLens/blob/main/r/src/support-functions/UMGYD-effect.functions.R\#L121}{code snippet}, the function \emph{being described here} is called, alongside the \hyperref[func.calc_UMGY_SmoothedDay_effect]{calc\_UMGY\_SmoothedDay\_effect}, within the \hyperref[func.train_UMGY_SmoothedDay_effect.cv]{train\_UMGY\_SmoothedDay\_effect.cv} (both described below in \emph{this Section}).

\end{noteblock}

\phantomsection\label{func.calc_day_general_effect.cv.args}
\subparagraph{Arguments}\label{arguments-67}

\hfill\break

\phantomsection\label{func.calc_day_general_effect.cv.args.lambda}
\begin{itemize}
\tightlist
\item
  \textbf{lambda:} (Optional, \emph{Numeric}, \(0\) by default) Regularization parameter \(\lambda\).
\end{itemize}

\phantomsection\label{func.calc_day_general_effect.cv.details}
\subparagraph{Details}\label{details-68}

\hfill\break

The \emph{Regularization technique}, described in \href{https://rafalab.dfci.harvard.edu/dsbook-part-2/highdim/regularization.html\#penalized-least-squares}{Section 24.2 \emph{Penalized least squares}} of the \emph{Course Textbook (New Edition)}\autocite{IDS2_24-2}, is applied to the model being trained only when the \emph{regularization parameter} \(\lambda\), passed in the \hyperref[func.calc_day_general_effect.cv.args.lambda]{lambda} argument, has a non-default value (to be more exact, is a positive number).

Internally, this function calls the custom helper functions \hyperref[func.calc_day_general_effect]{calc\_day\_general\_effect} function (described above in \emph{this Section}) for each iteration of the \emph{K-Fold Cross-Validation} applied to the \hyperref[appndx_b.edx_CV]{\texttt{edx\_CV} Object} described below in \hyperref[appndx_b]{Appendix B: Models Training Datasets}.
It also internally calls the user-defined auxiliary function \hyperref[func.union_cv_results]{union\_cv\_results}, described above in section \hyperref[models-training-functions]{Models Training Functions} of \emph{this Appendix}, to aggregate the cross-validation results.

\begin{noteblock}
The \(K\) value for the \emph{K-Fold Cross-Validation} is determined by the length of the \hyperref[appndx_b.edx_CV]{\texttt{edx\_CV} Object} (in \emph{this Project}, we use \(K = 5\)).

\end{noteblock}

\phantomsection\label{func.calc_day_general_effect.cv.value}
\subparagraph{Value}\label{value-65}

\hfill\break

\emph{Data frame} object, representing (optionally, \emph{Regularized}) \emph{Day General Effect}.

\newpage

\phantomsection\label{func.calc_day_general_effect.cv.code}
\subparagraph{Source Code}\label{source-code-72}

\hfill\break

Below is a simplified version of the source code of the \href{https://github.com/AzKurban-edX-DS/Capstone-MovieLens/blob/main/r/src/support-functions/UMGYD-effect.functions.R\#L33}{calc\_day\_general\_effect.cv} function:

\begin{Shaded}
\begin{Highlighting}[]
\NormalTok{calc\_day\_general\_effect.cv }\OtherTok{\textless{}{-}} \ControlFlowTok{function}\NormalTok{(}\AttributeTok{lambda =} \DecValTok{0}\NormalTok{)\{}
  
  \FunctionTok{put\_log1}\NormalTok{(}\StringTok{"Function \textasciigrave{}calc\_day\_general\_effect.cv\textasciigrave{}:}
\StringTok{Computing Day General Effect list for \%1{-}Fold Cross Validation samples..."}\NormalTok{, }
\NormalTok{           CVFolds\_N)}

\NormalTok{  gday\_effect\_ls }\OtherTok{\textless{}{-}} \FunctionTok{lapply}\NormalTok{(edx\_CV,  }\ControlFlowTok{function}\NormalTok{(cv\_fold\_dat)\{}
\NormalTok{    cv\_fold\_dat}\SpecialCharTok{$}\NormalTok{train\_set }\SpecialCharTok{|\textgreater{}} \FunctionTok{calc\_day\_general\_effect}\NormalTok{(lambda)}
\NormalTok{  \})}

\NormalTok{  gday\_effect\_united }\OtherTok{\textless{}{-}} \FunctionTok{union\_cv\_results}\NormalTok{(gday\_effect\_ls)}

\NormalTok{  gday\_effect\_united }\SpecialCharTok{|\textgreater{}}
    \FunctionTok{group\_by}\NormalTok{(days) }\SpecialCharTok{|\textgreater{}}
    \FunctionTok{summarise}\NormalTok{(}\AttributeTok{de =} \FunctionTok{mean}\NormalTok{(de), }\AttributeTok{year =} \FunctionTok{mean}\NormalTok{(year))}
  
\NormalTok{\}}
\end{Highlighting}
\end{Shaded}

\begin{noteblock}
The complete source code of the \href{https://github.com/AzKurban-edX-DS/Capstone-MovieLens/blob/main/r/src/support-functions/UMGYD-effect.functions.R\#L33}{calc\_day\_general\_effect.cv} function is available in the \href{https://github.com/AzKurban-edX-DS/Capstone-MovieLens/blob/main/r/src/support-functions/UMGYD-effect.functions.R\#L3}{Utility Functions} section of the \href{https://github.com/AzKurban-edX-DS/Capstone-MovieLens/blob/main/r/src/support-functions/UMGYD-effect.functions.R\#L33}{UMGYD-effect.functions.R} script on \emph{GitHub}.

\end{noteblock}

\newpage

\paragraph{\texorpdfstring{\href{https://github.com/AzKurban-edX-DS/Capstone-MovieLens/blob/main/r/src/support-functions/UMGYD-effect.functions.R\#L71}{calc\_UMGY\_SmoothedDay\_effect} Function}{calc\_UMGY\_SmoothedDay\_effect Function}}\label{func.calc_UMGY_SmoothedDay_effect}

\hfill\break

Builds the \emph{User+Movie+Genre+Year+SmoothedDay Effect} (hereafter \emph{UMGYDE} for short) \emph{Model} based on the \emph{Day General Effect} (\emph{DG Effect} or \emph{DGE} for short) data structure returned by the \hyperref[func.calc_day_general_effect]{calc\_day\_general\_effect} or \hyperref[func.calc_day_general_effect.cv]{calc\_day\_general\_effect.cv} function (both described above) and passed in the \hyperref[func.calc_UMGY_SmoothedDay_effect.args.date_general_effect]{date\_general\_effect} argument.

\phantomsection\label{func.calc_UMGY_SmoothedDay_effect.usage}
\subparagraph{Usage}\label{usage-71}

\hfill\break

\begin{Shaded}
\begin{Highlighting}[]
\NormalTok{  train\_set }\SpecialCharTok{|\textgreater{}} 
    \FunctionTok{calc\_day\_general\_effect}\NormalTok{(lambda) }\SpecialCharTok{|\textgreater{}}
    \FunctionTok{calc\_UMGY\_SmoothedDay\_effect}\NormalTok{(degree, span)}
\end{Highlighting}
\end{Shaded}

\begin{noteblock}
In the \href{https://github.com/AzKurban-edX-DS/Capstone-MovieLens/blob/main/r/src/support-functions/UMGYD-effect.functions.R\#L108}{code snippet} above, the function \emph{being described here} is called, alongside the \hyperref[func.calc_day_general_effect]{calc\_day\_general\_effect} (described above in \emph{this Section}), within the \hyperref[func.train_UMGY_SmoothedDay_effect]{train\_UMGY\_SmoothedDay\_effect} (described below in \emph{this Section}).

\end{noteblock}

\phantomsection\label{func.calc_UMGY_SmoothedDay_effect.args}
\subparagraph{Arguments}\label{arguments-68}

\hfill\break

\phantomsection\label{func.calc_UMGY_SmoothedDay_effect.args.date_general_effect}
\begin{itemize}
\tightlist
\item
  \textbf{day\_general\_effect:} Data frame object representing the \emph{DG Effect} (to pass in the \texttt{data} argument to the internally called \href{https://www.rdocumentation.org/packages/stats/versions/3.6.2/topics/loess}{stats::loess} function);
\end{itemize}

\phantomsection\label{func.calc_UMGY_SmoothedDay_effect.args.degree}
\begin{itemize}
\tightlist
\item
  \textbf{degree:} (Optional, \emph{Integer}, \texttt{NA} by default) The degree of the polynomials to pass in the same-named argument to the internally called \href{https://www.rdocumentation.org/packages/stats/versions/3.6.2/topics/loess}{stats::loess} (with the default value replaced by \texttt{degree\ =\ 1});
\end{itemize}

\phantomsection\label{func.calc_UMGY_SmoothedDay_effect.args.span}
\begin{itemize}
\tightlist
\item
  \textbf{span:} (Optional, \emph{Numeric}, \texttt{NA} by default) The parameter \(\alpha\) (which controls the degree of smoothing) to pass in the same-named argument to the internally called \href{https://www.rdocumentation.org/packages/stats/versions/3.6.2/topics/loess}{stats::loess} (with the default value replaced by \texttt{span\ =\ 0.75}).
\end{itemize}

\phantomsection\label{func.calc_UMGY_SmoothedDay_effect.details}
\subparagraph{Details}\label{details-69}

\hfill\break

This auxiliary function is a wrapper for the \href{https://www.rdocumentation.org/packages/stats/versions/3.6.2/topics/loess}{stats::loess} \texttt{R} function, intended to be used internally in the following custom helper functions (both described below in \emph{this Section}):

\begin{itemize}
\item
  \hyperref[func.train_UMGY_SmoothedDay_effect]{train\_UMGY\_SmoothedDay\_effect};
\item
  \hyperref[func.train_UMGY_SmoothedDay_effect.cv]{train\_UMGY\_SmoothedDay\_effect.cv}.
\end{itemize}

An example of using it in the \hyperref[func.train_UMGY_SmoothedDay_effect]{train\_UMGY\_SmoothedDay\_effect} function is shown above in the \hyperref[func.calc_UMGY_SmoothedDay_effect.usage]{Usage} section of \emph{this Description}:

\phantomsection\label{func.calc_UMGY_SmoothedDay_effect.value}
\subparagraph{Value}\label{value-66}

\hfill\break

Data frame object representing the \emph{UMGYDE Model}.

\phantomsection\label{func.calc_UMGY_SmoothedDay_effect.code}
\subparagraph{Source Code}\label{source-code-73}

\hfill\break

Below is a simplified version of the source code of the \href{https://github.com/AzKurban-edX-DS/Capstone-MovieLens/blob/main/r/src/support-functions/UMGYD-effect.functions.R\#L71}{calc\_UMGY\_SmoothedDay\_effect} function:

\begin{Shaded}
\begin{Highlighting}[]
\NormalTok{calc\_UMGY\_SmoothedDay\_effect }\OtherTok{\textless{}{-}} \ControlFlowTok{function}\NormalTok{(day\_gen\_effect, }
                                         \AttributeTok{degree =} \ConstantTok{NA}\NormalTok{, }
                                         \AttributeTok{span =} \ConstantTok{NA}\NormalTok{)\{}
  
  \ControlFlowTok{if}\NormalTok{(}\FunctionTok{is.na}\NormalTok{(degree)) degree }\OtherTok{=} \DecValTok{1}
  \ControlFlowTok{if}\NormalTok{(}\FunctionTok{is.na}\NormalTok{(span)) span }\OtherTok{=} \FloatTok{0.75}

\NormalTok{  fit }\OtherTok{\textless{}{-}} \FunctionTok{loess}\NormalTok{(de }\SpecialCharTok{\textasciitilde{}}\NormalTok{ days, }\AttributeTok{span =}\NormalTok{ span, }\AttributeTok{degree =}\NormalTok{ degree, }\AttributeTok{data =}\NormalTok{ day\_gen\_effect)}
\NormalTok{  smth\_day\_effect }\OtherTok{\textless{}{-}}\NormalTok{ day\_gen\_effect }\SpecialCharTok{|\textgreater{}} \FunctionTok{mutate}\NormalTok{(}\AttributeTok{de\_smoothed =}\NormalTok{ fit}\SpecialCharTok{$}\NormalTok{fitted)}
  
  \FunctionTok{put\_log2}\NormalTok{(}\StringTok{"Function \textasciigrave{}calc\_UMGY\_SmoothedDay\_effect\textasciigrave{}: }
\StringTok{Model has been trained using \textasciigrave{}loess\textasciigrave{} function with the following parameters:}
\StringTok{degree: \%1;}
\StringTok{span: \%2."}\NormalTok{,}
\NormalTok{           degree,}
\NormalTok{           span)}
  
\NormalTok{  smth\_day\_effect}
\NormalTok{\}}
\end{Highlighting}
\end{Shaded}

\begin{noteblock}
The complete source code of the \href{https://github.com/AzKurban-edX-DS/Capstone-MovieLens/blob/main/r/src/support-functions/UMGYD-effect.functions.R\#L71}{calc\_UMGY\_SmoothedDay\_effect} function is available in the \href{https://github.com/AzKurban-edX-DS/Capstone-MovieLens/blob/main/r/src/support-functions/UMGYD-effect.functions.R\#L3}{Utility Functions} section of the \href{https://github.com/AzKurban-edX-DS/Capstone-MovieLens/blob/main/r/src/support-functions/UMGYD-effect.functions.R\#L71}{UMGYD-effect.functions.R} script on \emph{GitHub}.

\end{noteblock}

\newpage

\paragraph{\texorpdfstring{\href{https://github.com/AzKurban-edX-DS/Capstone-MovieLens/blob/main/r/src/support-functions/UMGYD-effect.functions.R\#L99}{train\_UMGY\_SmoothedDay\_effect} Function}{train\_UMGY\_SmoothedDay\_effect Function}}\label{func.train_UMGY_SmoothedDay_effect}

\hfill\break

Trains (and optionally \emph{regularizes} if the \hyperref[func.train_UMGY_SmoothedDay_effect.args.lambda]{lambda} argument value is non-default) the \emph{User+Movie+Genre+Year+SmoothedDay Effect} (hereafter \emph{UMGYDE} for short) \emph{Model} using the training sample of the \texttt{Movielens} dataset passed in the \hyperref[func.train_UMGY_SmoothedDay_effect.args.train_set]{train\_set} argument.

\phantomsection\label{func.train_UMGY_SmoothedDay_effect.usage}
\subparagraph{\texorpdfstring{\href{https://github.com/AzKurban-edX-DS/Capstone-MovieLens/blob/main/r/src/capstone-movielens.main.R\#L1950}{Usage}}{Usage}}\label{usage-72}

\hfill\break

\begin{Shaded}
\begin{Highlighting}[]
\NormalTok{  lss.UMGYD\_effect }\OtherTok{\textless{}{-}}\NormalTok{ edx }\SpecialCharTok{|\textgreater{}} 
    \FunctionTok{train\_UMGY\_SmoothedDay\_effect}\NormalTok{(lss.best\_degree, lss.best\_span)}
\end{Highlighting}
\end{Shaded}

\phantomsection\label{func.train_UMGY_SmoothedDay_effect.args}
\subparagraph{Arguments}\label{arguments-69}

\hfill\break

\phantomsection\label{func.train_UMGY_SmoothedDay_effect.args.train_set}
\begin{itemize}
\tightlist
\item
  \textbf{train\_set:} Training sample of the \texttt{Movielens} dataset;
\end{itemize}

\phantomsection\label{func.train_UMGY_SmoothedDay_effect.args.degree}
\begin{itemize}
\tightlist
\item
  \textbf{degree:} (Optional, \emph{Integer}, \texttt{NA} by default) The degree of the polynomials to pass in the same-named argument to the internally called \hyperref[func.calc_UMGY_SmoothedDay_effect]{calc\_UMGY\_SmoothedDay\_effect} described above;
\end{itemize}

\phantomsection\label{func.train_UMGY_SmoothedDay_effect.args.span}
\begin{itemize}
\tightlist
\item
  \textbf{span:} (Optional, \emph{Numeric}, \texttt{NA} by default) The parameter \(\alpha\) (which controls the degree of smoothing) to pass in the same-named argument to the internally called \hyperref[func.calc_UMGY_SmoothedDay_effect]{calc\_UMGY\_SmoothedDay\_effect} described above.
\end{itemize}

\phantomsection\label{func.train_UMGY_SmoothedDay_effect.args.lambda}
\begin{itemize}
\tightlist
\item
  \textbf{lambda:} (Optional, \emph{Numeric}, \(0\) by default) Regularization parameter \(\lambda\).
\end{itemize}

\phantomsection\label{func.train_UMGY_SmoothedDay_effect.details}
\subparagraph{Details}\label{details-70}

\hfill\break

Internally, this function calls the helper functions \hyperref[func.calc_day_general_effect]{calc\_day\_general\_effect} and \hyperref[func.calc_UMGY_SmoothedDay_effect]{calc\_UMGY\_SmoothedDay\_effect} (both described above in this section), passing the values of the arguments \hyperref[func.train_UMGY_SmoothedDay_effect.args.degree]{degree} and \hyperref[func.train_UMGY_SmoothedDay_effect.args.span]{span} to the latter function in the same-named arguments.

\begin{noteblock}
The \emph{Regularization technique} described in \href{https://rafalab.dfci.harvard.edu/dsbook-part-2/highdim/regularization.html\#penalized-least-squares}{Section 24.2 \emph{Penalized least squares}} of the \emph{Course Textbook (New Edition)}\autocite{IDS2_24-2} is applied to the model being trained only when the \emph{regularization parameter} \(\lambda\) passed in the \hyperref[func.train_UMGY_SmoothedDay_effect.args.lambda]{lambda} argument, has a value other than the default (to be more exact, greater than zero).

\end{noteblock}

\phantomsection\label{func.train_UMGY_SmoothedDay_effect.value}
\subparagraph{Value}\label{value-67}

\hfill\break

Data frame object representing the trained (and optionally \emph{regularized}) \emph{UMGYDE Model}.

\phantomsection\label{func.train_UMGY_SmoothedDay_effect.code}
\subparagraph{Source Code}\label{source-code-74}

\hfill\break

The source code of the \href{https://github.com/AzKurban-edX-DS/Capstone-MovieLens/blob/main/r/src/support-functions/UMGYD-effect.functions.R\#L99}{train\_UMGY\_SmoothedDay\_effect} function is shown below:

\begin{Shaded}
\begin{Highlighting}[]
\NormalTok{train\_UMGY\_SmoothedDay\_effect }\OtherTok{\textless{}{-}} \ControlFlowTok{function}\NormalTok{(train\_set, }
                                          \AttributeTok{degree =} \ConstantTok{NA}\NormalTok{, }
                                          \AttributeTok{span =} \ConstantTok{NA}\NormalTok{,}
                                          \AttributeTok{lambda =} \DecValTok{0}\NormalTok{)\{}
  \ControlFlowTok{if}\NormalTok{ (}\FunctionTok{is.na}\NormalTok{(lambda)) \{}
    \FunctionTok{stop}\NormalTok{(}\StringTok{"Function: train\_UMGY\_SmoothedDay\_effect}
\StringTok{\textasciigrave{}lambda\textasciigrave{} is \textasciigrave{}NA\textasciigrave{}"}\NormalTok{)}
\NormalTok{  \}}
  
\NormalTok{  train\_set }\SpecialCharTok{|\textgreater{}} 
    \FunctionTok{calc\_day\_general\_effect}\NormalTok{(lambda) }\SpecialCharTok{|\textgreater{}}
    \FunctionTok{calc\_UMGY\_SmoothedDay\_effect}\NormalTok{(degree, span)}
\NormalTok{\}}
\end{Highlighting}
\end{Shaded}

\begin{noteblock}
The source code of the \href{https://github.com/AzKurban-edX-DS/Capstone-MovieLens/blob/main/r/src/support-functions/UMGYD-effect.functions.R\#L99}{train\_UMGY\_SmoothedDay\_effect} function is also available in the \href{https://github.com/AzKurban-edX-DS/Capstone-MovieLens/blob/main/r/src/support-functions/UMGYD-effect.functions.R\#L3}{Utility Functions} section of the \href{https://github.com/AzKurban-edX-DS/Capstone-MovieLens/blob/main/r/src/support-functions/UMGYD-effect.functions.R\#L99}{UMGYD-effect.functions.R} script on \emph{GitHub}.

\end{noteblock}

\newpage

\paragraph{\texorpdfstring{\href{https://github.com/AzKurban-edX-DS/Capstone-MovieLens/blob/main/r/src/support-functions/UMGYD-effect.functions.R\#L113}{train\_UMGY\_SmoothedDay\_effect.cv} Function}{train\_UMGY\_SmoothedDay\_effect.cv Function}}\label{func.train_UMGY_SmoothedDay_effect.cv}

\hfill\break

Trains (and optionally \emph{regularizes} if the \hyperref[func.train_UMGY_SmoothedDay_effect.cv.args.lambda]{lambda} argument value is non-default) the \emph{User+Movie+Genre+Year+SmoothedDay Effect (UMGYDE) Model} with use of \emph{K-Fold Cross-Validation} as described in \href{https://rafalab.dfci.harvard.edu/dsbook-part-2/ml/resampling-methods.html\#k-fold-cross-validation}{Section 29.6 \emph{Cross validation / K-fold cross validation}} of the \emph{Course Textbook (New Edition)}\autocite{IDS2_29-6_KFCV}.

\begin{noteblock}
The \(K\) value for the \emph{K-Fold Cross-Validation} is determined by the length of the \hyperref[appndx_b.edx_CV]{\texttt{edx\_CV} Object} described below in \hyperref[appndx_b]{Appendix B: Models Training Datasets} (in \emph{this Project}, we use \(K = 5\)).

\end{noteblock}

\phantomsection\label{func.train_UMGY_SmoothedDay_effect.cv.usage}
\subparagraph{Usage}\label{usage-73}

\hfill\break

\begin{Shaded}
\begin{Highlighting}[]
  \FunctionTok{train\_UMGY\_SmoothedDay\_effect.cv}\NormalTok{(best\_degree, }
\NormalTok{                                   best\_span, }
\NormalTok{                                   lambda)}
\end{Highlighting}
\end{Shaded}

\begin{noteblock}
In the above \href{https://github.com/AzKurban-edX-DS/Capstone-MovieLens/blob/main/r/src/support-functions/UMGYD-effect.functions.R\#L238}{code snippet}, the function \emph{being described here} is called internally from the \hyperref[func.regularize.train_UMGYD_effect.cv]{regularize.train\_UMGYD\_effect.cv} function described below in section \hyperref[umgyde-model-regularization-1]{UMGYDE Model: Regularization} of \emph{this Appendix}.

\end{noteblock}

\phantomsection\label{func.train_UMGY_SmoothedDay_effect.cv.args}
\subparagraph{Arguments}\label{arguments-70}

\hfill\break

\phantomsection\label{func.train_UMGY_SmoothedDay_effect.cv.args.degree}
\begin{itemize}
\tightlist
\item
  \textbf{degree:} (Optional, \emph{Integer}, \texttt{NA} by default) The degree of the polynomials to pass in the same-named argument to the internally called \hyperref[func.calc_UMGY_SmoothedDay_effect]{calc\_UMGY\_SmoothedDay\_effect} described above in \emph{this Section};
\end{itemize}

\phantomsection\label{func.train_UMGY_SmoothedDay_effect.cv.args.span}
\begin{itemize}
\tightlist
\item
  \textbf{span:} (Optional, \emph{Numeric}, \texttt{NA} by default) The parameter \(\alpha\) (which controls the degree of smoothing) to pass in the same-named argument to the internally called \hyperref[func.calc_UMGY_SmoothedDay_effect]{calc\_UMGY\_SmoothedDay\_effect} described above in \emph{this Section};
\end{itemize}

\phantomsection\label{func.train_UMGY_SmoothedDay_effect.cv.args.lambda}
\begin{itemize}
\tightlist
\item
  \textbf{lambda:} (Optional, \emph{Numeric}, \(0\) by default) Regularization parameter \(\lambda\).
\end{itemize}

\phantomsection\label{func.train_UMGY_SmoothedDay_effect.cv.details}
\subparagraph{Details}\label{details-71}

\hfill\break

Internally, this function calls the auxiliary functions \hyperref[func.calc_day_general_effect.cv]{calc\_day\_general\_effect.cv} and \hyperref[func.calc_UMGY_SmoothedDay_effect]{calc\_UMGY\_SmoothedDay\_effect} (both described above in this section), passing the values of the arguments \hyperref[func.train_UMGY_SmoothedDay_effect.cv.args.degree]{degree} and \hyperref[func.train_UMGY_SmoothedDay_effect.cv.args.span]{span} to the latter function in the same-named arguments.

This function is also called within other helper functions, such as the \hyperref[func.train_UMGY_SmoothedDay_effect.RMSE.cv]{train\_UMGY\_SmoothedDay\_effect.RMSE.cv} and \hyperref[func.regularize.train_UMGYD_effect.cv]{regularize.train\_UMGYD\_effect.cv} (described below in sections \hyperref[umgyde-model-tuning-loess-params]{UMGYDE Model: Tuning \texttt{loess} Params} and \hyperref[umgyde-model-regularization-1]{UMGYDE Model: Regularization} of \emph{this Appendix}, respectively). The usage in the latter function is shown in the \hyperref[func.train_UMGY_SmoothedDay_effect.cv.usage]{Usage} section of \emph{this Description} above.

\begin{noteblock}
The \emph{Regularization technique} described in \href{https://rafalab.dfci.harvard.edu/dsbook-part-2/highdim/regularization.html\#penalized-least-squares}{Section 24.2 \emph{Penalized least squares}} of the \emph{Course Textbook (New Edition)}\autocite{IDS2_24-2} is applied to the model being trained only when the \emph{regularization parameter} \(\lambda\) passed in the \hyperref[func.train_UMGY_SmoothedDay_effect.cv.args.lambda]{lambda} argument, has a value other than the default (to be more exact, greater than zero).

\end{noteblock}

\phantomsection\label{func.train_UMGY_SmoothedDay_effect.cv.value}
\subparagraph{Value}\label{value-68}

\hfill\break

Data frame object representing the trained (and optionally \emph{regularized}) \emph{UMGYDE Model}.

\phantomsection\label{func.train_UMGY_SmoothedDay_effect.cv.code}
\subparagraph{Source Code}\label{source-code-75}

\hfill\break

The source code of the \href{https://github.com/AzKurban-edX-DS/Capstone-MovieLens/blob/main/r/src/support-functions/UMGYD-effect.functions.R\#L113}{train\_UMGY\_SmoothedDay\_effect.cv} function is shown below:

\begin{Shaded}
\begin{Highlighting}[]
\NormalTok{train\_UMGY\_SmoothedDay\_effect.cv }\OtherTok{\textless{}{-}} \ControlFlowTok{function}\NormalTok{(}\AttributeTok{degree =} \ConstantTok{NA}\NormalTok{, }
                                             \AttributeTok{span =} \ConstantTok{NA}\NormalTok{,}
                                             \AttributeTok{lambda =} \DecValTok{0}\NormalTok{)\{}
  \ControlFlowTok{if}\NormalTok{ (}\FunctionTok{is.na}\NormalTok{(lambda)) \{}
    \FunctionTok{stop}\NormalTok{(}\StringTok{"Function: train\_UMGY\_SmoothedDay\_effect.cv}
\StringTok{\textasciigrave{}lambda\textasciigrave{} is \textasciigrave{}NA\textasciigrave{}"}\NormalTok{)}
\NormalTok{  \}}
  
  \FunctionTok{calc\_day\_general\_effect.cv}\NormalTok{(lambda) }\SpecialCharTok{|\textgreater{}}
    \FunctionTok{calc\_UMGY\_SmoothedDay\_effect}\NormalTok{(degree, span)}
\NormalTok{\}}
\end{Highlighting}
\end{Shaded}

\begin{noteblock}
The source code of the \href{https://github.com/AzKurban-edX-DS/Capstone-MovieLens/blob/main/r/src/support-functions/UMGYD-effect.functions.R\#L113}{train\_UMGY\_SmoothedDay\_effect.cv} function is also available in the \href{https://github.com/AzKurban-edX-DS/Capstone-MovieLens/blob/main/r/src/support-functions/UMGYD-effect.functions.R\#L3}{Utility Functions} section of the \href{https://github.com/AzKurban-edX-DS/Capstone-MovieLens/blob/main/r/src/support-functions/UMGYD-effect.functions.R\#L113}{UMGYD-effect.functions.R} script on \emph{GitHub}.

\end{noteblock}

\newpage

\paragraph{\texorpdfstring{\href{https://github.com/AzKurban-edX-DS/Capstone-MovieLens/blob/main/r/src/support-functions/UMGYD-effect.functions.R\#L125}{UMGY\_SmoothedDay\_effect.predict} Function}{UMGY\_SmoothedDay\_effect.predict Function}}\label{func.UMGY_SmoothedDay_effect.predict}

\hfill\break

Predicts \emph{user rating} values on the \emph{test dataset} passed in the \hyperref[func.UMGY_SmoothedDay_effect.predict.args.test_set]{test\_set} argument based on the data of the trained \emph{UMGYD Effect (UMGYDE) Model} passed in the \hyperref[func.UMGY_SmoothedDay_effect.predict.args.day_smoothed_effect]{day\_smoothed\_effect} argument.

\phantomsection\label{func.UMGY_SmoothedDay_effect.predict.usage}
\subparagraph{\texorpdfstring{\href{https://github.com/AzKurban-edX-DS/Capstone-MovieLens/blob/main/r/src/capstone-movielens.main.R\#L2300}{Usage}}{Usage}}\label{usage-74}

\hfill\break

\begin{Shaded}
\begin{Highlighting}[]
\CommentTok{\# Compute predictions from Regularized UMGYDE Model}
\NormalTok{final.UMGYDE.predicted }\OtherTok{\textless{}{-}}\NormalTok{ final\_holdout\_test }\SpecialCharTok{|\textgreater{}}
  \FunctionTok{UMGY\_SmoothedDay\_effect.predict}\NormalTok{(rglr.UMGYD\_effect)}
\end{Highlighting}
\end{Shaded}

\phantomsection\label{func.UMGY_SmoothedDay_effect.predict.args}
\subparagraph{Arguments}\label{arguments-71}

\hfill\break

\phantomsection\label{func.UMGY_SmoothedDay_effect.predict.args.test_set}
\begin{itemize}
\tightlist
\item
  \textbf{test\_set:} \emph{Validation} dataset;
\end{itemize}

\phantomsection\label{func.UMGY_SmoothedDay_effect.predict.args.day_smoothed_effect}
\begin{itemize}
\tightlist
\item
  \textbf{day\_smoothed\_effect:} \emph{Data frame} object representing the \emph{UMGYDE Model}.
\end{itemize}

\phantomsection\label{func.UMGY_SmoothedDay_effect.predict.details}
\subparagraph{Details}\label{details-72}

\hfill\break

This function implements an algorithm for predicting the \emph{user rating values} in accordance with the \emph{mathematical description} of the \emph{UMGYDE Model}, as described in section \hyperref[mathematical-description-of-the-umgyde-model]{Mathematical Description of the UMGYDE Model} of \emph{this Report}.

\begin{noteblock}
This function is also called within the custom helper function \hyperref[func.calc_UMGY_SmoothedDay_effect.MSE]{calc\_UMGY\_SmoothedDay\_effect.MSE} described below in \emph{this Section}.

\end{noteblock}

\phantomsection\label{func.UMGY_SmoothedDay_effect.predict.value}
\subparagraph{Value}\label{value-69}

\hfill\break

\emph{Data frame} object containing \emph{predicted user rating} values based on the \emph{UMGYDE Model}

\newpage

\phantomsection\label{func.UMGY_SmoothedDay_effect.predict.code}
\subparagraph{Source Code}\label{source-code-76}

\hfill\break

The source code of the \href{https://github.com/AzKurban-edX-DS/Capstone-MovieLens/blob/main/r/src/support-functions/UMGYD-effect.functions.R\#L125}{UMGY\_SmoothedDay\_effect.predict} function is shown below:

\begin{Shaded}
\begin{Highlighting}[]
\NormalTok{UMGY\_SmoothedDay\_effect.predict }\OtherTok{\textless{}{-}} \ControlFlowTok{function}\NormalTok{(test\_set, day\_smoothed\_effect) \{}
\NormalTok{  test\_set }\SpecialCharTok{|\textgreater{}}
    \FunctionTok{left\_join}\NormalTok{(edx.user\_effect, }\AttributeTok{by =} \StringTok{"userId"}\NormalTok{) }\SpecialCharTok{|\textgreater{}}
    \FunctionTok{left\_join}\NormalTok{(rglr.UM\_effect, }\AttributeTok{by =} \StringTok{"movieId"}\NormalTok{) }\SpecialCharTok{|\textgreater{}}
    \FunctionTok{left\_join}\NormalTok{(rglr.UMG\_effect, }\AttributeTok{by =} \StringTok{"movieId"}\NormalTok{) }\SpecialCharTok{|\textgreater{}}
    \FunctionTok{left\_join}\NormalTok{(date\_days\_map, }\AttributeTok{by =} \StringTok{"timestamp"}\NormalTok{) }\SpecialCharTok{|\textgreater{}}
    \FunctionTok{left\_join}\NormalTok{(rglr.UMGY\_effect, }\AttributeTok{by=}\StringTok{\textquotesingle{}year\textquotesingle{}}\NormalTok{) }\SpecialCharTok{|\textgreater{}}
    \FunctionTok{left\_join}\NormalTok{(day\_smoothed\_effect, }\AttributeTok{by=}\StringTok{\textquotesingle{}days\textquotesingle{}}\NormalTok{) }\SpecialCharTok{|\textgreater{}}
    \FunctionTok{mutate}\NormalTok{(}\AttributeTok{predicted =} \FunctionTok{clamp}\NormalTok{(mu }\SpecialCharTok{+}\NormalTok{ a }\SpecialCharTok{+}\NormalTok{ b }\SpecialCharTok{+}\NormalTok{ g }\SpecialCharTok{+} 
                               \FunctionTok{ifelse}\NormalTok{(}\FunctionTok{is.na}\NormalTok{(ye), }\DecValTok{0}\NormalTok{, ye) }\SpecialCharTok{+}
                               \FunctionTok{ifelse}\NormalTok{(}\FunctionTok{is.na}\NormalTok{(de\_smoothed), }
                                      \DecValTok{0}\NormalTok{, }
\NormalTok{                                      de\_smoothed))) }\SpecialCharTok{|\textgreater{}} 
    \FunctionTok{select}\NormalTok{(userId, movieId, timestamp, rating, predicted)}
\NormalTok{\}}
\end{Highlighting}
\end{Shaded}

\begin{noteblock}
The source code of the \href{https://github.com/AzKurban-edX-DS/Capstone-MovieLens/blob/main/r/src/support-functions/UMGYD-effect.functions.R\#L125}{UMGY\_SmoothedDay\_effect.predict} function is also available in the \href{https://github.com/AzKurban-edX-DS/Capstone-MovieLens/blob/main/r/src/support-functions/UMGYD-effect.functions.R\#L3}{Utility Functions} section of the \href{https://github.com/AzKurban-edX-DS/Capstone-MovieLens/blob/main/r/src/support-functions/UMGYD-effect.functions.R\#L125}{UMGYD-effect.functions.R} script on \emph{GitHub}.

\end{noteblock}

\newpage

\paragraph{\texorpdfstring{\href{https://github.com/AzKurban-edX-DS/Capstone-MovieLens/blob/main/r/src/support-functions/UMGYD-effect.functions.R\#L141}{calc\_UMGY\_SmoothedDay\_effect.MSE} Function}{calc\_UMGY\_SmoothedDay\_effect.MSE Function}}\label{func.calc_UMGY_SmoothedDay_effect.MSE}

\hfill\break

Validates the \emph{UMGYD Effect (UMGYDE) Model} (represented by the data object passed in the \hyperref[func.calc_UMGY_SmoothedDay_effect.MSE.args.day_smoothed_effect]{day\_smoothed\_effect} argument) on the \emph{Test dataset} passed in the \hyperref[func.calc_UMGY_SmoothedDay_effect.MSE.args.test_set]{test\_set} argument.

\phantomsection\label{func.calc_UMGY_SmoothedDay_effect.MSE.usage}
\subparagraph{Usage}\label{usage-75}

\hfill\break

In the \href{https://github.com/AzKurban-edX-DS/Capstone-MovieLens/blob/main/r/src/support-functions/UMGYD-effect.functions.R\#L153}{code snippet} below, the function \emph{being described here} is called from the \hyperref[func.calc_UMGY_SmoothedDay_effect.MSE.cv]{calc\_UMGY\_SmoothedDay\_effect.MSE.cv} function (described below in \emph{this Section}) for each iteration of the \emph{K-Fold Cross-Validation}, to calculate the \emph{MSE} for the corresponding validation dataset of the \hyperref[appndx_b.edx_CV]{\texttt{edx\_CV} Object} (described in \hyperref[appndx_b]{Appendix B: Models Training Datasets}):

\begin{Shaded}
\begin{Highlighting}[]
\NormalTok{  smth\_day\_effect\_MSEs }\OtherTok{\textless{}{-}} \FunctionTok{sapply}\NormalTok{(edx\_CV, }\ControlFlowTok{function}\NormalTok{(cv\_fold\_dat)\{}
\NormalTok{    cv\_fold\_dat}\SpecialCharTok{$}\NormalTok{validation\_set }\SpecialCharTok{|\textgreater{}} 
      \FunctionTok{calc\_UMGY\_SmoothedDay\_effect.MSE}\NormalTok{(day\_smoothed\_effect)}
\NormalTok{  \})}
\end{Highlighting}
\end{Shaded}

\begin{noteblock}
The \(K\) value for the \emph{K-Fold Cross-Validation} is determined by the length of the \hyperref[appndx_b.edx_CV]{\texttt{edx\_CV} Object} (in \emph{this Project}, we use \(K = 5\)).

\end{noteblock}

\phantomsection\label{func.calc_UMGY_SmoothedDay_effect.MSE.args}
\subparagraph{Arguments}\label{arguments-72}

\hfill\break

\phantomsection\label{func.calc_UMGY_SmoothedDay_effect.MSE.args.test_set}
\begin{itemize}
\tightlist
\item
  \textbf{test\_set:} \emph{Validation} dataset;
\end{itemize}

\phantomsection\label{func.calc_UMGY_SmoothedDay_effect.MSE.args.day_smoothed_effect}
\begin{itemize}
\tightlist
\item
  \textbf{day\_smoothed\_effect:} \emph{Data frame} object representing the \emph{UMGYDE Model}.
\end{itemize}

\phantomsection\label{func.calc_UMGY_SmoothedDay_effect.MSE.details}
\subparagraph{Details}\label{details-73}

\hfill\break

The function calculates the \emph{Mean Squared Error (MSE)} metric for the \emph{UMGYDE Model}.

In \emph{this Project}, the following functions (described above in \emph{this Section}) return the data object representing the \emph{UMGYDE Model} (to be passed in the \hyperref[func.calc_UMGY_SmoothedDay_effect.MSE.args.day_smoothed_effect]{day\_smoothed\_effect} argument):

\begin{itemize}
\item
  \hyperref[func.calc_UMGY_SmoothedDay_effect]{calc\_UMGY\_SmoothedDay\_effect}
\item
  \hyperref[func.train_UMGY_SmoothedDay_effect]{train\_UMGY\_SmoothedDay\_effect}
\item
  \hyperref[func.train_UMGY_SmoothedDay_effect.cv]{train\_UMGY\_SmoothedDay\_effect.cv}.
\end{itemize}

Internally, this function calls the user-defined auxiliary function \hyperref[func.UMGY_SmoothedDay_effect.predict]{UMGY\_SmoothedDay\_effect.predict} (described above in \emph{this Section}), passing to it the \hyperref[func.calc_UMGY_SmoothedDay_effect.MSE.args.day_smoothed_effect]{day\_smoothed\_effect} argument value in the \emph{same-named} argument.

\begin{noteblock}
This is an auxiliary function used internally in the \hyperref[func.calc_UMGY_SmoothedDay_effect.MSE.cv]{calc\_UMGY\_SmoothedDay\_effect.MSE.cv} function (described below in \emph{this Section}) as shown in the \hyperref[func.calc_UMGY_SmoothedDay_effect.MSE.usage]{Usage} section of \emph{This Description} above.

\end{noteblock}

\phantomsection\label{func.calc_UMGY_SmoothedDay_effect.MSE.value}
\subparagraph{Value}\label{value-70}

\hfill\break

\emph{MSE} value calculated for the \emph{UMGYDE Model}.

\phantomsection\label{func.calc_UMGY_SmoothedDay_effect.MSE.code}
\subparagraph{Source Code}\label{source-code-77}

\hfill\break

The source code of the \href{https://github.com/AzKurban-edX-DS/Capstone-MovieLens/blob/main/r/src/support-functions/UMGYD-effect.functions.R\#L141}{calc\_UMGY\_SmoothedDay\_effect.MSE} function is shown below:

\begin{Shaded}
\begin{Highlighting}[]
\NormalTok{calc\_UMGY\_SmoothedDay\_effect.MSE }\OtherTok{\textless{}{-}} \ControlFlowTok{function}\NormalTok{(test\_set, day\_smoothed\_effect) \{}
\NormalTok{  test\_set }\SpecialCharTok{|\textgreater{}}
    \FunctionTok{UMGY\_SmoothedDay\_effect.predict}\NormalTok{(day\_smoothed\_effect) }\SpecialCharTok{|\textgreater{}}
    \FunctionTok{mutate}\NormalTok{(}\AttributeTok{resid =}\NormalTok{ rating }\SpecialCharTok{{-}}\NormalTok{ predicted) }\SpecialCharTok{|\textgreater{}} 
    \FunctionTok{pull}\NormalTok{(resid) }\SpecialCharTok{|\textgreater{}} \FunctionTok{mse}\NormalTok{()}
\NormalTok{\}}
\end{Highlighting}
\end{Shaded}

\begin{noteblock}
The source code of the \href{https://github.com/AzKurban-edX-DS/Capstone-MovieLens/blob/main/r/src/support-functions/UMGYD-effect.functions.R\#L141}{calc\_UMGY\_SmoothedDay\_effect.MSE} function is also available in the \href{https://github.com/AzKurban-edX-DS/Capstone-MovieLens/blob/main/r/src/support-functions/UMGYD-effect.functions.R\#L3}{Utility Functions} section of the \href{https://github.com/AzKurban-edX-DS/Capstone-MovieLens/blob/main/r/src/support-functions/UMGYD-effect.functions.R\#L141}{UMGYD-effect.functions.R} script on \emph{GitHub}.

\end{noteblock}

\newpage

\paragraph{\texorpdfstring{\href{https://github.com/AzKurban-edX-DS/Capstone-MovieLens/blob/main/r/src/support-functions/UMGYD-effect.functions.R\#L147}{calc\_UMGY\_SmoothedDay\_effect.MSE.cv} Function}{calc\_UMGY\_SmoothedDay\_effect.MSE.cv Function}}\label{func.calc_UMGY_SmoothedDay_effect.MSE.cv}

\hfill\break

Validates the \emph{UMGYD Effect (UMGYDE) Model} represented by the data object passed in the \hyperref[func.calc_UMGY_SmoothedDay_effect.MSE.cv.args.day_smoothed_effect]{day\_smoothed\_effect} argument with use of \emph{K-Fold Cross-Validation} as described in \href{https://rafalab.dfci.harvard.edu/dsbook-part-2/ml/resampling-methods.html\#k-fold-cross-validation}{Section 29.6 \emph{Cross validation / K-fold cross validation}} of the \emph{Course Textbook (New Edition)}\autocite{IDS2_29-6_KFCV}.

\begin{noteblock}
The \(K\) value for the \emph{K-Fold Cross-Validation} is determined by the length of the \hyperref[appndx_b.edx_CV]{\texttt{edx\_CV} Object} (in \emph{this Project}, we use \(K = 5\)).

\end{noteblock}

\phantomsection\label{func.calc_UMGY_SmoothedDay_effect.MSE.cv.usage}
\subparagraph{Usage}\label{usage-76}

\hfill\break

\begin{Shaded}
\begin{Highlighting}[]
\NormalTok{calc\_UMGY\_SmoothedDay\_effect.RMSE.cv }\OtherTok{\textless{}{-}} \ControlFlowTok{function}\NormalTok{(day\_smoothed\_effect)\{}
  \FunctionTok{sqrt}\NormalTok{(}\FunctionTok{calc\_UMGY\_SmoothedDay\_effect.MSE.cv}\NormalTok{(day\_smoothed\_effect))}
\NormalTok{\}}
\end{Highlighting}
\end{Shaded}

\begin{noteblock}
In the \href{https://github.com/AzKurban-edX-DS/Capstone-MovieLens/blob/main/r/src/support-functions/UMGYD-effect.functions.R\#L169}{code snippet} above, the function \emph{being described here} is called internally from the \hyperref[func.calc_UMGY_SmoothedDay_effect.RMSE.cv]{calc\_UMGY\_SmoothedDay\_effect.RMSE.cv} function described below in \emph{this Section}.

\end{noteblock}

\phantomsection\label{func.calc_UMGY_SmoothedDay_effect.MSE.cv.args}
\subparagraph{Arguments}\label{arguments-73}

\hfill\break

\phantomsection\label{func.calc_UMGY_SmoothedDay_effect.MSE.cv.args.day_smoothed_effect}
\begin{itemize}
\tightlist
\item
  \textbf{day\_smoothed\_effect:} \emph{Data frame} object representing the \emph{UMGYDE Model}.
\end{itemize}

\phantomsection\label{func.calc_UMGY_SmoothedDay_effect.MSE.cv.details}
\subparagraph{Details}\label{details-74}

\hfill\break

The function calculates the \emph{MSE} metric for the \emph{UMGYDE Model} passed in the \hyperref[func.calc_UMGY_SmoothedDay_effect.MSE.cv.args.day_smoothed_effect]{day\_smoothed\_effect} argument using the validation datasets contained in the \hyperref[appndx_b.edx_CV]{\texttt{edx\_CV} Object} described below in \hyperref[appndx_b]{Appendix B: Models Training Datasets}.

In \emph{this Project}, the following functions (described above in \emph{this Section}) return the data object representing the \emph{UMGYDE Model} (to be passed in the \hyperref[func.calc_UMGY_SmoothedDay_effect.MSE.cv.args.day_smoothed_effect]{day\_smoothed\_effect} argument):

\begin{itemize}
\item
  \hyperref[func.calc_UMGY_SmoothedDay_effect]{calc\_UMGY\_SmoothedDay\_effect}
\item
  \hyperref[func.train_UMGY_SmoothedDay_effect]{train\_UMGY\_SmoothedDay\_effect}
\item
  \hyperref[func.train_UMGY_SmoothedDay_effect.cv]{train\_UMGY\_SmoothedDay\_effect.cv}.
\end{itemize}

Internally, this function calls the \hyperref[func.calc_UMGY_SmoothedDay_effect.MSE]{calc\_UMGY\_SmoothedDay\_effect.MSE} function for each iteration of the \emph{K-Fold Cross-Validation}, passing \hyperref[func.calc_UMGY_SmoothedDay_effect.MSE.cv.args.day_smoothed_effect]{day\_smoothed\_effect} value in the same-named argument to the called function and calculating the \emph{MSE} for the corresponding validation dataset of the \hyperref[appndx_b.edx_CV]{\texttt{edx\_CV} Object}. After that, it calculates the mean of the obtained results.

\begin{noteblock}
This is an auxiliary function used internally in the \hyperref[func.calc_UMGY_SmoothedDay_effect.RMSE.cv]{calc\_UMGY\_SmoothedDay\_effect.RMSE.cv} function (described below in this section) as shown in the \hyperref[func.calc_UMGY_SmoothedDay_effect.MSE.cv.usage]{Usage} section of \emph{This Description} above.

\end{noteblock}

\phantomsection\label{func.calc_UMGY_SmoothedDay_effect.MSE.cv.value}
\subparagraph{Value}\label{value-71}

\hfill\break

The \emph{MSE} value calculated for the \emph{UMGYDE Model} using \emph{K-Fold Cross-Validation} (see the \hyperref[func.calc_UMGY_SmoothedDay_effect.MSE.cv.details]{Details} section of \emph{this Description} above for more details).

\phantomsection\label{func.calc_UMGY_SmoothedDay_effect.MSE.cv.code}
\subparagraph{Source Code}\label{source-code-78}

\hfill\break

The slightly simplified version of the source code of the \href{https://github.com/AzKurban-edX-DS/Capstone-MovieLens/blob/main/r/src/support-functions/UMGYD-effect.functions.R\#L147}{calc\_UMGY\_SmoothedDay\_effect.MSE.cv} function is shown below:

\begin{Shaded}
\begin{Highlighting}[]
\NormalTok{calc\_UMGY\_SmoothedDay\_effect.MSE.cv }\OtherTok{\textless{}{-}} \ControlFlowTok{function}\NormalTok{(day\_smoothed\_effect)\{}
\NormalTok{  smth\_day\_effect\_MSEs }\OtherTok{\textless{}{-}} \FunctionTok{sapply}\NormalTok{(edx\_CV, }\ControlFlowTok{function}\NormalTok{(cv\_fold\_dat)\{}
\NormalTok{    cv\_fold\_dat}\SpecialCharTok{$}\NormalTok{validation\_set }\SpecialCharTok{|\textgreater{}} 
      \FunctionTok{calc\_UMGY\_SmoothedDay\_effect.MSE}\NormalTok{(day\_smoothed\_effect)}
\NormalTok{  \})}
  \FunctionTok{put\_end\_date}\NormalTok{(start)}
  \FunctionTok{put\_log1}\NormalTok{(}\StringTok{"Function \textasciigrave{}calc\_UMGY\_SmoothedDay\_effect.MSE.cv\textasciigrave{}:}
\StringTok{MSE value have been computed for the \%1{-}Fold Cross Validation samples."}\NormalTok{, }
\NormalTok{           CVFolds\_N)}
  \FunctionTok{mean}\NormalTok{(smth\_day\_effect\_MSEs)}
\NormalTok{\}}
\end{Highlighting}
\end{Shaded}

\begin{noteblock}
The completed source code of the \href{https://github.com/AzKurban-edX-DS/Capstone-MovieLens/blob/main/r/src/support-functions/UMGYD-effect.functions.R\#L147}{calc\_UMGY\_SmoothedDay\_effect.MSE.cv} function is available in the \href{https://github.com/AzKurban-edX-DS/Capstone-MovieLens/blob/main/r/src/support-functions/UMGYD-effect.functions.R\#L3}{Utility Functions} section of the \href{https://github.com/AzKurban-edX-DS/Capstone-MovieLens/blob/main/r/src/support-functions/UMGYD-effect.functions.R\#L147}{UMGYD-effect.functions.R} script on \emph{GitHub}.

\end{noteblock}

\newpage

\paragraph{\texorpdfstring{\href{https://github.com/AzKurban-edX-DS/Capstone-MovieLens/blob/main/r/src/support-functions/UMGYD-effect.functions.R\#L169}{calc\_UMGY\_SmoothedDay\_effect.RMSE.cv} Function}{calc\_UMGY\_SmoothedDay\_effect.RMSE.cv Function}}\label{func.calc_UMGY_SmoothedDay_effect.RMSE.cv}

\hfill\break

Validates the \emph{UMGYD Effect (UMGYDE) Model} represented by the data object passed in the \hyperref[func.calc_UMGY_SmoothedDay_effect.RMSE.cv.args.day_smoothed_effect]{day\_smoothed\_effect} argument with use of \emph{K-Fold Cross-Validation} as described in \href{https://rafalab.dfci.harvard.edu/dsbook-part-2/ml/resampling-methods.html\#k-fold-cross-validation}{Section 29.6 \emph{Cross validation / K-fold cross validation}} of the \emph{Course Textbook (New Edition)}\autocite{IDS2_29-6_KFCV}.

\begin{noteblock}
The \(K\) value for the \emph{K-Fold Cross-Validation} is determined by the length of the \hyperref[appndx_b.edx_CV]{\texttt{edx\_CV} Object} (in \emph{this Project}, we use \(K = 5\)).

\end{noteblock}

\phantomsection\label{func.calc_UMGY_SmoothedDay_effect.RMSE.cv.usage}
\subparagraph{\texorpdfstring{\href{https://github.com/AzKurban-edX-DS/Capstone-MovieLens/blob/main/r/src/capstone-movielens.main.R\#L2007}{Usage}}{Usage}}\label{usage-77}

\hfill\break

\begin{Shaded}
\begin{Highlighting}[]
\NormalTok{lss.UMGYD\_effect.RMSE }\OtherTok{\textless{}{-}} \FunctionTok{calc\_UMGY\_SmoothedDay\_effect.RMSE.cv}\NormalTok{(lss.UMGYD\_effect)}
\end{Highlighting}
\end{Shaded}

\begin{Shaded}
\begin{Highlighting}[]
\NormalTok{lss.UMGYD\_effect.RMSE}
\end{Highlighting}
\end{Shaded}

\begin{verbatim}
## [1] 0.8707855
\end{verbatim}

\phantomsection\label{func.calc_UMGY_SmoothedDay_effect.RMSE.cv.args}
\subparagraph{Arguments}\label{arguments-74}

\hfill\break

\phantomsection\label{func.calc_UMGY_SmoothedDay_effect.RMSE.cv.args.day_smoothed_effect}
\begin{itemize}
\tightlist
\item
  \textbf{day\_smoothed\_effect:} \emph{Data frame} object representing the \emph{UMGYDE Model}.
\end{itemize}

\phantomsection\label{func.calc_UMGY_SmoothedDay_effect.RMSE.cv.details}
\subparagraph{Details}\label{details-75}

\hfill\break

The function calculates \emph{Root Squared Error (RMSE)} for the \emph{UMGYDE Model} object using the validation datasets contained in the \hyperref[appndx_b.edx_CV]{\texttt{edx\_CV} Object} described below in \hyperref[appndx_b]{Appendix B: Models Training Datasets}.\newline

During its execution, it internally calls the \hyperref[func.calc_UMGY_SmoothedDay_effect.MSE.cv]{calc\_UMGY\_SmoothedDay\_effect.MSE.cv} auxiliary function described above in \emph{this Section} to compute the \emph{MSE} for the \emph{UMGYDE Model} and then calculates the \emph{square root} of the obtained result.

In \emph{this Project}, a \emph{UMGYDE} data object (representing the \emph{UMGYDE Model}) being passed in the \hyperref[func.calc_UMGY_SmoothedDay_effect.RMSE.cv.args.day_smoothed_effect]{day\_smoothed\_effect} argument) can be created by any of the following functions (described above in \emph{this Section}):

\begin{itemize}
\item
  \hyperref[func.calc_UMGY_SmoothedDay_effect]{calc\_UMGY\_SmoothedDay\_effect}
\item
  \hyperref[func.train_UMGY_SmoothedDay_effect]{train\_UMGY\_SmoothedDay\_effect}
\item
  \hyperref[func.train_UMGY_SmoothedDay_effect.cv]{train\_UMGY\_SmoothedDay\_effect.cv}.
\end{itemize}

\begin{noteblock}
This function is also called internally as an auxiliary function from the \hyperref[func.train_UMGY_SmoothedDay_effect.RMSE.cv]{train\_UMGY\_SmoothedDay\_effect.RMSE.cv} and \hyperref[func.regularize.test_lambda.UMGYD_effect.cv]{regularize.test\_lambda.UMGYD\_effect.cv} functions described below in sections \hyperref[umgyde-model-tuning-loess-params]{UMGYDE Model: Tuning \texttt{loess} Params} and \hyperref[umgyde-model-regularization-1]{UMGYDE Model: Regularization}, respectively.

\end{noteblock}

\phantomsection\label{func.calc_UMGY_SmoothedDay_effect.RMSE.cv.value}
\subparagraph{Value}\label{value-72}

\hfill\break

\emph{RMSE} value calculated for the \emph{UMGYDE model} using \emph{K-fold cross-validation}.

\phantomsection\label{func.calc_UMGY_SmoothedDay_effect.RMSE.cv.code}
\subparagraph{Source Code}\label{source-code-79}

\hfill\break

The source code of the \href{https://github.com/AzKurban-edX-DS/Capstone-MovieLens/blob/main/r/src/support-functions/UMGYD-effect.functions.R\#L169}{calc\_UMGY\_SmoothedDay\_effect.RMSE.cv} function is shown below:

\begin{Shaded}
\begin{Highlighting}[]
\NormalTok{calc\_UMGY\_SmoothedDay\_effect.RMSE.cv }\OtherTok{\textless{}{-}} \ControlFlowTok{function}\NormalTok{(day\_smoothed\_effect)\{}
  \FunctionTok{sqrt}\NormalTok{(}\FunctionTok{calc\_UMGY\_SmoothedDay\_effect.MSE.cv}\NormalTok{(day\_smoothed\_effect))}
\NormalTok{\}}
\end{Highlighting}
\end{Shaded}

\begin{noteblock}
The source code of the \href{https://github.com/AzKurban-edX-DS/Capstone-MovieLens/blob/main/r/src/support-functions/UMGYD-effect.functions.R\#L169}{calc\_UMGY\_SmoothedDay\_effect.RMSE.cv} function is also available in the \href{https://github.com/AzKurban-edX-DS/Capstone-MovieLens/blob/main/r/src/support-functions/UMGYD-effect.functions.R\#L3}{Utility Functions} section of the \href{https://github.com/AzKurban-edX-DS/Capstone-MovieLens/blob/main/r/src/support-functions/UMGYD-effect.functions.R\#L169}{UMGYD-effect.functions.R} script on \emph{GitHub}.

\end{noteblock}

\newpage

\subsubsection{\texorpdfstring{UMGYDE Model: Tuning \texttt{loess} Params}{UMGYDE Model: Tuning loess Params}}\label{umgyde-model-tuning-loess-params}

\hfill\break

\begin{noteblock}
The complete source code of the functions described in this section is available in the \href{https://github.com/AzKurban-edX-DS/Capstone-MovieLens/blob/main/r/src/support-functions/UMGYD-effect.functions.R\#L172}{Tuning \texttt{loess} Parameters} section of the \href{https://github.com/AzKurban-edX-DS/Capstone-MovieLens/blob/main/r/src/support-functions/UMGYD-effect.functions.R\#L172}{UMGYD-effect.functions.R} script on \emph{GitHub}.

\end{noteblock}

\paragraph{\texorpdfstring{\href{https://github.com/AzKurban-edX-DS/Capstone-MovieLens/blob/main/r/src/support-functions/UMGYD-effect.functions.R\#L173}{train\_UMGY\_SmoothedDay\_effect.RMSE.cv} Function}{train\_UMGY\_SmoothedDay\_effect.RMSE.cv Function}}\label{func.train_UMGY_SmoothedDay_effect.RMSE.cv}

\hfill\break

Trains and validate (without \emph{regularization}) the \emph{User+Movie+Genre+Year+SmoothedDay Effect (UMGYDE) Model} for the given \href{https://www.rdocumentation.org/packages/stats/versions/3.6.2/topics/loess}{stats::loess} parameters passed in the corresponding arguments (\hyperref[func.train_UMGY_SmoothedDay_effect.RMSE.cv.args.degree]{degree} and \hyperref[func.train_UMGY_SmoothedDay_effect.RMSE.cv.args.span]{span}), with use of \emph{K-Fold Cross-Validation} as described in \href{https://rafalab.dfci.harvard.edu/dsbook-part-2/ml/resampling-methods.html\#k-fold-cross-validation}{Section 29.6 \emph{Cross validation / K-fold cross validation}} of the \emph{Course Textbook (New Edition)}\autocite{IDS2_29-6_KFCV}.

\begin{noteblock}
The \(K\) value for the \emph{K-Fold Cross-Validation} is determined by the length of the \hyperref[appndx_b.edx_CV]{\texttt{edx\_CV} Object} described below in \hyperref[appndx_b]{Appendix B: Models Training Datasets} (in \emph{this Project}, we use \(K = 5\)).

\end{noteblock}

\phantomsection\label{func.train_UMGY_SmoothedDay_effect.RMSE.cv.usage}
\subparagraph{Usage}\label{usage-78}

\hfill\break

\begin{Shaded}
\begin{Highlighting}[]
\NormalTok{train\_UMGY\_SmoothedDay\_effect.RMSE.cv.degree1 }\OtherTok{\textless{}{-}} \ControlFlowTok{function}\NormalTok{(span) \{}
  \FunctionTok{train\_UMGY\_SmoothedDay\_effect.RMSE.cv}\NormalTok{(}\AttributeTok{degree =} \DecValTok{1}\NormalTok{, span)}
\NormalTok{\}}
\end{Highlighting}
\end{Shaded}

\begin{noteblock}
In the above \href{https://github.com/AzKurban-edX-DS/Capstone-MovieLens/blob/main/r/src/support-functions/UMGYD-effect.functions.R\#L195}{code snippet}, the function \emph{being described here} is called within the \hyperref[func.train_UMGY_SmoothedDay_effect.RMSE.cv.degree1]{train\_UMGY\_SmoothedDay\_effect.RMSE.cv.degree1} function described below in \emph{this Section}.

\end{noteblock}

\phantomsection\label{func.train_UMGY_SmoothedDay_effect.RMSE.cv.args}
\subparagraph{Arguments}\label{arguments-75}

\hfill\break

\phantomsection\label{func.train_UMGY_SmoothedDay_effect.RMSE.cv.args.degree}
\begin{itemize}
\tightlist
\item
  \textbf{degree:} (Optional, \emph{Integer}, \texttt{NA} by default) \emph{The \href{https://www.rdocumentation.org/packages/stats/versions/3.6.2/topics/loess}{loess} parameter}: the degree of the polynomials to pass in the same-named argument to the internally called \hyperref[func.train_UMGY_SmoothedDay_effect.cv]{train\_UMGY\_SmoothedDay\_effect.cv} described above;
\end{itemize}

\phantomsection\label{func.train_UMGY_SmoothedDay_effect.RMSE.cv.args.span}
\begin{itemize}
\tightlist
\item
  \textbf{span:} (Optional, \emph{Numeric}, \texttt{NA} by default) \emph{The \href{https://www.rdocumentation.org/packages/stats/versions/3.6.2/topics/loess}{loess} parameter}: the parameter \(\alpha\) (which controls the degree of smoothing) to pass in the same-named argument to the internally called \hyperref[func.train_UMGY_SmoothedDay_effect.cv]{train\_UMGY\_SmoothedDay\_effect.cv} described above.
\end{itemize}

\phantomsection\label{func.train_UMGY_SmoothedDay_effect.RMSE.cv.details}
\subparagraph{Details}\label{details-76}

\hfill\break

This is a wrapper for the function \hyperref[func.train_UMGY_SmoothedDay_effect.cv]{train\_UMGY\_SmoothedDay\_effect.cv} (described above in section \hyperref[umgyde-model-utility-functions]{UMGYDE Model: Utility Functions} of \emph{this Appendix}), which it calls internally, passing the \hyperref[func.train_UMGY_SmoothedDay_effect.RMSE.cv.args.degree]{degree} and \hyperref[func.train_UMGY_SmoothedDay_effect.RMSE.cv.args.span]{span} values as arguments of the same name.

Unlike the inner function \hyperref[func.train_UMGY_SmoothedDay_effect.cv]{train\_UMGY\_SmoothedDay\_effect.cv}, this one additionally validates the \emph{UMGYDE Model} (created and trained by this inner function), for which it calls another custom helper function, \hyperref[func.calc_UMGY_SmoothedDay_effect.RMSE.cv]{calc\_UMGY\_SmoothedDay\_effect.RMSE.cv} (also described above in section \hyperref[umgyde-model-utility-functions]{UMGYDE Model: Utility Functions}), computing \emph{RMSE} score for the model.

\begin{noteblock}

This is an auxiliary function intended to be called internally from the following helper functions (described below in this section), specifically designed for tuning the \texttt{degree} and \texttt{span} \href{https://www.rdocumentation.org/packages/stats/versions/3.6.2/topics/loess}{loess} parameters:

\begin{itemize}
\item
  \hyperref[func.train_UMGY_SmoothedDay_effect.RMSE.cv.degree0]{train\_UMGY\_SmoothedDay\_effect.RMSE.cv.degree0}: a helper function designed for tuning the \href{https://www.rdocumentation.org/packages/stats/versions/3.6.2/topics/loess}{loess} \texttt{span} parameter while keeping the \texttt{degree} parameter fixed at \(0\);
\item
  \hyperref[func.train_UMGY_SmoothedDay_effect.RMSE.cv.degree1]{train\_UMGY\_SmoothedDay\_effect.RMSE.cv.degree1}: a helper function designed for tuning the \href{https://www.rdocumentation.org/packages/stats/versions/3.6.2/topics/loess}{loess} \texttt{span} parameter while keeping the \texttt{degree} parameter fixed at \(1\);
\item
  \hyperref[func.train_UMGY_SmoothedDay_effect.RMSE.cv.degree2]{train\_UMGY\_SmoothedDay\_effect.RMSE.cv.degree2}: a helper function designed for tuning the \href{https://www.rdocumentation.org/packages/stats/versions/3.6.2/topics/loess}{loess} \texttt{span} parameter while keeping the \texttt{degree} parameter fixed at \(2\);
\end{itemize}

\end{noteblock}

\phantomsection\label{func.train_UMGY_SmoothedDay_effect.RMSE.cv.value}
\subparagraph{Value}\label{value-73}

\hfill\break

The \emph{RMSE} value calculated for the \emph{UMGYDE Model} fitted for the given \href{https://www.rdocumentation.org/packages/stats/versions/3.6.2/topics/loess}{loess} parameters (passed in the \hyperref[func.train_UMGY_SmoothedDay_effect.RMSE.cv.args.degree]{degree} and \hyperref[func.train_UMGY_SmoothedDay_effect.RMSE.cv.args.span]{span} arguments).

\phantomsection\label{func.train_UMGY_SmoothedDay_effect.RMSE.cv.code}
\subparagraph{Source Code}\label{source-code-80}

\hfill\break

The slightly simplified version of the source code of the \href{https://github.com/AzKurban-edX-DS/Capstone-MovieLens/blob/main/r/src/support-functions/UMGYD-effect.functions.R\#L173}{train\_UMGY\_SmoothedDay\_effect.RMSE.cv} function is shown below:

\begin{Shaded}
\begin{Highlighting}[]
\NormalTok{train\_UMGY\_SmoothedDay\_effect.RMSE.cv }\OtherTok{\textless{}{-}} \ControlFlowTok{function}\NormalTok{(}\AttributeTok{degree =} \ConstantTok{NA}\NormalTok{, }\AttributeTok{span =} \ConstantTok{NA}\NormalTok{) \{}
\NormalTok{  smth\_de.RMSE }\OtherTok{\textless{}{-}} \FunctionTok{train\_UMGY\_SmoothedDay\_effect.cv}\NormalTok{(degree, span) }\SpecialCharTok{|\textgreater{}}
    \FunctionTok{calc\_UMGY\_SmoothedDay\_effect.RMSE.cv}\NormalTok{()}
  
  \FunctionTok{put\_log2}\NormalTok{(}\StringTok{"Function \textasciigrave{}train\_UMGY\_SmoothedDay\_effect.RMSE.cv\textasciigrave{}:}
\StringTok{RMSE has been computed for the \textasciigrave{}loess\textasciigrave{} function parameters: }
\StringTok{degree = \%1, span = \%2."}\NormalTok{, }
\NormalTok{           degree,}
\NormalTok{           span)}
  
\NormalTok{  smth\_de.RMSE}
\NormalTok{\}}
\end{Highlighting}
\end{Shaded}

\begin{noteblock}
The complete source code of the \href{https://github.com/AzKurban-edX-DS/Capstone-MovieLens/blob/main/r/src/support-functions/UMGYD-effect.functions.R\#L173}{train\_UMGY\_SmoothedDay\_effect.RMSE.cv} function is available in the \href{https://github.com/AzKurban-edX-DS/Capstone-MovieLens/blob/main/r/src/support-functions/UMGYD-effect.functions.R\#L3}{Utility Functions} section of the \href{https://github.com/AzKurban-edX-DS/Capstone-MovieLens/blob/main/r/src/support-functions/UMGYD-effect.functions.R\#L173}{UMGYD-effect.functions.R} script on \emph{GitHub}.

\end{noteblock}

\newpage

\paragraph{\texorpdfstring{\href{https://github.com/AzKurban-edX-DS/Capstone-MovieLens/blob/main/r/src/support-functions/UMGYD-effect.functions.R\#L192}{train\_UMGY\_SmoothedDay\_effect.RMSE.cv.degree0} Function}{train\_UMGY\_SmoothedDay\_effect.RMSE.cv.degree0 Function}}\label{func.train_UMGY_SmoothedDay_effect.RMSE.cv.degree0}

\hfill\break

Trains and validates (without \emph{regularization}) the \emph{User+Movie+Genre+Year+SmoothedDay Effect (UMGYDE) Model} for the following values of the \href{https://www.rdocumentation.org/packages/stats/versions/3.6.2/topics/loess}{stats::loess} parameters:

\begin{itemize}
\tightlist
\item
  a fixed \texttt{degree} value of \(0\);
\item
  the \texttt{span} value passed in the \hyperref[func.train_UMGY_SmoothedDay_effect.RMSE.cv.degree0.args.span]{span} argument.
\end{itemize}

\phantomsection\label{func.train_UMGY_SmoothedDay_effect.RMSE.cv.degree0.usage}
\subparagraph{Usage}\label{usage-79}

\hfill\break

\begin{Shaded}
\begin{Highlighting}[]
\NormalTok{spans }\OtherTok{\textless{}{-}} \FunctionTok{seq}\NormalTok{(}\FloatTok{0.0005}\NormalTok{, }\DecValTok{1}\NormalTok{, }\FloatTok{0.001}\NormalTok{)}
\NormalTok{lss.UMGYDE.preset.degree0.result }\OtherTok{\textless{}{-}} 
  \FunctionTok{tune.model\_param}\NormalTok{(spans, train\_UMGY\_SmoothedDay\_effect.RMSE.cv.degree0)}
\end{Highlighting}
\end{Shaded}

\begin{Shaded}
\begin{Highlighting}[]
\FunctionTok{str}\NormalTok{(lss.UMGYDE.preset.degree0.result)}
\end{Highlighting}
\end{Shaded}

\begin{verbatim}
## List of 2
##  $ tuned.result:'data.frame':    5 obs. of  2 variables:
##   ..$ RMSE           : num [1:5] 0.872 0.871 0.871 0.872 0.872
##   ..$ parameter.value: num [1:5] 0.0005 0.0015 0.0025 0.0035 0.0045
##  $ best_result : Named num [1:2] 0.0015 0.8712
##   ..- attr(*, "names")= chr [1:2] "param.best_value" "best_RMSE"
\end{verbatim}

\begin{noteblock}
In the \href{https://github.com/AzKurban-edX-DS/Capstone-MovieLens/blob/main/r/src/capstone-movielens.main.R\#L1578}{code snippet} above the function \emph{being described here} is passed in the \hyperref[func.tune.model_param.args.fn_tune.test.param_value]{fn\_tune.test.param\_value} argument to the \hyperref[func.tune.model_param]{tune.model\_param} function described above in section \hyperref[appndx_a.CHF.model_tuning_utils]{Model Tuning Utils} of \emph{this Appendix}.

\end{noteblock}

\phantomsection\label{func.train_UMGY_SmoothedDay_effect.RMSE.cv.degree0.args}
\subparagraph{Arguments}\label{arguments-76}

\hfill\break

\phantomsection\label{func.train_UMGY_SmoothedDay_effect.RMSE.cv.degree0.args.span}
\begin{itemize}
\tightlist
\item
  \textbf{span:} (Optional, \emph{Numeric}, \texttt{NA} by default) \emph{The \href{https://www.rdocumentation.org/packages/stats/versions/3.6.2/topics/loess}{loess} parameter}: the parameter \(\alpha\) (which controls the degree of smoothing) to pass in the same-named argument to the internally called \hyperref[func.train_UMGY_SmoothedDay_effect.RMSE.cv]{train\_UMGY\_SmoothedDay\_effect.RMSE.cv} function described above.
\end{itemize}

\phantomsection\label{func.train_UMGY_SmoothedDay_effect.RMSE.cv.degree0.details}
\subparagraph{Details}\label{details-77}

\hfill\break

This function is a wrapper for the function \hyperref[func.train_UMGY_SmoothedDay_effect.RMSE.cv]{train\_UMGY\_SmoothedDay\_effect.RMSE.cv} (described above in \emph{this Section}) that it calls internally, passing it the following \href{https://www.rdocumentation.org/packages/stats/versions/3.6.2/topics/loess}{stats::loess} parameter values:

\begin{itemize}
\item
  \texttt{degree}: \(0\) in the \hyperref[func.train_UMGY_SmoothedDay_effect.RMSE.cv.args.degree]{degree} argument;
\item
  \texttt{span}: The \hyperref[func.train_UMGY_SmoothedDay_effect.RMSE.cv.degree0.args.span]{span} argument value (of this function) in the same-named argument (\hyperref[func.train_UMGY_SmoothedDay_effect.RMSE.cv.args.span]{span}).
\end{itemize}

\begin{noteblock}
This is an auxiliary function intended to be passed in the \texttt{fn\_tune.test.param\_value} argument to the \hyperref[func.tune.model_param]{tune.model\_param} and \hyperref[func.model.tune.param_range]{model.tune.param\_range} functions described above in section \hyperref[appndx_a.CHF.model_tuning_utils]{Model Tuning Utils} of \emph{this Appendix}.

\end{noteblock}

\phantomsection\label{func.train_UMGY_SmoothedDay_effect.RMSE.cv.degree0.value}
\subparagraph{Value}\label{value-74}

\hfill\break

The \emph{RMSE} value calculated for the \emph{UMGYDE Model} fitted for the given \href{https://www.rdocumentation.org/packages/stats/versions/3.6.2/topics/loess}{loess} parameters (\texttt{degree\ =\ 0} and the \texttt{span} value passed in the \hyperref[func.train_UMGY_SmoothedDay_effect.RMSE.cv.degree0.args.span]{span} argument), and trained with use of \emph{K-Fold Cross-Validation}, where the \(K\) is the length of the \hyperref[appndx_b.edx_CV]{\texttt{edx\_CV} Object} (in \emph{this Project}, we use \(K = 5\)).

\newpage

\phantomsection\label{func.train_UMGY_SmoothedDay_effect.RMSE.cv.degree0.code}
\subparagraph{Source Code}\label{source-code-81}

\hfill\break

The source code of the \href{https://github.com/AzKurban-edX-DS/Capstone-MovieLens/blob/main/r/src/support-functions/UMGYD-effect.functions.R\#L192}{train\_UMGY\_SmoothedDay\_effect.RMSE.cv.degree0} function is shown below:

\begin{Shaded}
\begin{Highlighting}[]
\NormalTok{train\_UMGY\_SmoothedDay\_effect.RMSE.cv.degree0 }\OtherTok{\textless{}{-}} \ControlFlowTok{function}\NormalTok{(span) \{}
  \FunctionTok{train\_UMGY\_SmoothedDay\_effect.RMSE.cv}\NormalTok{(}\AttributeTok{degree =} \DecValTok{0}\NormalTok{, span)}
\NormalTok{\}}
\end{Highlighting}
\end{Shaded}

\begin{noteblock}
The source code of the \href{https://github.com/AzKurban-edX-DS/Capstone-MovieLens/blob/main/r/src/support-functions/UMGYD-effect.functions.R\#L192}{train\_UMGY\_SmoothedDay\_effect.RMSE.cv.degree0} function is also available in the \href{https://github.com/AzKurban-edX-DS/Capstone-MovieLens/blob/main/r/src/support-functions/UMGYD-effect.functions.R\#L3}{Utility Functions} section of the \href{https://github.com/AzKurban-edX-DS/Capstone-MovieLens/blob/main/r/src/support-functions/UMGYD-effect.functions.R\#L192}{UMGYD-effect.functions.R} script on \emph{GitHub}.

\end{noteblock}

\newpage

\paragraph{\texorpdfstring{\href{https://github.com/AzKurban-edX-DS/Capstone-MovieLens/blob/main/r/src/support-functions/UMGYD-effect.functions.R\#L195}{train\_UMGY\_SmoothedDay\_effect.RMSE.cv.degree1} Function}{train\_UMGY\_SmoothedDay\_effect.RMSE.cv.degree1 Function}}\label{func.train_UMGY_SmoothedDay_effect.RMSE.cv.degree1}

\hfill\break

Trains and validates (without \emph{regularization}) the \emph{User+Movie+Genre+Year+SmoothedDay Effect (UMGYDE) Model} for the following values of the \href{https://www.rdocumentation.org/packages/stats/versions/3.6.2/topics/loess}{stats::loess} parameters:

\begin{itemize}
\tightlist
\item
  a fixed \texttt{degree} value of \(1\);
\item
  the \texttt{span} value passed in the \hyperref[func.train_UMGY_SmoothedDay_effect.RMSE.cv.degree1.args.span]{span} argument.
\end{itemize}

\phantomsection\label{func.train_UMGY_SmoothedDay_effect.RMSE.cv.degree1.usage}
\subparagraph{Usage}\label{usage-80}

\hfill\break

\begin{Shaded}
\begin{Highlighting}[]
\NormalTok{spans }\OtherTok{\textless{}{-}} \FunctionTok{seq}\NormalTok{(}\FloatTok{0.0005}\NormalTok{, }\DecValTok{1}\NormalTok{, }\FloatTok{0.001}\NormalTok{)}
\NormalTok{lss.UMGYDE.preset.degree1.result }\OtherTok{\textless{}{-}} 
    \FunctionTok{tune.model\_param}\NormalTok{(spans, train\_UMGY\_SmoothedDay\_effect.RMSE.cv.degree1)}
\end{Highlighting}
\end{Shaded}

\begin{Shaded}
\begin{Highlighting}[]
\FunctionTok{str}\NormalTok{(lss.UMGYDE.preset.degree1.result)}
\end{Highlighting}
\end{Shaded}

\begin{verbatim}
## List of 2
##  $ tuned.result:'data.frame':    5 obs. of  2 variables:
##   ..$ RMSE           : num [1:5] 0.872 0.871 0.871 0.872 0.872
##   ..$ parameter.value: num [1:5] 0.0005 0.0015 0.0025 0.0035 0.0045
##  $ best_result : Named num [1:2] 0.0015 0.8711
##   ..- attr(*, "names")= chr [1:2] "param.best_value" "best_RMSE"
\end{verbatim}

\begin{noteblock}
In the \href{https://github.com/AzKurban-edX-DS/Capstone-MovieLens/blob/main/r/src/capstone-movielens.main.R\#L1695}{code snippet} above the function \emph{being described here} is passed in the \hyperref[func.tune.model_param.args.fn_tune.test.param_value]{fn\_tune.test.param\_value} argument to the \hyperref[func.tune.model_param]{tune.model\_param} function described above in section \hyperref[appndx_a.CHF.model_tuning_utils]{Model Tuning Utils} of \emph{this Appendix}.

\end{noteblock}

\phantomsection\label{func.train_UMGY_SmoothedDay_effect.RMSE.cv.degree1.args}
\subparagraph{Arguments}\label{arguments-77}

\hfill\break

\phantomsection\label{func.train_UMGY_SmoothedDay_effect.RMSE.cv.degree1.args.span}
\begin{itemize}
\tightlist
\item
  \textbf{span:} (Optional, \emph{Numeric}, \texttt{NA} by default) \emph{The \href{https://www.rdocumentation.org/packages/stats/versions/3.6.2/topics/loess}{loess} parameter}: the parameter \(\alpha\) (which controls the degree of smoothing) to pass in the same-named argument to the internally called \hyperref[func.train_UMGY_SmoothedDay_effect.RMSE.cv]{train\_UMGY\_SmoothedDay\_effect.RMSE.cv} function described above.
\end{itemize}

\phantomsection\label{func.train_UMGY_SmoothedDay_effect.RMSE.cv.degree1.details}
\subparagraph{Details}\label{details-78}

\hfill\break

This function is a wrapper for the function \hyperref[func.train_UMGY_SmoothedDay_effect.RMSE.cv]{train\_UMGY\_SmoothedDay\_effect.RMSE.cv} (described above in \emph{this Section}) that it calls internally, passing it the following \href{https://www.rdocumentation.org/packages/stats/versions/3.6.2/topics/loess}{stats::loess} parameter values:

\begin{itemize}
\item
  \texttt{degree}: \(1\) in the \hyperref[func.train_UMGY_SmoothedDay_effect.RMSE.cv.args.degree]{degree} argument;
\item
  \texttt{span}: The \hyperref[func.train_UMGY_SmoothedDay_effect.RMSE.cv.degree1.args.span]{span} argument value (of this function) in the same-named argument (\hyperref[func.train_UMGY_SmoothedDay_effect.RMSE.cv.args.span]{span}).
\end{itemize}

\begin{noteblock}
This is an auxiliary function intended to be passed in the \texttt{fn\_tune.test.param\_value} argument to the \hyperref[func.tune.model_param]{tune.model\_param} and \hyperref[func.model.tune.param_range]{model.tune.param\_range} functions described above in section \hyperref[appndx_a.CHF.model_tuning_utils]{Model Tuning Utils} of \emph{this Appendix}.

\end{noteblock}

\phantomsection\label{func.train_UMGY_SmoothedDay_effect.RMSE.cv.degree1.value}
\subparagraph{Value}\label{value-75}

\hfill\break

The \emph{RMSE} value calculated for the \emph{UMGYDE Model} fitted for the given \href{https://www.rdocumentation.org/packages/stats/versions/3.6.2/topics/loess}{loess} parameters (\texttt{degree\ =\ 1} and the \texttt{span} value passed in the \hyperref[func.train_UMGY_SmoothedDay_effect.RMSE.cv.degree1.args.span]{span} argument), and trained with use of \emph{K-Fold Cross-Validation}, where the \(K\) is the length of the \hyperref[appndx_b.edx_CV]{\texttt{edx\_CV} Object} (in \emph{this Project}, we use \(K = 5\)).

\phantomsection\label{func.train_UMGY_SmoothedDay_effect.RMSE.cv.degree1.code}
\subparagraph{Source Code}\label{source-code-82}

\hfill\break

The source code of the \href{https://github.com/AzKurban-edX-DS/Capstone-MovieLens/blob/main/r/src/support-functions/UMGYD-effect.functions.R\#L195}{train\_UMGY\_SmoothedDay\_effect.RMSE.cv.degree1} function is shown below:

\begin{Shaded}
\begin{Highlighting}[]
\NormalTok{train\_UMGY\_SmoothedDay\_effect.RMSE.cv.degree1 }\OtherTok{\textless{}{-}} \ControlFlowTok{function}\NormalTok{(span) \{}
  \FunctionTok{train\_UMGY\_SmoothedDay\_effect.RMSE.cv}\NormalTok{(}\AttributeTok{degree =} \DecValTok{1}\NormalTok{, span)}
\NormalTok{\}}
\end{Highlighting}
\end{Shaded}

\begin{noteblock}
The source code of the \href{https://github.com/AzKurban-edX-DS/Capstone-MovieLens/blob/main/r/src/support-functions/UMGYD-effect.functions.R\#L195}{train\_UMGY\_SmoothedDay\_effect.RMSE.cv.degree1} function is also available in the \href{https://github.com/AzKurban-edX-DS/Capstone-MovieLens/blob/main/r/src/support-functions/UMGYD-effect.functions.R\#L3}{Utility Functions} section of the \href{https://github.com/AzKurban-edX-DS/Capstone-MovieLens/blob/main/r/src/support-functions/UMGYD-effect.functions.R\#L195}{UMGYD-effect.functions.R} script on \emph{GitHub}.

\end{noteblock}

\newpage

\paragraph{\texorpdfstring{\href{https://github.com/AzKurban-edX-DS/Capstone-MovieLens/blob/main/r/src/support-functions/UMGYD-effect.functions.R\#L198}{train\_UMGY\_SmoothedDay\_effect.RMSE.cv.degree2} Function}{train\_UMGY\_SmoothedDay\_effect.RMSE.cv.degree2 Function}}\label{func.train_UMGY_SmoothedDay_effect.RMSE.cv.degree2}

\hfill\break

Trains and validates (without \emph{regularization}) the \emph{User+Movie+Genre+Year+SmoothedDay Effect (UMGYDE) Model} for the following values of the \href{https://www.rdocumentation.org/packages/stats/versions/3.6.2/topics/loess}{stats::loess} parameters:

\begin{itemize}
\tightlist
\item
  a fixed \texttt{degree} value of \(2\);
\item
  the \texttt{span} value passed in the \hyperref[func.train_UMGY_SmoothedDay_effect.RMSE.cv.degree2.args.span]{span} argument.
\end{itemize}

\phantomsection\label{func.train_UMGY_SmoothedDay_effect.RMSE.cv.degree2.usage}
\subparagraph{Usage}\label{usage-81}

\hfill\break

\begin{Shaded}
\begin{Highlighting}[]
\NormalTok{spans }\OtherTok{\textless{}{-}} \FunctionTok{seq}\NormalTok{(}\FloatTok{0.0005}\NormalTok{, }\DecValTok{1}\NormalTok{, }\FloatTok{0.001}\NormalTok{)}
\NormalTok{lss.UMGYDE.preset.degree2.result }\OtherTok{\textless{}{-}} 
  \FunctionTok{tune.model\_param}\NormalTok{(spans, train\_UMGY\_SmoothedDay\_effect.RMSE.cv.degree2)}
\end{Highlighting}
\end{Shaded}

\begin{Shaded}
\begin{Highlighting}[]
\FunctionTok{str}\NormalTok{(lss.UMGYDE.preset.degree2.result)}
\end{Highlighting}
\end{Shaded}

\begin{verbatim}
## List of 2
##  $ tuned.result:'data.frame':    5 obs. of  2 variables:
##   ..$ RMSE           : num [1:5] 0.872 0.871 0.871 0.871 0.871
##   ..$ parameter.value: num [1:5] 0.0005 0.0015 0.0025 0.0035 0.0045
##  $ best_result : Named num [1:2] 0.0015 0.871
##   ..- attr(*, "names")= chr [1:2] "param.best_value" "best_RMSE"
\end{verbatim}

\begin{noteblock}
In the \href{https://github.com/AzKurban-edX-DS/Capstone-MovieLens/blob/main/r/src/capstone-movielens.main.R\#L1807}{code snippet} above the function \emph{being described here} is passed in the \hyperref[func.tune.model_param.args.fn_tune.test.param_value]{fn\_tune.test.param\_value} argument to the \hyperref[func.tune.model_param]{tune.model\_param} function described above in section \hyperref[appndx_a.CHF.model_tuning_utils]{Model Tuning Utils} of \emph{this Appendix}.

\end{noteblock}

\phantomsection\label{func.train_UMGY_SmoothedDay_effect.RMSE.cv.degree2.args}
\subparagraph{Arguments}\label{arguments-78}

\hfill\break

\phantomsection\label{func.train_UMGY_SmoothedDay_effect.RMSE.cv.degree2.args.span}
\begin{itemize}
\tightlist
\item
  \textbf{span:} (Optional, \emph{Numeric}, \texttt{NA} by default) \emph{The \href{https://www.rdocumentation.org/packages/stats/versions/3.6.2/topics/loess}{loess} parameter}: the parameter \(\alpha\) (which controls the degree of smoothing) to pass in the same-named argument to the internally called \hyperref[func.train_UMGY_SmoothedDay_effect.RMSE.cv]{train\_UMGY\_SmoothedDay\_effect.RMSE.cv} function described above.
\end{itemize}

\phantomsection\label{func.train_UMGY_SmoothedDay_effect.RMSE.cv.degree2.details}
\subparagraph{Details}\label{details-79}

\hfill\break

This function is a wrapper for the function \hyperref[func.train_UMGY_SmoothedDay_effect.RMSE.cv]{train\_UMGY\_SmoothedDay\_effect.RMSE.cv} (described above in \emph{this Section}) that it calls internally, passing it the following \href{https://www.rdocumentation.org/packages/stats/versions/3.6.2/topics/loess}{stats::loess} parameter values:

\begin{itemize}
\item
  \texttt{degree}: \(2\) in the \hyperref[func.train_UMGY_SmoothedDay_effect.RMSE.cv.args.degree]{degree} argument;
\item
  \texttt{span}: The \hyperref[func.train_UMGY_SmoothedDay_effect.RMSE.cv.degree2.args.span]{span} argument value (of this function) in the same-named argument (\hyperref[func.train_UMGY_SmoothedDay_effect.RMSE.cv.args.span]{span}).
\end{itemize}

\begin{noteblock}
This is an auxiliary function intended to be passed in the \texttt{fn\_tune.test.param\_value} argument to the \hyperref[func.tune.model_param]{tune.model\_param} and \hyperref[func.model.tune.param_range]{model.tune.param\_range} functions described above in section \hyperref[appndx_a.CHF.model_tuning_utils]{Model Tuning Utils} of \emph{this Appendix}.

\end{noteblock}

\phantomsection\label{func.train_UMGY_SmoothedDay_effect.RMSE.cv.degree2.value}
\subparagraph{Value}\label{value-76}

\hfill\break

The \emph{RMSE} value calculated for the \emph{UMGYDE Model} fitted for the given \href{https://www.rdocumentation.org/packages/stats/versions/3.6.2/topics/loess}{loess} parameters (\texttt{degree\ =\ 2} and the \texttt{span} value passed in the \hyperref[func.train_UMGY_SmoothedDay_effect.RMSE.cv.degree2.args.span]{span} argument), and trained with use of \emph{K-Fold Cross-Validation}, where the \(K\) is the length of the \hyperref[appndx_b.edx_CV]{\texttt{edx\_CV} Object} (in \emph{this Project}, we use \(K = 5\)).

\phantomsection\label{func.train_UMGY_SmoothedDay_effect.RMSE.cv.degree2.code}
\subparagraph{Source Code}\label{source-code-83}

\hfill\break

The source code of the \href{https://github.com/AzKurban-edX-DS/Capstone-MovieLens/blob/main/r/src/support-functions/UMGYD-effect.functions.R\#L198}{train\_UMGY\_SmoothedDay\_effect.RMSE.cv.degree2} function is shown below:

\begin{Shaded}
\begin{Highlighting}[]
\NormalTok{train\_UMGY\_SmoothedDay\_effect.RMSE.cv.degree2 }\OtherTok{\textless{}{-}} \ControlFlowTok{function}\NormalTok{(span) \{}
  \FunctionTok{train\_UMGY\_SmoothedDay\_effect.RMSE.cv}\NormalTok{(}\AttributeTok{degree =} \DecValTok{2}\NormalTok{, span)}
\NormalTok{\}}
\end{Highlighting}
\end{Shaded}

\begin{noteblock}
The source code of the \href{https://github.com/AzKurban-edX-DS/Capstone-MovieLens/blob/main/r/src/support-functions/UMGYD-effect.functions.R\#L198}{train\_UMGY\_SmoothedDay\_effect.RMSE.cv.degree2} function is also available in the \href{https://github.com/AzKurban-edX-DS/Capstone-MovieLens/blob/main/r/src/support-functions/UMGYD-effect.functions.R\#L3}{Utility Functions} section of the \href{https://github.com/AzKurban-edX-DS/Capstone-MovieLens/blob/main/r/src/support-functions/UMGYD-effect.functions.R\#L198}{UMGYD-effect.functions.R} script on \emph{GitHub}.

\end{noteblock}

\newpage

\subsubsection{UMGYDE Model: Regularization}\label{umgyde-model-regularization-1}

\hfill\break

\begin{noteblock}
The complete source code of the functions described in this section is available in the \href{https://github.com/AzKurban-edX-DS/Capstone-MovieLens/blob/main/r/src/support-functions/UMGYD-effect.functions.R\#L224}{Regularization} section of the \href{https://github.com/AzKurban-edX-DS/Capstone-MovieLens/blob/main/r/src/support-functions/UMGYD-effect.functions.R\#L224}{UMGYD-effect.functions.R} script on \emph{GitHub}.

\end{noteblock}

\paragraph{\texorpdfstring{\href{https://github.com/AzKurban-edX-DS/Capstone-MovieLens/blob/main/r/src/support-functions/UMGYD-effect.functions.R\#L225}{regularize.train\_UMGYD\_effect} Function}{regularize.train\_UMGYD\_effect Function}}\label{func.regularize.train_UMGYD_effect}

\hfill\break

Trains and \emph{regularizes} the \emph{User+Movie+Genre+Year+SmoothedDay Effect (UMGYDE) Model} for a given value of the parameter \(\lambda\) passed in the \hyperref[func.regularize.train_UMGYD_effect.args.lambda]{lambda} argument and the \emph{best values} of the \href{https://www.rdocumentation.org/packages/stats/versions/3.6.2/topics/loess}{stats::loess} parameters (\texttt{degree} and \texttt{span}, preliminary figured out and stored in the global variable \texttt{lss.UMGYDE.best\_params}), using the training set of the \texttt{Movielens} dataset passed in the \hyperref[func.regularize.train_UMGYD_effect.args.train_set]{train\_set} argument.

\phantomsection\label{func.regularize.train_UMGYD_effect.usage}
\subparagraph{\texorpdfstring{\href{https://github.com/AzKurban-edX-DS/Capstone-MovieLens/blob/main/r/src/capstone-movielens.main.R\#L2168}{Usage}}{Usage}}\label{usage-82}

\hfill\break

\begin{Shaded}
\begin{Highlighting}[]
\NormalTok{  rglr.UMGYD\_effect }\OtherTok{\textless{}{-}}\NormalTok{ edx }\SpecialCharTok{|\textgreater{}} 
    \FunctionTok{regularize.train\_UMGYD\_effect}\NormalTok{(UMGYDE.rglr.best\_lambda)}
\end{Highlighting}
\end{Shaded}

\begin{Shaded}
\begin{Highlighting}[]
\FunctionTok{str}\NormalTok{(rglr.UMGYD\_effect)}
\end{Highlighting}
\end{Shaded}

\begin{verbatim}
## tibble [4,640 x 4] (S3: tbl_df/tbl/data.frame)
##  $ days       : int [1:4640] 0 385 388 389 392 393 394 396 397 399 ...
##  $ de         : num [1:4640] 0.0331 0.1537 0.1174 0.239 -0.0277 ...
##  $ year       : num [1:4640] 1995 1996 1996 1996 1996 ...
##  $ de_smoothed: num [1:4640] 0.025 0.1492 0.1638 0.1565 0.0627 ...
\end{verbatim}

\phantomsection\label{func.regularize.train_UMGYD_effect.args}
\subparagraph{Arguments}\label{arguments-79}

\hfill\break

\phantomsection\label{func.regularize.train_UMGYD_effect.args.train_set}
\begin{itemize}
\tightlist
\item
  \textbf{train\_set:} Training sample of the \texttt{Movielens} dataset;
\end{itemize}

\phantomsection\label{func.regularize.train_UMGYD_effect.args.lambda}
\begin{itemize}
\tightlist
\item
  \textbf{lambda:} Regularization parameter \(\lambda\).
\end{itemize}

\newpage

\phantomsection\label{func.regularize.train_UMGYD_effect.details}
\subparagraph{Details}\label{details-80}

\hfill\break

This function is a wrapper for the function \hyperref[func.train_UMGY_SmoothedDay_effect]{train\_UMGY\_SmoothedDay\_effect} (described above in section \hyperref[umgyde-model-utility-functions]{UMGYDE Model: Utility Functions} of \emph{this Appendix}) that it calls internally, passing it the best values of the \href{https://www.rdocumentation.org/packages/stats/versions/3.6.2/topics/loess}{stats::loess} parameters \texttt{degree} and \texttt{span} in the corresponding arguments (\hyperref[func.train_UMGY_SmoothedDay_effect.args.degree]{degree} and \hyperref[func.train_UMGY_SmoothedDay_effect.args.span]{span}, respectively), stored in the global variable \texttt{lss.UMGYDE.best\_params}:

\begin{Shaded}
\begin{Highlighting}[]
\FunctionTok{print}\NormalTok{(lss.UMGYDE.best\_params)}
\end{Highlighting}
\end{Shaded}

\begin{verbatim}
##     degree       span       RMSE 
## 1.00000000 0.00109375 0.87083000
\end{verbatim}

\begin{noteblock}
The function \emph{being described here} is used in the \emph{regularization process} of the \emph{UMGYDE Model} using the \emph{penalized approach} to regularize predictions for the given \emph{regularization parameter} \(\lambda\) (passed in the \hyperref[func.regularize.train_UMGYD_effect.args.lambda]{lambda} argument) as explained in \href{https://rafalab.dfci.harvard.edu/dsbook-part-2/highdim/regularization.html\#penalized-least-squares}{Section 24.2 \emph{Penalized least squares}} of the \emph{Course Textbook (New Edition)}\autocite{IDS2_24-2}.

\end{noteblock}

\phantomsection\label{func.regularize.train_UMGYD_effect.value}
\subparagraph{Value}\label{value-77}

\hfill\break

\emph{Data frame} object representing the \emph{UMGYDE Model}, regularized by the given \emph{regularization parameter} \(\lambda\) and tuned for the \href{https://www.rdocumentation.org/packages/stats/versions/3.6.2/topics/loess}{loess} \emph{best parameters} \texttt{degree} and \texttt{span}.

\phantomsection\label{func.regularize.train_UMGYD_effect.code}
\subparagraph{Source Code}\label{source-code-84}

\hfill\break

The source code of the \href{https://github.com/AzKurban-edX-DS/Capstone-MovieLens/blob/main/r/src/support-functions/UMGYD-effect.functions.R\#L225}{regularize.train\_UMGYD\_effect} function is shown below:

\begin{Shaded}
\begin{Highlighting}[]
\NormalTok{regularize.train\_UMGYD\_effect }\OtherTok{\textless{}{-}} \ControlFlowTok{function}\NormalTok{(train\_set, lambda) \{}
\NormalTok{  best\_degree }\OtherTok{\textless{}{-}}\NormalTok{ lss.UMGYDE.best\_params[}\StringTok{"degree"}\NormalTok{]}
\NormalTok{  best\_span }\OtherTok{\textless{}{-}}\NormalTok{ lss.UMGYDE.best\_params[}\StringTok{"span"}\NormalTok{]}
  
\NormalTok{  train\_set }\SpecialCharTok{|\textgreater{}}
    \FunctionTok{train\_UMGY\_SmoothedDay\_effect}\NormalTok{(best\_degree, }
\NormalTok{                                  best\_span, }
\NormalTok{                                  lambda)}
\NormalTok{\}}
\end{Highlighting}
\end{Shaded}

\begin{noteblock}
The source code of the \href{https://github.com/AzKurban-edX-DS/Capstone-MovieLens/blob/main/r/src/support-functions/UMGYD-effect.functions.R\#L225}{regularize.train\_UMGYD\_effect} function is also available in the \href{https://github.com/AzKurban-edX-DS/Capstone-MovieLens/blob/main/r/src/support-functions/UMGYD-effect.functions.R\#L224}{Regularization} section of the \href{https://github.com/AzKurban-edX-DS/Capstone-MovieLens/blob/main/r/src/support-functions/UMGYD-effect.functions.R\#L225}{UMGYD-effect.functions.R} script on \emph{GitHub}.

\end{noteblock}

\newpage

\paragraph{\texorpdfstring{\href{https://github.com/AzKurban-edX-DS/Capstone-MovieLens/blob/main/r/src/support-functions/UMGYD-effect.functions.R\#L234}{regularize.train\_UMGYD\_effect.cv} Function}{regularize.train\_UMGYD\_effect.cv Function}}\label{func.regularize.train_UMGYD_effect.cv}

\hfill\break

Trains and \emph{regularizes} the \emph{User+Movie+Genre+Year+SmoothedDay Effect (UMGYDE) Model} for a given value of the parameter \(\lambda\) passed in the \hyperref[func.regularize.train_UMGYD_effect.cv.args.lambda]{lambda} argument and the \emph{best values} of the \href{https://www.rdocumentation.org/packages/stats/versions/3.6.2/topics/loess}{stats::loess} parameters (\texttt{degree} and \texttt{span}, preliminary figured out and stored in the global variable \texttt{lss.UMGYDE.best\_params}), with use of \emph{K-Fold Cross-Validation} as described in \href{https://rafalab.dfci.harvard.edu/dsbook-part-2/ml/resampling-methods.html\#k-fold-cross-validation}{Section 29.6 \emph{Cross validation / K-fold cross validation}} of the \emph{Course Textbook (New Edition)}\autocite{IDS2_29-6_KFCV}.

\begin{noteblock}
The \texttt{K} value for the \emph{K-Fold Cross-Validation} is determined by the length of the \hyperref[appndx_b.edx_CV]{\texttt{edx\_CV} Object} described below in \hyperref[appndx_b]{Appendix B: Models Training Datasets} (in \emph{this Project}, we use \texttt{K\ =\ 5}).

\end{noteblock}

\phantomsection\label{func.regularize.train_UMGYD_effect.cv.usage}
\subparagraph{Usage}\label{usage-83}

\hfill\break

\begin{Shaded}
\begin{Highlighting}[]
  \FunctionTok{regularize.train\_UMGYD\_effect.cv}\NormalTok{(lambda) }\SpecialCharTok{|\textgreater{}}
    \FunctionTok{calc\_UMGY\_SmoothedDay\_effect.RMSE.cv}\NormalTok{()}
\end{Highlighting}
\end{Shaded}

\begin{noteblock}
In the above \href{https://github.com/AzKurban-edX-DS/Capstone-MovieLens/blob/main/r/src/support-functions/UMGYD-effect.functions.R\#L248}{code snippet}, the function \emph{being described here} is called within the \hyperref[func.regularize.test_lambda.UMGYD_effect.cv]{regularize.test\_lambda.UMGYD\_effect.cv} function described below in \emph{this Section}, alongside the \hyperref[func.calc_UMGY_SmoothedDay_effect.RMSE.cv]{calc\_UMGY\_SmoothedDay\_effect.RMSE.cv} described above in section \hyperref[umgyde-model-utility-functions]{UMGYDE Model: Utility Functions}.

\end{noteblock}

\phantomsection\label{func.regularize.train_UMGYD_effect.cv.args}
\subparagraph{Arguments}\label{arguments-80}

\hfill\break

\phantomsection\label{func.regularize.train_UMGYD_effect.cv.args.lambda}
\begin{itemize}
\tightlist
\item
  \textbf{lambda:} Regularization parameter \(\lambda\).
\end{itemize}

\phantomsection\label{func.regularize.train_UMGYD_effect.cv.details}
\subparagraph{Details}\label{details-81}

\hfill\break

This function is a wrapper for the function \hyperref[func.train_UMGY_SmoothedDay_effect.cv]{train\_UMGY\_SmoothedDay\_effect.cv} (described above in section \hyperref[umgyde-model-utility-functions]{UMGYDE Model: Utility Functions} of \emph{this Appendix}) that it calls internally, passing it the best values of the \href{https://www.rdocumentation.org/packages/stats/versions/3.6.2/topics/loess}{stats::loess} parameters \texttt{degree} and \texttt{span} in the corresponding arguments (\hyperref[func.train_UMGY_SmoothedDay_effect.args.degree]{degree} and \hyperref[func.train_UMGY_SmoothedDay_effect.args.span]{span}, respectively), stored in the global variable \texttt{lss.UMGYDE.best\_params}:

\begin{Shaded}
\begin{Highlighting}[]
\FunctionTok{print}\NormalTok{(lss.UMGYDE.best\_params)}
\end{Highlighting}
\end{Shaded}

\begin{verbatim}
##     degree       span       RMSE 
## 1.00000000 0.00109375 0.87083000
\end{verbatim}

The function \emph{being described here} is used in the \emph{regularization process} of the \emph{UMGYDE Model} using the \emph{penalized approach} to regularize predictions for the given \emph{regularization parameter} \(\lambda\) (passed in the \hyperref[func.regularize.train_UMGYD_effect.cv.args.lambda]{lambda} argument) as explained in \href{https://rafalab.dfci.harvard.edu/dsbook-part-2/highdim/regularization.html\#penalized-least-squares}{Section 24.2 \emph{Penalized least squares}} of the \emph{Course Textbook (New Edition)}\autocite{IDS2_24-2}.

\begin{noteblock}
This is an auxiliary function intended to be called internally from the \hyperref[func.regularize.test_lambda.UMGYD_effect.cv]{regularize.test\_lambda.UMGYD\_effect.cv} function (described below in \emph{this Section}) as shown in the \hyperref[func.regularize.train_UMGYD_effect.cv.usage]{Usage} section of \emph{this Description} above.

\end{noteblock}

\phantomsection\label{func.regularize.train_UMGYD_effect.cv.value}
\subparagraph{Value}\label{value-78}

\hfill\break

\emph{Data frame} object representing the \emph{UMGYDE Model}, tuned for the \href{https://www.rdocumentation.org/packages/stats/versions/3.6.2/topics/loess}{loess} \emph{best parameters} \texttt{degree} and \texttt{span}, and trained using \emph{K-Fold Cross-Validation}.

\phantomsection\label{func.regularize.train_UMGYD_effect.cv.code}
\subparagraph{Source Code}\label{source-code-85}

\hfill\break

The source code of the \href{https://github.com/AzKurban-edX-DS/Capstone-MovieLens/blob/main/r/src/support-functions/UMGYD-effect.functions.R\#L234}{regularize.train\_UMGYD\_effect.cv} function is shown below:

\begin{Shaded}
\begin{Highlighting}[]
\NormalTok{regularize.train\_UMGYD\_effect.cv }\OtherTok{\textless{}{-}} \ControlFlowTok{function}\NormalTok{(lambda) \{}
\NormalTok{  best\_degree }\OtherTok{\textless{}{-}}\NormalTok{ lss.UMGYDE.best\_params[}\StringTok{"degree"}\NormalTok{]}
\NormalTok{  best\_span }\OtherTok{\textless{}{-}}\NormalTok{ lss.UMGYDE.best\_params[}\StringTok{"span"}\NormalTok{]}

  \FunctionTok{train\_UMGY\_SmoothedDay\_effect.cv}\NormalTok{(best\_degree, }
\NormalTok{                                   best\_span, }
\NormalTok{                                   lambda)}
\NormalTok{\}}
\end{Highlighting}
\end{Shaded}

\begin{noteblock}
The source code of the \href{https://github.com/AzKurban-edX-DS/Capstone-MovieLens/blob/main/r/src/support-functions/UMGYD-effect.functions.R\#L234}{regularize.train\_UMGYD\_effect.cv} function is also available in the \href{https://github.com/AzKurban-edX-DS/Capstone-MovieLens/blob/main/r/src/support-functions/UMGYD-effect.functions.R\#L224}{Regularization} section of the \href{https://github.com/AzKurban-edX-DS/Capstone-MovieLens/blob/main/r/src/support-functions/UMGYD-effect.functions.R\#L234}{UMGYD-effect.functions.R} script on \emph{GitHub}.

\end{noteblock}

\newpage

\paragraph{\texorpdfstring{\href{https://github.com/AzKurban-edX-DS/Capstone-MovieLens/blob/main/r/src/support-functions/UMGYD-effect.functions.R\#L242}{regularize.test\_lambda.UMGYD\_effect.cv} Function}{regularize.test\_lambda.UMGYD\_effect.cv Function}}\label{func.regularize.test_lambda.UMGYD_effect.cv}

\hfill\break

Validates the \emph{regularization parameter} \(\lambda\) passed in the \hyperref[func.regularize.test_lambda.UMGYD_effect.cv.args.lambda]{lambda} argument for the \emph{User+Movie+Genre+Year+SmoothedDay Effect (UMGYDE) Model} in the \emph{regularization process} using \emph{penalized approach} with use of \emph{K-Fold Cross-Validation}.

\phantomsection\label{func.regularize.test_lambda.UMGYD_effect.cv.usage}
\subparagraph{Usage}\label{usage-84}

\hfill\break

\begin{Shaded}
\begin{Highlighting}[]
\NormalTok{lambdas }\OtherTok{\textless{}{-}} \FunctionTok{seq}\NormalTok{(}\DecValTok{0}\NormalTok{, }\DecValTok{256}\NormalTok{, }\DecValTok{16}\NormalTok{)}
\NormalTok{cv.UMGYDE.preset.result }\OtherTok{\textless{}{-}} 
  \FunctionTok{tune.model\_param}\NormalTok{(lambdas, regularize.test\_lambda.UMGYD\_effect.cv)}
\end{Highlighting}
\end{Shaded}

\begin{Shaded}
\begin{Highlighting}[]
\FunctionTok{str}\NormalTok{(cv.UMGYDE.preset.result)}
\end{Highlighting}
\end{Shaded}

\begin{verbatim}
## List of 2
##  $ tuned.result:'data.frame':    5 obs. of  2 variables:
##   ..$ RMSE           : num [1:5] 0.871 0.871 0.871 0.871 0.871
##   ..$ parameter.value: num [1:5] 0 16 32 48 64
##  $ best_result : Named num [1:2] 16 0.871
##   ..- attr(*, "names")= chr [1:2] "param.best_value" "best_RMSE"
\end{verbatim}

\begin{noteblock}
In the \href{https://github.com/AzKurban-edX-DS/Capstone-MovieLens/blob/main/r/src/capstone-movielens.main.R\#L2070}{code snippet} above the function \emph{being described here} is passed in the \hyperref[func.tune.model_param.args.fn_tune.test.param_value]{fn\_tune.test.param\_value} argument to the \hyperref[func.tune.model_param]{tune.model\_param} function described above in section \hyperref[appndx_a.CHF.model_tuning_utils]{Model Tuning Utils} of \emph{this Appendix}.

Such a use of this function is also noted in the \hyperref[func.regularize.test_lambda.UMGYD_effect.cv.details]{Details} section of \emph{this Description} below.

\end{noteblock}

\phantomsection\label{func.regularize.test_lambda.UMGYD_effect.cv.args}
\subparagraph{Arguments}\label{arguments-81}

\hfill\break

\phantomsection\label{func.regularize.test_lambda.UMGYD_effect.cv.args.lambda}
\begin{itemize}
\tightlist
\item
  \textbf{lambda:} \emph{Regularization parameter} \(\lambda\) used to validate the model in the \emph{regularization process}.
\end{itemize}

\phantomsection\label{func.regularize.test_lambda.UMGYD_effect.cv.details}
\subparagraph{Details}\label{details-82}

\hfill\break

The function is used in the \emph{regularization process} of the \emph{UMGYDE Model} using the \emph{penalized approach} to regularize predictions for the given \emph{regularization parameter} \(\lambda\) (passed in the \hyperref[func.regularize.test_lambda.UMGYD_effect.cv.args.lambda]{lambda} argument) as explained in \href{https://rafalab.dfci.harvard.edu/dsbook-part-2/highdim/regularization.html\#penalized-least-squares}{Section 24.2 \emph{Penalized least squares}} of the \emph{Course Textbook (New Edition)}\autocite{IDS2_24-2}.

Internally it uses the \hyperref[func.regularize.train_UMGYD_effect.cv]{regularize.train\_UMGYD\_effect.cv} (described above in \emph{this Section}) and \hyperref[func.calc_UMGY_SmoothedDay_effect.RMSE.cv]{calc\_UMGY\_SmoothedDay\_effect.RMSE.cv} functions (described in section \hyperref[umgyde-model-utility-functions]{UMGYDE Model: Utility Functions} above) to validate the \emph{UMGYDE Model} for the given \(\lambda\) with use of the \emph{K-Fold Cross-Validation} method, as described in \href{https://rafalab.dfci.harvard.edu/dsbook-part-2/ml/resampling-methods.html\#k-fold-cross-validation}{Section 29.6 \emph{Cross validation / K-fold cross validation}} of the \emph{Course Textbook (New Edition)}\autocite{IDS2_29-6_KFCV}, where the value of \(K\) is determined by the length of the \hyperref[appndx_b.edx_CV]{\texttt{edx\_CV} Object} described below in \hyperref[appndx_b]{Appendix B: Models Training Datasets} (in \emph{this Project}, we use \(K = 5\)).

\begin{noteblock}
This function is an auxiliary function specifically designed to be passed in the \texttt{fn\_tune.test.param\_value} argument to the \hyperref[func.tune.model_param]{tune.model\_param} (as shown above in the \hyperref[func.regularize.test_lambda.UMGYD_effect.cv.usage]{Usage} section of \emph{this Description}) and the \hyperref[func.model.tune.param_range]{model.tune.param\_range} functions described above in section \hyperref[appndx_a.CHF.model_tuning_utils]{Model Tuning Utils} of \emph{this Appendix}.

\end{noteblock}

\phantomsection\label{func.regularize.test_lambda.UMGYD_effect.cv.value}
\subparagraph{Value}\label{value-79}

\hfill\break

The \emph{Root Mean Squared Errors (RMSE)} value computed for the UMGYDE Model applying the \emph{regularization parameter} \(\lambda\) passed in the \hyperref[func.regularize.test_lambda.UMGYD_effect.cv.args.lambda]{lambda} argument.

\phantomsection\label{func.regularize.test_lambda.UMGYD_effect.cv.code}
\subparagraph{Source Code}\label{source-code-86}

\hfill\break

The source code of the \href{https://github.com/AzKurban-edX-DS/Capstone-MovieLens/blob/main/r/src/support-functions/UMGYD-effect.functions.R\#L242}{regularize.test\_lambda.UMGYD\_effect.cv} function is shown below:

\begin{Shaded}
\begin{Highlighting}[]
\NormalTok{regularize.test\_lambda.UMGYD\_effect.cv }\OtherTok{\textless{}{-}} \ControlFlowTok{function}\NormalTok{(lambda)\{}
  \ControlFlowTok{if}\NormalTok{ (}\FunctionTok{is.na}\NormalTok{(lambda)) \{}
    \FunctionTok{stop}\NormalTok{(}\StringTok{"Function: regularize.test\_lambda.UMGYD\_effect.cv}
\StringTok{\textasciigrave{}lambda\textasciigrave{} is \textasciigrave{}NA\textasciigrave{}"}\NormalTok{)}
\NormalTok{  \}}
  
  \FunctionTok{regularize.train\_UMGYD\_effect.cv}\NormalTok{(lambda) }\SpecialCharTok{|\textgreater{}}
    \FunctionTok{calc\_UMGY\_SmoothedDay\_effect.RMSE.cv}\NormalTok{()}
\NormalTok{\}}
\end{Highlighting}
\end{Shaded}

\begin{noteblock}
The source code of the \href{https://github.com/AzKurban-edX-DS/Capstone-MovieLens/blob/main/r/src/support-functions/UMGYD-effect.functions.R\#L242}{regularize.test\_lambda.UMGYD\_effect.cv} function is also available in the \href{https://github.com/AzKurban-edX-DS/Capstone-MovieLens/blob/main/r/src/support-functions/UMGYD-effect.functions.R\#L224}{Regularization} section of the \href{https://github.com/AzKurban-edX-DS/Capstone-MovieLens/blob/main/r/src/support-functions/UMGYD-effect.functions.R\#L242}{UMGYD-effect.functions.R} script on \emph{GitHub}.

\end{noteblock}

\newpage

\subsubsection{UMGYDE Model: Matrix Factorization}\label{umgyde-model-matrix-factorization}

\hfill\break

\begin{noteblock}
The complete source code of the functions described in this section is available in the \href{https://github.com/AzKurban-edX-DS/Capstone-MovieLens/blob/main/r/src/support-functions/MF.functions.R}{MF.functions.R} script on \emph{GitHub}.

\end{noteblock}

\paragraph{\texorpdfstring{\href{https://github.com/AzKurban-edX-DS/Capstone-MovieLens/blob/main/r/src/support-functions/MF.functions.R\#L3}{mf.residual.dataframe} Function}{mf.residual.dataframe Function}}\label{func.mf.residual.dataframe}

\hfill\break

Computes residuals, defined as the difference between the observed and predicted values of the fully trained \emph{UMGYDE Model} as described in section \hyperref[usermoviegenreyearsmoothedday-effect-umgyde-model]{User+Movie+Genre+Year+SmoothedDay Effect (UMGYDE) Model}.

\phantomsection\label{func.mf.residual.dataframe.usage}
\subparagraph{\texorpdfstring{\href{https://github.com/AzKurban-edX-DS/Capstone-MovieLens/blob/main/r/src/capstone-movielens.main.R\#L2314}{Usage}}{Usage}}\label{usage-85}

\hfill\break

\begin{Shaded}
\begin{Highlighting}[]
\NormalTok{  mf.edx.residual }\OtherTok{\textless{}{-}} \FunctionTok{mf.residual.dataframe}\NormalTok{(edx)}
\end{Highlighting}
\end{Shaded}

\phantomsection\label{func.mf.residual.dataframe.args}
\subparagraph{Arguments}\label{arguments-82}

\hfill\break

\phantomsection\label{func.mf.residual.dataframe.args.train_set}
\begin{itemize}
\tightlist
\item
  \textbf{train\_set:} Training dataset;
\end{itemize}

\phantomsection\label{func.mf.residual.dataframe.details}
\subparagraph{Details}\label{details-83}

\hfill\break

This function calculates the \emph{residuals} on the \emph{training dataset} passed as the \hyperref[func.mf.residual.dataframe.args.train_set]{train\_set} argument, taking into account all the effects of the previously trained models, and creates a new \emph{data frame} that includes the results as a \emph{calculated column} \texttt{rsdsl}, along with the \texttt{movieId} and \texttt{userId} columns selected from the \emph{training dataset}.

\phantomsection\label{func.mf.residual.dataframe.value}
\subparagraph{Value}\label{value-80}

\hfill\break

\emph{Data frame} containing the following three variables:

\begin{itemize}
\tightlist
\item
  \textbf{\emph{rsdl}}: (\emph{Numeric}) Calculated column containing results of the function computation;
\item
  \textbf{\emph{userId}}: (\emph{Integer}) User Identifier;
\item
  \textbf{\emph{movieId}} (\emph{Integer}) Movie Identifier.
\end{itemize}

\newpage

\phantomsection\label{func.mf.residual.dataframe.code}
\subparagraph{Source Code}\label{source-code-87}

\hfill\break

The source code of the \href{https://github.com/AzKurban-edX-DS/Capstone-MovieLens/blob/main/r/src/support-functions/MF.functions.R\#L3}{mf.residual.dataframe} function is shown below:

\begin{Shaded}
\begin{Highlighting}[]
\NormalTok{mf.residual.dataframe }\OtherTok{\textless{}{-}} \ControlFlowTok{function}\NormalTok{(train\_set)\{}
\NormalTok{  train\_set }\SpecialCharTok{|\textgreater{}} 
    \FunctionTok{left\_join}\NormalTok{(edx.user\_effect, }\AttributeTok{by =} \StringTok{"userId"}\NormalTok{) }\SpecialCharTok{|\textgreater{}}
    \FunctionTok{left\_join}\NormalTok{(rglr.UM\_effect, }\AttributeTok{by =} \StringTok{"movieId"}\NormalTok{) }\SpecialCharTok{|\textgreater{}}
    \FunctionTok{left\_join}\NormalTok{(rglr.UMG\_effect, }\AttributeTok{by =} \StringTok{"movieId"}\NormalTok{) }\SpecialCharTok{|\textgreater{}}
    \FunctionTok{left\_join}\NormalTok{(date\_days\_map, }\AttributeTok{by =} \StringTok{"timestamp"}\NormalTok{) }\SpecialCharTok{|\textgreater{}}
    \FunctionTok{left\_join}\NormalTok{(rglr.UMGY\_effect, }\AttributeTok{by=}\StringTok{\textquotesingle{}year\textquotesingle{}}\NormalTok{) }\SpecialCharTok{|\textgreater{}}
    \FunctionTok{left\_join}\NormalTok{(rglr.UMGYD\_effect, }\AttributeTok{by=}\StringTok{\textquotesingle{}days\textquotesingle{}}\NormalTok{) }\SpecialCharTok{|\textgreater{}}
    \FunctionTok{mutate}\NormalTok{(}\AttributeTok{rsdl =}\NormalTok{ rating }\SpecialCharTok{{-}}\NormalTok{ (mu }\SpecialCharTok{+}\NormalTok{ a }\SpecialCharTok{+}\NormalTok{ b }\SpecialCharTok{+}\NormalTok{ g }\SpecialCharTok{+}\NormalTok{ ye }\SpecialCharTok{+}\NormalTok{ de\_smoothed)) }\SpecialCharTok{|\textgreater{}}
    \FunctionTok{select}\NormalTok{(userId, movieId, rsdl)}
\NormalTok{\}}
\end{Highlighting}
\end{Shaded}

\begin{noteblock}
The source code of the \href{https://github.com/AzKurban-edX-DS/Capstone-MovieLens/blob/main/r/src/support-functions/MF.functions.R\#L3}{mf.residual.dataframe} function is also available in the \href{https://github.com/AzKurban-edX-DS/Capstone-MovieLens/blob/main/r/src/support-functions/MF.functions.R\#L3}{MF.functions.R} script on \emph{GitHub}.

\end{noteblock}

\newpage

\section{Appendix B: Models Training Datasets}\label{appndx_b}

\hfill\break

\subsection{\texorpdfstring{\texttt{edx.mx} Matrix Object}{edx.mx Matrix Object}}\label{appndx_b.edx.mx}

\hfill\break

We use the array representation described in \href{https://rafalab.dfci.harvard.edu/dsbook-part-2/linear-models/treatment-effect-models.html\#sec-anova}{Section 17.6 \emph{Analysis of variance (ANOVA)}} of the \emph{Course Textbook (New Edition)}, for the training data\autocite{IDS2_17-6}. For this purpose, we will convert the \texttt{edx} dataset into a matrix using the \href{https://www.rdocumentation.org/packages/tidyr/versions/1.3.0/topics/pivot_wider}{tidyr::pivot\_wider} function:

\begin{Shaded}
\begin{Highlighting}[]
  \FunctionTok{put\_log}\NormalTok{(}\StringTok{"Function: \textasciigrave{}make\_source\_datasets\textasciigrave{}: Creating Rating Matrix from \textasciigrave{}edx\textasciigrave{} dataset..."}\NormalTok{)}
\NormalTok{  edx.mx }\OtherTok{\textless{}{-}}\NormalTok{ edx }\SpecialCharTok{|\textgreater{}} 
    \FunctionTok{mutate}\NormalTok{(}\AttributeTok{userId =} \FunctionTok{factor}\NormalTok{(userId),}
           \AttributeTok{movieId =} \FunctionTok{factor}\NormalTok{(movieId)) }\SpecialCharTok{|\textgreater{}}
    \FunctionTok{select}\NormalTok{(movieId, userId, rating) }\SpecialCharTok{|\textgreater{}}
    \FunctionTok{pivot\_wider}\NormalTok{(}\AttributeTok{names\_from =}\NormalTok{ movieId, }\AttributeTok{values\_from =}\NormalTok{ rating) }\SpecialCharTok{|\textgreater{}}
    \FunctionTok{column\_to\_rownames}\NormalTok{(}\StringTok{"userId"}\NormalTok{) }\SpecialCharTok{|\textgreater{}}
    \FunctionTok{as.matrix}\NormalTok{()}
  
  \FunctionTok{put\_log}\NormalTok{(}\StringTok{"Function: \textasciigrave{}make\_source\_datasets\textasciigrave{}:}
\StringTok{Matrix created: \textasciigrave{}edx.mx\textasciigrave{} of the following dimentions:"}\NormalTok{)}
  \FunctionTok{put}\NormalTok{(}\FunctionTok{dim}\NormalTok{(edx.mx))}
\end{Highlighting}
\end{Shaded}

To create the \texttt{edx.mx} object, the above \href{https://github.com/AzKurban-edX-DS/Capstone-MovieLens/blob/main/r/src/support-functions/data.helper.functions.R\#L92}{code snippet} is used in the \hyperref[func.make_source_datasets]{make\_source\_datasets} function described in section \hyperref[initializing-input-datasets]{Initializing Input Datasets} of \hyperref[appndx_a]{Appendix A}.

\begin{noteblock}
The \emph{code snippet} above uses \emph{custom logging function} \hyperref[func.put_log]{put\_log} described in section \hyperref[appndx_a.CHF.logging-funcs]{Logging Functions} of \hyperref[appndx_a]{Appendix A}.

\end{noteblock}

\begin{Shaded}
\begin{Highlighting}[]
  \FunctionTok{str}\NormalTok{(edx.mx)}
\end{Highlighting}
\end{Shaded}

\begin{verbatim}
##  num [1:69878, 1:10677] 5 NA NA NA NA NA NA NA NA NA ...
##  - attr(*, "dimnames")=List of 2
##   ..$ : chr [1:69878] "1" "2" "3" "4" ...
##   ..$ : chr [1:10677] "122" "185" "292" "316" ...
\end{verbatim}

\newpage

\subsection{\texorpdfstring{\texttt{edx.sgr} Object}{edx.sgr Object}}\label{appndx_b.edx.sgr}

\hfill\break

To account for the Movie Genre Effect more accurately, we need a dataset with split rows for movies belonging to multiple genres:

\begin{Shaded}
\begin{Highlighting}[]
  \FunctionTok{put\_log}\NormalTok{(}\StringTok{"Function: \textasciigrave{}make\_source\_datasets\textasciigrave{}: }
\StringTok{To account for the Movie Genre Effect, we need a dataset with split rows }
\StringTok{for movies belonging to multiple genres."}\NormalTok{)}
\NormalTok{  edx.sgr }\OtherTok{\textless{}{-}} \FunctionTok{splitGenreRows}\NormalTok{(edx)}
\end{Highlighting}
\end{Shaded}

To create the \texttt{edx.sgr} object, the above \href{https://github.com/AzKurban-edX-DS/Capstone-MovieLens/blob/main/r/src/support-functions/data.helper.functions.R\#L125}{code snippet} is used in the \hyperref[func.make_source_datasets]{make\_source\_datasets} function described in section \hyperref[initializing-input-datasets]{Initializing Input Datasets} of \hyperref[appndx_a]{Appendix A}.

\begin{noteblock}
The \emph{code snippet} above uses the \emph{user-defined helper function} \hyperref[func.splitGenreRows]{splitGenreRows} (described in section \hyperref[data-processing-functions]{Data Processing Functions}) to split the rows of the original dataset (\texttt{edx} in this case), as well as the \emph{custom logging function} \hyperref[func.put_log]{put\_log} described in section \hyperref[appndx_a.CHF.logging-funcs]{Logging Functions} of \hyperref[appndx_a]{Appendix A}.

\end{noteblock}

\begin{Shaded}
\begin{Highlighting}[]
\FunctionTok{str}\NormalTok{(edx.sgr)}
\end{Highlighting}
\end{Shaded}

\begin{verbatim}
## tibble [23,371,423 x 6] (S3: tbl_df/tbl/data.frame)
##  $ userId   : int [1:23371423] 1 1 1 1 1 1 1 1 1 1 ...
##  $ movieId  : int [1:23371423] 122 122 185 185 185 292 292 292 292 316 ...
##  $ rating   : num [1:23371423] 5 5 5 5 5 5 5 5 5 5 ...
##  $ timestamp: int [1:23371423] 838985046 838985046 838983525 838983525 838983525 838983421 838983421 838983421 838983421 838983392 ...
##  $ title    : chr [1:23371423] "Boomerang (1992)" "Boomerang (1992)" "Net, The (1995)" "Net, The (1995)" ...
##  $ genres   : chr [1:23371423] "Comedy" "Romance" "Action" "Crime" ...
\end{verbatim}

\begin{Shaded}
\begin{Highlighting}[]
\FunctionTok{summary}\NormalTok{(edx.sgr)}
\end{Highlighting}
\end{Shaded}

\begin{verbatim}
##      userId         movieId          rating        timestamp            title              genres         
##  Min.   :    1   Min.   :    1   Min.   :0.500   Min.   :7.897e+08   Length:23371423    Length:23371423   
##  1st Qu.:18140   1st Qu.:  616   1st Qu.:3.000   1st Qu.:9.472e+08   Class :character   Class :character  
##  Median :35784   Median : 1748   Median :4.000   Median :1.042e+09   Mode  :character   Mode  :character  
##  Mean   :35886   Mean   : 4277   Mean   :3.527   Mean   :1.035e+09                                        
##  3rd Qu.:53638   3rd Qu.: 3635   3rd Qu.:4.000   3rd Qu.:1.131e+09                                        
##  Max.   :71567   Max.   :65133   Max.   :5.000   Max.   :1.231e+09
\end{verbatim}

\newpage

\subsection{\texorpdfstring{\texttt{movie\_map} Object}{movie\_map Object}}\label{appndx_b.movie_map}

\hfill\break

To be able to map movie IDs to titles, we create the following lookup table:

\begin{Shaded}
\begin{Highlighting}[]
\NormalTok{movie\_map }\OtherTok{\textless{}{-}}\NormalTok{ edx }\SpecialCharTok{|\textgreater{}} \FunctionTok{select}\NormalTok{(movieId, title, genres) }\SpecialCharTok{|\textgreater{}} 
    \FunctionTok{distinct}\NormalTok{(movieId, }\AttributeTok{.keep\_all =} \ConstantTok{TRUE}\NormalTok{)}
  
  \FunctionTok{put\_log}\NormalTok{(}\StringTok{"Function: \textasciigrave{}make\_source\_datasets\textasciigrave{}: Dataset created: movie\_map"}\NormalTok{)}
\end{Highlighting}
\end{Shaded}

To create the \texttt{movie\_map} object, the above \href{https://github.com/AzKurban-edX-DS/Capstone-MovieLens/blob/main/r/src/support-functions/data.helper.functions.R\#L106}{code snippet} is used in the \hyperref[func.make_source_datasets]{make\_source\_datasets} function described in section \hyperref[initializing-input-datasets]{Initializing Input Datasets} of \hyperref[appndx_a]{Appendix A}.

\begin{noteblock}
The \emph{code snippet} above uses \emph{custom logging function} \hyperref[func.put_log]{put\_log} described in section \hyperref[appndx_a.CHF.logging-funcs]{Logging Functions} of \hyperref[appndx_a]{Appendix A}.

\end{noteblock}

\begin{Shaded}
\begin{Highlighting}[]
\FunctionTok{str}\NormalTok{(movie\_map)}
\end{Highlighting}
\end{Shaded}

\begin{verbatim}
## 'data.frame':    10677 obs. of  3 variables:
##  $ movieId: int  122 185 292 316 329 355 356 362 364 370 ...
##  $ title  : chr  "Boomerang (1992)" "Net, The (1995)" "Outbreak (1995)" "Stargate (1994)" ...
##  $ genres : chr  "Comedy|Romance" "Action|Crime|Thriller" "Action|Drama|Sci-Fi|Thriller" "Action|Adventure|Sci-Fi" ...
\end{verbatim}

\begin{Shaded}
\begin{Highlighting}[]
\FunctionTok{summary}\NormalTok{(movie\_map)}
\end{Highlighting}
\end{Shaded}

\begin{verbatim}
##     movieId         title              genres         
##  Min.   :    1   Length:10677       Length:10677      
##  1st Qu.: 2754   Class :character   Class :character  
##  Median : 5434   Mode  :character   Mode  :character  
##  Mean   :13105                                        
##  3rd Qu.: 8710                                        
##  Max.   :65133
\end{verbatim}

Note that titles cannot be considered unique, so we cannot use them as IDs\autocite{IDS2_24-1}.

\newpage

\subsection{\texorpdfstring{\texttt{date\_days\_map} Object}{date\_days\_map Object}}\label{appndx_b.date_days_map}

\hfill\break

We have a \texttt{timestamp} field in the \texttt{edx} dataset.
To be able to map the date, year, and number of days since the earliest record in the \texttt{edx} dataset with the corresponding value in this field, we create the following lookup table:

\begin{Shaded}
\begin{Highlighting}[]
  \FunctionTok{put\_log}\NormalTok{(}\StringTok{"Function: \textasciigrave{}make\_source\_datasets\textasciigrave{}: Creating Date{-}Days Map dataset..."}\NormalTok{)}
\NormalTok{  date\_days\_map }\OtherTok{\textless{}{-}}\NormalTok{ edx }\SpecialCharTok{|\textgreater{}}
    \FunctionTok{mutate}\NormalTok{(}\AttributeTok{date\_time =} \FunctionTok{as\_datetime}\NormalTok{(timestamp)) }\SpecialCharTok{|\textgreater{}}
    \FunctionTok{mutate}\NormalTok{(}\AttributeTok{date =} \FunctionTok{as\_date}\NormalTok{(date\_time)) }\SpecialCharTok{|\textgreater{}}
    \FunctionTok{mutate}\NormalTok{(}\AttributeTok{year =} \FunctionTok{year}\NormalTok{(date\_time)) }\SpecialCharTok{|\textgreater{}}
    \FunctionTok{mutate}\NormalTok{(}\AttributeTok{days =} \FunctionTok{as.integer}\NormalTok{(date }\SpecialCharTok{{-}} \FunctionTok{min}\NormalTok{(date))) }\SpecialCharTok{|\textgreater{}}
    \FunctionTok{select}\NormalTok{(timestamp, date\_time, date, year, days) }\SpecialCharTok{|\textgreater{}}
    \FunctionTok{distinct}\NormalTok{(timestamp, }\AttributeTok{.keep\_all =} \ConstantTok{TRUE}\NormalTok{)}
  
  \FunctionTok{put\_log}\NormalTok{(}\StringTok{"Function: \textasciigrave{}make\_source\_datasets\textasciigrave{}: Dataset created: date\_days\_map"}\NormalTok{)}
\end{Highlighting}
\end{Shaded}

To create the \texttt{date\_days\_map} object, the above \href{https://github.com/AzKurban-edX-DS/Capstone-MovieLens/blob/main/r/src/support-functions/data.helper.functions.R\#L112}{code snippet} is used in the \hyperref[func.make_source_datasets]{make\_source\_datasets} function described in section \hyperref[initializing-input-datasets]{Initializing Input Datasets} of \hyperref[appndx_a]{Appendix A}.

\begin{noteblock}
The \emph{code snippet} above uses \emph{custom logging function} \hyperref[func.put_log]{put\_log} described in section \hyperref[appndx_a.CHF.logging-funcs]{Logging Functions} of \hyperref[appndx_a]{Appendix A}.

\end{noteblock}

\begin{Shaded}
\begin{Highlighting}[]
  \FunctionTok{str}\NormalTok{(date\_days\_map)}
\end{Highlighting}
\end{Shaded}

\begin{verbatim}
## 'data.frame':    6519590 obs. of  5 variables:
##  $ timestamp: int  838985046 838983525 838983421 838983392 838984474 838983653 838984885 838983707 838984596 838983834 ...
##  $ date_time: POSIXct, format: "1996-08-02 11:24:06" "1996-08-02 10:58:45" "1996-08-02 10:57:01" "1996-08-02 10:56:32" ...
##  $ date     : Date, format: "1996-08-02" "1996-08-02" "1996-08-02" "1996-08-02" ...
##  $ year     : num  1996 1996 1996 1996 1996 ...
##  $ days     : int  571 571 571 571 571 571 571 571 571 571 ...
\end{verbatim}

\begin{Shaded}
\begin{Highlighting}[]
  \FunctionTok{summary}\NormalTok{(date\_days\_map)}
\end{Highlighting}
\end{Shaded}

\begin{verbatim}
##    timestamp           date_time                           date                 year           days     
##  Min.   :7.897e+08   Min.   :1995-01-09 11:46:49.00   Min.   :1995-01-09   Min.   :1995   Min.   :   0  
##  1st Qu.:9.783e+08   1st Qu.:2001-01-01 05:05:01.75   1st Qu.:2001-01-01   1st Qu.:2001   1st Qu.:2184  
##  Median :1.091e+09   Median :2004-08-03 01:08:18.50   Median :2004-08-03   Median :2004   Median :3494  
##  Mean   :1.066e+09   Mean   :2003-10-10 23:15:02.07   Mean   :2003-10-10   Mean   :2003   Mean   :3196  
##  3rd Qu.:1.152e+09   3rd Qu.:2006-07-04 20:41:57.50   3rd Qu.:2006-07-04   3rd Qu.:2006   3rd Qu.:4194  
##  Max.   :1.231e+09   Max.   :2009-01-05 05:02:16.00   Max.   :2009-01-05   Max.   :2009   Max.   :5110
\end{verbatim}

\newpage

\subsection{\texorpdfstring{\texttt{edx\_CV} Object}{edx\_CV Object}}\label{appndx_b.edx_CV}

\hfill\break

\texttt{edx\_CV} object is a list of sample objects we need to perform the \emph{K-Fold Cross-Validation}
as explained in \href{https://rafalab.dfci.harvard.edu/dsbook-part-2/ml/resampling-methods.html\#k-fold-cross-validation}{Section 29.6 \emph{Cross validation / K-fold cross validation}} of the \emph{Course Textbook (New Edition)}\autocite{IDS2_29-6_KFCV}:

\begin{Shaded}
\begin{Highlighting}[]
\NormalTok{  start }\OtherTok{\textless{}{-}} \FunctionTok{put\_start\_date}\NormalTok{()}
\NormalTok{  edx\_CV }\OtherTok{\textless{}{-}} \FunctionTok{lapply}\NormalTok{(kfold\_index,  }\ControlFlowTok{function}\NormalTok{(fold\_i)\{}
    
    \FunctionTok{put\_log1}\NormalTok{(}\StringTok{"Method \textasciigrave{}make\_source\_datasets\textasciigrave{}: }
\StringTok{Creating K{-}Fold Cross{-}Validation Datasets, Fold \%1"}\NormalTok{, fold\_i)}
    
    \CommentTok{\#\textgreater{} We split the initial datasets into training sets, which we will use to build }
    \CommentTok{\#\textgreater{} and train our models, and validation sets in which we will compute the accuracy }
    \CommentTok{\#\textgreater{} of our predictions, the way described in the \textasciigrave{}Section 23.1.1 Movielens data\textasciigrave{}}
    \CommentTok{\#\textgreater{} (https://rafalab.dfci.harvard.edu/dsbook{-}part{-}2/highdim/regularization.html\#movielens{-}data) }
    \CommentTok{\#\textgreater{} of the Course Textbook.}
    
\NormalTok{    split\_sets }\OtherTok{\textless{}{-}}\NormalTok{ edx }\SpecialCharTok{|\textgreater{}}
      \FunctionTok{sample\_train\_validation\_sets}\NormalTok{(fold\_i}\SpecialCharTok{*}\DecValTok{1000}\NormalTok{)}
    
\NormalTok{    train\_set }\OtherTok{\textless{}{-}}\NormalTok{ split\_sets}\SpecialCharTok{$}\NormalTok{train\_set}
\NormalTok{    validation\_set }\OtherTok{\textless{}{-}}\NormalTok{ split\_sets}\SpecialCharTok{$}\NormalTok{validation\_set}
    
    \FunctionTok{put\_log}\NormalTok{(}\StringTok{"Function: \textasciigrave{}make\_source\_datasets\textasciigrave{}: }
\StringTok{Sampling 20\% from the split{-}row version of the \textasciigrave{}edx\textasciigrave{} dataset..."}\NormalTok{)}
\NormalTok{    split\_sets.gs }\OtherTok{\textless{}{-}}\NormalTok{ edx.sgr }\SpecialCharTok{|\textgreater{}}
      \FunctionTok{sample\_train\_validation\_sets}\NormalTok{(fold\_i}\SpecialCharTok{*}\DecValTok{2000}\NormalTok{)}

\NormalTok{    train.sgr }\OtherTok{\textless{}{-}}\NormalTok{ split\_sets.gs}\SpecialCharTok{$}\NormalTok{train\_set}
\NormalTok{    validation.sgr }\OtherTok{\textless{}{-}}\NormalTok{ split\_sets.gs}\SpecialCharTok{$}\NormalTok{validation\_set}
    
    \CommentTok{\# put\_log("Function: \textasciigrave{}make\_source\_datasets\textasciigrave{}: Dataset created: validation.sgr")}
    \CommentTok{\# put(summary(validation.sgr))}
    
    \CommentTok{\#\textgreater{} We will use the array representation described in \textasciigrave{}Section 17.5 of the Textbook\textasciigrave{}}
    \CommentTok{\#\textgreater{} (https://rafalab.dfci.harvard.edu/dsbook{-}part{-}2/linear{-}models/treatment{-}effect{-}models.html\#sec{-}anova), }
    \CommentTok{\#\textgreater{} for the training data. }
    \CommentTok{\#\textgreater{} To create this matrix, we use \textasciigrave{}tidyr::pivot\_wider\textasciigrave{} function:}
    
    \FunctionTok{put\_log}\NormalTok{(}\StringTok{"Function: \textasciigrave{}make\_source\_datasets\textasciigrave{}: Creating Rating Matrix from Train Set..."}\NormalTok{)}
\NormalTok{    train\_mx }\OtherTok{\textless{}{-}}\NormalTok{ train\_set }\SpecialCharTok{|\textgreater{}} 
      \FunctionTok{mutate}\NormalTok{(}\AttributeTok{userId =} \FunctionTok{factor}\NormalTok{(userId),}
             \AttributeTok{movieId =} \FunctionTok{factor}\NormalTok{(movieId)) }\SpecialCharTok{|\textgreater{}}
      \FunctionTok{select}\NormalTok{(movieId, userId, rating) }\SpecialCharTok{|\textgreater{}}
      \FunctionTok{pivot\_wider}\NormalTok{(}\AttributeTok{names\_from =}\NormalTok{ movieId, }\AttributeTok{values\_from =}\NormalTok{ rating) }\SpecialCharTok{|\textgreater{}}
      \FunctionTok{column\_to\_rownames}\NormalTok{(}\StringTok{"userId"}\NormalTok{) }\SpecialCharTok{|\textgreater{}}
      \FunctionTok{as.matrix}\NormalTok{()}
    
    \FunctionTok{put\_log}\NormalTok{(}\StringTok{"Function: \textasciigrave{}make\_source\_datasets\textasciigrave{}:}
\StringTok{Matrix created: \textasciigrave{}train\_mx\textasciigrave{} of the following dimentions:"}\NormalTok{)}
    \FunctionTok{put}\NormalTok{(}\FunctionTok{dim}\NormalTok{(train\_mx))}

    \FunctionTok{list}\NormalTok{(}\AttributeTok{train\_set =}\NormalTok{ train\_set,}
         \AttributeTok{train\_mx =}\NormalTok{ train\_mx,}
         \AttributeTok{train.sgr =}\NormalTok{ train.sgr,}
         \AttributeTok{validation\_set =}\NormalTok{ validation\_set)}
\NormalTok{  \})}
  \FunctionTok{put\_end\_date}\NormalTok{(start)}
  \FunctionTok{put\_log}\NormalTok{(}\StringTok{"Function: \textasciigrave{}make\_source\_datasets\textasciigrave{}: }
\StringTok{Set of K{-}Fold Cross{-}Validation datasets created: edx\_CV"}\NormalTok{)}
\end{Highlighting}
\end{Shaded}

To create the \texttt{edx\_CV} object, the above \href{https://github.com/AzKurban-edX-DS/Capstone-MovieLens/blob/main/r/src/support-functions/data.helper.functions.R\#L141}{code snippet} is used in the \hyperref[func.make_source_datasets]{make\_source\_datasets} function described in section \hyperref[initializing-input-datasets]{Initializing Input Datasets} of \hyperref[appndx_a]{Appendix A}.

The \(K\) value for the \emph{K-Fold Cross-Validation} is determined by the length of the list composing the \texttt{edx\_CV} object. In \emph{this Project}, we choose \(K = 5\).

\begin{noteblock}

The \emph{code snippet} above uses the \emph{user-defined helper function} \hyperref[func.sample_train_validation_sets]{sample\_train\_validation\_sets} (described in section \hyperref[data-processing-functions]{Data Processing Functions}) to split the original dataset (\texttt{edx} in this case) into \emph{Train} and \emph{Validation} datasets, as well as the following \emph{custom logging function} described in section \hyperref[appndx_a.CHF.logging-funcs]{Logging Functions} of \hyperref[appndx_a]{Appendix A}:

\begin{itemize}
\tightlist
\item
  \hyperref[func.put_start_date]{put\_start\_date};
\item
  \hyperref[func.put_end_date]{put\_end\_date};
\item
  \hyperref[func.put_log]{put\_log};
\item
  \hyperref[func.put_log1]{put\_log1}.
\end{itemize}

\end{noteblock}

\begin{Shaded}
\begin{Highlighting}[]
\FunctionTok{str}\NormalTok{(edx\_CV)}
\end{Highlighting}
\end{Shaded}

\begin{verbatim}
## List of 5
##  $ :List of 4
##   ..$ train_set     :'data.frame':   7172311 obs. of  6 variables:
##   .. ..$ userId   : int [1:7172311] 1 1 1 1 1 1 1 1 1 1 ...
##   .. ..$ movieId  : int [1:7172311] 122 185 292 329 356 362 364 370 420 466 ...
##   .. ..$ rating   : num [1:7172311] 5 5 5 5 5 5 5 5 5 5 ...
##   .. ..$ timestamp: int [1:7172311] 838985046 838983525 838983421 838983392 838983653 838984885 838983707 838984596 838983834 838984679 ...
##   .. ..$ title    : chr [1:7172311] "Boomerang (1992)" "Net, The (1995)" "Outbreak (1995)" "Star Trek: Generations (1994)" ...
##   .. ..$ genres   : chr [1:7172311] "Comedy|Romance" "Action|Crime|Thriller" "Action|Drama|Sci-Fi|Thriller" "Action|Adventure|Drama|Sci-Fi" ...
##   ..$ train_mx      : num [1:69878, 1:10677] 5 NA NA NA NA NA NA NA NA NA ...
##   .. ..- attr(*, "dimnames")=List of 2
##   .. .. ..$ : chr [1:69878] "1" "2" "3" "4" ...
##   .. .. ..$ : chr [1:10677] "122" "185" "292" "329" ...
##   ..$ train.sgr     : tibble [18,669,190 x 6] (S3: tbl_df/tbl/data.frame)
##   .. ..$ userId   : int [1:18669190] 1 1 1 1 1 1 1 1 1 1 ...
##   .. ..$ movieId  : int [1:18669190] 122 122 185 185 292 292 292 292 316 316 ...
##   .. ..$ rating   : num [1:18669190] 5 5 5 5 5 5 5 5 5 5 ...
##   .. ..$ timestamp: int [1:18669190] 838985046 838985046 838983525 838983525 838983421 838983421 838983421 838983421 838983392 838983392 ...
##   .. ..$ title    : chr [1:18669190] "Boomerang (1992)" "Boomerang (1992)" "Net, The (1995)" "Net, The (1995)" ...
##   .. ..$ genres   : chr [1:18669190] "Comedy" "Romance" "Action" "Crime" ...
##   ..$ validation_set:'data.frame':   1827744 obs. of  6 variables:
##   .. ..$ userId   : int [1:1827744] 1 1 1 1 2 2 2 2 3 3 ...
##   .. ..$ movieId  : int [1:1827744] 316 355 377 588 260 376 648 1049 110 1252 ...
##   .. ..$ rating   : num [1:1827744] 5 5 5 5 5 3 2 3 4.5 4 ...
##   .. ..$ timestamp: int [1:1827744] 838983392 838984474 838983834 838983339 868244562 868245920 868244699 868245920 1136075500 1133571071 ...
##   .. ..$ title    : chr [1:1827744] "Stargate (1994)" "Flintstones, The (1994)" "Speed (1994)" "Aladdin (1992)" ...
##   .. ..$ genres   : chr [1:1827744] "Action|Adventure|Sci-Fi" "Children|Comedy|Fantasy" "Action|Romance|Thriller" "Adventure|Animation|Children|Comedy|Musical" ...
##  $ :List of 4
##   ..$ train_set     :'data.frame':   7172306 obs. of  6 variables:
##   .. ..$ userId   : int [1:7172306] 1 1 1 1 1 1 1 1 1 1 ...
##   .. ..$ movieId  : int [1:7172306] 122 185 292 316 329 355 356 364 370 377 ...
##   .. ..$ rating   : num [1:7172306] 5 5 5 5 5 5 5 5 5 5 ...
##   .. ..$ timestamp: int [1:7172306] 838985046 838983525 838983421 838983392 838983392 838984474 838983653 838983707 838984596 838983834 ...
##   .. ..$ title    : chr [1:7172306] "Boomerang (1992)" "Net, The (1995)" "Outbreak (1995)" "Stargate (1994)" ...
##   .. ..$ genres   : chr [1:7172306] "Comedy|Romance" "Action|Crime|Thriller" "Action|Drama|Sci-Fi|Thriller" "Action|Adventure|Sci-Fi" ...
##   ..$ train_mx      : num [1:69878, 1:10677] 5 NA NA NA NA NA NA NA NA NA ...
##   .. ..- attr(*, "dimnames")=List of 2
##   .. .. ..$ : chr [1:69878] "1" "2" "3" "4" ...
##   .. .. ..$ : chr [1:10677] "122" "185" "292" "316" ...
##   ..$ train.sgr     : tibble [18,669,201 x 6] (S3: tbl_df/tbl/data.frame)
##   .. ..$ userId   : int [1:18669201] 1 1 1 1 1 1 1 1 1 1 ...
##   .. ..$ movieId  : int [1:18669201] 122 122 185 185 185 292 292 316 316 329 ...
##   .. ..$ rating   : num [1:18669201] 5 5 5 5 5 5 5 5 5 5 ...
##   .. ..$ timestamp: int [1:18669201] 838985046 838985046 838983525 838983525 838983525 838983421 838983421 838983392 838983392 838983392 ...
##   .. ..$ title    : chr [1:18669201] "Boomerang (1992)" "Boomerang (1992)" "Net, The (1995)" "Net, The (1995)" ...
##   .. ..$ genres   : chr [1:18669201] "Comedy" "Romance" "Action" "Crime" ...
##   ..$ validation_set:'data.frame':   1827749 obs. of  6 variables:
##   .. ..$ userId   : int [1:1827749] 1 1 1 1 2 2 2 2 3 3 ...
##   .. ..$ movieId  : int [1:1827749] 362 520 539 594 539 590 733 1210 1252 1408 ...
##   .. ..$ rating   : num [1:1827749] 5 5 5 5 3 5 3 4 4 3.5 ...
##   .. ..$ timestamp: int [1:1827749] 838984885 838984679 838984068 838984679 868246262 868245608 868244562 868245644 1133571071 1133571145 ...
##   .. ..$ title    : chr [1:1827749] "Jungle Book, The (1994)" "Robin Hood: Men in Tights (1993)" "Sleepless in Seattle (1993)" "Snow White and the Seven Dwarfs (1937)" ...
##   .. ..$ genres   : chr [1:1827749] "Adventure|Children|Romance" "Comedy" "Comedy|Drama|Romance" "Animation|Children|Drama|Fantasy|Musical" ...
##  $ :List of 4
##   ..$ train_set     :'data.frame':   7172307 obs. of  6 variables:
##   .. ..$ userId   : int [1:7172307] 1 1 1 1 1 1 1 1 1 1 ...
##   .. ..$ movieId  : int [1:7172307] 122 185 292 316 329 355 362 370 377 420 ...
##   .. ..$ rating   : num [1:7172307] 5 5 5 5 5 5 5 5 5 5 ...
##   .. ..$ timestamp: int [1:7172307] 838985046 838983525 838983421 838983392 838983392 838984474 838984885 838984596 838983834 838983834 ...
##   .. ..$ title    : chr [1:7172307] "Boomerang (1992)" "Net, The (1995)" "Outbreak (1995)" "Stargate (1994)" ...
##   .. ..$ genres   : chr [1:7172307] "Comedy|Romance" "Action|Crime|Thriller" "Action|Drama|Sci-Fi|Thriller" "Action|Adventure|Sci-Fi" ...
##   ..$ train_mx      : num [1:69878, 1:10677] 5 NA NA NA NA NA NA NA NA NA ...
##   .. ..- attr(*, "dimnames")=List of 2
##   .. .. ..$ : chr [1:69878] "1" "2" "3" "4" ...
##   .. .. ..$ : chr [1:10677] "122" "185" "292" "316" ...
##   ..$ train.sgr     : tibble [18,669,195 x 6] (S3: tbl_df/tbl/data.frame)
##   .. ..$ userId   : int [1:18669195] 1 1 1 1 1 1 1 1 1 1 ...
##   .. ..$ movieId  : int [1:18669195] 122 122 185 185 185 292 292 292 316 329 ...
##   .. ..$ rating   : num [1:18669195] 5 5 5 5 5 5 5 5 5 5 ...
##   .. ..$ timestamp: int [1:18669195] 838985046 838985046 838983525 838983525 838983525 838983421 838983421 838983421 838983392 838983392 ...
##   .. ..$ title    : chr [1:18669195] "Boomerang (1992)" "Boomerang (1992)" "Net, The (1995)" "Net, The (1995)" ...
##   .. ..$ genres   : chr [1:18669195] "Comedy" "Romance" "Action" "Crime" ...
##   ..$ validation_set:'data.frame':   1827748 obs. of  6 variables:
##   .. ..$ userId   : int [1:1827748] 1 1 1 1 2 2 2 2 3 3 ...
##   .. ..$ movieId  : int [1:1827748] 356 364 539 616 590 719 780 786 151 213 ...
##   .. ..$ rating   : num [1:1827748] 5 5 5 5 5 3 3 3 4.5 5 ...
##   .. ..$ timestamp: int [1:1827748] 838983653 838983707 838984068 838984941 868245608 868246191 868244698 868244562 1133571026 1136075789 ...
##   .. ..$ title    : chr [1:1827748] "Forrest Gump (1994)" "Lion King, The (1994)" "Sleepless in Seattle (1993)" "Aristocats, The (1970)" ...
##   .. ..$ genres   : chr [1:1827748] "Comedy|Drama|Romance|War" "Adventure|Animation|Children|Drama|Musical" "Comedy|Drama|Romance" "Animation|Children" ...
##  $ :List of 4
##   ..$ train_set     :'data.frame':   7172311 obs. of  6 variables:
##   .. ..$ userId   : int [1:7172311] 1 1 1 1 1 1 1 1 1 1 ...
##   .. ..$ movieId  : int [1:7172311] 122 185 292 316 329 355 356 362 364 370 ...
##   .. ..$ rating   : num [1:7172311] 5 5 5 5 5 5 5 5 5 5 ...
##   .. ..$ timestamp: int [1:7172311] 838985046 838983525 838983421 838983392 838983392 838984474 838983653 838984885 838983707 838984596 ...
##   .. ..$ title    : chr [1:7172311] "Boomerang (1992)" "Net, The (1995)" "Outbreak (1995)" "Stargate (1994)" ...
##   .. ..$ genres   : chr [1:7172311] "Comedy|Romance" "Action|Crime|Thriller" "Action|Drama|Sci-Fi|Thriller" "Action|Adventure|Sci-Fi" ...
##   ..$ train_mx      : num [1:69878, 1:10677] 5 NA NA NA NA NA NA NA NA NA ...
##   .. ..- attr(*, "dimnames")=List of 2
##   .. .. ..$ : chr [1:69878] "1" "2" "3" "4" ...
##   .. .. ..$ : chr [1:10677] "122" "185" "292" "316" ...
##   ..$ train.sgr     : tibble [18,669,192 x 6] (S3: tbl_df/tbl/data.frame)
##   .. ..$ userId   : int [1:18669192] 1 1 1 1 1 1 1 1 1 1 ...
##   .. ..$ movieId  : int [1:18669192] 122 122 185 185 292 292 316 316 329 329 ...
##   .. ..$ rating   : num [1:18669192] 5 5 5 5 5 5 5 5 5 5 ...
##   .. ..$ timestamp: int [1:18669192] 838985046 838985046 838983525 838983525 838983421 838983421 838983392 838983392 838983392 838983392 ...
##   .. ..$ title    : chr [1:18669192] "Boomerang (1992)" "Boomerang (1992)" "Net, The (1995)" "Net, The (1995)" ...
##   .. ..$ genres   : chr [1:18669192] "Comedy" "Romance" "Action" "Thriller" ...
##   ..$ validation_set:'data.frame':   1827744 obs. of  6 variables:
##   .. ..$ userId   : int [1:1827744] 1 1 1 1 2 2 2 2 3 3 ...
##   .. ..$ movieId  : int [1:1827744] 377 520 588 616 110 648 1049 1356 1148 1276 ...
##   .. ..$ rating   : num [1:1827744] 5 5 5 5 5 2 3 3 4 3.5 ...
##   .. ..$ timestamp: int [1:1827744] 838983834 838984679 838983339 838984941 868245777 868244699 868245920 868244603 1133571121 1133571205 ...
##   .. ..$ title    : chr [1:1827744] "Speed (1994)" "Robin Hood: Men in Tights (1993)" "Aladdin (1992)" "Aristocats, The (1970)" ...
##   .. ..$ genres   : chr [1:1827744] "Action|Romance|Thriller" "Comedy" "Adventure|Animation|Children|Comedy|Musical" "Animation|Children" ...
##  $ :List of 4
##   ..$ train_set     :'data.frame':   7172301 obs. of  6 variables:
##   .. ..$ userId   : int [1:7172301] 1 1 1 1 1 1 1 1 1 1 ...
##   .. ..$ movieId  : int [1:7172301] 122 185 292 316 355 356 364 370 420 466 ...
##   .. ..$ rating   : num [1:7172301] 5 5 5 5 5 5 5 5 5 5 ...
##   .. ..$ timestamp: int [1:7172301] 838985046 838983525 838983421 838983392 838984474 838983653 838983707 838984596 838983834 838984679 ...
##   .. ..$ title    : chr [1:7172301] "Boomerang (1992)" "Net, The (1995)" "Outbreak (1995)" "Stargate (1994)" ...
##   .. ..$ genres   : chr [1:7172301] "Comedy|Romance" "Action|Crime|Thriller" "Action|Drama|Sci-Fi|Thriller" "Action|Adventure|Sci-Fi" ...
##   ..$ train_mx      : num [1:69878, 1:10677] 5 NA NA NA NA NA NA NA NA NA ...
##   .. ..- attr(*, "dimnames")=List of 2
##   .. .. ..$ : chr [1:69878] "1" "2" "3" "4" ...
##   .. .. ..$ : chr [1:10677] "122" "185" "292" "316" ...
##   ..$ train.sgr     : tibble [18,669,194 x 6] (S3: tbl_df/tbl/data.frame)
##   .. ..$ userId   : int [1:18669194] 1 1 1 1 1 1 1 1 1 1 ...
##   .. ..$ movieId  : int [1:18669194] 122 122 185 185 292 292 316 329 329 355 ...
##   .. ..$ rating   : num [1:18669194] 5 5 5 5 5 5 5 5 5 5 ...
##   .. ..$ timestamp: int [1:18669194] 838985046 838985046 838983525 838983525 838983421 838983421 838983392 838983392 838983392 838984474 ...
##   .. ..$ title    : chr [1:18669194] "Boomerang (1992)" "Boomerang (1992)" "Net, The (1995)" "Net, The (1995)" ...
##   .. ..$ genres   : chr [1:18669194] "Comedy" "Romance" "Crime" "Thriller" ...
##   ..$ validation_set:'data.frame':   1827754 obs. of  6 variables:
##   .. ..$ userId   : int [1:1827754] 1 1 1 1 2 2 2 2 3 3 ...
##   .. ..$ movieId  : int [1:1827754] 329 362 377 594 110 376 539 736 1252 1408 ...
##   .. ..$ rating   : num [1:1827754] 5 5 5 5 5 3 3 3 4 3.5 ...
##   .. ..$ timestamp: int [1:1827754] 838983392 838984885 838983834 838984679 868245777 868245920 868246262 868244698 1133571071 1133571145 ...
##   .. ..$ title    : chr [1:1827754] "Star Trek: Generations (1994)" "Jungle Book, The (1994)" "Speed (1994)" "Snow White and the Seven Dwarfs (1937)" ...
##   .. ..$ genres   : chr [1:1827754] "Action|Adventure|Drama|Sci-Fi" "Adventure|Children|Romance" "Action|Romance|Thriller" "Animation|Children|Drama|Fantasy|Musical" ...
\end{verbatim}

\newpage

\printbibliography

\end{document}
