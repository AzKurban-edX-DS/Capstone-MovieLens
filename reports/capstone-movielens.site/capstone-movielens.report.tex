% Options for packages loaded elsewhere
\PassOptionsToPackage{unicode}{hyperref}
\PassOptionsToPackage{hyphens}{url}
\PassOptionsToPackage{dvipsnames,svgnames,x11names}{xcolor}
%
\documentclass[
]{article}
\usepackage{amsmath,amssymb}
\usepackage{iftex}
\ifPDFTeX
  \usepackage[T1]{fontenc}
  \usepackage[utf8]{inputenc}
  \usepackage{textcomp} % provide euro and other symbols
\else % if luatex or xetex
  \usepackage{unicode-math} % this also loads fontspec
  \defaultfontfeatures{Scale=MatchLowercase}
  \defaultfontfeatures[\rmfamily]{Ligatures=TeX,Scale=1}
\fi
\usepackage{lmodern}
\ifPDFTeX\else
  % xetex/luatex font selection
\fi
% Use upquote if available, for straight quotes in verbatim environments
\IfFileExists{upquote.sty}{\usepackage{upquote}}{}
\IfFileExists{microtype.sty}{% use microtype if available
  \usepackage[]{microtype}
  \UseMicrotypeSet[protrusion]{basicmath} % disable protrusion for tt fonts
}{}
\makeatletter
\@ifundefined{KOMAClassName}{% if non-KOMA class
  \IfFileExists{parskip.sty}{%
    \usepackage{parskip}
  }{% else
    \setlength{\parindent}{0pt}
    \setlength{\parskip}{6pt plus 2pt minus 1pt}}
}{% if KOMA class
  \KOMAoptions{parskip=half}}
\makeatother
\usepackage{xcolor}
\usepackage[margin=1in]{geometry}
\usepackage{color}
\usepackage{fancyvrb}
\newcommand{\VerbBar}{|}
\newcommand{\VERB}{\Verb[commandchars=\\\{\}]}
\DefineVerbatimEnvironment{Highlighting}{Verbatim}{commandchars=\\\{\}}
% Add ',fontsize=\small' for more characters per line
\usepackage{framed}
\definecolor{shadecolor}{RGB}{248,248,248}
\newenvironment{Shaded}{\begin{snugshade}}{\end{snugshade}}
\newcommand{\AlertTok}[1]{\textcolor[rgb]{0.94,0.16,0.16}{#1}}
\newcommand{\AnnotationTok}[1]{\textcolor[rgb]{0.56,0.35,0.01}{\textbf{\textit{#1}}}}
\newcommand{\AttributeTok}[1]{\textcolor[rgb]{0.13,0.29,0.53}{#1}}
\newcommand{\BaseNTok}[1]{\textcolor[rgb]{0.00,0.00,0.81}{#1}}
\newcommand{\BuiltInTok}[1]{#1}
\newcommand{\CharTok}[1]{\textcolor[rgb]{0.31,0.60,0.02}{#1}}
\newcommand{\CommentTok}[1]{\textcolor[rgb]{0.56,0.35,0.01}{\textit{#1}}}
\newcommand{\CommentVarTok}[1]{\textcolor[rgb]{0.56,0.35,0.01}{\textbf{\textit{#1}}}}
\newcommand{\ConstantTok}[1]{\textcolor[rgb]{0.56,0.35,0.01}{#1}}
\newcommand{\ControlFlowTok}[1]{\textcolor[rgb]{0.13,0.29,0.53}{\textbf{#1}}}
\newcommand{\DataTypeTok}[1]{\textcolor[rgb]{0.13,0.29,0.53}{#1}}
\newcommand{\DecValTok}[1]{\textcolor[rgb]{0.00,0.00,0.81}{#1}}
\newcommand{\DocumentationTok}[1]{\textcolor[rgb]{0.56,0.35,0.01}{\textbf{\textit{#1}}}}
\newcommand{\ErrorTok}[1]{\textcolor[rgb]{0.64,0.00,0.00}{\textbf{#1}}}
\newcommand{\ExtensionTok}[1]{#1}
\newcommand{\FloatTok}[1]{\textcolor[rgb]{0.00,0.00,0.81}{#1}}
\newcommand{\FunctionTok}[1]{\textcolor[rgb]{0.13,0.29,0.53}{\textbf{#1}}}
\newcommand{\ImportTok}[1]{#1}
\newcommand{\InformationTok}[1]{\textcolor[rgb]{0.56,0.35,0.01}{\textbf{\textit{#1}}}}
\newcommand{\KeywordTok}[1]{\textcolor[rgb]{0.13,0.29,0.53}{\textbf{#1}}}
\newcommand{\NormalTok}[1]{#1}
\newcommand{\OperatorTok}[1]{\textcolor[rgb]{0.81,0.36,0.00}{\textbf{#1}}}
\newcommand{\OtherTok}[1]{\textcolor[rgb]{0.56,0.35,0.01}{#1}}
\newcommand{\PreprocessorTok}[1]{\textcolor[rgb]{0.56,0.35,0.01}{\textit{#1}}}
\newcommand{\RegionMarkerTok}[1]{#1}
\newcommand{\SpecialCharTok}[1]{\textcolor[rgb]{0.81,0.36,0.00}{\textbf{#1}}}
\newcommand{\SpecialStringTok}[1]{\textcolor[rgb]{0.31,0.60,0.02}{#1}}
\newcommand{\StringTok}[1]{\textcolor[rgb]{0.31,0.60,0.02}{#1}}
\newcommand{\VariableTok}[1]{\textcolor[rgb]{0.00,0.00,0.00}{#1}}
\newcommand{\VerbatimStringTok}[1]{\textcolor[rgb]{0.31,0.60,0.02}{#1}}
\newcommand{\WarningTok}[1]{\textcolor[rgb]{0.56,0.35,0.01}{\textbf{\textit{#1}}}}
\usepackage{longtable,booktabs,array}
\usepackage{calc} % for calculating minipage widths
% Correct order of tables after \paragraph or \subparagraph
\usepackage{etoolbox}
\makeatletter
\patchcmd\longtable{\par}{\if@noskipsec\mbox{}\fi\par}{}{}
\makeatother
% Allow footnotes in longtable head/foot
\IfFileExists{footnotehyper.sty}{\usepackage{footnotehyper}}{\usepackage{footnote}}
\makesavenoteenv{longtable}
\usepackage{graphicx}
\makeatletter
\newsavebox\pandoc@box
\newcommand*\pandocbounded[1]{% scales image to fit in text height/width
  \sbox\pandoc@box{#1}%
  \Gscale@div\@tempa{\textheight}{\dimexpr\ht\pandoc@box+\dp\pandoc@box\relax}%
  \Gscale@div\@tempb{\linewidth}{\wd\pandoc@box}%
  \ifdim\@tempb\p@<\@tempa\p@\let\@tempa\@tempb\fi% select the smaller of both
  \ifdim\@tempa\p@<\p@\scalebox{\@tempa}{\usebox\pandoc@box}%
  \else\usebox{\pandoc@box}%
  \fi%
}
% Set default figure placement to htbp
\def\fps@figure{htbp}
\makeatother
\setlength{\emergencystretch}{3em} % prevent overfull lines
\providecommand{\tightlist}{%
  \setlength{\itemsep}{0pt}\setlength{\parskip}{0pt}}
\setcounter{secnumdepth}{5}
\usepackage{awesomebox}
\usepackage{hyperref}
\usepackage{amsmath}
\usepackage[natbib=true, style=numeric, backref=true, sorting=none]{biblatex}
\hypersetup{backref, pdfpagemode=Normal, colorlinks=true, implicit=false}
\usepackage[]{biblatex}
\addbibresource{references.bib}
\usepackage{bookmark}
\IfFileExists{xurl.sty}{\usepackage{xurl}}{} % add URL line breaks if available
\urlstyle{same}
\hypersetup{
  pdftitle={Capstone Movielens Report},
  pdfauthor={Azamat Kurbanaev},
  colorlinks=true,
  linkcolor={red},
  filecolor={Maroon},
  citecolor={blue},
  urlcolor={Blue},
  pdfcreator={LaTeX via pandoc}}

\title{Capstone Movielens Report}
\author{Azamat Kurbanaev}
\date{2025-06-07}

\begin{document}
\maketitle

{
\hypersetup{linkcolor=}
\setcounter{tocdepth}{3}
\tableofcontents
}
\newenvironment{infobox}[1]
  {
  \begin{itemize}
  \renewcommand{\labelitemi}{
    \raisebox{-.7\height}[0pt][0pt]{
      {\setkeys{Gin}{width=3em,keepaspectratio}
        \includegraphics{images/#1}}
    }
  }
  \setlength{\fboxsep}{1em}
  \begin{blackbox}
  \item
  }
  {
  \end{blackbox}
  \end{itemize}
  }

\newpage

\section{Introduction / Overview / Executive Summary}\label{introduction-overview-executive-summary}

The goal of the project is to build a Recommendation System using a \href{http://grouplens.org/datasets/movielens/10m/}{10M version of the MovieLens dataset}.
Following the \href{https://archive.nytimes.com/bits.blogs.nytimes.com/2009/09/21/netflix-awards-1-million-prize-and-starts-a-new-contest/index.html}{Netflix Grand Prize Contest} requirements, we will evaluate the \emph{Root Mean Square Error} (\emph{RMSE}) score, which, as shown in \href{https://rafalab.dfci.harvard.edu/dsbook-part-2/highdim/regularization.html\#sec-netflix-loss-function}{Section 23.2 Loss function} of the \emph{Course Textbook}, is defined as:
\[
\mbox{RMSE} = \sqrt{\frac{1}{N} \sum_{i,j}^{N} (y_{i,j} - \hat{y}_{i,j})^2}
\]

with \(N\) being the number of user/movie combinations for which we make predictions and the sum occurring over all these combinations\autocite{IDS2_23-2}.

Our goal is to achieve a value of less than 0.86490 (compare with the \emph{Netflix Grand Prize} requirement: of at least 0.8563\autocite{BigChaosSln}).

\subsection{Datasets Overview}\label{datasets-overview}

To start with we have to generate two datasets derived from the \emph{MovieLens} one mentioned above:

\begin{itemize}
\tightlist
\item
  \textbf{edx:} we use it to develop and train our algorithms;
\item
  \textbf{final\_holdout\_test:} according to the course requirements, we use it exclusively to evaluate the \emph{\textbf{RMSE}} of our final algorithm.
\end{itemize}

For this purpose the following package has been developed by the author of this report: \texttt{edx.capstone.movielens.data}. The source code of the package is available \href{https://github.com/AzKurban-edX-DS/edx.capstone.movielens.data}{on GitHub}\autocite{edx_capstone_movielens_data}.

Let's install the development version of this package from the GitHub repository and attach the correspondent library to the global environment:

\begin{Shaded}
\begin{Highlighting}[]
\ControlFlowTok{if}\NormalTok{(}\SpecialCharTok{!}\FunctionTok{require}\NormalTok{(edx.capstone.movielens.data)) pak}\SpecialCharTok{::}\FunctionTok{pak}\NormalTok{(}\StringTok{"AzKurban{-}edX{-}DS/edx.capstone.movielens.data"}\NormalTok{)}

\FunctionTok{library}\NormalTok{(edx.capstone.movielens.data)}
\NormalTok{edx }\OtherTok{\textless{}{-}}\NormalTok{ edx.capstone.movielens.data}\SpecialCharTok{::}\NormalTok{edx}
\NormalTok{final\_holdout\_test }\OtherTok{\textless{}{-}}\NormalTok{ edx.capstone.movielens.data}\SpecialCharTok{::}\NormalTok{final\_holdout\_test}
\end{Highlighting}
\end{Shaded}

Now, we have the datasets listed above:

\begin{Shaded}
\begin{Highlighting}[]
\FunctionTok{summary}\NormalTok{(edx)}
\end{Highlighting}
\end{Shaded}

\begin{verbatim}
##      userId         movieId          rating        timestamp            title              genres         
##  Min.   :    1   Min.   :    1   Min.   :0.500   Min.   :7.897e+08   Length:9000055     Length:9000055    
##  1st Qu.:18124   1st Qu.:  648   1st Qu.:3.000   1st Qu.:9.468e+08   Class :character   Class :character  
##  Median :35738   Median : 1834   Median :4.000   Median :1.035e+09   Mode  :character   Mode  :character  
##  Mean   :35870   Mean   : 4122   Mean   :3.512   Mean   :1.033e+09                                        
##  3rd Qu.:53607   3rd Qu.: 3626   3rd Qu.:4.000   3rd Qu.:1.127e+09                                        
##  Max.   :71567   Max.   :65133   Max.   :5.000   Max.   :1.231e+09
\end{verbatim}

\begin{Shaded}
\begin{Highlighting}[]
\FunctionTok{summary}\NormalTok{(final\_holdout\_test)}
\end{Highlighting}
\end{Shaded}

\begin{verbatim}
##      userId         movieId          rating        timestamp            title              genres         
##  Min.   :    1   Min.   :    1   Min.   :0.500   Min.   :7.897e+08   Length:999999      Length:999999     
##  1st Qu.:18096   1st Qu.:  648   1st Qu.:3.000   1st Qu.:9.467e+08   Class :character   Class :character  
##  Median :35768   Median : 1827   Median :4.000   Median :1.035e+09   Mode  :character   Mode  :character  
##  Mean   :35870   Mean   : 4108   Mean   :3.512   Mean   :1.033e+09                                        
##  3rd Qu.:53621   3rd Qu.: 3624   3rd Qu.:4.000   3rd Qu.:1.127e+09                                        
##  Max.   :71567   Max.   :65133   Max.   :5.000   Max.   :1.231e+09
\end{verbatim}

\newpage

\subsubsection{\texorpdfstring{\texttt{edx} Dataset}{edx Dataset}}\label{edx-dataset}

\hfill\break
Let's look into the details of the \texttt{edx} dataset:

\begin{Shaded}
\begin{Highlighting}[]
\FunctionTok{str}\NormalTok{(edx)}
\end{Highlighting}
\end{Shaded}

\begin{verbatim}
## 'data.frame':    9000055 obs. of  6 variables:
##  $ userId   : int  1 1 1 1 1 1 1 1 1 1 ...
##  $ movieId  : int  122 185 292 316 329 355 356 362 364 370 ...
##  $ rating   : num  5 5 5 5 5 5 5 5 5 5 ...
##  $ timestamp: int  838985046 838983525 838983421 838983392 838983392 838984474 838983653 838984885 838983707 838984596 ...
##  $ title    : chr  "Boomerang (1992)" "Net, The (1995)" "Outbreak (1995)" "Stargate (1994)" ...
##  $ genres   : chr  "Comedy|Romance" "Action|Crime|Thriller" "Action|Drama|Sci-Fi|Thriller" "Action|Adventure|Sci-Fi" ...
\end{verbatim}

Note that we have 9000055 rows and six columns in there:

\begin{Shaded}
\begin{Highlighting}[]
\NormalTok{dim\_edx }\OtherTok{\textless{}{-}} \FunctionTok{dim}\NormalTok{(edx)}
\FunctionTok{print}\NormalTok{(dim\_edx)}
\end{Highlighting}
\end{Shaded}

\begin{verbatim}
## [1] 9000055       6
\end{verbatim}

First, let's note that we have 10677 different movies:

\begin{Shaded}
\begin{Highlighting}[]
\NormalTok{n\_movies }\OtherTok{\textless{}{-}} \FunctionTok{n\_distinct}\NormalTok{(edx}\SpecialCharTok{$}\NormalTok{movieId)}
\FunctionTok{print}\NormalTok{(n\_movies)}
\end{Highlighting}
\end{Shaded}

\begin{verbatim}
## [1] 10677
\end{verbatim}

and 69878 different users in the dataset:

\begin{Shaded}
\begin{Highlighting}[]
\NormalTok{n\_users }\OtherTok{\textless{}{-}} \FunctionTok{n\_distinct}\NormalTok{(edx}\SpecialCharTok{$}\NormalTok{userId)}
\FunctionTok{print}\NormalTok{(n\_users)}
\end{Highlighting}
\end{Shaded}

\begin{verbatim}
## [1] 69878
\end{verbatim}

Now, note the expressions below which confirm the fact explained in \href{https://rafalab.dfci.harvard.edu/dsbook-part-2/highdim/regularization.html\#movielens-data}{Section \emph{23.1.1 Movielens data}} of the \emph{Course Textbook}\autocite{IDS2} that not every user rated every movie:

\begin{Shaded}
\begin{Highlighting}[]
\NormalTok{max\_possible\_ratings }\OtherTok{\textless{}{-}}\NormalTok{ n\_movies}\SpecialCharTok{*}\NormalTok{n\_users}
\FunctionTok{sprintf}\NormalTok{(}\StringTok{"Maximum possible ratings: \%s"}\NormalTok{, max\_possible\_ratings)}
\end{Highlighting}
\end{Shaded}

\begin{verbatim}
## [1] "Maximum possible ratings: 746087406"
\end{verbatim}

\begin{Shaded}
\begin{Highlighting}[]
\FunctionTok{sprintf}\NormalTok{(}\StringTok{"Rows in \textasciigrave{}edx\textasciigrave{} dataset: \%s"}\NormalTok{, dim\_edx[}\DecValTok{1}\NormalTok{])}
\end{Highlighting}
\end{Shaded}

\begin{verbatim}
## [1] "Rows in `edx` dataset: 9000055"
\end{verbatim}

\begin{Shaded}
\begin{Highlighting}[]
\FunctionTok{sprintf}\NormalTok{(}\StringTok{"Not every movie was rated: \%s"}\NormalTok{, max\_possible\_ratings }\SpecialCharTok{\textgreater{}}\NormalTok{ dim\_edx[}\DecValTok{1}\NormalTok{])}
\end{Highlighting}
\end{Shaded}

\begin{verbatim}
## [1] "Not every movie was rated: TRUE"
\end{verbatim}

As also explained in that section, we can think of these data as a very large matrix, with users on the rows and movies on the columns, with many empty cells. Therefore, we can think of a recommendation system as filling in the \texttt{NA}s in the dataset for the movies that some or all the users do not rate. A sample from the \texttt{edx} data below illustrates this idea\autocite{IDS2_23-1-1}:

\begin{Shaded}
\begin{Highlighting}[]
\NormalTok{keep }\OtherTok{\textless{}{-}}\NormalTok{ edx }\SpecialCharTok{|\textgreater{}} 
\NormalTok{  dplyr}\SpecialCharTok{::}\FunctionTok{count}\NormalTok{(movieId) }\SpecialCharTok{|\textgreater{}} 
  \FunctionTok{top\_n}\NormalTok{(}\DecValTok{4}\NormalTok{, n) }\SpecialCharTok{|\textgreater{}} 
  \FunctionTok{pull}\NormalTok{(movieId)}

\NormalTok{tab }\OtherTok{\textless{}{-}}\NormalTok{ edx }\SpecialCharTok{|\textgreater{}} 
  \FunctionTok{filter}\NormalTok{(movieId }\SpecialCharTok{\%in\%}\NormalTok{ keep) }\SpecialCharTok{|\textgreater{}} 
  \FunctionTok{filter}\NormalTok{(userId }\SpecialCharTok{\%in\%} \FunctionTok{c}\NormalTok{(}\DecValTok{13}\SpecialCharTok{:}\DecValTok{20}\NormalTok{)) }\SpecialCharTok{|\textgreater{}} 
  \FunctionTok{select}\NormalTok{(userId, title, rating) }\SpecialCharTok{|\textgreater{}} 
  \FunctionTok{mutate}\NormalTok{(}\AttributeTok{title =} \FunctionTok{str\_remove}\NormalTok{(title, }\StringTok{", The"}\NormalTok{),}
         \AttributeTok{title =} \FunctionTok{str\_remove}\NormalTok{(title, }\StringTok{":.*"}\NormalTok{)) }\SpecialCharTok{|\textgreater{}}
  \FunctionTok{pivot\_wider}\NormalTok{(}\AttributeTok{names\_from =} \StringTok{"title"}\NormalTok{, }\AttributeTok{values\_from =} \StringTok{"rating"}\NormalTok{)}

\FunctionTok{print}\NormalTok{(tab)}
\end{Highlighting}
\end{Shaded}

\begin{verbatim}
## # A tibble: 5 x 5
##   userId `Pulp Fiction (1994)` `Jurassic Park (1993)` `Silence of the Lambs (1991)` `Forrest Gump (1994)`
##    <int>                 <dbl>                  <dbl>                         <dbl>                 <dbl>
## 1     13                     4                     NA                            NA                    NA
## 2     16                    NA                      3                            NA                    NA
## 3     17                    NA                     NA                             5                    NA
## 4     18                     5                      3                             5                    NA
## 5     19                    NA                      1                            NA                     4
\end{verbatim}

The following plot of the matrix for a random sample of 100 movies and 100 users with yellow indicating a user/movie combination for which we have a rating shows how \emph{sparse} the matrix is:

\begin{Shaded}
\begin{Highlighting}[]
\NormalTok{users }\OtherTok{\textless{}{-}} \FunctionTok{sample}\NormalTok{(}\FunctionTok{unique}\NormalTok{(edx}\SpecialCharTok{$}\NormalTok{userId), }\DecValTok{100}\NormalTok{)}

\NormalTok{rafalib}\SpecialCharTok{::}\FunctionTok{mypar}\NormalTok{()}
\NormalTok{edx}\SpecialCharTok{|\textgreater{}} 
  \FunctionTok{filter}\NormalTok{(userId }\SpecialCharTok{\%in\%}\NormalTok{ users) }\SpecialCharTok{|\textgreater{}} 
  \FunctionTok{select}\NormalTok{(userId, movieId, rating) }\SpecialCharTok{|\textgreater{}}
  \FunctionTok{mutate}\NormalTok{(}\AttributeTok{rating =} \DecValTok{1}\NormalTok{) }\SpecialCharTok{|\textgreater{}}
  \FunctionTok{pivot\_wider}\NormalTok{(}\AttributeTok{names\_from =}\NormalTok{ movieId, }\AttributeTok{values\_from =}\NormalTok{ rating) }\SpecialCharTok{|\textgreater{}} 
\NormalTok{  (\textbackslash{}(mat) mat[, }\FunctionTok{sample}\NormalTok{(}\FunctionTok{ncol}\NormalTok{(mat), }\DecValTok{100}\NormalTok{)])() }\SpecialCharTok{|\textgreater{}}
  \FunctionTok{as.matrix}\NormalTok{() }\SpecialCharTok{|\textgreater{}} 
  \FunctionTok{t}\NormalTok{() }\SpecialCharTok{|\textgreater{}}
  \FunctionTok{image}\NormalTok{(}\DecValTok{1}\SpecialCharTok{:}\DecValTok{100}\NormalTok{, }\DecValTok{1}\SpecialCharTok{:}\DecValTok{100}\NormalTok{, }\AttributeTok{z =}\NormalTok{ \_ , }\AttributeTok{xlab =} \StringTok{"Movies"}\NormalTok{, }\AttributeTok{ylab =} \StringTok{"Users"}\NormalTok{)}
\end{Highlighting}
\end{Shaded}

\includegraphics[width=0.4\linewidth]{capstone-movielens.report_files/figure-latex/sparsity-of-movie-recs-1}

Further observations highlighted there that, as we can see from the distributions the author presented, some movies get rated more than others, and some users are more active than others in rating movies:

\begin{Shaded}
\begin{Highlighting}[]
\NormalTok{p1 }\OtherTok{\textless{}{-}}\NormalTok{ edx }\SpecialCharTok{|\textgreater{}} 
  \FunctionTok{count}\NormalTok{(movieId) }\SpecialCharTok{|\textgreater{}} 
  \FunctionTok{ggplot}\NormalTok{(}\FunctionTok{aes}\NormalTok{(n)) }\SpecialCharTok{+} 
  \FunctionTok{geom\_histogram}\NormalTok{(}\AttributeTok{bins =} \DecValTok{30}\NormalTok{, }\AttributeTok{color =} \StringTok{"black"}\NormalTok{) }\SpecialCharTok{+} 
  \FunctionTok{scale\_x\_log10}\NormalTok{() }\SpecialCharTok{+} 
  \FunctionTok{ggtitle}\NormalTok{(}\StringTok{"Movies"}\NormalTok{)}

\NormalTok{p2 }\OtherTok{\textless{}{-}}\NormalTok{ edx }\SpecialCharTok{|\textgreater{}} 
  \FunctionTok{count}\NormalTok{(userId) }\SpecialCharTok{|\textgreater{}} 
  \FunctionTok{ggplot}\NormalTok{(}\FunctionTok{aes}\NormalTok{(n)) }\SpecialCharTok{+} 
  \FunctionTok{geom\_histogram}\NormalTok{(}\AttributeTok{bins =} \DecValTok{30}\NormalTok{, }\AttributeTok{color =} \StringTok{"black"}\NormalTok{) }\SpecialCharTok{+} 
  \FunctionTok{scale\_x\_log10}\NormalTok{() }\SpecialCharTok{+} 
  \FunctionTok{ggtitle}\NormalTok{(}\StringTok{"Users"}\NormalTok{)}

\NormalTok{gridExtra}\SpecialCharTok{::}\FunctionTok{grid.arrange}\NormalTok{(p2, p1, }\AttributeTok{ncol =} \DecValTok{2}\NormalTok{)}
\end{Highlighting}
\end{Shaded}

\pandocbounded{\includegraphics[keepaspectratio]{capstone-movielens.report_files/figure-latex/movie-id-and-user-hists-1.pdf}}

Finally, we can see that no movies have a rating of 0. Movies are rated from 0.5 to 5.0 in 0.5 increments:

\begin{Shaded}
\begin{Highlighting}[]
\CommentTok{\#library(dplyr)}
\NormalTok{s }\OtherTok{\textless{}{-}}\NormalTok{ edx }\SpecialCharTok{|\textgreater{}} \FunctionTok{group\_by}\NormalTok{(rating) }\SpecialCharTok{|\textgreater{}}
  \FunctionTok{summarise}\NormalTok{(}\AttributeTok{n =} \FunctionTok{n}\NormalTok{())}
\FunctionTok{print}\NormalTok{(s)}
\end{Highlighting}
\end{Shaded}

\begin{verbatim}
## # A tibble: 10 x 2
##    rating       n
##     <dbl>   <int>
##  1    0.5   85374
##  2    1    345679
##  3    1.5  106426
##  4    2    711422
##  5    2.5  333010
##  6    3   2121240
##  7    3.5  791624
##  8    4   2588430
##  9    4.5  526736
## 10    5   1390114
\end{verbatim}

\newpage

\section{Methods / Analysis}\label{methods-analysis}

\begin{noteblock}
All the source code of the R-scripts is available on the project's \href{https://github.com/AzKurban-edX-DS/Capstone-MovieLens/tree/main/r/src}{GitHub repository}\autocite{edx_capstone_movielens}.

\end{noteblock}

\subsection{Defining Logging and Time Measuring Helper Functions}\label{defining-logging-and-time-measuring-helper-functions}

\hfill\break

First, let's define some helper functions for logging and time-measuring features that we will use in our R scripts. Some of them are listed below:

\begin{Shaded}
\begin{Highlighting}[]
\CommentTok{\# Logging Helper functions {-}{-}{-}{-}{-}{-}{-}{-}{-}{-}{-}{-}{-}{-}{-}{-}{-}{-}{-}{-}{-}{-}{-}{-}{-}{-}{-}{-}{-}{-}{-}{-}{-}{-}{-}{-}{-}{-}{-}{-}{-}{-}{-}{-}{-}{-}{-}{-}{-}{-}{-}{-}{-}}
\NormalTok{open\_logfile }\OtherTok{\textless{}{-}} \ControlFlowTok{function}\NormalTok{(file\_name)\{}
\NormalTok{  log\_file\_name }\OtherTok{\textless{}{-}} \FunctionTok{as.character}\NormalTok{(}\FunctionTok{Sys.time}\NormalTok{()) }\SpecialCharTok{|\textgreater{}} 
    \FunctionTok{str\_replace\_all}\NormalTok{(}\StringTok{\textquotesingle{}:\textquotesingle{}}\NormalTok{, }\StringTok{\textquotesingle{}\_\textquotesingle{}}\NormalTok{) }\SpecialCharTok{|\textgreater{}} 
    \FunctionTok{str\_replace}\NormalTok{(}\StringTok{\textquotesingle{} \textquotesingle{}}\NormalTok{, }\StringTok{\textquotesingle{}T\textquotesingle{}}\NormalTok{) }\SpecialCharTok{|\textgreater{}}
    \FunctionTok{str\_c}\NormalTok{(file\_name)}
  
  \FunctionTok{log\_open}\NormalTok{(}\AttributeTok{file\_name =}\NormalTok{ log\_file\_name)}
\NormalTok{\}}
\NormalTok{print\_start\_date }\OtherTok{\textless{}{-}} \ControlFlowTok{function}\NormalTok{()\{}
  \FunctionTok{print}\NormalTok{(}\FunctionTok{date}\NormalTok{())}
  \FunctionTok{Sys.time}\NormalTok{()}
\NormalTok{\}}
\NormalTok{put\_start\_date }\OtherTok{\textless{}{-}} \ControlFlowTok{function}\NormalTok{()\{}
  \FunctionTok{put}\NormalTok{(}\FunctionTok{date}\NormalTok{())}
  \FunctionTok{Sys.time}\NormalTok{()}
\NormalTok{\}}
\NormalTok{print\_end\_date }\OtherTok{\textless{}{-}} \ControlFlowTok{function}\NormalTok{(start)\{}
  \FunctionTok{print}\NormalTok{(}\FunctionTok{date}\NormalTok{())}
  \FunctionTok{print}\NormalTok{(}\FunctionTok{Sys.time}\NormalTok{() }\SpecialCharTok{{-}}\NormalTok{ start)}
\NormalTok{\}}
\NormalTok{put\_end\_date }\OtherTok{\textless{}{-}} \ControlFlowTok{function}\NormalTok{(start)\{}
  \FunctionTok{put}\NormalTok{(}\FunctionTok{date}\NormalTok{())}
  \FunctionTok{put}\NormalTok{(}\FunctionTok{Sys.time}\NormalTok{() }\SpecialCharTok{{-}}\NormalTok{ start)}
\NormalTok{\}}

\NormalTok{msg.set\_arg }\OtherTok{\textless{}{-}} \ControlFlowTok{function}\NormalTok{(msg\_template, arg, }\AttributeTok{arg.name =} \StringTok{"\%1"}\NormalTok{) \{}
\NormalTok{  msg\_template }\SpecialCharTok{|\textgreater{}} 
    \FunctionTok{str\_replace\_all}\NormalTok{(arg.name, }\FunctionTok{as.character}\NormalTok{(arg))}
\NormalTok{\}}
\NormalTok{msg.glue }\OtherTok{\textless{}{-}} \ControlFlowTok{function}\NormalTok{(msg\_template, arg, }\AttributeTok{arg.name =} \StringTok{"\%1"}\NormalTok{)\{}
\NormalTok{  msg\_template }\SpecialCharTok{|\textgreater{}}
    \FunctionTok{msg.set\_arg}\NormalTok{(arg, arg.name) }\SpecialCharTok{|\textgreater{}}
    \FunctionTok{str\_glue}\NormalTok{()}
\NormalTok{\}}

\NormalTok{print\_log }\OtherTok{\textless{}{-}} \ControlFlowTok{function}\NormalTok{(msg)\{}
  \FunctionTok{print}\NormalTok{(}\FunctionTok{str\_glue}\NormalTok{(msg))}
\NormalTok{\}}
\NormalTok{put\_log }\OtherTok{\textless{}{-}} \ControlFlowTok{function}\NormalTok{(msg)\{}
  \FunctionTok{put}\NormalTok{(}\FunctionTok{str\_glue}\NormalTok{(msg))}
\NormalTok{\}}

\NormalTok{get\_log1 }\OtherTok{\textless{}{-}} \ControlFlowTok{function}\NormalTok{(msg\_template, arg1) \{}
  \FunctionTok{str\_glue}\NormalTok{(}\FunctionTok{str\_replace\_all}\NormalTok{(msg\_template, }\StringTok{"\%1"}\NormalTok{, }\FunctionTok{as.character}\NormalTok{(arg1)))}
\NormalTok{\}}
\NormalTok{print\_log1 }\OtherTok{\textless{}{-}} \ControlFlowTok{function}\NormalTok{(msg\_template, arg1)\{}
  \FunctionTok{print}\NormalTok{(}\FunctionTok{get\_log1}\NormalTok{(msg\_template, arg1))}
\NormalTok{\}}
\NormalTok{put\_log1 }\OtherTok{\textless{}{-}} \ControlFlowTok{function}\NormalTok{(msg\_template, arg1)\{}
  \FunctionTok{put}\NormalTok{(}\FunctionTok{get\_log1}\NormalTok{(msg\_template, arg1))}
\NormalTok{\}}

\NormalTok{get\_log2 }\OtherTok{\textless{}{-}} \ControlFlowTok{function}\NormalTok{(msg\_template, arg1, arg2) \{}
\NormalTok{  msg\_template }\SpecialCharTok{|\textgreater{}} 
    \FunctionTok{str\_replace\_all}\NormalTok{(}\StringTok{"\%1"}\NormalTok{, }\FunctionTok{as.character}\NormalTok{(arg1)) }\SpecialCharTok{|\textgreater{}}
    \FunctionTok{str\_replace\_all}\NormalTok{(}\StringTok{"\%2"}\NormalTok{, }\FunctionTok{as.character}\NormalTok{(arg2)) }\SpecialCharTok{|\textgreater{}}
    \FunctionTok{str\_glue}\NormalTok{()}
\NormalTok{\}}
\NormalTok{print\_log2 }\OtherTok{\textless{}{-}} \ControlFlowTok{function}\NormalTok{(msg\_template, arg1, arg2)\{}
  \FunctionTok{print}\NormalTok{(}\FunctionTok{get\_log1}\NormalTok{(msg\_template, arg1, arg2))}
\NormalTok{\}}
\NormalTok{put\_log2 }\OtherTok{\textless{}{-}} \ControlFlowTok{function}\NormalTok{(msg\_template, arg1, arg2)\{}
  \FunctionTok{put}\NormalTok{(}\FunctionTok{get\_log1}\NormalTok{(msg\_template, arg1, arg2))}
\NormalTok{\}}

\CommentTok{\# ...}
\end{Highlighting}
\end{Shaded}

\begin{noteblock}
The full source code of these functions is available in the \href{https://github.com/AzKurban-edX-DS/Capstone-MovieLens/blob/main/r/src/capstone-movielens.main.R\#L20}{Logging Helper functions} section of the \href{https://github.com/AzKurban-edX-DS/Capstone-MovieLens/blob/main/r/src/capstone-movielens.main.R\#L20}{capstone-movielens.main.R} script on \emph{GitHub}.

\end{noteblock}

\newpage

\subsection{Preparing train and set datasets}\label{preparing-train-and-set-datasets}

\hfill\break

We will split the \texttt{edx} dataset into a training set, which we will use to build and train our models, and a test set in which we will compute the accuracy of our predictions, the way described in \href{https://rafalab.dfci.harvard.edu/dsbook-part-2/highdim/regularization.html\#movielens-data}{Section 23.1.1 Movielens data} of the \emph{Course Textbook} mentioned above\autocite{IDS2_23-1-1}.We will also use the \emph{5-Fold Cross Validation} method as described in \href{https://rafalab.dfci.harvard.edu/dsbook-part-2/ml/resampling-methods.html\#cross-validation}{Section 29.6 Cross validation} of the \emph{Course Textbook}.
To prepare datasets for processing, we will use the following functions, specifically designed for these operations:

\phantomsection\label{func.make_source_datasets}
\begin{itemize}
\tightlist
\item
  \href{https://github.com/AzKurban-edX-DS/Capstone-MovieLens/blob/main/r/src/support-functions/data.helper.functions.R\#L87}{make\_source\_datasets}
\end{itemize}

\phantomsection\label{func.init_source_datasets}
\begin{itemize}
\tightlist
\item
  \href{https://github.com/AzKurban-edX-DS/Capstone-MovieLens/blob/main/r/src/support-functions/data.helper.functions.R\#L209}{init\_source\_datasets}
\end{itemize}

\begin{noteblock}
The full source code of the function listed above is available in the \href{https://github.com/AzKurban-edX-DS/Capstone-MovieLens/blob/main/r/src/support-functions/data.helper.functions.R\#L86}{Initialize input datasets} section of the \href{https://github.com/AzKurban-edX-DS/Capstone-MovieLens/blob/main/r/src/support-functions/data.helper.functions.R}{data.helper.functions.R} script on \emph{GitHub}.

\end{noteblock}

\subsubsection{\texorpdfstring{The \hyperref[func.make_source_datasets]{make\_source\_datasets} function}{The make\_source\_datasets function}}\label{the-make_source_datasets-function}

\hfill\break

Let's take a closer look at the objects we will receive as a result of executing this function.

\begin{Shaded}
\begin{Highlighting}[]
\NormalTok{make\_source\_datasets }\OtherTok{\textless{}{-}} \ControlFlowTok{function}\NormalTok{()\{}
  \CommentTok{\# ...}
  \FunctionTok{list}\NormalTok{(}\AttributeTok{edx\_CV =}\NormalTok{ edx\_CV,}
       \AttributeTok{edx.mx =}\NormalTok{ edx.mx,}
       \AttributeTok{edx.sgr =}\NormalTok{ edx.sgr,}
       \AttributeTok{tuning\_sets =}\NormalTok{ tuning\_sets,}
       \AttributeTok{movie\_map =}\NormalTok{ movie\_map,}
       \AttributeTok{date\_days\_map =}\NormalTok{ date\_days\_map)}
\NormalTok{\}}
\end{Highlighting}
\end{Shaded}

\paragraph{\texorpdfstring{\texttt{edx.mx} Matrix Object}{edx.mx Matrix Object}}\label{edx.mx-matrix-object}

\hfill\break

We will use the array representation described in \href{https://rafalab.dfci.harvard.edu/dsbook-part-2/linear-models/treatment-effect-models.html\#sec-anova}{Section 17.5 of the Textbook}, for the training data: we denote ranking for movie \(j\) by user \(i\) as \(y_{i,j}\). To create this matrix, we use \texttt{tidyr::pivot\_wider} function:

\begin{Shaded}
\begin{Highlighting}[]
  \FunctionTok{put\_log}\NormalTok{(}\StringTok{"Function: \textasciigrave{}make\_source\_datasets\textasciigrave{}: Creating Rating Matrix from \textasciigrave{}edx\textasciigrave{} dataset..."}\NormalTok{)}
\NormalTok{  edx.mx }\OtherTok{\textless{}{-}}\NormalTok{ edx }\SpecialCharTok{|\textgreater{}} 
    \FunctionTok{mutate}\NormalTok{(}\AttributeTok{userId =} \FunctionTok{factor}\NormalTok{(userId),}
           \AttributeTok{movieId =} \FunctionTok{factor}\NormalTok{(movieId)) }\SpecialCharTok{|\textgreater{}}
    \FunctionTok{select}\NormalTok{(movieId, userId, rating) }\SpecialCharTok{|\textgreater{}}
    \FunctionTok{pivot\_wider}\NormalTok{(}\AttributeTok{names\_from =}\NormalTok{ movieId, }\AttributeTok{values\_from =}\NormalTok{ rating) }\SpecialCharTok{|\textgreater{}}
    \FunctionTok{column\_to\_rownames}\NormalTok{(}\StringTok{"userId"}\NormalTok{) }\SpecialCharTok{|\textgreater{}}
    \FunctionTok{as.matrix}\NormalTok{()}
  
  \FunctionTok{put\_log}\NormalTok{(}\StringTok{"Function: \textasciigrave{}make\_source\_datasets\textasciigrave{}:}
\StringTok{Matrix created: \textasciigrave{}edx.mx\textasciigrave{} of the following dimentions:"}\NormalTok{)}
\end{Highlighting}
\end{Shaded}

\begin{Shaded}
\begin{Highlighting}[]
  \FunctionTok{str}\NormalTok{(edx.mx)}
\end{Highlighting}
\end{Shaded}

\begin{verbatim}
##  num [1:69878, 1:10677] 5 NA NA NA NA NA NA NA NA NA ...
##  - attr(*, "dimnames")=List of 2
##   ..$ : chr [1:69878] "1" "2" "3" "4" ...
##   ..$ : chr [1:10677] "122" "185" "292" "316" ...
\end{verbatim}

\paragraph{\texorpdfstring{\texttt{edx.sgr} Object}{edx.sgr Object}}\label{edx.sgr-object}

\hfill\break

To account for the Movie Genre Effect more accurately, we need a dataset with split rows for movies belonging to multiple genres:

\begin{Shaded}
\begin{Highlighting}[]
  \FunctionTok{put\_log}\NormalTok{(}\StringTok{"Function: \textasciigrave{}make\_source\_datasets\textasciigrave{}: }
\StringTok{To account for the Movie Genre Effect, we need a dataset with split rows }
\StringTok{for movies belonging to multiple genres."}\NormalTok{)}
\NormalTok{  edx.sgr }\OtherTok{\textless{}{-}} \FunctionTok{splitGenreRows}\NormalTok{(edx)}
\end{Highlighting}
\end{Shaded}

\begin{Shaded}
\begin{Highlighting}[]
\FunctionTok{str}\NormalTok{(edx.sgr)}
\end{Highlighting}
\end{Shaded}

\begin{verbatim}
## tibble [23,371,423 x 6] (S3: tbl_df/tbl/data.frame)
##  $ userId   : int [1:23371423] 1 1 1 1 1 1 1 1 1 1 ...
##  $ movieId  : int [1:23371423] 122 122 185 185 185 292 292 292 292 316 ...
##  $ rating   : num [1:23371423] 5 5 5 5 5 5 5 5 5 5 ...
##  $ timestamp: int [1:23371423] 838985046 838985046 838983525 838983525 838983525 838983421 838983421 838983421 838983421 838983392 ...
##  $ title    : chr [1:23371423] "Boomerang (1992)" "Boomerang (1992)" "Net, The (1995)" "Net, The (1995)" ...
##  $ genres   : chr [1:23371423] "Comedy" "Romance" "Action" "Crime" ...
\end{verbatim}

\begin{Shaded}
\begin{Highlighting}[]
\FunctionTok{summary}\NormalTok{(edx.sgr)}
\end{Highlighting}
\end{Shaded}

\begin{verbatim}
##      userId         movieId          rating        timestamp            title              genres         
##  Min.   :    1   Min.   :    1   Min.   :0.500   Min.   :7.897e+08   Length:23371423    Length:23371423   
##  1st Qu.:18140   1st Qu.:  616   1st Qu.:3.000   1st Qu.:9.472e+08   Class :character   Class :character  
##  Median :35784   Median : 1748   Median :4.000   Median :1.042e+09   Mode  :character   Mode  :character  
##  Mean   :35886   Mean   : 4277   Mean   :3.527   Mean   :1.035e+09                                        
##  3rd Qu.:53638   3rd Qu.: 3635   3rd Qu.:4.000   3rd Qu.:1.131e+09                                        
##  Max.   :71567   Max.   :65133   Max.   :5.000   Max.   :1.231e+09
\end{verbatim}

Note that we use the \texttt{splitGenreRows} function to split rows of the original dataset:

\begin{Shaded}
\begin{Highlighting}[]
\NormalTok{splitGenreRows }\OtherTok{\textless{}{-}} \ControlFlowTok{function}\NormalTok{(data)\{}
  \FunctionTok{put}\NormalTok{(}\StringTok{"Splitting dataset rows related to multiple genres..."}\NormalTok{)}
\NormalTok{  start }\OtherTok{\textless{}{-}} \FunctionTok{put\_start\_date}\NormalTok{()}
\NormalTok{  gs\_splitted }\OtherTok{\textless{}{-}}\NormalTok{ data }\SpecialCharTok{|\textgreater{}}
    \FunctionTok{separate\_rows}\NormalTok{(genres, }\AttributeTok{sep =} \StringTok{"}\SpecialCharTok{\textbackslash{}\textbackslash{}}\StringTok{|"}\NormalTok{)}
  \FunctionTok{put}\NormalTok{(}\StringTok{"Dataset rows related to multiple genres have been splitted to have single genre per row."}\NormalTok{)}
  \FunctionTok{put\_end\_date}\NormalTok{(start)}
\NormalTok{  gs\_splitted}
\NormalTok{\}}
\end{Highlighting}
\end{Shaded}

\begin{noteblock}
The source code of the function mentioned above is also available in the \href{https://github.com/AzKurban-edX-DS/Capstone-MovieLens/blob/main/r/src/support-functions/data.helper.functions.R\#L86}{Initialize input datasets} section of the \href{https://github.com/AzKurban-edX-DS/Capstone-MovieLens/blob/main/r/src/support-functions/data.helper.functions.R}{data.helper.functions.R} script on \emph{GitHub}.

\end{noteblock}

\paragraph{\texorpdfstring{\texttt{movie\_map} Object}{movie\_map Object}}\label{movie_map-object}

\hfill\break

To be able to map movie IDs to titles we create the following lookup table:

\begin{Shaded}
\begin{Highlighting}[]
\NormalTok{movie\_map }\OtherTok{\textless{}{-}}\NormalTok{ edx }\SpecialCharTok{|\textgreater{}} \FunctionTok{select}\NormalTok{(movieId, title, genres) }\SpecialCharTok{|\textgreater{}} 
    \FunctionTok{distinct}\NormalTok{(movieId, }\AttributeTok{.keep\_all =} \ConstantTok{TRUE}\NormalTok{)}
  
  \FunctionTok{put\_log}\NormalTok{(}\StringTok{"Function: \textasciigrave{}make\_source\_datasets\textasciigrave{}: Dataset created: movie\_map"}\NormalTok{)}
\end{Highlighting}
\end{Shaded}

\begin{Shaded}
\begin{Highlighting}[]
\FunctionTok{str}\NormalTok{(movie\_map)}
\end{Highlighting}
\end{Shaded}

\begin{verbatim}
## 'data.frame':    10677 obs. of  3 variables:
##  $ movieId: int  122 185 292 316 329 355 356 362 364 370 ...
##  $ title  : chr  "Boomerang (1992)" "Net, The (1995)" "Outbreak (1995)" "Stargate (1994)" ...
##  $ genres : chr  "Comedy|Romance" "Action|Crime|Thriller" "Action|Drama|Sci-Fi|Thriller" "Action|Adventure|Sci-Fi" ...
\end{verbatim}

\begin{Shaded}
\begin{Highlighting}[]
\FunctionTok{summary}\NormalTok{(movie\_map)}
\end{Highlighting}
\end{Shaded}

\begin{verbatim}
##     movieId         title              genres         
##  Min.   :    1   Length:10677       Length:10677      
##  1st Qu.: 2754   Class :character   Class :character  
##  Median : 5434   Mode  :character   Mode  :character  
##  Mean   :13105                                        
##  3rd Qu.: 8710                                        
##  Max.   :65133
\end{verbatim}

Note that titles cannot be considered unique, so we can't use them as IDs\autocite{IDS2_23-1-1}.

\paragraph{\texorpdfstring{\texttt{date\_days\_map} Object}{date\_days\_map Object}}\label{date_days_map-object}

\hfill\break

We have a \texttt{timestamp} field in the \texttt{edx} dataset.
To be able to map the date, year, and number of days since the earliest record in the \texttt{edx} dataset with the corresponding value in this field, we create the following lookup table:

\begin{Shaded}
\begin{Highlighting}[]
  \FunctionTok{put\_log}\NormalTok{(}\StringTok{"Function: \textasciigrave{}make\_source\_datasets\textasciigrave{}: Creating Date{-}Days Map dataset..."}\NormalTok{)}
\NormalTok{  date\_days\_map }\OtherTok{\textless{}{-}}\NormalTok{ edx }\SpecialCharTok{|\textgreater{}}
    \FunctionTok{mutate}\NormalTok{(}\AttributeTok{date\_time =} \FunctionTok{as\_datetime}\NormalTok{(timestamp)) }\SpecialCharTok{|\textgreater{}}
    \FunctionTok{mutate}\NormalTok{(}\AttributeTok{date =} \FunctionTok{as\_date}\NormalTok{(date\_time)) }\SpecialCharTok{|\textgreater{}}
    \FunctionTok{mutate}\NormalTok{(}\AttributeTok{year =} \FunctionTok{year}\NormalTok{(date\_time)) }\SpecialCharTok{|\textgreater{}}
    \FunctionTok{mutate}\NormalTok{(}\AttributeTok{days =} \FunctionTok{as.integer}\NormalTok{(date }\SpecialCharTok{{-}} \FunctionTok{min}\NormalTok{(date))) }\SpecialCharTok{|\textgreater{}}
    \FunctionTok{select}\NormalTok{(timestamp, date\_time, date, year, days) }\SpecialCharTok{|\textgreater{}}
    \FunctionTok{distinct}\NormalTok{(timestamp, }\AttributeTok{.keep\_all =} \ConstantTok{TRUE}\NormalTok{)}
  
  \FunctionTok{put\_log}\NormalTok{(}\StringTok{"Function: \textasciigrave{}make\_source\_datasets\textasciigrave{}: Dataset created: date\_days\_map"}\NormalTok{)}
\end{Highlighting}
\end{Shaded}

\begin{Shaded}
\begin{Highlighting}[]
  \FunctionTok{str}\NormalTok{(date\_days\_map)}
\end{Highlighting}
\end{Shaded}

\begin{verbatim}
## 'data.frame':    6519590 obs. of  5 variables:
##  $ timestamp: int  838985046 838983525 838983421 838983392 838984474 838983653 838984885 838983707 838984596 838983834 ...
##  $ date_time: POSIXct, format: "1996-08-02 11:24:06" "1996-08-02 10:58:45" "1996-08-02 10:57:01" "1996-08-02 10:56:32" ...
##  $ date     : Date, format: "1996-08-02" "1996-08-02" "1996-08-02" "1996-08-02" ...
##  $ year     : num  1996 1996 1996 1996 1996 ...
##  $ days     : int  571 571 571 571 571 571 571 571 571 571 ...
\end{verbatim}

\begin{Shaded}
\begin{Highlighting}[]
  \FunctionTok{summary}\NormalTok{(date\_days\_map)}
\end{Highlighting}
\end{Shaded}

\begin{verbatim}
##    timestamp           date_time                           date                 year           days     
##  Min.   :7.897e+08   Min.   :1995-01-09 11:46:49.00   Min.   :1995-01-09   Min.   :1995   Min.   :   0  
##  1st Qu.:9.783e+08   1st Qu.:2001-01-01 05:05:01.75   1st Qu.:2001-01-01   1st Qu.:2001   1st Qu.:2184  
##  Median :1.091e+09   Median :2004-08-03 01:08:18.50   Median :2004-08-03   Median :2004   Median :3494  
##  Mean   :1.066e+09   Mean   :2003-10-10 23:15:02.07   Mean   :2003-10-10   Mean   :2003   Mean   :3196  
##  3rd Qu.:1.152e+09   3rd Qu.:2006-07-04 20:41:57.50   3rd Qu.:2006-07-04   3rd Qu.:2006   3rd Qu.:4194  
##  Max.   :1.231e+09   Max.   :2009-01-05 05:02:16.00   Max.   :2009-01-05   Max.   :2009   Max.   :5110
\end{verbatim}

\paragraph{\texorpdfstring{\texttt{edx\_CV} Object}{edx\_CV Object}}\label{edx_cv-object}

\hfill\break

Here we have a list of sample objects we need to perform the \emph{5-Fold Cross Validation}
as explained in \href{https://rafalab.dfci.harvard.edu/dsbook-part-2/ml/resampling-methods.html\#k-fold-cross-validation}{Section 29.6.1 K-fold cross validation} of the \emph{Course Textbook}:

\begin{Shaded}
\begin{Highlighting}[]
\NormalTok{  start }\OtherTok{\textless{}{-}} \FunctionTok{put\_start\_date}\NormalTok{()}
\NormalTok{  edx\_CV }\OtherTok{\textless{}{-}} \FunctionTok{lapply}\NormalTok{(kfold\_index,  }\ControlFlowTok{function}\NormalTok{(fold\_i)\{}
    
    \FunctionTok{put\_log1}\NormalTok{(}\StringTok{"Method \textasciigrave{}make\_source\_datasets\textasciigrave{}: }
\StringTok{Creating K{-}Fold Cross Validation Datasets, Fold \%1"}\NormalTok{, fold\_i)}
    
    \CommentTok{\#\textgreater{} We split the initial datasets into training sets, which we will use to build }
    \CommentTok{\#\textgreater{} and train our models, and validation sets in which we will compute the accuracy }
    \CommentTok{\#\textgreater{} of our predictions, the way described in the \textasciigrave{}Section 23.1.1 Movielens data\textasciigrave{}}
    \CommentTok{\#\textgreater{} (https://rafalab.dfci.harvard.edu/dsbook{-}part{-}2/highdim/regularization.html\#movielens{-}data) }
    \CommentTok{\#\textgreater{} of the Course Textbook.}
    
\NormalTok{    split\_sets }\OtherTok{\textless{}{-}}\NormalTok{ edx }\SpecialCharTok{|\textgreater{}}
      \FunctionTok{sample\_train\_validation\_sets}\NormalTok{(fold\_i}\SpecialCharTok{*}\DecValTok{1000}\NormalTok{)}
    
\NormalTok{    train\_set }\OtherTok{\textless{}{-}}\NormalTok{ split\_sets}\SpecialCharTok{$}\NormalTok{train\_set}
\NormalTok{    validation\_set }\OtherTok{\textless{}{-}}\NormalTok{ split\_sets}\SpecialCharTok{$}\NormalTok{validation\_set}
    
    \FunctionTok{put\_log}\NormalTok{(}\StringTok{"Function: \textasciigrave{}make\_source\_datasets\textasciigrave{}: }
\StringTok{Sampling 20\% from the split{-}row version of the \textasciigrave{}edx\textasciigrave{} dataset..."}\NormalTok{)}
\NormalTok{    split\_sets.gs }\OtherTok{\textless{}{-}}\NormalTok{ edx.sgr }\SpecialCharTok{|\textgreater{}}
      \FunctionTok{sample\_train\_validation\_sets}\NormalTok{(fold\_i}\SpecialCharTok{*}\DecValTok{2000}\NormalTok{)}

\NormalTok{    train.sgr }\OtherTok{\textless{}{-}}\NormalTok{ split\_sets.gs}\SpecialCharTok{$}\NormalTok{train\_set}
\NormalTok{    validation.sgr }\OtherTok{\textless{}{-}}\NormalTok{ split\_sets.gs}\SpecialCharTok{$}\NormalTok{validation\_set}
    
    \CommentTok{\# put\_log("Function: \textasciigrave{}make\_source\_datasets\textasciigrave{}: Dataset created: validation.sgr")}
    \CommentTok{\# put(summary(validation.sgr))}
    
    \CommentTok{\#\textgreater{} We will use the array representation described in \textasciigrave{}Section 17.5 of the Textbook\textasciigrave{}}
    \CommentTok{\#\textgreater{} (https://rafalab.dfci.harvard.edu/dsbook{-}part{-}2/linear{-}models/treatment{-}effect{-}models.html\#sec{-}anova), }
    \CommentTok{\#\textgreater{} for the training data. }
    \CommentTok{\#\textgreater{} To create this matrix, we use \textasciigrave{}tidyr::pivot\_wider\textasciigrave{} function:}
    
    \FunctionTok{put\_log}\NormalTok{(}\StringTok{"Function: \textasciigrave{}make\_source\_datasets\textasciigrave{}: Creating Rating Matrix from Train Set..."}\NormalTok{)}
\NormalTok{    train\_mx }\OtherTok{\textless{}{-}}\NormalTok{ train\_set }\SpecialCharTok{|\textgreater{}} 
      \FunctionTok{mutate}\NormalTok{(}\AttributeTok{userId =} \FunctionTok{factor}\NormalTok{(userId),}
             \AttributeTok{movieId =} \FunctionTok{factor}\NormalTok{(movieId)) }\SpecialCharTok{|\textgreater{}}
      \FunctionTok{select}\NormalTok{(movieId, userId, rating) }\SpecialCharTok{|\textgreater{}}
      \FunctionTok{pivot\_wider}\NormalTok{(}\AttributeTok{names\_from =}\NormalTok{ movieId, }\AttributeTok{values\_from =}\NormalTok{ rating) }\SpecialCharTok{|\textgreater{}}
      \FunctionTok{column\_to\_rownames}\NormalTok{(}\StringTok{"userId"}\NormalTok{) }\SpecialCharTok{|\textgreater{}}
      \FunctionTok{as.matrix}\NormalTok{()}
    
    \FunctionTok{put\_log}\NormalTok{(}\StringTok{"Function: \textasciigrave{}make\_source\_datasets\textasciigrave{}:}
\StringTok{Matrix created: \textasciigrave{}train\_mx\textasciigrave{} of the following dimentions:"}\NormalTok{)}
    \FunctionTok{put}\NormalTok{(}\FunctionTok{dim}\NormalTok{(train\_mx))}

    \FunctionTok{list}\NormalTok{(}\AttributeTok{train\_set =}\NormalTok{ train\_set,}
         \AttributeTok{train\_mx =}\NormalTok{ train\_mx,}
         \AttributeTok{train.sgr =}\NormalTok{ train.sgr,}
         \AttributeTok{validation\_set =}\NormalTok{ validation\_set)}
\NormalTok{  \})}
  \FunctionTok{put\_end\_date}\NormalTok{(start)}
  \FunctionTok{put\_log}\NormalTok{(}\StringTok{"Function: \textasciigrave{}make\_source\_datasets\textasciigrave{}: }
\StringTok{Set of K{-}Fold Cross Validation datasets created: edx\_CV"}\NormalTok{)}
\end{Highlighting}
\end{Shaded}

\begin{Shaded}
\begin{Highlighting}[]
\FunctionTok{str}\NormalTok{(edx\_CV)}
\end{Highlighting}
\end{Shaded}

\begin{verbatim}
## List of 5
##  $ :List of 4
##   ..$ train_set     :'data.frame':   7172311 obs. of  6 variables:
##   .. ..$ userId   : int [1:7172311] 1 1 1 1 1 1 1 1 1 1 ...
##   .. ..$ movieId  : int [1:7172311] 122 185 292 329 356 362 364 370 420 466 ...
##   .. ..$ rating   : num [1:7172311] 5 5 5 5 5 5 5 5 5 5 ...
##   .. ..$ timestamp: int [1:7172311] 838985046 838983525 838983421 838983392 838983653 838984885 838983707 838984596 838983834 838984679 ...
##   .. ..$ title    : chr [1:7172311] "Boomerang (1992)" "Net, The (1995)" "Outbreak (1995)" "Star Trek: Generations (1994)" ...
##   .. ..$ genres   : chr [1:7172311] "Comedy|Romance" "Action|Crime|Thriller" "Action|Drama|Sci-Fi|Thriller" "Action|Adventure|Drama|Sci-Fi" ...
##   ..$ train_mx      : num [1:69878, 1:10677] 5 NA NA NA NA NA NA NA NA NA ...
##   .. ..- attr(*, "dimnames")=List of 2
##   .. .. ..$ : chr [1:69878] "1" "2" "3" "4" ...
##   .. .. ..$ : chr [1:10677] "122" "185" "292" "329" ...
##   ..$ train.sgr     : tibble [18,669,190 x 6] (S3: tbl_df/tbl/data.frame)
##   .. ..$ userId   : int [1:18669190] 1 1 1 1 1 1 1 1 1 1 ...
##   .. ..$ movieId  : int [1:18669190] 122 122 185 185 292 292 292 292 316 316 ...
##   .. ..$ rating   : num [1:18669190] 5 5 5 5 5 5 5 5 5 5 ...
##   .. ..$ timestamp: int [1:18669190] 838985046 838985046 838983525 838983525 838983421 838983421 838983421 838983421 838983392 838983392 ...
##   .. ..$ title    : chr [1:18669190] "Boomerang (1992)" "Boomerang (1992)" "Net, The (1995)" "Net, The (1995)" ...
##   .. ..$ genres   : chr [1:18669190] "Comedy" "Romance" "Action" "Crime" ...
##   ..$ validation_set:'data.frame':   1827744 obs. of  6 variables:
##   .. ..$ userId   : int [1:1827744] 1 1 1 1 2 2 2 2 3 3 ...
##   .. ..$ movieId  : int [1:1827744] 316 355 377 588 260 376 648 1049 110 1252 ...
##   .. ..$ rating   : num [1:1827744] 5 5 5 5 5 3 2 3 4.5 4 ...
##   .. ..$ timestamp: int [1:1827744] 838983392 838984474 838983834 838983339 868244562 868245920 868244699 868245920 1136075500 1133571071 ...
##   .. ..$ title    : chr [1:1827744] "Stargate (1994)" "Flintstones, The (1994)" "Speed (1994)" "Aladdin (1992)" ...
##   .. ..$ genres   : chr [1:1827744] "Action|Adventure|Sci-Fi" "Children|Comedy|Fantasy" "Action|Romance|Thriller" "Adventure|Animation|Children|Comedy|Musical" ...
##  $ :List of 4
##   ..$ train_set     :'data.frame':   7172306 obs. of  6 variables:
##   .. ..$ userId   : int [1:7172306] 1 1 1 1 1 1 1 1 1 1 ...
##   .. ..$ movieId  : int [1:7172306] 122 185 292 316 329 355 356 364 370 377 ...
##   .. ..$ rating   : num [1:7172306] 5 5 5 5 5 5 5 5 5 5 ...
##   .. ..$ timestamp: int [1:7172306] 838985046 838983525 838983421 838983392 838983392 838984474 838983653 838983707 838984596 838983834 ...
##   .. ..$ title    : chr [1:7172306] "Boomerang (1992)" "Net, The (1995)" "Outbreak (1995)" "Stargate (1994)" ...
##   .. ..$ genres   : chr [1:7172306] "Comedy|Romance" "Action|Crime|Thriller" "Action|Drama|Sci-Fi|Thriller" "Action|Adventure|Sci-Fi" ...
##   ..$ train_mx      : num [1:69878, 1:10677] 5 NA NA NA NA NA NA NA NA NA ...
##   .. ..- attr(*, "dimnames")=List of 2
##   .. .. ..$ : chr [1:69878] "1" "2" "3" "4" ...
##   .. .. ..$ : chr [1:10677] "122" "185" "292" "316" ...
##   ..$ train.sgr     : tibble [18,669,201 x 6] (S3: tbl_df/tbl/data.frame)
##   .. ..$ userId   : int [1:18669201] 1 1 1 1 1 1 1 1 1 1 ...
##   .. ..$ movieId  : int [1:18669201] 122 122 185 185 185 292 292 316 316 329 ...
##   .. ..$ rating   : num [1:18669201] 5 5 5 5 5 5 5 5 5 5 ...
##   .. ..$ timestamp: int [1:18669201] 838985046 838985046 838983525 838983525 838983525 838983421 838983421 838983392 838983392 838983392 ...
##   .. ..$ title    : chr [1:18669201] "Boomerang (1992)" "Boomerang (1992)" "Net, The (1995)" "Net, The (1995)" ...
##   .. ..$ genres   : chr [1:18669201] "Comedy" "Romance" "Action" "Crime" ...
##   ..$ validation_set:'data.frame':   1827749 obs. of  6 variables:
##   .. ..$ userId   : int [1:1827749] 1 1 1 1 2 2 2 2 3 3 ...
##   .. ..$ movieId  : int [1:1827749] 362 520 539 594 539 590 733 1210 1252 1408 ...
##   .. ..$ rating   : num [1:1827749] 5 5 5 5 3 5 3 4 4 3.5 ...
##   .. ..$ timestamp: int [1:1827749] 838984885 838984679 838984068 838984679 868246262 868245608 868244562 868245644 1133571071 1133571145 ...
##   .. ..$ title    : chr [1:1827749] "Jungle Book, The (1994)" "Robin Hood: Men in Tights (1993)" "Sleepless in Seattle (1993)" "Snow White and the Seven Dwarfs (1937)" ...
##   .. ..$ genres   : chr [1:1827749] "Adventure|Children|Romance" "Comedy" "Comedy|Drama|Romance" "Animation|Children|Drama|Fantasy|Musical" ...
##  $ :List of 4
##   ..$ train_set     :'data.frame':   7172307 obs. of  6 variables:
##   .. ..$ userId   : int [1:7172307] 1 1 1 1 1 1 1 1 1 1 ...
##   .. ..$ movieId  : int [1:7172307] 122 185 292 316 329 355 362 370 377 420 ...
##   .. ..$ rating   : num [1:7172307] 5 5 5 5 5 5 5 5 5 5 ...
##   .. ..$ timestamp: int [1:7172307] 838985046 838983525 838983421 838983392 838983392 838984474 838984885 838984596 838983834 838983834 ...
##   .. ..$ title    : chr [1:7172307] "Boomerang (1992)" "Net, The (1995)" "Outbreak (1995)" "Stargate (1994)" ...
##   .. ..$ genres   : chr [1:7172307] "Comedy|Romance" "Action|Crime|Thriller" "Action|Drama|Sci-Fi|Thriller" "Action|Adventure|Sci-Fi" ...
##   ..$ train_mx      : num [1:69878, 1:10677] 5 NA NA NA NA NA NA NA NA NA ...
##   .. ..- attr(*, "dimnames")=List of 2
##   .. .. ..$ : chr [1:69878] "1" "2" "3" "4" ...
##   .. .. ..$ : chr [1:10677] "122" "185" "292" "316" ...
##   ..$ train.sgr     : tibble [18,669,195 x 6] (S3: tbl_df/tbl/data.frame)
##   .. ..$ userId   : int [1:18669195] 1 1 1 1 1 1 1 1 1 1 ...
##   .. ..$ movieId  : int [1:18669195] 122 122 185 185 185 292 292 292 316 329 ...
##   .. ..$ rating   : num [1:18669195] 5 5 5 5 5 5 5 5 5 5 ...
##   .. ..$ timestamp: int [1:18669195] 838985046 838985046 838983525 838983525 838983525 838983421 838983421 838983421 838983392 838983392 ...
##   .. ..$ title    : chr [1:18669195] "Boomerang (1992)" "Boomerang (1992)" "Net, The (1995)" "Net, The (1995)" ...
##   .. ..$ genres   : chr [1:18669195] "Comedy" "Romance" "Action" "Crime" ...
##   ..$ validation_set:'data.frame':   1827748 obs. of  6 variables:
##   .. ..$ userId   : int [1:1827748] 1 1 1 1 2 2 2 2 3 3 ...
##   .. ..$ movieId  : int [1:1827748] 356 364 539 616 590 719 780 786 151 213 ...
##   .. ..$ rating   : num [1:1827748] 5 5 5 5 5 3 3 3 4.5 5 ...
##   .. ..$ timestamp: int [1:1827748] 838983653 838983707 838984068 838984941 868245608 868246191 868244698 868244562 1133571026 1136075789 ...
##   .. ..$ title    : chr [1:1827748] "Forrest Gump (1994)" "Lion King, The (1994)" "Sleepless in Seattle (1993)" "Aristocats, The (1970)" ...
##   .. ..$ genres   : chr [1:1827748] "Comedy|Drama|Romance|War" "Adventure|Animation|Children|Drama|Musical" "Comedy|Drama|Romance" "Animation|Children" ...
##  $ :List of 4
##   ..$ train_set     :'data.frame':   7172311 obs. of  6 variables:
##   .. ..$ userId   : int [1:7172311] 1 1 1 1 1 1 1 1 1 1 ...
##   .. ..$ movieId  : int [1:7172311] 122 185 292 316 329 355 356 362 364 370 ...
##   .. ..$ rating   : num [1:7172311] 5 5 5 5 5 5 5 5 5 5 ...
##   .. ..$ timestamp: int [1:7172311] 838985046 838983525 838983421 838983392 838983392 838984474 838983653 838984885 838983707 838984596 ...
##   .. ..$ title    : chr [1:7172311] "Boomerang (1992)" "Net, The (1995)" "Outbreak (1995)" "Stargate (1994)" ...
##   .. ..$ genres   : chr [1:7172311] "Comedy|Romance" "Action|Crime|Thriller" "Action|Drama|Sci-Fi|Thriller" "Action|Adventure|Sci-Fi" ...
##   ..$ train_mx      : num [1:69878, 1:10677] 5 NA NA NA NA NA NA NA NA NA ...
##   .. ..- attr(*, "dimnames")=List of 2
##   .. .. ..$ : chr [1:69878] "1" "2" "3" "4" ...
##   .. .. ..$ : chr [1:10677] "122" "185" "292" "316" ...
##   ..$ train.sgr     : tibble [18,669,192 x 6] (S3: tbl_df/tbl/data.frame)
##   .. ..$ userId   : int [1:18669192] 1 1 1 1 1 1 1 1 1 1 ...
##   .. ..$ movieId  : int [1:18669192] 122 122 185 185 292 292 316 316 329 329 ...
##   .. ..$ rating   : num [1:18669192] 5 5 5 5 5 5 5 5 5 5 ...
##   .. ..$ timestamp: int [1:18669192] 838985046 838985046 838983525 838983525 838983421 838983421 838983392 838983392 838983392 838983392 ...
##   .. ..$ title    : chr [1:18669192] "Boomerang (1992)" "Boomerang (1992)" "Net, The (1995)" "Net, The (1995)" ...
##   .. ..$ genres   : chr [1:18669192] "Comedy" "Romance" "Action" "Thriller" ...
##   ..$ validation_set:'data.frame':   1827744 obs. of  6 variables:
##   .. ..$ userId   : int [1:1827744] 1 1 1 1 2 2 2 2 3 3 ...
##   .. ..$ movieId  : int [1:1827744] 377 520 588 616 110 648 1049 1356 1148 1276 ...
##   .. ..$ rating   : num [1:1827744] 5 5 5 5 5 2 3 3 4 3.5 ...
##   .. ..$ timestamp: int [1:1827744] 838983834 838984679 838983339 838984941 868245777 868244699 868245920 868244603 1133571121 1133571205 ...
##   .. ..$ title    : chr [1:1827744] "Speed (1994)" "Robin Hood: Men in Tights (1993)" "Aladdin (1992)" "Aristocats, The (1970)" ...
##   .. ..$ genres   : chr [1:1827744] "Action|Romance|Thriller" "Comedy" "Adventure|Animation|Children|Comedy|Musical" "Animation|Children" ...
##  $ :List of 4
##   ..$ train_set     :'data.frame':   7172301 obs. of  6 variables:
##   .. ..$ userId   : int [1:7172301] 1 1 1 1 1 1 1 1 1 1 ...
##   .. ..$ movieId  : int [1:7172301] 122 185 292 316 355 356 364 370 420 466 ...
##   .. ..$ rating   : num [1:7172301] 5 5 5 5 5 5 5 5 5 5 ...
##   .. ..$ timestamp: int [1:7172301] 838985046 838983525 838983421 838983392 838984474 838983653 838983707 838984596 838983834 838984679 ...
##   .. ..$ title    : chr [1:7172301] "Boomerang (1992)" "Net, The (1995)" "Outbreak (1995)" "Stargate (1994)" ...
##   .. ..$ genres   : chr [1:7172301] "Comedy|Romance" "Action|Crime|Thriller" "Action|Drama|Sci-Fi|Thriller" "Action|Adventure|Sci-Fi" ...
##   ..$ train_mx      : num [1:69878, 1:10677] 5 NA NA NA NA NA NA NA NA NA ...
##   .. ..- attr(*, "dimnames")=List of 2
##   .. .. ..$ : chr [1:69878] "1" "2" "3" "4" ...
##   .. .. ..$ : chr [1:10677] "122" "185" "292" "316" ...
##   ..$ train.sgr     : tibble [18,669,194 x 6] (S3: tbl_df/tbl/data.frame)
##   .. ..$ userId   : int [1:18669194] 1 1 1 1 1 1 1 1 1 1 ...
##   .. ..$ movieId  : int [1:18669194] 122 122 185 185 292 292 316 329 329 355 ...
##   .. ..$ rating   : num [1:18669194] 5 5 5 5 5 5 5 5 5 5 ...
##   .. ..$ timestamp: int [1:18669194] 838985046 838985046 838983525 838983525 838983421 838983421 838983392 838983392 838983392 838984474 ...
##   .. ..$ title    : chr [1:18669194] "Boomerang (1992)" "Boomerang (1992)" "Net, The (1995)" "Net, The (1995)" ...
##   .. ..$ genres   : chr [1:18669194] "Comedy" "Romance" "Crime" "Thriller" ...
##   ..$ validation_set:'data.frame':   1827754 obs. of  6 variables:
##   .. ..$ userId   : int [1:1827754] 1 1 1 1 2 2 2 2 3 3 ...
##   .. ..$ movieId  : int [1:1827754] 329 362 377 594 110 376 539 736 1252 1408 ...
##   .. ..$ rating   : num [1:1827754] 5 5 5 5 5 3 3 3 4 3.5 ...
##   .. ..$ timestamp: int [1:1827754] 838983392 838984885 838983834 838984679 868245777 868245920 868246262 868244698 1133571071 1133571145 ...
##   .. ..$ title    : chr [1:1827754] "Star Trek: Generations (1994)" "Jungle Book, The (1994)" "Speed (1994)" "Snow White and the Seven Dwarfs (1937)" ...
##   .. ..$ genres   : chr [1:1827754] "Action|Adventure|Drama|Sci-Fi" "Adventure|Children|Romance" "Action|Romance|Thriller" "Animation|Children|Drama|Fantasy|Musical" ...
\end{verbatim}

\begin{noteblock}
This code snippet is a \href{https://github.com/AzKurban-edX-DS/Capstone-MovieLens/blob/main/r/src/support-functions/data.helper.functions.R\#L140}{part} of the \hyperref[func.make_source_datasets]{make\_source\_datasets} \emph{function} code described above.

\end{noteblock}

Note that we used the \href{https://github.com/AzKurban-edX-DS/Capstone-MovieLens/blob/main/r/src/support-functions/data.helper.functions.R\#L15}{sample\_train\_validation\_sets} function call to split the original dataset (\texttt{edx} in this case):

\begin{Shaded}
\begin{Highlighting}[]
\NormalTok{    split\_sets }\OtherTok{\textless{}{-}}\NormalTok{ edx }\SpecialCharTok{|\textgreater{}}
      \FunctionTok{sample\_train\_validation\_sets}\NormalTok{(fold\_i}\SpecialCharTok{*}\DecValTok{1000}\NormalTok{)}
\end{Highlighting}
\end{Shaded}

which returns a pair of train/validation sets:

\begin{Shaded}
\begin{Highlighting}[]
\NormalTok{sample\_train\_validation\_sets }\OtherTok{\textless{}{-}} \ControlFlowTok{function}\NormalTok{(data, seed)\{}
  \FunctionTok{put\_log}\NormalTok{(}\StringTok{"Function: \textasciigrave{}sample\_train\_validation\_sets\textasciigrave{}: Sampling 20\% of the \textasciigrave{}data\textasciigrave{} data..."}\NormalTok{)}
  \FunctionTok{set.seed}\NormalTok{(seed)}
\NormalTok{  validation\_ind }\OtherTok{\textless{}{-}} 
    \FunctionTok{sapply}\NormalTok{(}\FunctionTok{splitByUser}\NormalTok{(data),}
           \ControlFlowTok{function}\NormalTok{(i) }\FunctionTok{sample}\NormalTok{(i, }\FunctionTok{ceiling}\NormalTok{(}\FunctionTok{length}\NormalTok{(i)}\SpecialCharTok{*}\NormalTok{.}\DecValTok{2}\NormalTok{))) }\SpecialCharTok{|\textgreater{}} 
    \FunctionTok{unlist}\NormalTok{() }\SpecialCharTok{|\textgreater{}} 
    \FunctionTok{sort}\NormalTok{()}
  
  \FunctionTok{put\_log}\NormalTok{(}\StringTok{"Function: \textasciigrave{}sample\_train\_validation\_sets\textasciigrave{}: }
\StringTok{Extracting 80\% of the original \textasciigrave{}data\textasciigrave{} not used for the Validation Set, }
\StringTok{excluding data for users who provided no more than a specified number of ratings: \{min\_nratings\}."}\NormalTok{)}
  
\NormalTok{  train\_set }\OtherTok{\textless{}{-}}\NormalTok{ data[}\SpecialCharTok{{-}}\NormalTok{validation\_ind,]}
  
  \FunctionTok{put\_log}\NormalTok{(}\StringTok{"Function: \textasciigrave{}sample\_train\_validation\_sets\textasciigrave{}: Dataset created: train\_set"}\NormalTok{)}
  \FunctionTok{put}\NormalTok{(}\FunctionTok{summary}\NormalTok{(train\_set))}
  
  \FunctionTok{put\_log}\NormalTok{(}\StringTok{"Function: \textasciigrave{}sample\_train\_validation\_sets\textasciigrave{}: }
\StringTok{To make sure we don’t include movies in the Training Set that should not be there, }
\StringTok{we exclude entries using the semi\_join function from the Validation Set."}\NormalTok{)}
\NormalTok{  tmp.data }\OtherTok{\textless{}{-}}\NormalTok{ data[validation\_ind,]}
  
\NormalTok{  validation\_set }\OtherTok{\textless{}{-}}\NormalTok{ tmp.data }\SpecialCharTok{|\textgreater{}} 
    \FunctionTok{semi\_join}\NormalTok{(train\_set, }\AttributeTok{by =} \StringTok{"movieId"}\NormalTok{) }\SpecialCharTok{|\textgreater{}} 
    \FunctionTok{semi\_join}\NormalTok{(train\_set, }\AttributeTok{by =} \StringTok{"userId"}\NormalTok{) }\SpecialCharTok{|\textgreater{}}
    \FunctionTok{as.data.frame}\NormalTok{()}
  
  \CommentTok{\# Add rows excluded from \textasciigrave{}validation\_set\textasciigrave{} into \textasciigrave{}train\_set\textasciigrave{}}
\NormalTok{  tmp.excluded }\OtherTok{\textless{}{-}} \FunctionTok{anti\_join}\NormalTok{(tmp.data, validation\_set)}
\NormalTok{  train\_set }\OtherTok{\textless{}{-}} \FunctionTok{rbind}\NormalTok{(train\_set, tmp.excluded)}
  
  \FunctionTok{put\_log}\NormalTok{(}\StringTok{"Function: \textasciigrave{}sample\_train\_validation\_sets\textasciigrave{}: Dataset created: validation\_set"}\NormalTok{)}
  \FunctionTok{put}\NormalTok{(}\FunctionTok{summary}\NormalTok{(validation\_set))}

  \CommentTok{\# CV train \& test sets Consistency Test}
\NormalTok{  validation.left\_join.Nas }\OtherTok{\textless{}{-}}\NormalTok{ train\_set }\SpecialCharTok{|\textgreater{}}
    \FunctionTok{mutate}\NormalTok{(}\AttributeTok{tst.col =}\NormalTok{ rating) }\SpecialCharTok{|\textgreater{}}
    \FunctionTok{select}\NormalTok{(userId, movieId, tst.col) }\SpecialCharTok{|\textgreater{}}
    \FunctionTok{data.consistency.test}\NormalTok{(validation\_set)}
  
  \FunctionTok{put\_log}\NormalTok{(}\StringTok{"Function: \textasciigrave{}sample\_train\_validation\_sets\textasciigrave{}:}
\StringTok{Below are the data consistency verification results"}\NormalTok{)}
  \FunctionTok{put}\NormalTok{(validation.left\_join.Nas)}
  
  \CommentTok{\# Return result datassets {-}{-}{-}{-}{-}{-}{-}{-}{-}{-}{-}{-}{-}{-}{-}{-}{-}{-}{-}{-}{-}{-}{-}{-}{-}{-}{-}{-}{-}{-}{-}{-}{-}{-}{-}{-}{-}{-}{-}{-}{-}{-}{-}{-}{-}{-}{-}{-}{-}{-}{-}{-}    }
  \FunctionTok{list}\NormalTok{(}\AttributeTok{train\_set =}\NormalTok{ train\_set, }
       \AttributeTok{validation\_set =}\NormalTok{ validation\_set)}
\NormalTok{\}}
\end{Highlighting}
\end{Shaded}

\begin{noteblock}
The \href{https://github.com/AzKurban-edX-DS/Capstone-MovieLens/blob/main/r/src/support-functions/data.helper.functions.R\#L15}{sample\_train\_validation\_sets} function is defined in the same script as the \href{https://github.com/AzKurban-edX-DS/Capstone-MovieLens/blob/main/r/src/support-functions/data.helper.functions.R\#L87}{make\_source\_datasets}` one, from where it is called.

\end{noteblock}

\newpage

\subsubsection{Common Helper Functions}\label{common-helper-functions}

\hfill\break

For our further analysis, we are going to use the following \emph{common helper functions}:

\paragraph{\texorpdfstring{\href{https://github.com/AzKurban-edX-DS/Capstone-MovieLens/blob/main/r/src/support-functions/common-helper.functions.R\#L4}{clamp} function}{clamp function}}\label{func.clamp}

\hfill\break

As explained in \href{https://rafalab.dfci.harvard.edu/dsbook-part-2/highdim/regularization.html\#user-effects}{Section 24.4 User effects} of the \emph{Course Textbook} we know ratings can't be below 0.5 or above 5. For this reason, we will use the \texttt{clamp} function described in that section:

\begin{Shaded}
\begin{Highlighting}[]
\NormalTok{clamp }\OtherTok{\textless{}{-}} \ControlFlowTok{function}\NormalTok{(x, }\AttributeTok{min =} \FloatTok{0.5}\NormalTok{, }\AttributeTok{max =} \DecValTok{5}\NormalTok{) }\FunctionTok{pmax}\NormalTok{(}\FunctionTok{pmin}\NormalTok{(x, max), min)}
\end{Highlighting}
\end{Shaded}

\paragraph{\texorpdfstring{Functions to calculate \emph{(Root) Mean Squared Error}}{Functions to calculate (Root) Mean Squared Error}}\label{functions-to-calculate-root-mean-squared-error}

\hfill\break

We will need the following functions to calculate \emph{(R)MSEs}:

\begin{Shaded}
\begin{Highlighting}[]
\NormalTok{mse }\OtherTok{\textless{}{-}} \ControlFlowTok{function}\NormalTok{(r) }\FunctionTok{mean}\NormalTok{(r}\SpecialCharTok{\^{}}\DecValTok{2}\NormalTok{)}

\NormalTok{mse\_cv }\OtherTok{\textless{}{-}} \ControlFlowTok{function}\NormalTok{(r\_list) \{}
\NormalTok{  mses }\OtherTok{\textless{}{-}} \FunctionTok{sapply}\NormalTok{(r\_list, }\FunctionTok{mse}\NormalTok{(r))}
  \FunctionTok{mean}\NormalTok{(mses)}
\NormalTok{\}}

\NormalTok{rmse }\OtherTok{\textless{}{-}} \ControlFlowTok{function}\NormalTok{(r) }\FunctionTok{sqrt}\NormalTok{(}\FunctionTok{mse}\NormalTok{(r))}
\CommentTok{\# rmse\_cv \textless{}{-} function(r\_list) sqrt(mse\_cv(r\_list))}

\NormalTok{rmse2 }\OtherTok{\textless{}{-}} \ControlFlowTok{function}\NormalTok{(true\_ratings, predicted\_ratings) \{}
  \FunctionTok{rmse}\NormalTok{(true\_ratings }\SpecialCharTok{{-}}\NormalTok{ predicted\_ratings)}
\NormalTok{\}}
\end{Highlighting}
\end{Shaded}

\begin{noteblock}
All the \emph{common helper functions}, including those described above, are defined in the \href{https://github.com/AzKurban-edX-DS/Capstone-MovieLens/blob/main/r/src/support-functions/common-helper.functions.R}{common-helper.functions.R} script on \emph{GitHub}.

\end{noteblock}

\newpage

\subsection{Overall Mean Rating (OMR) Model}\label{overall-mean-rating-omr-model}

\hfill\break

Let's begin our analysis by evaluating the simplest model described in \href{https://rafalab.dfci.harvard.edu/dsbook-part-2/highdim/regularization.html\#a-first-model}{Section \emph{23.3 The First Model} of the \emph{Course Textbook}}, and then gradually refine it through further research.

\subsubsection{Mathematical Description of the OMR Model}\label{mathematical-description-of-the-omr-model}

\hfill\break

It is about a model that assumes the same rating for all movies and users with all the differences explained by random variation would look as follows:

\[
Y_{i,j} = \mu + \varepsilon_{i,j}
\]

with \(\varepsilon_{i,j}\) independent errors sampled from the same distribution centered at 0 and \(\mu\) the \emph{true} rating for all movies.

\subsubsection{OMR Model Building}\label{omr-model-building}

\hfill\break

We know that the estimate that minimizes the RMSE is the least squares estimate of \(\mu\) and, in this case, is the average of all ratings:

\begin{Shaded}
\begin{Highlighting}[]
\NormalTok{mu }\OtherTok{\textless{}{-}} \FunctionTok{mean}\NormalTok{(edx}\SpecialCharTok{$}\NormalTok{rating)}
\FunctionTok{print}\NormalTok{(mu)}
\end{Highlighting}
\end{Shaded}

\begin{verbatim}
## [1] 3.512465
\end{verbatim}

If we predict all unknown ratings with \(\hat{\mu}\), we obtain the following RMSE:

\begin{Shaded}
\begin{Highlighting}[]
\NormalTok{mu.MSEs }\OtherTok{\textless{}{-}} \FunctionTok{naive\_model\_MSEs}\NormalTok{(mu)}
\FunctionTok{data.frame}\NormalTok{(}\AttributeTok{fold\_No =} \DecValTok{1}\SpecialCharTok{:}\DecValTok{5}\NormalTok{, }\AttributeTok{MSE =}\NormalTok{ mu.MSEs) }\SpecialCharTok{|\textgreater{}}
  \FunctionTok{data.plot}\NormalTok{(}\AttributeTok{title =} \StringTok{"MSE resuls of the 5{-}fold CV method applied to the Overall Mean Rating Model"}\NormalTok{,}
              \AttributeTok{xname =} \StringTok{"fold\_No"}\NormalTok{, }
              \AttributeTok{yname =} \StringTok{"MSE"}\NormalTok{)}
\end{Highlighting}
\end{Shaded}

\pandocbounded{\includegraphics[keepaspectratio]{capstone-movielens.report_files/figure-latex/unnamed-chunk-74-1.pdf}}

\begin{Shaded}
\begin{Highlighting}[]
\NormalTok{mu.RMSE }\OtherTok{\textless{}{-}} \FunctionTok{sqrt}\NormalTok{(}\FunctionTok{mean}\NormalTok{(mu.MSEs))}
\NormalTok{mu.RMSE}
\end{Highlighting}
\end{Shaded}

\begin{verbatim}
## [1] 1.060346
\end{verbatim}

\begin{noteblock}
For the \emph{Mean Squared Errors} data visualization we use the custom data visualization function \hyperref[func.data.plot]{data.plot} described in the \hyperref[appndx_a.data_helper.functions]{Data Helper Functions} section of the \hyperref[appndx_a]{Appendix} to this report, which is defined in the \href{https://github.com/AzKurban-edX-DS/Capstone-MovieLens/blob/main/r/src/support-functions/data.helper.functions.R\#L447}{Data Visualization} section of the \href{https://github.com/AzKurban-edX-DS/Capstone-MovieLens/blob/main/r/src/support-functions/data.helper.functions.R\#L592}{data.helper.functions.R} script on \emph{GitHub}.

\end{noteblock}

Here we also used \href{https://github.com/AzKurban-edX-DS/Capstone-MovieLens/blob/main/r/src/support-functions/common-helper.functions.R\#L54}{naive\_model\_MSEs} function defined in the \href{https://github.com/AzKurban-edX-DS/Capstone-MovieLens/blob/main/r/src/support-functions/common-helper.functions.R}{common-helper.functions.R} script (already mentioned above) to compute \emph{Mean Squared Errors} using \emph{5-Fold Cross Validation} method:

\begin{Shaded}
\begin{Highlighting}[]
\NormalTok{naive\_model\_MSEs }\OtherTok{\textless{}{-}} \ControlFlowTok{function}\NormalTok{(val) \{}
  \FunctionTok{sapply}\NormalTok{(edx\_CV, }\ControlFlowTok{function}\NormalTok{(cv\_item)\{}
    \FunctionTok{mse}\NormalTok{(cv\_item}\SpecialCharTok{$}\NormalTok{validation\_set}\SpecialCharTok{$}\NormalTok{rating }\SpecialCharTok{{-}}\NormalTok{ val)}
\NormalTok{  \})}
\NormalTok{\}}
\end{Highlighting}
\end{Shaded}

One more function, defined in the \href{https://github.com/AzKurban-edX-DS/Capstone-MovieLens/blob/main/r/src/support-functions/common-helper.functions.R}{same script}, that we will need for further analysis of the current model, is the \href{https://github.com/AzKurban-edX-DS/Capstone-MovieLens/blob/main/r/src/support-functions/common-helper.functions.R\#L59}{naive\_model\_RMSE} one:

\begin{Shaded}
\begin{Highlighting}[]
\NormalTok{naive\_model\_RMSE }\OtherTok{\textless{}{-}} \ControlFlowTok{function}\NormalTok{(val)\{}
  \FunctionTok{sqrt}\NormalTok{(}\FunctionTok{mean}\NormalTok{(}\FunctionTok{naive\_model\_MSEs}\NormalTok{(val)))}
\NormalTok{\}}
\end{Highlighting}
\end{Shaded}

\newpage

\subsubsection{OMR Value Is the Best for the Current Model}\label{omr-value-is-the-best-for-the-current-model}

\hfill\break

If we plug in any other number, we will get a higher RMSE. Let's prove that by the following small investigation:

\begin{Shaded}
\begin{Highlighting}[]
\NormalTok{  deviation }\OtherTok{\textless{}{-}} \FunctionTok{seq}\NormalTok{(}\DecValTok{0}\NormalTok{, }\DecValTok{6}\NormalTok{, }\FloatTok{0.1}\NormalTok{) }\SpecialCharTok{{-}} \DecValTok{3}

\NormalTok{  deviation.RMSE }\OtherTok{\textless{}{-}} \FunctionTok{sapply}\NormalTok{(deviation, }\ControlFlowTok{function}\NormalTok{(delta)\{}
    \FunctionTok{naive\_model\_RMSE}\NormalTok{(mu }\SpecialCharTok{+}\NormalTok{ delta)}
\NormalTok{  \})}
\end{Highlighting}
\end{Shaded}

Let's make a quick investigation of the \texttt{deviation.RMSE} result we have just got:

\begin{Shaded}
\begin{Highlighting}[]
\FunctionTok{data.frame}\NormalTok{(}\AttributeTok{delta =}\NormalTok{ deviation, }
           \AttributeTok{delta.RMSE =}\NormalTok{ deviation.RMSE) }\SpecialCharTok{|\textgreater{}} 
\FunctionTok{data.plot}\NormalTok{(}\AttributeTok{title =} \FunctionTok{TeX}\NormalTok{(r}\StringTok{\textquotesingle{}[RMSE as a function of deviation ($\textbackslash{}delta$) from the Overall Mean Rating ($\textbackslash{}hat\{mu\}$)]\textquotesingle{}}\NormalTok{),}
              \AttributeTok{xname =} \StringTok{"delta"}\NormalTok{, }
              \AttributeTok{yname =} \StringTok{"delta.RMSE"}\NormalTok{, }
              \AttributeTok{xlabel =} \FunctionTok{TeX}\NormalTok{(r}\StringTok{\textquotesingle{}[$\textbackslash{}delta$]\textquotesingle{}}\NormalTok{), }
              \AttributeTok{ylabel =} \StringTok{"RMSE"}\NormalTok{)}
\end{Highlighting}
\end{Shaded}

\pandocbounded{\includegraphics[keepaspectratio]{capstone-movielens.report_files/figure-latex/unnamed-chunk-78-1.pdf}}

\begin{Shaded}
\begin{Highlighting}[]
\NormalTok{which\_min\_deviation }\OtherTok{\textless{}{-}}\NormalTok{ deviation[}\FunctionTok{which.min}\NormalTok{(deviation.RMSE)]}
\NormalTok{min\_rmse }\OtherTok{=} \FunctionTok{min}\NormalTok{(deviation.RMSE)}

\FunctionTok{print\_log1}\NormalTok{(}\StringTok{"Minimum RMSE is achieved when the deviation from the mean is: \%1"}\NormalTok{,}
\NormalTok{         which\_min\_deviation)}
\end{Highlighting}
\end{Shaded}

\begin{verbatim}
## Minimum RMSE is achieved when the deviation from the mean is: 0
\end{verbatim}

\begin{Shaded}
\begin{Highlighting}[]
\FunctionTok{print\_log1}\NormalTok{(}\StringTok{"Is the previously computed RMSE the best for the current model: \%1"}\NormalTok{,}
\NormalTok{         mu.RMSE }\SpecialCharTok{==}\NormalTok{ min\_rmse)}
\end{Highlighting}
\end{Shaded}

\begin{verbatim}
## Is the previously computed RMSE the best for the current model: TRUE
\end{verbatim}

\begin{Shaded}
\begin{Highlighting}[]
\NormalTok{RMSEs.ResultTibble.OMR }\OtherTok{\textless{}{-}}\NormalTok{ RMSEs.ResultTibble }\SpecialCharTok{|\textgreater{}} 
  \FunctionTok{RMSEs.AddRow}\NormalTok{(}\StringTok{"Overall Mean Rating Model"}\NormalTok{, mu.RMSE)}
\end{Highlighting}
\end{Shaded}

\begin{Shaded}
\begin{Highlighting}[]
\FunctionTok{RMSE\_kable}\NormalTok{(RMSEs.ResultTibble.OMR)}
\end{Highlighting}
\end{Shaded}

\begin{tabular}{>{\raggedright\arraybackslash}p{15em}>{\centering\arraybackslash}p{5em}>{\raggedright\arraybackslash}p{30em}}
\toprule
\textbf{Method} & \textbf{RMSE} & \textbf{Comment}\\
\midrule
Project Objective & 0.864900 & \\
Overall Mean Rating Model & 1.060346 & \\
\bottomrule
\end{tabular}

To win the grand prize of \$1,000,000, a participating team had to get an RMSE of at least 0.8563\autocite{BigChaosSln}. So we can definitely do better!\autocite{IDS2_23-3}

\newpage

\subsection{User Effect (UE) Model}\label{user-effect-ue-model}

\hfill\break

\subsubsection{User Effect Anasysis}\label{user-effect-anasysis}

\hfill\break

To improve our model let's now take into consideration user effects as explained in \href{https://rafalab.dfci.harvard.edu/dsbook-part-2/highdim/regularization.html\#user-effects}{Section \emph{23.4 User effects}} of the \emph{Course Textbook}\autocite{IDS2_23-4}.

If we visualize the average rating for each user the way the \href{https://x.com/rafalab}{the author} shows, we can see that there is substantial variability in the average ratings across users:

\begin{Shaded}
\begin{Highlighting}[]
\FunctionTok{hist}\NormalTok{(edx.user\_mean\_ratings}\SpecialCharTok{$}\NormalTok{mean\_rating, }\AttributeTok{nclass =} \DecValTok{30}\NormalTok{)}
\end{Highlighting}
\end{Shaded}

\pandocbounded{\includegraphics[keepaspectratio]{capstone-movielens.report_files/figure-latex/unnamed-chunk-89-1.pdf}}

\newpage

\paragraph{Ratings per user}\label{ratings-per-user}

\hfill\break
eponymous
homonymous

\begin{noteblock}
The code in this section is cited from \href{https://www.kaggle.com/code/amirmotefaker/movie-recommendation-system-using-r-best/notebook\#Ratings-per-user}{the eponymous section} of the article \emph{Movie Recommendation System using R - BEST}\autocite{MRS-R-BEST}.

\end{noteblock}

\subparagraph{User rating count (activity measure)}\label{user-rating-count-activity-measure}

\hfill\break

\begin{Shaded}
\begin{Highlighting}[]
\FunctionTok{print}\NormalTok{(edx }\SpecialCharTok{|\textgreater{}} 
  \FunctionTok{group\_by}\NormalTok{(userId) }\SpecialCharTok{|\textgreater{}} 
  \FunctionTok{summarize}\NormalTok{(}\AttributeTok{count =} \FunctionTok{n}\NormalTok{()) }\SpecialCharTok{|\textgreater{}}
  \FunctionTok{slice\_head}\NormalTok{(}\AttributeTok{n =} \DecValTok{10}\NormalTok{)}
\NormalTok{)}
\end{Highlighting}
\end{Shaded}

\begin{verbatim}
## # A tibble: 10 x 2
##    userId count
##     <int> <int>
##  1      1    19
##  2      2    17
##  3      3    31
##  4      4    35
##  5      5    74
##  6      6    39
##  7      7    96
##  8      8   727
##  9      9    21
## 10     10   112
\end{verbatim}

\subparagraph{User rating summary}\label{user-rating-summary}

\hfill\break

\begin{Shaded}
\begin{Highlighting}[]
\FunctionTok{summary}\NormalTok{(edx }\SpecialCharTok{|\textgreater{}} \FunctionTok{group\_by}\NormalTok{(userId) }\SpecialCharTok{|\textgreater{}} \FunctionTok{summarize}\NormalTok{(}\AttributeTok{count =} \FunctionTok{n}\NormalTok{()) }\SpecialCharTok{|\textgreater{}} \FunctionTok{select}\NormalTok{(count))}
\end{Highlighting}
\end{Shaded}

\begin{verbatim}
##      count       
##  Min.   :  10.0  
##  1st Qu.:  32.0  
##  Median :  62.0  
##  Mean   : 128.8  
##  3rd Qu.: 141.0  
##  Max.   :6616.0
\end{verbatim}

\newpage

\subparagraph{Ratings per user plot}\label{ratings-per-user-plot}

\hfill\break

\begin{Shaded}
\begin{Highlighting}[]
\NormalTok{edx }\SpecialCharTok{|\textgreater{}}
  \FunctionTok{group\_by}\NormalTok{(userId) }\SpecialCharTok{|\textgreater{}}
  \FunctionTok{summarize}\NormalTok{(}\AttributeTok{count =} \FunctionTok{n}\NormalTok{()) }\SpecialCharTok{|\textgreater{}}
  \FunctionTok{ggplot}\NormalTok{(}\FunctionTok{aes}\NormalTok{(}\AttributeTok{x =}\NormalTok{ userId, }\AttributeTok{y =}\NormalTok{ count)) }\SpecialCharTok{+}
  \FunctionTok{geom\_point}\NormalTok{(}\AttributeTok{alpha =} \FloatTok{0.2}\NormalTok{, }\AttributeTok{color =} \StringTok{"\#4020dd"}\NormalTok{) }\SpecialCharTok{+}
  \FunctionTok{geom\_smooth}\NormalTok{(}\AttributeTok{color =} \StringTok{"red"}\NormalTok{) }\SpecialCharTok{+}
  \FunctionTok{ggtitle}\NormalTok{(}\StringTok{"Ratings per user"}\NormalTok{) }\SpecialCharTok{+}
  \FunctionTok{xlab}\NormalTok{(}\StringTok{"Users"}\NormalTok{) }\SpecialCharTok{+}
  \FunctionTok{ylab}\NormalTok{(}\StringTok{"Number of ratings"}\NormalTok{) }\SpecialCharTok{+}
  \FunctionTok{scale\_y\_continuous}\NormalTok{(}\AttributeTok{labels =}\NormalTok{ comma) }\SpecialCharTok{+}
  \FunctionTok{scale\_x\_continuous}\NormalTok{(}\AttributeTok{n.breaks =} \DecValTok{10}\NormalTok{) }\SpecialCharTok{+}
  \FunctionTok{theme\_economist}\NormalTok{() }\SpecialCharTok{+}
  \FunctionTok{theme}\NormalTok{(}\AttributeTok{axis.title.x =} \FunctionTok{element\_text}\NormalTok{(}\AttributeTok{vjust =} \SpecialCharTok{{-}}\DecValTok{5}\NormalTok{, }\AttributeTok{face =} \StringTok{"bold"}\NormalTok{), }
        \AttributeTok{axis.title.y =} \FunctionTok{element\_text}\NormalTok{(}\AttributeTok{vjust =} \DecValTok{10}\NormalTok{, }\AttributeTok{face =} \StringTok{"bold"}\NormalTok{), }
        \AttributeTok{plot.margin =} \FunctionTok{margin}\NormalTok{(}\FloatTok{0.7}\NormalTok{, }\FloatTok{0.5}\NormalTok{, }\DecValTok{1}\NormalTok{, }\FloatTok{1.2}\NormalTok{, }\StringTok{"cm"}\NormalTok{))}
\end{Highlighting}
\end{Shaded}

\begin{verbatim}
## `geom_smooth()` using method = 'gam' and formula = 'y ~ s(x, bs = "cs")'
\end{verbatim}

\pandocbounded{\includegraphics[keepaspectratio]{capstone-movielens.report_files/figure-latex/unnamed-chunk-92-1.pdf}}
\newpage

\subparagraph{Users' rating histogram}\label{users-rating-histogram}

\hfill\break

\begin{Shaded}
\begin{Highlighting}[]
\NormalTok{edx }\SpecialCharTok{|\textgreater{}}
  \FunctionTok{group\_by}\NormalTok{(userId) }\SpecialCharTok{|\textgreater{}}
  \FunctionTok{summarize}\NormalTok{(}\AttributeTok{count =} \FunctionTok{n}\NormalTok{()) }\SpecialCharTok{|\textgreater{}}
  \FunctionTok{ggplot}\NormalTok{(}\FunctionTok{aes}\NormalTok{(}\AttributeTok{x =}\NormalTok{ count)) }\SpecialCharTok{+}
  \FunctionTok{geom\_histogram}\NormalTok{(}\AttributeTok{fill =} \StringTok{"\#8888ff"}\NormalTok{, }\AttributeTok{color =} \StringTok{"\#4020dd"}\NormalTok{) }\SpecialCharTok{+}
  \FunctionTok{ggtitle}\NormalTok{(}\StringTok{"Users\textquotesingle{} rating histogram"}\NormalTok{) }\SpecialCharTok{+}
  \FunctionTok{xlab}\NormalTok{(}\StringTok{"Rating count"}\NormalTok{) }\SpecialCharTok{+}
  \FunctionTok{ylab}\NormalTok{(}\StringTok{"Number of users"}\NormalTok{) }\SpecialCharTok{+}
  \FunctionTok{scale\_y\_continuous}\NormalTok{(}\AttributeTok{labels =}\NormalTok{ comma) }\SpecialCharTok{+}
  \FunctionTok{scale\_x\_log10}\NormalTok{(}\AttributeTok{n.breaks =} \DecValTok{10}\NormalTok{) }\SpecialCharTok{+}
  \FunctionTok{theme\_economist}\NormalTok{() }\SpecialCharTok{+}
  \FunctionTok{theme}\NormalTok{(}\AttributeTok{axis.title.x =} \FunctionTok{element\_text}\NormalTok{(}\AttributeTok{vjust =} \SpecialCharTok{{-}}\DecValTok{5}\NormalTok{, }\AttributeTok{face =} \StringTok{"bold"}\NormalTok{), }
        \AttributeTok{axis.title.y =} \FunctionTok{element\_text}\NormalTok{(}\AttributeTok{vjust =} \DecValTok{10}\NormalTok{, }\AttributeTok{face =} \StringTok{"bold"}\NormalTok{), }
        \AttributeTok{plot.margin =} \FunctionTok{margin}\NormalTok{(}\FloatTok{0.7}\NormalTok{, }\FloatTok{0.5}\NormalTok{, }\DecValTok{1}\NormalTok{, }\FloatTok{1.2}\NormalTok{, }\StringTok{"cm"}\NormalTok{))}
\end{Highlighting}
\end{Shaded}

\begin{verbatim}
## `stat_bin()` using `bins = 30`. Pick better value with `binwidth`.
\end{verbatim}

\pandocbounded{\includegraphics[keepaspectratio]{capstone-movielens.report_files/figure-latex/unnamed-chunk-93-1.pdf}}

\newpage

\subsubsection{Mathematical Description of the UE Model}\label{mathematical-description-of-the-ue-model}

\hfill\break

Following the author's further explanation, to account for this variability, we will use a linear model with a \emph{treatment effect} \(\alpha_i\) for each user. The sum \(\mu+\alpha_i\) can be interpreted as the typical rating user \(i\) gives to movies. So we write the model as follows:

\[
Y_{i,j} = \mu + \alpha_i + \varepsilon_{i,j}
\]

Statistics textbooks refer to the \(\alpha\)s as treatment effects. In the Netflix challenge papers, they refer to them as \emph{bias}\autocite{IDS2_23-4,MFT_RS}.

\subsubsection{UE Model Building}\label{ue-model-building}

\hfill\break

As it is stated here\autocite{IDS2_23-4}, it can be shown that the least squares estimate \(\hat{\alpha}_i\) is just the average of \(y_{i,j} - \hat{\mu}\) for each user \(i\). So we can compute them this way:

\begin{Shaded}
\begin{Highlighting}[]
\NormalTok{a }\OtherTok{\textless{}{-}} \FunctionTok{rowMeans}\NormalTok{(y }\SpecialCharTok{{-}}\NormalTok{ mu, }\AttributeTok{na.rm =} \ConstantTok{TRUE}\NormalTok{)}
\end{Highlighting}
\end{Shaded}

These considerations alows us to compute a \emph{User Mean Ratings} the following way:

\begin{Shaded}
\begin{Highlighting}[]
\FunctionTok{put\_log}\NormalTok{(}\StringTok{"Computing Average Ratings per User (User Mean Ratings)..."}\NormalTok{)}
\NormalTok{user.mean\_ratings }\OtherTok{\textless{}{-}} \FunctionTok{rowMeans}\NormalTok{(edx.mx, }\AttributeTok{na.rm =} \ConstantTok{TRUE}\NormalTok{)}
\NormalTok{user\_ratings.n }\OtherTok{\textless{}{-}} \FunctionTok{rowSums}\NormalTok{(}\SpecialCharTok{!}\FunctionTok{is.na}\NormalTok{(edx.mx))}
  
\NormalTok{edx.user\_mean\_ratings }\OtherTok{\textless{}{-}} 
  \FunctionTok{data.frame}\NormalTok{(}\AttributeTok{userId =} \FunctionTok{names}\NormalTok{(user.mean\_ratings), }
             \AttributeTok{mean\_rating =}\NormalTok{ user.mean\_ratings,}
             \AttributeTok{n =}\NormalTok{ user\_ratings.n)}
  
\FunctionTok{put\_log}\NormalTok{(}\StringTok{"User Mean Ratings have been computed."}\NormalTok{)}
\end{Highlighting}
\end{Shaded}

\begin{Shaded}
\begin{Highlighting}[]
\FunctionTok{str}\NormalTok{(edx.user\_mean\_ratings)}
\end{Highlighting}
\end{Shaded}

\begin{verbatim}
## 'data.frame':    69878 obs. of  3 variables:
##  $ userId     : chr  "1" "2" "3" "4" ...
##  $ mean_rating: num  5 3.29 3.94 4.06 3.92 ...
##  $ n          : num  19 17 31 35 74 39 96 727 21 112 ...
\end{verbatim}

And then we compute a \emph{User Effect} this way:

\begin{Shaded}
\begin{Highlighting}[]
\FunctionTok{put\_log}\NormalTok{(}\StringTok{"Computing User Effect per users ..."}\NormalTok{)}
\NormalTok{edx.user\_effect }\OtherTok{\textless{}{-}}\NormalTok{ edx.user\_mean\_ratings }\SpecialCharTok{|\textgreater{}}
  \FunctionTok{mutate}\NormalTok{(}\AttributeTok{userId =} \FunctionTok{as.integer}\NormalTok{(userId),}
         \AttributeTok{a =}\NormalTok{ mean\_rating }\SpecialCharTok{{-}}\NormalTok{ mu)}

\FunctionTok{put\_log}\NormalTok{(}\StringTok{"A User Effect Model has been builded"}\NormalTok{)}
\end{Highlighting}
\end{Shaded}

\begin{Shaded}
\begin{Highlighting}[]
\FunctionTok{par}\NormalTok{(}\AttributeTok{cex =} \FloatTok{0.7}\NormalTok{)}
\FunctionTok{hist}\NormalTok{(edx.user\_effect}\SpecialCharTok{$}\NormalTok{a, }\DecValTok{30}\NormalTok{, }\AttributeTok{xlab =} \FunctionTok{TeX}\NormalTok{(r}\StringTok{\textquotesingle{}[$\textbackslash{}hat\{alpha\}\_\{i\}$]\textquotesingle{}}\NormalTok{),}
     \AttributeTok{main =} \FunctionTok{TeX}\NormalTok{(r}\StringTok{\textquotesingle{}[Histogram of $\textbackslash{}hat\{alpha\}\_\{i\}$]\textquotesingle{}}\NormalTok{))}
\end{Highlighting}
\end{Shaded}

\pandocbounded{\includegraphics[keepaspectratio]{capstone-movielens.report_files/figure-latex/unnamed-chunk-98-1.pdf}}

\begin{Shaded}
\begin{Highlighting}[]
\FunctionTok{str}\NormalTok{(edx.user\_effect)}
\end{Highlighting}
\end{Shaded}

\begin{verbatim}
## 'data.frame':    69878 obs. of  4 variables:
##  $ userId     : int  1 2 3 4 5 6 7 8 9 10 ...
##  $ mean_rating: num  5 3.29 3.94 4.06 3.92 ...
##  $ n          : num  19 17 31 35 74 39 96 727 21 112 ...
##  $ a          : num  1.488 -0.218 0.423 0.545 0.406 ...
\end{verbatim}

\begin{noteblock}
The full source code of the \emph{User Effect} computation is available in the \href{https://github.com/AzKurban-edX-DS/Capstone-MovieLens/blob/main/r/src/capstone-movielens.main.R\#L531}{Model building: User Effect} section of the \href{https://github.com/AzKurban-edX-DS/Capstone-MovieLens/blob/main/r/src/capstone-movielens.main.R}{capstone-movielens.main.R} script on \emph{GitHub}.

\end{noteblock}

Now, we are ready to compute the \emph{Mean Squared Errors} from the \emph{5-Fold Cross Validation} (additionally using the \hyperref[func.clamp]{clamp} helper function we described above):

\begin{Shaded}
\begin{Highlighting}[]
\FunctionTok{put\_log}\NormalTok{(}\StringTok{"Computing the RMSE taking into account user effects..."}\NormalTok{)}
\NormalTok{start }\OtherTok{\textless{}{-}} \FunctionTok{put\_start\_date}\NormalTok{()}
\NormalTok{edx.user\_effect.MSEs }\OtherTok{\textless{}{-}} \FunctionTok{sapply}\NormalTok{(edx\_CV, }\ControlFlowTok{function}\NormalTok{(cv\_fold\_dat)\{}
\NormalTok{  cv\_fold\_dat}\SpecialCharTok{$}\NormalTok{validation\_set }\SpecialCharTok{|\textgreater{}}
    \FunctionTok{left\_join}\NormalTok{(edx.user\_effect, }\AttributeTok{by =} \StringTok{"userId"}\NormalTok{) }\SpecialCharTok{|\textgreater{}}
    \FunctionTok{mutate}\NormalTok{(}\AttributeTok{resid =}\NormalTok{ rating }\SpecialCharTok{{-}} \FunctionTok{clamp}\NormalTok{(mu }\SpecialCharTok{+}\NormalTok{ a)) }\SpecialCharTok{|\textgreater{}} 
    \FunctionTok{pull}\NormalTok{(resid) }\SpecialCharTok{|\textgreater{}} \FunctionTok{mse}\NormalTok{()}
\NormalTok{\})}
\FunctionTok{put\_end\_date}\NormalTok{(start)}
\end{Highlighting}
\end{Shaded}

\begin{Shaded}
\begin{Highlighting}[]
\FunctionTok{data.frame}\NormalTok{(}\AttributeTok{fold\_No =} \DecValTok{1}\SpecialCharTok{:}\DecValTok{5}\NormalTok{, }\AttributeTok{MSE =}\NormalTok{ edx.user\_effect.MSEs) }\SpecialCharTok{|\textgreater{}}
  \FunctionTok{data.plot}\NormalTok{(}\AttributeTok{title =} \StringTok{"MSE resuls of the 5{-}fold CV method applied to the User Effect Model"}\NormalTok{,}
            \AttributeTok{xname =} \StringTok{"fold\_No"}\NormalTok{, }
            \AttributeTok{yname =} \StringTok{"MSE"}\NormalTok{)}
\end{Highlighting}
\end{Shaded}

\pandocbounded{\includegraphics[keepaspectratio]{capstone-movielens.report_files/figure-latex/unnamed-chunk-100-1.pdf}}

\begin{noteblock}
For the \emph{Mean Squared Errors} data visualization we use the custom data visualization function \hyperref[func.data.plot]{data.plot} described in the \hyperref[appndx_a.data_helper.functions]{Data Helper Functions} section of the \hyperref[appndx_a]{Appendix} to this report, which is defined in the \href{https://github.com/AzKurban-edX-DS/Capstone-MovieLens/blob/main/r/src/support-functions/data.helper.functions.R\#L447}{Data Visualization} section of the \href{https://github.com/AzKurban-edX-DS/Capstone-MovieLens/blob/main/r/src/support-functions/data.helper.functions.R\#L592}{data.helper.functions.R} script on \emph{GitHub}.

\end{noteblock}

Finally, we calculate the \texttt{RMSE} as the \emph{square root} of the average of the \emph{Mean Squared Errors} we obtained through the \emph{5-Fold Cross Validation} above:

\begin{Shaded}
\begin{Highlighting}[]
\NormalTok{edx.user\_effect.RMSE }\OtherTok{\textless{}{-}} \FunctionTok{sqrt}\NormalTok{(}\FunctionTok{mean}\NormalTok{(edx.user\_effect.MSEs))}

\NormalTok{RMSEs.ResultTibble.UE }\OtherTok{\textless{}{-}}\NormalTok{ RMSEs.ResultTibble.OMR }\SpecialCharTok{|\textgreater{}} 
  \FunctionTok{RMSEs.AddRow}\NormalTok{(}\StringTok{"User Effect Model"}\NormalTok{, edx.user\_effect.RMSE)}
\end{Highlighting}
\end{Shaded}

\begin{Shaded}
\begin{Highlighting}[]
\FunctionTok{RMSE\_kable}\NormalTok{(RMSEs.ResultTibble.UE)}
\end{Highlighting}
\end{Shaded}

\begin{tabular}{>{\raggedright\arraybackslash}p{15em}>{\centering\arraybackslash}p{5em}>{\raggedright\arraybackslash}p{30em}}
\toprule
\textbf{Method} & \textbf{RMSE} & \textbf{Comment}\\
\midrule
Project Objective & 0.8649000 & \\
Overall Mean Rating Model & 1.0603462 & \\
User Effect Model & 0.9697962 & \\
\bottomrule
\end{tabular}

\begin{noteblock}
The full source code of the \emph{User Effect Model RMSE} computation is available in the \href{https://github.com/AzKurban-edX-DS/Capstone-MovieLens/blob/main/r/src/capstone-movielens.main.R\#L655}{Compute RMSE for User Effect Model} section of the \href{https://github.com/AzKurban-edX-DS/Capstone-MovieLens/blob/main/r/src/capstone-movielens.main.R}{capstone-movielens.main.R} script on \emph{GitHub}.

\end{noteblock}

\newpage

\subsection{User+Movie Effect (UME) Model}\label{usermovie-effect-ume-model}

\hfill\break

\subsubsection{Movie Effect Analysis}\label{movie-effect-analysis}

\hfill\break

\paragraph{Movies' Popularity}\label{movies-popularity}

\hfill\break

\phantomsection\label{textbook.23.5.movie_effects}
In \href{https://rafalab.dfci.harvard.edu/dsbook-part-2/highdim/regularization.html\#movie-effects}{23.5 Movie effects} section of the \emph{Course Textbook} the author draws our attention to the fact that some movies are generally rated higher than others.

To prove this fact, we can find out the movies with the highest number of ratings using the following code:

\begin{Shaded}
\begin{Highlighting}[]
\NormalTok{edx.ordered\_movie\_ratings }\OtherTok{\textless{}{-}}\NormalTok{ edx }\SpecialCharTok{|\textgreater{}} \FunctionTok{group\_by}\NormalTok{(movieId, title) }\SpecialCharTok{|\textgreater{}}
  \FunctionTok{summarize}\NormalTok{(}\AttributeTok{number\_of\_ratings =} \FunctionTok{n}\NormalTok{()) }\SpecialCharTok{|\textgreater{}}
  \FunctionTok{arrange}\NormalTok{(}\FunctionTok{desc}\NormalTok{(number\_of\_ratings))}
\end{Highlighting}
\end{Shaded}

\begin{Shaded}
\begin{Highlighting}[]
\FunctionTok{print}\NormalTok{(}\FunctionTok{head}\NormalTok{(edx.ordered\_movie\_ratings))}
\end{Highlighting}
\end{Shaded}

\begin{verbatim}
## # A tibble: 6 x 3
## # Groups:   movieId [6]
##   movieId title                            number_of_ratings
##     <int> <chr>                                        <int>
## 1     296 Pulp Fiction (1994)                          31362
## 2     356 Forrest Gump (1994)                          31079
## 3     593 Silence of the Lambs, The (1991)             30382
## 4     480 Jurassic Park (1993)                         29360
## 5     318 Shawshank Redemption, The (1994)             28015
## 6     110 Braveheart (1995)                            26212
\end{verbatim}

\newpage

Now, we can figure out the most given ratings in order from most to least:

\begin{Shaded}
\begin{Highlighting}[]
\NormalTok{edx.rating\_groups }\OtherTok{\textless{}{-}}\NormalTok{ edx }\SpecialCharTok{|\textgreater{}}  \FunctionTok{group\_by}\NormalTok{(rating) }\SpecialCharTok{|\textgreater{}}
     \FunctionTok{summarise}\NormalTok{(}\AttributeTok{count =} \FunctionTok{n}\NormalTok{()) }\SpecialCharTok{|\textgreater{}}
     \FunctionTok{arrange}\NormalTok{(}\FunctionTok{desc}\NormalTok{(count))}
\end{Highlighting}
\end{Shaded}

\begin{Shaded}
\begin{Highlighting}[]
\FunctionTok{print}\NormalTok{(edx.rating\_groups)}
\end{Highlighting}
\end{Shaded}

\begin{verbatim}
## # A tibble: 10 x 2
##    rating   count
##     <dbl>   <int>
##  1    4   2588430
##  2    3   2121240
##  3    5   1390114
##  4    3.5  791624
##  5    2    711422
##  6    4.5  526736
##  7    1    345679
##  8    2.5  333010
##  9    1.5  106426
## 10    0.5   85374
\end{verbatim}

\newpage

\paragraph{Rating Distribution}\label{rating-distribution}

\hfill\break

The following code allows us to summarize that in general, half-star ratings are less common than whole-star ratings (e.g., there are fewer ratings of 3.5 than there are ratings of 3 or 4, etc.):

\begin{Shaded}
\begin{Highlighting}[]
\FunctionTok{print}\NormalTok{(edx.rating\_groups }\SpecialCharTok{|\textgreater{}} \FunctionTok{arrange}\NormalTok{(rating))}
\end{Highlighting}
\end{Shaded}

\begin{verbatim}
## # A tibble: 10 x 2
##    rating   count
##     <dbl>   <int>
##  1    0.5   85374
##  2    1    345679
##  3    1.5  106426
##  4    2    711422
##  5    2.5  333010
##  6    3   2121240
##  7    3.5  791624
##  8    4   2588430
##  9    4.5  526736
## 10    5   1390114
\end{verbatim}

We can visually see that from the following plot:

\begin{Shaded}
\begin{Highlighting}[]
\NormalTok{edx.rating\_groups }\SpecialCharTok{|\textgreater{}}
  \FunctionTok{ggplot}\NormalTok{(}\FunctionTok{aes}\NormalTok{(}\AttributeTok{x =}\NormalTok{ rating, }\AttributeTok{y =}\NormalTok{ count)) }\SpecialCharTok{+}
  \FunctionTok{geom\_line}\NormalTok{() }
\end{Highlighting}
\end{Shaded}

\pandocbounded{\includegraphics[keepaspectratio]{capstone-movielens.report_files/figure-latex/unnamed-chunk-120-1.pdf}}

\newpage

The more sophisticated way of visualizing the rating distribution shown below is cited from the \href{https://www.kaggle.com/code/amirmotefaker/movie-recommendation-system-using-r-best/notebook\#Rating-distribution-plot}{Rating distribution plot} section of the article \emph{Movie Recommendation System using R - BEST}\autocite{MRS-R-BEST} already referenced above in the \hyperref[ratings-per-user]{Ratings per user} section of this report.

\begin{Shaded}
\begin{Highlighting}[]
\NormalTok{edx.rating\_groups }\SpecialCharTok{|\textgreater{}}
  \FunctionTok{ggplot}\NormalTok{(}\FunctionTok{aes}\NormalTok{(}\AttributeTok{x =}\NormalTok{ rating, }\AttributeTok{y =}\NormalTok{ count)) }\SpecialCharTok{+}
  \FunctionTok{geom\_bar}\NormalTok{(}\AttributeTok{stat =} \StringTok{"identity"}\NormalTok{, }\AttributeTok{fill =} \StringTok{"\#8888ff"}\NormalTok{) }\SpecialCharTok{+}
  \FunctionTok{ggtitle}\NormalTok{(}\StringTok{"Rating Distribution"}\NormalTok{) }\SpecialCharTok{+}
  \FunctionTok{xlab}\NormalTok{(}\StringTok{"Rating"}\NormalTok{) }\SpecialCharTok{+}
  \FunctionTok{ylab}\NormalTok{(}\StringTok{"Occurrences Count"}\NormalTok{) }\SpecialCharTok{+}
  \FunctionTok{scale\_y\_continuous}\NormalTok{(}\AttributeTok{labels =}\NormalTok{ comma) }\SpecialCharTok{+}
  \FunctionTok{scale\_x\_continuous}\NormalTok{(}\AttributeTok{n.breaks =} \DecValTok{10}\NormalTok{) }\SpecialCharTok{+}
  \FunctionTok{theme\_economist}\NormalTok{() }\SpecialCharTok{+}
  \FunctionTok{theme}\NormalTok{(}\AttributeTok{axis.title.x =} \FunctionTok{element\_text}\NormalTok{(}\AttributeTok{vjust =} \SpecialCharTok{{-}}\DecValTok{5}\NormalTok{, }\AttributeTok{face =} \StringTok{"bold"}\NormalTok{), }
        \AttributeTok{axis.title.y =} \FunctionTok{element\_text}\NormalTok{(}\AttributeTok{vjust =} \DecValTok{10}\NormalTok{, }\AttributeTok{face =} \StringTok{"bold"}\NormalTok{), }
        \AttributeTok{plot.margin =} \FunctionTok{margin}\NormalTok{(}\FloatTok{0.7}\NormalTok{, }\FloatTok{0.5}\NormalTok{, }\DecValTok{1}\NormalTok{, }\FloatTok{1.2}\NormalTok{, }\StringTok{"cm"}\NormalTok{))}
\end{Highlighting}
\end{Shaded}

\pandocbounded{\includegraphics[keepaspectratio]{capstone-movielens.report_files/figure-latex/unnamed-chunk-121-1.pdf}}

\begin{quote}
This graph is another confirmation of what we found out above: rounded ratings occur more often than half-stared ones. The upward trend previously discussed is now perfectly clear, although it seems to top right between the 3 and 4-star ratings lowering the occurrences count afterward. That might be due to users being more hesitant to rate with the highest mark for whichever reasons they might hold\autocite{MRS-R-BEST}.
\end{quote}

The two more plots below, from the same article, give us an additional visual representation of the movies' popularity.

\newpage

\subparagraph{\texorpdfstring{\href{https://www.kaggle.com/code/amirmotefaker/movie-recommendation-system-using-r-best/notebook\#Ratings-per-movie-plot}{Ratings per movie plot}}{Ratings per movie plot}}\label{ratings-per-movie-plot}

\hfill\break

\begin{Shaded}
\begin{Highlighting}[]
\NormalTok{edx.movie\_groups }\OtherTok{\textless{}{-}}\NormalTok{ edx }\SpecialCharTok{|\textgreater{}}
  \FunctionTok{group\_by}\NormalTok{(movieId) }\SpecialCharTok{|\textgreater{}}
  \FunctionTok{summarize}\NormalTok{(}\AttributeTok{count =} \FunctionTok{n}\NormalTok{())}
  
\NormalTok{edx.movie\_groups }\SpecialCharTok{|\textgreater{}}
  \FunctionTok{ggplot}\NormalTok{(}\FunctionTok{aes}\NormalTok{(}\AttributeTok{x =}\NormalTok{ movieId, }\AttributeTok{y =}\NormalTok{ count)) }\SpecialCharTok{+}
  \FunctionTok{geom\_point}\NormalTok{(}\AttributeTok{alpha =} \FloatTok{0.2}\NormalTok{, }\AttributeTok{color =} \StringTok{"\#4020dd"}\NormalTok{) }\SpecialCharTok{+}
  \FunctionTok{geom\_smooth}\NormalTok{(}\AttributeTok{color =} \StringTok{"red"}\NormalTok{) }\SpecialCharTok{+}
  \FunctionTok{ggtitle}\NormalTok{(}\StringTok{"Ratings per movie"}\NormalTok{) }\SpecialCharTok{+}
  \FunctionTok{xlab}\NormalTok{(}\StringTok{"Movies"}\NormalTok{) }\SpecialCharTok{+}
  \FunctionTok{ylab}\NormalTok{(}\StringTok{"Number of ratings"}\NormalTok{) }\SpecialCharTok{+}
  \FunctionTok{scale\_y\_continuous}\NormalTok{(}\AttributeTok{labels =}\NormalTok{ comma) }\SpecialCharTok{+}
  \FunctionTok{scale\_x\_continuous}\NormalTok{(}\AttributeTok{n.breaks =} \DecValTok{10}\NormalTok{) }\SpecialCharTok{+}
  \FunctionTok{theme\_economist}\NormalTok{() }\SpecialCharTok{+}
  \FunctionTok{theme}\NormalTok{(}\AttributeTok{axis.title.x =} \FunctionTok{element\_text}\NormalTok{(}\AttributeTok{vjust =} \SpecialCharTok{{-}}\DecValTok{5}\NormalTok{, }\AttributeTok{face =} \StringTok{"bold"}\NormalTok{), }
        \AttributeTok{axis.title.y =} \FunctionTok{element\_text}\NormalTok{(}\AttributeTok{vjust =} \DecValTok{10}\NormalTok{, }\AttributeTok{face =} \StringTok{"bold"}\NormalTok{), }
        \AttributeTok{plot.margin =} \FunctionTok{margin}\NormalTok{(}\FloatTok{0.7}\NormalTok{, }\FloatTok{0.5}\NormalTok{, }\DecValTok{1}\NormalTok{, }\FloatTok{1.2}\NormalTok{, }\StringTok{"cm"}\NormalTok{))}
\end{Highlighting}
\end{Shaded}

\begin{verbatim}
## `geom_smooth()` using method = 'gam' and formula = 'y ~ s(x, bs = "cs")'
\end{verbatim}

\pandocbounded{\includegraphics[keepaspectratio]{capstone-movielens.report_files/figure-latex/unnamed-chunk-122-1.pdf}}
\newpage

\subparagraph{\texorpdfstring{\href{https://www.kaggle.com/code/amirmotefaker/movie-recommendation-system-using-r-best/notebook\#Movies'-rating-histogram}{Movies' rating histogram}}{Movies' rating histogram}}\label{movies-rating-histogram}

\hfill\break

\begin{Shaded}
\begin{Highlighting}[]
\NormalTok{edx.movie\_groups }\SpecialCharTok{|\textgreater{}}
  \FunctionTok{ggplot}\NormalTok{(}\FunctionTok{aes}\NormalTok{(}\AttributeTok{x =}\NormalTok{ count)) }\SpecialCharTok{+}
  \FunctionTok{geom\_histogram}\NormalTok{(}\AttributeTok{fill =} \StringTok{"\#8888ff"}\NormalTok{, }\AttributeTok{color =} \StringTok{"\#4020dd"}\NormalTok{) }\SpecialCharTok{+}
  \FunctionTok{ggtitle}\NormalTok{(}\StringTok{"Movies\textquotesingle{} rating histogram"}\NormalTok{) }\SpecialCharTok{+}
  \FunctionTok{xlab}\NormalTok{(}\StringTok{"Rating count"}\NormalTok{) }\SpecialCharTok{+}
  \FunctionTok{ylab}\NormalTok{(}\StringTok{"Number of movies"}\NormalTok{) }\SpecialCharTok{+}
  \FunctionTok{scale\_y\_continuous}\NormalTok{(}\AttributeTok{labels =}\NormalTok{ comma) }\SpecialCharTok{+}
  \FunctionTok{scale\_x\_log10}\NormalTok{(}\AttributeTok{n.breaks =} \DecValTok{10}\NormalTok{) }\SpecialCharTok{+}
  \FunctionTok{theme\_economist}\NormalTok{() }\SpecialCharTok{+}
  \FunctionTok{theme}\NormalTok{(}\AttributeTok{axis.title.x =} \FunctionTok{element\_text}\NormalTok{(}\AttributeTok{vjust =} \SpecialCharTok{{-}}\DecValTok{5}\NormalTok{, }\AttributeTok{face =} \StringTok{"bold"}\NormalTok{), }
        \AttributeTok{axis.title.y =} \FunctionTok{element\_text}\NormalTok{(}\AttributeTok{vjust =} \DecValTok{10}\NormalTok{, }\AttributeTok{face =} \StringTok{"bold"}\NormalTok{), }
        \AttributeTok{plot.margin =} \FunctionTok{margin}\NormalTok{(}\FloatTok{0.7}\NormalTok{, }\FloatTok{0.5}\NormalTok{, }\DecValTok{1}\NormalTok{, }\FloatTok{1.2}\NormalTok{, }\StringTok{"cm"}\NormalTok{))}
\end{Highlighting}
\end{Shaded}

\begin{verbatim}
## `stat_bin()` using `bins = 30`. Pick better value with `binwidth`.
\end{verbatim}

\pandocbounded{\includegraphics[keepaspectratio]{capstone-movielens.report_files/figure-latex/unnamed-chunk-123-1.pdf}}

\begin{quote}
This histogram better conveys the information provided by the summary() function, where the quantiles' values state that half the movies are rated between 30 and 565 times\autocite{MRS-R-BEST}.
\end{quote}

\newpage

\subsubsection{Mathematical Description of the UME Model}\label{mathematical-description-of-the-ume-model}

\hfill\break

The author of the \emph{Course Textbook} mentioned \hyperref[textbook.23.5.movie_effects]{above} also explains that in this case one can use a linear model with a \emph{treatment effect} \(\beta_j\) for each movie, which can be interpreted as the movie effect, or the difference between the average rating for movie \(j\) and the overall average \(\mu\):

\[
Y_{i,j} = \mu + \alpha_i + \beta_j +\varepsilon_{i,j}
\]
The author then shows how to use an approximation by first computing the least square estimate \(\hat{\mu}\) and \(\hat{\alpha}_i\), and then estimating \(\hat{\beta}_j\) as the average of the residuals \(y_{i,j} - \hat{\mu} - \hat{\alpha}_i\):

\begin{Shaded}
\begin{Highlighting}[]
\NormalTok{b }\OtherTok{\textless{}{-}} \FunctionTok{colMeans}\NormalTok{(y }\SpecialCharTok{{-}}\NormalTok{ mu }\SpecialCharTok{{-}}\NormalTok{ a, }\AttributeTok{na.rm =} \ConstantTok{TRUE}\NormalTok{)}
\end{Highlighting}
\end{Shaded}

Inspired by this idea, a few support functions were developed by the author of this report, which we will use for our further analysis.

\subsubsection{UME Model: Support Functions}\label{ume-model-support-functions}

\hfill\break

\begin{noteblock}
The complete source code of the functions described in this section is available in the \href{https://github.com/AzKurban-edX-DS/Capstone-MovieLens/blob/main/r/src/support-functions/UM-effect.functions.R\#L1}{UME Model Support Functions} section of the \href{https://github.com/AzKurban-edX-DS/Capstone-MovieLens/blob/main/r/src/support-functions/UM-effect.functions.R}{UM-effect.functions.R} script on \emph{GitHub}.

\end{noteblock}

\paragraph{\texorpdfstring{\href{https://github.com/AzKurban-edX-DS/Capstone-MovieLens/blob/main/r/src/support-functions/UM-effect.functions.R\#L2}{train\_user\_movie\_effect} Function}{train\_user\_movie\_effect Function}}\label{func.train_user_movie_effect}

\hfill\break

We use this function to build and train our model using the \texttt{train\_set} dataset:

\begin{Shaded}
\begin{Highlighting}[]
\NormalTok{train\_user\_movie\_effect }\OtherTok{\textless{}{-}} \ControlFlowTok{function}\NormalTok{(train\_set, }\AttributeTok{lambda =} \DecValTok{0}\NormalTok{)\{}
  \ControlFlowTok{if}\NormalTok{ (}\FunctionTok{is.na}\NormalTok{(lambda)) \{}
    \FunctionTok{stop}\NormalTok{(}\StringTok{"Function: train\_user\_movie\_effect}
\StringTok{\textasciigrave{}lambda\textasciigrave{} is \textasciigrave{}NA\textasciigrave{}"}\NormalTok{)}
\NormalTok{  \}}

\NormalTok{  UM.effect }\OtherTok{\textless{}{-}}\NormalTok{ train\_set }\SpecialCharTok{|\textgreater{}}
    \FunctionTok{left\_join}\NormalTok{(edx.user\_effect, }\AttributeTok{by =} \StringTok{"userId"}\NormalTok{) }\SpecialCharTok{|\textgreater{}}
    \FunctionTok{mutate}\NormalTok{(}\AttributeTok{resid =}\NormalTok{ rating }\SpecialCharTok{{-}}\NormalTok{ (mu }\SpecialCharTok{+}\NormalTok{ a)) }\SpecialCharTok{|\textgreater{}} 
    \FunctionTok{group\_by}\NormalTok{(movieId) }\SpecialCharTok{|\textgreater{}}
    \FunctionTok{summarise}\NormalTok{(}\AttributeTok{b =} \FunctionTok{mean\_reg}\NormalTok{(resid, lambda), }\AttributeTok{n =} \FunctionTok{n}\NormalTok{())}
  
  \FunctionTok{stopifnot}\NormalTok{(}\SpecialCharTok{!}\FunctionTok{is.na}\NormalTok{(}\FunctionTok{mean}\NormalTok{(UM.effect}\SpecialCharTok{$}\NormalTok{b)))}
\NormalTok{  UM.effect}
\NormalTok{\}}
\end{Highlighting}
\end{Shaded}

\begin{noteblock}
The function described above accepts the \texttt{lambda} parameter, which we will need later for the \emph{Model Regularization} method. We also use the \href{https://github.com/AzKurban-edX-DS/Capstone-MovieLens/blob/main/r/src/support-functions/common-helper.functions.R\#L63}{mean\_reg} function call, which we will need later for the \emph{Regularization} techniques (for details, see the \hyperref[func.mean_reg]{mean\_reg} function description in the Section \hyperref[appndx_a.CHF.regularization]{Common Helper Functions: Regularization} section of the \hyperref[appndx_a]{Appendix} to this report).
We will explain that later in the {[}User+Movie Effect Model Regularization{]} section.
For now, we omit the \texttt{lambda} parameter, accepting its default value \texttt{lambda\ =\ 0}. In this case, the \hyperref[func.mean_reg]{mean\_reg} function is equivalent to the standard R function \href{https://www.rdocumentation.org/packages/base/versions/3.6.2/topics/mean}{base::mean}.

\end{noteblock}

\paragraph{\texorpdfstring{\texttt{train\_user\_movie\_effect.cv} Function}{train\_user\_movie\_effect.cv Function}}\label{func.train_user_movie_effect.cv}

\hfill\break

We use the \href{https://github.com/AzKurban-edX-DS/Capstone-MovieLens/blob/main/r/src/support-functions/UM-effect.functions.R\#L17}{train\_user\_movie\_effect.cv} function to build and train our model using the \texttt{5-Fold\ Cross\ Validation} method. Below, we provide the most important part of the code of that function:

\begin{Shaded}
\begin{Highlighting}[]
\NormalTok{train\_user\_movie\_effect.cv }\OtherTok{\textless{}{-}} \ControlFlowTok{function}\NormalTok{(}\AttributeTok{lambda =} \DecValTok{0}\NormalTok{)\{}
\CommentTok{\# ...}
\NormalTok{  start }\OtherTok{\textless{}{-}} \FunctionTok{put\_start\_date}\NormalTok{()}
\NormalTok{  user\_movie\_effects\_ls }\OtherTok{\textless{}{-}} \FunctionTok{lapply}\NormalTok{(edx\_CV, }\ControlFlowTok{function}\NormalTok{(cv\_fold\_dat)\{}
\NormalTok{    cv\_fold\_dat}\SpecialCharTok{$}\NormalTok{train\_set }\SpecialCharTok{|\textgreater{}} \FunctionTok{train\_user\_movie\_effect}\NormalTok{(lambda)}
\NormalTok{  \})}
  \FunctionTok{put\_end\_date}\NormalTok{(start)}
  \FunctionTok{put\_log}\NormalTok{(}\StringTok{"Function: train\_user\_movie\_effect.cv:}
\StringTok{User+Movie Effect list have been computed"}\NormalTok{)}
  
\NormalTok{  user\_movie\_effects\_united }\OtherTok{\textless{}{-}} \FunctionTok{union\_cv\_results}\NormalTok{(user\_movie\_effects\_ls)}

\NormalTok{  user\_movie\_effect }\OtherTok{\textless{}{-}}\NormalTok{ user\_movie\_effects\_united }\SpecialCharTok{|\textgreater{}}
    \FunctionTok{group\_by}\NormalTok{(movieId) }\SpecialCharTok{|\textgreater{}}
    \FunctionTok{summarise}\NormalTok{(}\AttributeTok{b =} \FunctionTok{mean}\NormalTok{(b), }\AttributeTok{n =} \FunctionTok{mean}\NormalTok{(n))}
  \CommentTok{\# ...}
\NormalTok{  user\_movie\_effect}
\NormalTok{\}}
\end{Highlighting}
\end{Shaded}

\begin{noteblock}
Here we use the function call \href{https://github.com/AzKurban-edX-DS/Capstone-MovieLens/blob/main/r/src/support-functions/data.helper.functions.R\#L432}{union\_cv\_results}, which is defined in the script \href{https://github.com/AzKurban-edX-DS/Capstone-MovieLens/blob/main/r/src/support-functions/data.helper.functions.R}{common-helper.functions.R}, to aggregate the \emph{5-Fold Cross Validation} method results (for details, see the \hyperref[func.union_cv_results]{union\_cv\_results} function description in the \hyperref[appndx_a.data_helper.functions]{Data Helper Functions} section of the \hyperref[appndx_a]{Appendix} to this report).

\end{noteblock}

\paragraph{\texorpdfstring{\texttt{calc\_user\_movie\_effect\_MSE} Function}{calc\_user\_movie\_effect\_MSE Function}}\label{calc_user_movie_effect_mse-function}

\hfill\break

The source code of the function \href{https://github.com/AzKurban-edX-DS/Capstone-MovieLens/blob/main/r/src/support-functions/UM-effect.functions.R\#L55}{calc\_user\_movie\_effect\_MSE} defined in the \href{https://github.com/AzKurban-edX-DS/Capstone-MovieLens/blob/main/r/src/support-functions/UM-effect.functions.R}{UM-effect.functions.R} script to calculate the \emph{Mean Squared Error (MSE)} of the \emph{UME Model} for the given \texttt{Test\ Set} is provided below:

\begin{Shaded}
\begin{Highlighting}[]
\NormalTok{calc\_user\_movie\_effect\_MSE }\OtherTok{\textless{}{-}} \ControlFlowTok{function}\NormalTok{(test\_set, um\_effect)\{}
\NormalTok{  mse.result }\OtherTok{\textless{}{-}}\NormalTok{ test\_set }\SpecialCharTok{|\textgreater{}}
    \FunctionTok{left\_join}\NormalTok{(edx.user\_effect, }\AttributeTok{by =} \StringTok{"userId"}\NormalTok{) }\SpecialCharTok{|\textgreater{}}
    \FunctionTok{left\_join}\NormalTok{(um\_effect, }\AttributeTok{by =} \StringTok{"movieId"}\NormalTok{) }\SpecialCharTok{|\textgreater{}}
    \FunctionTok{mutate}\NormalTok{(}\AttributeTok{resid =}\NormalTok{ rating }\SpecialCharTok{{-}} \FunctionTok{clamp}\NormalTok{(mu }\SpecialCharTok{+}\NormalTok{ a }\SpecialCharTok{+}\NormalTok{ b)) }\SpecialCharTok{|\textgreater{}} 
    \FunctionTok{pull}\NormalTok{(resid) }\SpecialCharTok{|\textgreater{}} \FunctionTok{mse}\NormalTok{()}
  
  \FunctionTok{stopifnot}\NormalTok{(}\SpecialCharTok{!}\FunctionTok{is.na}\NormalTok{(mse.result))}
\NormalTok{  mse.result}
\NormalTok{\}}
\end{Highlighting}
\end{Shaded}

\paragraph{\texorpdfstring{\texttt{calc\_user\_movie\_effect\_MSE.cv} Function}{calc\_user\_movie\_effect\_MSE.cv Function}}\label{calc_user_movie_effect_mse.cv-function}

\hfill\break

The source code of the function \href{https://github.com/AzKurban-edX-DS/Capstone-MovieLens/blob/main/r/src/support-functions/UM-effect.functions.R\#L72}{calc\_user\_movie\_effect\_MSE.cv} defined in the \href{https://github.com/AzKurban-edX-DS/Capstone-MovieLens/blob/main/r/src/support-functions/UM-effect.functions.R}{UM-effect.functions.R} script to calculate the \emph{5-Fold Cross Validation} \textbf{MSE}\_ result of the \emph{UME Model} is provided below:

\begin{Shaded}
\begin{Highlighting}[]
\NormalTok{calc\_user\_movie\_effect\_MSE.cv }\OtherTok{\textless{}{-}} \ControlFlowTok{function}\NormalTok{(um\_effect)\{}
  \FunctionTok{put\_log}\NormalTok{(}\StringTok{"Function: user\_movie\_effects\_MSE.cv:}
\StringTok{Computing the RMSE taking into account User+Movie Effects..."}\NormalTok{)}
\NormalTok{  start }\OtherTok{\textless{}{-}} \FunctionTok{put\_start\_date}\NormalTok{()}
\NormalTok{  user\_movie\_effects\_MSEs }\OtherTok{\textless{}{-}} \FunctionTok{sapply}\NormalTok{(edx\_CV, }\ControlFlowTok{function}\NormalTok{(cv\_fold\_dat)\{}
\NormalTok{    cv\_fold\_dat}\SpecialCharTok{$}\NormalTok{validation\_set }\SpecialCharTok{|\textgreater{}} \FunctionTok{calc\_user\_movie\_effect\_MSE}\NormalTok{(um\_effect)}
\NormalTok{  \})}
  \FunctionTok{put\_end\_date}\NormalTok{(start)}
  
  \FunctionTok{put\_log1}\NormalTok{(}\StringTok{"Function: user\_movie\_effects\_MSE.cv:}
\StringTok{MSE values have been plotted for the \%1{-}Fold Cross Validation samples."}\NormalTok{, }
\NormalTok{           CVFolds\_N)}
  
  \FunctionTok{mean}\NormalTok{(user\_movie\_effects\_MSEs)}
\NormalTok{\}}
\end{Highlighting}
\end{Shaded}

\paragraph{\texorpdfstring{\texttt{calc\_user\_movie\_effect\_RMSE} Function}{calc\_user\_movie\_effect\_RMSE Function}}\label{calc_user_movie_effect_rmse-function}

\hfill\break

The source code of the function \href{https://github.com/AzKurban-edX-DS/Capstone-MovieLens/blob/main/r/src/support-functions/UM-effect.functions.R\#L51}{calc\_user\_movie\_effect\_RMSE} defined in the \href{https://github.com/AzKurban-edX-DS/Capstone-MovieLens/blob/main/r/src/support-functions/UM-effect.functions.R}{UM-effect.functions.R} script to calculate the \texttt{Root\ Mean\ Squared\ Error\ (RMSE)} of the \texttt{UME\ Model} for the given \texttt{Test\ Set} is provided below:

\begin{Shaded}
\begin{Highlighting}[]
\NormalTok{calc\_user\_movie\_effect\_RMSE }\OtherTok{\textless{}{-}} \ControlFlowTok{function}\NormalTok{(test\_set, um\_effect)\{}
\NormalTok{  mse }\OtherTok{\textless{}{-}}\NormalTok{ test\_set }\SpecialCharTok{|\textgreater{}} \FunctionTok{calc\_user\_movie\_effect\_MSE}\NormalTok{(um\_effect)}
  \FunctionTok{sqrt}\NormalTok{(mse)}
\NormalTok{\}}
\end{Highlighting}
\end{Shaded}

\paragraph{\texorpdfstring{\texttt{calc\_user\_movie\_effect\_RMSE.cv} Function}{calc\_user\_movie\_effect\_RMSE.cv Function}}\label{func.calc_user_movie_effect_RMSE.cv}

\hfill\break

The source code of the function \href{https://github.com/AzKurban-edX-DS/Capstone-MovieLens/blob/main/r/src/support-functions/UM-effect.functions.R\#L65}{calc\_user\_movie\_effect\_RMSE.cv} defined in the \href{https://github.com/AzKurban-edX-DS/Capstone-MovieLens/blob/main/r/src/support-functions/UM-effect.functions.R}{UM-effect.functions.R} script to calculate the \texttt{5-Fold\ Cross\ Validation} RMSE result of the \texttt{UME\ Model} is provided below:

\begin{Shaded}
\begin{Highlighting}[]
\NormalTok{calc\_user\_movie\_effect\_RMSE.cv }\OtherTok{\textless{}{-}} \ControlFlowTok{function}\NormalTok{(um\_effect)\{}
\NormalTok{  user\_movie\_effects\_MSE }\OtherTok{\textless{}{-}} \FunctionTok{calc\_user\_movie\_effect\_MSE.cv}\NormalTok{(um\_effect)}
\NormalTok{  um\_effect\_RMSE }\OtherTok{\textless{}{-}} \FunctionTok{sqrt}\NormalTok{(user\_movie\_effects\_MSE)}
  \FunctionTok{put\_log2}\NormalTok{(}\StringTok{"Function: user\_movie\_effects\_RMSE.cv:}
\StringTok{\%1{-}Fold Cross Validation ultimate RMSE: \%2"}\NormalTok{, CVFolds\_N, um\_effect\_RMSE)}
\NormalTok{  um\_effect\_RMSE}
\NormalTok{\}}
\end{Highlighting}
\end{Shaded}

\newpage

\subsubsection{UME Model Building}\label{ume-model-building}

\hfill\break

\begin{noteblock}
The complete source code of builing and training the current model is available in the \href{https://github.com/AzKurban-edX-DS/Capstone-MovieLens/blob/main/r/src/capstone-movielens.main.R\#L721}{Model building: User+Movie Effect} section of the \href{https://github.com/AzKurban-edX-DS/Capstone-MovieLens/blob/main/r/src/capstone-movielens.main.R}{capstone-movielens.main.R} script on \emph{GitHub}.

\end{noteblock}

Below, we provide the most significant part of the code for training our model using the \texttt{5-Fold\ Cross\ Validation} method:

\begin{Shaded}
\begin{Highlighting}[]
\NormalTok{  cv.UM\_effect }\OtherTok{\textless{}{-}} \FunctionTok{train\_user\_movie\_effect.cv}\NormalTok{()}
\end{Highlighting}
\end{Shaded}

\begin{Shaded}
\begin{Highlighting}[]
\FunctionTok{str}\NormalTok{(cv.UM\_effect)}
\end{Highlighting}
\end{Shaded}

\begin{verbatim}
## tibble [10,677 x 3] (S3: tbl_df/tbl/data.frame)
##  $ movieId: int [1:10677] 1 2 3 4 5 6 7 8 9 10 ...
##  $ b      : num [1:10677] 0.335 -0.306 -0.365 -0.598 -0.444 ...
##  $ n      : num [1:10677] 18907 8593 5574 1253 5065 ...
\end{verbatim}

\begin{Shaded}
\begin{Highlighting}[]
\FunctionTok{par}\NormalTok{(}\AttributeTok{cex =} \FloatTok{0.7}\NormalTok{)}
\FunctionTok{hist}\NormalTok{(cv.UM\_effect}\SpecialCharTok{$}\NormalTok{b, }\DecValTok{30}\NormalTok{, }\AttributeTok{xlab =} \FunctionTok{TeX}\NormalTok{(r}\StringTok{\textquotesingle{}[$\textbackslash{}hat\{beta\}\_\{j\}$)]\textquotesingle{}}\NormalTok{),}
     \AttributeTok{main =} \FunctionTok{TeX}\NormalTok{(r}\StringTok{\textquotesingle{}[Histogram of $\textbackslash{}hat\{beta\}\_\{j\}$]\textquotesingle{}}\NormalTok{))}
\end{Highlighting}
\end{Shaded}

\pandocbounded{\includegraphics[keepaspectratio]{capstone-movielens.report_files/figure-latex/unnamed-chunk-132-1.pdf}}

We can now construct predictors and see how much the \texttt{RMSE} improves\autocite{IDS2_23-5}:

\begin{Shaded}
\begin{Highlighting}[]
\NormalTok{cv.UM\_effect.RMSE }\OtherTok{\textless{}{-}} \FunctionTok{calc\_user\_movie\_effect\_RMSE.cv}\NormalTok{(cv.UM\_effect)}

\NormalTok{RMSEs.ResultTibble.UME }\OtherTok{\textless{}{-}}\NormalTok{ RMSEs.ResultTibble.UE }\SpecialCharTok{|\textgreater{}} 
  \FunctionTok{RMSEs.AddRow}\NormalTok{(}\StringTok{"User+Movie Effect Model"}\NormalTok{, cv.UM\_effect.RMSE)}
\end{Highlighting}
\end{Shaded}

\begin{Shaded}
\begin{Highlighting}[]
\FunctionTok{RMSE\_kable}\NormalTok{(RMSEs.ResultTibble.UME)}
\end{Highlighting}
\end{Shaded}

\begin{tabular}{>{\raggedright\arraybackslash}p{15em}>{\centering\arraybackslash}p{5em}>{\raggedright\arraybackslash}p{30em}}
\toprule
\textbf{Method} & \textbf{RMSE} & \textbf{Comment}\\
\midrule
Project Objective & 0.8649000 & \\
Overall Mean Rating Model & 1.0603462 & \\
User Effect Model & 0.9697962 & \\
User+Movie Effect Model & 0.8732081 & \\
\bottomrule
\end{tabular}

\newpage

\subsubsection{UME Model Regularization}\label{ume-model-regularization}

\hfill\break

\href{https://rafalab.dfci.harvard.edu/dsbook-part-2/highdim/regularization.html\#penalized-least-squares}{Section \emph{23.6 Penalized least squares} of the \emph{Course Textbook}} explains why and how we should use \emph{Penalized least squares} to improve our predictions. The author also explains that the general idea of penalized regression is to control the total variability of the movie effects:
\[
\sum_{j=1}^{n} \beta_j^2
\]
Specifically, instead of minimizing the least squares equation, we minimize an equation that adds a penalty:

\begin{equation}
\sum_{i,j} \left(y_{u,i} - \mu - \alpha_i - \beta_j \right)^2 + \lambda \sum_{j} \beta_j^2 
\label{eq:ME-penalty}
\end{equation}

The first term is just the sum of squares and the second is a penalty that gets larger when many \(\beta_i\)s are large. Using calculus, we can actually show that the values of \(\beta_i\) that minimize this equation are:

\begin{equation}
\hat{\beta}_j(\lambda) = \frac{1}{\lambda + n_j} \sum_{i=1}^{n_j} \left(Y_{i,j} - \mu - \alpha_i\right)
\label{eq:ME-regularized}
\end{equation}

where \(n_j\) is the number of ratings made for movie \(j\).

This approach will have our desired effect: when our sample size \(n_j\) is very large, we obtain a stable estimate and the penalty \(\lambda\) is effectively ignored since \(n_j+\lambda \approx n_j\). Yet when the \(n_j\) is small, then the estimate \(\hat{\beta}_i(\lambda)\) is shrunken towards 0. The larger the \(\lambda\), the more we shrink\autocite{IDS2_23-6}.

We will implement the \emph{Regularization} method on our models (starting from the current model) in following three steps:

\begin{enumerate}
\def\labelenumi{\arabic{enumi}.}
\item
  \textbf{Pre-configuration:} Preliminary determination of the optimal range of \(\lambda\) values for the \texttt{5-Fold\ Cross\ Validation} samples;
\item
  \textbf{Fine-tuning:} figuring out the value of \(\lambda\) that minimizes the model's RMSE.
\item
  \textbf{Retraining:} retraining the model with the best value of the parameter \(\lambda\) obtained in the previous step.
\end{enumerate}

\newpage

\paragraph{UME Model Regularization: Support Functions}\label{ume-model-regularization-support-functions}

\hfill\break

\begin{noteblock}
The \href{}{regularize.test\_lambda.UM\_effect.cv} function described below are defined in the \href{}{Regularization} section of the \href{}{UM-effect.functions.R} script.

\end{noteblock}

\subparagraph{\texorpdfstring{\href{https://github.com/AzKurban-edX-DS/Capstone-MovieLens/blob/main/r/src/support-functions/UM-effect.functions.R\#L96}{regularize.test\_lambda.UM\_effect.cv} Function}{regularize.test\_lambda.UM\_effect.cv Function}}\label{func.regularize.test_lambda.UM_effect.cv}

\hfill\break

This function calculates \emph{RMSE} of the \emph{UME Model} using \emph{5-Fold Cross Validation} method for the given \(\lambda\) parameter value:

\begin{Shaded}
\begin{Highlighting}[]
\NormalTok{regularize.test\_lambda.UM\_effect.cv }\OtherTok{\textless{}{-}} \ControlFlowTok{function}\NormalTok{(lambda)\{}
  \ControlFlowTok{if}\NormalTok{ (}\FunctionTok{is.na}\NormalTok{(lambda)) \{}
    \FunctionTok{stop}\NormalTok{(}\StringTok{"Function: regularize.test\_lambda.UM\_effect.cv}
\StringTok{\textasciigrave{}lambda\textasciigrave{} is \textasciigrave{}NA\textasciigrave{}"}\NormalTok{)}
\NormalTok{  \}}
\NormalTok{  um\_effect }\OtherTok{\textless{}{-}} \FunctionTok{train\_user\_movie\_effect.cv}\NormalTok{(lambda)}
  \FunctionTok{calc\_user\_movie\_effect\_RMSE.cv}\NormalTok{(um\_effect)}
\NormalTok{\}}
\end{Highlighting}
\end{Shaded}

\begin{noteblock}
Note that we reuse the function \hyperref[func.train_user_movie_effect.cv]{train\_user\_movie\_effect.cv} calling it from the \hyperref[func.regularize.test_lambda.UM_effect.cv]{regularize.test\_lambda.UM\_effect.cv}, but now with the \(\lambda\) parameter different from the default (`lambda = 0') value.

\end{noteblock}

\newpage

\paragraph{UME Model Regularization: Pre-configuration}\label{ume-model-regularization-pre-configuration}

\hfill\break

Let's perform the preconfiguration to determine the appropriate range of \(\lambda\) for subsequent fine-tuning of our current model:

We are going to use the \hyperref[func.tune.model_param]{tune.model\_param} function described in the \hyperref[appndx_a.CHF.regularization]{Common Helper Functions: Regularization} section of the \hyperref[appndx_a]{Appendix} to this report, passing the \hyperref[func.regularize.test_lambda.UM_effect.cv]{regularize.test\_lambda.UM\_effect.cv} function as the value of the \hyperref[func.tune.model_param.params.fn_tune.test.param_value]{fn\_tune.test.param\_value} parameter.

Below we provide the most significant part of the code that performs this operation:

\begin{Shaded}
\begin{Highlighting}[]
\NormalTok{lambdas }\OtherTok{\textless{}{-}} \FunctionTok{seq}\NormalTok{(}\DecValTok{0}\NormalTok{, }\DecValTok{1}\NormalTok{, }\FloatTok{0.1}\NormalTok{)}

\NormalTok{cv.UME.preset.result }\OtherTok{\textless{}{-}} 
  \FunctionTok{tune.model\_param}\NormalTok{(lambdas, regularize.test\_lambda.UM\_effect.cv)}

\FunctionTok{put\_log1}\NormalTok{(}\StringTok{"Preliminary regularization set{-}up of \textasciigrave{}lambda\textasciigrave{}s range for the UME Model has been completed}
\StringTok{for the \%1{-}Fold Cross Validation samples."}\NormalTok{,}
\NormalTok{CVFolds\_N)}
\end{Highlighting}
\end{Shaded}

\begin{Shaded}
\begin{Highlighting}[]
\FunctionTok{str}\NormalTok{(cv.UME.preset.result)}
\end{Highlighting}
\end{Shaded}

\begin{verbatim}
## List of 2
##  $ tuned.result:'data.frame':    8 obs. of  2 variables:
##   ..$ RMSE           : num [1:8] 0.873 0.873 0.873 0.873 0.873 ...
##   ..$ parameter.value: num [1:8] 0 0.1 0.2 0.3 0.4 0.5 0.6 0.7
##  $ best_result : Named num [1:2] 0.4 0.873
##   ..- attr(*, "names")= chr [1:2] "param.best_value" "best_RMSE"
\end{verbatim}

\begin{noteblock}
The complete version of the source code provided in this section can be found in the \href{https://github.com/AzKurban-edX-DS/Capstone-MovieLens/blob/main/r/src/capstone-movielens.main.R\#L814}{UME Model Regularization: Pre-configuration} section of the \href{https://github.com/AzKurban-edX-DS/Capstone-MovieLens/blob/main/r/src/capstone-movielens.main.R}{capstone-movielens.main.R} script.

\end{noteblock}

Now, let's visualize the results of the \(\lambda\) range preconfiguration:

\begin{Shaded}
\begin{Highlighting}[]
\NormalTok{cv.UME.preset.result}\SpecialCharTok{$}\NormalTok{tuned.result }\SpecialCharTok{|\textgreater{}}
  \FunctionTok{data.plot}\NormalTok{(}\AttributeTok{title =} \FunctionTok{TeX}\NormalTok{(r}\StringTok{\textquotesingle{}[UME Model Regularization: $\textbackslash{}lambda$ Range Pre{-}configuration]\textquotesingle{}}\NormalTok{),}
              \AttributeTok{xname =} \StringTok{"parameter.value"}\NormalTok{, }
              \AttributeTok{yname =} \StringTok{"RMSE"}\NormalTok{, }
              \AttributeTok{xlabel =} \FunctionTok{TeX}\NormalTok{(r}\StringTok{\textquotesingle{}[$\textbackslash{}lambda$]\textquotesingle{}}\NormalTok{), }
              \AttributeTok{ylabel =} \StringTok{"RMSE"}\NormalTok{)}
\end{Highlighting}
\end{Shaded}

\pandocbounded{\includegraphics[keepaspectratio]{capstone-movielens.report_files/figure-latex/unnamed-chunk-138-1.pdf}}

\begin{noteblock}
We use the custom data visualization function \hyperref[func.data.plot]{data.plot} described in the \hyperref[appndx_a.data_helper.functions]{Data Helper Functions} section of the \hyperref[appndx_a]{Appendix} to this report, which is defined in the \href{https://github.com/AzKurban-edX-DS/Capstone-MovieLens/blob/main/r/src/support-functions/data.helper.functions.R\#L447}{Data Visualization} section of the \href{https://github.com/AzKurban-edX-DS/Capstone-MovieLens/blob/main/r/src/support-functions/data.helper.functions.R\#L592}{data.helper.functions.R} script on \emph{GitHub}.

\end{noteblock}

\newpage

\paragraph{UME Model Regularization: Fine-tuning}\label{ume-model-regularization-fine-tuning}

\hfill\break

We are now ready to perform the fine-tuning step of our model \emph{regularization} process to determine the best value for the \(\lambda\) parameter.

Here we are going to use the \hyperref[func.model.tune.param_range]{model.tune.param\_range} function described in the \hyperref[appndx_a.CHF.regularization]{Common Helper Functions: Regularization} section of the \hyperref[appndx_a]{Appendix} to this report, passing the \hyperref[func.regularize.test_lambda.UM_effect.cv]{regularize.test\_lambda.UM\_effect.cv} function as the value of the \hyperref[func.model.tune.param_range.params.fn_tune.test.param_value]{fn\_tune.test.param\_value} parameter.

Below we provide the most significant part of the code that performs this operation:

\begin{Shaded}
\begin{Highlighting}[]
\NormalTok{endpoints }\OtherTok{\textless{}{-}} 
  \FunctionTok{get\_fine\_tune.param.endpoints}\NormalTok{(cv.UME.preset.result}\SpecialCharTok{$}\NormalTok{tuned.result)}

\NormalTok{UM\_effect.loop\_starter }\OtherTok{\textless{}{-}} \FunctionTok{c}\NormalTok{(endpoints[}\StringTok{"start"}\NormalTok{], }
\NormalTok{                            endpoints[}\StringTok{"end"}\NormalTok{], }
                            \DecValTok{8}\NormalTok{)}
\NormalTok{UM\_effect.loop\_starter}
\CommentTok{\#\textgreater{} [1] 0.3 0.5 8.0}

\NormalTok{UME.rglr.fine\_tune.cache.base\_name }\OtherTok{\textless{}{-}} \StringTok{"UME.rglr.fine{-}tune"}

\NormalTok{UME.rglr.fine\_tune.results }\OtherTok{\textless{}{-}} 
  \FunctionTok{model.tune.param\_range}\NormalTok{(UM\_effect.loop\_starter,}
\NormalTok{                         UME.rglr.fine\_tune.cache.path,}
\NormalTok{                         UME.rglr.fine\_tune.cache.base\_name,}
\NormalTok{                         regularize.test\_lambda.UM\_effect.cv)}
                         \CommentTok{\#endpoint.min\_diff = 1e{-}07/4)}

\NormalTok{UME.rglr.fine\_tune.RMSE.best }\OtherTok{\textless{}{-}}\NormalTok{ UME.rglr.fine\_tune.results}\SpecialCharTok{$}\NormalTok{best\_result[}\StringTok{"best\_RMSE"}\NormalTok{]}
\end{Highlighting}
\end{Shaded}

\begin{verbatim}
## *** Fine-tuning results object data structure ***
\end{verbatim}

\begin{verbatim}
## List of 3
##  $ best_result           : Named num [1:2] 0.387 0.873
##   ..- attr(*, "names")= chr [1:2] "param.best_value" "best_RMSE"
##  $ param_values.endpoints: Named num [1:3] 3.87e-01 3.87e-01 9.54e-07
##   ..- attr(*, "names")= chr [1:3] "" "" ""
##  $ tuned.result          :'data.frame':  9 obs. of  2 variables:
##   ..$ parameter.value: num [1:9] 0.387 0.387 0.387 0.387 0.387 ...
##   ..$ RMSE           : num [1:9] 0.873 0.873 0.873 0.873 0.873 ...
\end{verbatim}

\begin{verbatim}
## *** Fine-tuning: best results ***
\end{verbatim}

\begin{verbatim}
## param.best_value        best_RMSE 
##        0.3874500        0.8732057
\end{verbatim}

\begin{noteblock}
The complete version of the source code provided in this section can be also found in the \href{https://github.com/AzKurban-edX-DS/Capstone-MovieLens/blob/main/r/src/capstone-movielens.main.R\#L867}{Fine-tuning Step of the Regularization Method for the User+Movie Model} section of the \href{https://github.com/AzKurban-edX-DS/Capstone-MovieLens/blob/main/r/src/capstone-movielens.main.R}{capstone-movielens.main.R} script on \emph{GitHub}.

\end{noteblock}

Let's visualize the fine-tuning results:

\begin{Shaded}
\begin{Highlighting}[]
\NormalTok{UME.rglr.fine\_tune.results}\SpecialCharTok{$}\NormalTok{tuned.result }\SpecialCharTok{|\textgreater{}}
  \FunctionTok{data.plot}\NormalTok{(}\AttributeTok{title =} \StringTok{"UME Model Regularization: Fine{-}tuned result"}\NormalTok{,}
              \AttributeTok{xname =} \StringTok{"parameter.value"}\NormalTok{,}
              \AttributeTok{yname =} \StringTok{"RMSE"}\NormalTok{,}
              \AttributeTok{xlabel =} \FunctionTok{TeX}\NormalTok{(r}\StringTok{\textquotesingle{}[$\textbackslash{}lambda$]\textquotesingle{}}\NormalTok{),}
              \AttributeTok{ylabel =} \FunctionTok{str\_glue}\NormalTok{(}\StringTok{"Deviation from the best RMSE value ("}\NormalTok{,}
                                \FunctionTok{as.character}\NormalTok{(}\FunctionTok{round}\NormalTok{(UME.rglr.fine\_tune.RMSE.best, }\AttributeTok{digits =} \DecValTok{7}\NormalTok{)), }
                                \StringTok{")"}\NormalTok{),}
              \AttributeTok{normalize =} \ConstantTok{TRUE}\NormalTok{)}
\end{Highlighting}
\end{Shaded}

\pandocbounded{\includegraphics[keepaspectratio]{capstone-movielens.report_files/figure-latex/unnamed-chunk-141-1.pdf}}
\newpage

\paragraph{\texorpdfstring{UME Model Regularization: Retraining Model with the best \(\lambda\)}{UME Model Regularization: Retraining Model with the best \textbackslash lambda}}\label{ume-model-regularization-retraining-model-with-the-best-lambda}

\hfill\break

Now, we can calculate the \emph{Regularized User+Movie Effect} by retraining our model on the entire \texttt{edx} dataset with the best value of the \(\lambda\) parameter we just calculated, for the definitive \emph{Root Mean Squared Error} calculation and use in subsequent models.

\begin{Shaded}
\begin{Highlighting}[]
\NormalTok{  UME.rglr.best\_lambda }\OtherTok{\textless{}{-}}\NormalTok{ best\_result[}\StringTok{"param.best\_value"}\NormalTok{]}
\NormalTok{  UME.rglr.best\_lambda}

\NormalTok{  rglr.UM\_effect }\OtherTok{\textless{}{-}} \FunctionTok{train\_user\_movie\_effect}\NormalTok{(edx, UME.rglr.best\_lambda)}
\end{Highlighting}
\end{Shaded}

\begin{verbatim}
## *** The Best Fine-tuning Results ***
\end{verbatim}

\begin{verbatim}
## param.best_value        best_RMSE 
##        0.3874500        0.8732057
\end{verbatim}

\begin{verbatim}
## *** Regularized User+Movie Effect Structure ***
\end{verbatim}

\begin{verbatim}
## tibble [10,677 x 3] (S3: tbl_df/tbl/data.frame)
##  $ movieId: int [1:10677] 1 2 3 4 5 6 7 8 9 10 ...
##  $ b      : Named num [1:10677] 0.331 -0.305 -0.364 -0.599 -0.443 ...
##   ..- attr(*, "names")= chr [1:10677] "param.best_value" "param.best_value" "param.best_value" "param.best_value" ...
##  $ n      : int [1:10677] 23790 10779 7028 1577 6400 12346 7259 821 2278 15187 ...
\end{verbatim}

\begin{verbatim}
## Regularized User+Movie Effect Model has been re-trained for the best `lambda`: 0.38745002746582.
\end{verbatim}

\begin{noteblock}
The complete version of the source code provided in this section are available in the \href{https://github.com/AzKurban-edX-DS/Capstone-MovieLens/blob/main/r/src/capstone-movielens.main.R\#L910}{Re-training Regularized User+Movie Effect Model for the best \(\lambda\)} section of the \href{https://github.com/AzKurban-edX-DS/Capstone-MovieLens/blob/main/r/src/capstone-movielens.main.R\#L910}{capstone-movielens.main.R} script on \emph{GitHub}.

\end{noteblock}

We calculate the \emph{Root Mean Squared Error} for the ultimately computed \emph{User+Movie Effect} using \hyperref[func.calc_user_movie_effect_RMSE.cv]{calc\_user\_movie\_effect\_RMSE.cv} function described above as follows:

\begin{Shaded}
\begin{Highlighting}[]
\NormalTok{UME.rglr.retrain.RMSE }\OtherTok{\textless{}{-}} \FunctionTok{calc\_user\_movie\_effect\_RMSE.cv}\NormalTok{(rglr.UM\_effect)}
\end{Highlighting}
\end{Shaded}

\begin{Shaded}
\begin{Highlighting}[]
\FunctionTok{print\_log1}\NormalTok{(}\StringTok{"The best RMSE after being regularized: \%1"}\NormalTok{,}
\NormalTok{         UME.rglr.retrain.RMSE)}
\end{Highlighting}
\end{Shaded}

\begin{verbatim}
## The best RMSE after being regularized: 0.872972999076497
\end{verbatim}

Finally, we add the definitive result for the current model to our \emph{Result Table}:

\begin{Shaded}
\begin{Highlighting}[]
\NormalTok{RMSEs.ResultTibble.rglr.UME }\OtherTok{\textless{}{-}}\NormalTok{ RMSEs.ResultTibble.UME }\SpecialCharTok{|\textgreater{}} 
  \FunctionTok{RMSEs.AddRow}\NormalTok{(}\StringTok{"Regularized User+Movie Effect Model"}\NormalTok{, }
\NormalTok{               UME.rglr.retrain.RMSE,}
               \AttributeTok{comment =} \StringTok{"Computed for \textasciigrave{}lambda\textasciigrave{} = \%1"} \SpecialCharTok{|\textgreater{}}
                 \FunctionTok{msg.glue}\NormalTok{(UME.rglr.best\_lambda))}
\end{Highlighting}
\end{Shaded}

\begin{Shaded}
\begin{Highlighting}[]
\FunctionTok{RMSE\_kable}\NormalTok{(RMSEs.ResultTibble.rglr.UME)}
\end{Highlighting}
\end{Shaded}

\begin{tabular}{>{\raggedright\arraybackslash}p{15em}>{\centering\arraybackslash}p{5em}>{\raggedright\arraybackslash}p{30em}}
\toprule
\textbf{Method} & \textbf{RMSE} & \textbf{Comment}\\
\midrule
Project Objective & 0.8649000 & \\
Overall Mean Rating Model & 1.0603462 & \\
User Effect Model & 0.9697962 & \\
User+Movie Effect Model & 0.8732081 & \\
Regularized UME Model & 0.8729730 & Computed for `lambda` = 0.38745002746582\\
\bottomrule
\end{tabular}

\newpage

\subsection{User+Movie+Genre Effect (UMGE) Model}\label{usermoviegenre-effect-umge-model}

\hfill\break

As mentioned in \href{https://rafalab.dfci.harvard.edu/dsbook-part-2/highdim/regularization.html\#exercises}{Section 23.7: Exercises} of the \emph{Chapter ``23 Regularization'' of the Course Textbook} the \texttt{Movielens} dataset also has a genres column. This column includes every genre that applies to the movie (some movies fall under several genres)\autocite{IDS2_23-7}.

\subsubsection{Movie Genres Effect Anasysis}\label{movie-genres-effect-anasysis}

\hfill\break

The plot below shows strong evidence of a genre effect (for illustrative purposes, the plot shows only categories with more than 20, 000 ratings).

\begin{Shaded}
\begin{Highlighting}[]
\CommentTok{\# Preparing data for plotting:}
\NormalTok{genre\_ratins\_grp }\OtherTok{\textless{}{-}}\NormalTok{ edx }\SpecialCharTok{|\textgreater{}} 
  \FunctionTok{mutate}\NormalTok{(}\AttributeTok{genre\_categories =} \FunctionTok{as.factor}\NormalTok{(genres)) }\SpecialCharTok{|\textgreater{}}
  \FunctionTok{group\_by}\NormalTok{(genre\_categories) }\SpecialCharTok{|\textgreater{}}
  \FunctionTok{summarize}\NormalTok{(}\AttributeTok{n =} \FunctionTok{n}\NormalTok{(), }\AttributeTok{rating\_avg =} \FunctionTok{mean}\NormalTok{(rating), }\AttributeTok{se =} \FunctionTok{sd}\NormalTok{(rating)}\SpecialCharTok{/}\FunctionTok{sqrt}\NormalTok{(}\FunctionTok{n}\NormalTok{())) }\SpecialCharTok{|\textgreater{}}
  \FunctionTok{filter}\NormalTok{(n }\SpecialCharTok{\textgreater{}} \DecValTok{40000}\NormalTok{) }\SpecialCharTok{|\textgreater{}} 
  \FunctionTok{mutate}\NormalTok{(}\AttributeTok{genres =} \FunctionTok{reorder}\NormalTok{(genre\_categories, rating\_avg)) }\SpecialCharTok{|\textgreater{}}
  \FunctionTok{select}\NormalTok{(genres, rating\_avg, se, n)}

\FunctionTok{dim}\NormalTok{(genre\_ratins\_grp)}
\end{Highlighting}
\end{Shaded}

\begin{verbatim}
## [1] 42  4
\end{verbatim}

\begin{Shaded}
\begin{Highlighting}[]
\NormalTok{genre\_ratins\_grp\_sorted }\OtherTok{\textless{}{-}}\NormalTok{ genre\_ratins\_grp }\SpecialCharTok{|\textgreater{}} \FunctionTok{sort\_by.data.frame}\NormalTok{(}\SpecialCharTok{\textasciitilde{}}\NormalTok{ rating\_avg)}
\FunctionTok{print}\NormalTok{(genre\_ratins\_grp\_sorted)}
\end{Highlighting}
\end{Shaded}

\begin{verbatim}
## # A tibble: 42 x 4
##    genres                  rating_avg      se      n
##    <fct>                        <dbl>   <dbl>  <int>
##  1 Horror                        2.88 0.00472  68738
##  2 Children|Comedy               2.92 0.00460  63483
##  3 Action|Comedy                 3.00 0.00480  51289
##  4 Comedy|Sci-Fi                 3.08 0.00526  44599
##  5 Horror|Thriller               3.12 0.00447  75000
##  6 Action|Adventure|Comedy       3.20 0.00498  45118
##  7 Comedy                        3.24 0.00133 700889
##  8 Action|Sci-Fi                 3.25 0.00561  49733
##  9 Action|Drama|Thriller         3.27 0.00492  45246
## 10 Action|Thriller               3.27 0.00319  96535
## # i 32 more rows
\end{verbatim}

\begin{Shaded}
\begin{Highlighting}[]
\CommentTok{\# Creating plot:}
\NormalTok{genre\_ratins\_grp }\SpecialCharTok{|\textgreater{}} 
  \FunctionTok{ggplot}\NormalTok{(}\FunctionTok{aes}\NormalTok{(}\AttributeTok{x =}\NormalTok{ genres, }\AttributeTok{y =}\NormalTok{ rating\_avg, }\AttributeTok{ymin =}\NormalTok{ rating\_avg }\SpecialCharTok{{-}} \DecValTok{2}\SpecialCharTok{*}\NormalTok{se, }\AttributeTok{ymax =}\NormalTok{ rating\_avg }\SpecialCharTok{+} \DecValTok{2}\SpecialCharTok{*}\NormalTok{se)) }\SpecialCharTok{+} 
  \FunctionTok{geom\_point}\NormalTok{() }\SpecialCharTok{+}
  \FunctionTok{geom\_errorbar}\NormalTok{() }\SpecialCharTok{+} 
  \FunctionTok{ggtitle}\NormalTok{(}\StringTok{"Average rating per Genre"}\NormalTok{) }\SpecialCharTok{+}
  \FunctionTok{ylab}\NormalTok{(}\StringTok{"Average rating"}\NormalTok{) }\SpecialCharTok{+}
  \FunctionTok{theme}\NormalTok{(}\AttributeTok{axis.text.x =} \FunctionTok{element\_text}\NormalTok{(}\AttributeTok{angle =} \DecValTok{90}\NormalTok{, }\AttributeTok{hjust =} \DecValTok{1}\NormalTok{))}
\end{Highlighting}
\end{Shaded}

\pandocbounded{\includegraphics[keepaspectratio]{capstone-movielens.report_files/figure-latex/unnamed-chunk-174-1.pdf}}
Below are worst and best ratings categories:

\begin{verbatim}
## [1] "The worst ratings are for the genre category: Horror"
\end{verbatim}

\begin{verbatim}
## [1] "The best ratings are for the genre category: Comedy|Drama|Romance|War"
\end{verbatim}

Another way of visualizing a genre effect is shown in the section \href{https://www.kaggle.com/code/amirmotefaker/movie-recommendation-system-using-r-best/notebook\#Average-rating-for-each-genre}{Average rating for each genre} of the article \href{https://www.kaggle.com/code/amirmotefaker/movie-recommendation-system-using-r-best/notebook}{Movie Recommendation System using R - BEST} mentioned above\autocite{MRS-R-BEST}:

\begin{Shaded}
\begin{Highlighting}[]
\CommentTok{\# For better visibility, we reduce the data for plotting }
\CommentTok{\# while keeping the worst and best rating rows:}
\NormalTok{plot\_ind }\OtherTok{\textless{}{-}} \FunctionTok{odd}\NormalTok{(}\DecValTok{1}\SpecialCharTok{:}\FunctionTok{nrow}\NormalTok{(genre\_ratins\_grp))}
\NormalTok{plot\_dat }\OtherTok{\textless{}{-}}\NormalTok{ genre\_ratins\_grp\_sorted[plot\_ind,] }

\NormalTok{plot\_dat }\SpecialCharTok{|\textgreater{}}
  \FunctionTok{ggplot}\NormalTok{(}\FunctionTok{aes}\NormalTok{(}\AttributeTok{x =}\NormalTok{ rating\_avg, }\AttributeTok{y =}\NormalTok{ genres)) }\SpecialCharTok{+}
  \FunctionTok{ggtitle}\NormalTok{(}\StringTok{"Genre Average Rating"}\NormalTok{) }\SpecialCharTok{+}
  \FunctionTok{geom\_bar}\NormalTok{(}\AttributeTok{stat =} \StringTok{"identity"}\NormalTok{, }\AttributeTok{width =} \FloatTok{0.6}\NormalTok{, }\AttributeTok{fill =} \StringTok{"\#8888ff"}\NormalTok{) }\SpecialCharTok{+}
  \FunctionTok{xlab}\NormalTok{(}\StringTok{"Average ratings"}\NormalTok{) }\SpecialCharTok{+}
  \FunctionTok{ylab}\NormalTok{(}\StringTok{"Genres"}\NormalTok{) }\SpecialCharTok{+}
  \FunctionTok{scale\_x\_continuous}\NormalTok{(}\AttributeTok{labels =}\NormalTok{ comma, }\AttributeTok{limits =} \FunctionTok{c}\NormalTok{(}\FloatTok{0.0}\NormalTok{, }\FloatTok{5.0}\NormalTok{)) }\SpecialCharTok{+}
  \FunctionTok{theme\_economist}\NormalTok{() }\SpecialCharTok{+}
  \FunctionTok{theme}\NormalTok{(}\AttributeTok{plot.title =} \FunctionTok{element\_text}\NormalTok{(}\AttributeTok{vjust =} \FloatTok{3.5}\NormalTok{),}
        \AttributeTok{axis.title.x =} \FunctionTok{element\_text}\NormalTok{(}\AttributeTok{vjust =} \SpecialCharTok{{-}}\DecValTok{5}\NormalTok{, }\AttributeTok{face =} \StringTok{"bold"}\NormalTok{),}
        \AttributeTok{axis.title.y =} \FunctionTok{element\_text}\NormalTok{(}\AttributeTok{vjust =} \DecValTok{10}\NormalTok{, }\AttributeTok{face =} \StringTok{"bold"}\NormalTok{),}
        \AttributeTok{axis.text.x =} \FunctionTok{element\_text}\NormalTok{(}\AttributeTok{vjust =} \DecValTok{1}\NormalTok{, }\AttributeTok{hjust =} \DecValTok{1}\NormalTok{, }\AttributeTok{angle =} \DecValTok{0}\NormalTok{),}
        \AttributeTok{axis.text.y =} \FunctionTok{element\_text}\NormalTok{(}\AttributeTok{vjust =} \FloatTok{0.25}\NormalTok{, }\AttributeTok{hjust =} \DecValTok{1}\NormalTok{, }\AttributeTok{size =} \DecValTok{8}\NormalTok{),}
        \AttributeTok{plot.margin =} \FunctionTok{margin}\NormalTok{(}\FloatTok{0.7}\NormalTok{, }\FloatTok{0.5}\NormalTok{, }\DecValTok{1}\NormalTok{, }\FloatTok{1.2}\NormalTok{, }\StringTok{"cm"}\NormalTok{))}
\end{Highlighting}
\end{Shaded}

\pandocbounded{\includegraphics[keepaspectratio]{capstone-movielens.report_files/figure-latex/unnamed-chunk-176-1.pdf}}

\subsubsection{Mathematical Description of the UMGE Model}\label{mathematical-description-of-the-umge-model}

\hfill\break

To account for a \emph{genre effect} we will use the model suggested in the \href{https://rafalab.dfci.harvard.edu/dsbook-part-2/highdim/regularization.html\#exercises}{Section 23.7: Exercises} of the \emph{Chapter 23 ``Regularization'' of the Course Textbook}\autocite{IDS2_23-7}:

If we define a \emph{genre treatment effect} \(g_{i,j}\) for user's \(i\) rating of movie \(j\), we can use the following models to account for the \texttt{genre} effect:

\begin{equation}
Y_{i,j} = \mu + \alpha_i + \beta_j + g_{i,j} + \varepsilon_{i,j}
\label{eq:UMGE-model}
\end{equation}

where \(g_{i,j}\) is an \emph{aggregation function} which is explained in detail in \emph{Section 22.3: ``Review of Aggregation Functions'' of ``Recommender Systems Handbook''} (\emph{Chapter 22: ``Aggregation of Preferences in Recommender Systems''}, p.~712) book\autocite{RRSK_RS_HB}.

In the formula above \(g_{i,j}\) denotes a \emph{genre effect} for user's \(i\) rating of movie \(j\), so that:

\[
g_{i,j} = \sum_{k=1}^K x_{i,j}^k \gamma_k
\]

with \(x^k_{i,j} = 1\) if \(g_{i,j}\) includes genre \(k\), and \(x^k_{i,j} = 0\) otherwise.

Therefore, for our current model, we can compute a predicted value
\[
\hat{g}_{i,j} = g_{i,j} + \varepsilon_{i,j}
\]
as a residual:

\begin{equation}
\hat{g}_{i,j} = Y_{i,j} - (\mu + \alpha_i + \beta_j)
\label{eq:genre-effect}
\end{equation}

\subsubsection{UMGE Model: Support Functions}\label{umge-model-support-functions}

\hfill\break

\begin{noteblock}
Some of the functions described in this section accepts the \texttt{lambda} parameter, which we will need later for the \emph{Model Regularization} method. We use the \href{https://github.com/AzKurban-edX-DS/Capstone-MovieLens/blob/main/r/src/support-functions/common-helper.functions.R\#L63}{mean\_reg} function call as well, which we will also need for the \emph{Regularization} techniques to apply to our models (for details, see the \hyperref[func.mean_reg]{mean\_reg} function description in the Section \hyperref[appndx_a.CHF.regularization]{Common Helper Functions: Regularization} section of the \hyperref[appndx_a]{Appendix} to this report).
We have already explained that in the \hyperref[ume-model-regularization]{UME Model Regularization} section.
For now, we omit the \texttt{lambda} parameter, accepting its default value \texttt{lambda\ =\ 0}. Let's recall that in this case, the \hyperref[func.mean_reg]{mean\_reg} function is equivalent to the standard R function \href{https://www.rdocumentation.org/packages/base/versions/3.6.2/topics/mean}{base::mean}.

\end{noteblock}

\paragraph{\texorpdfstring{\href{https://github.com/AzKurban-edX-DS/Capstone-MovieLens/blob/main/r/src/support-functions/UMG-effect.functions.R\#L40}{train\_user\_movie\_genre\_effect} Function}{train\_user\_movie\_genre\_effect Function}}\label{func.train_user_movie_genre_effect}

\hfill\break

We use this function to build and train our model using a \emph{Train Set} (passing it to the \texttt{train.sgr} parameter), which can be any sample of the \texttt{edx.sgr} dataset (or the entire dataset itself) described above in the \hyperref[edx.sgr-object]{\texttt{edx.sgr} Object} section. Here, I only remind that it is the dataset created from the \texttt{edx} one by splitting its rows to ensure each belongs to a single \emph{genre}. Here is the source code of the function:

\begin{Shaded}
\begin{Highlighting}[]
\NormalTok{train\_user\_movie\_genre\_effect }\OtherTok{\textless{}{-}} \ControlFlowTok{function}\NormalTok{(train.sgr, }\AttributeTok{lambda =} \DecValTok{0}\NormalTok{)\{}
  \ControlFlowTok{if}\NormalTok{ (}\FunctionTok{is.na}\NormalTok{(lambda)) \{}
    \FunctionTok{stop}\NormalTok{(}\StringTok{"Function: train\_user\_movie\_genre\_effect}
\StringTok{\textasciigrave{}lambda\textasciigrave{} is \textasciigrave{}NA\textasciigrave{}"}\NormalTok{)}
\NormalTok{  \}}

\NormalTok{  genre\_bias }\OtherTok{\textless{}{-}}\NormalTok{ train.sgr }\SpecialCharTok{|\textgreater{}}
      \FunctionTok{left\_join}\NormalTok{(edx.user\_effect, }\AttributeTok{by =} \StringTok{"userId"}\NormalTok{) }\SpecialCharTok{|\textgreater{}}
      \FunctionTok{left\_join}\NormalTok{(rglr.UM\_effect, }\AttributeTok{by =} \StringTok{"movieId"}\NormalTok{) }\SpecialCharTok{|\textgreater{}}
      \FunctionTok{mutate}\NormalTok{(}\AttributeTok{resid =}\NormalTok{ rating }\SpecialCharTok{{-}}\NormalTok{ (mu }\SpecialCharTok{+}\NormalTok{ a }\SpecialCharTok{+}\NormalTok{ b)) }\SpecialCharTok{|\textgreater{}}
      \FunctionTok{group\_by}\NormalTok{(genres) }\SpecialCharTok{|\textgreater{}}
      \FunctionTok{summarise}\NormalTok{(}\AttributeTok{g =} \FunctionTok{mean\_reg}\NormalTok{(resid, lambda), }\AttributeTok{n =} \FunctionTok{n}\NormalTok{())}

\NormalTok{    train.sgr }\SpecialCharTok{|\textgreater{}}
      \FunctionTok{left\_join}\NormalTok{(genre\_bias, }\AttributeTok{by =} \StringTok{"genres"}\NormalTok{) }\SpecialCharTok{|\textgreater{}}
      \FunctionTok{left\_join}\NormalTok{(rglr.UM\_effect, }\AttributeTok{by =} \StringTok{"movieId"}\NormalTok{) }\SpecialCharTok{|\textgreater{}}
      \FunctionTok{group\_by}\NormalTok{(movieId) }\SpecialCharTok{|\textgreater{}}
      \FunctionTok{summarise}\NormalTok{(}\AttributeTok{g =} \FunctionTok{mean}\NormalTok{(g))}
\NormalTok{\}}
\end{Highlighting}
\end{Shaded}

\paragraph{\texorpdfstring{\href{https://github.com/AzKurban-edX-DS/Capstone-MovieLens/blob/main/r/src/support-functions/UMG-effect.functions.R\#L59}{train\_user\_movie\_genre\_effect.cv} Function}{train\_user\_movie\_genre\_effect.cv Function}}\label{func.train_user_movie_genre_effect.cv}

\hfill\break

We use this function, defined in the \href{https://github.com/AzKurban-edX-DS/Capstone-MovieLens/blob/main/r/src/support-functions/UMG-effect.functions.R\#L59}{UMG-effect.functions.R} script, to train the current model using the \emph{5-Fold Cross Validation} method. Below, we provide a slightly simplified version of the source code of that function:

\begin{Shaded}
\begin{Highlighting}[]
\NormalTok{train\_user\_movie\_genre\_effect.cv }\OtherTok{\textless{}{-}} \ControlFlowTok{function}\NormalTok{(}\AttributeTok{lambda =} \DecValTok{0}\NormalTok{)\{}
  \CommentTok{\# ...}

\FunctionTok{put\_log1}\NormalTok{(}\StringTok{"Function \textasciigrave{}train\_user\_movie\_genre\_effect.cv\textasciigrave{}:}
\StringTok{Computing User+Movie+Genre Effects list for \%1{-}Fold Cross Validation samples..."}\NormalTok{, }
\NormalTok{           CVFolds\_N)}
  
\NormalTok{  start }\OtherTok{\textless{}{-}} \FunctionTok{put\_start\_date}\NormalTok{()}
\NormalTok{  user\_movie\_genre\_effects\_ls }\OtherTok{\textless{}{-}} \FunctionTok{lapply}\NormalTok{(kfold\_index, }\ControlFlowTok{function}\NormalTok{(fold\_i)\{}
\NormalTok{    cv\_fold\_dat }\OtherTok{\textless{}{-}}\NormalTok{ edx\_CV[[fold\_i]]}
    
    \FunctionTok{put\_log2}\NormalTok{(}\StringTok{"Processing User+Movie+Genre Effects for \%1{-}Fold Cross Validation samples (Fold \%2)..."}\NormalTok{,}
\NormalTok{             CVFolds\_N,}
\NormalTok{             fold\_i)}
\NormalTok{    umg\_effect }\OtherTok{\textless{}{-}}\NormalTok{ cv\_fold\_dat}\SpecialCharTok{$}\NormalTok{train.sgr }\SpecialCharTok{|\textgreater{}} \FunctionTok{train\_user\_movie\_genre\_effect}\NormalTok{(lambda)}
    
    
    \FunctionTok{put\_log2}\NormalTok{(}\StringTok{"User+Movie+Genre Effects have been computed for the Fold \%1 }
\StringTok{of the \%2{-}Fold Cross Validation samples."}\NormalTok{,}
\NormalTok{             fold\_i,}
\NormalTok{             CVFolds\_N)}
\NormalTok{    umg\_effect}
\NormalTok{  \})}
  \FunctionTok{put\_end\_date}\NormalTok{(start)}
  \FunctionTok{put\_log1}\NormalTok{(}\StringTok{"Function \textasciigrave{}train\_user\_movie\_genre\_effect.cv\textasciigrave{}:}
\StringTok{User+Movie+Genre Effects list has been computed for \%1{-}Fold Cross Validation samples."}\NormalTok{, }
\NormalTok{           CVFolds\_N)}

\NormalTok{  user\_movie\_genre\_effects\_united }\OtherTok{\textless{}{-}} \FunctionTok{union\_cv\_results}\NormalTok{(user\_movie\_genre\_effects\_ls)}

\NormalTok{  user\_movie\_genre\_effect }\OtherTok{\textless{}{-}}\NormalTok{ user\_movie\_genre\_effects\_united }\SpecialCharTok{|\textgreater{}}
    \FunctionTok{group\_by}\NormalTok{(movieId) }\SpecialCharTok{|\textgreater{}}
    \FunctionTok{summarise}\NormalTok{(}\AttributeTok{g =} \FunctionTok{mean}\NormalTok{(g))}
  
  \ControlFlowTok{if}\NormalTok{(lambda }\SpecialCharTok{==} \DecValTok{0}\NormalTok{) }\FunctionTok{put\_log}\NormalTok{(}\StringTok{"Function \textasciigrave{}train\_user\_movie\_genre\_effect.cv\textasciigrave{}:}
\StringTok{Training completed: User+Movie+Genre Effects model."}\NormalTok{)}
  \ControlFlowTok{else} \FunctionTok{put\_log1}\NormalTok{(}\StringTok{"Function \textasciigrave{}train\_user\_movie\_genre\_effect.cv\textasciigrave{}:}
\StringTok{Training completed: User+Movie+Genre Effects model for lambda: \%1..."}\NormalTok{,}
\NormalTok{                lambda)}
  
\NormalTok{  user\_movie\_genre\_effect}
\NormalTok{\}}
\end{Highlighting}
\end{Shaded}

\begin{noteblock}
Here we use the function call \href{https://github.com/AzKurban-edX-DS/Capstone-MovieLens/blob/main/r/src/support-functions/data.helper.functions.R\#L432}{union\_cv\_results} to aggregate the \emph{5-Fold Cross Validation} method results (for details, see the \hyperref[func.union_cv_results]{union\_cv\_results} function description in the \hyperref[appndx_a.data_helper.functions]{Data Helper Functions} section of the \hyperref[appndx_a]{Appendix} to this report).

\end{noteblock}

\paragraph{\texorpdfstring{\texttt{calc\_user\_movie\_genre\_effect\_MSE} Function}{calc\_user\_movie\_genre\_effect\_MSE Function}}\label{calc_user_movie_genre_effect_mse-function}

\hfill\break

The source code of the function \href{https://github.com/AzKurban-edX-DS/Capstone-MovieLens/blob/main/r/src/support-functions/UMG-effect.functions.R\#L110}{calc\_user\_movie\_genre\_effect\_MSE} defined in the \href{https://github.com/AzKurban-edX-DS/Capstone-MovieLens/blob/main/r/src/support-functions/UMG-effect.functions.R}{UMG-effect.functions.R} script to calculate the \emph{Mean Squared Error (MSE)} of the \emph{UMGE Model} for the given \emph{Test Set} is provided below:

\begin{Shaded}
\begin{Highlighting}[]
\NormalTok{calc\_user\_movie\_genre\_effect\_MSE }\OtherTok{\textless{}{-}} \ControlFlowTok{function}\NormalTok{(test\_set, umg\_effect)\{}
\NormalTok{  test\_set }\SpecialCharTok{|\textgreater{}}
    \FunctionTok{left\_join}\NormalTok{(edx.user\_effect, }\AttributeTok{by =} \StringTok{"userId"}\NormalTok{) }\SpecialCharTok{|\textgreater{}}
    \FunctionTok{left\_join}\NormalTok{(rglr.UM\_effect, }\AttributeTok{by =} \StringTok{"movieId"}\NormalTok{) }\SpecialCharTok{|\textgreater{}}
    \FunctionTok{left\_join}\NormalTok{(umg\_effect, }\AttributeTok{by =} \StringTok{"movieId"}\NormalTok{) }\SpecialCharTok{|\textgreater{}}
    \FunctionTok{mutate}\NormalTok{(}\AttributeTok{resid =}\NormalTok{ rating }\SpecialCharTok{{-}} \FunctionTok{clamp}\NormalTok{(mu }\SpecialCharTok{+}\NormalTok{ a }\SpecialCharTok{+}\NormalTok{ b }\SpecialCharTok{+}\NormalTok{ g)) }\SpecialCharTok{|\textgreater{}} 
    \FunctionTok{pull}\NormalTok{(resid) }\SpecialCharTok{|\textgreater{}} \FunctionTok{mse}\NormalTok{()}
\NormalTok{\}}
\end{Highlighting}
\end{Shaded}

\newpage

\paragraph{\texorpdfstring{\texttt{calc\_user\_movie\_genre\_effect\_MSE.cv} Function}{calc\_user\_movie\_genre\_effect\_MSE.cv Function}}\label{calc_user_movie_genre_effect_mse.cv-function}

\hfill\break

The source code of the function \href{https://github.com/AzKurban-edX-DS/Capstone-MovieLens/blob/main/r/src/support-functions/UMG-effect.functions.R\#L122}{calc\_user\_movie\_genre\_effect\_MSE.cv} defined in the \href{https://github.com/AzKurban-edX-DS/Capstone-MovieLens/blob/main/r/src/support-functions/UMG-effect.functions.R}{UMG-effect.functions.R} script to calculate the \emph{5-Fold Cross Validation MSE} result of the \emph{UMGE Model} is provided below:

\begin{Shaded}
\begin{Highlighting}[]
\NormalTok{calc\_user\_movie\_genre\_effect\_MSE.cv }\OtherTok{\textless{}{-}} \ControlFlowTok{function}\NormalTok{(umg\_effect)\{}
  \FunctionTok{put\_log}\NormalTok{(}\StringTok{"Computing RMSEs.ResultTibble on Validation Sets..."}\NormalTok{)}
\NormalTok{  start }\OtherTok{\textless{}{-}} \FunctionTok{put\_start\_date}\NormalTok{()}
\NormalTok{  user\_movie\_genre\_effects\_MSEs }\OtherTok{\textless{}{-}} \FunctionTok{sapply}\NormalTok{(edx\_CV, }\ControlFlowTok{function}\NormalTok{(cv\_dat)\{}
\NormalTok{    cv\_dat}\SpecialCharTok{$}\NormalTok{validation\_set }\SpecialCharTok{|\textgreater{}} \FunctionTok{calc\_user\_movie\_genre\_effect\_MSE}\NormalTok{(umg\_effect)}
\NormalTok{  \})}
  \FunctionTok{put\_end\_date}\NormalTok{(start)}
  
  \FunctionTok{plot}\NormalTok{(user\_movie\_genre\_effects\_MSEs)}
  \FunctionTok{put\_log1}\NormalTok{(}\StringTok{"MSE values have been plotted for the \%1{-}Fold Cross Validation samples."}\NormalTok{, }
\NormalTok{           CVFolds\_N)}
 
  \FunctionTok{mean}\NormalTok{(user\_movie\_genre\_effects\_MSEs)}
\NormalTok{\}}
\end{Highlighting}
\end{Shaded}

\paragraph{\texorpdfstring{\texttt{calc\_user\_movie\_genre\_effect\_RMSE} Function}{calc\_user\_movie\_genre\_effect\_RMSE Function}}\label{calc_user_movie_genre_effect_rmse-function}

\hfill\break

The source code of the function \href{https://github.com/AzKurban-edX-DS/Capstone-MovieLens/blob/main/r/src/support-functions/UMG-effect.functions.R\#L118}{calc\_user\_movie\_genre\_effect\_RMSE} defined in the \href{https://github.com/AzKurban-edX-DS/Capstone-MovieLens/blob/main/r/src/support-functions/UMG-effect.functions.R}{UMG-effect.functions.R} script to calculate the \emph{Root Mean Squared Error (RMSE)} of the \emph{UMGE Model} for the given \emph{Test Set} is provided below:

\begin{Shaded}
\begin{Highlighting}[]
\NormalTok{calc\_user\_movie\_genre\_effect\_RMSE }\OtherTok{\textless{}{-}} \ControlFlowTok{function}\NormalTok{(test\_set, umg\_effect)\{}
\NormalTok{  umg\_mse }\OtherTok{\textless{}{-}}\NormalTok{ test\_set }\SpecialCharTok{|\textgreater{}} \FunctionTok{calc\_user\_movie\_genre\_effect\_MSE}\NormalTok{(umg\_effect)}
  \FunctionTok{sqrt}\NormalTok{(umg\_mse)}
\NormalTok{\}}
\end{Highlighting}
\end{Shaded}

\paragraph{\texorpdfstring{\texttt{calc\_user\_movie\_genre\_effect\_RMSE.cv} Function}{calc\_user\_movie\_genre\_effect\_RMSE.cv Function}}\label{func.calc_user_movie_genre_effect_RMSE.cv}

\hfill\break

The source code of the function \href{https://github.com/AzKurban-edX-DS/Capstone-MovieLens/blob/main/r/src/support-functions/UMG-effect.functions.R\#L136}{calc\_user\_movie\_genre\_effect\_RMSE.cv} defined in the \href{https://github.com/AzKurban-edX-DS/Capstone-MovieLens/blob/main/r/src/support-functions/UMG-effect.functions.R}{UMG-effect.functions.R} script to calculate the \emph{5-Fold Cross Validation \textbf{RMSE}} result of the \emph{UMGE Model} is provided below:

\begin{Shaded}
\begin{Highlighting}[]
\NormalTok{calc\_user\_movie\_genre\_effect\_RMSE.cv }\OtherTok{\textless{}{-}} \ControlFlowTok{function}\NormalTok{(umg\_effect)\{}
\NormalTok{  umg\_effect\_RMSE }\OtherTok{\textless{}{-}} \FunctionTok{sqrt}\NormalTok{(}\FunctionTok{calc\_user\_movie\_genre\_effect\_MSE.cv}\NormalTok{(umg\_effect))}
  \FunctionTok{put\_log2}\NormalTok{(}\StringTok{"\%1{-}Fold Cross Validation ultimate RMSE: \%2"}\NormalTok{, }
\NormalTok{           CVFolds\_N, }
\NormalTok{           umg\_effect\_RMSE)}
  
\NormalTok{  umg\_effect\_RMSE}
\NormalTok{\}}
\end{Highlighting}
\end{Shaded}

\newpage

\subsubsection{UMGE Model Building}\label{umge-model-building}

\hfill\break

\begin{noteblock}
The complete source code of builing and training the current model is available in the \href{https://github.com/AzKurban-edX-DS/Capstone-MovieLens/blob/main/r/src/capstone-movielens.main.R\#L1149}{Model building: User+Movie+Genre Effect} section of the \href{https://github.com/AzKurban-edX-DS/Capstone-MovieLens/blob/main/r/src/capstone-movielens.main.R}{capstone-movielens.main.R} script on \emph{GitHub}.

\end{noteblock}

Below, we provide the most significant part of the code for training our model using the \texttt{5-Fold\ Cross\ Validation} method:

\begin{Shaded}
\begin{Highlighting}[]
\NormalTok{cv.UMG\_effect }\OtherTok{\textless{}{-}} \FunctionTok{train\_user\_movie\_genre\_effect.cv}\NormalTok{()}
\end{Highlighting}
\end{Shaded}

\begin{Shaded}
\begin{Highlighting}[]
\FunctionTok{str}\NormalTok{(cv.UMG\_effect)}
\end{Highlighting}
\end{Shaded}

\begin{verbatim}
## tibble [10,677 x 2] (S3: tbl_df/tbl/data.frame)
##  $ movieId: int [1:10677] 1 2 3 4 5 6 7 8 9 10 ...
##  $ g      : num [1:10677] -9.28e-06 -7.23e-05 -1.07e-04 -6.46e-05 4.56e-05 ...
\end{verbatim}

\begin{Shaded}
\begin{Highlighting}[]
\FunctionTok{par}\NormalTok{(}\AttributeTok{cex =} \FloatTok{0.7}\NormalTok{)}
\FunctionTok{hist}\NormalTok{(cv.UMG\_effect}\SpecialCharTok{$}\NormalTok{g, }\DecValTok{30}\NormalTok{, }\AttributeTok{xlab =} \FunctionTok{TeX}\NormalTok{(r}\StringTok{\textquotesingle{}[$\textbackslash{}hat\{g\}\_\{i,j\}$]\textquotesingle{}}\NormalTok{),}
     \AttributeTok{main =} \FunctionTok{TeX}\NormalTok{(r}\StringTok{\textquotesingle{}[Histogram of $\textbackslash{}hat\{g\}\_\{i,j\}$]\textquotesingle{}}\NormalTok{))}
\end{Highlighting}
\end{Shaded}

\pandocbounded{\includegraphics[keepaspectratio]{capstone-movielens.report_files/figure-latex/hist.cv.UMG_effect-1.pdf}}

\newpage

We can now construct predictors and calculate the \emph{RMSE} of the current model using the
\hyperref[func.calc_user_movie_genre_effect_RMSE.cv]{calc\_user\_movie\_genre\_effect\_RMSE.cv} function described above:

\begin{Shaded}
\begin{Highlighting}[]
\NormalTok{cv.UMG\_effect.RMSE }\OtherTok{\textless{}{-}} \FunctionTok{calc\_user\_movie\_genre\_effect\_RMSE.cv}\NormalTok{(cv.UMG\_effect)}

\NormalTok{RMSEs.ResultTibble.UMGE }\OtherTok{\textless{}{-}}\NormalTok{ RMSEs.ResultTibble.rglr.UME }\SpecialCharTok{|\textgreater{}} 
  \FunctionTok{RMSEs.AddRow}\NormalTok{(}\StringTok{"User+Movie+Genre Effect (UMGE) Model"}\NormalTok{, cv.UMG\_effect.RMSE)}
\end{Highlighting}
\end{Shaded}

\begin{Shaded}
\begin{Highlighting}[]
\FunctionTok{RMSE\_kable}\NormalTok{(RMSEs.ResultTibble.UMGE)}
\end{Highlighting}
\end{Shaded}

\begin{tabular}{>{\raggedright\arraybackslash}p{15em}>{\centering\arraybackslash}p{5em}>{\raggedright\arraybackslash}p{30em}}
\toprule
\textbf{Method} & \textbf{RMSE} & \textbf{Comment}\\
\midrule
Project Objective & 0.8649000 & \\
Overall Mean Rating Model & 1.0603462 & \\
User Effect Model & 0.9697962 & \\
User+Movie Effect Model & 0.8732081 & \\
Regularized UME Model & 0.8729730 & Computed for `lambda` = 0.38745002746582\\
\addlinespace
UMGE Model & 0.8729730 & User+Movie+Genre Effect (UMGE) Model\\
\bottomrule
\end{tabular}

\begin{noteblock}
Unfortunately, for some reason, we do not see any improvement here yet.

\end{noteblock}

\newpage

\subsubsection{UMGE Model Regularization}\label{umge-model-regularization}

\hfill\break

We have already explained the idea of the \emph{Linear Model Regularization} in the \hyperref[ume-model-regularization]{UME Model Regularization} section above. Let's extend the concept outlined there to our current model.

In this case, the formula \eqref{eq:ME-penalty} for adding a penalty takes the form:

\begin{equation}
\sum_{i,j} \left(y_{u,i} - \mu - \alpha_i - \beta_j - g_{i,j}\right)^2 + \lambda \sum_{i,j} g_{i,j}^2
\label{eq:GE-penalty}
\end{equation}

And the formula \eqref{eq:ME-regularized} for calculating the values of the \emph{treatment effect} that minimizes the equation will take the form:

\begin{equation}
\hat{g}_{i,j}(\lambda) = \frac{1}{\lambda + n_g} \sum_{r=1}^{n_{g}} \left(Y_{i,j} - \mu - \alpha_i - \beta_j\right)
\label{eq:GE-regularized}
\end{equation}

where \(n_{g}\) is the number of ratings made for genre \(g\).

As stated in the \hyperref[ume-model-regularization]{UME Model Regularization} section, we implement the \emph{Regularization} method for our models in the following three steps:

\begin{enumerate}
\def\labelenumi{\arabic{enumi}.}
\item
  \textbf{Pre-configuration:} Preliminary determination of the optimal range of \(\lambda\) values for the \texttt{5-Fold\ Cross\ Validation} samples;
\item
  \textbf{Fine-tuning:} figuring out the value of \(\lambda\) that minimizes the model's RMSE.
\item
  \textbf{Retraining:} retraining the model with the best value of the parameter \(\lambda\) obtained in the previous step.
\end{enumerate}

\newpage

\paragraph{UMGE Model Regularization: Support Functions}\label{umge-model-regularization-support-functions}

\hfill\break

\begin{noteblock}
The \href{https://github.com/AzKurban-edX-DS/Capstone-MovieLens/blob/main/r/src/support-functions/UMG-effect.functions.R\#L146}{regularize.test\_lambda.UMG\_effect.cv} function described below are defined in the \href{https://github.com/AzKurban-edX-DS/Capstone-MovieLens/blob/main/r/src/support-functions/UMG-effect.functions.R\#L145}{Regularization} section of the \href{https://github.com/AzKurban-edX-DS/Capstone-MovieLens/blob/main/r/src/support-functions/UMG-effect.functions.R\#L146}{UM-effect.functions.R} script.

\end{noteblock}

\subparagraph{\texorpdfstring{\href{https://github.com/AzKurban-edX-DS/Capstone-MovieLens/blob/main/r/src/support-functions/UMG-effect.functions.R\#L146}{regularize.test\_lambda.UMG\_effect.cv} Function}{regularize.test\_lambda.UMG\_effect.cv Function}}\label{func.regularize.test_lambda.UMG_effect.cv}

\hfill\break

This function calculates \emph{RMSE} of the \emph{UMGE Model} using \emph{5-Fold Cross Validation} method for the given \(\lambda\) parameter value:

\begin{Shaded}
\begin{Highlighting}[]
\NormalTok{regularize.test\_lambda.UMG\_effect.cv }\OtherTok{\textless{}{-}} \ControlFlowTok{function}\NormalTok{(lambda)\{}
  \ControlFlowTok{if}\NormalTok{ (}\FunctionTok{is.na}\NormalTok{(lambda)) \{}
    \FunctionTok{stop}\NormalTok{(}\StringTok{"Function: regularize.test\_lambda.user\_movie\_genre\_effect.cv}
\StringTok{\textasciigrave{}lambda\textasciigrave{} is \textasciigrave{}NA\textasciigrave{}"}\NormalTok{)}
\NormalTok{  \}}
  
\NormalTok{  umg\_effect }\OtherTok{\textless{}{-}} \FunctionTok{train\_user\_movie\_genre\_effect.cv}\NormalTok{(lambda)}
  \FunctionTok{calc\_user\_movie\_genre\_effect\_RMSE.cv}\NormalTok{(umg\_effect)}
\NormalTok{\}}
\end{Highlighting}
\end{Shaded}

\begin{noteblock}
Note that we reuse the function \hyperref[func.train_user_movie_genre_effect.cv]{train\_user\_movie\_genre\_effect.cv} calling it from the \hyperref[func.regularize.test_lambda.UMG_effect.cv]{regularize.test\_lambda.UMG\_effect.cv}, but now with the \(\lambda\) parameter different from the default (`lambda = 0') value.

\end{noteblock}

\newpage

\paragraph{UMGE Model Regularization: Pre-configuration}\label{umge-model-regularization-pre-configuration}

\hfill\break

Let's perform the preconfiguration to determine the appropriate range of \(\lambda\) for subsequent fine-tuning of our current model:

We are going to use the \hyperref[func.tune.model_param]{tune.model\_param} function described in the \hyperref[appndx_a.CHF.regularization]{Common Helper Functions: Regularization} section of the \hyperref[appndx_a]{Appendix} to this report, passing the \hyperref[func.regularize.test_lambda.UMG_effect.cv]{regularize.test\_lambda.UMG\_effect.cv} function as the value of the \hyperref[func.tune.model_param.params.fn_tune.test.param_value]{fn\_tune.test.param\_value} parameter.

Below we provide the most significant part of the code that performs this operation:

\begin{Shaded}
\begin{Highlighting}[]
  \FunctionTok{put\_log1}\NormalTok{(}\StringTok{"Preliminary setting{-}up of \textasciigrave{}lambda\textasciigrave{}s range for \%1{-}Fold Cross Validation samples..."}\NormalTok{,}
\NormalTok{           CVFolds\_N)}
  
\NormalTok{  start }\OtherTok{\textless{}{-}} \FunctionTok{put\_start\_date}\NormalTok{()}
\NormalTok{  lambdas }\OtherTok{\textless{}{-}} \FunctionTok{seq}\NormalTok{(}\DecValTok{0}\NormalTok{, }\FloatTok{0.2}\NormalTok{, }\FloatTok{0.01}\NormalTok{)}
\NormalTok{  cv.UMGE.preset.result }\OtherTok{\textless{}{-}} 
    \FunctionTok{tune.model\_param}\NormalTok{(lambdas, regularize.test\_lambda.UMG\_effect.cv)}
  \FunctionTok{put\_end\_date}\NormalTok{(start)}
  \FunctionTok{put\_log1}\NormalTok{(}\StringTok{"Preliminary regularization set{-}up of \textasciigrave{}lambda\textasciigrave{}s range for the UMGE Model has been completed}
\StringTok{for the \%1{-}Fold Cross Validation samples."}\NormalTok{,}
\NormalTok{CVFolds\_N)}
\end{Highlighting}
\end{Shaded}

\begin{Shaded}
\begin{Highlighting}[]
\FunctionTok{str}\NormalTok{(cv.UMGE.preset.result)}
\end{Highlighting}
\end{Shaded}

\begin{verbatim}
## List of 2
##  $ tuned.result:'data.frame':    8 obs. of  2 variables:
##   ..$ RMSE           : num [1:8] 0.873 0.873 0.873 0.873 0.873 ...
##   ..$ parameter.value: num [1:8] 0 0.01 0.02 0.03 0.04 0.05 0.06 0.07
##  $ best_result : Named num [1:2] 0.04 0.873
##   ..- attr(*, "names")= chr [1:2] "param.best_value" "best_RMSE"
\end{verbatim}

\begin{Shaded}
\begin{Highlighting}[]
\NormalTok{cv.UMGE.preset.result}\SpecialCharTok{$}\NormalTok{best\_result}
\end{Highlighting}
\end{Shaded}

\begin{verbatim}
## param.best_value        best_RMSE 
##         0.040000         0.872973
\end{verbatim}

\begin{noteblock}
The complete version of the source code provided in this section can be found in the \href{https://github.com/AzKurban-edX-DS/Capstone-MovieLens/blob/main/r/src/capstone-movielens.main.R\#L814}{UMGE Model Regularization: Pre-configuration} section of the \href{https://github.com/AzKurban-edX-DS/Capstone-MovieLens/blob/main/r/src/capstone-movielens.main.R}{capstone-movielens.main.R} script.

\end{noteblock}

Now, let's visualize the results of the \(\lambda\) range preconfiguration:

\begin{Shaded}
\begin{Highlighting}[]
\NormalTok{cv.UMGE.preset.result}\SpecialCharTok{$}\NormalTok{tuned.result }\SpecialCharTok{|\textgreater{}}
  \FunctionTok{data.plot}\NormalTok{(}\AttributeTok{title =} \FunctionTok{TeX}\NormalTok{(r}\StringTok{\textquotesingle{}[UMGE Model Regularization: $\textbackslash{}lambda$ Range Pre{-}configuration]\textquotesingle{}}\NormalTok{),}
              \AttributeTok{xname =} \StringTok{"parameter.value"}\NormalTok{, }
              \AttributeTok{yname =} \StringTok{"RMSE"}\NormalTok{, }
              \AttributeTok{xlabel =} \FunctionTok{TeX}\NormalTok{(r}\StringTok{\textquotesingle{}[$\textbackslash{}lambda$]\textquotesingle{}}\NormalTok{), }
              \AttributeTok{ylabel =} \FunctionTok{str\_glue}\NormalTok{(}\StringTok{"Deviation from the best RMSE value ("}\NormalTok{,}
                                \FunctionTok{as.character}\NormalTok{(}\FunctionTok{round}\NormalTok{(cv.UMGE.preset.result}\SpecialCharTok{$}\NormalTok{best\_result[}\StringTok{"best\_RMSE"}\NormalTok{], }
                                                   \AttributeTok{digits =} \DecValTok{7}\NormalTok{)),}
                                \StringTok{")"}\NormalTok{),}
              \AttributeTok{normalize =} \ConstantTok{TRUE}\NormalTok{)}
\end{Highlighting}
\end{Shaded}

\pandocbounded{\includegraphics[keepaspectratio]{capstone-movielens.report_files/figure-latex/unnamed-chunk-188-1.pdf}}

\begin{noteblock}
We use the custom data visualization function \hyperref[func.data.plot]{data.plot} described in the \hyperref[appndx_a.data_helper.functions]{Data Helper Functions} section of the \hyperref[appndx_a]{Appendix} to this report, which is defined in the \href{https://github.com/AzKurban-edX-DS/Capstone-MovieLens/blob/main/r/src/support-functions/data.helper.functions.R\#L447}{Data Visualization} section of the \href{https://github.com/AzKurban-edX-DS/Capstone-MovieLens/blob/main/r/src/support-functions/data.helper.functions.R\#L592}{data.helper.functions.R} script on \emph{GitHub}.

\end{noteblock}

\newpage

\paragraph{UMGE Model Regularization: Fine-tuning}\label{umge-model-regularization-fine-tuning}

\hfill\break

We are now ready to perform the fine-tuning step of our model \emph{regularization} process to determine the best value for the \(\lambda\) parameter.

Here we are going to use the \hyperref[func.model.tune.param_range]{model.tune.param\_range} function described in the \hyperref[appndx_a.CHF.regularization]{Common Helper Functions: Regularization} section of the \hyperref[appndx_a]{Appendix} to this report, passing the \hyperref[func.regularize.test_lambda.UMG_effect.cv]{regularize.test\_lambda.UMG\_effect.cv} function as the value of the \hyperref[func.model.tune.param_range.params.fn_tune.test.param_value]{fn\_tune.test.param\_value} parameter.

Below we provide the most significant part of the code that performs this operation:

\begin{Shaded}
\begin{Highlighting}[]
\NormalTok{endpoints }\OtherTok{\textless{}{-}} 
  \FunctionTok{get\_fine\_tune.param.endpoints}\NormalTok{(cv.UMGE.preset.result}\SpecialCharTok{$}\NormalTok{tuned.result)}

\NormalTok{UMG\_effect.loop\_starter }\OtherTok{\textless{}{-}} \FunctionTok{c}\NormalTok{(endpoints[}\StringTok{"start"}\NormalTok{], }
\NormalTok{                            endpoints[}\StringTok{"end"}\NormalTok{], }
                            \DecValTok{8}\NormalTok{)}
\NormalTok{UMG\_effect.loop\_starter}
\CommentTok{\#\textgreater{} [1] 0.0   0.1   8.0}

\NormalTok{UMGE.rglr.fine\_tune.cache.base\_name }\OtherTok{\textless{}{-}} \StringTok{"UMGE.rglr.fine{-}tuning"}

\NormalTok{UMGE.rglr.fine\_tune.results }\OtherTok{\textless{}{-}} 
  \FunctionTok{model.tune.param\_range}\NormalTok{(UMG\_effect.loop\_starter,}
\NormalTok{                         UMGE.rglr.fine\_tune.cache.path,}
\NormalTok{                         UMGE.rglr.fine\_tune.cache.base\_name,}
\NormalTok{                         regularize.test\_lambda.UMG\_effect.cv)}

\NormalTok{UMGE.rglr.fine\_tune.RMSE.best }\OtherTok{\textless{}{-}}\NormalTok{ UMGE.rglr.fine\_tune.results}\SpecialCharTok{$}\NormalTok{best\_result[}\StringTok{"best\_RMSE"}\NormalTok{]}
\CommentTok{\# best\_RMSE }
\CommentTok{\#  0.872973 }
\end{Highlighting}
\end{Shaded}

\begin{Shaded}
\begin{Highlighting}[]
\FunctionTok{str}\NormalTok{(UMGE.rglr.fine\_tune.results)}
\end{Highlighting}
\end{Shaded}

\begin{verbatim}
## List of 3
##  $ best_result           : Named num [1:2] 0.0359 0.873
##   ..- attr(*, "names")= chr [1:2] "param.best_value" "best_RMSE"
##  $ param_values.endpoints: Named num [1:3] 0.035 0.0425 0.000937
##   ..- attr(*, "names")= chr [1:3] "" "" ""
##  $ tuned.result          :'data.frame':  9 obs. of  2 variables:
##   ..$ parameter.value: num [1:9] 0.035 0.0359 0.0369 0.0378 0.0388 ...
##   ..$ RMSE           : num [1:9] 0.873 0.873 0.873 0.873 0.873 ...
\end{verbatim}

\begin{Shaded}
\begin{Highlighting}[]
\NormalTok{UMGE.rglr.fine\_tune.results}\SpecialCharTok{$}\NormalTok{best\_result}
\end{Highlighting}
\end{Shaded}

\begin{verbatim}
## param.best_value        best_RMSE 
##        0.0359375        0.8729730
\end{verbatim}

\begin{noteblock}
The complete version of the source code provided in this section can be also found in the \href{https://github.com/AzKurban-edX-DS/Capstone-MovieLens/blob/main/r/src/capstone-movielens.main.R\#L867}{Fine-tuning Step of the Regularization Method for the User+Movie Model} section of the \href{https://github.com/AzKurban-edX-DS/Capstone-MovieLens/blob/main/r/src/capstone-movielens.main.R}{capstone-movielens.main.R} script on \emph{GitHub}.

\end{noteblock}

Let's visualize the fine-tuning results:

\begin{Shaded}
\begin{Highlighting}[]
\NormalTok{UMGE.rglr.fine\_tune.results}\SpecialCharTok{$}\NormalTok{tuned.result }\SpecialCharTok{|\textgreater{}}
  \FunctionTok{data.plot}\NormalTok{(}\AttributeTok{title =} \StringTok{"UMGE Model Regularization: Fine{-}tuned result"}\NormalTok{,}
              \AttributeTok{xname =} \StringTok{"parameter.value"}\NormalTok{,}
              \AttributeTok{yname =} \StringTok{"RMSE"}\NormalTok{,}
              \AttributeTok{xlabel =} \FunctionTok{TeX}\NormalTok{(r}\StringTok{\textquotesingle{}[$\textbackslash{}lambda$]\textquotesingle{}}\NormalTok{),}
              \AttributeTok{ylabel =} \FunctionTok{str\_glue}\NormalTok{(}\StringTok{"Deviation from the best RMSE value ("}\NormalTok{,}
                                \FunctionTok{as.character}\NormalTok{(}\FunctionTok{round}\NormalTok{(UMGE.rglr.fine\_tune.RMSE.best, }\AttributeTok{digits =} \DecValTok{7}\NormalTok{)),}
                                \StringTok{")"}\NormalTok{),}
              \AttributeTok{normalize =} \ConstantTok{TRUE}\NormalTok{)}
\end{Highlighting}
\end{Shaded}

\pandocbounded{\includegraphics[keepaspectratio]{capstone-movielens.report_files/figure-latex/unnamed-chunk-191-1.pdf}}
\newpage

\paragraph{\texorpdfstring{UMGE Model Regularization: Retraining Model with the best \(\lambda\)}{UMGE Model Regularization: Retraining Model with the best \textbackslash lambda}}\label{umge-model-regularization-retraining-model-with-the-best-lambda}

\hfill\break

Now, we can calculate the \emph{Regularized UMG Effect} by retraining our model on the entire \texttt{edx} dataset with the best value of the \(\lambda\) parameter we just calculated, for the definitive \emph{Root Mean Squared Error} calculation and use in subsequent models.

\begin{Shaded}
\begin{Highlighting}[]
\NormalTok{  best\_result }\OtherTok{\textless{}{-}}\NormalTok{ UMGE.rglr.fine\_tune.results}\SpecialCharTok{$}\NormalTok{best\_result}
  \CommentTok{\# param.best\_value        best\_RMSE }
  \CommentTok{\#     0.03554688       0.87297303 }
  
\NormalTok{  UMGE.rglr.best\_lambda }\OtherTok{\textless{}{-}}\NormalTok{ best\_result[}\StringTok{"param.best\_value"}\NormalTok{]}
\NormalTok{  UMGE.rglr.best\_RMSE }\OtherTok{\textless{}{-}}\NormalTok{ best\_result[}\StringTok{"best\_RMSE"}\NormalTok{]}
  
  \FunctionTok{put\_log1}\NormalTok{(}\StringTok{"Re{-}training Regularized User+Movie+Genre Effect Model for the best \textasciigrave{}lambda\textasciigrave{}: \%1..."}\NormalTok{,}
\NormalTok{           UMGE.rglr.best\_lambda)}
  
\NormalTok{  rglr.UMG\_effect }\OtherTok{\textless{}{-}}\NormalTok{ edx.sgr }\SpecialCharTok{|\textgreater{}} \FunctionTok{train\_user\_movie\_genre\_effect}\NormalTok{(UMGE.rglr.best\_lambda)}
\end{Highlighting}
\end{Shaded}

\begin{Shaded}
\begin{Highlighting}[]
\FunctionTok{str}\NormalTok{(rglr.UMG\_effect)}
\end{Highlighting}
\end{Shaded}

\begin{verbatim}
## tibble [10,677 x 2] (S3: tbl_df/tbl/data.frame)
##  $ movieId: int [1:10677] 1 2 3 4 5 6 7 8 9 10 ...
##  $ g      : num [1:10677] -1.35e-05 -5.11e-05 -1.08e-04 -2.77e-05 -2.46e-04 ...
\end{verbatim}

\begin{noteblock}
The complete version of the source code provided in this section are available in the \href{https://github.com/AzKurban-edX-DS/Capstone-MovieLens/blob/main/r/src/capstone-movielens.main.R\#L910}{Re-training Regularized UMG Effect Model for the best \(\lambda\)} section of the \href{https://github.com/AzKurban-edX-DS/Capstone-MovieLens/blob/main/r/src/capstone-movielens.main.R\#L910}{capstone-movielens.main.R} script on \emph{GitHub}.

\end{noteblock}

We calculate the \emph{Root Mean Squared Error} for the ultimately computed \emph{UMG Effect} using \hyperref[func.calc_UMG_effect_RMSE.cv]{calc\_UMG\_effect\_RMSE.cv} function described above as follows:

\begin{Shaded}
\begin{Highlighting}[]
\NormalTok{  rglr.UMG\_effect.RMSE }\OtherTok{\textless{}{-}} \FunctionTok{calc\_user\_movie\_genre\_effect\_RMSE.cv}\NormalTok{(rglr.UMG\_effect)}
\end{Highlighting}
\end{Shaded}

\begin{verbatim}
## Regularized User+Movie+Genre Effect RMSE has been computed for the best `lambda = 0.0359375`: 0.872972778604307.
\end{verbatim}

Finally, we add the definitive result for the current model to our \emph{Result Table}:

\begin{Shaded}
\begin{Highlighting}[]
\NormalTok{RMSEs.ResultTibble.rglr.UMGE }\OtherTok{\textless{}{-}}\NormalTok{ RMSEs.ResultTibble.UMGE }\SpecialCharTok{|\textgreater{}} 
  \FunctionTok{RMSEs.AddRow}\NormalTok{(}\StringTok{"Regularized User+Movie+Genre Effect Model"}\NormalTok{, }
\NormalTok{               rglr.UMG\_effect.RMSE,}
               \AttributeTok{comment =} \StringTok{"Computed for \textasciigrave{}lambda\textasciigrave{} = \%1"} \SpecialCharTok{|\textgreater{}}
                 \FunctionTok{msg.glue}\NormalTok{(UMGE.rglr.best\_lambda))}
\end{Highlighting}
\end{Shaded}

\begin{Shaded}
\begin{Highlighting}[]
\FunctionTok{RMSE\_kable}\NormalTok{(RMSEs.ResultTibble.rglr.UMGE)}
\end{Highlighting}
\end{Shaded}

\begin{tabular}{>{\raggedright\arraybackslash}p{15em}>{\centering\arraybackslash}p{5em}>{\raggedright\arraybackslash}p{30em}}
\toprule
\textbf{Method} & \textbf{RMSE} & \textbf{Comment}\\
\midrule
Project Objective & 0.8649000 & \\
Overall Mean Rating Model & 1.0603462 & \\
User Effect Model & 0.9697962 & \\
User+Movie Effect Model & 0.8732081 & \\
Regularized UME Model & 0.8729730 & Computed for `lambda` = 0.38745002746582\\
\addlinespace
UMGE Model & 0.8729730 & User+Movie+Genre Effect (UMGE) Model\\
Regularized UMGE Model & 0.8729728 & Computed for `lambda` = 0.0359375\\
\bottomrule
\end{tabular}

\begin{noteblock}
As we can see, the current model still does not show much improvement after \emph{regularization}, even though the data analysis we made in the {[}Movie Genres Effect{]} section showed strong evidence of a genre effect.
It looks like we need a better model to account for a genre effect more efficiently.

Or, maybe, we have implemented the current model not quite correctly (?)

\end{noteblock}

\newpage

\subsection{User+Movie+Genre+Year Effect (UMGYE) Model}\label{usermoviegenreyear-effect-umgye-model}

\hfill\break

\subsubsection{Year Effect Analysis}\label{year-effect-analysis}

\begin{noteblock}
The \emph{Year Effect} visualization and analysis code in this section are cited from the article \href{https://www.kaggle.com/code/amirmotefaker/movie-recommendation-system-using-r-best/notebook}{Movie Recommendation System using R - BEST} mentioned earlier in this report\autocite{MRS-R-BEST}.

\end{noteblock}

\paragraph{\texorpdfstring{Yearly rating count\autocite{MRS-R-BEST}}{Yearly rating count{[}@MRS-R-BEST{]}}}\label{yearly-rating-countmrs-r-best}

\hfill\break

\begin{Shaded}
\begin{Highlighting}[]
\FunctionTok{print}\NormalTok{(edx }\SpecialCharTok{|\textgreater{}} 
  \FunctionTok{mutate}\NormalTok{(}\AttributeTok{year =} \FunctionTok{year}\NormalTok{(}\FunctionTok{as\_datetime}\NormalTok{(timestamp, }\AttributeTok{origin =} \StringTok{"1970{-}01{-}01"}\NormalTok{))) }\SpecialCharTok{|\textgreater{}}
  \FunctionTok{group\_by}\NormalTok{(year) }\SpecialCharTok{|\textgreater{}}
  \FunctionTok{summarize}\NormalTok{(}\AttributeTok{count =} \FunctionTok{n}\NormalTok{())}
\NormalTok{)}
\end{Highlighting}
\end{Shaded}

\begin{verbatim}
## # A tibble: 15 x 2
##     year   count
##    <dbl>   <int>
##  1  1995       2
##  2  1996  942772
##  3  1997  414101
##  4  1998  181634
##  5  1999  709893
##  6  2000 1144349
##  7  2001  683355
##  8  2002  524959
##  9  2003  619938
## 10  2004  691429
## 11  2005 1059277
## 12  2006  689315
## 13  2007  629168
## 14  2008  696740
## 15  2009   13123
\end{verbatim}

\paragraph{\texorpdfstring{Average rating per year plot\autocite{MRS-R-BEST}}{Average rating per year plot{[}@MRS-R-BEST{]}}}\label{average-rating-per-year-plotmrs-r-best}

\hfill\break

\begin{Shaded}
\begin{Highlighting}[]
\NormalTok{edx }\SpecialCharTok{|\textgreater{}} 
  \FunctionTok{mutate}\NormalTok{(}\AttributeTok{year =} \FunctionTok{year}\NormalTok{(}\FunctionTok{as\_datetime}\NormalTok{(timestamp, }\AttributeTok{origin =} \StringTok{"1970{-}01{-}01"}\NormalTok{))) }\SpecialCharTok{|\textgreater{}}
  \FunctionTok{group\_by}\NormalTok{(year) }\SpecialCharTok{|\textgreater{}}
  \FunctionTok{summarize}\NormalTok{(}\AttributeTok{rating\_avg =} \FunctionTok{mean}\NormalTok{(rating)) }\SpecialCharTok{|\textgreater{}}
  \FunctionTok{ggplot}\NormalTok{(}\FunctionTok{aes}\NormalTok{(}\AttributeTok{x =}\NormalTok{ year, }\AttributeTok{y =}\NormalTok{ rating\_avg)) }\SpecialCharTok{+}
  \FunctionTok{geom\_bar}\NormalTok{(}\AttributeTok{stat =} \StringTok{"identity"}\NormalTok{, }\AttributeTok{fill =} \StringTok{"\#8888ff"}\NormalTok{) }\SpecialCharTok{+} 
  \FunctionTok{ggtitle}\NormalTok{(}\StringTok{"Average rating per year"}\NormalTok{) }\SpecialCharTok{+}
  \FunctionTok{xlab}\NormalTok{(}\StringTok{"Year"}\NormalTok{) }\SpecialCharTok{+}
  \FunctionTok{ylab}\NormalTok{(}\StringTok{"Average rating"}\NormalTok{) }\SpecialCharTok{+}
  \FunctionTok{scale\_y\_continuous}\NormalTok{(}\AttributeTok{labels =}\NormalTok{ comma) }\SpecialCharTok{+} 
  \FunctionTok{theme\_economist}\NormalTok{() }\SpecialCharTok{+}
  \FunctionTok{theme}\NormalTok{(}\AttributeTok{axis.title.x =} \FunctionTok{element\_text}\NormalTok{(}\AttributeTok{vjust =} \SpecialCharTok{{-}}\DecValTok{5}\NormalTok{, }\AttributeTok{face =} \StringTok{"bold"}\NormalTok{), }
        \AttributeTok{axis.title.y =} \FunctionTok{element\_text}\NormalTok{(}\AttributeTok{vjust =} \DecValTok{10}\NormalTok{, }\AttributeTok{face =} \StringTok{"bold"}\NormalTok{), }
        \AttributeTok{plot.margin =} \FunctionTok{margin}\NormalTok{(}\FloatTok{0.7}\NormalTok{, }\FloatTok{0.5}\NormalTok{, }\DecValTok{1}\NormalTok{, }\FloatTok{1.2}\NormalTok{, }\StringTok{"cm"}\NormalTok{))}
\end{Highlighting}
\end{Shaded}

\pandocbounded{\includegraphics[keepaspectratio]{capstone-movielens.report_files/figure-latex/unnamed-chunk-220-1.pdf}}

\newpage

\subsubsection{Mathematical Description of the UMGYE Model}\label{mathematical-description-of-the-umgye-model}

\hfill\break

If we define \(\gamma(v_{i,j})\) as the \emph{year effect} for the year (here denotes by \(v_{i,j}\)) of rating a movie \(j\) by a user \(i\), the formula \eqref{eq:UMGE-model} describing the \emph{UMGE Model}, for the current model, takes the form:

\begin{equation}
Y_{i,j} = \mu + \alpha_i + \beta_j + g_{i,j} + \gamma(v_{i,j}) + \varepsilon_{i,j}
\label{eq:UMGYE-model}
\end{equation}

Therefore, the formula \eqref{eq:genre-effect} for calculation the prediction of a \emph{genre effect} as a residual, for a \emph{year effect}
\[
\hat{\gamma}(v_{i,j}) = \gamma(v_{i,j}) + \varepsilon_{i,j}
\]

takes the following form:

\begin{equation}
\hat{\gamma}(v_{i,j}) = Y_{i,j} - (\mu + \alpha_i + \beta_j + g_{i,j})
\label{eq:year-effect}
\end{equation}

\newpage

\subsubsection{UMGYE Model: Support Functions}\label{umgye-model-support-functions}

\hfill\break

\begin{noteblock}
Some of the functions described in this section accepts the \texttt{lambda} parameter, which we will need later for the \emph{Model Regularization} method. We use the \href{https://github.com/AzKurban-edX-DS/Capstone-MovieLens/blob/main/r/src/support-functions/common-helper.functions.R\#L63}{mean\_reg} function call as well, which we will also need for the \emph{Regularization} techniques to apply to our models (for details, see the \hyperref[func.mean_reg]{mean\_reg} function description in the \hyperref[appndx_a.CHF.regularization]{Common Helper Functions: Regularization} section of the \hyperref[appndx_a]{Appendix} to this report).
We have already explained that in the \hyperref[ume-model-regularization]{UME Model Regularization} section.
For now, we omit the \texttt{lambda} parameter, accepting its default value \texttt{lambda\ =\ 0}. Let's recall that in this case, the \hyperref[func.mean_reg]{mean\_reg} function is equivalent to the standard R function \href{https://www.rdocumentation.org/packages/base/versions/3.6.2/topics/mean}{base::mean}.

\end{noteblock}

\paragraph{\texorpdfstring{\href{https://github.com/AzKurban-edX-DS/Capstone-MovieLens/blob/main/r/src/support-functions/UMGY-effect.functions.R\#L2}{calc\_date\_general\_effect} Function}{calc\_date\_general\_effect Function}}\label{func.calc_date_general_effect}

\hfill\break

This is a helper function that calculates the \emph{date general effect} of the user ratings and is intended to be used in other functions and scripts (such as \hyperref[func.train_UMGY_effect]{train\_UMGY\_effect}) described below.

The following is a slightly simplified version of the function's source code:

\begin{Shaded}
\begin{Highlighting}[]
\NormalTok{calc\_date\_general\_effect }\OtherTok{\textless{}{-}} \ControlFlowTok{function}\NormalTok{(train\_set, }\AttributeTok{lambda =} \DecValTok{0}\NormalTok{)\{}
  \ControlFlowTok{if}\NormalTok{ (}\FunctionTok{is.na}\NormalTok{(lambda)) \{}
    \FunctionTok{stop}\NormalTok{(}\StringTok{"Function: calc\_date\_general\_effect}
\StringTok{\textasciigrave{}lambda\textasciigrave{} is \textasciigrave{}NA\textasciigrave{}"}\NormalTok{)}
\NormalTok{  \}}
  
  \ControlFlowTok{if}\NormalTok{(lambda }\SpecialCharTok{==} \DecValTok{0}\NormalTok{) }\FunctionTok{put\_log}\NormalTok{(}\StringTok{"Function \textasciigrave{}calc\_date\_general\_effect\textasciigrave{}:}
\StringTok{Computing Date Global Effect for given Train Set data..."}\NormalTok{)}
  \ControlFlowTok{else} \FunctionTok{put\_log1}\NormalTok{(}\StringTok{"Function \textasciigrave{}calc\_date\_general\_effect\textasciigrave{}:}
\StringTok{Computing Date Global Effect for lambda: \%1..."}\NormalTok{,}
\NormalTok{                lambda)}
\NormalTok{  dg\_effect }\OtherTok{\textless{}{-}}\NormalTok{ train\_set }\SpecialCharTok{|\textgreater{}} 
    \FunctionTok{left\_join}\NormalTok{(edx.user\_effect, }\AttributeTok{by =} \StringTok{"userId"}\NormalTok{) }\SpecialCharTok{|\textgreater{}}
    \FunctionTok{left\_join}\NormalTok{(rglr.UM\_effect, }\AttributeTok{by =} \StringTok{"movieId"}\NormalTok{) }\SpecialCharTok{|\textgreater{}}
    \FunctionTok{left\_join}\NormalTok{(rglr.UMG\_effect, }\AttributeTok{by =} \StringTok{"movieId"}\NormalTok{) }\SpecialCharTok{|\textgreater{}}
    \FunctionTok{left\_join}\NormalTok{(date\_days\_map, }\AttributeTok{by =} \StringTok{"timestamp"}\NormalTok{) }\SpecialCharTok{|\textgreater{}}
    \FunctionTok{mutate}\NormalTok{(}\AttributeTok{resid =}\NormalTok{ rating }\SpecialCharTok{{-}}\NormalTok{ (mu }\SpecialCharTok{+}\NormalTok{ a }\SpecialCharTok{+}\NormalTok{ b }\SpecialCharTok{+}\NormalTok{ g)) }\SpecialCharTok{|\textgreater{}}
  \FunctionTok{group\_by}\NormalTok{(days) }\SpecialCharTok{|\textgreater{}}
    \FunctionTok{summarise}\NormalTok{(}\AttributeTok{de =} \FunctionTok{mean\_reg}\NormalTok{(resid, lambda), }
              \AttributeTok{year =} \FunctionTok{mean}\NormalTok{(year))}
\NormalTok{  dg\_effect}
\NormalTok{\}}
\end{Highlighting}
\end{Shaded}

\begin{noteblock}
The complete version of the source code of the \href{https://github.com/AzKurban-edX-DS/Capstone-MovieLens/blob/main/r/src/support-functions/UMGY-effect.functions.R\#L2}{calc\_date\_general\_effect} function is available in the \href{https://github.com/AzKurban-edX-DS/Capstone-MovieLens/blob/main/r/src/support-functions/UMGY-effect.functions.R\#L1}{UMGYE Model Support Functions} section of the \href{https://github.com/AzKurban-edX-DS/Capstone-MovieLens/blob/main/r/src/support-functions/UMGY-effect.functions.R\#L2}{UMGY-effect.functions.R} script on \emph{GitHub}.

\end{noteblock}

\subparagraph{Return}\label{return}

A data structure containing information about \emph{Date General Effect}.

\paragraph{\texorpdfstring{\href{https://github.com/AzKurban-edX-DS/Capstone-MovieLens/blob/main/r/src/support-functions/UMGY-effect.functions.R\#L30}{calc\_date\_general\_effect.cv} Function}{calc\_date\_general\_effect.cv Function}}\label{func.calc_date_general_effect.cv}

\hfill\break

This is a helper function that calculates the \emph{date general effect} of the user ratings using the \emph{5-Fold Cross Validation} method and is intended to be used in other functions and scripts (such as \hyperref[func.train_UMGY_effect.cv]{train\_UMGY\_effect.cv}) described below.

\subparagraph{Souurce Code}\label{souurce-code}

\hfill\break

\begin{noteblock}
The complete version of the source code of the \href{https://github.com/AzKurban-edX-DS/Capstone-MovieLens/blob/main/r/src/support-functions/UMGY-effect.functions.R\#L30}{calc\_date\_general\_effect.cv} function is available in the \href{https://github.com/AzKurban-edX-DS/Capstone-MovieLens/blob/main/r/src/support-functions/UMGY-effect.functions.R\#L1}{UMGYE Model Support Functions} section of the \href{https://github.com/AzKurban-edX-DS/Capstone-MovieLens/blob/main/r/src/support-functions/UMGY-effect.functions.R\#L30}{UMGY-effect.functions.R} script on \emph{GitHub}.

\end{noteblock}

The following is a simplified version of the function's source code:

\begin{Shaded}
\begin{Highlighting}[]
\NormalTok{calc\_date\_general\_effect.cv }\OtherTok{\textless{}{-}} \ControlFlowTok{function}\NormalTok{(}\AttributeTok{lambda =} \DecValTok{0}\NormalTok{)\{}
  \ControlFlowTok{if}\NormalTok{ (}\FunctionTok{is.na}\NormalTok{(lambda)) \{}
    \FunctionTok{stop}\NormalTok{(}\StringTok{"Function: calc\_date\_general\_effect.cv}
\StringTok{\textasciigrave{}lambda\textasciigrave{} is \textasciigrave{}NA\textasciigrave{}"}\NormalTok{)}
\NormalTok{  \}}
  
\NormalTok{  date\_general\_effect\_ls }\OtherTok{\textless{}{-}} \FunctionTok{lapply}\NormalTok{(edx\_CV,  }\ControlFlowTok{function}\NormalTok{(cv\_fold\_dat)\{}
\NormalTok{    cv\_fold\_dat}\SpecialCharTok{$}\NormalTok{train\_set }\SpecialCharTok{|\textgreater{}} \FunctionTok{calc\_date\_general\_effect}\NormalTok{(lambda)}
\NormalTok{  \})}
  \FunctionTok{str}\NormalTok{(date\_general\_effect\_ls)}

\NormalTok{  date\_general\_effect\_united }\OtherTok{\textless{}{-}} \FunctionTok{union\_cv\_results}\NormalTok{(date\_general\_effect\_ls)}

\NormalTok{  date\_general\_effect\_united }\SpecialCharTok{|\textgreater{}}
    \FunctionTok{group\_by}\NormalTok{(days) }\SpecialCharTok{|\textgreater{}}
    \FunctionTok{summarise}\NormalTok{(}\AttributeTok{de =} \FunctionTok{mean}\NormalTok{(de, }\AttributeTok{na.rm =} \ConstantTok{TRUE}\NormalTok{), }\AttributeTok{year =} \FunctionTok{mean}\NormalTok{(year, }\AttributeTok{na.rm =} \ConstantTok{TRUE}\NormalTok{))}
\NormalTok{\}}
\end{Highlighting}
\end{Shaded}

\begin{noteblock}
Here we use the function call \href{https://github.com/AzKurban-edX-DS/Capstone-MovieLens/blob/main/r/src/support-functions/data.helper.functions.R\#L432}{union\_cv\_results} to aggregate the \emph{5-Fold Cross Validation} results (for details, see the \hyperref[func.union_cv_results]{union\_cv\_results} function description in the \hyperref[appndx_a.data_helper.functions]{Data Helper Functions} section of the \hyperref[appndx_a]{Appendix} to this report).

\end{noteblock}

\subparagraph{Return}\label{return-1}

A data structure containing information about \emph{Date General Effect} as result of \emph{5-Fold Cross Validation}.

\newpage

\paragraph{\texorpdfstring{\href{https://github.com/AzKurban-edX-DS/Capstone-MovieLens/blob/main/r/src/support-functions/UMGY-effect.functions.R\#L69}{calc\_UMGY\_effect} Function}{calc\_UMGY\_effect Function}}\label{func.calc_UMGY_effect}

\hfill\break

This is a helper function that calculates the \emph{UMGY effect} of the user ratings and is intended to be used in other functions and scripts (such as \hyperref[func.train_UMGY_effect]{train\_UMGY\_effect}) described below.

\subparagraph{Parameters}\label{parameters}

\hfill\break

The function accepts the only parameter \texttt{date\_general\_effect}, which is a data structure returned by any of the following functions described above:

\begin{itemize}
\tightlist
\item
  \hyperref[func.calc_date_general_effect]{calc\_date\_general\_effect};
\item
  \hyperref[func.calc_date_general_effect.cv]{calc\_date\_general\_effect.cv}.
\end{itemize}

\subparagraph{Source Code}\label{source-code}

\hfill\break

The following is the source code of the function:

\begin{Shaded}
\begin{Highlighting}[]
\NormalTok{calc\_UMGY\_effect }\OtherTok{\textless{}{-}} \ControlFlowTok{function}\NormalTok{(date\_general\_effect)\{}
\NormalTok{  date\_general\_effect }\SpecialCharTok{|\textgreater{}}
    \FunctionTok{group\_by}\NormalTok{(year) }\SpecialCharTok{|\textgreater{}}
    \FunctionTok{summarise}\NormalTok{(}\AttributeTok{ye =} \FunctionTok{mean}\NormalTok{(de, }\AttributeTok{na.rm =} \ConstantTok{TRUE}\NormalTok{))}
\NormalTok{\}}
\end{Highlighting}
\end{Shaded}

\begin{noteblock}
The source code of the \href{https://github.com/AzKurban-edX-DS/Capstone-MovieLens/blob/main/r/src/support-functions/UMGY-effect.functions.R\#L69}{calc\_UMGY\_effect} function is also available in the \href{https://github.com/AzKurban-edX-DS/Capstone-MovieLens/blob/main/r/src/support-functions/UMGY-effect.functions.R\#L1}{UMGYE Model Support Functions} section of the \href{https://github.com/AzKurban-edX-DS/Capstone-MovieLens/blob/main/r/src/support-functions/UMGY-effect.functions.R\#L69}{UMGY-effect.functions.R} script on \emph{GitHub}.

\end{noteblock}

\subparagraph{Return}\label{return-2}

A data structure containing information about \emph{User+Movie+Genre+Year (UMGY) Effect}.

\paragraph{\texorpdfstring{\href{https://github.com/AzKurban-edX-DS/Capstone-MovieLens/blob/main/r/src/support-functions/UMGY-effect.functions.R\#L74}{train\_UMGY\_effect} Function}{train\_UMGY\_effect Function}}\label{func.train_UMGY_effect}

\hfill\break

This function is used to train the current model using the \emph{Train Set} dataset passed in the \texttt{train\_set} parameter.
The following is the source code of the function:

\begin{Shaded}
\begin{Highlighting}[]
\NormalTok{train\_UMGY\_effect }\OtherTok{\textless{}{-}} \ControlFlowTok{function}\NormalTok{(train\_set, }\AttributeTok{lambda =} \DecValTok{0}\NormalTok{)\{}
  \ControlFlowTok{if}\NormalTok{ (}\FunctionTok{is.na}\NormalTok{(lambda)) \{}
    \FunctionTok{stop}\NormalTok{(}\StringTok{"Function: train\_UMGY\_effect}
\StringTok{\textasciigrave{}lambda\textasciigrave{} is \textasciigrave{}NA\textasciigrave{}"}\NormalTok{)}
\NormalTok{  \}}
  
\NormalTok{  train\_set }\SpecialCharTok{|\textgreater{}}
    \FunctionTok{calc\_date\_general\_effect}\NormalTok{(lambda) }\SpecialCharTok{|\textgreater{}}
    \FunctionTok{calc\_UMGY\_effect}\NormalTok{()}
\NormalTok{\}}
\end{Highlighting}
\end{Shaded}

\begin{noteblock}
The source code of the \href{https://github.com/AzKurban-edX-DS/Capstone-MovieLens/blob/main/r/src/support-functions/UMGY-effect.functions.R\#L74}{train\_UMGY\_effect} function is also available in the \href{https://github.com/AzKurban-edX-DS/Capstone-MovieLens/blob/main/r/src/support-functions/UMGY-effect.functions.R\#L1}{UMGYE Model Support Functions} section of the \href{https://github.com/AzKurban-edX-DS/Capstone-MovieLens/blob/main/r/src/support-functions/UMGY-effect.functions.R\#L74}{UMGY-effect.functions.R} script on \emph{GitHub}.

\end{noteblock}

\subparagraph{Return}\label{return-3}

A data structure containing information about \emph{User+Movie+Genre+Year (UMGY) Effect}.

\paragraph{\texorpdfstring{\href{https://github.com/AzKurban-edX-DS/Capstone-MovieLens/blob/main/r/src/support-functions/UMGY-effect.functions.R\#L84}{train\_UMGY\_effect.cv} Function}{train\_UMGY\_effect.cv Function}}\label{func.train_UMGY_effect.cv}

\hfill\break

This function is used to train the current model using the \emph{5-Fold Cross Validation}.
The following is the source code of the function:

\begin{Shaded}
\begin{Highlighting}[]
\NormalTok{train\_UMGY\_effect.cv }\OtherTok{\textless{}{-}} \ControlFlowTok{function}\NormalTok{(}\AttributeTok{lambda =} \DecValTok{0}\NormalTok{)\{}
  \ControlFlowTok{if}\NormalTok{ (}\FunctionTok{is.na}\NormalTok{(lambda)) \{}
    \FunctionTok{stop}\NormalTok{(}\StringTok{"Function: train\_UMGY\_effect.cv}
\StringTok{\textasciigrave{}lambda\textasciigrave{} is \textasciigrave{}NA\textasciigrave{}"}\NormalTok{)}
\NormalTok{  \}}
  
  \FunctionTok{calc\_date\_general\_effect.cv}\NormalTok{(lambda) }\SpecialCharTok{|\textgreater{}} \FunctionTok{calc\_UMGY\_effect}\NormalTok{()  }
\NormalTok{\}}
\end{Highlighting}
\end{Shaded}

\begin{noteblock}
The source code of the \href{https://github.com/AzKurban-edX-DS/Capstone-MovieLens/blob/main/r/src/support-functions/UMGY-effect.functions.R\#L84}{train\_UMGY\_effect.cv} function is also available in the \href{https://github.com/AzKurban-edX-DS/Capstone-MovieLens/blob/main/r/src/support-functions/UMGY-effect.functions.R\#L1}{UMGYE Model Support Functions} section of the \href{https://github.com/AzKurban-edX-DS/Capstone-MovieLens/blob/main/r/src/support-functions/UMGY-effect.functions.R\#L84}{UMGY-effect.functions.R} script on \emph{GitHub}.

\end{noteblock}

\subparagraph{Return}\label{return-4}

A data structure containing information about \emph{User+Movie+Genre+Year (UMGY) Effect}.

\paragraph{\texorpdfstring{\href{https://github.com/AzKurban-edX-DS/Capstone-MovieLens/blob/main/r/src/support-functions/UMGY-effect.functions.R\#L92}{calc\_UMGY\_effect\_MSE} Function}{calc\_UMGY\_effect\_MSE Function}}\label{func.calc_UMGY_effect_MSE}

\hfill\break

The function calculates the \emph{Mean Squared Error (MSE)} of the \emph{UMGYE Model} for the given \emph{Test Set} passed in the \texttt{test\_set} parameter.

\subparagraph{Parameters}\label{parameters-1}

\hfill\break

\begin{itemize}
\tightlist
\item
  \textbf{test\_set:} A \emph{Test Set} dataset used for \emph{UMGYE Model} validation;
\item
  \textbf{UMGY\_effect:} A data structure returned by any of the following functions described above:

  \begin{itemize}
  \tightlist
  \item
    \hyperref[func.train_UMGY_effect]{calc\_UMGY\_effect};
  \item
    \hyperref[func.train_UMGY_effect]{train\_UMGY\_effect};
  \item
    \hyperref[func.train_UMGY_effect.cv]{train\_UMGY\_effect.cv}.
  \end{itemize}
\end{itemize}

\newpage

\subparagraph{Source Code}\label{source-code-1}

\hfill\break

The source code of the function is provided below:

\begin{Shaded}
\begin{Highlighting}[]
\NormalTok{calc\_UMGY\_effect\_MSE }\OtherTok{\textless{}{-}} \ControlFlowTok{function}\NormalTok{(test\_set, UMGY\_effect)\{}
\NormalTok{  test\_set }\SpecialCharTok{|\textgreater{}}
    \FunctionTok{left\_join}\NormalTok{(edx.user\_effect, }\AttributeTok{by =} \StringTok{"userId"}\NormalTok{) }\SpecialCharTok{|\textgreater{}}
    \FunctionTok{left\_join}\NormalTok{(rglr.UM\_effect, }\AttributeTok{by =} \StringTok{"movieId"}\NormalTok{) }\SpecialCharTok{|\textgreater{}}
    \FunctionTok{left\_join}\NormalTok{(rglr.UMG\_effect, }\AttributeTok{by =} \StringTok{"movieId"}\NormalTok{) }\SpecialCharTok{|\textgreater{}}
    \FunctionTok{left\_join}\NormalTok{(date\_days\_map, }\AttributeTok{by =} \StringTok{"timestamp"}\NormalTok{) }\SpecialCharTok{|\textgreater{}}
    \FunctionTok{left\_join}\NormalTok{(UMGY\_effect, }\AttributeTok{by=}\StringTok{\textquotesingle{}year\textquotesingle{}}\NormalTok{) }\SpecialCharTok{|\textgreater{}}
    \FunctionTok{mutate}\NormalTok{(}\AttributeTok{resid =}\NormalTok{ rating }\SpecialCharTok{{-}} \FunctionTok{clamp}\NormalTok{(mu }\SpecialCharTok{+}\NormalTok{ a }\SpecialCharTok{+}\NormalTok{ b }\SpecialCharTok{+}\NormalTok{ g }\SpecialCharTok{+}\NormalTok{ ye)) }\SpecialCharTok{|\textgreater{}} 
    \FunctionTok{pull}\NormalTok{(resid) }\SpecialCharTok{|\textgreater{}} \FunctionTok{mse}\NormalTok{()}
\NormalTok{\}}
\end{Highlighting}
\end{Shaded}

\begin{noteblock}
The source code of the function \href{https://github.com/AzKurban-edX-DS/Capstone-MovieLens/blob/main/r/src/support-functions/UMGY-effect.functions.R\#L92}{calc\_UMGY\_effect\_MSE} is also available in the \href{https://github.com/AzKurban-edX-DS/Capstone-MovieLens/blob/main/r/src/support-functions/UMGY-effect.functions.R\#L92}{UMGYE Model Support Functions} script.

\end{noteblock}

\paragraph{\texorpdfstring{\href{https://github.com/AzKurban-edX-DS/Capstone-MovieLens/blob/main/r/src/support-functions/UMGY-effect.functions.R\#L102}{calc\_UMGY\_effect\_MSE.cv} Function}{calc\_UMGY\_effect\_MSE.cv Function}}\label{func.calc_UMGY_effect_MSE.cv}

\hfill\break

The function calculates the \emph{5-Fold Cross Validation MSE} result for the \emph{UMGY Effect} data passed in the \texttt{UMGY\_effect} parameter.

\subparagraph{Parameters}\label{parameters-2}

\hfill\break

The function accepts the only parameter: \texttt{UMGY\_effect}, which is a data structure returned by the one of the following functions described above:

\begin{itemize}
\tightlist
\item
  \hyperref[func.train_UMGY_effect]{calc\_UMGY\_effect};
\item
  \hyperref[func.train_UMGY_effect]{train\_UMGY\_effect};
\item
  \hyperref[func.train_UMGY_effect.cv]{train\_UMGY\_effect.cv}.
\end{itemize}

\subparagraph{Source Code}\label{source-code-2}

\hfill\break

The source code of the function is provided below:

\begin{Shaded}
\begin{Highlighting}[]
\NormalTok{calc\_UMGY\_effect\_MSE.cv }\OtherTok{\textless{}{-}} \ControlFlowTok{function}\NormalTok{(UMGY\_effect)\{}
\NormalTok{  start }\OtherTok{\textless{}{-}} \FunctionTok{put\_start\_date}\NormalTok{()}
\NormalTok{  UMGY\_effect\_MSEs }\OtherTok{\textless{}{-}} \FunctionTok{sapply}\NormalTok{(edx\_CV, }\ControlFlowTok{function}\NormalTok{(cv\_fold\_dat)\{}
\NormalTok{    cv\_fold\_dat}\SpecialCharTok{$}\NormalTok{validation\_set }\SpecialCharTok{|\textgreater{}} \FunctionTok{calc\_UMGY\_effect\_MSE}\NormalTok{(UMGY\_effect)}
\NormalTok{  \})}
  \FunctionTok{put\_end\_date}\NormalTok{(start)}
  \FunctionTok{put\_log1}\NormalTok{(}\StringTok{"Function: calc\_UMGY\_effect\_MSE.cv}
\StringTok{Date (Year) Effect MSE values have been computed for the \%1{-}Fold Cross Validation samples."}\NormalTok{, }
\NormalTok{           CVFolds\_N)}
  
  \FunctionTok{mean}\NormalTok{(UMGY\_effect\_MSEs)}
\NormalTok{\}}
\end{Highlighting}
\end{Shaded}

\begin{noteblock}
The source code of the function \href{https://github.com/AzKurban-edX-DS/Capstone-MovieLens/blob/main/r/src/support-functions/UMGY-effect.functions.R\#L102}{calc\_UMGY\_effect\_MSE.cv} is also available in the \href{https://github.com/AzKurban-edX-DS/Capstone-MovieLens/blob/main/r/src/support-functions/UMGY-effect.functions.R\#L102}{UMGYE Model Support Functions} script.

\end{noteblock}

\paragraph{\texorpdfstring{\href{https://github.com/AzKurban-edX-DS/Capstone-MovieLens/blob/main/r/src/support-functions/UMGY-effect.functions.R\#L114}{calc\_UMGY\_effect\_RMSE} Function}{calc\_UMGY\_effect\_RMSE Function}}\label{func.calc_UMGY_effect_RMSE}

\hfill\break

The function calculates the \emph{Mean Squared Error (MSE)} of the \emph{UMGYE Model} for the given \emph{Test Set} passed in the \texttt{test\_set} parameter.

\subparagraph{Parameters}\label{parameters-3}

\hfill\break

\begin{itemize}
\tightlist
\item
  \textbf{test\_set:} A \emph{Test Set} dataset used for \emph{UMGYE Model} validation;
\item
  \textbf{UMGY\_effect:} A data structure returned by any of the following functions described above:

  \begin{itemize}
  \tightlist
  \item
    \hyperref[func.train_UMGY_effect]{calc\_UMGY\_effect};
  \item
    \hyperref[func.train_UMGY_effect]{train\_UMGY\_effect};
  \item
    \hyperref[func.train_UMGY_effect.cv]{train\_UMGY\_effect.cv}.
  \end{itemize}
\end{itemize}

\subparagraph{Source Code}\label{source-code-3}

\hfill\break

The source code of the function is provided below:

\begin{Shaded}
\begin{Highlighting}[]
\NormalTok{calc\_UMGY\_effect\_RMSE }\OtherTok{\textless{}{-}} \ControlFlowTok{function}\NormalTok{(test\_set, UMGY\_effect)\{}
\NormalTok{  mse }\OtherTok{\textless{}{-}}\NormalTok{ test\_set }\SpecialCharTok{|\textgreater{}} \FunctionTok{calc\_UMGY\_effect\_MSE}\NormalTok{(UMGY\_effect)}
  \FunctionTok{sqrt}\NormalTok{(mse)}
\NormalTok{\}}
\end{Highlighting}
\end{Shaded}

\begin{noteblock}
The source code of the function \href{https://github.com/AzKurban-edX-DS/Capstone-MovieLens/blob/main/r/src/support-functions/UMGY-effect.functions.R\#L114}{calc\_UMGY\_effect\_RMSE} is also available in the \href{https://github.com/AzKurban-edX-DS/Capstone-MovieLens/blob/main/r/src/support-functions/UMGY-effect.functions.R\#L114}{UMGYE Model Support Functions} script.

\end{noteblock}

\paragraph{\texorpdfstring{\href{https://github.com/AzKurban-edX-DS/Capstone-MovieLens/blob/main/r/src/support-functions/UMGY-effect.functions.R\#L118}{calc\_UMGY\_effect\_RMSE.cv} Function}{calc\_UMGY\_effect\_RMSE.cv Function}}\label{func.calc_UMGY_effect_RMSE.cv}

\hfill\break

The function calculates the \emph{5-Fold Cross Validation MSE} result for the \emph{UMGY Effect} data passed in the \texttt{UMGY\_effect} parameter.

\subparagraph{Parameters}\label{parameters-4}

\hfill\break

The function accepts the only parameter: \texttt{UMGY\_effect}, which is a data structure returned by the one of the following functions described above:

\begin{itemize}
\tightlist
\item
  \hyperref[func.train_UMGY_effect]{calc\_UMGY\_effect};
\item
  \hyperref[func.train_UMGY_effect]{train\_UMGY\_effect};
\item
  \hyperref[func.train_UMGY_effect.cv]{train\_UMGY\_effect.cv}.
\end{itemize}

\subparagraph{Source Code}\label{source-code-4}

\hfill\break

The source code of the function is provided below:

\begin{Shaded}
\begin{Highlighting}[]
\NormalTok{calc\_UMGY\_effect\_RMSE.cv }\OtherTok{\textless{}{-}} \ControlFlowTok{function}\NormalTok{(UMGY\_effect)\{}
\NormalTok{  UMGY\_effect\_RMSE }\OtherTok{\textless{}{-}} \FunctionTok{sqrt}\NormalTok{(}\FunctionTok{calc\_UMGY\_effect\_MSE.cv}\NormalTok{(UMGY\_effect))}
  \FunctionTok{put\_log2}\NormalTok{(}\StringTok{"\%1{-}Fold Cross Validation ultimate RMSE: \%2"}\NormalTok{, }
\NormalTok{           CVFolds\_N, }
\NormalTok{           UMGY\_effect\_RMSE)}
  
\NormalTok{  UMGY\_effect\_RMSE}
\NormalTok{\}}
\end{Highlighting}
\end{Shaded}

\begin{noteblock}
The source code of the function \href{https://github.com/AzKurban-edX-DS/Capstone-MovieLens/blob/main/r/src/support-functions/UMGY-effect.functions.R\#L118}{calc\_UMGY\_effect\_RMSE.cv} is also available in the \href{https://github.com/AzKurban-edX-DS/Capstone-MovieLens/blob/main/r/src/support-functions/UMGY-effect.functions.R\#L118}{UMGYE Model Support Functions} script.

\end{noteblock}

\newpage

\subsubsection{UMGYE Model Building}\label{umgye-model-building}

\hfill\break

\begin{noteblock}
The complete source code of building and training the current model is available in the \href{https://github.com/AzKurban-edX-DS/Capstone-MovieLens/blob/main/r/src/capstone-movielens.main.R\#L1432}{UMGYE Model Building} section of the \href{https://github.com/AzKurban-edX-DS/Capstone-MovieLens/blob/main/r/src/capstone-movielens.main.R\#L1432}{capstone-movielens.main.R} script on \emph{GitHub}.

\end{noteblock}

Below, we provide the most significant line of the code for training our model using the \texttt{5-Fold\ Cross\ Validation} method:

\begin{Shaded}
\begin{Highlighting}[]
\NormalTok{  cv.UMGY\_effect }\OtherTok{\textless{}{-}} \FunctionTok{train\_UMGY\_effect.cv}\NormalTok{()}
\end{Highlighting}
\end{Shaded}

\begin{Shaded}
\begin{Highlighting}[]
\FunctionTok{str}\NormalTok{(cv.UMGY\_effect)}
\end{Highlighting}
\end{Shaded}

\begin{verbatim}
## tibble [15 x 2] (S3: tbl_df/tbl/data.frame)
##  $ year: num [1:15] 1995 1996 1997 1998 1999 ...
##  $ ye  : num [1:15] 0.6694 0.1152 0.0118 0.0262 0.0335 ...
\end{verbatim}

\begin{Shaded}
\begin{Highlighting}[]
\FunctionTok{summary}\NormalTok{(cv.UMGY\_effect)}
\end{Highlighting}
\end{Shaded}

\begin{verbatim}
##       year            ye          
##  Min.   :1995   Min.   :-0.06477  
##  1st Qu.:1998   1st Qu.:-0.02190  
##  Median :2002   Median : 0.01182  
##  Mean   :2002   Mean   : 0.04811  
##  3rd Qu.:2006   3rd Qu.: 0.02937  
##  Max.   :2009   Max.   : 0.66940
\end{verbatim}

\begin{noteblock}
We use the \hyperref[func.train_UMGY_effect.cv]{train\_UMGY\_effect.cv} function described above to compute the \emph{UMGY Effect}.

\end{noteblock}

We can now construct predictors and calculate the \emph{RMSE} of the current model using the \hyperref[func.calc_UMGY_effect_RMSE.cv]{calc\_UMGY\_effect\_RMSE.cv} function described above:

\begin{Shaded}
\begin{Highlighting}[]
\NormalTok{cv.UMGY\_effect.RMSE }\OtherTok{\textless{}{-}} \FunctionTok{calc\_UMGY\_effect\_RMSE.cv}\NormalTok{(cv.UMGY\_effect)}

\NormalTok{RMSEs.ResultTibble.UMGYE }\OtherTok{\textless{}{-}}\NormalTok{ RMSEs.ResultTibble.rglr.UMGE }\SpecialCharTok{|\textgreater{}} 
  \FunctionTok{RMSEs.AddRow}\NormalTok{(}\StringTok{"UMGYE Model"}\NormalTok{, }
\NormalTok{               cv.UMGY\_effect.RMSE,}
               \AttributeTok{comment =} \StringTok{"User+Movie+Genre+Year Effect (UMGYE) Model"}\NormalTok{)}
\end{Highlighting}
\end{Shaded}

\begin{Shaded}
\begin{Highlighting}[]
\FunctionTok{RMSE\_kable}\NormalTok{(RMSEs.ResultTibble.UMGYE)}
\end{Highlighting}
\end{Shaded}

\begin{tabular}{>{\raggedright\arraybackslash}p{15em}>{\centering\arraybackslash}p{5em}>{\raggedright\arraybackslash}p{30em}}
\toprule
\textbf{Method} & \textbf{RMSE} & \textbf{Comment}\\
\midrule
Project Objective & 0.8649000 & \\
Overall Mean Rating Model & 1.0603462 & \\
User Effect Model & 0.9697962 & \\
User+Movie Effect Model & 0.8732081 & \\
Regularized UME Model & 0.8729730 & Computed for `lambda` = 0.38745002746582\\
\addlinespace
UMGE Model & 0.8729730 & User+Movie+Genre Effect (UMGE) Model\\
Regularized UMGE Model & 0.8729728 & Computed for `lambda` = 0.0359375\\
UMGYE Model & 0.8723973 & User+Movie+Genre+Year Effect (UMGYE) Model\\
\bottomrule
\end{tabular}

\begin{noteblock}
Now, we have a bit better result than one for the previous model.

\end{noteblock}

\newpage

\subsubsection{UMGYE Model Regularization}\label{umgye-model-regularization}

\hfill\break

We have already explained the idea of \emph{Linear Model Regularization} in the \hyperref[ume-model-regularization]{UME Model Regularization} section above. We have also seen how the formula \eqref{eq:ME-penalty} for adding a penalty to the \emph{UME Model} is transformed into the formula \eqref{eq:GE-penalty} for the \emph{UMGE Model}. For the current model, this formula takes the form:

\begin{equation}
\sum_{i,j} \left(y_{u,i} - \mu - \alpha_i - \beta_j - g_{i,j} - \gamma(v_{i,j})\right)^2 + \lambda \sum_{i,j} \gamma(v_{i,j})^2
\label{eq:YE-penalty}
\end{equation}

And the formula \eqref{eq:ME-regularized} for calculating the values of the \emph{treatment effect} that minimizes the equation will take the form:

\begin{equation}
\hat{\gamma}(v_{i,j}, \lambda) = \frac{1}{\lambda + n_v} \sum_{r=1}^{n_v} \left(Y_{i,j} - \mu - \alpha_i - \beta_j - g_{i,j}\right)
\label{eq:YE-regularized}
\end{equation}

where \(n_v\) is the number of ratings made in year \(v\).

As with the previous models, we implement the \emph{Regularization} method for the current model in the following three steps:

\begin{enumerate}
\def\labelenumi{\arabic{enumi}.}
\item
  \textbf{Pre-configuration:} Preliminary determination of the optimal range of \(\lambda\) values for the \texttt{5-Fold\ Cross\ Validation} samples;
\item
  \textbf{Fine-tuning:} figuring out the value of \(\lambda\) that minimizes the model's RMSE.
\item
  \textbf{Retraining:} retraining the model with the best value of the parameter \(\lambda\) obtained in the previous step.
\end{enumerate}

\newpage

\paragraph{UMGYE Model Regularization: Support Functions}\label{umgye-model-regularization-support-functions}

\hfill\break

\begin{noteblock}
The \href{https://github.com/AzKurban-edX-DS/Capstone-MovieLens/blob/main/r/src/support-functions/UMGY-effect.functions.R\#L128}{regularize.test\_lambda.UMGY\_effect.cv} function described below are defined in the \href{https://github.com/AzKurban-edX-DS/Capstone-MovieLens/blob/main/r/src/support-functions/UMGY-effect.functions.R\#L127}{Regularization} section of the \href{https://github.com/AzKurban-edX-DS/Capstone-MovieLens/blob/main/r/src/support-functions/UMGY-effect.functions.R\#L128}{UM-effect.functions.R} script.

\end{noteblock}

\subparagraph{\texorpdfstring{\href{https://github.com/AzKurban-edX-DS/Capstone-MovieLens/blob/main/r/src/support-functions/UMGY-effect.functions.R\#128}{regularize.test\_lambda.UMGY\_effect.cv} Function}{regularize.test\_lambda.UMGY\_effect.cv Function}}\label{func.regularize.test_lambda.UMGY_effect.cv}

\hfill\break

This function calculates \emph{RMSE} of the \emph{UMGYE Model} using \emph{5-Fold Cross Validation} for the given \(\lambda\) parameter value:

\begin{Shaded}
\begin{Highlighting}[]
\NormalTok{regularize.test\_lambda.UMGY\_effect.cv }\OtherTok{\textless{}{-}} \ControlFlowTok{function}\NormalTok{(lambda)\{}
  \ControlFlowTok{if}\NormalTok{ (}\FunctionTok{is.na}\NormalTok{(lambda)) \{}
    \FunctionTok{stop}\NormalTok{(}\StringTok{"Function: regularize.test\_lambda.UMGY\_effect.cv}
\StringTok{\textasciigrave{}lambda\textasciigrave{} is \textasciigrave{}NA\textasciigrave{}"}\NormalTok{)}
\NormalTok{  \}}
  
  \FunctionTok{train\_UMGY\_effect.cv}\NormalTok{(lambda) }\SpecialCharTok{|\textgreater{}}
    \FunctionTok{calc\_UMGY\_effect\_RMSE.cv}\NormalTok{()}
\NormalTok{\}}
\end{Highlighting}
\end{Shaded}

\begin{noteblock}
Note that we reuse the function \hyperref[func.train_UMGY_effect.cv]{train\_UMGY\_effect.cv} calling it from the \hyperref[func.regularize.test_lambda.UMGY_effect.cv]{regularize.test\_lambda.UMGY\_effect.cv}, but now with the \(\lambda\) parameter different from the default (`lambda = 0') value.

\end{noteblock}

\newpage

\paragraph{UMGYE Model Regularization: Pre-configuration}\label{umgye-model-regularization-pre-configuration}

\hfill\break

Let's perform the preconfiguration to determine the appropriate range of \(\lambda\) for subsequent fine-tuning of our current model:

We are going to use the \hyperref[func.tune.model_param]{tune.model\_param} function described in the \hyperref[appndx_a.CHF.regularization]{Common Helper Functions: Regularization} section of the \hyperref[appndx_a]{Appendix} to this report, passing the \hyperref[func.regularize.test_lambda.UMGY_effect.cv]{regularize.test\_lambda.UMGY\_effect.cv} function as the value of the \hyperref[func.tune.model_param.params.fn_tune.test.param_value]{fn\_tune.test.param\_value} parameter.

Below we provide the most significant part of the code that performs this operation:

\begin{Shaded}
\begin{Highlighting}[]
\NormalTok{  lambdas }\OtherTok{\textless{}{-}} \FunctionTok{seq}\NormalTok{(}\DecValTok{0}\NormalTok{, }\DecValTok{512}\NormalTok{, }\DecValTok{32}\NormalTok{)}
\NormalTok{  cv.UMGYE.preset.result }\OtherTok{\textless{}{-}} 
    \FunctionTok{tune.model\_param}\NormalTok{(lambdas, }
\NormalTok{                     regularize.test\_lambda.UMGY\_effect.cv,}
                     \AttributeTok{steps.beyond\_min =} \DecValTok{16}\NormalTok{)}
\end{Highlighting}
\end{Shaded}

\begin{Shaded}
\begin{Highlighting}[]
\FunctionTok{str}\NormalTok{(cv.UMGYE.preset.result)}
\end{Highlighting}
\end{Shaded}

\begin{verbatim}
## List of 2
##  $ tuned.result:'data.frame':    17 obs. of  2 variables:
##   ..$ RMSE           : num [1:17] 0.872 0.872 0.872 0.872 0.872 ...
##   ..$ parameter.value: num [1:17] 0 32 64 96 128 160 192 224 256 288 ...
##  $ best_result : Named num [1:2] 224 0.872
##   ..- attr(*, "names")= chr [1:2] "param.best_value" "best_RMSE"
\end{verbatim}

\begin{Shaded}
\begin{Highlighting}[]
\NormalTok{cv.UMGYE.preset.result}\SpecialCharTok{$}\NormalTok{best\_result}
\end{Highlighting}
\end{Shaded}

\begin{verbatim}
## param.best_value        best_RMSE 
##      224.0000000        0.8721852
\end{verbatim}

\begin{noteblock}
The complete version of the source code provided in this section can be found in the \href{https://github.com/AzKurban-edX-DS/Capstone-MovieLens/blob/main/r/src/capstone-movielens.main.R\#L1493}{UMGYE Model Regularization: Pre-configuration} section of the \href{https://github.com/AzKurban-edX-DS/Capstone-MovieLens/blob/main/r/src/capstone-movielens.main.R\#L1493}{capstone-movielens.main.R} script on \emph{GitHub}.

\end{noteblock}

Now, let's visualize the results of the \(\lambda\) range preconfiguration:

\begin{Shaded}
\begin{Highlighting}[]
\NormalTok{cv.UMGYE.preset.result}\SpecialCharTok{$}\NormalTok{tuned.result }\SpecialCharTok{|\textgreater{}}
  \FunctionTok{data.plot}\NormalTok{(}\AttributeTok{title =} \FunctionTok{TeX}\NormalTok{(r}\StringTok{\textquotesingle{}[UMGYE Model Regularization: $\textbackslash{}lambda$ Range Pre{-}configuration]\textquotesingle{}}\NormalTok{),}
              \AttributeTok{xname =} \StringTok{"parameter.value"}\NormalTok{, }
              \AttributeTok{yname =} \StringTok{"RMSE"}\NormalTok{, }
              \AttributeTok{xlabel =} \FunctionTok{TeX}\NormalTok{(r}\StringTok{\textquotesingle{}[$\textbackslash{}lambda$]\textquotesingle{}}\NormalTok{), }
              \AttributeTok{ylabel =} \StringTok{"RMSE"}\NormalTok{)}
\end{Highlighting}
\end{Shaded}

\pandocbounded{\includegraphics[keepaspectratio]{capstone-movielens.report_files/figure-latex/unnamed-chunk-237-1.pdf}}

\begin{noteblock}
We use the custom data visualization function \hyperref[func.data.plot]{data.plot} described in the \hyperref[appndx_a.data_helper.functions]{Data Helper Functions} section of the \hyperref[appndx_a]{Appendix} to this report, which is defined in the \href{https://github.com/AzKurban-edX-DS/Capstone-MovieLens/blob/main/r/src/support-functions/data.helper.functions.R\#L447}{Data Visualization} section of the \href{https://github.com/AzKurban-edX-DS/Capstone-MovieLens/blob/main/r/src/support-functions/data.helper.functions.R\#L592}{data.helper.functions.R} script on \emph{GitHub}.

\end{noteblock}

\newpage

\paragraph{UMGYE Model Regularization: Fine-tuning}\label{umgye-model-regularization-fine-tuning}

\hfill\break

\begin{noteblock}
The complete version of the source code provided in this section can be found in the \href{https://github.com/AzKurban-edX-DS/Capstone-MovieLens/blob/main/r/src/capstone-movielens.main.R\#L867}{Fine-tuning Step of the Regularization Method for the User+Movie Model} section of the \href{https://github.com/AzKurban-edX-DS/Capstone-MovieLens/blob/main/r/src/capstone-movielens.main.R}{capstone-movielens.main.R} script on \emph{GitHub}.

\end{noteblock}

We are now ready to perform the fine-tuning step of our model \emph{regularization} process to determine the best value for the \(\lambda\) parameter.

Here we are going to use the \hyperref[func.model.tune.param_range]{model.tune.param\_range} function described in the \hyperref[appndx_a.CHF.regularization]{Common Helper Functions: Regularization} section of the \hyperref[appndx_a]{Appendix} to this report, passing the \hyperref[func.regularize.test_lambda.UMGY_effect.cv]{regularize.test\_lambda.UMGY\_effect.cv} function as the value of the \hyperref[func.model.tune.param_range.params.fn_tune.test.param_value]{fn\_tune.test.param\_value} parameter.

Below we provide the most significant part of the code that performs this operation:

\begin{Shaded}
\begin{Highlighting}[]
\NormalTok{endpoints }\OtherTok{\textless{}{-}} 
  \FunctionTok{get\_fine\_tune.param.endpoints}\NormalTok{(cv.UMGYE.preset.result}\SpecialCharTok{$}\NormalTok{tuned.result)}

\NormalTok{UMGYE.loop\_starter }\OtherTok{\textless{}{-}} \FunctionTok{c}\NormalTok{(endpoints[}\StringTok{"start"}\NormalTok{], }
\NormalTok{                        endpoints[}\StringTok{"end"}\NormalTok{], }
                        \DecValTok{8}\NormalTok{)}
\NormalTok{UMGYE.loop\_starter}
\CommentTok{\#\textgreater{} [1] }

\NormalTok{cache.base\_name }\OtherTok{\textless{}{-}} \StringTok{"UMGYE.rglr.fine{-}tuning"}

\NormalTok{UMGYE.rglr.fine\_tune.results }\OtherTok{\textless{}{-}} 
  \FunctionTok{model.tune.param\_range}\NormalTok{(UMGYE.loop\_starter,}
\NormalTok{                         UMGYE.rglr.fine\_tune.cache.path,}
\NormalTok{                         cache.base\_name,}
\NormalTok{                         regularize.test\_lambda.UMGY\_effect.cv)}
\end{Highlighting}
\end{Shaded}

\begin{Shaded}
\begin{Highlighting}[]
\FunctionTok{str}\NormalTok{(UMGYE.rglr.fine\_tune.results)}
\end{Highlighting}
\end{Shaded}

\begin{verbatim}
## List of 3
##  $ best_result           : Named num [1:2] 233.777 0.872
##   ..- attr(*, "names")= chr [1:2] "param.best_value" "best_RMSE"
##  $ param_values.endpoints: Named num [1:3] 2.34e+02 2.34e+02 7.63e-05
##   ..- attr(*, "names")= chr [1:3] "" "" ""
##  $ tuned.result          :'data.frame':  9 obs. of  2 variables:
##   ..$ parameter.value: num [1:9] 234 234 234 234 234 ...
##   ..$ RMSE           : num [1:9] 0.872 0.872 0.872 0.872 0.872 ...
\end{verbatim}

\begin{Shaded}
\begin{Highlighting}[]
\NormalTok{UMGYE.rglr.fine\_tune.results}\SpecialCharTok{$}\NormalTok{best\_result}
\end{Highlighting}
\end{Shaded}

\begin{verbatim}
## param.best_value        best_RMSE 
##      233.7766876        0.8721851
\end{verbatim}

\begin{noteblock}
Under the hood, this code block internally calls the helper function \hyperref[func.get_fine_tune.param.endpoints]{get\_fine\_tune.param.endpoints} described in the \hyperref[appndx_a.CHF.regularization]{Common Helper Functions: Regularization} section of the \hyperref[appndx_a]{Appendix} to this report to figure out the boundaries of the range of values from the nearest neighborhood of the value corresponding to the best \texttt{RMSE} in the data passed in as a parameter.

\end{noteblock}

\newpage

Let's visualize the fine-tuning results:

\begin{Shaded}
\begin{Highlighting}[]
\NormalTok{UMGYE.rglr.fine\_tune.results}\SpecialCharTok{$}\NormalTok{tuned.result }\SpecialCharTok{|\textgreater{}}
  \FunctionTok{data.plot}\NormalTok{(}\AttributeTok{title =} \StringTok{"UMGYE Model Regularization: Fine{-}tuned result"}\NormalTok{,}
              \AttributeTok{xname =} \StringTok{"parameter.value"}\NormalTok{,}
              \AttributeTok{yname =} \StringTok{"RMSE"}\NormalTok{,}
              \AttributeTok{xlabel =} \FunctionTok{TeX}\NormalTok{(r}\StringTok{\textquotesingle{}[$\textbackslash{}lambda$]\textquotesingle{}}\NormalTok{),}
              \AttributeTok{ylabel =} \FunctionTok{str\_glue}\NormalTok{(}\StringTok{"Deviation from the best RMSE value ("}\NormalTok{,}
                                \FunctionTok{as.character}\NormalTok{(}\FunctionTok{round}\NormalTok{(UMGYE.rglr.fine\_tune.RMSE.best, }\AttributeTok{digits =} \DecValTok{7}\NormalTok{)),}
                                \StringTok{")"}\NormalTok{),}
              \AttributeTok{normalize =} \ConstantTok{TRUE}\NormalTok{)}
\end{Highlighting}
\end{Shaded}

\pandocbounded{\includegraphics[keepaspectratio]{capstone-movielens.report_files/figure-latex/unnamed-chunk-240-1.pdf}}

\begin{noteblock}
We use the custom data visualization function \hyperref[func.data.plot]{data.plot} described in the \hyperref[appndx_a.data_helper.functions]{Data Helper Functions} section of the \hyperref[appndx_a]{Appendix} to this report, which is defined in the \href{https://github.com/AzKurban-edX-DS/Capstone-MovieLens/blob/main/r/src/support-functions/data.helper.functions.R\#L447}{Data Visualization} section of the \href{https://github.com/AzKurban-edX-DS/Capstone-MovieLens/blob/main/r/src/support-functions/data.helper.functions.R\#L592}{data.helper.functions.R} script on \emph{GitHub}.

\end{noteblock}

\newpage

\paragraph{\texorpdfstring{UMGYE Model Regularization: Retraining Model with the best \(\lambda\)}{UMGYE Model Regularization: Retraining Model with the best \textbackslash lambda}}\label{umgye-model-regularization-retraining-model-with-the-best-lambda}

\hfill\break

\begin{noteblock}
The complete version of the source code provided in this section are available in the \href{https://github.com/AzKurban-edX-DS/Capstone-MovieLens/blob/main/r/src/capstone-movielens.main.R\#L1589}{Re-train Regularized UMGYE Model for the best \texttt{lambda}} section of the \href{https://github.com/AzKurban-edX-DS/Capstone-MovieLens/blob/main/r/src/capstone-movielens.main.R\#L1589}{capstone-movielens.main.R} script on \emph{GitHub}.

\end{noteblock}

Now, we can calculate the \emph{Regularized UMGYE Effect} by retraining our model on the entire \texttt{edx} dataset with the best value of the \(\lambda\) parameter we just calculated, for the definitive \emph{Root Mean Squared Error} calculation and use in subsequent models.

\begin{Shaded}
\begin{Highlighting}[]
\NormalTok{  rglr.UMGY\_effect }\OtherTok{\textless{}{-}}\NormalTok{ edx }\SpecialCharTok{|\textgreater{}} \FunctionTok{train\_UMGY\_effect}\NormalTok{(UMGYE.rglr.best\_lambda)}
\end{Highlighting}
\end{Shaded}

\begin{noteblock}
Here we use the helper function \hyperref[func.train_UMGY_effect]{train\_UMGY\_effect} described above in the \hyperref[umgye-model-regularization-support-functions]{UMGYE Model Regularization: Support Functions} section.

\end{noteblock}

We calculate the \emph{Root Mean Squared Error} for the ultimately computed \emph{UMGYE Effect} using the helper function \hyperref[func.calc_UMGY_effect_RMSE.cv]{calc\_UMGY\_effect\_RMSE.cv} described above in the \hyperref[umgye-model-regularization-support-functions]{UMGYE Model Regularization: Support Functions} section as follows:

\begin{Shaded}
\begin{Highlighting}[]
\NormalTok{  rglr.UMGY\_effect.RMSE }\OtherTok{\textless{}{-}} \FunctionTok{calc\_UMGY\_effect\_RMSE.cv}\NormalTok{(rglr.UMGY\_effect)}
\end{Highlighting}
\end{Shaded}

Finally, we add the definitive result for the current model to our \emph{Result Table}:

\begin{Shaded}
\begin{Highlighting}[]
\NormalTok{RMSEs.ResultTibble.rglr.UMGYE }\OtherTok{\textless{}{-}}\NormalTok{ RMSEs.ResultTibble.UMGYE }\SpecialCharTok{|\textgreater{}} 
  \FunctionTok{RMSEs.AddRow}\NormalTok{(}\StringTok{"Regularized UMGYE Model"}\NormalTok{, }
\NormalTok{               rglr.UMGY\_effect.RMSE,}
               \AttributeTok{comment =} \StringTok{"Computed for \textasciigrave{}lambda\textasciigrave{} = \%1"} \SpecialCharTok{|\textgreater{}}
                 \FunctionTok{msg.glue}\NormalTok{(UMGYE.rglr.best\_lambda))}
\end{Highlighting}
\end{Shaded}

\begin{Shaded}
\begin{Highlighting}[]
\FunctionTok{RMSE\_kable}\NormalTok{(RMSEs.ResultTibble.rglr.UMGYE)}
\end{Highlighting}
\end{Shaded}

\begin{tabular}{>{\raggedright\arraybackslash}p{15em}>{\centering\arraybackslash}p{5em}>{\raggedright\arraybackslash}p{30em}}
\toprule
\textbf{Method} & \textbf{RMSE} & \textbf{Comment}\\
\midrule
Project Objective & 0.8649000 & \\
Overall Mean Rating Model & 1.0603462 & \\
User Effect Model & 0.9697962 & \\
User+Movie Effect Model & 0.8732081 & \\
Regularized UME Model & 0.8729730 & Computed for `lambda` = 0.38745002746582\\
\addlinespace
UMGE Model & 0.8729730 & User+Movie+Genre Effect (UMGE) Model\\
Regularized UMGE Model & 0.8729728 & Computed for `lambda` = 0.0359375\\
UMGYE Model & 0.8723973 & User+Movie+Genre+Year Effect (UMGYE) Model\\
Regularized UMGYE Model & 0.8721857 & Computed for `lambda` = 233.77668762207\\
\bottomrule
\end{tabular}

\newpage

\subsection{User+Movie+Genre+Year+Day Effect (UMGYDE) Model}\label{usermoviegenreyearday-effect-umgyde-model}

\hfill\break
\#\#\# Mathematical Description of the UMGYDE Model\\

If we define \(d_{i,j}\) as the \emph{number of days since the earliest record} in the \texttt{edx} dataset for movie \(j\) rated by user \(i\), then the formula \eqref{eq:UMGYDE-model} describing the \emph{UMGYDE Model}, for the current model, takes the form:

\begin{equation}
Y_{i,j} = \mu + \alpha_i + \beta_j + g_{i,j} + \gamma(v_{i,j}) + s(d_{i,j}) + \varepsilon_{i,j}
\label{eq:UMGYDE-model}
\end{equation}

with \(s\) a \emph{smoothed day} function of \(d_{i,j}\)

Therefore, the formula \eqref{eq:year-effect} for calculation the prediction of a \emph{year effect} as a residual, for a \emph{smoothed day effect}
\[
\hat{s}(d_{i,j}) = s(d_{i,j}) + \varepsilon_{i,j}
\]

takes the following form:

\begin{equation}
\hat{s}(d_{i,j}) = Y_{i,j} - (\mu + \alpha_i + \beta_j + g_{i,j} + \gamma(v_{i,j}))
\label{eq:smoothed-day-effect}
\end{equation}

\newpage

\subsubsection{UMGYDE Model: Helper Functions}\label{umgyde-model-helper-functions}

\hfill\break

\phantomsection\label{sec.umgydem.support_functions.note.lambda}
\begin{noteblock}
Some of the functions described in this section accepts the \texttt{lambda} parameter, which we will need later for the \emph{Model Regularization} method. We use the \href{https://github.com/AzKurban-edX-DS/Capstone-MovieLens/blob/main/r/src/support-functions/common-helper.functions.R\#L63}{mean\_reg} function call as well, which we will also need for the \emph{Regularization} techniques to apply to our models (for details, see the \hyperref[func.mean_reg]{mean\_reg} function description in the \hyperref[appndx_a.rglr.common_helper.functions]{Regularization: Common Helper Functions} section of the \hyperref[appndx_a]{Appendix} to this report).
We have already explained that in the \hyperref[ume-model-regularization]{UME Model Regularization} section.
For now, we omit the \texttt{lambda} parameter, implicitly using its default value \texttt{lambda\ =\ 0}. Let's recall that in this case, the \hyperref[func.mean_reg]{mean\_reg} function is equivalent to the standard R function \href{https://www.rdocumentation.org/packages/base/versions/3.6.2/topics/mean}{base::mean}.

\end{noteblock}

\paragraph{\texorpdfstring{\href{https://github.com/AzKurban-edX-DS/Capstone-MovieLens/blob/main/r/src/support-functions/UMGYD-effect.functions.R\#L2}{calc\_day\_general\_effect} Function}{calc\_day\_general\_effect Function}}\label{func.calc_day_general_effect}

\hfill\break

This is a helper function that calculates the \emph{Day General Effect} of user ratings on a \emph{given day since the earliest record} in the \texttt{edx} dataset and is intended to be used in other functions and scripts (such as \hyperref[func.train_UMGY_SmoothedDay_effect]{train\_UMGY\_SmoothedDay\_effect}) described below.

\subparagraph{Parameters}\label{parameters-5}

\hfill\break

\textbf{\emph{train\_set}}

\hfill\break
The \emph{Train Set} dataset.

\textbf{\emph{lambda}}

\hfill\break
As explained \hyperref[sec.umgydem.support_functions.note.lambda]{above} in this section, this parameter is used for the \emph{Regularization} techniques. Otherwise, it is omitted, meaning its default (zero) value is used.

\subparagraph{Source Code}\label{source-code-5}

\hfill\break

The following is a simplified version of the function's source code:

\begin{Shaded}
\begin{Highlighting}[]
\NormalTok{calc\_day\_general\_effect }\OtherTok{\textless{}{-}} \ControlFlowTok{function}\NormalTok{(train\_set, }\AttributeTok{lambda =} \DecValTok{0}\NormalTok{)\{}
\NormalTok{  train\_set }\SpecialCharTok{|\textgreater{}} 
    \FunctionTok{left\_join}\NormalTok{(edx.user\_effect, }\AttributeTok{by =} \StringTok{"userId"}\NormalTok{) }\SpecialCharTok{|\textgreater{}}
    \FunctionTok{left\_join}\NormalTok{(rglr.UM\_effect, }\AttributeTok{by =} \StringTok{"movieId"}\NormalTok{) }\SpecialCharTok{|\textgreater{}}
    \FunctionTok{left\_join}\NormalTok{(rglr.UMG\_effect, }\AttributeTok{by =} \StringTok{"movieId"}\NormalTok{) }\SpecialCharTok{|\textgreater{}}
    \FunctionTok{left\_join}\NormalTok{(date\_days\_map, }\AttributeTok{by =} \StringTok{"timestamp"}\NormalTok{) }\SpecialCharTok{|\textgreater{}}
    \FunctionTok{left\_join}\NormalTok{(rglr.UMGY\_effect, }\AttributeTok{by=}\StringTok{\textquotesingle{}year\textquotesingle{}}\NormalTok{) }\SpecialCharTok{|\textgreater{}}
    \FunctionTok{mutate}\NormalTok{(}\AttributeTok{resid =}\NormalTok{ rating }\SpecialCharTok{{-}}\NormalTok{ (mu }\SpecialCharTok{+}\NormalTok{ a }\SpecialCharTok{+}\NormalTok{ b }\SpecialCharTok{+}\NormalTok{ g }\SpecialCharTok{+}\NormalTok{ ye)) }\SpecialCharTok{|\textgreater{}}
    \FunctionTok{group\_by}\NormalTok{(days) }\SpecialCharTok{|\textgreater{}}
    \FunctionTok{summarise}\NormalTok{(}\AttributeTok{de =} \FunctionTok{mean\_reg}\NormalTok{(resid, lambda), }
              \AttributeTok{year =} \FunctionTok{mean}\NormalTok{(year))}
\NormalTok{\}}
\end{Highlighting}
\end{Shaded}

\begin{noteblock}
The complete version of the source code of the \href{https://github.com/AzKurban-edX-DS/Capstone-MovieLens/blob/main/r/src/support-functions/UMGYD-effect.functions.R\#L2}{calc\_day\_general\_effect} function is available in the \href{https://github.com/AzKurban-edX-DS/Capstone-MovieLens/blob/main/r/src/support-functions/UMGYD-effect.functions.R\#L1}{UMGYDE Model Support Functions} section of the \href{https://github.com/AzKurban-edX-DS/Capstone-MovieLens/blob/main/r/src/support-functions/UMGYD-effect.functions.R\#L2}{UMGYD-effect.functions.R} script on \emph{GitHub}.

\end{noteblock}

\subparagraph{Return}\label{return-5}

\hfill\break
A data frame object representing the \emph{Day General Effect}.

\newpage

\paragraph{\texorpdfstring{\href{https://github.com/AzKurban-edX-DS/Capstone-MovieLens/blob/main/r/src/support-functions/UMGYD-effect.functions.R\#L31}{calc\_day\_general\_effect.cv} Function}{calc\_day\_general\_effect.cv Function}}\label{func.calc_day_general_effect.cv}

\hfill\break

This is a helper function that calculates the \emph{Day General Effect} of user ratings on a \emph{given day since the earliest record} in the \texttt{edx} dataset using the \emph{5-Fold Cross Validation} and is intended to be used in other functions and scripts (such as \hyperref[func.train_UMGY_SmoothedDay_effect.cv]{train\_UMGY\_SmoothedDay\_effect.cv}) described below.

\subparagraph{Parameters}\label{parameters-6}

\hfill\break

\textbf{\emph{lambda}}

\hfill\break
As explained \hyperref[sec.umgydem.support_functions.note.lambda]{above} in this section, this parameter is used for the \emph{Regularization} techniques. Otherwise, it is omitted, meaning its default (zero) value is used.

\subparagraph{Souurce Code}\label{souurce-code-1}

\hfill\break

\begin{noteblock}
The complete version of the source code of the \href{https://github.com/AzKurban-edX-DS/Capstone-MovieLens/blob/main/r/src/support-functions/UMGYD-effect.functions.R\#L31}{calc\_day\_general\_effect.cv} function is available in the \href{https://github.com/AzKurban-edX-DS/Capstone-MovieLens/blob/main/r/src/support-functions/UMGYD-effect.functions.R\#L1}{UMGYDE Model Support Functions} section of the \href{https://github.com/AzKurban-edX-DS/Capstone-MovieLens/blob/main/r/src/support-functions/UMGYD-effect.functions.R\#L30}{UMGYD-effect.functions.R} script on \emph{GitHub}.

\end{noteblock}

The following is a simplified version of the function's source code:

\begin{Shaded}
\begin{Highlighting}[]
\NormalTok{calc\_day\_general\_effect.cv }\OtherTok{\textless{}{-}} \ControlFlowTok{function}\NormalTok{(}\AttributeTok{lambda =} \DecValTok{0}\NormalTok{)\{}
\NormalTok{  gday\_effect\_ls }\OtherTok{\textless{}{-}} \FunctionTok{lapply}\NormalTok{(edx\_CV,  }\ControlFlowTok{function}\NormalTok{(cv\_fold\_dat)\{}
\NormalTok{    cv\_fold\_dat}\SpecialCharTok{$}\NormalTok{train\_set }\SpecialCharTok{|\textgreater{}} \FunctionTok{calc\_day\_general\_effect}\NormalTok{(lambda)}
\NormalTok{  \})}

\NormalTok{  gday\_effect\_united }\OtherTok{\textless{}{-}} \FunctionTok{union\_cv\_results}\NormalTok{(gday\_effect\_ls)}

\NormalTok{  gday\_effect\_united }\SpecialCharTok{|\textgreater{}}
    \FunctionTok{group\_by}\NormalTok{(days) }\SpecialCharTok{|\textgreater{}}
    \FunctionTok{summarise}\NormalTok{(}\AttributeTok{de =} \FunctionTok{mean}\NormalTok{(de), }\AttributeTok{year =} \FunctionTok{mean}\NormalTok{(year))}
\NormalTok{\}}
\end{Highlighting}
\end{Shaded}

\begin{noteblock}
Here we use the function call \href{https://github.com/AzKurban-edX-DS/Capstone-MovieLens/blob/main/r/src/support-functions/data.helper.functions.R\#L432}{union\_cv\_results} to aggregate the \emph{5-Fold Cross Validation} results (for details, see the \hyperref[func.union_cv_results]{union\_cv\_results} function description in the \hyperref[appndx_a.data_helper.functions]{Data Helper Functions} section of the \hyperref[appndx_a]{Appendix} to this report).

\end{noteblock}

\subparagraph{Return}\label{return-6}

\hfill\break
A data frame object representing the \emph{Day General Effect}.

\newpage

\paragraph{\texorpdfstring{\href{https://github.com/AzKurban-edX-DS/Capstone-MovieLens/blob/main/r/src/support-functions/UMGYD-effect.functions.R\#L69}{loess\_de} Function}{loess\_de Function}}\label{func.loess_de}

\hfill\break

This is a helper function that calculates the \emph{Smoothed Day Effect} of user ratings and is intended to be used in other functions and scripts (such as \hyperref[func.calc_UMGY_SmoothedDay_effect]{calc\_UMGY\_SmoothedDay\_effect}) described below.

\subparagraph{Parameters}\label{parameters-7}

\hfill\break

\textbf{\emph{de\_bias.dat}}

\hfill\break
A data frame object representing the \emph{Day General Effect}. It can be returned by the following functions described above:

\begin{itemize}
\tightlist
\item
  \hyperref[func.calc_day_general_effect]{calc\_day\_general\_effect};
\item
  \hyperref[func.calc_day_general_effect.cv]{calc\_day\_general\_effect.cv}.
\end{itemize}

\textbf{\emph{degree}}

\hfill\break
A parameter to pass to the internally called \href{https://www.rdocumentation.org/packages/stats/versions/3.6.2/topics/loess}{stats::loess} function (see the \hyperref[func.loess_de.details]{Details} section below), meaning the degree of the polynomials to be used.

\textbf{\emph{span}}

\hfill\break
A parameter to pass to the internally called \href{https://www.rdocumentation.org/packages/stats/versions/3.6.2/topics/loess}{stats::loess} function (see the \hyperref[func.loess_de.details]{Details} section below) to control the degree of smoothing.

\phantomsection\label{func.loess_de.details}
\subparagraph{Details}\label{details}

\hfill\break
The function is a wrapper for the \href{https://www.rdocumentation.org/packages/stats/versions/3.6.2/topics/loess}{stats::loess} and calls it under the hood.

\subparagraph{Source Code}\label{source-code-6}

\hfill\break

The following is the source code of the function:

\begin{Shaded}
\begin{Highlighting}[]
\NormalTok{loess\_de }\OtherTok{\textless{}{-}} \ControlFlowTok{function}\NormalTok{(de\_bias.dat, }\AttributeTok{degree =} \ConstantTok{NA}\NormalTok{, }\AttributeTok{span =} \ConstantTok{NA}\NormalTok{)\{}
  \ControlFlowTok{if}\NormalTok{(}\FunctionTok{is.na}\NormalTok{(degree)) degree }\OtherTok{=} \DecValTok{2}
  \ControlFlowTok{if}\NormalTok{(}\FunctionTok{is.na}\NormalTok{(span)) span }\OtherTok{=} \FloatTok{0.75}
  \FunctionTok{loess}\NormalTok{(de }\SpecialCharTok{\textasciitilde{}}\NormalTok{ days, }\AttributeTok{span =}\NormalTok{ span, }\AttributeTok{degree =}\NormalTok{ degree, }\AttributeTok{data =}\NormalTok{ de\_bias.dat)}
\NormalTok{\}}
\end{Highlighting}
\end{Shaded}

\begin{noteblock}
The source code of the \href{https://github.com/AzKurban-edX-DS/Capstone-MovieLens/blob/main/r/src/support-functions/UMGYD-effect.functions.R\#L69}{loess\_de} function is also available in the \href{https://github.com/AzKurban-edX-DS/Capstone-MovieLens/blob/main/r/src/support-functions/UMGYD-effect.functions.R\#L1}{UMGYDE Model Support Functions} section of the \href{https://github.com/AzKurban-edX-DS/Capstone-MovieLens/blob/main/r/src/support-functions/UMGYD-effect.functions.R\#L69}{UMGYD-effect.functions.R} script on \emph{GitHub}.

\end{noteblock}

\subparagraph{Return}\label{return-7}

\hfill\break
An object of class \texttt{loess} returned by the internally called function \href{https://www.rdocumentation.org/packages/stats/versions/3.6.2/topics/loess}{stats::loess} (see the \hyperref[func.loess_de.details]{Details} section above).

\newpage

\paragraph{\texorpdfstring{\href{https://github.com/AzKurban-edX-DS/Capstone-MovieLens/blob/main/r/src/support-functions/UMGYD-effect.functions.R\#L74}{calc\_UMGY\_SmoothedDay\_effect} Function}{calc\_UMGY\_SmoothedDay\_effect Function}}\label{func.calc_UMGY_SmoothedDay_effect}

\hfill\break

This is a helper function that calculates the \emph{Smoothed Day Effect} of the user ratings and is intended to be used in other functions and scripts (such as \hyperref[func.train_UMGY_SmoothedDay_effect]{train\_UMGY\_SmoothedDay\_effect}) described below.

\subparagraph{Parameters}\label{parameters-8}

\hfill\break

\phantomsection\label{func.calc_UMGY_SmoothedDay_effect.params.day_gen_effect}
\textbf{\emph{day\_gen\_effect}}

\hfill\break
A data frame object representing the \emph{Day General Effect}. It can be returned by the following functions described above:

\begin{itemize}
\tightlist
\item
  \hyperref[func.calc_day_general_effect]{calc\_day\_general\_effect};
\item
  \hyperref[func.calc_day_general_effect.cv]{calc\_day\_general\_effect.cv}.
\end{itemize}

\textbf{\emph{degree}}

\hfill\break
A parameter to pass to the internally called \hyperref[func.loess_de]{loess\_de} function (see the \hyperref[func.calc_UMGY_SmoothedDay_effect.details]{Details} section below), meaning the degree of the polynomials to be used.

\textbf{\emph{span}}

\hfill\break
A parameter to pass to the internally called \hyperref[func.loess_de]{loess\_de} function (see the \hyperref[func.calc_UMGY_SmoothedDay_effect.details]{Details} section below) to control the degree of smoothing.

\phantomsection\label{func.calc_UMGY_SmoothedDay_effect.details}
\subparagraph{Details}\label{details-1}

\hfill\break
Under the hood, the function internally calls the \hyperref[func.loess_de]{loess\_de} described above and constructs the \emph{User+Movie+Genre+Year+(Smoothed)Day (UMGYD) Effect} data object based on the object of class \texttt{loess} it returns along with the data passed by the \hyperref[func.calc_UMGY_SmoothedDay_effect.params.day_gen_effect]{day\_gen\_effect} parameter.

\subparagraph{Source Code}\label{source-code-7}

\hfill\break

The following is the source code of the function:

\begin{Shaded}
\begin{Highlighting}[]
\NormalTok{calc\_UMGY\_SmoothedDay\_effect }\OtherTok{\textless{}{-}} \ControlFlowTok{function}\NormalTok{(day\_gen\_effect, }
                                         \AttributeTok{degree =} \ConstantTok{NA}\NormalTok{, }
                                         \AttributeTok{span =} \ConstantTok{NA}\NormalTok{)\{}
  \FunctionTok{put\_log2}\NormalTok{(}\StringTok{"Function \textasciigrave{}calc\_UMGY\_SmoothedDay\_effect\textasciigrave{}: }
\StringTok{Training model using \textasciigrave{}loess\textasciigrave{} function with the following parameters:}
\StringTok{degree: \%1;}
\StringTok{span: \%2}
\StringTok{..."}\NormalTok{,}
\NormalTok{           degree,}
\NormalTok{           span)}
  \CommentTok{\# browser()}
\NormalTok{  fit }\OtherTok{\textless{}{-}}\NormalTok{ day\_gen\_effect }\SpecialCharTok{|\textgreater{}} \FunctionTok{loess\_de}\NormalTok{(degree, span)}
\NormalTok{  smth\_day\_effect }\OtherTok{\textless{}{-}}\NormalTok{ day\_gen\_effect }\SpecialCharTok{|\textgreater{}} \FunctionTok{mutate}\NormalTok{(}\AttributeTok{de\_smoothed =}\NormalTok{ fit}\SpecialCharTok{$}\NormalTok{fitted)}
  
  \FunctionTok{put\_log2}\NormalTok{(}\StringTok{"Function \textasciigrave{}calc\_UMGY\_SmoothedDay\_effect\textasciigrave{}: }
\StringTok{Model has been trained using \textasciigrave{}loess\textasciigrave{} function with the following parameters:}
\StringTok{degree: \%1;}
\StringTok{span: \%2."}\NormalTok{,}
\NormalTok{           degree,}
\NormalTok{           span)}
  
\NormalTok{  smth\_day\_effect}
\NormalTok{\}}
\end{Highlighting}
\end{Shaded}

\begin{noteblock}
The source code of the \href{https://github.com/AzKurban-edX-DS/Capstone-MovieLens/blob/main/r/src/support-functions/UMGYD-effect.functions.R\#L74}{calc\_UMGY\_SmoothedDay\_effect} function is also available in the \href{https://github.com/AzKurban-edX-DS/Capstone-MovieLens/blob/main/r/src/support-functions/UMGYD-effect.functions.R\#L1}{UMGYDE Model Support Functions} section of the \href{https://github.com/AzKurban-edX-DS/Capstone-MovieLens/blob/main/r/src/support-functions/UMGYD-effect.functions.R\#L74}{UMGYD-effect.functions.R} script on \emph{GitHub}.

\end{noteblock}

\subparagraph{Return}\label{return-8}

\hfill\break
A data frame object representing the \emph{User+Movie+Genre+Year+(Smoothed)Day (UMGYD) Effect}.

\newpage

\paragraph{\texorpdfstring{\href{https://github.com/AzKurban-edX-DS/Capstone-MovieLens/blob/main/r/src/support-functions/UMGYD-effect.functions.R\#L97}{train\_UMGY\_SmoothedDay\_effect} Function}{train\_UMGY\_SmoothedDay\_effect Function}}\label{func.train_UMGY_SmoothedDay_effect}

\hfill\break

This is a helper function that calculates the \emph{Smoothed Day Effect} of the user ratings and is intended to be used in other functions and scripts (such as \hyperref[func.train_UMGY_SmoothedDay_effect]{train\_UMGY\_SmoothedDay\_effect}) described below.

\subparagraph{Parameters}\label{parameters-9}

\hfill\break

\textbf{\emph{train\_set}}

\hfill\break
The \emph{Train Set} dataset.

\phantomsection\label{func.train_UMGY_SmoothedDay_effect.params.degree}
\textbf{\emph{degree}}

\hfill\break
A parameter to pass to the internally called \hyperref[func.calc_UMGY_SmoothedDay_effect]{calc\_UMGY\_SmoothedDay\_effect} function (see the \hyperref[func.train_UMGY_SmoothedDay_effect.details]{Details} section below), meaning the degree of the polynomials to be used.

\phantomsection\label{func.train_UMGY_SmoothedDay_effect.params.span}
\textbf{\emph{span}}

\hfill\break
A parameter to pass to the internally called \hyperref[func.calc_UMGY_SmoothedDay_effect]{calc\_UMGY\_SmoothedDay\_effect} function (see the \hyperref[func.train_UMGY_SmoothedDay_effect.details]{Details} section below) to control the degree of smoothing.

\textbf{\emph{lambda}}

\hfill\break
As explained \hyperref[sec.umgydem.support_functions.note.lambda]{above} in this section, this parameter is used for the \emph{Regularization} techniques. Otherwise, it is omitted, meaning its default (zero) value is used.

\phantomsection\label{func.train_UMGY_SmoothedDay_effect.details}
\subparagraph{Details}\label{details-2}

\hfill\break
Under the hood, the function internally calls the \hyperref[func.calc_day_general_effect]{calc\_day\_general\_effect} and then \hyperref[func.calc_UMGY_SmoothedDay_effect]{calc\_UMGY\_SmoothedDay\_effect} described above, and returns the \emph{User+Movie+Genre+Year+(Smoothed)Day (UMGYD) Effect} data object that the latter inner function returns.

\subparagraph{Source Code}\label{source-code-8}

\hfill\break

The following is the source code of the function:

\begin{Shaded}
\begin{Highlighting}[]
\NormalTok{train\_UMGY\_SmoothedDay\_effect }\OtherTok{\textless{}{-}} \ControlFlowTok{function}\NormalTok{(train\_set, }
                                          \AttributeTok{degree =} \ConstantTok{NA}\NormalTok{, }
                                          \AttributeTok{span =} \ConstantTok{NA}\NormalTok{,}
                                          \AttributeTok{lambda =} \DecValTok{0}\NormalTok{)\{}
  \ControlFlowTok{if}\NormalTok{ (}\FunctionTok{is.na}\NormalTok{(lambda)) \{}
    \FunctionTok{stop}\NormalTok{(}\StringTok{"Function: train\_UMGY\_SmoothedDay\_effect}
\StringTok{\textasciigrave{}lambda\textasciigrave{} is \textasciigrave{}NA\textasciigrave{}"}\NormalTok{)}
\NormalTok{  \}}
  
\NormalTok{  train\_set }\SpecialCharTok{|\textgreater{}} 
    \FunctionTok{calc\_day\_general\_effect}\NormalTok{(lambda) }\SpecialCharTok{|\textgreater{}}
    \FunctionTok{train\_UMGY\_SmoothedDay\_effect}\NormalTok{(degree, span)}
\NormalTok{\}}
\end{Highlighting}
\end{Shaded}

\begin{noteblock}
The source code of the \href{https://github.com/AzKurban-edX-DS/Capstone-MovieLens/blob/main/r/src/support-functions/UMGYD-effect.functions.R\#L97}{train\_UMGY\_SmoothedDay\_effect} function is also available in the \href{https://github.com/AzKurban-edX-DS/Capstone-MovieLens/blob/main/r/src/support-functions/UMGYD-effect.functions.R\#L1}{UMGYDE Model Support Functions} section of the \href{https://github.com/AzKurban-edX-DS/Capstone-MovieLens/blob/main/r/src/support-functions/UMGYD-effect.functions.R\#L97}{UMGYD-effect.functions.R} script on \emph{GitHub}.

\end{noteblock}

\subparagraph{Return}\label{return-9}

\hfill\break
A data frame object representing the \emph{User+Movie+Genre+Year+(Smoothed)Day (UMGYD) Effect}.

\newpage

\paragraph{\texorpdfstring{\href{https://github.com/AzKurban-edX-DS/Capstone-MovieLens/blob/main/r/src/support-functions/UMGYD-effect.functions.R\#L111}{train\_UMGY\_SmoothedDay\_effect.cv} Function}{train\_UMGY\_SmoothedDay\_effect.cv Function}}\label{func.train_UMGY_SmoothedDay_effect.cv}

\hfill\break

This is a helper function that calculates the \emph{Smoothed Day Effect} of the user ratings and is intended to be used in other functions and scripts (such as \hyperref[func.train_UMGY_SmoothedDay_effect.cv]{train\_UMGY\_SmoothedDay\_effect.cv}) described below.

\subparagraph{Parameters}\label{parameters-10}

\hfill\break

\textbf{\emph{degree}}

\hfill\break
A parameter to pass to the internally called \hyperref[func.calc_UMGY_SmoothedDay_effect]{calc\_UMGY\_SmoothedDay\_effect} function (see the \hyperref[func.train_UMGY_SmoothedDay_effect.cv.details]{Details} section below), meaning the degree of the polynomials to be used.

\textbf{\emph{span}}

\hfill\break
A parameter to pass to the internally called \hyperref[func.calc_UMGY_SmoothedDay_effect]{calc\_UMGY\_SmoothedDay\_effect} function (see the \hyperref[func.train_UMGY_SmoothedDay_effect.cv.details]{Details} section below) to control the degree of smoothing.

\textbf{\emph{lambda}}

\hfill\break
As explained \hyperref[sec.umgydem.support_functions.note.lambda]{above} in this section, this parameter is used for the \emph{Regularization} techniques. Otherwise, it is omitted, meaning its default (zero) value is used.

\phantomsection\label{func.train_UMGY_SmoothedDay_effect.cv.details}
\subparagraph{Details}\label{details-3}

\hfill\break
Under the hood, the function internally calls the \hyperref[func.calc_day_general_effect.cv]{calc\_day\_general\_effect.cv} and then \hyperref[func.calc_UMGY_SmoothedDay_effect]{calc\_UMGY\_SmoothedDay\_effect} described above, and returns the \emph{User+Movie+Genre+Year+(Smoothed)Day (UMGYD) Effect} data object that the latter inner function returns.

\subparagraph{Source Code}\label{source-code-9}

\hfill\break

The following is the source code of the function:

\begin{Shaded}
\begin{Highlighting}[]
\NormalTok{train\_UMGY\_SmoothedDay\_effect.cv }\OtherTok{\textless{}{-}} \ControlFlowTok{function}\NormalTok{(}\AttributeTok{degree =} \ConstantTok{NA}\NormalTok{, }
                                             \AttributeTok{span =} \ConstantTok{NA}\NormalTok{,}
                                             \AttributeTok{lambda =} \DecValTok{0}\NormalTok{)\{}
  \ControlFlowTok{if}\NormalTok{ (}\FunctionTok{is.na}\NormalTok{(lambda)) \{}
    \FunctionTok{stop}\NormalTok{(}\StringTok{"Function: train\_UMGY\_SmoothedDay\_effect.cv}
\StringTok{\textasciigrave{}lambda\textasciigrave{} is \textasciigrave{}NA\textasciigrave{}"}\NormalTok{)}
\NormalTok{  \}}
  
  \FunctionTok{calc\_day\_general\_effect.cv.cv}\NormalTok{(lambda) }\SpecialCharTok{|\textgreater{}}
    \FunctionTok{calc\_UMGY\_SmoothedDay\_effect}\NormalTok{(degree, span)}
\NormalTok{\}}
\end{Highlighting}
\end{Shaded}

\begin{noteblock}
The source code of the \href{https://github.com/AzKurban-edX-DS/Capstone-MovieLens/blob/main/r/src/support-functions/UMGYD-effect.functions.R\#L111}{train\_UMGY\_SmoothedDay\_effect.cv} function is also available in the \href{https://github.com/AzKurban-edX-DS/Capstone-MovieLens/blob/main/r/src/support-functions/UMGYD-effect.functions.R\#L1}{UMGYDE Model Support Functions} section of the \href{https://github.com/AzKurban-edX-DS/Capstone-MovieLens/blob/main/r/src/support-functions/UMGYD-effect.functions.R\#L111}{UMGYD-effect.functions.R} script on \emph{GitHub}.

\end{noteblock}

\subparagraph{Return}\label{return-10}

\hfill\break
A data frame object representing the \emph{User+Movie+Genre+Year+(Smoothed)Day (UMGYD) Effect}.

\paragraph{\texorpdfstring{\href{https://github.com/AzKurban-edX-DS/Capstone-MovieLens/blob/main/r/src/support-functions/UMGYD-effect.functions.R\#L123}{UMGY\_SmoothedDay\_effect.predict} Function}{UMGY\_SmoothedDay\_effect.predict Function}}\label{func.UMGY_SmoothedDay_effect.predict}

\hfill\break

This is a helper function that predicts user rating values on the \emph{Test Set} passed by the \hyperref[func.UMGY_SmoothedDay_effect.predict.params.test_set]{test\_set} parameter, taking into account the \emph{User+Movie+Genre+Year+(Smoothed)Day (UMGYD) Effect} passed by the \hyperref[func.UMGY_SmoothedDay_effect.predict.params.day_smoothed_effect]{day\_smoothed\_effect} parameter.

\subparagraph{Parameters}\label{parameters-11}

\hfill\break

\phantomsection\label{func.UMGY_SmoothedDay_effect.predict.params.test_set}
\textbf{\emph{test\_set}}

\hfill\break
A \emph{Test Set} dataset for the \emph{UMGYDE Model} validation.

\phantomsection\label{func.UMGY_SmoothedDay_effect.predict.params.day_smoothed_effect}
\textbf{\emph{day\_smoothed\_effect}}

\hfill\break
A data object representing the \emph{UMGYD Effect}, which can be returned by any of the following functions described above:

\begin{itemize}
\tightlist
\item
  \hyperref[func.train_UMGY_SmoothedDay_effect]{calc\_UMGY\_SmoothedDay\_effect};
\item
  \hyperref[func.train_UMGY_SmoothedDay_effect]{train\_UMGY\_SmoothedDay\_effect};
\item
  \hyperref[func.train_UMGY_SmoothedDay_effect.cv]{train\_UMGY\_SmoothedDay\_effect.cv}.
\end{itemize}

\subparagraph{Source Code}\label{source-code-10}

\hfill\break

Below is the source code of the function:

\begin{Shaded}
\begin{Highlighting}[]
\NormalTok{UMGY\_SmoothedDay\_effect.predict }\OtherTok{\textless{}{-}} \ControlFlowTok{function}\NormalTok{(test\_set, day\_smoothed\_effect) \{}
\NormalTok{  test\_set }\SpecialCharTok{|\textgreater{}}
    \FunctionTok{left\_join}\NormalTok{(edx.user\_effect, }\AttributeTok{by =} \StringTok{"userId"}\NormalTok{) }\SpecialCharTok{|\textgreater{}}
    \FunctionTok{left\_join}\NormalTok{(rglr.UM\_effect, }\AttributeTok{by =} \StringTok{"movieId"}\NormalTok{) }\SpecialCharTok{|\textgreater{}}
    \FunctionTok{left\_join}\NormalTok{(rglr.UMG\_effect, }\AttributeTok{by =} \StringTok{"movieId"}\NormalTok{) }\SpecialCharTok{|\textgreater{}}
    \FunctionTok{left\_join}\NormalTok{(date\_days\_map, }\AttributeTok{by =} \StringTok{"timestamp"}\NormalTok{) }\SpecialCharTok{|\textgreater{}}
    \FunctionTok{left\_join}\NormalTok{(rglr.UMGY\_effect, }\AttributeTok{by=}\StringTok{\textquotesingle{}year\textquotesingle{}}\NormalTok{) }\SpecialCharTok{|\textgreater{}}
    \FunctionTok{left\_join}\NormalTok{(day\_smoothed\_effect, }\AttributeTok{by=}\StringTok{\textquotesingle{}days\textquotesingle{}}\NormalTok{) }\SpecialCharTok{|\textgreater{}}
    \FunctionTok{mutate}\NormalTok{(}\AttributeTok{predicted =} \FunctionTok{clamp}\NormalTok{(mu }\SpecialCharTok{+}\NormalTok{ a }\SpecialCharTok{+}\NormalTok{ b }\SpecialCharTok{+}\NormalTok{ g }\SpecialCharTok{+} 
                               \FunctionTok{ifelse}\NormalTok{(}\FunctionTok{is.na}\NormalTok{(ye), }\DecValTok{0}\NormalTok{, ye) }\SpecialCharTok{+}
                               \FunctionTok{ifelse}\NormalTok{(}\FunctionTok{is.na}\NormalTok{(de\_smoothed), }
                                      \DecValTok{0}\NormalTok{, }
\NormalTok{                                      de\_smoothed))) }\SpecialCharTok{|\textgreater{}} 
    \FunctionTok{select}\NormalTok{(userId, movieId, timestamp, rating, predicted)}
\NormalTok{\}}
\end{Highlighting}
\end{Shaded}

\begin{noteblock}
The source code of the \href{https://github.com/AzKurban-edX-DS/Capstone-MovieLens/blob/main/r/src/support-functions/UMGYD-effect.functions.R\#L123}{UMGY\_SmoothedDay\_effect.predict} function is also available in the \href{https://github.com/AzKurban-edX-DS/Capstone-MovieLens/blob/main/r/src/support-functions/UMGYD-effect.functions.R\#L1}{UMGYDE Model Support Functions} section of the \href{https://github.com/AzKurban-edX-DS/Capstone-MovieLens/blob/main/r/src/support-functions/UMGYD-effect.functions.R\#L123}{UMGYD-effect.functions.R} script on \emph{GitHub}.

\end{noteblock}

\subparagraph{Return}\label{return-11}

\hfill\break

A data frame object containing predicted values.

\newpage

\paragraph{\texorpdfstring{\href{https://github.com/AzKurban-edX-DS/Capstone-MovieLens/blob/main/r/src/support-functions/UMGYD-effect.functions.R\#L139}{calc\_UMGY\_SmoothedDay\_effect.MSE} Function}{calc\_UMGY\_SmoothedDay\_effect.MSE Function}}\label{func.calc_UMGY_SmoothedDay_effect.MSE}

\hfill\break

The function calculates the \emph{Mean Squared Error (MSE)} of the \emph{UMGYDE Model} for the given \emph{Test Set} passed by the \hyperref[func.calc_UMGY_SmoothedDay_effect.MSE.params.test_set]{test\_set} parameter.

\subparagraph{Parameters}\label{parameters-12}

\hfill\break

\phantomsection\label{func.calc_UMGY_SmoothedDay_effect.MSE.params.test_set}
\textbf{\emph{test\_set}}

\hfill\break
A \emph{Test Set} dataset for the \emph{UMGYDE Model} validation.

\phantomsection\label{func.calc_UMGY_SmoothedDay_effect.MSE.params.day_smoothed_effect}
\textbf{\emph{day\_smoothed\_effect}}

\hfill\break
A data object representing the \emph{UMGYD Effect}, which can be returned by any of the following functions described above:

\begin{itemize}
\tightlist
\item
  \hyperref[func.train_UMGY_SmoothedDay_effect]{calc\_UMGY\_SmoothedDay\_effect};
\item
  \hyperref[func.train_UMGY_SmoothedDay_effect]{train\_UMGY\_SmoothedDay\_effect};
\item
  \hyperref[func.train_UMGY_SmoothedDay_effect.cv]{train\_UMGY\_SmoothedDay\_effect.cv}.
\end{itemize}

\subparagraph{Source Code}\label{source-code-11}

\hfill\break

The source code of the function is provided below:

\begin{Shaded}
\begin{Highlighting}[]
\NormalTok{calc\_UMGY\_SmoothedDay\_effect.MSE }\OtherTok{\textless{}{-}} \ControlFlowTok{function}\NormalTok{(test\_set, day\_smoothed\_effect) \{}
\NormalTok{  test\_set }\SpecialCharTok{|\textgreater{}}
    \FunctionTok{UMGY\_SmoothedDay\_effect.predict}\NormalTok{(day\_smoothed\_effect) }\SpecialCharTok{|\textgreater{}}
    \FunctionTok{mutate}\NormalTok{(}\AttributeTok{resid =}\NormalTok{ rating }\SpecialCharTok{{-}}\NormalTok{ predicted) }\SpecialCharTok{|\textgreater{}} 
    \FunctionTok{pull}\NormalTok{(resid) }\SpecialCharTok{|\textgreater{}} \FunctionTok{mse}\NormalTok{()}
\NormalTok{\}}
\end{Highlighting}
\end{Shaded}

\begin{noteblock}
The source code of the function \href{https://github.com/AzKurban-edX-DS/Capstone-MovieLens/blob/main/r/src/support-functions/UMGYD-effect.functions.R\#L139}{calc\_UMGY\_SmoothedDay\_effect.MSE} is also available in the \href{https://github.com/AzKurban-edX-DS/Capstone-MovieLens/blob/main/r/src/support-functions/UMGYD-effect.functions.R\#L1}{UMGYDE Model Support Functions} section of the \href{https://github.com/AzKurban-edX-DS/Capstone-MovieLens/blob/main/r/src/support-functions/UMGYD-effect.functions.R\#L139}{UMGYD-effect.functions.R} script on \emph{GitHub}.

\end{noteblock}

\subparagraph{Return}\label{return-12}

\hfill\break
The \emph{Mean Squared Error (MSE)} value of the \emph{UMGYDE Model} computed on the \emph{Test Set} passed by the \hyperref[func.calc_UMGY_SmoothedDay_effect.MSE.params.test_set]{test\_set} parameter.

\paragraph{\texorpdfstring{\href{https://github.com/AzKurban-edX-DS/Capstone-MovieLens/blob/main/r/src/support-functions/UMGYD-effect.functions.R\#L145}{calc\_UMGY\_SmoothedDay\_effect.MSE.cv} Function}{calc\_UMGY\_SmoothedDay\_effect.MSE.cv Function}}\label{func.calc_UMGY_SmoothedDay_effect.MSE.cv}

\hfill\break

The function calculates the \emph{5-Fold Cross Validation MSE} result for the \emph{UMGYD Effect} data passed by the \hyperref[func.calc_UMGY_SmoothedDay_effect.MSE.cv.params.day_smoothed_effect]{day\_smoothed\_effect} parameter.

\subparagraph{Parameters}\label{parameters-13}

\hfill\break

\phantomsection\label{func.calc_UMGY_SmoothedDay_effect.MSE.cv.params.day_smoothed_effect}
\textbf{\emph{day\_smoothed\_effect}}

\hfill\break
A data object representing the \emph{UMGYD Effect}, which can be returned by any of the following functions described above:

\begin{itemize}
\tightlist
\item
  \hyperref[func.train_UMGY_SmoothedDay_effect]{calc\_UMGY\_SmoothedDay\_effect};
\item
  \hyperref[func.train_UMGY_SmoothedDay_effect]{train\_UMGY\_SmoothedDay\_effect};
\item
  \hyperref[func.train_UMGY_SmoothedDay_effect.cv]{train\_UMGY\_SmoothedDay\_effect.cv}.
\end{itemize}

\subparagraph{Source Code}\label{source-code-12}

\hfill\break

Below is a simplified version of the function's source code:

\begin{Shaded}
\begin{Highlighting}[]
\NormalTok{calc\_UMGY\_SmoothedDay\_effect.MSE.cv }\OtherTok{\textless{}{-}} \ControlFlowTok{function}\NormalTok{(day\_smoothed\_effect)\{}
\NormalTok{  smth\_day\_effect\_MSEs }\OtherTok{\textless{}{-}} \FunctionTok{sapply}\NormalTok{(edx\_CV, }\ControlFlowTok{function}\NormalTok{(cv\_fold\_dat)\{}
\NormalTok{    cv\_fold\_dat}\SpecialCharTok{$}\NormalTok{validation\_set }\SpecialCharTok{|\textgreater{}} 
      \FunctionTok{calc\_UMGY\_SmoothedDay\_effect.MSE}\NormalTok{(day\_smoothed\_effect)}
\NormalTok{  \})}
  
  \FunctionTok{mean}\NormalTok{(smth\_day\_effect\_MSEs)}
\NormalTok{\}}
\end{Highlighting}
\end{Shaded}

\begin{noteblock}
The complete version of the source code of the \href{https://github.com/AzKurban-edX-DS/Capstone-MovieLens/blob/main/r/src/support-functions/UMGYD-effect.functions.R\#L145}{calc\_UMGY\_SmoothedDay\_effect.MSE.cv} function is available in the \href{https://github.com/AzKurban-edX-DS/Capstone-MovieLens/blob/main/r/src/support-functions/UMGYD-effect.functions.R\#L1}{UMGYDE Model Support Functions} section of the \href{https://github.com/AzKurban-edX-DS/Capstone-MovieLens/blob/main/r/src/support-functions/UMGYD-effect.functions.R\#L145}{UMGYD-effect.functions.R} script on \emph{GitHub}.

\end{noteblock}

\subparagraph{Return}\label{return-13}

\hfill\break
The \emph{Mean Squared Error (MSE)} value of the \emph{UMGYDE Model} computed as the mean of the \emph{5-Fold Cross Validation} results.

\paragraph{\texorpdfstring{\href{https://github.com/AzKurban-edX-DS/Capstone-MovieLens/blob/main/r/src/support-functions/UMGYD-effect.functions.R\#L161}{calc\_UMGY\_SmoothedDay\_effect.RMSE} Function}{calc\_UMGY\_SmoothedDay\_effect.RMSE Function}}\label{func.calc_UMGY_SmoothedDay_effect.RMSE}

\hfill\break

The function calculates the \emph{Root Mean Squared Error (RMSE)} of the \emph{UMGYDE Model} for the given \emph{Test Set} passed by the \hyperref[func.calc_UMGY_SmoothedDay_effect.RMSE.params.test_set]{test\_set} parameter.

\subparagraph{Parameters}\label{parameters-14}

\hfill\break

\phantomsection\label{func.calc_UMGY_SmoothedDay_effect.RMSE.params.test_set}
\textbf{\emph{test\_set}}

\hfill\break
A \emph{Test Set} dataset for the \emph{UMGYDE Model} validation.

\phantomsection\label{func.calc_UMGY_SmoothedDay_effect.RMSE.params.day_smoothed_effect}
\textbf{\emph{day\_smoothed\_effect}}

\hfill\break
A data object representing the \emph{UMGYD Effect}, which can be returned by any of the following functions described above:

\begin{itemize}
\tightlist
\item
  \hyperref[func.train_UMGY_SmoothedDay_effect]{calc\_UMGY\_SmoothedDay\_effect};
\item
  \hyperref[func.train_UMGY_SmoothedDay_effect]{train\_UMGY\_SmoothedDay\_effect};
\item
  \hyperref[func.train_UMGY_SmoothedDay_effect.cv]{train\_UMGY\_SmoothedDay\_effect.cv}.
\end{itemize}

\subparagraph{Source Code}\label{source-code-13}

\hfill\break

The source code of the function is provided below:

\begin{Shaded}
\begin{Highlighting}[]
\NormalTok{calc\_UMGY\_SmoothedDay\_effect.RMSE }\OtherTok{\textless{}{-}} \ControlFlowTok{function}\NormalTok{(test\_set, day\_smoothed\_effect)\{}
\NormalTok{  day\_smth\_effect\_MSe }\OtherTok{\textless{}{-}}\NormalTok{ test\_set }\SpecialCharTok{|\textgreater{}} 
    \FunctionTok{calc\_UMGY\_SmoothedDay\_effect.MSE}\NormalTok{(day\_smoothed\_effect)}
  
  \FunctionTok{sqrt}\NormalTok{(day\_smth\_effect\_MSe)}
\NormalTok{\}}
\end{Highlighting}
\end{Shaded}

\begin{noteblock}
The source code of the function \href{https://github.com/AzKurban-edX-DS/Capstone-MovieLens/blob/main/r/src/support-functions/UMGYD-effect.functions.R\#L161}{calc\_UMGY\_SmoothedDay\_effect.RMSE} is also available in the \href{https://github.com/AzKurban-edX-DS/Capstone-MovieLens/blob/main/r/src/support-functions/UMGYD-effect.functions.R\#L1}{UMGYDE Model Support Functions} section of the \href{https://github.com/AzKurban-edX-DS/Capstone-MovieLens/blob/main/r/src/support-functions/UMGYD-effect.functions.R\#L161}{UMGYD-effect.functions.R} script.

\end{noteblock}

\subparagraph{Return}\label{return-14}

\hfill\break
The \emph{Root Mean Squared Error (RMSE)} value of the \emph{UMGYDE Model} computed on the \emph{Test Set} passed by the \hyperref[func.calc_UMGY_SmoothedDay_effect.RMSE.params.test_set]{test\_set} parameter.

\paragraph{\texorpdfstring{\href{https://github.com/AzKurban-edX-DS/Capstone-MovieLens/blob/main/r/src/support-functions/UMGYD-effect.functions.R\#L167}{calc\_UMGY\_SmoothedDay\_effect.RMSE.cv} Function}{calc\_UMGY\_SmoothedDay\_effect.RMSE.cv Function}}\label{func.calc_UMGY_SmoothedDay_effect.RMSE.cv}

\hfill\break

The function calculates the \emph{5-Fold Cross Validation RMSE} result for the \emph{UMGYD Effect} data passed by the \hyperref[func.calc_UMGY_SmoothedDay_effect.RMSE.cv.params.day_smoothed_effect]{day\_smoothed\_effect} parameter.

\subparagraph{Parameters}\label{parameters-15}

\hfill\break

\phantomsection\label{func.calc_UMGY_SmoothedDay_effect.RMSE.cv.params.day_smoothed_effect}
\textbf{\emph{day\_smoothed\_effect}}

\hfill\break
A data object representing the \emph{UMGYD Effect}, which can be returned by any of the following functions described above:

\begin{itemize}
\tightlist
\item
  \hyperref[func.train_UMGY_SmoothedDay_effect]{calc\_UMGY\_SmoothedDay\_effect};
\item
  \hyperref[func.train_UMGY_SmoothedDay_effect]{train\_UMGY\_SmoothedDay\_effect};
\item
  \hyperref[func.train_UMGY_SmoothedDay_effect.cv]{train\_UMGY\_SmoothedDay\_effect.cv}.
\end{itemize}

\subparagraph{Source Code}\label{source-code-14}

\hfill\break

The source code of the function is provided below:

\begin{Shaded}
\begin{Highlighting}[]
\NormalTok{calc\_UMGY\_SmoothedDay\_effect.RMSE.cv }\OtherTok{\textless{}{-}} \ControlFlowTok{function}\NormalTok{(day\_smoothed\_effect)\{}
  \FunctionTok{sqrt}\NormalTok{(}\FunctionTok{calc\_UMGY\_SmoothedDay\_effect.MSE.cv}\NormalTok{(day\_smoothed\_effect))}
\NormalTok{\}}
\end{Highlighting}
\end{Shaded}

\begin{noteblock}
The source code of the function \href{https://github.com/AzKurban-edX-DS/Capstone-MovieLens/blob/main/r/src/support-functions/UMGYD-effect.functions.R\#L167}{calc\_UMGY\_SmoothedDay\_effect.RMSE.cv} is also available in the \href{https://github.com/AzKurban-edX-DS/Capstone-MovieLens/blob/main/r/src/support-functions/UMGYD-effect.functions.R\#L1}{UMGYDE Model Support Functions} section of the \href{https://github.com/AzKurban-edX-DS/Capstone-MovieLens/blob/main/r/src/support-functions/UMGYD-effect.functions.R\#L167}{UMGYD-effect.functions.R} script.

\end{noteblock}

\subparagraph{Return}\label{return-15}

\hfill\break
The \emph{Root Mean Squared Error (RMSE)} of the \emph{UMGYD Model} computed as the square root of the \emph{5-Fold Cross Validation MSE} result.

\newpage

\subsubsection{UMGYDE Model Building With Default Parameters}\label{umgyde-model-building-with-default-parameters}

\hfill\break

\begin{noteblock}
The complete source code of building and training the current model is available in the \href{https://github.com/AzKurban-edX-DS/Capstone-MovieLens/blob/main/r/src/capstone-movielens.main.R\#L1745}{UMGYDEM Training with default parameters} section of the \href{https://github.com/AzKurban-edX-DS/Capstone-MovieLens/blob/main/r/src/capstone-movielens.main.R\#L1745}{capstone-movielens.main.R} script on \emph{GitHub}.

\end{noteblock}

We are going to use our helper function \hyperref[func.train_UMGY_SmoothedDay_effect.cv]{train\_UMGY\_SmoothedDay\_effect.cv} described in the \hyperref[umgyde-model-helper-functions]{UMGYDE Model: Helper Functions} section, which, as we explained, internally calls the R function \href{https://www.rdocumentation.org/packages/stats/versions/3.6.2/topics/loess}{stats::loess}.

We will start by calling the \texttt{train\_UMGY\_SmoothedDay\_effect.cv} with default parameters, which means that the \texttt{loess} function will be implicitly called with \texttt{degree\ =\ 1} and \texttt{span\ =\ 0.75} parameters (the default values for our custom \hyperref[func.loess_de]{loess\_de} function, which, in turn, calls the \texttt{loess} under the hood).

Below, we provide the most significant line of the code for training our model using the \texttt{5-Fold\ Cross\ Validation} method:

\begin{Shaded}
\begin{Highlighting}[]
\NormalTok{  cv.UMGYDE.default\_params }\OtherTok{\textless{}{-}} \FunctionTok{train\_UMGY\_SmoothedDay\_effect.cv}\NormalTok{()}
\end{Highlighting}
\end{Shaded}

\begin{Shaded}
\begin{Highlighting}[]
\FunctionTok{str}\NormalTok{(cv.UMGYDE.default\_params)}
\end{Highlighting}
\end{Shaded}

\begin{verbatim}
## tibble [4,640 x 4] (S3: tbl_df/tbl/data.frame)
##  $ days       : int [1:4640] 0 385 388 389 392 393 394 396 397 399 ...
##  $ de         : num [1:4640] 0.6676 0.2939 0.1661 0.3206 -0.0871 ...
##  $ year       : num [1:4640] 1995 1996 1996 1996 1996 ...
##  $ de_smoothed: num [1:4640] 0.0834 0.0553 0.0552 0.0551 0.0549 ...
\end{verbatim}

Below is the visual representation of the UMGYD Effect data we have just computed:

\begin{Shaded}
\begin{Highlighting}[]
\NormalTok{cv.UMGYDE.default\_params }\SpecialCharTok{|\textgreater{}}
  \FunctionTok{ggplot}\NormalTok{(}\FunctionTok{aes}\NormalTok{(}\AttributeTok{x =}\NormalTok{ days)) }\SpecialCharTok{+}
  \FunctionTok{geom\_point}\NormalTok{(}\FunctionTok{aes}\NormalTok{(}\AttributeTok{y =}\NormalTok{ de), }\AttributeTok{size =} \DecValTok{3}\NormalTok{, }\AttributeTok{alpha =}\NormalTok{ .}\DecValTok{5}\NormalTok{, }\AttributeTok{color =} \StringTok{"grey"}\NormalTok{) }\SpecialCharTok{+} 
  \FunctionTok{geom\_line}\NormalTok{(}\FunctionTok{aes}\NormalTok{(}\AttributeTok{y =}\NormalTok{ de\_smoothed), }\AttributeTok{color =} \StringTok{"red"}\NormalTok{)}
\end{Highlighting}
\end{Shaded}

\pandocbounded{\includegraphics[keepaspectratio]{capstone-movielens.report_files/figure-latex/unnamed-chunk-281-1.pdf}}

Let's calculate the \emph{RMSE} of the current model using the \hyperref[func.calc_UMGY_SmoothedDay_effect.RMSE.cv]{calc\_UMGY\_SmoothedDay\_effect.RMSE.cv} function described above:

\begin{Shaded}
\begin{Highlighting}[]
\NormalTok{cv.UMGYDE.default\_params.RMSE }\OtherTok{\textless{}{-}}\NormalTok{ cv.UMGYDE.default\_params }\SpecialCharTok{|\textgreater{}}
  \FunctionTok{calc\_UMGY\_SmoothedDay\_effect.RMSE.cv}\NormalTok{()}

\NormalTok{RMSEs.ResultTibble.UMGYDE }\OtherTok{\textless{}{-}}\NormalTok{ RMSEs.ResultTibble.rglr.UMGYE }\SpecialCharTok{|\textgreater{}} 
  \FunctionTok{RMSEs.AddRow}\NormalTok{(}\StringTok{"UMGYDE (Default) Model"}\NormalTok{, }
\NormalTok{               cv.UMGYDE.default\_params.RMSE,}
               \AttributeTok{comment =} \StringTok{"User+Movie+Genre+Year+Day Effect (UMGYDE) Model }
\StringTok{computed using \textasciigrave{}stats::loess\textasciigrave{} function with \textasciigrave{}degree=1\textasciigrave{} \& \textasciigrave{}span=0.75\textasciigrave{} parameter values."}\NormalTok{)}
\end{Highlighting}
\end{Shaded}

\begin{Shaded}
\begin{Highlighting}[]
\FunctionTok{RMSE\_kable}\NormalTok{(RMSEs.ResultTibble.UMGYDE)}
\end{Highlighting}
\end{Shaded}

\begin{tabular}{>{\raggedright\arraybackslash}p{15em}>{\centering\arraybackslash}p{5em}>{\raggedright\arraybackslash}p{30em}}
\toprule
\textbf{Method} & \textbf{RMSE} & \textbf{Comment}\\
\midrule
Project Objective & 0.8649000 & \\
Overall Mean Rating Model & 1.0603462 & \\
User Effect Model & 0.9697962 & \\
User+Movie Effect Model & 0.8732081 & \\
Regularized UME Model & 0.8729730 & Computed for `lambda` = 0.38745002746582\\
\addlinespace
UMGE Model & 0.8729730 & User+Movie+Genre Effect (UMGE) Model\\
Regularized UMGE Model & 0.8729728 & Computed for `lambda` = 0.0359375\\
UMGYE Model & 0.8723973 & User+Movie+Genre+Year Effect (UMGYE) Model\\
Regularized UMGYE Model & 0.8721857 & Computed for `lambda` = 233.77668762207\\
UMGYDE (Default) Model & 0.8722410 & User+Movie+Genre+Year+Day Effect (UMGYDE) Model 
computed using `stats::loess` function with `degree=1` \& `span=0.75` parameter values.\\
\bottomrule
\end{tabular}

Unfortunately, with the default parameters, we obtained an even worse result than before.
Let's tune the current model by adjusting the \texttt{degree} and \texttt{span} parameters and see what we get.

\begin{noteblock}
Since the \texttt{degree} parameter can only take three discrete values, 0, 1, and 2, we will perform the tuning in three steps - one per each possible value of the \texttt{degree} parameter.

\end{noteblock}

\newpage

\subsubsection{\texorpdfstring{UMGYDE Model Tuning by \texttt{degree} and \texttt{span} Parameters}{UMGYDE Model Tuning by degree and span Parameters}}\label{umgyde-model-tuning-by-degree-and-span-parameters}

\hfill\break

\paragraph{UMGYDE Model Tuning: Support Functions}\label{umgyde-model-tuning-support-functions}

\hfill\break

\subparagraph{\texorpdfstring{\href{https://github.com/AzKurban-edX-DS/Capstone-MovieLens/blob/main/r/src/support-functions/UMGYD-effect.functions.R\#L171}{train\_UMGY\_SmoothedDay\_effect.RMSE.cv} Function}{train\_UMGY\_SmoothedDay\_effect.RMSE.cv Function}}\label{func.train_UMGY_SmoothedDay_effect.RMSE.cv}

\hfill\break
The function is designed to tune the \href{https://www.rdocumentation.org/packages/stats/versions/3.6.2/topics/loess}{stats::loess} function's parameters (\hyperref[func.train_UMGY_SmoothedDay_effect.RMSE.cv.params.degree]{degree} and \hyperref[func.train_UMGY_SmoothedDay_effect.RMSE.cv.params.span]{span}), which is ultimately called internally under the hood.

Parameters

\hfill\break

\phantomsection\label{func.train_UMGY_SmoothedDay_effect.RMSE.cv.params.degree}
\begin{itemize}
\tightlist
\item
  \textbf{\emph{degree:}}
  A parameter to pass to the internally called \hyperref[func.train_UMGY_SmoothedDay_effect.cv]{train\_UMGY\_SmoothedDay\_effect.cv} function (see the \hyperref[func.train_UMGY_SmoothedDay_effect.RMSE.cv.details]{Details} section below) and, ultimately under the hood, to the \href{https://www.rdocumentation.org/packages/stats/versions/3.6.2/topics/loess}{stats::loess} to specify the degree of the polynomials to be used.
\end{itemize}

\phantomsection\label{func.train_UMGY_SmoothedDay_effect.RMSE.cv.params.span}
\begin{itemize}
\tightlist
\item
  \textbf{\emph{span:}}
  A parameter to pass to the internally called \hyperref[func.train_UMGY_SmoothedDay_effect.cv]{train\_UMGY\_SmoothedDay\_effect.cv} function (see the \hyperref[func.train_UMGY_SmoothedDay_effect.RMSE.cv.details]{Details} section below) and, ultimately under the hood, to the \href{https://www.rdocumentation.org/packages/stats/versions/3.6.2/topics/loess}{stats::loess} to control the degree of smoothing.
\end{itemize}

\phantomsection\label{func.train_UMGY_SmoothedDay_effect.RMSE.cv.details}
Details

\hfill\break
Under the hood, the function internally calls the \hyperref[func.train_UMGY_SmoothedDay_effect.cv]{train\_UMGY\_SmoothedDay\_effect.cv} and then \hyperref[func.calc_UMGY_SmoothedDay_effect.RMSE.cv]{calc\_UMGY\_SmoothedDay\_effect.RMSE.cv} described above to compute the \texttt{RMSE} of the model to be tuned.

Source Code

\hfill\break
Below is a simplified version of the function's source code:

\begin{Shaded}
\begin{Highlighting}[]
\NormalTok{train\_UMGY\_SmoothedDay\_effect.RMSE.cv }\OtherTok{\textless{}{-}} \ControlFlowTok{function}\NormalTok{(}\AttributeTok{degree =} \ConstantTok{NA}\NormalTok{, }\AttributeTok{span =} \ConstantTok{NA}\NormalTok{) \{}
  \FunctionTok{train\_UMGY\_SmoothedDay\_effect.cv}\NormalTok{(degree, span) }\SpecialCharTok{|\textgreater{}}
    \FunctionTok{calc\_UMGY\_SmoothedDay\_effect.RMSE.cv}\NormalTok{()}
\NormalTok{\}}
\end{Highlighting}
\end{Shaded}

\begin{noteblock}
The complete version of the source code of the \href{https://github.com/AzKurban-edX-DS/Capstone-MovieLens/blob/main/r/src/support-functions/UMGYD-effect.functions.R\#L171}{train\_UMGY\_SmoothedDay\_effect.RMSE.cv} function is available in the \href{https://github.com/AzKurban-edX-DS/Capstone-MovieLens/blob/main/r/src/support-functions/UMGYD-effect.functions.R\#L170}{Tuning \texttt{loess} Parameters} section of the \href{https://github.com/AzKurban-edX-DS/Capstone-MovieLens/blob/main/r/src/support-functions/UMGYD-effect.functions.R\#L171}{UMGYD-effect.functions.R} script on \emph{GitHub}.

\end{noteblock}

Return

\hfill\break
The \emph{Root Mean Squared Error (RMSE)} of the \emph{UMGYD Model}.

\subparagraph{\texorpdfstring{\href{https://github.com/AzKurban-edX-DS/Capstone-MovieLens/blob/main/r/src/support-functions/UMGYD-effect.functions.R\#L190}{train\_UMGY\_SmoothedDay\_effect.RMSE.cv.degree0} Function}{train\_UMGY\_SmoothedDay\_effect.RMSE.cv.degree0 Function}}\label{func.train_UMGY_SmoothedDay_effect.RMSE.cv.degree0}

\hfill\break
The overloaded version of the \hyperref[func.train_UMGY_SmoothedDay_effect.RMSE.cv]{train\_UMGY\_SmoothedDay\_effect.RMSE.cv} function described above with the fixed \hyperref[func.train_UMGY_SmoothedDay_effect.RMSE.cv.params.degree]{degree} parameter value of 0.

Parameters

\hfill\break

\phantomsection\label{func.train_UMGY_SmoothedDay_effect.RMSE.cv.degree0.params.span}
\begin{itemize}
\tightlist
\item
  \textbf{\emph{span:}}
  A parameter to pass to the internally called \hyperref[func.train_UMGY_SmoothedDay_effect.RMSE.cv]{train\_UMGY\_SmoothedDay\_effect.RMSE.cv} function (see the \hyperref[func.train_UMGY_SmoothedDay_effect.RMSE.cv.degree0.details]{Details} section below) and, ultimately under the hood, to the \href{https://www.rdocumentation.org/packages/stats/versions/3.6.2/topics/loess}{stats::loess} to control the degree of smoothing.
\end{itemize}

\phantomsection\label{func.train_UMGY_SmoothedDay_effect.RMSE.cv.degree0.details}
Details

\hfill\break
Under the hood, the function internally calls the \hyperref[func.train_UMGY_SmoothedDay_effect.RMSE.cv]{train\_UMGY\_SmoothedDay\_effect.RMSE.cv} described above with the fixed \hyperref[func.train_UMGY_SmoothedDay_effect.RMSE.cv.params.degree]{degree} parameter value of 0 and the \hyperref[func.train_UMGY_SmoothedDay_effect.RMSE.cv.params.span]{span} parameter value taken from the \hyperref[func.train_UMGY_SmoothedDay_effect.RMSE.cv.degree0.params.span]{parameter of the same name} of this function to compute the \texttt{RMSE} of the model to be tuned.

Source Code

\hfill\break
Below is the function's source code:

\begin{Shaded}
\begin{Highlighting}[]
\NormalTok{train\_UMGY\_SmoothedDay\_effect.RMSE.cv.degree0 }\OtherTok{\textless{}{-}} \ControlFlowTok{function}\NormalTok{(span) \{}
  \FunctionTok{train\_UMGY\_SmoothedDay\_effect.RMSE.cv}\NormalTok{(}\AttributeTok{degree =} \DecValTok{0}\NormalTok{, span)}
\NormalTok{\}}
\end{Highlighting}
\end{Shaded}

\begin{noteblock}
The source code of the \href{https://github.com/AzKurban-edX-DS/Capstone-MovieLens/blob/main/r/src/support-functions/UMGYD-effect.functions.R\#L190}{train\_UMGY\_SmoothedDay\_effect.RMSE.cv.degree0} function is also available in the \href{https://github.com/AzKurban-edX-DS/Capstone-MovieLens/blob/main/r/src/support-functions/UMGYD-effect.functions.R\#L170}{Tuning \texttt{loess} Parameters} section of the \href{https://github.com/AzKurban-edX-DS/Capstone-MovieLens/blob/main/r/src/support-functions/UMGYD-effect.functions.R\#L190}{UMGYD-effect.functions.R} script on \emph{GitHub}.

\end{noteblock}

Return

\hfill\break
\emph{Root Mean Square Error (RMSE)} of the \emph{UMGYD} model trained using the \href{https://www.rdocumentation.org/packages/stats/versions/3.6.2/topics/loess}{stats::loess} function called with the given \hyperref[func.train_UMGY_SmoothedDay_effect.RMSE.cv.degree0.params.span]{span} parameter and the \texttt{degree} parameter value equal to 0.

\newpage

\subparagraph{\texorpdfstring{\href{https://github.com/AzKurban-edX-DS/Capstone-MovieLens/blob/main/r/src/support-functions/UMGYD-effect.functions.R\#L193}{train\_UMGY\_SmoothedDay\_effect.RMSE.cv.degree1} Function}{train\_UMGY\_SmoothedDay\_effect.RMSE.cv.degree1 Function}}\label{func.train_UMGY_SmoothedDay_effect.RMSE.cv.degree1}

\hfill\break
The overloaded version of the \hyperref[func.train_UMGY_SmoothedDay_effect.RMSE.cv]{train\_UMGY\_SmoothedDay\_effect.RMSE.cv} function described above with the fixed \hyperref[func.train_UMGY_SmoothedDay_effect.RMSE.cv.params.degree]{degree} parameter value of 1.

Parameters

\hfill\break

\phantomsection\label{func.train_UMGY_SmoothedDay_effect.RMSE.cv.degree1.params.span}
\begin{itemize}
\tightlist
\item
  \textbf{\emph{span:}}
  A parameter to pass to the internally called \hyperref[func.train_UMGY_SmoothedDay_effect.RMSE.cv]{train\_UMGY\_SmoothedDay\_effect.RMSE.cv} function (see the \hyperref[func.train_UMGY_SmoothedDay_effect.RMSE.cv.degree1.details]{Details} section below) and, ultimately under the hood, to the \href{https://www.rdocumentation.org/packages/stats/versions/3.6.2/topics/loess}{stats::loess} to control the degree of smoothing.
\end{itemize}

\phantomsection\label{func.train_UMGY_SmoothedDay_effect.RMSE.cv.degree1.details}
Details

\hfill\break
Under the hood, the function internally calls the \hyperref[func.train_UMGY_SmoothedDay_effect.RMSE.cv]{train\_UMGY\_SmoothedDay\_effect.RMSE.cv} described above with the fixed \hyperref[func.train_UMGY_SmoothedDay_effect.RMSE.cv.params.degree]{degree} parameter value of 1 and the \hyperref[func.train_UMGY_SmoothedDay_effect.RMSE.cv.params.span]{span} parameter value taken from the \hyperref[func.train_UMGY_SmoothedDay_effect.RMSE.cv.degree1.params.span]{parameter of the same name} of this function to compute the \texttt{RMSE} of the model to be tuned.

Source Code

\hfill\break
Below is the function's source code:

\begin{Shaded}
\begin{Highlighting}[]
\NormalTok{train\_UMGY\_SmoothedDay\_effect.RMSE.cv.degree1 }\OtherTok{\textless{}{-}} \ControlFlowTok{function}\NormalTok{(span) \{}
  \FunctionTok{train\_UMGY\_SmoothedDay\_effect.RMSE.cv}\NormalTok{(}\AttributeTok{degree =} \DecValTok{1}\NormalTok{, span)}
\NormalTok{\}}
\end{Highlighting}
\end{Shaded}

\begin{noteblock}
The source code of the \href{https://github.com/AzKurban-edX-DS/Capstone-MovieLens/blob/main/r/src/support-functions/UMGYD-effect.functions.R\#L193}{train\_UMGY\_SmoothedDay\_effect.RMSE.cv.degree1} function is also available in the \href{https://github.com/AzKurban-edX-DS/Capstone-MovieLens/blob/main/r/src/support-functions/UMGYD-effect.functions.R\#L170}{Tuning \texttt{loess} Parameters} section of the \href{https://github.com/AzKurban-edX-DS/Capstone-MovieLens/blob/main/r/src/support-functions/UMGYD-effect.functions.R\#L193}{UMGYD-effect.functions.R} script on \emph{GitHub}.

\end{noteblock}

Return

\hfill\break
\emph{Root Mean Square Error (RMSE)} of the \emph{UMGYD} model trained using the \href{https://www.rdocumentation.org/packages/stats/versions/3.6.2/topics/loess}{stats::loess} function called with the given \hyperref[func.train_UMGY_SmoothedDay_effect.RMSE.cv.degree1.params.span]{span} parameter and the \texttt{degree} parameter value equal to 1.

\subparagraph{\texorpdfstring{\href{https://github.com/AzKurban-edX-DS/Capstone-MovieLens/blob/main/r/src/support-functions/UMGYD-effect.functions.R\#L196}{train\_UMGY\_SmoothedDay\_effect.RMSE.cv.degree2} Function}{train\_UMGY\_SmoothedDay\_effect.RMSE.cv.degree2 Function}}\label{func.train_UMGY_SmoothedDay_effect.RMSE.cv.degree2}

\hfill\break
The overloaded version of the \hyperref[func.train_UMGY_SmoothedDay_effect.RMSE.cv]{train\_UMGY\_SmoothedDay\_effect.RMSE.cv} function described above with the fixed \hyperref[func.train_UMGY_SmoothedDay_effect.RMSE.cv.params.degree]{degree} parameter value of 2.

Parameters

\hfill\break

\phantomsection\label{func.train_UMGY_SmoothedDay_effect.RMSE.cv.degree2.params.span}
\begin{itemize}
\tightlist
\item
  \textbf{\emph{span:}}
  A parameter to pass to the internally called \hyperref[func.train_UMGY_SmoothedDay_effect.RMSE.cv]{train\_UMGY\_SmoothedDay\_effect.RMSE.cv} function (see the \hyperref[func.train_UMGY_SmoothedDay_effect.RMSE.cv.degree2.details]{Details} section below) and, ultimately under the hood, to the \href{https://www.rdocumentation.org/packages/stats/versions/3.6.2/topics/loess}{stats::loess} to control the degree of smoothing.
\end{itemize}

\phantomsection\label{func.train_UMGY_SmoothedDay_effect.RMSE.cv.degree2.details}
Details

\hfill\break
Under the hood, the function internally calls the \hyperref[func.train_UMGY_SmoothedDay_effect.RMSE.cv]{train\_UMGY\_SmoothedDay\_effect.RMSE.cv} described above with the fixed \hyperref[func.train_UMGY_SmoothedDay_effect.RMSE.cv.params.degree]{degree} parameter value of 2 and the \hyperref[func.train_UMGY_SmoothedDay_effect.RMSE.cv.params.span]{span} parameter value taken from the \hyperref[func.train_UMGY_SmoothedDay_effect.RMSE.cv.degree2.params.span]{parameter of the same name} of this function to compute the \texttt{RMSE} of the model to be tuned.

Source Code

\hfill\break
Below is the function's source code:

\begin{Shaded}
\begin{Highlighting}[]
\NormalTok{train\_UMGY\_SmoothedDay\_effect.RMSE.cv.degree2 }\OtherTok{\textless{}{-}} \ControlFlowTok{function}\NormalTok{(span) \{}
  \FunctionTok{train\_UMGY\_SmoothedDay\_effect.RMSE.cv}\NormalTok{(}\AttributeTok{degree =} \DecValTok{2}\NormalTok{, span)}
\NormalTok{\}}
\end{Highlighting}
\end{Shaded}

\begin{noteblock}
The source code of the \href{https://github.com/AzKurban-edX-DS/Capstone-MovieLens/blob/main/r/src/support-functions/UMGYD-effect.functions.R\#L196}{train\_UMGY\_SmoothedDay\_effect.RMSE.cv.degree2} function is also available in the \href{https://github.com/AzKurban-edX-DS/Capstone-MovieLens/blob/main/r/src/support-functions/UMGYD-effect.functions.R\#L170}{Tuning \texttt{loess} Parameters} section of the \href{https://github.com/AzKurban-edX-DS/Capstone-MovieLens/blob/main/r/src/support-functions/UMGYD-effect.functions.R\#L196}{UMGYD-effect.functions.R} script on \emph{GitHub}.

\end{noteblock}

Return

\hfill\break
\emph{Root Mean Square Error (RMSE)} of the \emph{UMGYD} model trained using the \href{https://www.rdocumentation.org/packages/stats/versions/3.6.2/topics/loess}{stats::loess} function called with the given \hyperref[func.train_UMGY_SmoothedDay_effect.RMSE.cv.degree2.params.span]{span} parameter and the \texttt{degree} parameter value of 2.

\newpage

\paragraph{\texorpdfstring{UMGYDE Model Tuning: Step 1 (\texttt{degree\ =\ 0})}{UMGYDE Model Tuning: Step 1 (degree = 0)}}\label{umgyde-model-tuning-step-1-degree-0}

\hfill\break

\subparagraph{\texorpdfstring{Pre-configuration: \texttt{span} Range Determination}{Pre-configuration: span Range Determination}}\label{pre-configuration-span-range-determination}

\hfill\break

\begin{noteblock}
The complete source code of the solution described in this section is available in the \href{https://github.com/AzKurban-edX-DS/Capstone-MovieLens/blob/main/r/src/capstone-movielens.main.R\#L1827}{UMGYDE Model Tuning: Pre-configuration (\texttt{degree\ =\ 0})} section of the \href{https://github.com/AzKurban-edX-DS/Capstone-MovieLens/blob/main/r/src/capstone-movielens.main.R\#L1827}{capstone-movielens.main.R} script on \emph{GitHub}.

\end{noteblock}

Below the most significant part of the code performing the current model tuning, the \emph{Pre-configuration Step} is provided:

\begin{Shaded}
\begin{Highlighting}[]
\NormalTok{  spans }\OtherTok{\textless{}{-}} \FunctionTok{seq}\NormalTok{(}\FloatTok{0.0005}\NormalTok{, }\DecValTok{1}\NormalTok{, }\FloatTok{0.001}\NormalTok{)}
\NormalTok{  lss.UMGYDE.preset.degree0.result }\OtherTok{\textless{}{-}} 
    \FunctionTok{tune.model\_param}\NormalTok{(spans, train\_UMGY\_SmoothedDay\_effect.RMSE.cv.degree0)}
\end{Highlighting}
\end{Shaded}

\begin{Shaded}
\begin{Highlighting}[]
\FunctionTok{str}\NormalTok{(lss.UMGYDE.preset.degree0.result)}
\end{Highlighting}
\end{Shaded}

\begin{verbatim}
## List of 2
##  $ tuned.result:'data.frame':    5 obs. of  2 variables:
##   ..$ RMSE           : num [1:5] 0.872 0.871 0.871 0.872 0.872
##   ..$ parameter.value: num [1:5] 0.0005 0.0015 0.0025 0.0035 0.0045
##  $ best_result : Named num [1:2] 0.0015 0.8712
##   ..- attr(*, "names")= chr [1:2] "param.best_value" "best_RMSE"
\end{verbatim}

Below is the visual representation of the \texttt{span} parameter values range we have figured out in this step of tuning:

\begin{Shaded}
\begin{Highlighting}[]
\NormalTok{plt.title }\OtherTok{=} \StringTok{"UMGYDE Model, \textasciigrave{}loess\textasciigrave{} function args: range of \textasciigrave{}span\textasciigrave{} values for \textasciigrave{}degree = 0\textasciigrave{}"}
\NormalTok{lss.UMGYDE.preset.degree0.result}\SpecialCharTok{$}\NormalTok{tuned.result }\SpecialCharTok{|\textgreater{}}
  \FunctionTok{data.plot}\NormalTok{(plt.title,}
              \AttributeTok{xname =} \StringTok{"parameter.value"}\NormalTok{, }
              \AttributeTok{yname =} \StringTok{"RMSE"}\NormalTok{, }
              \AttributeTok{xlabel =} \StringTok{"spans"}\NormalTok{, }
              \AttributeTok{ylabel =} \StringTok{"RMSE"}\NormalTok{)}
\end{Highlighting}
\end{Shaded}

\pandocbounded{\includegraphics[keepaspectratio]{capstone-movielens.report_files/figure-latex/unnamed-chunk-290-1.pdf}}

\begin{Shaded}
\begin{Highlighting}[]
\FunctionTok{rm}\NormalTok{(plt.title)}
\end{Highlighting}
\end{Shaded}

\subparagraph{\texorpdfstring{UMGYDE Model Fine-tuning (\texttt{degree\ =\ 0})}{UMGYDE Model Fine-tuning (degree = 0)}}\label{umgyde-model-fine-tuning-degree-0}

\hfill\break

\begin{noteblock}
The complete source code of the solution described in this section is available in the \href{https://github.com/AzKurban-edX-DS/Capstone-MovieLens/blob/main/r/src/capstone-movielens.main.R\#L1893}{UMGYDE Model Fine-tuning (\texttt{degree\ =\ 0})} section of the \href{https://github.com/AzKurban-edX-DS/Capstone-MovieLens/blob/main/r/src/capstone-movielens.main.R\#L1893}{capstone-movielens.main.R} script on \emph{GitHub}.

\end{noteblock}

Below the most significant part of the code performing the current model \emph{Fine-tuning} is provided:

\begin{Shaded}
\begin{Highlighting}[]
\NormalTok{lss.fine\_tune.loop\_starter }\OtherTok{\textless{}{-}} 
  \FunctionTok{c}\NormalTok{(lss.UMGYDE.preset.degree0.result}\SpecialCharTok{$}\NormalTok{tuned.result}\SpecialCharTok{$}\NormalTok{parameter.value[}\DecValTok{1}\NormalTok{], }
\NormalTok{    lss.UMGYDE.preset.degree0.result}\SpecialCharTok{$}\NormalTok{tuned.result}\SpecialCharTok{$}\NormalTok{parameter.value[}\DecValTok{3}\NormalTok{], }
    \DecValTok{8}\NormalTok{)}

\NormalTok{cache\_file.base\_name }\OtherTok{\textless{}{-}} \StringTok{"UMGYDE.degree0.tuning{-}span"}

\NormalTok{lss.UMGYDE.fine\_tune.degree0.result }\OtherTok{\textless{}{-}} 
  \FunctionTok{model.tune.param\_range}\NormalTok{(lss.fine\_tune.loop\_starter,}
\NormalTok{                         UMGYDE.fine\_tune.degree0.data.path,}
\NormalTok{                         cache\_file.base\_name,}
\NormalTok{                         train\_UMGY\_SmoothedDay\_effect.RMSE.cv.degree0)}
\end{Highlighting}
\end{Shaded}

\begin{Shaded}
\begin{Highlighting}[]
\NormalTok{lss.fine\_tune.loop\_starter}
\end{Highlighting}
\end{Shaded}

\begin{verbatim}
## [1] 0.0005 0.0025 8.0000
\end{verbatim}

\begin{Shaded}
\begin{Highlighting}[]
\FunctionTok{str}\NormalTok{(lss.UMGYDE.fine\_tune.degree0.result)}
\end{Highlighting}
\end{Shaded}

\begin{verbatim}
## List of 3
##  $ best_result           : Named num [1:2] 0.00109 0.87111
##   ..- attr(*, "names")= chr [1:2] "param.best_value" "best_RMSE"
##  $ param_values.endpoints: num [1:3] 1.06e-03 1.31e-03 3.13e-05
##  $ tuned.result          :'data.frame':  9 obs. of  2 variables:
##   ..$ parameter.value: num [1:9] 0.00106 0.00109 0.00113 0.00116 0.00119 ...
##   ..$ RMSE           : num [1:9] 0.871 0.871 0.871 0.871 0.871 ...
\end{verbatim}

Below is the visual representation of the current model \emph{Fine-tuning} results for the parameter value: \texttt{degree\ =\ 0}:

\begin{Shaded}
\begin{Highlighting}[]
\NormalTok{plt.title }\OtherTok{=} \StringTok{"Fine{-}tuned UMGYDE Model with \textasciigrave{}loess\textasciigrave{} parameter: \textasciigrave{}degree\textasciigrave{} = 0"}

\NormalTok{lss.UMGYDE.fine\_tune.degree0.result}\SpecialCharTok{$}\NormalTok{tuned.result }\SpecialCharTok{|\textgreater{}}
  \FunctionTok{data.plot}\NormalTok{(plt.title, }
              \AttributeTok{xname =} \StringTok{"parameter.value"}\NormalTok{, }
              \AttributeTok{yname =} \StringTok{"RMSE"}\NormalTok{, }
              \AttributeTok{xlabel =} \StringTok{"spans"}\NormalTok{, }
              \AttributeTok{ylabel =} \StringTok{"RMSE"}\NormalTok{)}
\end{Highlighting}
\end{Shaded}

\pandocbounded{\includegraphics[keepaspectratio]{capstone-movielens.report_files/figure-latex/unnamed-chunk-293-1.pdf}}

\begin{Shaded}
\begin{Highlighting}[]
\FunctionTok{rm}\NormalTok{(plt.title)}
\end{Highlighting}
\end{Shaded}

Now, we can add the \emph{UMGYDE Model Tuning} result for \texttt{degree\ =\ 0} to our \emph{Result Table}:

\begin{Shaded}
\begin{Highlighting}[]
\NormalTok{RMSEs.ResultTibble.UMGYDE0 }\OtherTok{\textless{}{-}}\NormalTok{ RMSEs.ResultTibble.UMGYDE }\SpecialCharTok{|\textgreater{}} 
  \FunctionTok{RMSEs.AddRow}\NormalTok{(}\StringTok{"Tuned UMGYDE.d0 Model"}\NormalTok{, }
\NormalTok{               lss.UMGYDE.fine\_tune.degree0.result.best\_RMSE,}
               \AttributeTok{comment =} \StringTok{"UMGYDE Model computed using function call: \textasciigrave{}loess(degree = 0, span = \%1)\textasciigrave{}"} \SpecialCharTok{|\textgreater{}}
                 \FunctionTok{msg.glue}\NormalTok{(lss.UMGYDE.fine\_tune.degree0.result.best\_span))}
\end{Highlighting}
\end{Shaded}

\begin{Shaded}
\begin{Highlighting}[]
\FunctionTok{RMSE\_kable}\NormalTok{(RMSEs.ResultTibble.UMGYDE0)}
\end{Highlighting}
\end{Shaded}

\begin{tabular}{>{\raggedright\arraybackslash}p{15em}>{\centering\arraybackslash}p{5em}>{\raggedright\arraybackslash}p{30em}}
\toprule
\textbf{Method} & \textbf{RMSE} & \textbf{Comment}\\
\midrule
Project Objective & 0.8649000 & \\
Overall Mean Rating Model & 1.0603462 & \\
User Effect Model & 0.9697962 & \\
User+Movie Effect Model & 0.8732081 & \\
Regularized UME Model & 0.8729730 & Computed for `lambda` = 0.38745002746582\\
\addlinespace
UMGE Model & 0.8729730 & User+Movie+Genre Effect (UMGE) Model\\
Regularized UMGE Model & 0.8729728 & Computed for `lambda` = 0.0359375\\
UMGYE Model & 0.8723973 & User+Movie+Genre+Year Effect (UMGYE) Model\\
Regularized UMGYE Model & 0.8721857 & Computed for `lambda` = 233.77668762207\\
UMGYDE (Default) Model & 0.8722410 & User+Movie+Genre+Year+Day Effect (UMGYDE) Model 
computed using `loess` function with default `degree` \& `span` parameters.\\
\addlinespace
Tuned UMGYDE.d0 Model & 0.8711131 & UMGYDE Model computed using function call: `loess(degree = 0, span = 0.00109375)`\\
\bottomrule
\end{tabular}

\paragraph{\texorpdfstring{UMGYDE Model Tuning: Step 2 (\texttt{degree\ =\ 1})}{UMGYDE Model Tuning: Step 2 (degree = 1)}}\label{umgyde-model-tuning-step-2-degree-1}

\hfill\break

\subparagraph{\texorpdfstring{Pre-configuration: \texttt{span} Range Determination}{Pre-configuration: span Range Determination}}\label{pre-configuration-span-range-determination-1}

\hfill\break

\begin{noteblock}
The complete source code of the solution described in this section is available in the \href{https://github.com/AzKurban-edX-DS/Capstone-MovieLens/blob/main/r/src/capstone-movielens.main.R\#L1944}{UMGYDE Model Tuning: Pre-configuration (\texttt{degree\ =\ 1})} section of the \href{https://github.com/AzKurban-edX-DS/Capstone-MovieLens/blob/main/r/src/capstone-movielens.main.R\#L1944}{capstone-movielens.main.R} script on \emph{GitHub}.

\end{noteblock}

Below the most significant part of the code performing the current model tuning, the \emph{Pre-configuration Step} is provided:

\begin{Shaded}
\begin{Highlighting}[]
\NormalTok{  lss.UMGYDE.preset.degree1.result }\OtherTok{\textless{}{-}} 
    \FunctionTok{tune.model\_param}\NormalTok{(spans, train\_UMGY\_SmoothedDay\_effect.RMSE.cv.degree1)}
\end{Highlighting}
\end{Shaded}

\begin{Shaded}
\begin{Highlighting}[]
\FunctionTok{str}\NormalTok{(lss.UMGYDE.preset.degree1.result)}
\end{Highlighting}
\end{Shaded}

\begin{verbatim}
## List of 2
##  $ tuned.result:'data.frame':    5 obs. of  2 variables:
##   ..$ RMSE           : num [1:5] 0.872 0.871 0.871 0.872 0.872
##   ..$ parameter.value: num [1:5] 0.0005 0.0015 0.0025 0.0035 0.0045
##  $ best_result : Named num [1:2] 0.0015 0.8711
##   ..- attr(*, "names")= chr [1:2] "param.best_value" "best_RMSE"
\end{verbatim}

Below is the visual representation of the \texttt{span} parameter values range we have figured out in this step of tuning:

\begin{Shaded}
\begin{Highlighting}[]
\NormalTok{lss.UMGYDE.preset.degree1.result}\SpecialCharTok{$}\NormalTok{tuned.result }\SpecialCharTok{|\textgreater{}}
  \FunctionTok{data.plot}\NormalTok{(}\AttributeTok{title =} \StringTok{"UMGYDE Model, \textasciigrave{}loess\textasciigrave{} function args: range of \textasciigrave{}span\textasciigrave{} values for \textasciigrave{}degree = 1\textasciigrave{}"}\NormalTok{,}
              \AttributeTok{xname =} \StringTok{"parameter.value"}\NormalTok{, }
              \AttributeTok{yname =} \StringTok{"RMSE"}\NormalTok{, }
              \AttributeTok{xlabel =} \StringTok{"spans"}\NormalTok{, }
              \AttributeTok{ylabel =} \StringTok{"RMSE"}\NormalTok{)}
\end{Highlighting}
\end{Shaded}

\pandocbounded{\includegraphics[keepaspectratio]{capstone-movielens.report_files/figure-latex/unnamed-chunk-298-1.pdf}}

\subparagraph{\texorpdfstring{UMGYDE Model Fine-tuning (\texttt{degree\ =\ 1})}{UMGYDE Model Fine-tuning (degree = 1)}}\label{umgyde-model-fine-tuning-degree-1}

\hfill\break

\begin{noteblock}
The complete source code of the solution described in this section is available in the \href{https://github.com/AzKurban-edX-DS/Capstone-MovieLens/blob/main/r/src/capstone-movielens.main.R\#L2}{UMGYDE Model Fine-tuning (\texttt{degree\ =\ 1})} section of the \href{https://github.com/AzKurban-edX-DS/Capstone-MovieLens/blob/main/r/src/capstone-movielens.main.R\#L2007}{capstone-movielens.main.R} script on \emph{GitHub}.

\end{noteblock}

Below the most significant part of the code performing the current model \emph{Fine-tuning} is provided:

\begin{Shaded}
\begin{Highlighting}[]
\NormalTok{lss.fine\_tune.loop\_starter }\OtherTok{\textless{}{-}} 
  \FunctionTok{c}\NormalTok{(lss.UMGYDE.preset.degree1.result}\SpecialCharTok{$}\NormalTok{tuned.result}\SpecialCharTok{$}\NormalTok{parameter.value[}\DecValTok{1}\NormalTok{], }
\NormalTok{    lss.UMGYDE.preset.degree1.result}\SpecialCharTok{$}\NormalTok{tuned.result}\SpecialCharTok{$}\NormalTok{parameter.value[}\DecValTok{3}\NormalTok{], }
    \DecValTok{8}\NormalTok{)}

\NormalTok{cache\_file.base\_name }\OtherTok{\textless{}{-}} \StringTok{"UMGYDE.degree1.tuning{-}span"}

\NormalTok{lss.UMGYDE.fine\_tune.degree1.result }\OtherTok{\textless{}{-}} 
  \FunctionTok{model.tune.param\_range}\NormalTok{(lss.fine\_tune.loop\_starter,}
\NormalTok{                         UMGYDE.fine\_tune.degree1.data.path,}
\NormalTok{                         cache\_file.base\_name,}
\NormalTok{                         train\_UMGY\_SmoothedDay\_effect.RMSE.cv.degree1)}
\end{Highlighting}
\end{Shaded}

\begin{Shaded}
\begin{Highlighting}[]
\NormalTok{lss.fine\_tune.loop\_starter}
\end{Highlighting}
\end{Shaded}

\begin{verbatim}
## [1] 0.0005 0.0025 8.0000
\end{verbatim}

\begin{Shaded}
\begin{Highlighting}[]
\FunctionTok{str}\NormalTok{(lss.UMGYDE.fine\_tune.degree1.result)}
\end{Highlighting}
\end{Shaded}

\begin{verbatim}
## List of 3
##  $ best_result           : Named num [1:2] 0.00109 0.87083
##   ..- attr(*, "names")= chr [1:2] "param.best_value" "best_RMSE"
##  $ param_values.endpoints: num [1:3] 1.06e-03 1.31e-03 3.13e-05
##  $ tuned.result          :'data.frame':  9 obs. of  2 variables:
##   ..$ parameter.value: num [1:9] 0.00106 0.00109 0.00113 0.00116 0.00119 ...
##   ..$ RMSE           : num [1:9] 0.871 0.871 0.871 0.871 0.871 ...
\end{verbatim}

Below is the visual representation of the current model \emph{Fine-tuning} results for the parameter value: \texttt{degree\ =\ 1}:

\begin{Shaded}
\begin{Highlighting}[]
\NormalTok{lss.UMGYDE.fine\_tune.degree1.result}\SpecialCharTok{$}\NormalTok{tuned.result }\SpecialCharTok{|\textgreater{}}
  \FunctionTok{data.plot}\NormalTok{(}\AttributeTok{title =} \StringTok{"Fine{-}tuned UMGYDE Model with \textasciigrave{}loess\textasciigrave{} parameter: \textasciigrave{}degree = 1\textasciigrave{}"}\NormalTok{, }
                             \AttributeTok{xname =} \StringTok{"parameter.value"}\NormalTok{, }
                             \AttributeTok{yname =} \StringTok{"RMSE"}\NormalTok{, }
                             \AttributeTok{xlabel =} \StringTok{"spans"}\NormalTok{, }
                             \AttributeTok{ylabel =} \StringTok{"RMSE"}\NormalTok{)}
\end{Highlighting}
\end{Shaded}

\pandocbounded{\includegraphics[keepaspectratio]{capstone-movielens.report_files/figure-latex/unnamed-chunk-301-1.pdf}}

Now, we can add the \emph{UMGYDE Model Tuning} result for \texttt{degree\ =\ 1} to our \emph{Result Table}:

\begin{Shaded}
\begin{Highlighting}[]
\NormalTok{RMSEs.ResultTibble.UMGYDE1 }\OtherTok{\textless{}{-}}\NormalTok{ RMSEs.ResultTibble.UMGYDE0 }\SpecialCharTok{|\textgreater{}} 
  \FunctionTok{RMSEs.AddRow}\NormalTok{(}\StringTok{"Tuned UMGYDE.d1 Model"}\NormalTok{, }
\NormalTok{               lss.UMGYDE.fine\_tune.degree1.result.best\_RMSE,}
               \AttributeTok{comment =} \StringTok{"UMGYDE Model computed using function call: \textasciigrave{}loess(degree = 1, span = \%1)\textasciigrave{}"} \SpecialCharTok{|\textgreater{}}
                 \FunctionTok{msg.glue}\NormalTok{(lss.UMGYDE.fine\_tune.degree1.result.best\_span))}
\end{Highlighting}
\end{Shaded}

\begin{Shaded}
\begin{Highlighting}[]
\FunctionTok{RMSE\_kable}\NormalTok{(RMSEs.ResultTibble.UMGYDE1)}
\end{Highlighting}
\end{Shaded}

\begin{tabular}{>{\raggedright\arraybackslash}p{15em}>{\centering\arraybackslash}p{5em}>{\raggedright\arraybackslash}p{30em}}
\toprule
\textbf{Method} & \textbf{RMSE} & \textbf{Comment}\\
\midrule
Project Objective & 0.8649000 & \\
Overall Mean Rating Model & 1.0603462 & \\
User Effect Model & 0.9697962 & \\
User+Movie Effect Model & 0.8732081 & \\
Regularized UME Model & 0.8729730 & Computed for `lambda` = 0.38745002746582\\
\addlinespace
UMGE Model & 0.8729730 & User+Movie+Genre Effect (UMGE) Model\\
Regularized UMGE Model & 0.8729728 & Computed for `lambda` = 0.0359375\\
UMGYE Model & 0.8723973 & User+Movie+Genre+Year Effect (UMGYE) Model\\
Regularized UMGYE Model & 0.8721857 & Computed for `lambda` = 233.77668762207\\
UMGYDE (Default) Model & 0.8722410 & User+Movie+Genre+Year+Day Effect (UMGYDE) Model 
computed using `loess` function with default `degree` \& `span` parameters.\\
\addlinespace
Tuned UMGYDE.d0 Model & 0.8711131 & UMGYDE Model computed using function call: `loess(degree = 0, span = 0.00109375)`\\
Tuned UMGYDE.d1 Model & 0.8708295 & UMGYDE Model computed using function call: `loess(degree = 1, span = 0.00109375)`\\
\bottomrule
\end{tabular}

\paragraph{\texorpdfstring{UMGYDE Model Tuning: Step 3 (\texttt{degree\ =\ 2})}{UMGYDE Model Tuning: Step 3 (degree = 2)}}\label{umgyde-model-tuning-step-3-degree-2}

\hfill\break

\subparagraph{\texorpdfstring{Pre-configuration: \texttt{span} Range Determination}{Pre-configuration: span Range Determination}}\label{pre-configuration-span-range-determination-2}

\hfill\break

\begin{noteblock}
The complete source code of the solution described in this section is available in the \href{https://github.com/AzKurban-edX-DS/Capstone-MovieLens/blob/main/r/src/capstone-movielens.main.R\#L2056}{UMGYDE Model Tuning: Pre-configuration (\texttt{degree\ =\ 2})} section of the \href{https://github.com/AzKurban-edX-DS/Capstone-MovieLens/blob/main/r/src/capstone-movielens.main.R\#L2057}{capstone-movielens.main.R} script on \emph{GitHub}.

\end{noteblock}

Below the most significant part of the code performing the current model tuning, the \emph{Pre-configuration Step} is provided:

\begin{Shaded}
\begin{Highlighting}[]
\NormalTok{  spans }\OtherTok{\textless{}{-}} \FunctionTok{seq}\NormalTok{(}\FloatTok{0.0005}\NormalTok{, }\DecValTok{1}\NormalTok{, }\FloatTok{0.001}\NormalTok{)}
\NormalTok{  lss.UMGYDE.preset.degree2.result }\OtherTok{\textless{}{-}} 
    \FunctionTok{tune.model\_param}\NormalTok{(spans, train\_UMGY\_SmoothedDay\_effect.RMSE.cv.degree2)}
\end{Highlighting}
\end{Shaded}

\begin{Shaded}
\begin{Highlighting}[]
\FunctionTok{str}\NormalTok{(lss.UMGYDE.preset.degree2.result)}
\end{Highlighting}
\end{Shaded}

\begin{verbatim}
## List of 2
##  $ tuned.result:'data.frame':    5 obs. of  2 variables:
##   ..$ RMSE           : num [1:5] 0.872 0.871 0.871 0.871 0.871
##   ..$ parameter.value: num [1:5] 0.0005 0.0015 0.0025 0.0035 0.0045
##  $ best_result : Named num [1:2] 0.0015 0.871
##   ..- attr(*, "names")= chr [1:2] "param.best_value" "best_RMSE"
\end{verbatim}

Below is the visual representation of the \texttt{span} parameter values range we have figured out in this step of tuning:

\begin{Shaded}
\begin{Highlighting}[]
\NormalTok{lss.UMGYDE.preset.degree2.result}\SpecialCharTok{$}\NormalTok{tuned.result }\SpecialCharTok{|\textgreater{}}
  \FunctionTok{data.plot}\NormalTok{(}\AttributeTok{title =} \StringTok{"UMGYDE Model, \textasciigrave{}loess\textasciigrave{} function args: range of \textasciigrave{}span\textasciigrave{} values for \textasciigrave{}degree = 2\textasciigrave{}"}\NormalTok{,}
              \AttributeTok{xname =} \StringTok{"parameter.value"}\NormalTok{, }
              \AttributeTok{yname =} \StringTok{"RMSE"}\NormalTok{, }
              \AttributeTok{xlabel =} \StringTok{"spans"}\NormalTok{, }
              \AttributeTok{ylabel =} \StringTok{"RMSE"}\NormalTok{)}
\end{Highlighting}
\end{Shaded}

\pandocbounded{\includegraphics[keepaspectratio]{capstone-movielens.report_files/figure-latex/unnamed-chunk-306-1.pdf}}

\subparagraph{\texorpdfstring{UMGYDE Model Fine-tuning (\texttt{degree\ =\ 2})}{UMGYDE Model Fine-tuning (degree = 2)}}\label{umgyde-model-fine-tuning-degree-2}

\hfill\break

\begin{noteblock}
The complete source code of the solution described in this section is available in the \href{https://github.com/AzKurban-edX-DS/Capstone-MovieLens/blob/main/r/src/capstone-movielens.main.R\#L2119}{UMGYDE Model Fine-tuning (\texttt{degree\ =\ 2})} section of the \href{https://github.com/AzKurban-edX-DS/Capstone-MovieLens/blob/main/r/src/capstone-movielens.main.R\#L2119}{capstone-movielens.main.R} script on \emph{GitHub}.

\end{noteblock}

Below the most significant part of the code performing the current model \emph{Fine-tuning} is provided:

\begin{Shaded}
\begin{Highlighting}[]
\NormalTok{lss.fine\_tune.loop\_starter }\OtherTok{\textless{}{-}} 
  \FunctionTok{c}\NormalTok{(lss.UMGYDE.preset.degree2.result}\SpecialCharTok{$}\NormalTok{tuned.result}\SpecialCharTok{$}\NormalTok{parameter.value[}\DecValTok{1}\NormalTok{], }
\NormalTok{    lss.UMGYDE.preset.degree2.result}\SpecialCharTok{$}\NormalTok{tuned.result}\SpecialCharTok{$}\NormalTok{parameter.value[}\DecValTok{3}\NormalTok{], }
    \DecValTok{8}\NormalTok{)}

\NormalTok{cache\_file.base\_name }\OtherTok{\textless{}{-}} \StringTok{"UMGYDE.degree2.tuning{-}span"}

\NormalTok{lss.UMGYDE.fine\_tune.degree2.result }\OtherTok{\textless{}{-}} 
  \FunctionTok{model.tune.param\_range}\NormalTok{(lss.fine\_tune.loop\_starter,}
\NormalTok{                         UMGYDE.fine\_tune.degree2.data.path,}
\NormalTok{                         cache\_file.base\_name,}
\NormalTok{                         train\_UMGY\_SmoothedDay\_effect.RMSE.cv.degree2)}
\end{Highlighting}
\end{Shaded}

\begin{Shaded}
\begin{Highlighting}[]
\NormalTok{lss.fine\_tune.loop\_starter}
\end{Highlighting}
\end{Shaded}

\begin{verbatim}
## [1] 0.0005 0.0025 8.0000
\end{verbatim}

\begin{Shaded}
\begin{Highlighting}[]
\FunctionTok{str}\NormalTok{(lss.UMGYDE.fine\_tune.degree2.result)}
\end{Highlighting}
\end{Shaded}

\begin{verbatim}
## List of 3
##  $ best_result           : Named num [1:2] 0.00153 0.87099
##   ..- attr(*, "names")= chr [1:2] "param.best_value" "best_RMSE"
##  $ param_values.endpoints: num [1:3] 1.50e-03 1.75e-03 3.13e-05
##  $ tuned.result          :'data.frame':  9 obs. of  2 variables:
##   ..$ parameter.value: num [1:9] 0.0015 0.00153 0.00156 0.00159 0.00163 ...
##   ..$ RMSE           : num [1:9] 0.871 0.871 0.871 0.871 0.871 ...
\end{verbatim}

Below is the visual representation of the current model \emph{Fine-tuning} results for the parameter value: \texttt{degree\ =\ 2}:

\begin{Shaded}
\begin{Highlighting}[]
\NormalTok{lss.UMGYDE.fine\_tune.degree2.result}\SpecialCharTok{$}\NormalTok{tuned.result }\SpecialCharTok{|\textgreater{}}
  \FunctionTok{data.plot.left\_detailed}\NormalTok{(}\AttributeTok{title =} \StringTok{"Fine{-}tuned UMGYDE Model with \textasciigrave{}loess\textasciigrave{} parameter: \textasciigrave{}degree = 2\textasciigrave{}"}\NormalTok{, }
                             \AttributeTok{title.left =} \StringTok{"Left Part of the Chart Above (Zoomed in)"}\NormalTok{,}
                             \AttributeTok{left.n =} \DecValTok{8}\NormalTok{,}
                             \AttributeTok{xname =} \StringTok{"parameter.value"}\NormalTok{, }
                             \AttributeTok{yname =} \StringTok{"RMSE"}\NormalTok{, }
                             \AttributeTok{xlabel1 =} \StringTok{"spans"}\NormalTok{, }
                             \AttributeTok{ylabel1 =} \StringTok{"RMSE"}\NormalTok{)}
\end{Highlighting}
\end{Shaded}

\pandocbounded{\includegraphics[keepaspectratio]{capstone-movielens.report_files/figure-latex/unnamed-chunk-309-1.pdf}}

\begin{noteblock}
The custom data visualization function \hyperref[func.data.plot.left_detailed]{data.plot.left\_detailed} we used above is described in the \hyperref[appndx_a.data_helper.functions]{Data Helper Functions} section of the \hyperref[appndx_a]{Appendix} to this report.

\end{noteblock}

Now, we can add the \emph{UMGYDE Model Tuning} result for \texttt{degree\ =\ 2} to our \emph{Result Table}:

\begin{Shaded}
\begin{Highlighting}[]
\NormalTok{RMSEs.ResultTibble.UMGYDE2 }\OtherTok{\textless{}{-}}\NormalTok{ RMSEs.ResultTibble.UMGYDE1 }\SpecialCharTok{|\textgreater{}} 
  \FunctionTok{RMSEs.AddRow}\NormalTok{(}\StringTok{"Tuned UMGYDE.d2 Model"}\NormalTok{, }
\NormalTok{               lss.UMGYDE.fine\_tune.degree2.result.best\_RMSE,}
               \AttributeTok{comment =} \StringTok{"UMGYDE Model computed using function call: \textasciigrave{}loess(degree = 2, span = \%1)\textasciigrave{}"} \SpecialCharTok{|\textgreater{}}
                 \FunctionTok{msg.glue}\NormalTok{(lss.UMGYDE.fine\_tune.degree2.result.best\_span))}
\end{Highlighting}
\end{Shaded}

\begin{Shaded}
\begin{Highlighting}[]
\FunctionTok{RMSE\_kable}\NormalTok{(RMSEs.ResultTibble.UMGYDE2)}
\end{Highlighting}
\end{Shaded}

\begin{tabular}{>{\raggedright\arraybackslash}p{15em}>{\centering\arraybackslash}p{5em}>{\raggedright\arraybackslash}p{30em}}
\toprule
\textbf{Method} & \textbf{RMSE} & \textbf{Comment}\\
\midrule
Project Objective & 0.8649000 & \\
Overall Mean Rating Model & 1.0603462 & \\
User Effect Model & 0.9697962 & \\
User+Movie Effect Model & 0.8732081 & \\
Regularized UME Model & 0.8729730 & Computed for `lambda` = 0.38745002746582\\
\addlinespace
UMGE Model & 0.8729730 & User+Movie+Genre Effect (UMGE) Model\\
Regularized UMGE Model & 0.8729728 & Computed for `lambda` = 0.0359375\\
UMGYE Model & 0.8723973 & User+Movie+Genre+Year Effect (UMGYE) Model\\
Regularized UMGYE Model & 0.8721857 & Computed for `lambda` = 233.77668762207\\
UMGYDE (Default) Model & 0.8722410 & User+Movie+Genre+Year+Day Effect (UMGYDE) Model 
computed using `loess` function with default `degree` \& `span` parameters.\\
\addlinespace
Tuned UMGYDE.d0 Model & 0.8711131 & UMGYDE Model computed using function call: `loess(degree = 0, span = 0.00109375)`\\
Tuned UMGYDE.d1 Model & 0.8708295 & UMGYDE Model computed using function call: `loess(degree = 1, span = 0.00109375)`\\
Tuned UMGYDE.d2 Model & 0.8709886 & UMGYDE Model computed using function call: `loess(degree = 2, span = 0.00153125)`\\
\bottomrule
\end{tabular}

\newpage

\paragraph{\texorpdfstring{UMGYDE Model Tuning: Re-training with the Best \texttt{degree} \& \texttt{span} Parameters}{UMGYDE Model Tuning: Re-training with the Best degree \& span Parameters}}\label{umgyde-model-tuning-re-training-with-the-best-degree-span-parameters}

\hfill\break

\begin{noteblock}
The complete version of the source code provided in this section are available in the \href{https://github.com/AzKurban-edX-DS/Capstone-MovieLens/blob/main/r/src/capstone-movielens.main.R\#2169}{UMGYDE Tuned Model: Retraining with the best params} section of the \href{https://github.com/AzKurban-edX-DS/Capstone-MovieLens/blob/main/r/src/capstone-movielens.main.R\#L2169}{capstone-movielens.main.R} script on \emph{GitHub}.

\end{noteblock}

Now, we can calculate the \emph{Regularized UMGYDE Effect} by retraining our model on the entire \texttt{edx} dataset with the best values of the \texttt{degree} and \texttt{span} parameters we figured out above, for the definitive \emph{Root Mean Squared Error} calculation and use in subsequent processing.

The most significant part of the code that performs this operation is shown below:

\begin{Shaded}
\begin{Highlighting}[]
\NormalTok{lss.best\_results }\OtherTok{\textless{}{-}} \FunctionTok{data.frame}\NormalTok{(}\AttributeTok{degree =}\NormalTok{ degree, }
                         \AttributeTok{span =} \FunctionTok{c}\NormalTok{(lss.UMGYDE.fine\_tune.degree0.result.best\_span,}
\NormalTok{                                  lss.UMGYDE.fine\_tune.degree1.result.best\_span,}
\NormalTok{                                  lss.UMGYDE.fine\_tune.degree2.result.best\_span),}
                         
                         \AttributeTok{RMSE =} \FunctionTok{c}\NormalTok{(lss.UMGYDE.fine\_tune.degree0.result.best\_RMSE, }
\NormalTok{                                  lss.UMGYDE.fine\_tune.degree1.result.best\_RMSE,}
\NormalTok{                                  lss.UMGYDE.fine\_tune.degree2.result.best\_RMSE))}

\NormalTok{lss.best\_RMSE.idx }\OtherTok{\textless{}{-}} \FunctionTok{which.min}\NormalTok{(lss.best\_results}\SpecialCharTok{$}\NormalTok{RMSE)}


\NormalTok{lss.UMGYDE.best\_params }\OtherTok{\textless{}{-}} 
  \FunctionTok{c}\NormalTok{(}\AttributeTok{degree =}\NormalTok{  lss.best\_results[lss.best\_RMSE.idx, }\StringTok{"degree"}\NormalTok{],  }\CommentTok{\# 1}
    \AttributeTok{span =}\NormalTok{ lss.best\_results[lss.best\_RMSE.idx, }\StringTok{"span"}\NormalTok{], }\CommentTok{\# 0.00087,}
    \AttributeTok{RMSE =}\NormalTok{ lss.best\_results[lss.best\_RMSE.idx, }\StringTok{"RMSE"}\NormalTok{]) }\CommentTok{\# 0.8568619}

\NormalTok{lss.best\_degree }\OtherTok{\textless{}{-}}\NormalTok{ lss.UMGYDE.best\_params[}\StringTok{"degree"}\NormalTok{]}
\NormalTok{lss.best\_span }\OtherTok{\textless{}{-}}\NormalTok{ lss.UMGYDE.best\_params[}\StringTok{"span"}\NormalTok{]}
\NormalTok{lss.best\_RMSE }\OtherTok{\textless{}{-}}\NormalTok{ lss.UMGYDE.best\_params[}\StringTok{"RMSE"}\NormalTok{]}

\FunctionTok{put\_log2}\NormalTok{(}\StringTok{"Re{-}training model using \textasciigrave{}loess\textasciigrave{} function with the best parameters: }
\StringTok{span = \%1, degree = \%2"}\NormalTok{, lss.best\_span, lss.best\_degree)}

\NormalTok{lss.UMGYD\_effect }\OtherTok{\textless{}{-}}\NormalTok{ edx }\SpecialCharTok{|\textgreater{}} 
  \FunctionTok{train\_UMGY\_SmoothedDay\_effect}\NormalTok{(lss.best\_degree, lss.best\_span)}
\end{Highlighting}
\end{Shaded}

\begin{Shaded}
\begin{Highlighting}[]
\NormalTok{lss.UMGYDE.best\_params}
\end{Highlighting}
\end{Shaded}

\begin{verbatim}
##     degree       span       RMSE 
## 1.00000000 0.00109375 0.87082947
\end{verbatim}

\begin{Shaded}
\begin{Highlighting}[]
\FunctionTok{str}\NormalTok{(lss.UMGYD\_effect)}
\end{Highlighting}
\end{Shaded}

\begin{verbatim}
## tibble [4,640 x 4] (S3: tbl_df/tbl/data.frame)
##  $ days       : int [1:4640] 0 385 388 389 392 393 394 396 397 399 ...
##  $ de         : num [1:4640] 0.2106 0.2506 0.1395 0.2797 -0.0392 ...
##  $ year       : num [1:4640] 1995 1996 1996 1996 1996 ...
##  $ de_smoothed: num [1:4640] 0.228 0.242 0.205 0.183 0.44 ...
\end{verbatim}

\begin{noteblock}
Here we use the helper function \hyperref[func.train_UMGY_SmoothedDay_effect]{train\_UMGY\_SmoothedDay\_effect} described above in the \hyperref[umgyde-model-helper-functions]{UMGYDE Model: Helper Functions} section.

\end{noteblock}

Let's now visualize final Smoothed Day effect we have obtained:

\begin{Shaded}
\begin{Highlighting}[]
\NormalTok{lss.UMGYD\_effect }\SpecialCharTok{|\textgreater{}}
  \FunctionTok{ggplot}\NormalTok{(}\FunctionTok{aes}\NormalTok{(}\AttributeTok{x =}\NormalTok{ days)) }\SpecialCharTok{+}
  \FunctionTok{geom\_point}\NormalTok{(}\FunctionTok{aes}\NormalTok{(}\AttributeTok{y =}\NormalTok{ de), }\AttributeTok{size =} \DecValTok{3}\NormalTok{, }\AttributeTok{alpha =}\NormalTok{ .}\DecValTok{5}\NormalTok{, }\AttributeTok{color =} \StringTok{"grey"}\NormalTok{) }\SpecialCharTok{+} 
  \FunctionTok{geom\_line}\NormalTok{(}\FunctionTok{aes}\NormalTok{(}\AttributeTok{y =}\NormalTok{ de\_smoothed), }\AttributeTok{color =} \StringTok{"red"}\NormalTok{)}
\end{Highlighting}
\end{Shaded}

\pandocbounded{\includegraphics[keepaspectratio]{capstone-movielens.report_files/figure-latex/unnamed-chunk-314-1.pdf}}

\begin{noteblock}
It should be noted, however, that the graph above is not as smooth as before, although the data now provides better results.

\end{noteblock}

We calculate the \emph{Root Mean Squared Error} for the ultimately computed \emph{UMGYDE Effect} using the helper function \hyperref[func.calc_UMGY_SmoothedDay_effect.RMSE.cv]{calc\_UMGY\_SmoothedDay\_effect.RMSE.cv} described above in the \hyperref[umgyde-model-helper-functions]{UMGYDE Model: Helper Functions} section above as follows:

\begin{Shaded}
\begin{Highlighting}[]
\NormalTok{lss.UMGYD\_effect.RMSE }\OtherTok{\textless{}{-}} \FunctionTok{calc\_UMGY\_SmoothedDay\_effect.RMSE.cv}\NormalTok{(lss.UMGYD\_effect)}
\end{Highlighting}
\end{Shaded}

Finally, we add the definitive result for the current model to our \emph{Result Table}:

\begin{Shaded}
\begin{Highlighting}[]
\NormalTok{RMSEs.ResultTibble.UMGYDE.tuned }\OtherTok{\textless{}{-}}\NormalTok{ RMSEs.ResultTibble.UMGYDE2 }\SpecialCharTok{|\textgreater{}} 
  \FunctionTok{RMSEs.AddRow}\NormalTok{(}\StringTok{"Tuned UMGYDE Best Model"}\NormalTok{, }
\NormalTok{               lss.UMGYD\_effect.RMSE,}
               \AttributeTok{comment =} \StringTok{"UMGYDE Model computed using \textasciigrave{}loess\textasciigrave{} function call with the best degree \& span values."}\NormalTok{)}
\end{Highlighting}
\end{Shaded}

\begin{Shaded}
\begin{Highlighting}[]
\FunctionTok{RMSE\_kable}\NormalTok{(RMSEs.ResultTibble.UMGYDE.tuned)}
\end{Highlighting}
\end{Shaded}

\begin{tabular}{>{\raggedright\arraybackslash}p{15em}>{\centering\arraybackslash}p{5em}>{\raggedright\arraybackslash}p{30em}}
\toprule
\textbf{Method} & \textbf{RMSE} & \textbf{Comment}\\
\midrule
Project Objective & 0.8649000 & \\
Overall Mean Rating Model & 1.0603462 & \\
User Effect Model & 0.9697962 & \\
User+Movie Effect Model & 0.8732081 & \\
Regularized UME Model & 0.8729730 & Computed for `lambda` = 0.38745002746582\\
\addlinespace
UMGE Model & 0.8729730 & User+Movie+Genre Effect (UMGE) Model\\
Regularized UMGE Model & 0.8729728 & Computed for `lambda` = 0.0359375\\
UMGYE Model & 0.8723973 & User+Movie+Genre+Year Effect (UMGYE) Model\\
Regularized UMGYE Model & 0.8721857 & Computed for `lambda` = 233.77668762207\\
UMGYDE (Default) Model & 0.8722410 & User+Movie+Genre+Year+Day Effect (UMGYDE) Model 
computed using `loess` function with default `degree` \& `span` parameters.\\
\addlinespace
Tuned UMGYDE.d0 Model & 0.8711131 & UMGYDE Model computed using function call: `loess(degree = 0, span = 0.00109375)`\\
Tuned UMGYDE.d1 Model & 0.8708295 & UMGYDE Model computed using function call: `loess(degree = 1, span = 0.00109375)`\\
Tuned UMGYDE.d2 Model & 0.8709886 & UMGYDE Model computed using function call: `loess(degree = 2, span = 0.00153125)`\\
Tuned UMGYDE Best Model & 0.8707850 & UMGYDE Model computed using `loess` function call with the best degree \& span values.\\
\bottomrule
\end{tabular}

Finally, let's regularize our tuned model as we did with the previous ones.

\newpage

\subsubsection{UMGYDE Model Regularization}\label{umgyde-model-regularization}

\hfill\break

\paragraph{UMGYDE Model Regularization: Mathematical Description}\label{umgyde-model-regularization-mathematical-description}

\hfill\break

We have already explained the idea of \emph{Linear Model Regularization} in the \hyperref[ume-model-regularization]{UME Model Regularization} section above. We have also seen how the formula \eqref{eq:ME-penalty} for adding a penalty to the \emph{UME Model} is transformed into the formula \eqref{eq:GE-penalty} for the \emph{UMGE Model} and then into the formula \eqref{eq:GYE-penalty} for the \emph{UMGYDE Model}. For the current model, this formula takes the form:

\begin{equation}
\sum_{i,j} \left(y_{u,i} - \mu - \alpha_i - \beta_j - g_{i,j} - \gamma(v_{i,j}) - s(d_{i,j})\right)^2 + \lambda \sum_{i,j} s(d_{i,j})^2
\label{eq:DE-penalty}
\end{equation}

And the formula \eqref{eq:ME-regularized} for calculating the values of the \emph{treatment effect} that minimizes the equation will take the form:

\begin{equation}
\hat{s}(d_{i,j}, \lambda) = \frac{1}{\lambda + n_d} \sum_{r=1}^{n_{d}} \left(Y_{i,j} - \mu - \alpha_i - \beta_j - g_{i,j} - \gamma(v_{i,j})\right)
\label{eq:DE-regularized}
\end{equation}

where \(n_d\) is the number of ratings made on the day \(d\).

As with the previous models, we implement the \emph{Regularization} method for the current model in the following three steps:

\begin{enumerate}
\def\labelenumi{\arabic{enumi}.}
\item
  \textbf{Pre-configuration:} Preliminary determination of the optimal range of \(\lambda\) values for the \texttt{5-Fold\ Cross\ Validation} samples;
\item
  \textbf{Fine-tuning:} figuring out the value of \(\lambda\) that minimizes the model's RMSE.
\item
  \textbf{Retraining:} retraining the model with the best value of the parameter \(\lambda\) obtained in the previous step.
\end{enumerate}

\newpage

\paragraph{UMGYDE Model Regularization: Helper Functions}\label{umgyde-model-regularization-helper-functions}

\hfill\break

\subparagraph{\texorpdfstring{\href{https://github.com/AzKurban-edX-DS/Capstone-MovieLens/blob/main/r/src/support-functions/UMGYD-effect.functions.R\#L223}{regularize.train\_UMGYD\_effect} Function}{regularize.train\_UMGYD\_effect Function}}\label{func.regularize.train_UMGYD_effect}

\hfill\break

The function is an overloaded version of the \hyperref[func.train_UMGY_SmoothedDay_effect]{train\_UMGY\_SmoothedDay\_effect} function described above, with the fixed values of the \hyperref[func.train_UMGY_SmoothedDay_effect.params.degree]{degree} and \hyperref[func.train_UMGY_SmoothedDay_effect.params.span]{span} parameters set to the best values figured out during the tuning process described in the {[}UMGYDE Model Tuning{]} section above.

Parameters

\hfill\break

\phantomsection\label{func.regularize.train_UMGYD_effect.params.train_set}
\begin{itemize}
\tightlist
\item
  \textbf{\emph{train\_set:}}
  The \emph{Train Set} dataset .
\end{itemize}

\phantomsection\label{func.regularize.train_UMGYD_effect.params.lambda}
\begin{itemize}
\tightlist
\item
  \textbf{\emph{lambda:}}
  In accordance with the description in the \hyperref[umgyde-model-regularization-mathematical-description]{UMGYDE Model Regularization: Mathematical Description} section above, this is the \emph{Regularization parameter} \(\lambda\) used for the \emph{UMGYDE Model Regularization} process.
\end{itemize}

Source Code

\hfill\break

The following is the source code of the function:

\begin{Shaded}
\begin{Highlighting}[]
\NormalTok{regularize.train\_UMGYD\_effect }\OtherTok{\textless{}{-}} \ControlFlowTok{function}\NormalTok{(train\_set, lambda) \{}
\NormalTok{  best\_degree }\OtherTok{\textless{}{-}}\NormalTok{ lss.UMGYDE.best\_params[}\StringTok{"degree"}\NormalTok{]}
\NormalTok{  best\_span }\OtherTok{\textless{}{-}}\NormalTok{ lss.UMGYDE.best\_params[}\StringTok{"span"}\NormalTok{]}
  
\NormalTok{  train\_set }\SpecialCharTok{|\textgreater{}}
    \FunctionTok{train\_UMGY\_SmoothedDay\_effect}\NormalTok{(best\_degree, }
\NormalTok{                                  best\_span, }
\NormalTok{                                  lambda)}
\NormalTok{\}}
\end{Highlighting}
\end{Shaded}

\begin{noteblock}
The source code of the \href{https://github.com/AzKurban-edX-DS/Capstone-MovieLens/blob/main/r/src/support-functions/UMGYD-effect.functions.R\#L223}{regularize.train\_UMGYD\_effect} function is also available in the \href{https://github.com/AzKurban-edX-DS/Capstone-MovieLens/blob/main/r/src/support-functions/UMGYD-effect.functions.R\#L222}{Regularization} section of the \href{https://github.com/AzKurban-edX-DS/Capstone-MovieLens/blob/main/r/src/support-functions/UMGYD-effect.functions.R\#L223}{UMGYD-effect.functions.R} script on \emph{GitHub}.

\end{noteblock}

Return

\hfill\break
A data frame object representing the \emph{UMGYD Effect}.

\newpage

\subparagraph{\texorpdfstring{\href{https://github.com/AzKurban-edX-DS/Capstone-MovieLens/blob/main/r/src/support-functions/UMGYD-effect.functions.R\#L232}{regularize.train\_UMGYD\_effect.cv} Function}{regularize.train\_UMGYD\_effect.cv Function}}\label{func.regularize.train_UMGYD_effect.cv}

\hfill\break

\begin{noteblock}
The source code of the \href{https://github.com/AzKurban-edX-DS/Capstone-MovieLens/blob/main/r/src/support-functions/UMGYD-effect.functions.R\#L232}{regularize.train\_UMGYD\_effect.cv} function is defined in the \href{https://github.com/AzKurban-edX-DS/Capstone-MovieLens/blob/main/r/src/support-functions/UMGYD-effect.functions.R\#L222}{Regularization} section of the \href{https://github.com/AzKurban-edX-DS/Capstone-MovieLens/blob/main/r/src/support-functions/UMGYD-effect.functions.R\#L232}{UMGYD-effect.functions.R} script on \emph{GitHub}.

\end{noteblock}

The function is an overloaded version of the \hyperref[func.train_UMGY_SmoothedDay_effect.cv]{train\_UMGY\_SmoothedDay\_effect.cv} function described above, with the fixed values of the \hyperref[func.train_UMGY_SmoothedDay_effect.cv.params.degree]{degree} and \hyperref[func.train_UMGY_SmoothedDay_effect.cv.params.span]{span} parameters set to the best values figured out during the tuning process described in the {[}UMGYDE Model Tuning{]} section above.

Parameters

\hfill\break

\phantomsection\label{func.regularize.train_UMGYD_effect.cv.params.lambda}
\begin{itemize}
\tightlist
\item
  \textbf{\emph{lambda:}}
  In accordance with the description in the \hyperref[umgyde-model-regularization-mathematical-description]{UMGYDE Model Regularization: Mathematical Description} section above, this is the \emph{Regularization parameter} \(\lambda\) used for the \emph{UMGYDE Model Regularization} process.
\end{itemize}

Source Code

\hfill\break
The source code of the function is provided below:

\begin{Shaded}
\begin{Highlighting}[]
\NormalTok{regularize.train\_UMGYD\_effect.cv }\OtherTok{\textless{}{-}} \ControlFlowTok{function}\NormalTok{(lambda) \{}
\NormalTok{  best\_degree }\OtherTok{\textless{}{-}}\NormalTok{ lss.UMGYDE.best\_params[}\StringTok{"degree"}\NormalTok{]}
\NormalTok{  best\_span }\OtherTok{\textless{}{-}}\NormalTok{ lss.UMGYDE.best\_params[}\StringTok{"span"}\NormalTok{]}

  \FunctionTok{train\_UMGY\_SmoothedDay\_effect.cv}\NormalTok{(best\_degree, }
\NormalTok{                                   best\_span, }
\NormalTok{                                   lambda)}
\NormalTok{\}}
\end{Highlighting}
\end{Shaded}

\begin{noteblock}
Note that we reuse the function \hyperref[func.train_UMGY_SmoothedDay_effect.cv]{train\_UMGY\_SmoothedDay\_effect.cv} calling it internally from the \hyperref[func.regularize.train_UMGYD_effect.cv]{regularize.train\_UMGYD\_effect.cv}, but now with the \(\lambda\) parameter different from the default (`lambda = 0') value.

\end{noteblock}

Return

\hfill\break
A data frame object representing the \emph{UMGYD Effect}.

\newpage

\subparagraph{\texorpdfstring{\href{https://github.com/AzKurban-edX-DS/Capstone-MovieLens/blob/main/r/src/support-functions/UMGYD-effect.functions.R\#L240}{regularize.test\_lambda.UMGYD\_effect.cv} Function}{regularize.test\_lambda.UMGYD\_effect.cv Function}}\label{func.regularize.test_lambda.UMGYD_effect.cv}

\hfill\break

The function calculates \emph{RMSE} of the \emph{UMGYDE Model} using \emph{5-Fold Cross Validation} for the given \(\lambda\) parameter value.

\phantomsection\label{func.regularize.test_lambda.UMGYD_effect.cv.params.lambda}
\begin{itemize}
\tightlist
\item
  \textbf{\emph{lambda:}}
  In accordance with the description in the \hyperref[umgyde-model-regularization-mathematical-description]{UMGYDE Model Regularization: Mathematical Description} section above, this is the \emph{Regularization parameter} \(\lambda\) used for the \emph{UMGYDE Model Regularization} process.
\end{itemize}

\phantomsection\label{func.regularize.test_lambda.UMGYD_effect.cv.details}
Details

\hfill\break
Under the hood, the function internally calls the \hyperref[func.regularize.train_UMGYD_effect.cv]{regularize.train\_UMGYD\_effect.cv} and then \hyperref[func.calc_UMGY_SmoothedDay_effect.RMSE.cv]{calc\_UMGY\_SmoothedDay\_effect.RMSE.cv} described above, and returns the \emph{RMSE} value that the latter inner function calculates for the given \hyperref[func.regularize.test_lambda.UMGYD_effect.cv.params.lambda]{lambda} parameter value passed to the former one.

Source Code

\hfill\break
The source code of the function is provided below:

\begin{Shaded}
\begin{Highlighting}[]
\NormalTok{regularize.test\_lambda.UMGYD\_effect.cv }\OtherTok{\textless{}{-}} \ControlFlowTok{function}\NormalTok{(lambda)\{}
  \ControlFlowTok{if}\NormalTok{ (}\FunctionTok{is.na}\NormalTok{(lambda)) \{}
    \FunctionTok{stop}\NormalTok{(}\StringTok{"Function: regularize.test\_lambda.UMGY\_effect.cv}
\StringTok{\textasciigrave{}lambda\textasciigrave{} is \textasciigrave{}NA\textasciigrave{}"}\NormalTok{)}
\NormalTok{  \}}
  
  \FunctionTok{regularize.train\_UMGYD\_effect.cv}\NormalTok{(lambda) }\SpecialCharTok{|\textgreater{}}
    \FunctionTok{calc\_UMGY\_SmoothedDay\_effect.RMSE.cv}\NormalTok{()}
\NormalTok{\}}
\end{Highlighting}
\end{Shaded}

\begin{noteblock}
The source code of the \href{https://github.com/AzKurban-edX-DS/Capstone-MovieLens/blob/main/r/src/support-functions/UMGYD-effect.functions.R\#L240}{regularize.test\_lambda.UMGYD\_effect.cv} function is also available in the \href{https://github.com/AzKurban-edX-DS/Capstone-MovieLens/blob/main/r/src/support-functions/UMGYD-effect.functions.R\#L222}{Regularization} section of the \href{https://github.com/AzKurban-edX-DS/Capstone-MovieLens/blob/main/r/src/support-functions/UMGYD-effect.functions.R\#L240}{UMGYD-effect.functions.R} script on \emph{GitHub}.

\end{noteblock}

Return

\hfill\break
\emph{Root Mean Squared Error (RMSE)} of the \emph{UMGYD Model} trained for the given \hyperref[func.regularize.train_UMGYD_effect.cv.params.span]{lambda} parameter value.

\newpage

\paragraph{UMGYDE Model Regularization: Pre-configuration}\label{umgyde-model-regularization-pre-configuration}

\hfill\break

\begin{noteblock}
The complete version of the source code shown in this section is available in the \href{https://github.com/AzKurban-edX-DS/Capstone-MovieLens/blob/main/r/src/capstone-movielens.main.R\#L2322}{UMGYDE Model Regularization: Pre-configuration} section of the \href{https://github.com/AzKurban-edX-DS/Capstone-MovieLens/blob/main/r/src/capstone-movielens.main.R\#L2322}{capstone-movielens.main.R} script on \emph{GitHub}.

\end{noteblock}

Let's perform the preconfiguration to determine the appropriate range of \(\lambda\) for subsequent fine-tuning of our current model:

We are going to use the \hyperref[func.tune.model_param]{tune.model\_param} function described in the \hyperref[appndx_a.rglr.common_helper.functions]{Regularization: Common Helper Functions} section of the \hyperref[appndx_a]{Appendix} to this report, passing the \hyperref[func.regularize.test_lambda.UMGYD_effect.cv]{regularize.test\_lambda.UMGYD\_effect.cv} function as the value of the \hyperref[func.tune.model_param.params.fn_tune.test.param_value]{fn\_tune.test.param\_value} parameter.

Below we provide the most significant part of the code that performs this operation:

\begin{Shaded}
\begin{Highlighting}[]
\NormalTok{  lambdas }\OtherTok{\textless{}{-}} \FunctionTok{seq}\NormalTok{(}\DecValTok{0}\NormalTok{, }\DecValTok{256}\NormalTok{, }\DecValTok{16}\NormalTok{)}
\NormalTok{  cv.UMGYDE.preset.result }\OtherTok{\textless{}{-}} 
    \FunctionTok{tune.model\_param}\NormalTok{(lambdas, regularize.test\_lambda.UMGYD\_effect.cv)}
\end{Highlighting}
\end{Shaded}

\begin{Shaded}
\begin{Highlighting}[]
\FunctionTok{str}\NormalTok{(cv.UMGYDE.preset.result)}
\end{Highlighting}
\end{Shaded}

\begin{verbatim}
## List of 2
##  $ tuned.result:'data.frame':    5 obs. of  2 variables:
##   ..$ RMSE           : num [1:5] 0.871 0.871 0.871 0.871 0.871
##   ..$ parameter.value: num [1:5] 0 16 32 48 64
##  $ best_result : Named num [1:2] 16 0.871
##   ..- attr(*, "names")= chr [1:2] "param.best_value" "best_RMSE"
\end{verbatim}

\begin{Shaded}
\begin{Highlighting}[]
\NormalTok{cv.UMGYDE.preset.result}\SpecialCharTok{$}\NormalTok{best\_result}
\end{Highlighting}
\end{Shaded}

\begin{verbatim}
## param.best_value        best_RMSE 
##       16.0000000        0.8708197
\end{verbatim}

Now, let's visualize the results of the \(\lambda\) range preconfiguration:

\begin{Shaded}
\begin{Highlighting}[]
\NormalTok{cv.UMGYDE.preset.result}\SpecialCharTok{$}\NormalTok{tuned.result }\SpecialCharTok{|\textgreater{}}
  \FunctionTok{data.plot}\NormalTok{(}\AttributeTok{title =} \FunctionTok{TeX}\NormalTok{(r}\StringTok{\textquotesingle{}[UMGYDE Model Regularization: $\textbackslash{}lambda$ Range Pre{-}configuration]\textquotesingle{}}\NormalTok{),}
              \AttributeTok{xname =} \StringTok{"parameter.value"}\NormalTok{, }
              \AttributeTok{yname =} \StringTok{"RMSE"}\NormalTok{, }
              \AttributeTok{xlabel =} \FunctionTok{TeX}\NormalTok{(r}\StringTok{\textquotesingle{}[$\textbackslash{}lambda$]\textquotesingle{}}\NormalTok{), }
              \AttributeTok{ylabel =} \StringTok{"RMSE"}\NormalTok{)}
\end{Highlighting}
\end{Shaded}

\pandocbounded{\includegraphics[keepaspectratio]{capstone-movielens.report_files/figure-latex/unnamed-chunk-323-1.pdf}}

\begin{noteblock}
We use the custom data visualization function \hyperref[func.data.plot]{data.plot} described in the \hyperref[appndx_a.data_helper.functions]{Data Helper Functions} section of the \hyperref[appndx_a]{Appendix} to this report, which is defined in the \href{https://github.com/AzKurban-edX-DS/Capstone-MovieLens/blob/main/r/src/support-functions/data.helper.functions.R\#L447}{Data Visualization} section of the \href{https://github.com/AzKurban-edX-DS/Capstone-MovieLens/blob/main/r/src/support-functions/data.helper.functions.R\#L592}{data.helper.functions.R} script on \emph{GitHub}.

\end{noteblock}

\newpage

\paragraph{UMGYDE Model Regularization: Fine-tuning}\label{umgyde-model-regularization-fine-tuning}

\hfill\break

\begin{noteblock}
The complete version of the source code shown in this section can be found in the \href{https://github.com/AzKurban-edX-DS/Capstone-MovieLens/blob/main/r/src/capstone-movielens.main.R\#L2378}{UMGYDE Model Regularization: Fine-tuning} section of the \href{https://github.com/AzKurban-edX-DS/Capstone-MovieLens/blob/main/r/src/capstone-movielens.main.R\#L2378}{capstone-movielens.main.R} script on \emph{GitHub}.

\end{noteblock}

We are now ready to perform the fine-tuning step of our model \emph{regularization} process to determine the best value for the \(\lambda\) parameter.

Here we are going to use the \hyperref[func.model.tune.param_range]{model.tune.param\_range} function described in the \hyperref[appndx_a.rglr.common_helper.functions]{Regularization: Common Helper Functions} section of the \hyperref[appndx_a]{Appendix} to this report, passing the \hyperref[func.regularize.test_lambda.UMGYD_effect.cv]{regularize.test\_lambda.UMGYD\_effect.cv} function as the value of the \hyperref[func.model.tune.param_range.params.fn_tune.test.param_value]{fn\_tune.test.param\_value} parameter.

Below we provide the most significant part of the code that performs this operation:

\begin{Shaded}
\begin{Highlighting}[]
\NormalTok{endpoints }\OtherTok{\textless{}{-}} 
  \FunctionTok{get\_fine\_tune.param.endpoints}\NormalTok{(cv.UMGYDE.preset.result}\SpecialCharTok{$}\NormalTok{tuned.result)}

\NormalTok{UMGYDE.loop\_starter }\OtherTok{\textless{}{-}} \FunctionTok{c}\NormalTok{(endpoints[}\StringTok{"start"}\NormalTok{], }
\NormalTok{                        endpoints[}\StringTok{"end"}\NormalTok{], }
                        \DecValTok{8}\NormalTok{)}

\NormalTok{cache.base\_name }\OtherTok{\textless{}{-}} \StringTok{"UMGYDE.rglr.fine{-}tuning"}

\NormalTok{UMGYDE.rglr.fine\_tune.results }\OtherTok{\textless{}{-}} 
  \FunctionTok{model.tune.param\_range}\NormalTok{(UMGYDE.loop\_starter,}
\NormalTok{                         UMGYDE.rglr.fine\_tune.cache.path,}
\NormalTok{                         cache.base\_name,}
\NormalTok{                         regularize.test\_lambda.UMGYD\_effect.cv)}
\end{Highlighting}
\end{Shaded}

\begin{Shaded}
\begin{Highlighting}[]
\FunctionTok{str}\NormalTok{(UMGYDE.rglr.fine\_tune.results)}
\end{Highlighting}
\end{Shaded}

\begin{verbatim}
## List of 3
##  $ best_result           : Named num [1:2] 10.714 0.871
##   ..- attr(*, "names")= chr [1:2] "param.best_value" "best_RMSE"
##  $ param_values.endpoints: Named num [1:3] 1.07e+01 1.07e+01 1.53e-05
##   ..- attr(*, "names")= chr [1:3] "" "" ""
##  $ tuned.result          :'data.frame':  9 obs. of  2 variables:
##   ..$ parameter.value: num [1:9] 10.7 10.7 10.7 10.7 10.7 ...
##   ..$ RMSE           : num [1:9] 0.871 0.871 0.871 0.871 0.871 ...
\end{verbatim}

\begin{Shaded}
\begin{Highlighting}[]
\NormalTok{UMGYDE.rglr.fine\_tune.results}\SpecialCharTok{$}\NormalTok{best\_result}
\end{Highlighting}
\end{Shaded}

\begin{verbatim}
## param.best_value        best_RMSE 
##       10.7142639        0.8708192
\end{verbatim}

\begin{noteblock}
Under the hood, this code block internally calls the helper function \hyperref[func.get_fine_tune.param.endpoints]{get\_fine\_tune.param.endpoints} described in the \hyperref[appndx_a.rglr.common_helper.functions]{Regularization: Common Helper Functions} section of the \hyperref[appndx_a]{Appendix} to this report to figure out the boundaries of the range of values from the nearest neighborhood of the value corresponding to the best \texttt{RMSE} in the data passed in as a parameter.

\end{noteblock}

\newpage

Let's visualize the fine-tuning results:

\begin{Shaded}
\begin{Highlighting}[]
\NormalTok{UMGYDE.rglr.fine\_tune.results}\SpecialCharTok{$}\NormalTok{tuned.result }\SpecialCharTok{|\textgreater{}}
  \FunctionTok{data.plot}\NormalTok{(}\AttributeTok{title =} \StringTok{"UMGYDE Model Regularization: Fine{-}tuned result"}\NormalTok{,}
              \AttributeTok{xname =} \StringTok{"parameter.value"}\NormalTok{,}
              \AttributeTok{yname =} \StringTok{"RMSE"}\NormalTok{,}
              \AttributeTok{xlabel =} \FunctionTok{TeX}\NormalTok{(r}\StringTok{\textquotesingle{}[$\textbackslash{}lambda$]\textquotesingle{}}\NormalTok{),}
              \AttributeTok{ylabel =} \FunctionTok{str\_glue}\NormalTok{(}\StringTok{"Deviation from the best RMSE value ("}\NormalTok{,}
                                \FunctionTok{as.character}\NormalTok{(}\FunctionTok{round}\NormalTok{(UMGYDE.rglr.fine\_tune.RMSE.best, }\AttributeTok{digits =} \DecValTok{7}\NormalTok{)),}
                                \StringTok{")"}\NormalTok{),}
              \AttributeTok{normalize =} \ConstantTok{TRUE}\NormalTok{)}
\end{Highlighting}
\end{Shaded}

\pandocbounded{\includegraphics[keepaspectratio]{capstone-movielens.report_files/figure-latex/unnamed-chunk-326-1.pdf}}

\begin{noteblock}
We use the custom data visualization function \hyperref[func.data.plot]{data.plot} described in the \hyperref[appndx_a.data_helper.functions]{Data Helper Functions} section of the \hyperref[appndx_a]{Appendix} to this report, which is defined in the \href{https://github.com/AzKurban-edX-DS/Capstone-MovieLens/blob/main/r/src/support-functions/data.helper.functions.R\#L447}{Data Visualization} section of the \href{https://github.com/AzKurban-edX-DS/Capstone-MovieLens/blob/main/r/src/support-functions/data.helper.functions.R\#L592}{data.helper.functions.R} script on \emph{GitHub}.

\end{noteblock}

\newpage

\paragraph{\texorpdfstring{UMGYDE Model Regularization: Retraining Model with the best \(\lambda\)}{UMGYDE Model Regularization: Retraining Model with the best \textbackslash lambda}}\label{umgyde-model-regularization-retraining-model-with-the-best-lambda}

\hfill\break

\begin{noteblock}
The complete version of the source code shown in this section is available in the \href{https://github.com/AzKurban-edX-DS/Capstone-MovieLens/blob/main/r/src/capstone-movielens.main.R\#L2417}{UMGYDE Regularized Model: Retraining with the best params} section of the \href{https://github.com/AzKurban-edX-DS/Capstone-MovieLens/blob/main/r/src/capstone-movielens.main.R\#L2417}{capstone-movielens.main.R} script on \emph{GitHub}.

\end{noteblock}

Now, we can compute the \emph{Regularized UMGYDE Effect} by re-training our model on the entire \texttt{edx} dataset with the best value of the \(\lambda\) parameter we have just figured out and use it in subsequent analysis.

The most significant part of the code for this operation is shown below:

\begin{Shaded}
\begin{Highlighting}[]
\NormalTok{  best\_result }\OtherTok{\textless{}{-}}\NormalTok{ UMGYDE.rglr.fine\_tune.results}\SpecialCharTok{$}\NormalTok{best\_result}
\NormalTok{  UMGYDE.rglr.best\_lambda }\OtherTok{\textless{}{-}}\NormalTok{ best\_result[}\StringTok{"param.best\_value"}\NormalTok{]}
\NormalTok{  UMGYDE.rglr.best\_RMSE }\OtherTok{\textless{}{-}}\NormalTok{ best\_result[}\StringTok{"best\_RMSE"}\NormalTok{]}
  
  \FunctionTok{put\_log1}\NormalTok{(}\StringTok{"Re{-}training Regularized User+Movie+Genre+Year+(Smoothed)Day Effect Model for the best \textasciigrave{}lambda\textasciigrave{}: \%1..."}\NormalTok{,}
\NormalTok{           UMGYDE.rglr.best\_lambda)}
  
\NormalTok{  rglr.UMGYD\_effect }\OtherTok{\textless{}{-}}\NormalTok{ edx }\SpecialCharTok{|\textgreater{}} 
    \FunctionTok{regularize.train\_UMGYD\_effect}\NormalTok{(UMGYDE.rglr.best\_lambda)}
\end{Highlighting}
\end{Shaded}

\begin{Shaded}
\begin{Highlighting}[]
  \FunctionTok{str}\NormalTok{(rglr.UMGYD\_effect)}
\end{Highlighting}
\end{Shaded}

\begin{verbatim}
## tibble [4,640 x 4] (S3: tbl_df/tbl/data.frame)
##  $ days       : int [1:4640] 0 385 388 389 392 393 394 396 397 399 ...
##  $ de         : Named num [1:4640] 0.0331 0.1537 0.1174 0.239 -0.0277 ...
##   ..- attr(*, "names")= chr [1:4640] "param.best_value" "param.best_value" "param.best_value" "param.best_value" ...
##  $ year       : num [1:4640] 1995 1996 1996 1996 1996 ...
##  $ de_smoothed: num [1:4640] 0.025 0.1492 0.1638 0.1565 0.0627 ...
\end{verbatim}

\begin{noteblock}
Here we use the helper function \hyperref[func.regularize.train_UMGYD_effect]{regularize.train\_UMGYD\_effect} described above in the \hyperref[umgyde-model-helper-functions]{UMGYDE Model: Helper Functions} section.

\end{noteblock}

We calculate the \emph{Root Mean Squared Error} for the ultimately computed \emph{Regularized UMGYDE Effect} using the helper function \hyperref[func.calc_UMGY_SmoothedDay_effect.RMSE.cv]{calc\_UMGY\_SmoothedDay\_effect.RMSE.cv} described above in the \hyperref[umgyde-model-helper-functions]{UMGYDE Model: Helper Functions} section as follows:

\begin{Shaded}
\begin{Highlighting}[]
\NormalTok{  rglr.UMGYD\_effect.RMSE }\OtherTok{\textless{}{-}} \FunctionTok{calc\_UMGY\_SmoothedDay\_effect.RMSE.cv}\NormalTok{(rglr.UMGYD\_effect)}
\end{Highlighting}
\end{Shaded}

As always, we add the definitive result for the current model to our \emph{Result Table}:

\begin{Shaded}
\begin{Highlighting}[]
\NormalTok{RMSEs.ResultTibble.UMGYDE.rglr.tuned }\OtherTok{\textless{}{-}}\NormalTok{ RMSEs.ResultTibble.UMGYDE.tuned }\SpecialCharTok{|\textgreater{}} 
  \FunctionTok{RMSEs.AddRow}\NormalTok{(}\StringTok{"Regularized UMGYDE Model"}\NormalTok{, }
\NormalTok{               rglr.UMGYD\_effect.RMSE,}
               \AttributeTok{comment =} \StringTok{"The best tuned and regularized UMGYDE Model."}\NormalTok{)}
\end{Highlighting}
\end{Shaded}

\begin{Shaded}
\begin{Highlighting}[]
\FunctionTok{RMSE\_kable}\NormalTok{(RMSEs.ResultTibble.UMGYDE.rglr.tuned)}
\end{Highlighting}
\end{Shaded}

\begin{tabular}{>{\raggedright\arraybackslash}p{15em}>{\centering\arraybackslash}p{5em}>{\raggedright\arraybackslash}p{30em}}
\toprule
\textbf{Method} & \textbf{RMSE} & \textbf{Comment}\\
\midrule
Project Objective & 0.8649000 & \\
Overall Mean Rating Model & 1.0603462 & \\
User Effect Model & 0.9697962 & \\
User+Movie Effect Model & 0.8732081 & \\
Regularized UME Model & 0.8729730 & Computed for `lambda` = 0.38745002746582\\
\addlinespace
UMGE Model & 0.8729730 & User+Movie+Genre Effect (UMGE) Model\\
Regularized UMGE Model & 0.8729728 & Computed for `lambda` = 0.0359375\\
UMGYE Model & 0.8723973 & User+Movie+Genre+Year Effect (UMGYE) Model\\
Regularized UMGYE Model & 0.8721857 & Computed for `lambda` = 233.77668762207\\
UMGYDE (Default) Model & 0.8722410 & User+Movie+Genre+Year+Day Effect (UMGYDE) Model 
computed using `loess` function with default `degree` \& `span` parameters.\\
\addlinespace
Tuned UMGYDE.d0 Model & 0.8711131 & UMGYDE Model computed using function call: `loess(degree = 0, span = 0.00109375)`\\
Tuned UMGYDE.d1 Model & 0.8708295 & UMGYDE Model computed using function call: `loess(degree = 1, span = 0.00109375)`\\
Tuned UMGYDE.d2 Model & 0.8709886 & UMGYDE Model computed using function call: `loess(degree = 2, span = 0.00153125)`\\
Tuned UMGYDE Best Model & 0.8707850 & UMGYDE Model computed using `loess` function call with the best degree \& span values.\\
Regularized UMGYDE Model & 0.8707750 & The best tuned and regularized UMGYDE Model.\\
\bottomrule
\end{tabular}

\begin{noteblock}
Now, we have a better result than the ones for the previous models, but still insufficient to meet the Project Objective. Let's see what else we can do to achieve our ultimate goal.

\end{noteblock}

\newpage

\paragraph{UMGYDE Model: Final Holdout Test (Preliminary Assessment)}\label{umgyde-model-final-holdout-test-preliminary-assessment}

\hfill\break

\begin{noteblock}
The complete version of the source code shown in this section is available in the \href{https://github.com/AzKurban-edX-DS/Capstone-MovieLens/blob/main/r/src/capstone-movielens.main.R\#L2498}{UMGYDE Model: Final Holdout Test} section of the \href{https://github.com/AzKurban-edX-DS/Capstone-MovieLens/blob/main/r/src/capstone-movielens.main.R\#L2498}{capstone-movielens.main.R} script on \emph{GitHub}.

\end{noteblock}

Let's see what we have achieved so far, using the \texttt{final\_holdout\_test} dataset for the first time.
Firstly, we compute predictions for our current model:

\begin{Shaded}
\begin{Highlighting}[]
\NormalTok{final.UMGYDE.predicted }\OtherTok{\textless{}{-}}\NormalTok{ final\_holdout\_test }\SpecialCharTok{|\textgreater{}}
  \FunctionTok{UMGY\_SmoothedDay\_effect.predict}\NormalTok{(rglr.UMGYD\_effect)}
\end{Highlighting}
\end{Shaded}

\begin{Shaded}
\begin{Highlighting}[]
\FunctionTok{str}\NormalTok{(final.UMGYDE.predicted)}
\end{Highlighting}
\end{Shaded}

\begin{verbatim}
## 'data.frame':    999999 obs. of  5 variables:
##  $ userId   : int  1 1 1 2 2 2 3 3 4 4 ...
##  $ movieId  : int  231 480 586 151 858 1544 590 4995 34 432 ...
##  $ timestamp: int  838983392 838983653 838984068 868246450 868245645 868245920 1136075494 1133571200 844416936 844417070 ...
##  $ rating   : num  5 5 5 3 2 3 3.5 4.5 5 3 ...
##  $ predicted: num  4.49 5 4.63 3.3 4.07 ...
\end{verbatim}

\begin{noteblock}
Here we use the helper function \hyperref[func.UMGY_SmoothedDay_effect.predict]{UMGY\_SmoothedDay\_effect.predict} described above in the \hyperref[umgyde-model-helper-functions]{UMGYDE Model: Helper Functions} section.

\end{noteblock}

Then we compute the \emph{RMSE} and add the result to our \emph{Result Table}:

\begin{Shaded}
\begin{Highlighting}[]
\NormalTok{final.UMGYDE.predicted.RMSE }\OtherTok{\textless{}{-}} \FunctionTok{rmse2}\NormalTok{(final\_holdout\_test}\SpecialCharTok{$}\NormalTok{rating,}
\NormalTok{                                     final.UMGYDE.predicted}\SpecialCharTok{$}\NormalTok{predicted)}

\NormalTok{final.RMSEs.ResultTibble.UMGYDE.rglr.tuned }\OtherTok{\textless{}{-}}\NormalTok{ RMSEs.ResultTibble.UMGYDE.rglr.tuned }\SpecialCharTok{|\textgreater{}} 
  \FunctionTok{RMSEs.AddRow}\NormalTok{(}\StringTok{"Best UMGYDE Model (Final Test)"}\NormalTok{, }
\NormalTok{               final.UMGYDE.predicted.RMSE,}
               \AttributeTok{comment =} \StringTok{"Final Holdout Test of the best tuned and regularized UMGYDE Model."}\NormalTok{)}
\end{Highlighting}
\end{Shaded}

\begin{noteblock}
Here we use the helper function \href{https://github.com/AzKurban-edX-DS/Capstone-MovieLens/blob/main/r/src/support-functions/common-helper.functions.R\#L28}{rmse2} to calculate the \emph{RMSE} that is defined in the \href{https://github.com/AzKurban-edX-DS/Capstone-MovieLens/blob/main/r/src/support-functions/common-helper.functions.R\#L18}{(R)MSE-related functions} section of the \href{https://github.com/AzKurban-edX-DS/Capstone-MovieLens/blob/main/r/src/support-functions/common-helper.functions.R\#L28}{common-helper.functions.R} script on \emph{GitHub}.

\end{noteblock}

And finally, we print out the results:

\begin{Shaded}
\begin{Highlighting}[]
\FunctionTok{RMSE\_kable}\NormalTok{(final.RMSEs.ResultTibble.UMGYDE.rglr.tuned)}
\end{Highlighting}
\end{Shaded}

\begin{tabular}{>{\raggedright\arraybackslash}p{15em}>{\centering\arraybackslash}p{5em}>{\raggedright\arraybackslash}p{30em}}
\toprule
\textbf{Method} & \textbf{RMSE} & \textbf{Comment}\\
\midrule
Project Objective & 0.8649000 & \\
Overall Mean Rating Model & 1.0603462 & \\
User Effect Model & 0.9697962 & \\
User+Movie Effect Model & 0.8732081 & \\
Regularized UME Model & 0.8729730 & Computed for `lambda` = 0.38745002746582\\
\addlinespace
UMGE Model & 0.8729730 & User+Movie+Genre Effect (UMGE) Model\\
Regularized UMGE Model & 0.8729728 & Computed for `lambda` = 0.0359375\\
UMGYE Model & 0.8723973 & User+Movie+Genre+Year Effect (UMGYE) Model\\
Regularized UMGYE Model & 0.8721857 & Computed for `lambda` = 233.77668762207\\
UMGYDE (Default) Model & 0.8722410 & User+Movie+Genre+Year+Day Effect (UMGYDE) Model 
computed using `loess` function with default `degree` \& `span` parameters.\\
\addlinespace
Tuned UMGYDE.d0 Model & 0.8711131 & UMGYDE Model computed using function call: `loess(degree = 0, span = 0.00109375)`\\
Tuned UMGYDE.d1 Model & 0.8708295 & UMGYDE Model computed using function call: `loess(degree = 1, span = 0.00109375)`\\
Tuned UMGYDE.d2 Model & 0.8709886 & UMGYDE Model computed using function call: `loess(degree = 2, span = 0.00153125)`\\
Tuned UMGYDE Best Model & 0.8707850 & UMGYDE Model computed using `loess` function call with the best degree \& span values.\\
Regularized UMGYDE Model & 0.8707750 & The best tuned and regularized UMGYDE Model.\\
\addlinespace
Best UMGYDE Model (Final Test) & 0.8804225 & Final Holdout Test of the best tuned and regularized UMGYDE Model.\\
\bottomrule
\end{tabular}

\begin{noteblock}
As expected, we have not yet achieved a result that meets our Project Objective.

\end{noteblock}

As explained in the \href{https://rafalab.dfci.harvard.edu/dsbook-part-2/highdim/matrix-factorization.html}{Chapter 24 \texttt{Matrix\ Factorization}} of the \emph{Course Textbook}, so far our models \emph{``ignore an important source of information related to the fact that groups of movies have similar rating patterns and groups of users have similar rating patterns as well\ldots{}''} \autocite{IDS2_24}

Then the author shows that in this case, the \emph{Matrix Factorization} method can considerably improve our results. In the next section, we will apply this method to our current model and see what we get.

\newpage

\subsection{UMGYDE Model: Matrix Factorization (MF)}\label{umgyde-model-matrix-factorization-mf}

\hfill\break

For our next solution, we will use the \href{https://cran.r-project.org/web/packages/recosystem/index.html}{\texttt{recosystem} package} to perform the \emph{Parallel Matrix Factorization}.

\subsubsection{MF: Matematical Description}\label{mf-matematical-description}

\hfill\break

As outlined in \href{https://cran.r-project.org/web/packages/recosystem/vignettes/introduction.html}{this article}\autocite{yqiu_recosystem230505}, the idea of the \emph{Matrix Factorization} method is to approximate the whole rating matrix \(R_{m \times n}\) by the product of two matrices of lower dimensions, \(P_{n \times k}\) and \(Q_{n\times k}\), such that

\begin{equation}
\mathbb{R} \approx \mathbb{PQ^T}
\label{eq:R-PQ-MF}
\end{equation}

In relation to our model, the expression \eqref{eq:R-PQ-MF} will take the form:

\begin{equation}
\mathbf{R} \thicksim \mathbf{\hat{R}} + \mathbf{PQ^T} +  \varepsilon
\label{eq:UMGYDE-MF}
\end{equation}

where:

\begin{itemize}
\tightlist
\item
  \(\mathbf{R}\) is the \(U_\mathrm{m} \times M_n\) rating matrix with \(U_m\) users and \(M_n\) movies;
\item
  \(\mathbf{\hat{R}}\) represents the predictions from our best model: \emph{Regularized UMGYDE Model};
\item
  \(\mathbf{P}\) and \(\mathbf{Q}\) are \(P_{m \times k}\) and \(Q_{n \times k}\) matrices, respectively, where \(k\) is the number of \emph{latent features} to be found.
\end{itemize}

If we denote the \(u\)-th row of \(\mathbf{P}\) as \(\mathbf{p}_u\) and the \(v\)-th row of \(\mathbf{Q}\) as \(\mathbf{q}_v\), then the unknown rating \(\mathbf{r}_{u,v}\) given by user \(u\) on movie item \(v\) for our model can be estimated as \(\hat{\mathbf{r}}_{u,v} + \mathbf{p}_{u}\mathbf{q}^T_{v}\), where the \(\hat{\mathbf{r}}_{u,v}\) is the prediction given by our last \emph{Regularized UMGYDE Model}.

A typical solution for \(\mathbf{P}\) and \(\mathbf{Q}\) is given by the following optimization problem \autocite{chin_FPSGM_MF_SMS2015a,chin_LRS_SGM_MF2015b}:

\[
\begin{array}{c} { min} \\
{\scriptstyle P,Q} \end{array} \sum_{(u,v)\in R}[f(\mathbf{p}_u, \mathbf{q}_v; \mathbf{r}_{u,v}) + \mu_P\parallel\mathbf{p}_u\parallel_1 + \mu_Q\parallel\mathbf{q}_v\parallel_1 + \frac{\lambda_P}2\parallel\mathbf{p}_u\parallel^2_2 + \frac{\lambda_Q}2\parallel\mathbf{q}_v\parallel^2_2]
\]

where \((u, v)\) are locations of observed entries in \(\mathbf{R}, \mathbf{r}_{u,v}\) is the observed ratings, \(f\) is the loss function, and \(\mu_P\), \(\mu_Q\), \(\lambda_P\), \(\lambda_Q\) are penalty parameters to avoid overfitting.\autocite{yqiu_recosystem230505}

\begin{quote}
\emph{The process of solving the matrices \(P\) and \(Q\) is referred to as model training, and the selection of penalty parameters is called parameter tuning. In \texttt{recosystem}, we provide convenient functions for these two tasks, and additionally have functions for model exporting (outputting \(P\) and \(Q\) matrices) and prediction.}\autocite{yqiu_recosystem230505}
\end{quote}

\newpage

\subsubsection{MF: Helper Functions}\label{mf-helper-functions}

\paragraph{\texorpdfstring{\href{https://github.com/AzKurban-edX-DS/Capstone-MovieLens/blob/main/r/src/support-functions/MF.functions.R\#L1}{mf.residual.dataframe} Function}{mf.residual.dataframe Function}}\label{func.mf.residual.dataframe}

\hfill\break

\phantomsection\label{func.mf.residual.dataframe.params.description}
\subparagraph{Description}\label{description}

\hfill\break

Computes residuals, defined as the difference between the observed and predicted values of the fully trained \emph{UMGYDE Model} as described in the \hyperref[usermoviegenreyearday-effect-umgyde-model]{User+Movie+Genre+Year+Day Effect (UMGYDE) Model} section.

\subparagraph{Parameters}\label{parameters-22}

\hfill\break

\textbf{\emph{train\_set}}

\hfill\break

\phantomsection\label{func.mf.residual.dataframe.params.train_set}
The dataset to be trained.

\subparagraph{Source Code}\label{source-code-22}

\hfill\break

The folowing is the source code of the function:

\begin{Shaded}
\begin{Highlighting}[]
\NormalTok{mf.residual.dataframe }\OtherTok{\textless{}{-}} \ControlFlowTok{function}\NormalTok{(train\_set)\{}
\NormalTok{  train\_set }\SpecialCharTok{|\textgreater{}} 
    \FunctionTok{left\_join}\NormalTok{(edx.user\_effect, }\AttributeTok{by =} \StringTok{"userId"}\NormalTok{) }\SpecialCharTok{|\textgreater{}}
    \FunctionTok{left\_join}\NormalTok{(rglr.UM\_effect, }\AttributeTok{by =} \StringTok{"movieId"}\NormalTok{) }\SpecialCharTok{|\textgreater{}}
    \FunctionTok{left\_join}\NormalTok{(rglr.UMG\_effect, }\AttributeTok{by =} \StringTok{"movieId"}\NormalTok{) }\SpecialCharTok{|\textgreater{}}
    \FunctionTok{left\_join}\NormalTok{(date\_days\_map, }\AttributeTok{by =} \StringTok{"timestamp"}\NormalTok{) }\SpecialCharTok{|\textgreater{}}
    \FunctionTok{left\_join}\NormalTok{(rglr.UMGY\_effect, }\AttributeTok{by=}\StringTok{\textquotesingle{}year\textquotesingle{}}\NormalTok{) }\SpecialCharTok{|\textgreater{}}
    \FunctionTok{left\_join}\NormalTok{(rglr.UMGYD\_effect, }\AttributeTok{by=}\StringTok{\textquotesingle{}days\textquotesingle{}}\NormalTok{) }\SpecialCharTok{|\textgreater{}}
    \FunctionTok{mutate}\NormalTok{(}\AttributeTok{rsdl =}\NormalTok{ rating }\SpecialCharTok{{-}}\NormalTok{ (mu }\SpecialCharTok{+}\NormalTok{ a }\SpecialCharTok{+}\NormalTok{ b }\SpecialCharTok{+}\NormalTok{ g }\SpecialCharTok{+}\NormalTok{ ye }\SpecialCharTok{+}\NormalTok{ de\_smoothed)) }\SpecialCharTok{|\textgreater{}}
    \FunctionTok{select}\NormalTok{(userId, movieId, rsdl)}
\NormalTok{\}}
\end{Highlighting}
\end{Shaded}

\begin{noteblock}
The source code of the \href{https://github.com/AzKurban-edX-DS/Capstone-MovieLens/blob/main/r/src/support-functions/MF.functions.R\#L1}{mf.residual.dataframe} function is also available in the \href{https://github.com/AzKurban-edX-DS/Capstone-MovieLens/blob/main/r/src/support-functions/MF.functions.R\#L1}{MF.functions.R} script on \emph{GitHub}.

\end{noteblock}

\subparagraph{Return}\label{return-23}

\hfill\break

A data frame object containing the residuals (see the \hyperref[func.mf.residual.dataframe.params.description]{Description} above for the details).

\newpage

\paragraph{\texorpdfstring{\href{https://github.com/AzKurban-edX-DS/Capstone-MovieLens/blob/main/r/src/support-functions/MF.functionsR\#L12}{UMGYDE\_model.predict} Function}{UMGYDE\_model.predict Function}}\label{func.UMGYDE_model.predict}

\hfill\break

This is a wrapper for the \hyperref[func.UMGY_SmoothedDay_effect.predict]{UMGY\_SmoothedDay\_effect.predict} function described in the \hyperref[umgyde-model-helper-functions]{UMGYDE Model: Helper Functions} section above with the fixed \hyperref[func.UMGY_SmoothedDay_effect.predict.params.day_smoothed_effect]{day\_smoothed\_effect} parameter value obtained after the \emph{UMGYDE Model} (described in the \hyperref[usermoviegenreyearday-effect-umgyde-model]{User+Movie+Genre+Year+Day Effect (UMGYDE) Model} section) has been fully trained.

\subparagraph{Parameters}\label{parameters-23}

\hfill\break

\phantomsection\label{func.UMGYDE_model.predict.params.test_set}
\textbf{\emph{test\_set}}

\hfill\break

The dataset for testing the \emph{UMGYDE Model} after it has been fully trained.

\subparagraph{Source Code}\label{source-code-23}

\hfill\break

The following is the source code of the function:

\begin{Shaded}
\begin{Highlighting}[]
\NormalTok{UMGYDE\_model.predict }\OtherTok{\textless{}{-}} \ControlFlowTok{function}\NormalTok{(test\_set) \{}
\NormalTok{  test\_set }\SpecialCharTok{|\textgreater{}}
    \FunctionTok{UMGY\_SmoothedDay\_effect.predict}\NormalTok{(rglr.UMGYD\_effect)}
\NormalTok{\}}
\end{Highlighting}
\end{Shaded}

\begin{noteblock}
The source code of the \href{https://github.com/AzKurban-edX-DS/Capstone-MovieLens/blob/main/r/src/support-functions/MF.functionsR\#L12}{UMGYDE\_model.predict} function is also available in the \href{https://github.com/AzKurban-edX-DS/Capstone-MovieLens/blob/main/r/src/support-functions/MF.functionsR\#L12}{MF.functions.R} script on \emph{GitHub}.

\end{noteblock}

\subparagraph{Return}\label{return-24}

\hfill\break

A data frame object containing predicted values.

\newpage

\subsubsection{MF: Performing Operation}\label{mf-performing-operation}

\begin{noteblock}
The complete source code shown in this section is available in the \href{https://github.com/AzKurban-edX-DS/Capstone-MovieLens/blob/main/r/src/capstone-movielens.main.R\#L2567}{Perform the Matrix Factorization \& Final Test} section of the \href{https://github.com/AzKurban-edX-DS/Capstone-MovieLens/blob/main/r/src/capstone-movielens.main.R\#L2567}{capstone-movielens.main.R} script on \emph{GitHub}.

\end{noteblock}

We assume that the \emph{Matrix Factorization} method is going to be the last one to provide sufficient results, so we will use the entire \texttt{edx} dataset as a \emph{Training Set} and \texttt{final\_holdout\_test} as a \emph{Test Set} for our final model, skipping the intermediate \emph{train/test} operations on the split \emph{train/validation} datasets.

\paragraph{MF: Getting Residuals From the UMGYDE Model Prediction Values}\label{mf-getting-residuals-from-the-umgyde-model-prediction-values}

\hfill\break

First, we get residuals from the \emph{UMGYDE Model} prediction values using the \hyperref[func.mf.residual.dataframe]{mf.residual.dataframe} function described above:

\begin{Shaded}
\begin{Highlighting}[]
\NormalTok{  mf.edx.residual }\OtherTok{\textless{}{-}} \FunctionTok{mf.residual.dataframe}\NormalTok{(edx)}
\end{Highlighting}
\end{Shaded}

\begin{Shaded}
\begin{Highlighting}[]
  \FunctionTok{str}\NormalTok{(mf.edx.residual)}
\end{Highlighting}
\end{Shaded}

\begin{verbatim}
## 'data.frame':    9000055 obs. of  3 variables:
##  $ userId : int  1 1 1 1 1 1 1 1 1 1 ...
##  $ movieId: int  122 185 292 316 329 355 356 362 364 370 ...
##  $ rsdl   : Named num  0.4868 0.3223 0.0386 0.1137 0.1269 ...
##   ..- attr(*, "names")= chr [1:9000055] "param.best_value" "param.best_value" "param.best_value" "param.best_value" ...
\end{verbatim}

\paragraph{\texorpdfstring{MF: Transforming Input Data to Be Compatible With the \texttt{recosystem} Package}{MF: Transforming Input Data to Be Compatible With the recosystem Package}}\label{mf-transforming-input-data-to-be-compatible-with-the-recosystem-package}

\hfill\break

Now, let's convert the residuals data and \texttt{final\_holdout\_test} dataset to the input data compatible with the \href{https://cran.r-project.org/web/packages/recosystem/index.html}{recosystem} package to process:

\begin{Shaded}
\begin{Highlighting}[]
  \FunctionTok{set.seed}\NormalTok{(}\DecValTok{5430}\NormalTok{)}
\NormalTok{  mf.edx.residual.reco }\OtherTok{\textless{}{-}} \FunctionTok{with}\NormalTok{(mf.edx.residual, }
                               \FunctionTok{data\_memory}\NormalTok{(}\AttributeTok{user\_index =}\NormalTok{ userId, }
                                           \AttributeTok{item\_index =}\NormalTok{ movieId,}
                                           \AttributeTok{rating =}\NormalTok{ rsdl))}

\NormalTok{  final\_holdout\_test.reco }\OtherTok{\textless{}{-}} \FunctionTok{with}\NormalTok{(final\_holdout\_test, }
                          \FunctionTok{data\_memory}\NormalTok{(}\AttributeTok{user\_index =}\NormalTok{ userId, }
                                      \AttributeTok{item\_index =}\NormalTok{ movieId, }
                                      \AttributeTok{rating =}\NormalTok{ rating))}
\end{Highlighting}
\end{Shaded}

\begin{Shaded}
\begin{Highlighting}[]
\FunctionTok{str}\NormalTok{(mf.edx.residual.reco)}
\end{Highlighting}
\end{Shaded}

\begin{verbatim}
## Formal class 'DataSource' [package "recosystem"] with 3 slots
##   ..@ source:List of 3
##   .. ..$ : int [1:9000055] 1 1 1 1 1 1 1 1 1 1 ...
##   .. ..$ : int [1:9000055] 122 185 292 316 329 355 356 362 364 370 ...
##   .. ..$ : num [1:9000055] 0.4868 0.3223 0.0386 0.1137 0.1269 ...
##   ..@ index1: logi FALSE
##   ..@ type  : chr "memory"
\end{verbatim}

\begin{Shaded}
\begin{Highlighting}[]
\FunctionTok{str}\NormalTok{(final\_holdout\_test.reco)}
\end{Highlighting}
\end{Shaded}

\begin{verbatim}
## Formal class 'DataSource' [package "recosystem"] with 3 slots
##   ..@ source:List of 3
##   .. ..$ : int [1:999999] 1 1 1 2 2 2 3 3 4 4 ...
##   .. ..$ : int [1:999999] 231 480 586 151 858 1544 590 4995 34 432 ...
##   .. ..$ : num [1:999999] 5 5 5 3 2 3 3.5 4.5 5 3 ...
##   ..@ index1: logi FALSE
##   ..@ type  : chr "memory"
\end{verbatim}

\paragraph{\texorpdfstring{MF: Creating and Tuning the \texttt{Reco} Object}{MF: Creating and Tuning the Reco Object}}\label{mf-creating-and-tuning-the-reco-object}

\hfill\break

We then create a \texttt{reco} object of the \texttt{Reco} class and tune it using the \texttt{reco\$tune} method, passing to it as a \texttt{train\_data} parameter residual data prepared in the previous step:

\begin{Shaded}
\begin{Highlighting}[]
\NormalTok{  reco }\OtherTok{\textless{}{-}} \FunctionTok{Reco}\NormalTok{()}
  
\NormalTok{  reco.tuned }\OtherTok{\textless{}{-}}\NormalTok{ reco}\SpecialCharTok{$}\FunctionTok{tune}\NormalTok{(mf.edx.residual.reco, }\AttributeTok{opts =} \FunctionTok{list}\NormalTok{(}\AttributeTok{dim =} \FunctionTok{c}\NormalTok{(}\DecValTok{10}\NormalTok{, }\DecValTok{20}\NormalTok{, }\DecValTok{30}\NormalTok{),}
                                                \AttributeTok{lrate    =} \FunctionTok{c}\NormalTok{(}\FloatTok{0.1}\NormalTok{, }\FloatTok{0.2}\NormalTok{),}
                                                \AttributeTok{nthread  =} \DecValTok{4}\NormalTok{,}
                                                \AttributeTok{niter    =} \DecValTok{10}\NormalTok{,}
                                                \AttributeTok{verbose  =} \ConstantTok{TRUE}\NormalTok{))}
\end{Highlighting}
\end{Shaded}

\begin{Shaded}
\begin{Highlighting}[]
\FunctionTok{str}\NormalTok{(reco)}
\end{Highlighting}
\end{Shaded}

\begin{verbatim}
## Reference class 'RecoSys' [package "recosystem"] with 2 fields
##  $ model     :Reference class 'RecoModel' [package "recosystem"] with 5 fields
##   ..$ path    : chr ""
##   ..$ nuser   : int 0
##   ..$ nitem   : int 0
##   ..$ nfac    : int 0
##   ..$ matrices: list()
##   ..and 16 methods, of which 2 are  possibly relevant:
##   ..  initialize, show#envRefClass
##  $ train_pars: list()
##  and 19 methods, of which 5 are  possibly relevant:
##    output, predict, show#envRefClass, train, tune
\end{verbatim}

\begin{Shaded}
\begin{Highlighting}[]
\FunctionTok{str}\NormalTok{(reco.tuned)}
\end{Highlighting}
\end{Shaded}

\begin{verbatim}
## List of 2
##  $ min:List of 7
##   ..$ dim     : int 30
##   ..$ costp_l1: num 0
##   ..$ costp_l2: num 0.01
##   ..$ costq_l1: num 0
##   ..$ costq_l2: num 0.1
##   ..$ lrate   : num 0.1
##   ..$ loss_fun: num 0.795
##  $ res:'data.frame': 96 obs. of  7 variables:
##   ..$ dim     : int [1:96] 10 20 30 10 20 30 10 20 30 10 ...
##   ..$ costp_l1: num [1:96] 0 0 0 0.1 0.1 0.1 0 0 0 0.1 ...
##   ..$ costp_l2: num [1:96] 0.01 0.01 0.01 0.01 0.01 0.01 0.1 0.1 0.1 0.1 ...
##   ..$ costq_l1: num [1:96] 0 0 0 0 0 0 0 0 0 0 ...
##   ..$ costq_l2: num [1:96] 0.01 0.01 0.01 0.01 0.01 0.01 0.01 0.01 0.01 0.01 ...
##   ..$ lrate   : num [1:96] 0.1 0.1 0.1 0.1 0.1 0.1 0.1 0.1 0.1 0.1 ...
##   ..$ loss_fun: num [1:96] 0.807 0.812 0.821 0.86 0.87 ...
##   ..- attr(*, "out.attrs")=List of 2
##   .. ..$ dim     : Named int [1:6] 3 2 2 2 2 2
##   .. .. ..- attr(*, "names")= chr [1:6] "dim" "costp_l1" "costp_l2" "costq_l1" ...
##   .. ..$ dimnames:List of 6
##   .. .. ..$ dim     : chr [1:3] "dim=10" "dim=20" "dim=30"
##   .. .. ..$ costp_l1: chr [1:2] "costp_l1=0.0" "costp_l1=0.1"
##   .. .. ..$ costp_l2: chr [1:2] "costp_l2=0.01" "costp_l2=0.10"
##   .. .. ..$ costq_l1: chr [1:2] "costq_l1=0.0" "costq_l1=0.1"
##   .. .. ..$ costq_l2: chr [1:2] "costq_l2=0.01" "costq_l2=0.10"
##   .. .. ..$ lrate   : chr [1:2] "lrate=0.1" "lrate=0.2"
\end{verbatim}

\paragraph{MF: Final Training}\label{mf-final-training}

\hfill\break

Finally, we train the model for the last time using the \texttt{reco\$train} method, passing to it as a \texttt{train\_data} parameter the same residual data we used for the tuning:

\begin{Shaded}
\begin{Highlighting}[]
\NormalTok{  reco}\SpecialCharTok{$}\FunctionTok{train}\NormalTok{(mf.edx.residual.reco, }\AttributeTok{opts =} \FunctionTok{c}\NormalTok{(reco.tuned}\SpecialCharTok{$}\NormalTok{min,}
                                       \AttributeTok{niter =} \DecValTok{20}\NormalTok{, }
                                       \AttributeTok{nthread =} \DecValTok{4}\NormalTok{)) }
\end{Highlighting}
\end{Shaded}

\paragraph{MF: Final Holdout Test}\label{mf-final-holdout-test}

\hfill\break

From now on, we will work with the \texttt{final\_holdout\_test} dataset for the final testing of our model, obtaining final predictions, and computing the \emph{Root Mean Squared Error}.

First, we get the final predictions combining the predicted residuals from the \texttt{reco} objects and predicted values we lastly get from the \emph{UMGYDE Model} as described in the \hyperref[umgyde-model-final-holdout-test-preliminary-assessment]{UMGYDE Model: Final Holdout Test (Preliminary Assessment)} section:

\begin{Shaded}
\begin{Highlighting}[]
\NormalTok{  mf.reco.residual }\OtherTok{\textless{}{-}}\NormalTok{ reco}\SpecialCharTok{$}\FunctionTok{predict}\NormalTok{(final\_holdout\_test.reco, }\FunctionTok{out\_memory}\NormalTok{())}

\NormalTok{  mf.predicted\_ratings }\OtherTok{\textless{}{-}} 
    \FunctionTok{clamp}\NormalTok{(final.UMGYDE.predicted}\SpecialCharTok{$}\NormalTok{predicted }\SpecialCharTok{+}\NormalTok{ mf.reco.residual)}
\end{Highlighting}
\end{Shaded}

\begin{Shaded}
\begin{Highlighting}[]
\FunctionTok{str}\NormalTok{(mf.reco.residual)}
\end{Highlighting}
\end{Shaded}

\begin{verbatim}
##  num [1:999999] 0.107 -0.264 0.128 0.245 0.682 ...
\end{verbatim}

\begin{Shaded}
\begin{Highlighting}[]
\FunctionTok{str}\NormalTok{(mf.predicted\_ratings)}
\end{Highlighting}
\end{Shaded}

\begin{verbatim}
##  num [1:999999] 4.59 4.74 4.76 3.55 4.76 ...
\end{verbatim}

Next, we compute the \emph{RMSE} using the \href{https://github.com/AzKurban-edX-DS/Capstone-MovieLens/blob/main/r/src/support-functions/common-helper.functions.R\#L28}{rmse2} custom helper function, already mentioned above in the \hyperref[umgyde-model-final-holdout-test-preliminary-assessment]{UMGYDE Model: Final Holdout Test (Preliminary Assessment)} section:

\begin{Shaded}
\begin{Highlighting}[]
\NormalTok{final\_holdout\_test.RMSE }\OtherTok{\textless{}{-}} \FunctionTok{rmse2}\NormalTok{(final\_holdout\_test}\SpecialCharTok{$}\NormalTok{rating,}
\NormalTok{                                     mf.predicted\_ratings)}
\end{Highlighting}
\end{Shaded}

\begin{noteblock}
The function \href{https://github.com/AzKurban-edX-DS/Capstone-MovieLens/blob/main/r/src/support-functions/common-helper.functions.R\#L28}{rmse2} is defined in the \href{https://github.com/AzKurban-edX-DS/Capstone-MovieLens/blob/main/r/src/support-functions/common-helper.functions.R\#L18}{(R)MSE-related functions} section of the \href{https://github.com/AzKurban-edX-DS/Capstone-MovieLens/blob/main/r/src/support-functions/common-helper.functions.R\#L28}{common-helper.functions.R} script on \emph{GitHub}.

\end{noteblock}

And add the result to our \emph{Result Table}:

\begin{Shaded}
\begin{Highlighting}[]
\NormalTok{final.MF.RMSEs.ResultTibble }\OtherTok{\textless{}{-}}\NormalTok{ final.RMSEs.ResultTibble.UMGYDE.rglr.tuned }\SpecialCharTok{|\textgreater{}} 
  \FunctionTok{RMSEs.AddRow}\NormalTok{(}\StringTok{"MF (Final Test)"}\NormalTok{, }
\NormalTok{               final\_holdout\_test.RMSE,}
               \AttributeTok{comment =} \StringTok{"Matrix Factorization of the Best Model Residuals, Final Holdout Test"}\NormalTok{)}
\end{Highlighting}
\end{Shaded}

\begin{Shaded}
\begin{Highlighting}[]
\FunctionTok{RMSE\_kable}\NormalTok{(final.MF.RMSEs.ResultTibble)}
\end{Highlighting}
\end{Shaded}

\begin{tabular}{>{\raggedright\arraybackslash}p{15em}>{\centering\arraybackslash}p{5em}>{\raggedright\arraybackslash}p{30em}}
\toprule
\textbf{Method} & \textbf{RMSE} & \textbf{Comment}\\
\midrule
Project Objective & 0.8649000 & \\
Overall Mean Rating Model & 1.0603462 & \\
User Effect Model & 0.9697962 & \\
User+Movie Effect Model & 0.8732081 & \\
Regularized UME Model & 0.8729730 & Computed for `lambda` = 0.38745002746582\\
\addlinespace
UMGE Model & 0.8729730 & User+Movie+Genre Effect (UMGE) Model\\
Regularized UMGE Model & 0.8729728 & Computed for `lambda` = 0.0359375\\
UMGYE Model & 0.8723973 & User+Movie+Genre+Year Effect (UMGYE) Model\\
Regularized UMGYE Model & 0.8721857 & Computed for `lambda` = 233.77668762207\\
UMGYDE (Default) Model & 0.8722410 & User+Movie+Genre+Year+Day Effect (UMGYDE) Model 
computed using `loess` function with default `degree` \& `span` parameters.\\
\addlinespace
Tuned UMGYDE.d0 Model & 0.8711131 & UMGYDE Model computed using function call: `loess(degree = 0, span = 0.00109375)`\\
Tuned UMGYDE.d1 Model & 0.8708295 & UMGYDE Model computed using function call: `loess(degree = 1, span = 0.00109375)`\\
Tuned UMGYDE.d2 Model & 0.8709886 & UMGYDE Model computed using function call: `loess(degree = 2, span = 0.00153125)`\\
Tuned UMGYDE Best Model & 0.8707850 & UMGYDE Model computed using `loess` function call with the best degree \& span values.\\
Regularized UMGYDE Model & 0.8707750 & The best tuned and regularized UMGYDE Model.\\
\addlinespace
Best UMGYDE Model (Final Test) & 0.8804225 & Final Holdout Test of the best tuned and regularized UMGYDE Model.\\
MF (Final Test) & 0.7875645 & Matrix Factorization of the Best Model Residuals, Final Holdout Test\\
\bottomrule
\end{tabular}

\begin{noteblock}
Finally, we have reached our ultimate goal, since the \emph{Root Mean Squared Error} we obtained has achieved the required \emph{Project Objective}.

\end{noteblock}

\newpage

\section{Conclusion}\label{conclusion}

\hfill\break

In this project, a \emph{Machine Learning Algorithm} was successfully developed for a \emph{Recommender System} (hereinafter referred to as \emph{RS MLA}) to predict movie ratings using the \href{http://grouplens.org/datasets/movielens/10m/}{10M version of the MovieLens} dataset.

Since it was needed to convert the original dataset into a \emph{Data Source}, suitable for processing by the \emph{RS MLA}, a package {[}edx.capstone.movielens.data{]}, specifically designed for this purpose, was developed and used, providing the \emph{RS MLA} with the necessary datasets.

Following the Netflix Grand Prize Contest requirements, the \emph{Root Mean Squared Error (RMSE)} metric was used to evaluate the accuracy of our models.

By incorporating essential factors such as overall average rating, movie, user, and some other observable effects, as well as efficient techniques such as \emph{Regularization} and \emph{Matrix Factorization} methods, the accuracy was significantly improved compared to the baseline model.

The baseline model was the \emph{Overall Mean Rating} model described in the \hyperref[overall-mean-rating-omr-model]{Overall Mean Rating (OMR) Model} section.

The following effects were taken into account:

\begin{itemize}
\tightlist
\item
  \textbf{\emph{User:}} based on which the \emph{User Effect (UE) Model} described in the \hyperref[user-effect-ue-model]{User Effect (UE) Model} section was built (significantly improved the \emph{RMSE} compared to the baseline model);
\item
  \textbf{\emph{Movie:}} based on which (and the \emph{User Effect}) the \emph{Regularized User+Movie Effect (UME) Model} described in the \hyperref[usermovie-effect-ume-model]{User+Movie Effect (UME) Model} section was eventually built (even better improved the \emph{RMSE} compared to the previous (\emph{UE}) model);
\item
  \textbf{\emph{Genre:}} based on which (and the previous effects) the \emph{Regularized User+Movie+Genre Effect (UMGE) Model} described in the \hyperref[usermoviegenre-effect-umge-model]{User+Movie+Genre Effect (UMGE) Model} section was built (even after \emph{regularization} it showed very little improvement in \emph{RMSE} compared to the previous (\emph{UME}) model; before regularization, for some reason, there was no improvement at all);
\item
  \textbf{\emph{Year:}} based on which (and the previous effects) the \emph{Regularized User+Movie+Genre+Year Effect (UMGYE) Model} described in the \hyperref[usermoviegenreyear-effect-umgye-model]{User+Movie+Genre+Year Effect (UMGYE) Model} section was built (provided a much better improvement in \emph{RMSE} compared to the previous (\emph{UMGE}) model but still two orders of magnitude less than the \emph{UE} and \emph{UME} models showed);
\item
  \textbf{\emph{Day}} based on which (and the previous effects) the \emph{Regularized (and Tuned) User+Movie+Genre+Year+Day Effect (UMGYDE) Model} described in the \hyperref[usermoviegenreyearday-effect-umgyde-model]{User+Movie+Genre+Year+Day Effect (UMGYDE) Model} section was built (showed almost twice better improvement than the previous (\emph{UMGYE}) model but still not good enough to meet the \emph{Project Objective});.
\item
  \textbf{\emph{Day (Matrix Factorization):}} (described in the \hyperref[umgyde-model-matrix-factorization-mf]{UMGYDE Model: Matrix Factorization (MF)} section) \emph{Matrix Factorization} method performed on the \emph{Regularized UMGYDE Model} provided substantial improvement comparable to that shown by the \emph{UE} and \emph{UME} models, and finally permitted to reach the \emph{Project Objective}.
\end{itemize}

The \emph{Result Summary Table} constructed below shows in detail the progress of our \emph{Recommender System} improvement:

\begin{Shaded}
\begin{Highlighting}[]
\NormalTok{total.RMSEs.ResultTibble }\OtherTok{\textless{}{-}} \FunctionTok{RMSEs.AddDiffColumn}\NormalTok{(final.MF.RMSEs.ResultTibble)}
\end{Highlighting}
\end{Shaded}

\begin{Shaded}
\begin{Highlighting}[]
\FunctionTok{RMSE.Total\_kable}\NormalTok{(total.RMSEs.ResultTibble)}
\end{Highlighting}
\end{Shaded}

\begin{tabular}{>{\raggedright\arraybackslash}p{15em}>{\centering\arraybackslash}p{7em}>{\raggedright\arraybackslash}p{5em}>{\raggedright\arraybackslash}p{25em}}
\toprule
\textbf{Method} & \textbf{RMSE} & \textbf{Diff} & \textbf{Comment}\\
\midrule
Project Objective & 0.8649000 & 0.0000000 & \\
Overall Mean Rating Model & 1.0603462 & -0.1954462 & \\
User Effect Model & 0.9697962 & 0.0905500 & \\
User+Movie Effect Model & 0.8732081 & 0.0965881 & \\
Regularized UME Model & 0.8729730 & 0.0002351 & Computed for `lambda` = 0.38745002746582\\
\addlinespace
UMGE Model & 0.8729730 & 0.0000000 & User+Movie+Genre Effect (UMGE) Model\\
Regularized UMGE Model & 0.8729728 & 0.0000003 & Computed for `lambda` = 0.0359375\\
UMGYE Model & 0.8723973 & 0.0005755 & User+Movie+Genre+Year Effect (UMGYE) Model\\
Regularized UMGYE Model & 0.8721857 & 0.0002117 & Computed for `lambda` = 233.77668762207\\
UMGYDE (Default) Model & 0.8722410 & -0.0000553 & User+Movie+Genre+Year+Day Effect (UMGYDE) Model 
computed using `loess` function with default `degree` \& `span` parameters.\\
\addlinespace
Tuned UMGYDE.d0 Model & 0.8711131 & 0.0011279 & UMGYDE Model computed using function call: `loess(degree = 0, span = 0.00109375)`\\
Tuned UMGYDE.d1 Model & 0.8708295 & 0.0002836 & UMGYDE Model computed using function call: `loess(degree = 1, span = 0.00109375)`\\
Tuned UMGYDE.d2 Model & 0.8709886 & -0.0001591 & UMGYDE Model computed using function call: `loess(degree = 2, span = 0.00153125)`\\
Tuned UMGYDE Best Model & 0.8707850 & 0.0002036 & UMGYDE Model computed using `loess` function call with the best degree \& span values.\\
Regularized UMGYDE Model & 0.8707750 & 0.0000100 & The best tuned and regularized UMGYDE Model.\\
\addlinespace
Best UMGYDE Model (Final Test) & 0.8804225 & -0.0096476 & Final Holdout Test of the best tuned and regularized UMGYDE Model.\\
MF (Final Test) & 0.7875645 & 0.0928580 & Matrix Factorization of the Best Model Residuals, Final Holdout Test\\
\bottomrule
\end{tabular}

\newpage

\section{Appendix}\label{appndx_a}

\subsection{Data Helper Functions}\label{appndx_a.data_helper.functions}

\hfill\break

\begin{noteblock}
The complete source code of the functions described in this section is available in the \href{https://github.com/AzKurban-edX-DS/Capstone-MovieLens/blob/main/r/src/support-functions/data.helper.functions.R\#L592}{data.helper.functions.R} script on \emph{GitHub}.

\end{noteblock}

\subsubsection{\texorpdfstring{\href{https://github.com/AzKurban-edX-DS/Capstone-MovieLens/blob/main/r/src/support-functions/data.helper.functions.R\#L432}{union\_cv\_results} Function}{union\_cv\_results Function}}\label{func.union_cv_results}

\hfill\break

Aggregates the input data frames into a single data frame.

\phantomsection\label{func.union_cv_results.Usage}
\paragraph{Usage}\label{usage}

\hfill\break

\begin{Shaded}
\begin{Highlighting}[]
\FunctionTok{union\_cv\_results}\NormalTok{(}\AttributeTok{data\_list =}\NormalTok{ data)}
\end{Highlighting}
\end{Shaded}

\phantomsection\label{func.union_cv_results.params}
\paragraph{Parameters}\label{parameters-24}

\hfill\break

\phantomsection\label{func.union_cv_results.params.data_list}
\begin{itemize}
\tightlist
\item
  \textbf{data\_list:} List of data frames representing the results of the \emph{N-Fold Cross Validation} method execution.;
\end{itemize}

\phantomsection\label{func.union_cv_results.details}
\paragraph{Details}\label{details-9}

\hfill\break

The function is used to aggregate the \emph{N-Fold Cross Validation} method result data into a single data frame for further processing.

\phantomsection\label{func.union_cv_results.value}
\paragraph{Value}\label{value}

\hfill\break

A data frame that is a union of the input data frames.

\phantomsection\label{func.union_cv_results.code}
\paragraph{Source Code}\label{source-code-24}

\hfill\break

The soure code of the \href{https://github.com/AzKurban-edX-DS/Capstone-MovieLens/blob/main/r/src/support-functions/data.helper.functions.R\#L432}{union\_cv\_results} function is shown below:

\begin{Shaded}
\begin{Highlighting}[]
\NormalTok{union\_cv\_results }\OtherTok{\textless{}{-}} \ControlFlowTok{function}\NormalTok{(data\_list) \{}
\NormalTok{  out\_dat }\OtherTok{\textless{}{-}}\NormalTok{ data\_list[[}\DecValTok{1}\NormalTok{]]}
  
  \ControlFlowTok{for}\NormalTok{ (i }\ControlFlowTok{in} \DecValTok{2}\SpecialCharTok{:}\NormalTok{CVFolds\_N)\{}
\NormalTok{    out\_dat }\OtherTok{\textless{}{-}} \FunctionTok{union}\NormalTok{(out\_dat, }
\NormalTok{                     data\_list[[i]])}
\NormalTok{  \}}
  
\NormalTok{  out\_dat}
\NormalTok{\}}
\end{Highlighting}
\end{Shaded}

\begin{noteblock}
The source code of the \href{https://github.com/AzKurban-edX-DS/Capstone-MovieLens/blob/main/r/src/support-functions/data.helper.functions.R\#L432}{union\_cv\_results} is also available in the \href{https://github.com/AzKurban-edX-DS/Capstone-MovieLens/blob/main/r/src/support-functions/data.helper.functions.R\#L431}{Model training} section of the \href{https://github.com/AzKurban-edX-DS/Capstone-MovieLens/blob/main/r/src/support-functions/data.helper.functions.R\#L432}{data.helper.functions.R} script on \emph{GitHub}.

\end{noteblock}

\subsubsection{\texorpdfstring{\href{https://github.com/AzKurban-edX-DS/Capstone-MovieLens/blob/main/r/src/support-functions/data.helper.functions.R\#L592}{data.plot} Function}{data.plot Function}}\label{func.data.plot}

\hfill\break

Plots a graph based on the dataset passed in the \hyperref[func.data.plot.params.data]{data} parameter, using both points and a line.

\phantomsection\label{func.data.plot.Usage}
\paragraph{Usage}\label{usage-1}

\hfill\break

\begin{Shaded}
\begin{Highlighting}[]
\FunctionTok{data.plot}\NormalTok{(data, }
\NormalTok{          title, }
\NormalTok{          xname, }
\NormalTok{          yname, }
          \AttributeTok{xlabel =} \ConstantTok{NULL}\NormalTok{, }
          \AttributeTok{ylabel =} \ConstantTok{NULL}\NormalTok{,}
          \AttributeTok{line\_col =} \StringTok{"blue"}\NormalTok{,}
          \AttributeTok{normalize =} \ConstantTok{FALSE}\NormalTok{)}
\end{Highlighting}
\end{Shaded}

\phantomsection\label{func.data.plot.params}
\paragraph{Parameters}\label{parameters-25}

\hfill\break

\phantomsection\label{func.data.plot.params.data}
\begin{itemize}
\tightlist
\item
  \textbf{data:} Dataset to use for the plot;
\end{itemize}

\phantomsection\label{func.data.plot.params.title}
\begin{itemize}
\tightlist
\item
  \textbf{title:} Title of the plot;
\end{itemize}

\phantomsection\label{func.data.plot.params.xname}
\begin{itemize}
\tightlist
\item
  \textbf{xname:} The name of the dataset column used as the source of the \(x\) variable for the plot;
\end{itemize}

\phantomsection\label{func.data.plot.params.yname}
\begin{itemize}
\tightlist
\item
  \textbf{yname:} The name of the dataset column used as the source of the \(y\) variable for the plot;
\end{itemize}

\phantomsection\label{func.data.plot.params.xlabel}
\begin{itemize}
\tightlist
\item
  \emph{xlabel = NULL}:** The \(x\) axis label. If \texttt{NULL}, the value of the \texttt{xname} parameter is used for the label;
\end{itemize}

\phantomsection\label{func.data.plot.params.ylabel}
\begin{itemize}
\tightlist
\item
  \textbf{ylabel = NULL:} The \(y\) axis label. If \texttt{NULL}, the value of the \texttt{yname} parameter is used for the label.;
\end{itemize}

\phantomsection\label{func.data.plot.params.line_col}
\begin{itemize}
\tightlist
\item
  \textbf{line\_col = blue:} The line color;
\end{itemize}

\phantomsection\label{func.data.plot.params.normalize}
\begin{itemize}
\tightlist
\item
  \textbf{normalize = FALSE:} If \texttt{TRUE}, the deviation of \(y\) from its mean is used to plot, rather than \(y\) values. Otherwise, the value of \(y\) is used.
\end{itemize}

\phantomsection\label{func.data.plot.details}
\paragraph{Details}\label{details-10}

\hfill\break

The function is a wrapper for the \href{https://ggplot2.tidyverse.org/reference/ggplot.html}{ggplot2::ggplot} function and is a simple and convenient tool for visualizing the data of this project.

\phantomsection\label{func.data.plot.code}
\paragraph{Source Code}\label{source-code-25}

\hfill\break

The soure code of the {[}data.plot{]}((\url{https://github.com/AzKurban-edX-DS/Capstone-MovieLens/blob/main/r/src/support-functions/data.helper.functions.R\#L592}) function is shown below:

\begin{Shaded}
\begin{Highlighting}[]
\NormalTok{data.plot }\OtherTok{\textless{}{-}} \ControlFlowTok{function}\NormalTok{(data, }
\NormalTok{                      title, }
\NormalTok{                      xname, }
\NormalTok{                      yname, }
                      \AttributeTok{xlabel =} \ConstantTok{NULL}\NormalTok{, }
                      \AttributeTok{ylabel =} \ConstantTok{NULL}\NormalTok{,}
                      \AttributeTok{line\_col =} \StringTok{"blue"}\NormalTok{,}
                      \AttributeTok{normalize =} \ConstantTok{FALSE}\NormalTok{) \{}
\NormalTok{  y }\OtherTok{\textless{}{-}}\NormalTok{ data[, yname]}
  
  \ControlFlowTok{if}\NormalTok{ (normalize) \{}
\NormalTok{    y }\OtherTok{\textless{}{-}}\NormalTok{ y }\SpecialCharTok{{-}} \FunctionTok{min}\NormalTok{(y)}
\NormalTok{  \}}
  
  \ControlFlowTok{if}\NormalTok{ (}\FunctionTok{is.null}\NormalTok{(xlabel)) \{}
\NormalTok{    xlabel }\OtherTok{=}\NormalTok{ xname}
\NormalTok{  \}}
  \ControlFlowTok{if}\NormalTok{ (}\FunctionTok{is.null}\NormalTok{(ylabel)) \{}
\NormalTok{    ylabel }\OtherTok{=}\NormalTok{ yname}
\NormalTok{  \}}
  
\NormalTok{  aes\_mapping }\OtherTok{\textless{}{-}} \FunctionTok{aes}\NormalTok{(}\AttributeTok{x =}\NormalTok{ data[, xname], }\AttributeTok{y =}\NormalTok{ y)}
  
\NormalTok{  data }\SpecialCharTok{|\textgreater{}} 
    \FunctionTok{ggplot}\NormalTok{(}\AttributeTok{mapping =}\NormalTok{ aes\_mapping) }\SpecialCharTok{+}
    \FunctionTok{ggtitle}\NormalTok{(title) }\SpecialCharTok{+}
    \FunctionTok{xlab}\NormalTok{(xlabel) }\SpecialCharTok{+}
    \FunctionTok{ylab}\NormalTok{(ylabel) }\SpecialCharTok{+}
    \FunctionTok{geom\_point}\NormalTok{() }\SpecialCharTok{+} 
    \FunctionTok{geom\_line}\NormalTok{(}\AttributeTok{color=}\NormalTok{line\_col)}
\NormalTok{\}}
\end{Highlighting}
\end{Shaded}

\begin{noteblock}
The source code of the \href{https://github.com/AzKurban-edX-DS/Capstone-MovieLens/blob/main/r/src/support-functions/data.helper.functions.R\#L592}{data.plot} is also available in the \href{https://github.com/AzKurban-edX-DS/Capstone-MovieLens/blob/main/r/src/support-functions/data.helper.functions.R\#L447}{Data Visualization} section of the \href{https://github.com/AzKurban-edX-DS/Capstone-MovieLens/blob/main/r/src/support-functions/data.helper.functions.R\#L592}{data.helper.functions.R} script on \emph{GitHub}.

\end{noteblock}

\subsubsection{\texorpdfstring{\href{https://github.com/AzKurban-edX-DS/Capstone-MovieLens/blob/main/r/src/support-functions/data.helper.functions.R\#L493}{data.plot.left.n} Function}{data.plot.left.n Function}}\label{func.data.plot.left.n}

\hfill\break

Plots a graph based on the first \(n\) rows (the \(n\) is specified by the \hyperref[func.data.plot.left.n.params.left.n]{left.n} parameter) of the dataset passed in the \hyperref[func.data.plot.left.n.params.data]{data} parameter, using both points and a line.

\begin{noteblock}
The dataset passed in the \hyperref[func.data.plot.left.n.params.data]{data} parameter is assumed to be sorted by the column used for the \(x\) axis (specified by the column name passed in the \hyperref[func.data.plot.left.n.params.xname]{xname} parameter).

\end{noteblock}

\phantomsection\label{func.data.plot.left.n.Usage}
\paragraph{Usage}\label{usage-2}

\hfill\break

\begin{Shaded}
\begin{Highlighting}[]
\NormalTok{data.plot.left.n }\OtherTok{\textless{}{-}} \ControlFlowTok{function}\NormalTok{(data, }
                             \AttributeTok{left.n =} \DecValTok{0}\NormalTok{,}
\NormalTok{                             title, }
\NormalTok{                             xname, }
\NormalTok{                             yname, }
\NormalTok{                             xlabel, }
\NormalTok{                             ylabel,}
                             \AttributeTok{line\_col =} \StringTok{"red"}\NormalTok{,}
                             \AttributeTok{normalize =} \ConstantTok{FALSE}\NormalTok{)}
\end{Highlighting}
\end{Shaded}

\phantomsection\label{func.data.plot.left.n.params}
\paragraph{Parameters}\label{parameters-26}

\hfill\break

\phantomsection\label{func.data.plot.left.n.params.data}
\begin{itemize}
\tightlist
\item
  \textbf{data:} Dataset to use for the plot.
\end{itemize}

\phantomsection\label{func.data.plot.left.n.params.left.n}
\begin{itemize}
\tightlist
\item
  \textbf{left.n:} If a positive numeric value, specifies the number of the first rows of the dataset to plot. Otherwise, the entire dataset data is set to plot.
\end{itemize}

\phantomsection\label{func.data.plot.left.n.params.title}
\begin{itemize}
\tightlist
\item
  \textbf{title:} Title of the plot.
\end{itemize}

\phantomsection\label{func.data.plot.left.n.params.xname}
\begin{itemize}
\tightlist
\item
  \textbf{xname:} The name of the dataset column used as the source of the \(x\) variable for the plot.
\end{itemize}

\phantomsection\label{func.data.plot.left.n.params.yname}
\begin{itemize}
\tightlist
\item
  \textbf{yname:} The name of the dataset column used as the source of the \(y\) variable for the plot.
\end{itemize}

\phantomsection\label{func.data.plot.left.n.params.xlabel}
\begin{itemize}
\tightlist
\item
  \emph{xlabel = NULL}:** The \(x\) axis label. If \texttt{NULL}, the value of the \texttt{xname} parameter is used for the label.
\end{itemize}

\phantomsection\label{func.data.plot.left.n.params.ylabel}
\begin{itemize}
\tightlist
\item
  \textbf{ylabel = NULL:} The \(y\) axis label. If \texttt{NULL}, the value of the \texttt{yname} parameter is used for the label.
\end{itemize}

\phantomsection\label{func.data.plot.left.n.params.line_col}
\begin{itemize}
\tightlist
\item
  \textbf{line\_col = blue:} The line color.
\end{itemize}

\phantomsection\label{func.data.plot.left.n.params.normalize}
\begin{itemize}
\tightlist
\item
  \textbf{normalize = FALSE:} If \texttt{TRUE}, the deviation of \(y\) from its mean is used to plot, rather than \(y\) values. Otherwise, the value of \(y\) is used.
\end{itemize}

\phantomsection\label{func.data.plot.left.n.details}
\paragraph{Details}\label{details-11}

\hfill\break

The function is a wrapper for the \hyperref[func.data.plot]{data.plot} function described above and is a simple and convenient tool for more detailed visualization of the left part of the graph, initially built based on the entire data set passed in the \hyperref[func.data.plot.left.n.params.data]{data} parameter.

\phantomsection\label{func.data.plot.left.n.code}
\paragraph{Source Code}\label{source-code-26}

\hfill\break

The source code of the \href{https://github.com/AzKurban-edX-DS/Capstone-MovieLens/blob/main/r/src/support-functions/data.helper.functions.R\#L493}{data.plot.left.n} function is shown below:

\begin{Shaded}
\begin{Highlighting}[]
\NormalTok{data.plot.left.n }\OtherTok{\textless{}{-}} \ControlFlowTok{function}\NormalTok{(data, }
                             \AttributeTok{left.n =} \DecValTok{0}\NormalTok{,}
\NormalTok{                             title, }
\NormalTok{                             xname, }
\NormalTok{                             yname, }
\NormalTok{                             xlabel, }
\NormalTok{                             ylabel,}
                             \AttributeTok{line\_col =} \StringTok{"red"}\NormalTok{,}
                             \AttributeTok{normalize =} \ConstantTok{FALSE}\NormalTok{) \{}
\NormalTok{  x\_col }\OtherTok{\textless{}{-}}\NormalTok{ data[, xname] }
\NormalTok{  y\_col }\OtherTok{\textless{}{-}}\NormalTok{ data[, yname]}
  
  
\NormalTok{  data.left }\OtherTok{\textless{}{-}}\NormalTok{ data}
  
  \ControlFlowTok{if}\NormalTok{ (left.n }\SpecialCharTok{\textgreater{}} \DecValTok{0}\NormalTok{) \{}
\NormalTok{    data.left }\OtherTok{\textless{}{-}}\NormalTok{ data }\SpecialCharTok{|\textgreater{}} 
      \FunctionTok{head}\NormalTok{(left.n)}
\NormalTok{  \}}
  
\NormalTok{  data.left }\SpecialCharTok{|\textgreater{}}
    \FunctionTok{data.plot}\NormalTok{(}\AttributeTok{title =}\NormalTok{ title,}
              \AttributeTok{xname =}\NormalTok{ xname,}
              \AttributeTok{yname =}\NormalTok{ yname,}
              \AttributeTok{xlabel =}\NormalTok{ xlabel,}
              \AttributeTok{ylabel =}\NormalTok{ ylabel,}
              \AttributeTok{line\_col =}\NormalTok{ line\_col,}
              \AttributeTok{normalize =}\NormalTok{ normalize)}
\NormalTok{\}}
\end{Highlighting}
\end{Shaded}

\begin{noteblock}
The source code of the \href{https://github.com/AzKurban-edX-DS/Capstone-MovieLens/blob/main/r/src/support-functions/data.helper.functions.R\#L493}{data.plot.left.n} is also available in the \href{https://github.com/AzKurban-edX-DS/Capstone-MovieLens/blob/main/r/src/support-functions/data.helper.functions.R\#L447}{Data Visualization} section of the \href{https://github.com/AzKurban-edX-DS/Capstone-MovieLens/blob/main/r/src/support-functions/data.helper.functions.R\#L493}{data.helper.functions.R} script on \emph{GitHub}.

\end{noteblock}

\subsubsection{\texorpdfstring{\href{https://github.com/AzKurban-edX-DS/Capstone-MovieLens/blob/main/r/src/support-functions/data.helper.functions.R\#L448}{data.plot.left\_detailed} Function}{data.plot.left\_detailed Function}}\label{func.data.plot.left_detailed}

\hfill\break

Combines two graphs plotted by the internally called helper functions in a grid: the first is a standard graph plotted by the \hyperref[func.data.plot]{data.plot} function described above, the second is a graph based on the first \(n\) rows (the \(n\) is specified by the \hyperref[func.data.plot.left.n.params.left.n]{left.n} parameter) of the dataset plotted by the \hyperref[func.data.plot.left.n]{data.plot.left.n} function also described above (The data for both graphs is passed in the {[}data{]}).(\#func.data.plot.left.n.params.data) parameter).

\begin{noteblock}
The dataset passed in the \hyperref[func.data.plot.left.n.params.data]{data} parameter is assumed to be sorted by the column used for the \(x\) axis (specified by the column name passed in the \hyperref[func.data.plot.left.n.params.xname]{xname} parameter).

\end{noteblock}

\phantomsection\label{func.data.plot.left_detailed.Usage}
\paragraph{Usage}\label{usage-3}

\hfill\break

\begin{Shaded}
\begin{Highlighting}[]
\NormalTok{data.plot.left\_detailed }\OtherTok{\textless{}{-}} \ControlFlowTok{function}\NormalTok{(data, }
                                    \AttributeTok{left.n =} \DecValTok{0}\NormalTok{,}
                                    \AttributeTok{title =} \ConstantTok{NULL}\NormalTok{,}
                                    \AttributeTok{title.left =} \ConstantTok{NULL}\NormalTok{,}
\NormalTok{                                    xname, }
\NormalTok{                                    yname, }
                                    \AttributeTok{xlabel1 =} \ConstantTok{NULL}\NormalTok{, }
                                    \AttributeTok{xlabel2 =} \ConstantTok{NULL}\NormalTok{, }
                                    \AttributeTok{ylabel1 =} \ConstantTok{NULL}\NormalTok{,}
                                    \AttributeTok{ylabel2 =} \ConstantTok{NULL}\NormalTok{,}
                                    \AttributeTok{line\_col1 =} \StringTok{"blue"}\NormalTok{,}
                                    \AttributeTok{line\_col2 =} \StringTok{"red"}\NormalTok{,}
                                    \AttributeTok{normalize =} \ConstantTok{FALSE}\NormalTok{)}
\end{Highlighting}
\end{Shaded}

\phantomsection\label{func.data.plot.left_detailed.params}
\paragraph{Parameters}\label{parameters-27}

\hfill\break

\phantomsection\label{func.data.plot.left_detailed.params.data}
\begin{itemize}
\tightlist
\item
  \textbf{data:} Dataset to use for the plot.
\end{itemize}

\phantomsection\label{func.data.plot.left.n.params.left.n}
\begin{itemize}
\tightlist
\item
  \textbf{left.n:} If a positive numeric value, specifies the number of the first rows of the dataset to plot. Otherwise, the entire dataset data is set to plot.
\end{itemize}

\phantomsection\label{func.data.plot.left_detailed.params.title}
\begin{itemize}
\tightlist
\item
  \textbf{title:} Title of the plot.
\end{itemize}

\phantomsection\label{func.data.plot.left_detailed.params.xname}
\begin{itemize}
\tightlist
\item
  \textbf{xname:} The name of the dataset column used as the source of the \(x\) variable for the plot.
\end{itemize}

\phantomsection\label{func.data.plot.left_detailed.params.yname}
\begin{itemize}
\tightlist
\item
  \textbf{yname:} The name of the dataset column used as the source of the \(y\) variable for the plot.
\end{itemize}

\phantomsection\label{func.data.plot.left_detailed.params.xlabel1}
\begin{itemize}
\tightlist
\item
  \emph{xlabel1 = NULL}:** The \(x\) axis label for the first plot. If \texttt{NULL}, the value of the \texttt{xname} parameter is used for the label.
\end{itemize}

\phantomsection\label{func.data.plot.left_detailed.params.xlabel2}
\begin{itemize}
\tightlist
\item
  \emph{xlabel2 = NULL}:** The \(x\) axis label for the second plot. If \texttt{NULL}, the value of the \texttt{xname} parameter is used for the label.
\end{itemize}

\phantomsection\label{func.data.plot.left_detailed.params.ylabel1}
\begin{itemize}
\tightlist
\item
  \textbf{ylabel1 = NULL:} The \(y\) axis label for the first plot. If \texttt{NULL}, the value of the \texttt{yname} parameter is used for the label.
\end{itemize}

\phantomsection\label{func.data.plot.left_detailed.params.ylabel2}
\begin{itemize}
\tightlist
\item
  \textbf{ylabel2 = NULL:} The \(y\) axis label for the second plot. If \texttt{NULL}, the value of the \texttt{yname} parameter is used for the label.
\end{itemize}

\phantomsection\label{func.data.plot.left_detailed.params.line_col}
\begin{itemize}
\tightlist
\item
  \textbf{line\_col = blue:} The line color.
\end{itemize}

\phantomsection\label{func.data.plot.left_detailed.params.normalize}
\begin{itemize}
\tightlist
\item
  \textbf{normalize = FALSE:} If \texttt{TRUE}, the deviation of \(y\) from its mean is used to plot, rather than \(y\) values. Otherwise, the value of \(y\) is used.
\end{itemize}

\phantomsection\label{func.data.plot.left_detailed.details}
\paragraph{Details}\label{details-12}

\hfill\break

Under the hood, the function calls internally the \hyperref[func.data.plot]{data.plot} and \hyperref[func.data.plot.left.n]{data.plot.left.n} functions (described above) to build two graphs and combines them into a grid.

\phantomsection\label{func.data.plot.left_detailed.code}
\paragraph{Source Code}\label{source-code-27}

\hfill\break

The soure code of the \href{https://github.com/AzKurban-edX-DS/Capstone-MovieLens/blob/main/r/src/support-functions/data.helper.functions.R\#L448}{data.plot.left\_detailed} function is shown below:

\begin{Shaded}
\begin{Highlighting}[]
\NormalTok{data.plot.left\_detailed }\OtherTok{\textless{}{-}} \ControlFlowTok{function}\NormalTok{(data, }
                                    \AttributeTok{left.n =} \DecValTok{0}\NormalTok{,}
                                    \AttributeTok{title =} \ConstantTok{NULL}\NormalTok{,}
                                    \AttributeTok{title.left =} \ConstantTok{NULL}\NormalTok{,}
\NormalTok{                                    xname, }
\NormalTok{                                    yname, }
                                    \AttributeTok{xlabel1 =} \ConstantTok{NULL}\NormalTok{, }
                                    \AttributeTok{xlabel2 =} \ConstantTok{NULL}\NormalTok{, }
                                    \AttributeTok{ylabel1 =} \ConstantTok{NULL}\NormalTok{,}
                                    \AttributeTok{ylabel2 =} \ConstantTok{NULL}\NormalTok{,}
                                    \AttributeTok{line\_col1 =} \StringTok{"blue"}\NormalTok{,}
                                    \AttributeTok{line\_col2 =} \StringTok{"red"}\NormalTok{,}
                                    \AttributeTok{normalize =} \ConstantTok{FALSE}\NormalTok{) \{}
  \ControlFlowTok{if}\NormalTok{(}\FunctionTok{is.null}\NormalTok{(xlabel1)) \{}
\NormalTok{    xlabel1 }\OtherTok{\textless{}{-}}\NormalTok{ xname}
\NormalTok{  \}}
  \ControlFlowTok{if}\NormalTok{(}\FunctionTok{is.null}\NormalTok{(xlabel2)) \{}
\NormalTok{    xlabel2 }\OtherTok{\textless{}{-}} \FunctionTok{str\_glue}\NormalTok{(xlabel1, }\StringTok{" (left part)"}\NormalTok{)}
\NormalTok{  \}}
  \ControlFlowTok{if}\NormalTok{(}\FunctionTok{is.null}\NormalTok{(ylabel1)) \{}
\NormalTok{    ylabel1 }\OtherTok{\textless{}{-}}\NormalTok{ yname}
\NormalTok{  \}}
  \ControlFlowTok{if}\NormalTok{(}\FunctionTok{is.null}\NormalTok{(ylabel2)) \{}
\NormalTok{    ylabel2 }\OtherTok{\textless{}{-}}\NormalTok{ ylabel1}
\NormalTok{  \}}
  
\NormalTok{  p1 }\OtherTok{\textless{}{-}}\NormalTok{ data }\SpecialCharTok{|\textgreater{}}
    \FunctionTok{data.plot}\NormalTok{(}\AttributeTok{title =}\NormalTok{ title,}
              \AttributeTok{xname =}\NormalTok{ xname, }
              \AttributeTok{yname =}\NormalTok{ yname, }
              \AttributeTok{xlabel =}\NormalTok{ xlabel1, }
              \AttributeTok{ylabel =}\NormalTok{ ylabel1,}
              \AttributeTok{line\_col =}\NormalTok{ line\_col1)}
  
\NormalTok{  p2 }\OtherTok{\textless{}{-}}\NormalTok{ data }\SpecialCharTok{|\textgreater{}}
    \FunctionTok{data.plot.left.n}\NormalTok{(}\AttributeTok{left.n =}\NormalTok{ left.n,}
                     \AttributeTok{title =}\NormalTok{ title.left,}
                     \AttributeTok{xname =}\NormalTok{ xname, }
                     \AttributeTok{yname =}\NormalTok{ yname, }
                     \AttributeTok{xlabel =}\NormalTok{ xlabel2, }
                     \AttributeTok{ylabel =}\NormalTok{ ylabel2,}
                     \AttributeTok{line\_col =}\NormalTok{ line\_col2,}
                     \AttributeTok{normalize =}\NormalTok{ normalize)}
  \FunctionTok{grid.arrange}\NormalTok{(p1, p2)}
\NormalTok{\}}
\end{Highlighting}
\end{Shaded}

\begin{noteblock}
The source code of the \href{https://github.com/AzKurban-edX-DS/Capstone-MovieLens/blob/main/r/src/support-functions/data.helper.functions.R\#L448}{data.plot.left\_detailed} is also available in the \href{https://github.com/AzKurban-edX-DS/Capstone-MovieLens/blob/main/r/src/support-functions/data.helper.functions.R\#L447}{Data Visualization} section of the \href{https://github.com/AzKurban-edX-DS/Capstone-MovieLens/blob/main/r/src/support-functions/data.helper.functions.R\#L448}{data.helper.functions.R} script on \emph{GitHub}.

\end{noteblock}

\newpage

\subsection{Common Helper Functions}\label{common-helper-functions-1}

\subsubsection{Common Helper Functions: Result RMSEs Table}\label{common-helper-functions-result-rmses-table}

\hfill\break

\begin{noteblock}
The complete source code of the functions described in this section are available in the \href{https://github.com/AzKurban-edX-DS/Capstone-MovieLens/blob/main/r/src/support-functions/common-helper.functions.R\#L74}{Model Tuning} section of the \href{https://github.com/AzKurban-edX-DS/Capstone-MovieLens/blob/main/r/src/support-functions/common-helper.functions.R}{common-helper.functions.R} script on \emph{GitHub}.

\end{noteblock}

\paragraph{\texorpdfstring{\href{https://github.com/AzKurban-edX-DS/Capstone-MovieLens/blob/main/r/src/support-functions/common-helper.functions.R\#L}{RMSEs.} Function}{RMSEs. Function}}\label{func.RMSEs.}

\hfill\break

\begin{noteblock}
The complete source code of this function is available in the \href{https://github.com/AzKurban-edX-DS/Capstone-MovieLens/blob/main/r/src/support-functions/common-helper.functions.R\#L32}{RMSEs Result Tibble} section of the \href{https://github.com/AzKurban-edX-DS/Capstone-MovieLens/blob/main/r/src/support-functions/common-helper.functions.R\#L}{common-helper.functions.R} script on \emph{GitHub}.

\end{noteblock}

\phantomsection\label{func.RMSEs.usage}
\subparagraph{Usage}\label{usage-4}

\hfill\break

\phantomsection\label{func.RMSEs..params}
\subparagraph{Parameters}\label{parameters-28}

\hfill\break

\phantomsection\label{func.RMSEs..params.RMSEs}
\begin{itemize}
\tightlist
\item
  \textbf{\emph{RMSEs:}} \emph{Resut (RMSEs) Table} to modify;
\end{itemize}

\phantomsection\label{func.RMSEs..params.}
\begin{itemize}
\tightlist
\item
  \textbf{\emph{:}} ;
\end{itemize}

\phantomsection\label{func.RMSEs..params.}
\begin{itemize}
\tightlist
\item
  \textbf{\emph{:}} ;
\end{itemize}

\phantomsection\label{func.RMSEs..params.}
\begin{itemize}
\tightlist
\item
  \textbf{\emph{:}} ;
\end{itemize}

\phantomsection\label{func.RMSEs..params.}
\begin{itemize}
\tightlist
\item
  \textbf{\emph{:}} ;
\end{itemize}

\phantomsection\label{func.RMSEs..params.}
\begin{itemize}
\tightlist
\item
  \textbf{\emph{:}} ;
\end{itemize}

\phantomsection\label{func.RMSEs..params.}
\begin{itemize}
\tightlist
\item
  \textbf{\emph{:}} ;
\end{itemize}

\phantomsection\label{func.RMSEs..params.}
\begin{itemize}
\tightlist
\item
  \textbf{\emph{:}} ;
\end{itemize}

\phantomsection\label{func.RMSEs..params.}
\begin{itemize}
\tightlist
\item
  \textbf{\emph{:}} ;
\end{itemize}

\phantomsection\label{func.RMSEs.details}
\subparagraph{Details}\label{details-13}

\hfill\break

\phantomsection\label{func.RMSEs.code}
\subparagraph{Source Code}\label{source-code-28}

\hfill\break

The soure code of the function is provided below:

\subparagraph{Return}\label{return-25}

The modified \emph{Resut (RMSEs) Table}.

\paragraph{\texorpdfstring{\href{https://github.com/AzKurban-edX-DS/Capstone-MovieLens/blob/main/r/src/support-functions/common-helper.functions.R\#L38}{RMSEs.AddRow} Function}{RMSEs.AddRow Function}}\label{func.RMSEs.AddRow}

\hfill\break

\begin{noteblock}
The complete source code of this function is available in the \href{https://github.com/AzKurban-edX-DS/Capstone-MovieLens/blob/main/r/src/support-functions/common-helper.functions.R\#L32}{RMSEs Result Tibble} section of the \href{https://github.com/AzKurban-edX-DS/Capstone-MovieLens/blob/main/r/src/support-functions/common-helper.functions.R\#L38}{common-helper.functions.R} script on \emph{GitHub}.

\end{noteblock}

\phantomsection\label{func.RMSEs.AddRow.usage}
\subparagraph{Usage}\label{usage-5}

\hfill\break

\begin{Shaded}
\begin{Highlighting}[]
\FunctionTok{RMSEs.AddRow}\NormalTok{(RMSEs, }
\NormalTok{             method, }
\NormalTok{             value, }
             \AttributeTok{comment =} \StringTok{""}\NormalTok{, }
             \AttributeTok{before =} \ConstantTok{NULL}\NormalTok{)}
\end{Highlighting}
\end{Shaded}

\phantomsection\label{func.RMSEs.AddRow.params}
\subparagraph{Parameters}\label{parameters-29}

\hfill\break

\phantomsection\label{func.RMSEs.AddRow.params.RMSEs}
\begin{itemize}
\tightlist
\item
  \textbf{\emph{RMSEs:}} \emph{Resut (RMSEs) Table} to modify;
\end{itemize}

\phantomsection\label{func.RMSEs.AddRow.params.method}
\begin{itemize}
\tightlist
\item
  \textbf{\emph{method:}} The \emph{Method or Model} being evaluated;
\end{itemize}

\phantomsection\label{func.RMSEs.AddRow.params.value}
\begin{itemize}
\tightlist
\item
  \textbf{\emph{value:}} The \emph{Root Mean Squared Error (RMSE)} value;
\end{itemize}

\phantomsection\label{func.RMSEs.AddRow.params.comment}
\begin{itemize}
\tightlist
\item
  \textbf{\emph{comment:}} Detailed information about the record or any other comment (if needed);
\end{itemize}

\phantomsection\label{func.RMSEs.AddRow.params.before}
\begin{itemize}
\tightlist
\item
  \textbf{\emph{before:}} One-based row index where to add the new rows, if NULL (default): after last row;
\end{itemize}

\phantomsection\label{func.RMSEs.AddRow.code}
\subparagraph{Source Code}\label{source-code-29}

\hfill\break

The soure code of the function is provided below:

\begin{Shaded}
\begin{Highlighting}[]
\NormalTok{RMSEs.AddRow }\OtherTok{\textless{}{-}} \ControlFlowTok{function}\NormalTok{(RMSEs, }
\NormalTok{                         method, }
\NormalTok{                         value, }
                         \AttributeTok{comment =} \StringTok{""}\NormalTok{, }
                         \AttributeTok{before =} \ConstantTok{NULL}\NormalTok{)\{}
\NormalTok{  RMSEs }\SpecialCharTok{|\textgreater{}}
    \FunctionTok{add\_row}\NormalTok{(}\AttributeTok{Method =}\NormalTok{ method,}
            \AttributeTok{RMSE =}\NormalTok{ value,}
            \AttributeTok{Comment =}\NormalTok{ comment,}
            \AttributeTok{.before =}\NormalTok{ before)}
\NormalTok{\}}
\end{Highlighting}
\end{Shaded}

\subparagraph{Return}\label{return-26}

The modified \emph{Resut (RMSEs) Table}.

\paragraph{\texorpdfstring{\href{https://github.com/AzKurban-edX-DS/Capstone-MovieLens/blob/main/r/src/support-functions/common-helper.functions.R\#L49}{RMSEs.AddDiffColumn} Function}{RMSEs.AddDiffColumn Function}}\label{func.RMSEs.AddDiffColumn}

\hfill\break

\begin{noteblock}
The complete source code of this function is available in the \href{https://github.com/AzKurban-edX-DS/Capstone-MovieLens/blob/main/r/src/support-functions/common-helper.functions.R\#L32}{RMSEs Result Tibble} section of the \href{https://github.com/AzKurban-edX-DS/Capstone-MovieLens/blob/main/r/src/support-functions/common-helper.functions.R\#L49}{common-helper.functions.R} script on \emph{GitHub}.

\end{noteblock}

Adds a calculated column named \emph{Diff} to the \emph{Result(RMSEs)} table (passed in the \hyperref[func.RMSEs.AddDiffColumn.params.RMSEs]{RMSEs} parameter) that contains, for each row, the difference between the value of the \emph{RMSE} column for the previous row and the value of the same column for the current row. This is true for each row except the first, for which the value is always zero.

\phantomsection\label{func.RMSEs.AddDiffColumn.usage}
\subparagraph{Usage}\label{usage-6}

\hfill\break

\begin{Shaded}
\begin{Highlighting}[]
\NormalTok{RMSEs.ModifiedTable }\OtherTok{\textless{}{-}} \FunctionTok{RMSEs.AddDiffColumn}\NormalTok{(RMSEs)}
\end{Highlighting}
\end{Shaded}

\phantomsection\label{func.RMSEs.AddDiffColumn.params}
\subparagraph{Parameters}\label{parameters-30}

\hfill\break

\phantomsection\label{func.RMSEs.AddDiffColumn.params.RMSEs}
\begin{itemize}
\tightlist
\item
  \textbf{\emph{RMSEs:}} \emph{Resut (RMSEs) Table} to modify;
\end{itemize}

\phantomsection\label{func.RMSEs.AddDiffColumn.code}
\subparagraph{Source Code}\label{source-code-30}

\hfill\break

The soure code of the function is provided below:

\begin{Shaded}
\begin{Highlighting}[]
\NormalTok{RMSEs.AddDiffColumn }\OtherTok{\textless{}{-}} \ControlFlowTok{function}\NormalTok{(RMSEs)\{}
\NormalTok{  RMSEs.Diff }\OtherTok{\textless{}{-}}\NormalTok{ RMSEs }\SpecialCharTok{|\textgreater{}}
    \FunctionTok{RMSEs.AddRow}\NormalTok{(}\ConstantTok{NULL}\NormalTok{, project\_objective, }\AttributeTok{before =} \DecValTok{1}\NormalTok{) }\CommentTok{\#|\textgreater{}}
    \CommentTok{\#RMSEs.AddRow(NULL, 0, before = 1)}
    
  
\NormalTok{  RMSEs.Diff }\OtherTok{\textless{}{-}}\NormalTok{ RMSEs.Diff[}\SpecialCharTok{{-}}\FunctionTok{nrow}\NormalTok{(RMSEs.Diff),]}
    
\NormalTok{  RMSEs }\SpecialCharTok{|\textgreater{}}
  \FunctionTok{add\_column}\NormalTok{(}\AttributeTok{Diff =}\NormalTok{ RMSEs.Diff}\SpecialCharTok{$}\NormalTok{RMSE }\SpecialCharTok{{-}}\NormalTok{ RMSEs}\SpecialCharTok{$}\NormalTok{RMSE, }
             \AttributeTok{.before =} \StringTok{"Comment"}\NormalTok{)}
  
\NormalTok{\}}
\end{Highlighting}
\end{Shaded}

\begin{noteblock}
Under the hood, the function calls internally the \hyperref[func.RMSEs.AddRow]{RMSEs.AddRow} function described above.

\end{noteblock}

\subparagraph{Return}\label{return-27}

The modified \emph{Resut (RMSEs) Table}.

\paragraph{\texorpdfstring{\href{https://github.com/AzKurban-edX-DS/Capstone-MovieLens/blob/main/r/src/support-functions/common-helper.functions.R\#L74}{RMSE\_kable} Function}{RMSE\_kable Function}}\label{func.RMSE_kable}

\hfill\break

\begin{noteblock}
The complete source code of this function is available in the \href{https://github.com/AzKurban-edX-DS/Capstone-MovieLens/blob/main/r/src/support-functions/common-helper.functions.R\#L32}{RMSEs Result Tibble} section of the \href{https://github.com/AzKurban-edX-DS/Capstone-MovieLens/blob/main/r/src/support-functions/common-helper.functions.R\#L74}{common-helper.functions.R} script on \emph{GitHub}.

\end{noteblock}

Prints the \emph{Resut (RMSEs) Table}.

\begin{noteblock}

It is assumed that the table has the following three columns:

\begin{enumerate}
\def\labelenumi{\arabic{enumi}.}
\tightlist
\item
  \emph{Method};
\item
  \emph{RMSE};
\item
  \emph{Comment};
\end{enumerate}

\end{noteblock}

\phantomsection\label{func.RMSE_kableusage}
\subparagraph{Usage}\label{usage-7}

\hfill\break

\begin{Shaded}
\begin{Highlighting}[]
\NormalTok{RMSE\_kable }\OtherTok{\textless{}{-}} \ControlFlowTok{function}\NormalTok{(RMSEs,}
                       \AttributeTok{col1width =} \DecValTok{15}\NormalTok{,}
                       \AttributeTok{col2width =} \DecValTok{5}\NormalTok{,}
                       \AttributeTok{col3width =} \DecValTok{30}\NormalTok{)}
\end{Highlighting}
\end{Shaded}

\phantomsection\label{func.RMSE_kable.params}
\subparagraph{Parameters}\label{parameters-31}

\hfill\break

\phantomsection\label{func.RMSE_kable.params.RMSEs}
\begin{itemize}
\tightlist
\item
  \textbf{\emph{RMSEs:}} \emph{Resut (RMSEs) Table} to print;
\end{itemize}

\phantomsection\label{func.RMSE_kable.params.col1width}
\begin{itemize}
\tightlist
\item
  \textbf{\emph{col1width:}} Width of the first column;
\end{itemize}

\phantomsection\label{func.RMSE_kable.params.col2width}
\begin{itemize}
\tightlist
\item
  \textbf{\emph{col2width:}} Width of the second column;
\end{itemize}

\phantomsection\label{func.RMSE_kable.params.col3width}
\begin{itemize}
\tightlist
\item
  \textbf{\emph{col3width:}} Width of the third column;
\end{itemize}

\phantomsection\label{func.RMSE_kablecode}
\subparagraph{Source Code}\label{source-code-31}

\hfill\break

The soure code of the function is provided below:

\begin{Shaded}
\begin{Highlighting}[]
\NormalTok{RMSE\_kable }\OtherTok{\textless{}{-}} \ControlFlowTok{function}\NormalTok{(RMSEs,}
                       \AttributeTok{col1width =} \DecValTok{15}\NormalTok{,}
                       \AttributeTok{col2width =} \DecValTok{5}\NormalTok{,}
                       \AttributeTok{col3width =} \DecValTok{30}\NormalTok{)\{}
\NormalTok{  RMSEs }\SpecialCharTok{|\textgreater{}}
    \FunctionTok{kable}\NormalTok{(}\AttributeTok{align=}\StringTok{\textquotesingle{}lcl\textquotesingle{}}\NormalTok{, }\AttributeTok{booktabs =}\NormalTok{ T, }\AttributeTok{padding =} \DecValTok{5}\NormalTok{) }\SpecialCharTok{|\textgreater{}} 
    \FunctionTok{row\_spec}\NormalTok{(}\DecValTok{0}\NormalTok{, }\AttributeTok{bold =}\NormalTok{ T) }\SpecialCharTok{|\textgreater{}}
    \FunctionTok{column\_spec}\NormalTok{(}\AttributeTok{column =} \DecValTok{1}\NormalTok{, }\AttributeTok{width =} \FunctionTok{RSME.tibble.col\_width}\NormalTok{(col1width)) }\SpecialCharTok{|\textgreater{}}
    \FunctionTok{column\_spec}\NormalTok{(}\AttributeTok{column =} \DecValTok{2}\NormalTok{, }\AttributeTok{width =} \FunctionTok{RSME.tibble.col\_width}\NormalTok{(col2width)) }\SpecialCharTok{|\textgreater{}}
    \FunctionTok{column\_spec}\NormalTok{(}\AttributeTok{column =} \DecValTok{3}\NormalTok{, }\AttributeTok{width =} \FunctionTok{RSME.tibble.col\_width}\NormalTok{(col3width)) }
\NormalTok{\}}
\end{Highlighting}
\end{Shaded}

\paragraph{\texorpdfstring{\href{https://github.com/AzKurban-edX-DS/Capstone-MovieLens/blob/main/r/src/support-functions/common-helper.functions.R\#L85}{RMSE.Total\_kable} Function}{RMSE.Total\_kable Function}}\label{func.RMSE.Total_kable}

\hfill\break

\begin{noteblock}
The complete source code of this function is available in the \href{https://github.com/AzKurban-edX-DS/Capstone-MovieLens/blob/main/r/src/support-functions/common-helper.functions.R\#L32}{RMSEs Result Tibble} section of the \href{https://github.com/AzKurban-edX-DS/Capstone-MovieLens/blob/main/r/src/support-functions/common-helper.functions.R\#L85}{common-helper.functions.R} script on \emph{GitHub}.

\end{noteblock}

Prints the \emph{Resut (RMSEs) Table}.

\begin{noteblock}

It is assumed that the table has the following four columns:

\begin{enumerate}
\def\labelenumi{\arabic{enumi}.}
\tightlist
\item
  \emph{Method};
\item
  \emph{RMSE};
\item
  \emph{Diff};
\item
  \emph{Comment};
\end{enumerate}

\end{noteblock}

\phantomsection\label{func.RMSE.Total_kableusage}
\subparagraph{Usage}\label{usage-8}

\hfill\break

\begin{Shaded}
\begin{Highlighting}[]
\NormalTok{RMSE.Total\_kable }\OtherTok{\textless{}{-}} \ControlFlowTok{function}\NormalTok{(RMSEs,}
                       \AttributeTok{col1width =} \DecValTok{15}\NormalTok{,}
                       \AttributeTok{col2width =} \DecValTok{7}\NormalTok{,}
                       \AttributeTok{col3width =} \DecValTok{5}\NormalTok{,}
                       \AttributeTok{col4width =} \DecValTok{25}\NormalTok{)}
\end{Highlighting}
\end{Shaded}

\phantomsection\label{func.RMSE.Total_kable.params}
\subparagraph{Parameters}\label{parameters-32}

\hfill\break

\phantomsection\label{func.RMSE.Total_kable.params.RMSEs}
\begin{itemize}
\tightlist
\item
  \textbf{\emph{RMSEs:}} \emph{Resut (RMSEs) Table} to print;
\end{itemize}

\phantomsection\label{func.RMSE.Total_kable.params.}
\begin{itemize}
\tightlist
\item
  \textbf{\emph{col1width:}} Width of the first column;
\end{itemize}

\phantomsection\label{func.RMSE.Total_kable.params.}
\begin{itemize}
\tightlist
\item
  \textbf{\emph{col2width:}} Width of the second column;
\end{itemize}

\phantomsection\label{func.RMSE.Total_kable.params.}
\begin{itemize}
\tightlist
\item
  \textbf{\emph{col3width:}} Width of the third column;
\end{itemize}

\phantomsection\label{func.RMSE.Total_kable.params.}
\begin{itemize}
\tightlist
\item
  \textbf{\emph{col4width:}} Width of the fourth column;
\end{itemize}

\phantomsection\label{func.RMSE.Total_kablecode}
\subparagraph{Source Code}\label{source-code-32}

\hfill\break

The soure code of the function is provided below:

\begin{Shaded}
\begin{Highlighting}[]
\NormalTok{RMSE.Total\_kable }\OtherTok{\textless{}{-}} \ControlFlowTok{function}\NormalTok{(RMSEs,}
                       \AttributeTok{col1width =} \DecValTok{15}\NormalTok{,}
                       \AttributeTok{col2width =} \DecValTok{7}\NormalTok{,}
                       \AttributeTok{col3width =} \DecValTok{5}\NormalTok{, }
                       \AttributeTok{col4width =} \DecValTok{25}\NormalTok{)\{}
\NormalTok{  RMSEs }\SpecialCharTok{|\textgreater{}} 
    \FunctionTok{RMSE\_kable}\NormalTok{(col1width,}
\NormalTok{               col2width,}
\NormalTok{               col3width) }\SpecialCharTok{|\textgreater{}}
    \FunctionTok{column\_spec}\NormalTok{(}\AttributeTok{column =} \DecValTok{4}\NormalTok{, }
                \AttributeTok{width =} \FunctionTok{RSME.tibble.col\_width}\NormalTok{(col4width)) }
\NormalTok{\}}
\end{Highlighting}
\end{Shaded}

\begin{noteblock}
Under the hood, the function calls internally the \hyperref[func.RMSE_kable]{RMSE\_kable} function described above.

\end{noteblock}

\subsubsection{Common Helper Functions: Regularization}\label{appndx_a.CHF.regularization}

\hfill\break

\begin{noteblock}
The complete source code of the functions described in this section are available in the \href{https://github.com/AzKurban-edX-DS/Capstone-MovieLens/blob/main/r/src/support-functions/common-helper.functions.R\#L74}{Model Tuning} section of the \href{https://github.com/AzKurban-edX-DS/Capstone-MovieLens/blob/main/r/src/support-functions/common-helper.functions.R}{common-helper.functions.R} script on \emph{GitHub}.

\end{noteblock}

\paragraph{\texorpdfstring{\href{https://github.com/AzKurban-edX-DS/Capstone-MovieLens/blob/main/r/src/support-functions/common-helper.functions.R\#L63}{mean\_reg} Function}{mean\_reg Function}}\label{func.mean_reg}

\hfill\break

A general function for computing the arithmetic mean given a \emph{regularization parameter} \(\lambda\). We will call it the \emph{penaized mean}.

\phantomsection\label{func.mean_reg.usage}
\subparagraph{Usage}\label{usage-9}

\hfill\break

\begin{Shaded}
\begin{Highlighting}[]
\FunctionTok{mean\_reg}\NormalTok{(vals, }\AttributeTok{lambda =} \DecValTok{0}\NormalTok{, }\AttributeTok{na.rm =} \ConstantTok{TRUE}\NormalTok{)}
\end{Highlighting}
\end{Shaded}

\phantomsection\label{func.mean_reg.params}
\subparagraph{Parameters}\label{parameters-33}

\hfill\break

\phantomsection\label{func.mean_reg.params.vals}
\begin{itemize}
\tightlist
\item
  \textbf{vals:} values (usually a numeric vector) to calculate the \emph{penalized mean};
\end{itemize}

\phantomsection\label{func.mean_reg.params.lambda}
\begin{itemize}
\tightlist
\item
  \textbf{lambda:} a \emph{regularization parameter} \(\lambda\) used in the \emph{Regularization techniques}.
\end{itemize}

\begin{itemize}
\tightlist
\item
  \textbf{na.rm:} a logical evaluating to TRUE or FALSE indicating whether NA values should be stripped before the computation proceeds.
\end{itemize}

\phantomsection\label{func.mean_reg.details}
\subparagraph{Details}\label{details-14}

\hfill\break
We call this function internally from the functions that compute the \emph{penalized estimates} as described in section \href{https://rafalab.dfci.harvard.edu/dsbook-part-2/highdim/regularization.html\#penalized-least-squares}{23.6 Penalized Least Squares} of the \emph{Course Textbook}\autocite{IDS2_23-6}.

\begin{noteblock}
If the \hyperref[func.mean_reg.params.lambda]{lambda} parameter is \texttt{0} (the default), the function is equivalent to the standard R function \href{https://www.rdocumentation.org/packages/base/versions/3.6.2/topics/mean}{base::mean}, calling with the parameter \texttt{trim\ =\ 0} (the default for the \texttt{base::mean} function).

\end{noteblock}

\phantomsection\label{func.mean_reg.code}
\subparagraph{Source Code}\label{source-code-33}

\hfill\break

The soure code of the \href{https://github.com/AzKurban-edX-DS/Capstone-MovieLens/blob/main/r/src/support-functions/common-helper.functions.R\#L63}{mean\_reg} function is shown below:

\begin{Shaded}
\begin{Highlighting}[]
\NormalTok{mean\_reg }\OtherTok{\textless{}{-}} \ControlFlowTok{function}\NormalTok{(vals, }\AttributeTok{lambda =} \DecValTok{0}\NormalTok{, }\AttributeTok{na.rm =} \ConstantTok{TRUE}\NormalTok{)\{}
  \ControlFlowTok{if}\NormalTok{ (}\FunctionTok{is.na}\NormalTok{(lambda)) \{}
    \FunctionTok{stop}\NormalTok{(}\StringTok{"Function: mean\_reg}
\StringTok{\textasciigrave{}lambda\textasciigrave{} is \textasciigrave{}NA\textasciigrave{}"}\NormalTok{)}
\NormalTok{  \}}
  
  \FunctionTok{names}\NormalTok{(lambda) }\OtherTok{\textless{}{-}} \ConstantTok{NULL}
\NormalTok{  sums }\OtherTok{\textless{}{-}} \FunctionTok{sum}\NormalTok{(vals, }\AttributeTok{na.rm =}\NormalTok{ na.rm)}
\NormalTok{  N }\OtherTok{\textless{}{-}} \FunctionTok{ifelse}\NormalTok{(na.rm, }\FunctionTok{sum}\NormalTok{(}\SpecialCharTok{!}\FunctionTok{is.na}\NormalTok{(vals)), }\FunctionTok{length}\NormalTok{(vals))}
\NormalTok{  sums}\SpecialCharTok{/}\NormalTok{(N }\SpecialCharTok{+}\NormalTok{ lambda)}
\NormalTok{\}}
\end{Highlighting}
\end{Shaded}

\begin{noteblock}
The source code of the \href{https://github.com/AzKurban-edX-DS/Capstone-MovieLens/blob/main/r/src/support-functions/common-helper.functions.R\#L63}{mean\_reg} is also available in the \href{https://github.com/AzKurban-edX-DS/Capstone-MovieLens/blob/main/r/src/support-functions/common-helper.functions.R\#L62}{Regularization} section of the \href{https://github.com/AzKurban-edX-DS/Capstone-MovieLens/blob/main/r/src/support-functions/common-helper.functions.R\#L63}{common-helper.functions.R} script on \emph{GitHub}.

\end{noteblock}

\phantomsection\label{func.mean_reg.value}
\subparagraph{Value}\label{value-1}

\hfill\break

The function calculates the \emph{penalized mean} of the \hyperref[func.mean_reg.params.vals]{vals} parameter as follows:
\[
\mu(\lambda) = \frac{1}{\lambda + N} \sum_{i=1}^{N} x_i
\]

When the \hyperref[func.mean_reg.params.lambda]{lambda} parameter is \texttt{0} (meaning the \(\lambda = 0\)), the function calculates the simple arithmetic mean as follows:
\[
\mu = \frac{1}{N} \sum_{i=1}^{N} x_i
\]

\paragraph{\texorpdfstring{\href{https://github.com/AzKurban-edX-DS/Capstone-MovieLens/blob/main/r/src/support-functions/common-helper.functions.R\#L116}{tune.model\_param} Function}{tune.model\_param Function}}\label{func.tune.model_param}

\hfill\break

The function searches for the parameter value corresponding to the minimum value of the \texttt{RMSE} from the list of values specified by the \texttt{param\_values} parameter.

\subparagraph{Signature}\label{signature}

\hfill\break

\begin{Shaded}
\begin{Highlighting}[]
\NormalTok{tune.model\_param }\OtherTok{\textless{}{-}} \ControlFlowTok{function}\NormalTok{(param\_values, }
\NormalTok{                             fn\_tune.test.param\_value, }
                             \AttributeTok{break.if\_min =} \ConstantTok{TRUE}\NormalTok{,}
                             \AttributeTok{steps.beyond\_min =} \DecValTok{2}\NormalTok{)\{}

\CommentTok{\# ...}
  \FunctionTok{list}\NormalTok{(}\AttributeTok{tuned.result =} \FunctionTok{data.frame}\NormalTok{(}\AttributeTok{RMSE =}\NormalTok{ RMSEs\_tmp,}
                                \AttributeTok{parameter.value =}\NormalTok{ param\_vals\_tmp),}
       \AttributeTok{best\_result =}\NormalTok{ param\_values.best\_result)}
\NormalTok{\}  }
\end{Highlighting}
\end{Shaded}

\subparagraph{Parameters}\label{parameters-34}

\hfill\break

\begin{itemize}
\tightlist
\item
  \textbf{param\_values:} A list of values to search for the value corresponding to the minimum value of the \texttt{RMSE} ;
\end{itemize}

\phantomsection\label{func.tune.model_param.params.fn_tune.test.param_value}
\begin{itemize}
\tightlist
\item
  \textbf{fn\_tune.test.param\_value:} A helper function that calculates the value of the \texttt{RMSE} for a given parameter value.;
\end{itemize}

\begin{itemize}
\item
  \textbf{break.if\_min = TRUE:} A Boolean parameter that determines whether the function should terminate after completing the number of steps specified by the parameter \texttt{steps.beyond\_min}, after the minimum value of the \texttt{RMSE} has been found;
\item
  \textbf{steps.beyond\_min = 2:} (takes effect only if \texttt{break.if\_min} parameter is \texttt{TRUE}) Specifies the number of steps after finding the minimum value of the \texttt{RMSE}, upon completion of which the function should terminate.
\end{itemize}

\subparagraph{Details}\label{details-15}

\hfill\break
During execution, the function uses a helper function specified by the \texttt{fn\_tune.test.param\_value} parameter, which calculates the \texttt{RMSE} value for the given parameter from the list determined by the \texttt{param\_values} parameter.

\begin{noteblock}
Note that the algorithm assumes that the dependence of the \texttt{RMSE} on the input parameter is a monotonically decreasing function until a minimum is reached and monotonically increasing thereafter. That is, it is assumed that the function has a single minimum on the given interval.

\end{noteblock}

\phantomsection\label{func.tune.model_param.value}
\subparagraph{Value}\label{value-2}

\hfill\break
The function returns a data structure containing the found value of the input parameter \texttt{param\_values} for which the \texttt{RMSE} value is minimal, as well as the minimum \texttt{RMSE} value itself, along with a sequence of all calculated \texttt{RMSE} values:

\begin{Shaded}
\begin{Highlighting}[]
  \FunctionTok{list}\NormalTok{(}\AttributeTok{tuned.result =} \FunctionTok{data.frame}\NormalTok{(}\AttributeTok{RMSE =}\NormalTok{ RMSEs\_tmp,}
                                \AttributeTok{parameter.value =}\NormalTok{ param\_vals\_tmp),}
       \AttributeTok{best\_result =}\NormalTok{ param\_values.best\_result)}
\end{Highlighting}
\end{Shaded}

\subparagraph{Source Code}\label{source-code-34}

\hfill\break
Below is the most significant part of the source code of the \href{https://github.com/AzKurban-edX-DS/Capstone-MovieLens/blob/main/r/src/support-functions/common-helper.functions.R\#L116}{tune.model\_param} function:

\begin{Shaded}
\begin{Highlighting}[]
\NormalTok{tune.model\_param }\OtherTok{\textless{}{-}} \ControlFlowTok{function}\NormalTok{(param\_values, }
\NormalTok{                             fn\_tune.test.param\_value, }
                             \AttributeTok{break.if\_min =} \ConstantTok{TRUE}\NormalTok{,}
                             \AttributeTok{steps.beyond\_min =} \DecValTok{2}\NormalTok{)\{}
\NormalTok{  n }\OtherTok{\textless{}{-}} \FunctionTok{length}\NormalTok{(param\_values)}
\NormalTok{  param\_vals\_tmp }\OtherTok{\textless{}{-}} \FunctionTok{numeric}\NormalTok{()}
\NormalTok{  RMSEs\_tmp }\OtherTok{\textless{}{-}} \FunctionTok{numeric}\NormalTok{()}
\NormalTok{  RMSE\_min }\OtherTok{\textless{}{-}} \ConstantTok{Inf}
\NormalTok{  i\_max.beyond\_RMSE\_min }\OtherTok{\textless{}{-}} \ConstantTok{Inf}
\NormalTok{  prm\_val.best }\OtherTok{\textless{}{-}} \ConstantTok{NA}

  \CommentTok{\# ...}
  
  \ControlFlowTok{for}\NormalTok{ (i }\ControlFlowTok{in} \DecValTok{1}\SpecialCharTok{:}\NormalTok{n) \{}
    \FunctionTok{put\_log1}\NormalTok{(}\StringTok{"Function: \textasciigrave{}tune.model\_param\textasciigrave{}:}
\StringTok{Iteration \%1"}\NormalTok{, i)}
\NormalTok{    prm\_val }\OtherTok{\textless{}{-}}\NormalTok{ param\_values[i]}
\NormalTok{    param\_vals\_tmp[i] }\OtherTok{\textless{}{-}}\NormalTok{ prm\_val}
    
\NormalTok{    RMSE\_tmp }\OtherTok{\textless{}{-}} \FunctionTok{fn\_tune.test.param\_value}\NormalTok{(prm\_val)}
\NormalTok{    RMSEs\_tmp[i] }\OtherTok{\textless{}{-}}\NormalTok{ RMSE\_tmp}

    \FunctionTok{plot}\NormalTok{(param\_vals\_tmp[RMSEs\_tmp], RMSEs\_tmp[RMSEs\_tmp])}

    \ControlFlowTok{if}\NormalTok{(RMSE\_tmp }\SpecialCharTok{\textgreater{}}\NormalTok{ RMSE\_min)\{}
      \FunctionTok{warning}\NormalTok{(}\StringTok{"Function: \textasciigrave{}tune.model\_param\textasciigrave{}:}
\StringTok{\textasciigrave{}RSME\textasciigrave{} reached its minimum: "}\NormalTok{, RMSE\_min, }\StringTok{"}
\StringTok{for parameter value: "}\NormalTok{, prm\_val)}
      \FunctionTok{put\_log2}\NormalTok{(}\StringTok{"Function: \textasciigrave{}tune.model\_param\textasciigrave{}:}
\StringTok{Current \textasciigrave{}RMSE\textasciigrave{} value is \%1 related to parameter value: \%2"}\NormalTok{,}
\NormalTok{               RMSE\_tmp,}
\NormalTok{               prm\_val)}
      
      \ControlFlowTok{if}\NormalTok{ (i }\SpecialCharTok{\textgreater{}}\NormalTok{ i\_max.beyond\_RMSE\_min) \{}
        \FunctionTok{warning}\NormalTok{(}\StringTok{"Function: \textasciigrave{}tune.model\_param\textasciigrave{}:}
\StringTok{Operation is breaked (after \textasciigrave{}RSME\textasciigrave{} reached its minimum) on the following step: "}\NormalTok{, i)}
        \ControlFlowTok{break}
\NormalTok{      \}}
      \ControlFlowTok{next}
\NormalTok{    \}}

\NormalTok{    RMSE\_min }\OtherTok{\textless{}{-}}\NormalTok{ RMSE\_tmp}
\NormalTok{    prm\_val.best }\OtherTok{\textless{}{-}}\NormalTok{ prm\_val}
    
    \ControlFlowTok{if}\NormalTok{ (break.if\_min) \{}
\NormalTok{      i\_max.beyond\_RMSE\_min }\OtherTok{\textless{}{-}}\NormalTok{ i }\SpecialCharTok{+}\NormalTok{ steps.beyond\_min}
\NormalTok{    \}}
\NormalTok{  \}}
  
\NormalTok{  param\_values.best\_result }\OtherTok{\textless{}{-}} \FunctionTok{c}\NormalTok{(}\AttributeTok{param.best\_value =}\NormalTok{ prm\_val.best, }
                                \AttributeTok{best\_RMSE =}\NormalTok{ RMSE\_min)}
  
  
  \FunctionTok{list}\NormalTok{(}\AttributeTok{tuned.result =} \FunctionTok{data.frame}\NormalTok{(}\AttributeTok{RMSE =}\NormalTok{ RMSEs\_tmp,}
                                \AttributeTok{parameter.value =}\NormalTok{ param\_vals\_tmp),}
       \AttributeTok{best\_result =}\NormalTok{ param\_values.best\_result)}
\NormalTok{\}}
\end{Highlighting}
\end{Shaded}

\newpage

\paragraph{\texorpdfstring{\href{https://github.com/AzKurban-edX-DS/Capstone-MovieLens/blob/main/r/src/support-functions/common-helper.functions.R\#L198}{model.tune.param\_range} Function}{model.tune.param\_range Function}}\label{func.model.tune.param_range}

\hfill\break

The function fine-tunes the model by searching for the best possible value of the input parameter over a given interval for which the corresponding \texttt{RMSE} value is minimal.

\subparagraph{Signature}\label{signature-1}

\hfill\break

\begin{Shaded}
\begin{Highlighting}[]
\NormalTok{model.tune.param\_range }\OtherTok{\textless{}{-}} \ControlFlowTok{function}\NormalTok{(loop\_starter,}
\NormalTok{                             tune\_dir\_path,}
\NormalTok{                             cache\_file\_base\_name,}
\NormalTok{                             fn\_tune.test.param\_value,}
                             \AttributeTok{max.identical.min\_RMSE.count =} \DecValTok{4}\NormalTok{,}
                             \AttributeTok{endpoint.min\_diff =} \DecValTok{0}\NormalTok{,}
                             \AttributeTok{break.if\_min =} \ConstantTok{TRUE}\NormalTok{,}
                             \AttributeTok{steps.beyond\_min =} \DecValTok{2}\NormalTok{)\{}
  \CommentTok{\# ...}
  \FunctionTok{list}\NormalTok{(}\AttributeTok{best\_result =}\NormalTok{ param\_values.best\_result,}
       \AttributeTok{param\_values.endpoints =} \FunctionTok{c}\NormalTok{(prm\_val.leftmost, prm\_val.rightmost, seq\_increment),}
       \AttributeTok{tuned.result =} \FunctionTok{data.frame}\NormalTok{(}\AttributeTok{parameter.value =}\NormalTok{ parameter.value,}
                                 \AttributeTok{RMSE =}\NormalTok{ result.RMSE))}
\NormalTok{\}}
\end{Highlighting}
\end{Shaded}

\subparagraph{\texorpdfstring{\textbf{Parameters}}{Parameters}}\label{parameters-35}

\hfill\break

\phantomsection\label{func.model.tune.param_range.params.loop_starter}
\textbf{\emph{loop\_starter}}

\hfill\break
A numeric vector of the form \texttt{c(start,\ end,\ dvs)}, where \texttt{start} and \texttt{end} are the endpoints of the interval on which the parameter value that minimizes \texttt{RMSE} is sought.
\texttt{dvs} is a divisor for splitting the interval to transform it into a sequence of values among which the value that minimizes \texttt{RMSE} is sought.
For this purpose, the sequence step is calculated as follows:
\[
step = \frac{end - start}{dvs}
\]

The sequence obtained as a result of the transformation is equivalent to the one generated by the function \texttt{seq} as follows:

\begin{Shaded}
\begin{Highlighting}[]
\FunctionTok{seq}\NormalTok{(start, end, step)}
\end{Highlighting}
\end{Shaded}

In fact, the \texttt{seq} function is called internally to generate the sequence during the execution of the \texttt{model.tune.param\_range} function.

\phantomsection\label{func.model.tune.param_range.params.tune_dir_path}
\textbf{\emph{tune\_dir\_path}}

\hfill\break
To improve performance, the algorithm caches intermediate results in the file system. This parameter specifies the path to the directory where the files are cached.

\phantomsection\label{func.model.tune.param_range.params.cache_file_base_name}
\textbf{\emph{cache\_file\_base\_name}}

\hfill\break
The algorithm generates unique names for cache files based on this and the \texttt{loop\_starter} parameter, as well as some other intermediate values calculated during the execution.

\phantomsection\label{func.model.tune.param_range.params.fn_tune.test.param_value}
\textbf{\emph{fn\_tune.test.param\_value}}

\hfill\break
This is a helper function name that is passed to the same-named parameter of the \href{https://github.com/AzKurban-edX-DS/Capstone-MovieLens/blob/main/r/src/support-functions/common-helper.functions.R\#L116}{tune.model\_param} function that is called internally during the execution (see the description of the \texttt{tune.model\_param} function \hyperref[func.tune.model_param]{above}).

\phantomsection\label{func.model.tune.param_range.params.max.identical.min_RMSE.count}
\textbf{\emph{max.identical.min\_RMSE.count = 4}}

\hfill\break
If more than one identical minimum \texttt{RMSE} value is calculated during execution, the number of identical minimums is limited by the value of this parameter. When it is reached, the algorithm considers the task execution to be complete.

\phantomsection\label{func.model.tune.param_range.params.endpoint.min_diff}
\textbf{\emph{endpoint.min\_diff = 0}}

\hfill\break
Defines the sensitivity threshold for determining the neighborhood boundaries of the minimum \texttt{RMSE} value (for details, see the \texttt{Details} section \hyperref[func.model.tune.param_range.details]{below}).

\phantomsection\label{func.model.tune.param_range.params.break.if_min}
\textbf{\emph{break.if\_min = TRUE}}

\hfill\break
This is a parameter that is required for the \href{https://github.com/AzKurban-edX-DS/Capstone-MovieLens/blob/main/r/src/support-functions/common-helper.functions.R\#L116}{tune.model\_param} function that is called internally during execution (see the description of the \texttt{tune.model\_param} function \hyperref[func.tune.model_param]{above}).

\phantomsection\label{func.model.tune.param_range.params.steps.beyond_min}
\textbf{\emph{steps.beyond\_min = 2}}

\hfill\break
This is a parameter that is required for the \href{https://github.com/AzKurban-edX-DS/Capstone-MovieLens/blob/main/r/src/support-functions/common-helper.functions.R\#L116}{tune.model\_param} function that is called internally during execution (see the description of the \texttt{tune.model\_param} function \hyperref[func.tune.model_param]{above}).

\phantomsection\label{func.model.tune.param_range.details}
\subparagraph{\texorpdfstring{\textbf{Details}}{Details}}\label{details-16}

\hfill\break
First, the function generates a sequence of the \texttt{input\ parameter\ values} based on the \texttt{loop\_starter} parameter values, as described above (see the \hyperref[func.model.tune.param_range.params.loop_starter]{loop\_starter} parameter description for the details), which is used as one of the input parameters for the helper function \hyperref[func.tune.model_param]{tune.model\_param}, which is repeatedly called during the execution, performing fine-tuning of the model.

The \texttt{tune.model\_param} function returns a range of input parameter values associated with the set of corresponding \emph{Root Mean Squared Errors} that is also guaranteed to include their minimum value, as described in the \hyperref[func.tune.model_param.value]{Value} subsection of the \hyperref[func.tune.model_param]{tune.model\_param} function description.

Next, the function figures out the boundary indices of a range of values from the neighborhood of the minimum \texttt{RMSE} by calling internally another helper function \hyperref[func.get_fine_tune.param.endpoints.idx]{get\_fine\_tune.param.endpoints.idx} (see the description below) and, using one more helper function \hyperref[func.get_best_param.result]{get\_best\_param.result} (see the description below), a pair of values, corresponding to the best result: minimal \texttt{RMSE} and corresponding input parameter value (which is considered the best).

The found values of the boundary indices are then used as the endpoints of a new interval to regenerate the sequence based on it in the next iteration, just as it was done based on the values of the \texttt{loop\_starter} parameter at the very beginning of the execution. The value of \texttt{step\ divisor}, used to calculate the step of the generated sequence, remains unchanged during the entire execution (see the description of the \hyperref[func.model.tune.param_range.params.loop_starter]{loop\_strarter} parameter above for details).

Thus, with each subsequent iteration, the minimum \texttt{RMSE} value is calculated more accurately, over an ever-decreasing interval, with the boundary values tending to the minimum \texttt{RMSE} value, and the sequence generated with an ever-decreasing step.

The calculation is completed when subsequent calculated values of the minimum \texttt{RMSE} stop improving and reach the most accurate possible value.

\subparagraph{\texorpdfstring{\textbf{Value}}{Value}}\label{value-3}

\hfill\break

A data structure containing the minimum \texttt{RMSE} value reached during the fine-tuning process, the corresponding input parameter value, and information about the final sequence, on which the best output values were found:

\begin{Shaded}
\begin{Highlighting}[]
  \FunctionTok{list}\NormalTok{(}\AttributeTok{best\_result =}\NormalTok{ param\_values.best\_result,}
       \AttributeTok{param\_values.endpoints =} \FunctionTok{c}\NormalTok{(prm\_val.leftmost, prm\_val.rightmost, seq\_increment),}
       \AttributeTok{tuned.result =} \FunctionTok{data.frame}\NormalTok{(}\AttributeTok{parameter.value =}\NormalTok{ parameter.value,}
                                 \AttributeTok{RMSE =}\NormalTok{ result.RMSE))}
\end{Highlighting}
\end{Shaded}

\phantomsection\label{func.model.tune.param_range.code}
\subparagraph{Source Code}\label{source-code-35}

\hfill\break
Below is the simplified version of the source code of the \href{https://github.com/AzKurban-edX-DS/Capstone-MovieLens/blob/main/r/src/support-functions/common-helper.functions.R\#L198}{model.tune.param\_range} function:

\begin{Shaded}
\begin{Highlighting}[]
\NormalTok{model.tune.param\_range }\OtherTok{\textless{}{-}} \ControlFlowTok{function}\NormalTok{(loop\_starter,}
\NormalTok{                             tune\_dir\_path,}
\NormalTok{                             cache\_file\_base\_name,}
\NormalTok{                             fn\_tune.test.param\_value,}
                             \AttributeTok{max.identical.min\_RMSE.count =} \DecValTok{4}\NormalTok{,}
                             \AttributeTok{is.cv =} \ConstantTok{TRUE}\NormalTok{,}
                             \AttributeTok{endpoint.min\_diff =} \DecValTok{0}\NormalTok{, }\CommentTok{\#1e{-}07,}
                             \AttributeTok{break.if\_min =} \ConstantTok{TRUE}\NormalTok{,}
                             \AttributeTok{steps.beyond\_min =} \DecValTok{2}\NormalTok{)\{}

\NormalTok{  seq\_start }\OtherTok{\textless{}{-}}\NormalTok{ loop\_starter[}\DecValTok{1}\NormalTok{]}
\NormalTok{  seq\_end }\OtherTok{\textless{}{-}}\NormalTok{ loop\_starter[}\DecValTok{2}\NormalTok{]}
\NormalTok{  interval\_divisor }\OtherTok{\textless{}{-}}\NormalTok{ loop\_starter[}\DecValTok{3}\NormalTok{]}
  
  \ControlFlowTok{if}\NormalTok{ (interval\_divisor }\SpecialCharTok{\textless{}} \DecValTok{4}\NormalTok{) \{}
\NormalTok{    interval\_divisor }\OtherTok{\textless{}{-}} \DecValTok{4}
\NormalTok{  \}}
  
\NormalTok{  prm\_val.leftmost }\OtherTok{\textless{}{-}}\NormalTok{ seq\_start}
\NormalTok{  prm\_val.rightmost }\OtherTok{\textless{}{-}}\NormalTok{ seq\_end}
  
\NormalTok{  RMSE.leftmost }\OtherTok{\textless{}{-}} \ConstantTok{NA}
\NormalTok{  RMSE.rightmost }\OtherTok{\textless{}{-}} \ConstantTok{NA}

\NormalTok{  best\_RMSE }\OtherTok{\textless{}{-}} \ConstantTok{NA}
\NormalTok{  param.best\_value }\OtherTok{\textless{}{-}} \DecValTok{0}
  
  
\NormalTok{  param\_values.best\_result }\OtherTok{\textless{}{-}} \FunctionTok{c}\NormalTok{(}\AttributeTok{param.best\_value =}\NormalTok{ param.best\_value, }
                                \AttributeTok{best\_RMSE =}\NormalTok{ best\_RMSE)}
  \CommentTok{\# Start repeat loop}
  \ControlFlowTok{repeat}\NormalTok{\{}
\NormalTok{    seq\_increment }\OtherTok{\textless{}{-}}\NormalTok{ (seq\_end }\SpecialCharTok{{-}}\NormalTok{ seq\_start)}\SpecialCharTok{/}\NormalTok{interval\_divisor }
    
    \ControlFlowTok{if}\NormalTok{ (seq\_increment }\SpecialCharTok{\textless{}} \FloatTok{0.0000000000001}\NormalTok{) \{}
      \FunctionTok{warning}\NormalTok{(}\StringTok{"Function \textasciigrave{}model.tune.param\_range\textasciigrave{}:}
\StringTok{parameter value increment is too small."}\NormalTok{)}
      \ControlFlowTok{break}
\NormalTok{    \}}
    
\NormalTok{    test\_param\_vals }\OtherTok{\textless{}{-}} \FunctionTok{seq}\NormalTok{(seq\_start, seq\_end, seq\_increment)}
    
\NormalTok{    tuned\_result }\OtherTok{\textless{}{-}} \FunctionTok{tune.model\_param}\NormalTok{(test\_param\_vals, }
\NormalTok{                                  fn\_tune.test.param\_value,}
\NormalTok{                                  break.if\_min,}
\NormalTok{                                  steps.beyond\_min)}
    
\NormalTok{    tuned.result }\OtherTok{\textless{}{-}}\NormalTok{ tuned\_result}\SpecialCharTok{$}\NormalTok{tuned.result}
    \FunctionTok{plot}\NormalTok{(tuned.result}\SpecialCharTok{$}\NormalTok{parameter.value, tuned.result}\SpecialCharTok{$}\NormalTok{RMSE)}
    
\NormalTok{    bound.idx }\OtherTok{\textless{}{-}} \FunctionTok{get\_fine\_tune.param.endpoints.idx}\NormalTok{(tuned.result)}
\NormalTok{    start.idx }\OtherTok{\textless{}{-}}\NormalTok{ bound.idx[}\StringTok{"start"}\NormalTok{]}
\NormalTok{    end.idx }\OtherTok{\textless{}{-}}\NormalTok{ bound.idx[}\StringTok{"end"}\NormalTok{]}
\NormalTok{    best\_RMSE.idx }\OtherTok{\textless{}{-}}\NormalTok{ bound.idx[}\StringTok{"best"}\NormalTok{]}
    
\NormalTok{    prm\_val.leftmost.tmp }\OtherTok{\textless{}{-}}\NormalTok{ tuned.result}\SpecialCharTok{$}\NormalTok{parameter.value[start.idx]}
\NormalTok{    RMSE.leftmost.tmp }\OtherTok{\textless{}{-}}\NormalTok{ tuned.result}\SpecialCharTok{$}\NormalTok{RMSE[start.idx]}

\NormalTok{    prm\_val.rightmost.tmp }\OtherTok{\textless{}{-}}\NormalTok{ tuned.result}\SpecialCharTok{$}\NormalTok{parameter.value[end.idx]}
\NormalTok{    RMSE.rightmost.tmp }\OtherTok{\textless{}{-}}\NormalTok{ tuned.result}\SpecialCharTok{$}\NormalTok{RMSE[end.idx]}
    
\NormalTok{    min\_RMSE }\OtherTok{\textless{}{-}}\NormalTok{ tuned.result}\SpecialCharTok{$}\NormalTok{RMSE[best\_RMSE.idx]}
\NormalTok{    min\_RMSE.prm\_val }\OtherTok{\textless{}{-}}\NormalTok{ tuned.result}\SpecialCharTok{$}\NormalTok{parameter.value[best\_RMSE.idx]}

\NormalTok{    seq\_start }\OtherTok{\textless{}{-}}\NormalTok{ prm\_val.leftmost.tmp}
\NormalTok{    seq\_end }\OtherTok{\textless{}{-}}\NormalTok{ prm\_val.rightmost.tmp}
    
    \ControlFlowTok{if}\NormalTok{ (}\FunctionTok{is.na}\NormalTok{(best\_RMSE)) \{}
\NormalTok{      prm\_val.leftmost }\OtherTok{\textless{}{-}}\NormalTok{ prm\_val.leftmost.tmp}
\NormalTok{      RMSE.leftmost }\OtherTok{\textless{}{-}}\NormalTok{ RMSE.leftmost.tmp}
      
\NormalTok{      prm\_val.rightmost }\OtherTok{\textless{}{-}}\NormalTok{ prm\_val.rightmost.tmp}
\NormalTok{      RMSE.rightmost }\OtherTok{\textless{}{-}}\NormalTok{ RMSE.rightmost.tmp}
      
\NormalTok{      param.best\_value }\OtherTok{\textless{}{-}}\NormalTok{ min\_RMSE.prm\_val}
\NormalTok{      best\_RMSE }\OtherTok{\textless{}{-}}\NormalTok{ min\_RMSE}
\NormalTok{    \}}
    
    \ControlFlowTok{if}\NormalTok{ (RMSE.leftmost.tmp }\SpecialCharTok{{-}}\NormalTok{ min\_RMSE }\SpecialCharTok{\textgreater{}=}\NormalTok{ endpoint.min\_diff) \{}
\NormalTok{      prm\_val.leftmost }\OtherTok{\textless{}{-}}\NormalTok{ prm\_val.leftmost.tmp}
\NormalTok{      RMSE.leftmost }\OtherTok{\textless{}{-}}\NormalTok{ RMSE.leftmost.tmp}
\NormalTok{    \} }
    
    \ControlFlowTok{if}\NormalTok{ (RMSE.rightmost.tmp }\SpecialCharTok{{-}}\NormalTok{ min\_RMSE }\SpecialCharTok{\textgreater{}=}\NormalTok{ endpoint.min\_diff) \{}
\NormalTok{      prm\_val.rightmost }\OtherTok{\textless{}{-}}\NormalTok{ prm\_val.rightmost.tmp}
\NormalTok{      RMSE.rightmost }\OtherTok{\textless{}{-}}\NormalTok{ RMSE.rightmost.tmp}
\NormalTok{    \} }
    
    \ControlFlowTok{if}\NormalTok{ (end.idx }\SpecialCharTok{{-}}\NormalTok{ start.idx }\SpecialCharTok{\textless{}=} \DecValTok{0}\NormalTok{) \{}
      \FunctionTok{warning}\NormalTok{(}\StringTok{"\textasciigrave{}tuned.result$parameter.value\textasciigrave{} sequential start index are the same or greater than end one."}\NormalTok{)}
      \ControlFlowTok{break}
\NormalTok{    \}}
    
    \ControlFlowTok{if}\NormalTok{ (best\_RMSE }\SpecialCharTok{==}\NormalTok{ min\_RMSE) \{}
      \FunctionTok{warning}\NormalTok{(}\StringTok{"Currently computed minimal RMSE equals the previously reached best one: "}\NormalTok{,}
\NormalTok{              best\_RMSE, }\StringTok{"}
\StringTok{Currently computed minial value is: "}\NormalTok{, min\_RMSE)}

      \ControlFlowTok{if}\NormalTok{ (}\FunctionTok{sum}\NormalTok{(tuned.result}\SpecialCharTok{$}\NormalTok{RMSE[tuned.result}\SpecialCharTok{$}\NormalTok{RMSE }\SpecialCharTok{==}\NormalTok{ min\_RMSE]) }\SpecialCharTok{\textgreater{}=}\NormalTok{ max.identical.min\_RMSE.count) \{}
        \FunctionTok{warning}\NormalTok{(}\StringTok{"Minimal \textasciigrave{}RMSE\textasciigrave{}identical values count reached it maximum allowed value: "}\NormalTok{,}
\NormalTok{                max.identical.min\_RMSE.count)}

\NormalTok{        param\_values.best\_result }\OtherTok{\textless{}{-}}
          \FunctionTok{get\_best\_param.result}\NormalTok{(tuned.result}\SpecialCharTok{$}\NormalTok{parameter.value,}
\NormalTok{                                tuned.result}\SpecialCharTok{$}\NormalTok{RMSE)}
        \ControlFlowTok{break}
\NormalTok{      \}}

\NormalTok{    \} }\ControlFlowTok{else} \ControlFlowTok{if}\NormalTok{ (best\_RMSE }\SpecialCharTok{\textless{}}\NormalTok{ min\_RMSE) \{}
      \FunctionTok{stop}\NormalTok{(}\StringTok{"Current minimal RMSE is greater than previously computed best value: "}\NormalTok{,}
\NormalTok{           best\_RMSE, }\StringTok{"}
\StringTok{Currently computed minial value is: "}\NormalTok{, min\_RMSE)}
\NormalTok{    \}}

\NormalTok{    best\_RMSE }\OtherTok{\textless{}{-}}\NormalTok{ min\_RMSE}
\NormalTok{    param.best\_value }\OtherTok{\textless{}{-}}\NormalTok{ min\_RMSE.prm\_val}

\NormalTok{    param\_values.best\_result }\OtherTok{\textless{}{-}} 
      \FunctionTok{get\_best\_param.result}\NormalTok{(tuned.result}\SpecialCharTok{$}\NormalTok{parameter.value, }
\NormalTok{                          tuned.result}\SpecialCharTok{$}\NormalTok{RMSE)}
\NormalTok{  \}}
  \CommentTok{\# End repeat loop}
  
\NormalTok{  n }\OtherTok{\textless{}{-}} \FunctionTok{length}\NormalTok{(tuned.result}\SpecialCharTok{$}\NormalTok{parameter.value)}
\NormalTok{  parameter.value }\OtherTok{\textless{}{-}}\NormalTok{ tuned.result}\SpecialCharTok{$}\NormalTok{parameter.value}
\NormalTok{  result.RMSE }\OtherTok{\textless{}{-}}\NormalTok{ tuned.result}\SpecialCharTok{$}\NormalTok{RMSE}
  
  \ControlFlowTok{if}\NormalTok{ (result.RMSE[}\DecValTok{1}\NormalTok{] }\SpecialCharTok{==}\NormalTok{ best\_RMSE) \{}
\NormalTok{    parameter.value[}\DecValTok{1}\NormalTok{] }\OtherTok{\textless{}{-}}\NormalTok{ prm\_val.leftmost}
\NormalTok{    result.RMSE[}\DecValTok{1}\NormalTok{] }\OtherTok{\textless{}{-}}\NormalTok{ RMSE.leftmost}
\NormalTok{  \}}
  \ControlFlowTok{if}\NormalTok{ (result.RMSE[n] }\SpecialCharTok{==}\NormalTok{ best\_RMSE) \{}
\NormalTok{    parameter.value[n}\SpecialCharTok{+}\DecValTok{1}\NormalTok{] }\OtherTok{\textless{}{-}}\NormalTok{ prm\_val.rightmost}
\NormalTok{    result.RMSE[n}\SpecialCharTok{+}\DecValTok{1}\NormalTok{] }\OtherTok{\textless{}{-}}\NormalTok{ RMSE.rightmost}
\NormalTok{  \}}
  
  \FunctionTok{list}\NormalTok{(}\AttributeTok{best\_result =}\NormalTok{ param\_values.best\_result,}
       \AttributeTok{param\_values.endpoints =} \FunctionTok{c}\NormalTok{(prm\_val.leftmost, prm\_val.rightmost, seq\_increment),}
       \AttributeTok{tuned.result =} \FunctionTok{data.frame}\NormalTok{(}\AttributeTok{parameter.value =}\NormalTok{ parameter.value,}
                                 \AttributeTok{RMSE =}\NormalTok{ result.RMSE))}
\NormalTok{\}}
\end{Highlighting}
\end{Shaded}

\begin{noteblock}
The complete version of the source code of the \href{https://github.com/AzKurban-edX-DS/Capstone-MovieLens/blob/main/r/src/support-functions/common-helper.functions.R\#L198}{model.tune.param\_range} is available in the \href{https://github.com/AzKurban-edX-DS/Capstone-MovieLens/blob/main/r/src/support-functions/common-helper.functions.R\#L74}{Model Tuning} section of the \href{https://github.com/AzKurban-edX-DS/Capstone-MovieLens/blob/main/r/src/support-functions/common-helper.functions.R\#L198}{common-helper.functions.R} script on \emph{GitHub}.

\end{noteblock}

\newpage

\paragraph{\texorpdfstring{\href{https://github.com/AzKurban-edX-DS/Capstone-MovieLens/blob/main/r/src/support-functions/common-helper.functions.R\#L80}{get\_fine\_tune.param.endpoints} Function}{get\_fine\_tune.param.endpoints Function}}\label{func.get_fine_tune.param.endpoints}

\hfill\break

The algorithm finds the values of the nearest neighboring elements of the element corresponding to the minimum RMSE value in the given sequence.

\subparagraph{Signature}\label{signature-2}

\hfill\break

\begin{Shaded}
\begin{Highlighting}[]
\NormalTok{get\_fine\_tune.param.endpoints }\OtherTok{\textless{}{-}} \ControlFlowTok{function}\NormalTok{(preset.result) \{}
  \CommentTok{\# ...}
  
  \FunctionTok{c}\NormalTok{(}\AttributeTok{start =}\NormalTok{ preset.result}\SpecialCharTok{$}\NormalTok{parameter.value[i], }
    \AttributeTok{end =}\NormalTok{ preset.result}\SpecialCharTok{$}\NormalTok{parameter.value[j],}
    \AttributeTok{best =}\NormalTok{ preset.result}\SpecialCharTok{$}\NormalTok{parameter.value[best.idx])}
\NormalTok{\}}
\end{Highlighting}
\end{Shaded}

\subparagraph{Parameters}\label{parameters-36}

\hfill\break

\phantomsection\label{func.get_fine_tune.param.endpoints.params.preset.result}
\textbf{\emph{preset.result}}

\hfill\break
A data structure compatible with the \texttt{tuned.result} item of the data structure returned by the \hyperref[func.tune.model_param]{tune.model\_param} function (see the \hyperref[func.tune.model_param.value]{Value} subsection of the \texttt{tune.model\_param} function description section for more details). Here is a sample of using this parameter in the \texttt{model.tune.param\_range} function \hyperref[func.model.tune.param_range.code]{code}:

\begin{Shaded}
\begin{Highlighting}[]
\NormalTok{    tuned\_result }\OtherTok{\textless{}{-}} \FunctionTok{tune.model\_param}\NormalTok{(test\_param\_vals, }
\NormalTok{                                  fn\_tune.test.param\_value,}
\NormalTok{                                  break.if\_min,}
\NormalTok{                                  steps.beyond\_min)}
    
\NormalTok{    tuned.result }\OtherTok{\textless{}{-}}\NormalTok{ tuned\_result}\SpecialCharTok{$}\NormalTok{tuned.result}
    \CommentTok{\# tuned.result = data.frame(RMSE, parameter.value)}
    \CommentTok{\# ...}
\NormalTok{    bound.idx }\OtherTok{\textless{}{-}} \FunctionTok{get\_fine\_tune.param.endpoints}\NormalTok{(tuned.result)}
    \CommentTok{\# bounnd.idx =  c(start.index, end.index, min\_RMSE.index)}
\end{Highlighting}
\end{Shaded}

\subparagraph{Details}\label{details-17}

\hfill\break

Under the hood, the function internally calls another helper function \hyperref[func.get_fine_tune.param.endpoints.idx]{get\_fine\_tune.param.endpoints.idx} described below to figure out the boundary indices of the range of values from the nearest neighborhood of the value corresponding to the best \texttt{RMSE} in the data passed in the \hyperref[func.get_fine_tune.param.endpoints.params.preset.result]{preset.result} parameter.

\subparagraph{Value}\label{value-4}

\hfill\break
A vector containing the element value corresponding to the minimum \texttt{RMSE} in a given sequence, along with the values of its nearest neighboring elements:

\begin{Shaded}
\begin{Highlighting}[]
  \FunctionTok{c}\NormalTok{(}\AttributeTok{start =}\NormalTok{ preset.result}\SpecialCharTok{$}\NormalTok{parameter.value[i], }
    \AttributeTok{end =}\NormalTok{ preset.result}\SpecialCharTok{$}\NormalTok{parameter.value[j],}
    \AttributeTok{best =}\NormalTok{ preset.result}\SpecialCharTok{$}\NormalTok{parameter.value[best.idx])}
\end{Highlighting}
\end{Shaded}

\subparagraph{Source Code}\label{source-code-36}

\hfill\break
The source code of the \href{https://github.com/AzKurban-edX-DS/Capstone-MovieLens/blob/main/r/src/support-functions/common-helper.functions.R\#L80}{get\_fine\_tune.param.endpoints} function is provided below:

\begin{Shaded}
\begin{Highlighting}[]
\NormalTok{get\_fine\_tune.param.endpoints }\OtherTok{\textless{}{-}} \ControlFlowTok{function}\NormalTok{(preset.result) \{}

\NormalTok{  preset.result.idx }\OtherTok{\textless{}{-}} \FunctionTok{get\_fine\_tune.param.endpoints.idx}\NormalTok{(preset.result)}
  
\NormalTok{  i }\OtherTok{\textless{}{-}}\NormalTok{ preset.result.idx[}\StringTok{"start"}\NormalTok{]}
\NormalTok{  j }\OtherTok{\textless{}{-}}\NormalTok{ preset.result.idx[}\StringTok{"end"}\NormalTok{]}
\NormalTok{  best.idx }\OtherTok{\textless{}{-}}\NormalTok{ preset.result.idx[}\StringTok{"best"}\NormalTok{]}
  
  \FunctionTok{c}\NormalTok{(}\AttributeTok{start =}\NormalTok{ preset.result}\SpecialCharTok{$}\NormalTok{parameter.value[i], }
    \AttributeTok{end =}\NormalTok{ preset.result}\SpecialCharTok{$}\NormalTok{parameter.value[j],}
    \AttributeTok{best =}\NormalTok{ preset.result}\SpecialCharTok{$}\NormalTok{parameter.value[best.idx])}
\NormalTok{\}}
\end{Highlighting}
\end{Shaded}

\newpage

\paragraph{\texorpdfstring{\href{https://github.com/AzKurban-edX-DS/Capstone-MovieLens/blob/main/r/src/support-functions/common-helper.functions.R\#L87}{get\_fine\_tune.param.endpoints.idx} Function}{get\_fine\_tune.param.endpoints.idx Function}}\label{func.get_fine_tune.param.endpoints.idx}

\hfill\break

\subparagraph{Signature}\label{signature-3}

\hfill\break

\begin{Shaded}
\begin{Highlighting}[]
\NormalTok{get\_fine\_tune.param.endpoints.idx }\OtherTok{\textless{}{-}} \ControlFlowTok{function}\NormalTok{(preset.result) \{}
  \CommentTok{\# ...}
    
  \FunctionTok{c}\NormalTok{(}\AttributeTok{start =}\NormalTok{ i, }
    \AttributeTok{end =}\NormalTok{ j,}
    \AttributeTok{best =}\NormalTok{ best\_RMSE.idx)}
\NormalTok{\}}
\end{Highlighting}
\end{Shaded}

\subparagraph{Parameters}\label{parameters-37}

\hfill\break

\phantomsection\label{func.get_fine_tune.param.endpoints.idx.params.preset.result}
\textbf{\emph{preset.result}}

\hfill\break
A data structure compatible with the \texttt{tuned.result} item of the data structure returned by the \hyperref[func.tune.model_param]{tune.model\_param} function (see the \hyperref[func.tune.model_param.value]{Value} subsection of the \texttt{tune.model\_param} function description section for more details). Here is a sample of using this parameter in the \texttt{model.tune.param\_range} function \hyperref[func.model.tune.param_range.code]{code}:

\begin{Shaded}
\begin{Highlighting}[]
\NormalTok{    tuned\_result }\OtherTok{\textless{}{-}} \FunctionTok{tune.model\_param}\NormalTok{(test\_param\_vals, }
\NormalTok{                                  fn\_tune.test.param\_value,}
\NormalTok{                                  break.if\_min,}
\NormalTok{                                  steps.beyond\_min)}
    
\NormalTok{    tuned.result }\OtherTok{\textless{}{-}}\NormalTok{ tuned\_result}\SpecialCharTok{$}\NormalTok{tuned.result}
    \CommentTok{\# tuned.result = data.frame(RMSE, parameter.value)}
    \CommentTok{\# ...}
\NormalTok{    bound.idx }\OtherTok{\textless{}{-}} \FunctionTok{get\_fine\_tune.param.endpoints.idx}\NormalTok{(tuned.result)}
    \CommentTok{\# bounnd.idx =  c(start.index, end.index, min\_RMSE.index)}
\end{Highlighting}
\end{Shaded}

\subparagraph{Details}\label{details-18}

\hfill\break
The algorithm finds the indices of the nearest neighboring elements of the element corresponding to the minimum RMSE value in the given sequence.

\subparagraph{Value}\label{value-5}

\hfill\break
A vector containing the index of the element corresponding to the minimum \texttt{RMSE} value in a given sequence, along with the indices of its nearest neighboring elements:

\begin{Shaded}
\begin{Highlighting}[]
  \FunctionTok{c}\NormalTok{(start, }\CommentTok{\# interval start index}
\NormalTok{    end,   }\CommentTok{\# interval end index}
\NormalTok{    best)  }\CommentTok{\# index of the best \textasciigrave{}RMSE\textasciigrave{}}
\end{Highlighting}
\end{Shaded}

\subparagraph{Source Code}\label{source-code-37}

\hfill\break
The source code of the \href{https://github.com/AzKurban-edX-DS/Capstone-MovieLens/blob/main/r/src/support-functions/common-helper.functions.R\#L87}{get\_fine\_tune.param.endpoints.idx} function is shown below:

\begin{Shaded}
\begin{Highlighting}[]
\NormalTok{get\_fine\_tune.param.endpoints.idx }\OtherTok{\textless{}{-}} \ControlFlowTok{function}\NormalTok{(preset.result) \{}
\NormalTok{  best\_RMSE }\OtherTok{\textless{}{-}} \FunctionTok{min}\NormalTok{(preset.result}\SpecialCharTok{$}\NormalTok{RMSE)}
\NormalTok{  best\_RMSE.idx }\OtherTok{\textless{}{-}} \FunctionTok{which.min}\NormalTok{(preset.result}\SpecialCharTok{$}\NormalTok{RMSE)}
  \CommentTok{\# best\_lambda \textless{}{-} preset.result$parameter.value[best\_RMSE.idx]}
  
\NormalTok{  preset.result.N }\OtherTok{\textless{}{-}} \FunctionTok{length}\NormalTok{(preset.result}\SpecialCharTok{$}\NormalTok{RMSE)}
\NormalTok{  i }\OtherTok{\textless{}{-}}\NormalTok{ best\_RMSE.idx}
\NormalTok{  j }\OtherTok{\textless{}{-}}\NormalTok{ i}
  
  \ControlFlowTok{while}\NormalTok{ (i }\SpecialCharTok{\textgreater{}} \DecValTok{1}\NormalTok{) \{}
\NormalTok{    i }\OtherTok{\textless{}{-}}\NormalTok{ i }\SpecialCharTok{{-}} \DecValTok{1}
    
    \ControlFlowTok{if}\NormalTok{ (preset.result}\SpecialCharTok{$}\NormalTok{RMSE[i] }\SpecialCharTok{\textgreater{}}\NormalTok{ best\_RMSE) \{}
      \ControlFlowTok{break}
\NormalTok{    \}}
\NormalTok{  \}}
  
  \ControlFlowTok{while}\NormalTok{ (j }\SpecialCharTok{\textless{}}\NormalTok{ preset.result.N) \{}
\NormalTok{    j }\OtherTok{\textless{}{-}}\NormalTok{ j }\SpecialCharTok{+} \DecValTok{1}
    
    \ControlFlowTok{if}\NormalTok{ (preset.result}\SpecialCharTok{$}\NormalTok{RMSE[j] }\SpecialCharTok{\textgreater{}}\NormalTok{ best\_RMSE) \{}
      \ControlFlowTok{break}
\NormalTok{    \}}
\NormalTok{  \}}
  
  \FunctionTok{c}\NormalTok{(}\AttributeTok{start =}\NormalTok{ i, }
    \AttributeTok{end =}\NormalTok{ j,}
    \AttributeTok{best =}\NormalTok{ best\_RMSE.idx)}
\NormalTok{\}}
\end{Highlighting}
\end{Shaded}

\newpage

\paragraph{\texorpdfstring{\href{https://github.com/AzKurban-edX-DS/Capstone-MovieLens/blob/main/r/src/support-functions/data.helper.functions.R\#L442}{get\_best\_param.result} Function}{get\_best\_param.result Function}}\label{func.get_best_param.result}

\hfill\break

\subparagraph{Signature}\label{signature-4}

\hfill\break

\begin{Shaded}
\begin{Highlighting}[]
\NormalTok{get\_best\_param.result }\OtherTok{\textless{}{-}} \ControlFlowTok{function}\NormalTok{(param\_values, rmses)\{}
  \CommentTok{\# ...}
  
  \FunctionTok{c}\NormalTok{(}\AttributeTok{param.best\_value =}\NormalTok{ param\_values[best\_pvalue\_idx], }
    \AttributeTok{best\_RMSE =}\NormalTok{ rmses[best\_pvalue\_idx])}
\end{Highlighting}
\end{Shaded}

\subparagraph{Parameters}\label{parameters-38}

\hfill\break
A pair of matched vectors, the first of which contains the values of the input parameters for tuning, and the second, the corresponding values of the \texttt{RMSE}.

\begin{itemize}
\tightlist
\item
  \textbf{param\_values:} A vector containing the values of the input parameters for tuning;;
\item
  \textbf{rmses:} A vector containing values of the corresponding \texttt{Root\ Mean\ Squared\ Errors}..
\end{itemize}

\subparagraph{Details}\label{details-19}

\hfill\break
Extracts the best values from the input data and returns them as a vector containing a pair of the best \emph{input parameter} value and the corresponding \emph{minimum \texttt{RMSE}}.

\subparagraph{Value}\label{value-6}

\hfill\break
A vector containing a pair of the best \emph{input parameter} value and the corresponding \emph{minimum \texttt{RMSE}}:

\begin{Shaded}
\begin{Highlighting}[]
\FunctionTok{c}\NormalTok{(param.best\_value, best\_RMSE)}
\end{Highlighting}
\end{Shaded}

\subparagraph{Source Code}\label{source-code-38}

\hfill\break
The source code of the \href{https://github.com/AzKurban-edX-DS/Capstone-MovieLens/blob/main/r/src/support-functions/data.helper.functions.R\#L442}{get\_best\_param.result} function is shown below:

\begin{Shaded}
\begin{Highlighting}[]
\NormalTok{get\_best\_param.result }\OtherTok{\textless{}{-}} \ControlFlowTok{function}\NormalTok{(param\_values, rmses)\{}
\NormalTok{  best\_pvalue\_idx }\OtherTok{\textless{}{-}} \FunctionTok{which.min}\NormalTok{(rmses)}
  \FunctionTok{c}\NormalTok{(}\AttributeTok{param.best\_value =}\NormalTok{ param\_values[best\_pvalue\_idx], }
    \AttributeTok{best\_RMSE =}\NormalTok{ rmses[best\_pvalue\_idx])}
\end{Highlighting}
\end{Shaded}


\printbibliography

\end{document}
