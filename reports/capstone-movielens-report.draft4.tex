% Options for packages loaded elsewhere
\PassOptionsToPackage{unicode}{hyperref}
\PassOptionsToPackage{hyphens}{url}
\PassOptionsToPackage{dvipsnames,svgnames,x11names}{xcolor}
%
\documentclass[
]{article}
\usepackage{amsmath,amssymb}
\usepackage{iftex}
\ifPDFTeX
  \usepackage[T1]{fontenc}
  \usepackage[utf8]{inputenc}
  \usepackage{textcomp} % provide euro and other symbols
\else % if luatex or xetex
  \usepackage{unicode-math} % this also loads fontspec
  \defaultfontfeatures{Scale=MatchLowercase}
  \defaultfontfeatures[\rmfamily]{Ligatures=TeX,Scale=1}
\fi
\usepackage{lmodern}
\ifPDFTeX\else
  % xetex/luatex font selection
\fi
% Use upquote if available, for straight quotes in verbatim environments
\IfFileExists{upquote.sty}{\usepackage{upquote}}{}
\IfFileExists{microtype.sty}{% use microtype if available
  \usepackage[]{microtype}
  \UseMicrotypeSet[protrusion]{basicmath} % disable protrusion for tt fonts
}{}
\makeatletter
\@ifundefined{KOMAClassName}{% if non-KOMA class
  \IfFileExists{parskip.sty}{%
    \usepackage{parskip}
  }{% else
    \setlength{\parindent}{0pt}
    \setlength{\parskip}{6pt plus 2pt minus 1pt}}
}{% if KOMA class
  \KOMAoptions{parskip=half}}
\makeatother
\usepackage{xcolor}
\usepackage[margin=1in]{geometry}
\usepackage{color}
\usepackage{fancyvrb}
\newcommand{\VerbBar}{|}
\newcommand{\VERB}{\Verb[commandchars=\\\{\}]}
\DefineVerbatimEnvironment{Highlighting}{Verbatim}{commandchars=\\\{\}}
% Add ',fontsize=\small' for more characters per line
\usepackage{framed}
\definecolor{shadecolor}{RGB}{248,248,248}
\newenvironment{Shaded}{\begin{snugshade}}{\end{snugshade}}
\newcommand{\AlertTok}[1]{\textcolor[rgb]{0.94,0.16,0.16}{#1}}
\newcommand{\AnnotationTok}[1]{\textcolor[rgb]{0.56,0.35,0.01}{\textbf{\textit{#1}}}}
\newcommand{\AttributeTok}[1]{\textcolor[rgb]{0.13,0.29,0.53}{#1}}
\newcommand{\BaseNTok}[1]{\textcolor[rgb]{0.00,0.00,0.81}{#1}}
\newcommand{\BuiltInTok}[1]{#1}
\newcommand{\CharTok}[1]{\textcolor[rgb]{0.31,0.60,0.02}{#1}}
\newcommand{\CommentTok}[1]{\textcolor[rgb]{0.56,0.35,0.01}{\textit{#1}}}
\newcommand{\CommentVarTok}[1]{\textcolor[rgb]{0.56,0.35,0.01}{\textbf{\textit{#1}}}}
\newcommand{\ConstantTok}[1]{\textcolor[rgb]{0.56,0.35,0.01}{#1}}
\newcommand{\ControlFlowTok}[1]{\textcolor[rgb]{0.13,0.29,0.53}{\textbf{#1}}}
\newcommand{\DataTypeTok}[1]{\textcolor[rgb]{0.13,0.29,0.53}{#1}}
\newcommand{\DecValTok}[1]{\textcolor[rgb]{0.00,0.00,0.81}{#1}}
\newcommand{\DocumentationTok}[1]{\textcolor[rgb]{0.56,0.35,0.01}{\textbf{\textit{#1}}}}
\newcommand{\ErrorTok}[1]{\textcolor[rgb]{0.64,0.00,0.00}{\textbf{#1}}}
\newcommand{\ExtensionTok}[1]{#1}
\newcommand{\FloatTok}[1]{\textcolor[rgb]{0.00,0.00,0.81}{#1}}
\newcommand{\FunctionTok}[1]{\textcolor[rgb]{0.13,0.29,0.53}{\textbf{#1}}}
\newcommand{\ImportTok}[1]{#1}
\newcommand{\InformationTok}[1]{\textcolor[rgb]{0.56,0.35,0.01}{\textbf{\textit{#1}}}}
\newcommand{\KeywordTok}[1]{\textcolor[rgb]{0.13,0.29,0.53}{\textbf{#1}}}
\newcommand{\NormalTok}[1]{#1}
\newcommand{\OperatorTok}[1]{\textcolor[rgb]{0.81,0.36,0.00}{\textbf{#1}}}
\newcommand{\OtherTok}[1]{\textcolor[rgb]{0.56,0.35,0.01}{#1}}
\newcommand{\PreprocessorTok}[1]{\textcolor[rgb]{0.56,0.35,0.01}{\textit{#1}}}
\newcommand{\RegionMarkerTok}[1]{#1}
\newcommand{\SpecialCharTok}[1]{\textcolor[rgb]{0.81,0.36,0.00}{\textbf{#1}}}
\newcommand{\SpecialStringTok}[1]{\textcolor[rgb]{0.31,0.60,0.02}{#1}}
\newcommand{\StringTok}[1]{\textcolor[rgb]{0.31,0.60,0.02}{#1}}
\newcommand{\VariableTok}[1]{\textcolor[rgb]{0.00,0.00,0.00}{#1}}
\newcommand{\VerbatimStringTok}[1]{\textcolor[rgb]{0.31,0.60,0.02}{#1}}
\newcommand{\WarningTok}[1]{\textcolor[rgb]{0.56,0.35,0.01}{\textbf{\textit{#1}}}}
\usepackage{graphicx}
\makeatletter
\def\maxwidth{\ifdim\Gin@nat@width>\linewidth\linewidth\else\Gin@nat@width\fi}
\def\maxheight{\ifdim\Gin@nat@height>\textheight\textheight\else\Gin@nat@height\fi}
\makeatother
% Scale images if necessary, so that they will not overflow the page
% margins by default, and it is still possible to overwrite the defaults
% using explicit options in \includegraphics[width, height, ...]{}
\setkeys{Gin}{width=\maxwidth,height=\maxheight,keepaspectratio}
% Set default figure placement to htbp
\makeatletter
\def\fps@figure{htbp}
\makeatother
\setlength{\emergencystretch}{3em} % prevent overfull lines
\providecommand{\tightlist}{%
  \setlength{\itemsep}{0pt}\setlength{\parskip}{0pt}}
\setcounter{secnumdepth}{-\maxdimen} % remove section numbering
\usepackage{awesomebox}
\usepackage{hyperref}
\usepackage[natbib=true, style=numeric, backref=true, sorting=none]{biblatex}
\hypersetup{backref, pdfpagemode=Normal, colorlinks=true, implicit=false}
\usepackage{booktabs}
\usepackage{longtable}
\usepackage{array}
\usepackage{multirow}
\usepackage{wrapfig}
\usepackage{float}
\usepackage{colortbl}
\usepackage{pdflscape}
\usepackage{tabu}
\usepackage{threeparttable}
\usepackage{threeparttablex}
\usepackage[normalem]{ulem}
\usepackage{makecell}
\usepackage{xcolor}
\ifLuaTeX
  \usepackage{selnolig}  % disable illegal ligatures
\fi
\usepackage[]{biblatex}
\addbibresource{references.bib}
\usepackage{bookmark}
\IfFileExists{xurl.sty}{\usepackage{xurl}}{} % add URL line breaks if available
\urlstyle{same}
\hypersetup{
  pdftitle={Capstone Movielens Report},
  pdfauthor={Azamat Kurbanaev},
  colorlinks=true,
  linkcolor={red},
  filecolor={Maroon},
  citecolor={blue},
  urlcolor={Blue},
  pdfcreator={LaTeX via pandoc}}

\title{Capstone Movielens Report}
\author{Azamat Kurbanaev}
\date{2025-05-11}

\begin{document}
\maketitle

{
\hypersetup{linkcolor=}
\setcounter{tocdepth}{2}
\tableofcontents
}
\newenvironment{infobox}[1]
  {
  \begin{itemize}
  \renewcommand{\labelitemi}{
    \raisebox{-.7\height}[0pt][0pt]{
      {\setkeys{Gin}{width=3em,keepaspectratio}
        \includegraphics{images/#1}}
    }
  }
  \setlength{\fboxsep}{1em}
  \begin{blackbox}
  \item
  }
  {
  \end{blackbox}
  \end{itemize}
  }

\subsection{Introduction / Overview / Executive
Summary}\label{introduction-overview-executive-summary}

The goal of the project is to build a Recommendation System using a
\href{http://grouplens.org/datasets/movielens/10m/}{10M version of the
MovieLens dataset}. Following the
\href{https://archive.nytimes.com/bits.blogs.nytimes.com/2009/09/21/netflix-awards-1-million-prize-and-starts-a-new-contest/index.html}{Netflix
Grand Prize Contest} requirements, we will evaluate the \emph{Root Mean
Square Error} (\emph{RMSE}) score, which, as shown in
\href{https://rafalab.dfci.harvard.edu/dsbook-part-2/highdim/regularization.html\#sec-netflix-loss-function}{Section
23.2 Loss function} of the \emph{Course Textbook}, is defined as: \[
\mbox{RMSE} = \sqrt{\frac{1}{N} \sum_{i,j}^{N} (y_{i,j} - \hat{y}_{i,j})^2}
\]

with \(N\) being the number of user/movie combinations for which we make
predictions and the sum occurring over all these
combinations\autocite{IDS2_23-2}.

Our goal is to achieve a value of less than 0.86490 (compare with the
\emph{Netflix Grand Prize} requirement: of at least
0.8563\autocite{BigChaosSln}).

\subsubsection{Datasets Overview}\label{datasets-overview}

To start with we have to generate two datasets derived from the
\emph{MovieLens} one mentioned above:

\begin{itemize}
\tightlist
\item
  \textbf{edx:} we use it to develop and train our algorithms;
\item
  \textbf{final\_holdout\_test:} according to the course requirements,
  we use it exclusively to evaluate the \emph{\textbf{RMSE}} of our
  final algorithm.
\end{itemize}

For this purpose the following package has been developed by the author
of this report: \texttt{edx.capstone.movielens.data}. The source code of
the package is available
\href{https://github.com/AzKurban-edX-DS/edx.capstone.movielens.data}{on
GitHub}\autocite{edx_capstone_movielens_data}.

Let's install the development version of this package from the GitHub
repository and attach the correspondent library to the global
environment:

\begin{Shaded}
\begin{Highlighting}[]
\ControlFlowTok{if}\NormalTok{(}\SpecialCharTok{!}\FunctionTok{require}\NormalTok{(edx.capstone.movielens.data)) pak}\SpecialCharTok{::}\FunctionTok{pak}\NormalTok{(}\StringTok{"AzKurban{-}edX{-}DS/edx.capstone.movielens.data"}\NormalTok{)}

\FunctionTok{library}\NormalTok{(edx.capstone.movielens.data)}
\NormalTok{edx }\OtherTok{\textless{}{-}}\NormalTok{ edx.capstone.movielens.data}\SpecialCharTok{::}\NormalTok{edx}
\NormalTok{final\_holdout\_test }\OtherTok{\textless{}{-}}\NormalTok{ edx.capstone.movielens.data}\SpecialCharTok{::}\NormalTok{final\_holdout\_test}
\end{Highlighting}
\end{Shaded}

Now, we have the datasets listed above:

\begin{Shaded}
\begin{Highlighting}[]
\FunctionTok{summary}\NormalTok{(edx)}
\end{Highlighting}
\end{Shaded}

\begin{verbatim}
##      userId         movieId          rating        timestamp            title              genres         
##  Min.   :    1   Min.   :    1   Min.   :0.500   Min.   :7.897e+08   Length:9000055     Length:9000055    
##  1st Qu.:18124   1st Qu.:  648   1st Qu.:3.000   1st Qu.:9.468e+08   Class :character   Class :character  
##  Median :35738   Median : 1834   Median :4.000   Median :1.035e+09   Mode  :character   Mode  :character  
##  Mean   :35870   Mean   : 4122   Mean   :3.512   Mean   :1.033e+09                                        
##  3rd Qu.:53607   3rd Qu.: 3626   3rd Qu.:4.000   3rd Qu.:1.127e+09                                        
##  Max.   :71567   Max.   :65133   Max.   :5.000   Max.   :1.231e+09
\end{verbatim}

\begin{Shaded}
\begin{Highlighting}[]
\FunctionTok{summary}\NormalTok{(final\_holdout\_test)}
\end{Highlighting}
\end{Shaded}

\begin{verbatim}
##      userId         movieId          rating        timestamp            title              genres         
##  Min.   :    1   Min.   :    1   Min.   :0.500   Min.   :7.897e+08   Length:999999      Length:999999     
##  1st Qu.:18096   1st Qu.:  648   1st Qu.:3.000   1st Qu.:9.467e+08   Class :character   Class :character  
##  Median :35768   Median : 1827   Median :4.000   Median :1.035e+09   Mode  :character   Mode  :character  
##  Mean   :35870   Mean   : 4108   Mean   :3.512   Mean   :1.033e+09                                        
##  3rd Qu.:53621   3rd Qu.: 3624   3rd Qu.:4.000   3rd Qu.:1.127e+09                                        
##  Max.   :71567   Max.   :65133   Max.   :5.000   Max.   :1.231e+09
\end{verbatim}

\paragraph{\texorpdfstring{\texttt{edx}
Dataset}{edx Dataset}}\label{edx-dataset}

\hfill\break
Let's look into the details of the \texttt{edx} dataset:

\begin{Shaded}
\begin{Highlighting}[]
\FunctionTok{str}\NormalTok{(edx)}
\end{Highlighting}
\end{Shaded}

\begin{verbatim}
## 'data.frame':    9000055 obs. of  6 variables:
##  $ userId   : int  1 1 1 1 1 1 1 1 1 1 ...
##  $ movieId  : int  122 185 292 316 329 355 356 362 364 370 ...
##  $ rating   : num  5 5 5 5 5 5 5 5 5 5 ...
##  $ timestamp: int  838985046 838983525 838983421 838983392 838983392 838984474 838983653 838984885 838983707 838984596 ...
##  $ title    : chr  "Boomerang (1992)" "Net, The (1995)" "Outbreak (1995)" "Stargate (1994)" ...
##  $ genres   : chr  "Comedy|Romance" "Action|Crime|Thriller" "Action|Drama|Sci-Fi|Thriller" "Action|Adventure|Sci-Fi" ...
\end{verbatim}

Note that we have 9000055 rows and six columns in there:

\begin{Shaded}
\begin{Highlighting}[]
\NormalTok{dim\_edx }\OtherTok{\textless{}{-}} \FunctionTok{dim}\NormalTok{(edx)}
\FunctionTok{print}\NormalTok{(dim\_edx)}
\end{Highlighting}
\end{Shaded}

\begin{verbatim}
## [1] 9000055       6
\end{verbatim}

First, let's note that we have 10677 different movies:

\begin{Shaded}
\begin{Highlighting}[]
\NormalTok{n\_movies }\OtherTok{\textless{}{-}} \FunctionTok{n\_distinct}\NormalTok{(edx}\SpecialCharTok{$}\NormalTok{movieId)}
\FunctionTok{print}\NormalTok{(n\_movies)}
\end{Highlighting}
\end{Shaded}

\begin{verbatim}
## [1] 10677
\end{verbatim}

and 69878 different users in the dataset:

\begin{Shaded}
\begin{Highlighting}[]
\NormalTok{n\_users }\OtherTok{\textless{}{-}} \FunctionTok{n\_distinct}\NormalTok{(edx}\SpecialCharTok{$}\NormalTok{userId)}
\FunctionTok{print}\NormalTok{(n\_users)}
\end{Highlighting}
\end{Shaded}

\begin{verbatim}
## [1] 69878
\end{verbatim}

Now, note the expressions below which confirm the fact explained in
\href{https://rafalab.dfci.harvard.edu/dsbook-part-2/highdim/regularization.html\#movielens-data}{Section
\emph{23.1.1 Movielens data}} of the \emph{Course
Textbook}\autocite{IDS2} that not every user rated every movie:

\begin{Shaded}
\begin{Highlighting}[]
\NormalTok{max\_possible\_ratings }\OtherTok{\textless{}{-}}\NormalTok{ n\_movies}\SpecialCharTok{*}\NormalTok{n\_users}
\FunctionTok{sprintf}\NormalTok{(}\StringTok{"Maximum possible ratings: \%s"}\NormalTok{, max\_possible\_ratings)}
\end{Highlighting}
\end{Shaded}

\begin{verbatim}
## [1] "Maximum possible ratings: 746087406"
\end{verbatim}

\begin{Shaded}
\begin{Highlighting}[]
\FunctionTok{sprintf}\NormalTok{(}\StringTok{"Rows in \textasciigrave{}edx\textasciigrave{} dataset: \%s"}\NormalTok{, dim\_edx[}\DecValTok{1}\NormalTok{])}
\end{Highlighting}
\end{Shaded}

\begin{verbatim}
## [1] "Rows in `edx` dataset: 9000055"
\end{verbatim}

\begin{Shaded}
\begin{Highlighting}[]
\FunctionTok{sprintf}\NormalTok{(}\StringTok{"Not every movie was rated: \%s"}\NormalTok{, max\_possible\_ratings }\SpecialCharTok{\textgreater{}}\NormalTok{ dim\_edx[}\DecValTok{1}\NormalTok{])}
\end{Highlighting}
\end{Shaded}

\begin{verbatim}
## [1] "Not every movie was rated: TRUE"
\end{verbatim}

As also explained in that section, we can think of these data as a very
large matrix, with users on the rows and movies on the columns, with
many empty cells. Therefore, we can think of a recommendation system as
filling in the \texttt{NA}s in the dataset for the movies that some or
all the users do not rate. A sample from the \texttt{edx} data below
illustrates this idea\autocite{IDS2_23-1-1}:

\begin{Shaded}
\begin{Highlighting}[]
\NormalTok{keep }\OtherTok{\textless{}{-}}\NormalTok{ edx }\SpecialCharTok{|\textgreater{}} 
\NormalTok{  dplyr}\SpecialCharTok{::}\FunctionTok{count}\NormalTok{(movieId) }\SpecialCharTok{|\textgreater{}} 
  \FunctionTok{top\_n}\NormalTok{(}\DecValTok{4}\NormalTok{, n) }\SpecialCharTok{|\textgreater{}} 
  \FunctionTok{pull}\NormalTok{(movieId)}

\NormalTok{tab }\OtherTok{\textless{}{-}}\NormalTok{ edx }\SpecialCharTok{|\textgreater{}} 
  \FunctionTok{filter}\NormalTok{(movieId }\SpecialCharTok{\%in\%}\NormalTok{ keep) }\SpecialCharTok{|\textgreater{}} 
  \FunctionTok{filter}\NormalTok{(userId }\SpecialCharTok{\%in\%} \FunctionTok{c}\NormalTok{(}\DecValTok{13}\SpecialCharTok{:}\DecValTok{20}\NormalTok{)) }\SpecialCharTok{|\textgreater{}} 
  \FunctionTok{select}\NormalTok{(userId, title, rating) }\SpecialCharTok{|\textgreater{}} 
  \FunctionTok{mutate}\NormalTok{(}\AttributeTok{title =} \FunctionTok{str\_remove}\NormalTok{(title, }\StringTok{", The"}\NormalTok{),}
         \AttributeTok{title =} \FunctionTok{str\_remove}\NormalTok{(title, }\StringTok{":.*"}\NormalTok{)) }\SpecialCharTok{|\textgreater{}}
  \FunctionTok{pivot\_wider}\NormalTok{(}\AttributeTok{names\_from =} \StringTok{"title"}\NormalTok{, }\AttributeTok{values\_from =} \StringTok{"rating"}\NormalTok{)}

\FunctionTok{print}\NormalTok{(tab)}
\end{Highlighting}
\end{Shaded}

\begin{verbatim}
## # A tibble: 5 x 5
##   userId `Pulp Fiction (1994)` `Jurassic Park (1993)` `Silence of the Lambs (1991)` `Forrest Gump (1994)`
##    <int>                 <dbl>                  <dbl>                         <dbl>                 <dbl>
## 1     13                     4                     NA                            NA                    NA
## 2     16                    NA                      3                            NA                    NA
## 3     17                    NA                     NA                             5                    NA
## 4     18                     5                      3                             5                    NA
## 5     19                    NA                      1                            NA                     4
\end{verbatim}

The following plot of the matrix for a random sample of 100 movies and
100 users with yellow indicating a user/movie combination for which we
have a rating shows how \emph{sparse} the matrix is:

\begin{Shaded}
\begin{Highlighting}[]
\NormalTok{users }\OtherTok{\textless{}{-}} \FunctionTok{sample}\NormalTok{(}\FunctionTok{unique}\NormalTok{(edx}\SpecialCharTok{$}\NormalTok{userId), }\DecValTok{100}\NormalTok{)}

\NormalTok{rafalib}\SpecialCharTok{::}\FunctionTok{mypar}\NormalTok{()}
\NormalTok{edx}\SpecialCharTok{|\textgreater{}} 
  \FunctionTok{filter}\NormalTok{(userId }\SpecialCharTok{\%in\%}\NormalTok{ users) }\SpecialCharTok{|\textgreater{}} 
  \FunctionTok{select}\NormalTok{(userId, movieId, rating) }\SpecialCharTok{|\textgreater{}}
  \FunctionTok{mutate}\NormalTok{(}\AttributeTok{rating =} \DecValTok{1}\NormalTok{) }\SpecialCharTok{|\textgreater{}}
  \FunctionTok{pivot\_wider}\NormalTok{(}\AttributeTok{names\_from =}\NormalTok{ movieId, }\AttributeTok{values\_from =}\NormalTok{ rating) }\SpecialCharTok{|\textgreater{}} 
\NormalTok{  (\textbackslash{}(mat) mat[, }\FunctionTok{sample}\NormalTok{(}\FunctionTok{ncol}\NormalTok{(mat), }\DecValTok{100}\NormalTok{)])() }\SpecialCharTok{|\textgreater{}}
  \FunctionTok{as.matrix}\NormalTok{() }\SpecialCharTok{|\textgreater{}} 
  \FunctionTok{t}\NormalTok{() }\SpecialCharTok{|\textgreater{}}
  \FunctionTok{image}\NormalTok{(}\DecValTok{1}\SpecialCharTok{:}\DecValTok{100}\NormalTok{, }\DecValTok{1}\SpecialCharTok{:}\DecValTok{100}\NormalTok{, }\AttributeTok{z =}\NormalTok{ \_ , }\AttributeTok{xlab =} \StringTok{"Movies"}\NormalTok{, }\AttributeTok{ylab =} \StringTok{"Users"}\NormalTok{)}
\end{Highlighting}
\end{Shaded}

\includegraphics[width=0.4\linewidth]{capstone-movielens-report.draft4_files/figure-latex/sparsity-of-movie-recs-1}

Further observations highlighted there that, as we can see from the
distributions the author presented, some movies get rated more than
others, and some users are more active than others in rating movies:

\begin{Shaded}
\begin{Highlighting}[]
\NormalTok{p1 }\OtherTok{\textless{}{-}}\NormalTok{ edx }\SpecialCharTok{|\textgreater{}} 
  \FunctionTok{count}\NormalTok{(movieId) }\SpecialCharTok{|\textgreater{}} 
  \FunctionTok{ggplot}\NormalTok{(}\FunctionTok{aes}\NormalTok{(n)) }\SpecialCharTok{+} 
  \FunctionTok{geom\_histogram}\NormalTok{(}\AttributeTok{bins =} \DecValTok{30}\NormalTok{, }\AttributeTok{color =} \StringTok{"black"}\NormalTok{) }\SpecialCharTok{+} 
  \FunctionTok{scale\_x\_log10}\NormalTok{() }\SpecialCharTok{+} 
  \FunctionTok{ggtitle}\NormalTok{(}\StringTok{"Movies"}\NormalTok{)}

\NormalTok{p2 }\OtherTok{\textless{}{-}}\NormalTok{ edx }\SpecialCharTok{|\textgreater{}} 
  \FunctionTok{count}\NormalTok{(userId) }\SpecialCharTok{|\textgreater{}} 
  \FunctionTok{ggplot}\NormalTok{(}\FunctionTok{aes}\NormalTok{(n)) }\SpecialCharTok{+} 
  \FunctionTok{geom\_histogram}\NormalTok{(}\AttributeTok{bins =} \DecValTok{30}\NormalTok{, }\AttributeTok{color =} \StringTok{"black"}\NormalTok{) }\SpecialCharTok{+} 
  \FunctionTok{scale\_x\_log10}\NormalTok{() }\SpecialCharTok{+} 
  \FunctionTok{ggtitle}\NormalTok{(}\StringTok{"Users"}\NormalTok{)}

\NormalTok{gridExtra}\SpecialCharTok{::}\FunctionTok{grid.arrange}\NormalTok{(p2, p1, }\AttributeTok{ncol =} \DecValTok{2}\NormalTok{)}
\end{Highlighting}
\end{Shaded}

\includegraphics{capstone-movielens-report.draft4_files/figure-latex/movie-id-and-user-hists-1.pdf}

Finally, we can see that no movies have a rating of 0. Movies are rated
from 0.5 to 5.0 in 0.5 increments:

\begin{Shaded}
\begin{Highlighting}[]
\CommentTok{\#library(dplyr)}
\NormalTok{s }\OtherTok{\textless{}{-}}\NormalTok{ edx }\SpecialCharTok{|\textgreater{}} \FunctionTok{group\_by}\NormalTok{(rating) }\SpecialCharTok{|\textgreater{}}
  \FunctionTok{summarise}\NormalTok{(}\AttributeTok{n =} \FunctionTok{n}\NormalTok{())}
\FunctionTok{print}\NormalTok{(s)}
\end{Highlighting}
\end{Shaded}

\begin{verbatim}
## # A tibble: 10 x 2
##    rating       n
##     <dbl>   <int>
##  1    0.5   85374
##  2    1    345679
##  3    1.5  106426
##  4    2    711422
##  5    2.5  333010
##  6    3   2121240
##  7    3.5  791624
##  8    4   2588430
##  9    4.5  526736
## 10    5   1390114
\end{verbatim}

Further analysis of the \texttt{edx} dataset have been also inspired by
the article mentioned above\autocite{MRS-R-BEST}, from which the code
and explanatory notes below were cited.

\subparagraph{\texorpdfstring{Rating distribution
plot\autocite{MRS-R-BEST}}{Rating distribution plot{[}@MRS-R-BEST{]}}}\label{rating-distribution-plotmrs-r-best}

\hfill\break
The code below demonstrates another way of visualizing the rating
distribution:

\begin{Shaded}
\begin{Highlighting}[]
\NormalTok{edx }\SpecialCharTok{|\textgreater{}}
  \FunctionTok{group\_by}\NormalTok{(rating) }\SpecialCharTok{|\textgreater{}}
  \FunctionTok{summarize}\NormalTok{(}\AttributeTok{count =} \FunctionTok{n}\NormalTok{()) }\SpecialCharTok{|\textgreater{}}
  \FunctionTok{ggplot}\NormalTok{(}\FunctionTok{aes}\NormalTok{(}\AttributeTok{x =}\NormalTok{ rating, }\AttributeTok{y =}\NormalTok{ count)) }\SpecialCharTok{+}
  \FunctionTok{geom\_bar}\NormalTok{(}\AttributeTok{stat =} \StringTok{"identity"}\NormalTok{, }\AttributeTok{fill =} \StringTok{"\#8888ff"}\NormalTok{) }\SpecialCharTok{+}
  \FunctionTok{ggtitle}\NormalTok{(}\StringTok{"Rating Distribution"}\NormalTok{) }\SpecialCharTok{+}
  \FunctionTok{xlab}\NormalTok{(}\StringTok{"Rating"}\NormalTok{) }\SpecialCharTok{+}
  \FunctionTok{ylab}\NormalTok{(}\StringTok{"Occurrences Count"}\NormalTok{) }\SpecialCharTok{+}
  \FunctionTok{scale\_y\_continuous}\NormalTok{(}\AttributeTok{labels =}\NormalTok{ comma) }\SpecialCharTok{+}
  \FunctionTok{scale\_x\_continuous}\NormalTok{(}\AttributeTok{n.breaks =} \DecValTok{10}\NormalTok{) }\SpecialCharTok{+}
  \FunctionTok{theme\_economist}\NormalTok{() }\SpecialCharTok{+}
  \FunctionTok{theme}\NormalTok{(}\AttributeTok{axis.title.x =} \FunctionTok{element\_text}\NormalTok{(}\AttributeTok{vjust =} \SpecialCharTok{{-}}\DecValTok{5}\NormalTok{, }\AttributeTok{face =} \StringTok{"bold"}\NormalTok{), }
        \AttributeTok{axis.title.y =} \FunctionTok{element\_text}\NormalTok{(}\AttributeTok{vjust =} \DecValTok{10}\NormalTok{, }\AttributeTok{face =} \StringTok{"bold"}\NormalTok{), }
        \AttributeTok{plot.margin =} \FunctionTok{margin}\NormalTok{(}\FloatTok{0.7}\NormalTok{, }\FloatTok{0.5}\NormalTok{, }\DecValTok{1}\NormalTok{, }\FloatTok{1.2}\NormalTok{, }\StringTok{"cm"}\NormalTok{))}
\end{Highlighting}
\end{Shaded}

\includegraphics{capstone-movielens-report.draft4_files/figure-latex/unnamed-chunk-11-1.pdf}

This graph is another confirmation of what we found out above: rounded
ratings occur more often than half-stared ones. The upward trend
previously discussed is now perfectly clear, although it seems to top
right between the 3 and 4-star ratings lowering the occurrences count
afterward. That might be due to users being more hesitant to rate with
the highest mark for whichever reasons they might
hold\autocite{MRS-R-BEST}.

\subparagraph{Ratings per movie}\label{ratings-per-movie}

\hfill\break

Movie popularity count\autocite{MRS-R-BEST}

\hfill\break

\begin{Shaded}
\begin{Highlighting}[]
\FunctionTok{print}\NormalTok{(edx }\SpecialCharTok{|\textgreater{}} 
  \FunctionTok{group\_by}\NormalTok{(movieId) }\SpecialCharTok{|\textgreater{}} 
  \FunctionTok{summarize}\NormalTok{(}\AttributeTok{count =} \FunctionTok{n}\NormalTok{()) }\SpecialCharTok{|\textgreater{}}
  \FunctionTok{slice\_head}\NormalTok{(}\AttributeTok{n =} \DecValTok{10}\NormalTok{)}
\NormalTok{)}
\end{Highlighting}
\end{Shaded}

\begin{verbatim}
## # A tibble: 10 x 2
##    movieId count
##      <int> <int>
##  1       1 23790
##  2       2 10779
##  3       3  7028
##  4       4  1577
##  5       5  6400
##  6       6 12346
##  7       7  7259
##  8       8   821
##  9       9  2278
## 10      10 15187
\end{verbatim}

\begin{Shaded}
\begin{Highlighting}[]
\FunctionTok{summary}\NormalTok{(edx }\SpecialCharTok{|\textgreater{}} \FunctionTok{group\_by}\NormalTok{(movieId) }\SpecialCharTok{|\textgreater{}} \FunctionTok{summarize}\NormalTok{(}\AttributeTok{count =} \FunctionTok{n}\NormalTok{()) }\SpecialCharTok{|\textgreater{}} \FunctionTok{select}\NormalTok{(count))}
\end{Highlighting}
\end{Shaded}

\begin{verbatim}
##      count        
##  Min.   :    1.0  
##  1st Qu.:   30.0  
##  Median :  122.0  
##  Mean   :  842.9  
##  3rd Qu.:  565.0  
##  Max.   :31362.0
\end{verbatim}

Ratings per movie plot\autocite{MRS-R-BEST}

\hfill\break

\begin{Shaded}
\begin{Highlighting}[]
\NormalTok{edx }\SpecialCharTok{|\textgreater{}}
  \FunctionTok{group\_by}\NormalTok{(movieId) }\SpecialCharTok{|\textgreater{}}
  \FunctionTok{summarize}\NormalTok{(}\AttributeTok{count =} \FunctionTok{n}\NormalTok{()) }\SpecialCharTok{|\textgreater{}}
  \FunctionTok{ggplot}\NormalTok{(}\FunctionTok{aes}\NormalTok{(}\AttributeTok{x =}\NormalTok{ movieId, }\AttributeTok{y =}\NormalTok{ count)) }\SpecialCharTok{+}
  \FunctionTok{geom\_point}\NormalTok{(}\AttributeTok{alpha =} \FloatTok{0.2}\NormalTok{, }\AttributeTok{color =} \StringTok{"\#4020dd"}\NormalTok{) }\SpecialCharTok{+}
  \FunctionTok{geom\_smooth}\NormalTok{(}\AttributeTok{color =} \StringTok{"red"}\NormalTok{) }\SpecialCharTok{+}
  \FunctionTok{ggtitle}\NormalTok{(}\StringTok{"Ratings per movie"}\NormalTok{) }\SpecialCharTok{+}
  \FunctionTok{xlab}\NormalTok{(}\StringTok{"Movies"}\NormalTok{) }\SpecialCharTok{+}
  \FunctionTok{ylab}\NormalTok{(}\StringTok{"Number of ratings"}\NormalTok{) }\SpecialCharTok{+}
  \FunctionTok{scale\_y\_continuous}\NormalTok{(}\AttributeTok{labels =}\NormalTok{ comma) }\SpecialCharTok{+}
  \FunctionTok{scale\_x\_continuous}\NormalTok{(}\AttributeTok{n.breaks =} \DecValTok{10}\NormalTok{) }\SpecialCharTok{+}
  \FunctionTok{theme\_economist}\NormalTok{() }\SpecialCharTok{+}
  \FunctionTok{theme}\NormalTok{(}\AttributeTok{axis.title.x =} \FunctionTok{element\_text}\NormalTok{(}\AttributeTok{vjust =} \SpecialCharTok{{-}}\DecValTok{5}\NormalTok{, }\AttributeTok{face =} \StringTok{"bold"}\NormalTok{), }
        \AttributeTok{axis.title.y =} \FunctionTok{element\_text}\NormalTok{(}\AttributeTok{vjust =} \DecValTok{10}\NormalTok{, }\AttributeTok{face =} \StringTok{"bold"}\NormalTok{), }
        \AttributeTok{plot.margin =} \FunctionTok{margin}\NormalTok{(}\FloatTok{0.7}\NormalTok{, }\FloatTok{0.5}\NormalTok{, }\DecValTok{1}\NormalTok{, }\FloatTok{1.2}\NormalTok{, }\StringTok{"cm"}\NormalTok{))}
\end{Highlighting}
\end{Shaded}

\begin{verbatim}
## `geom_smooth()` using method = 'gam' and formula = 'y ~ s(x, bs = "cs")'
\end{verbatim}

\includegraphics{capstone-movielens-report.draft4_files/figure-latex/unnamed-chunk-14-1.pdf}

Movies' rating histogram\autocite{MRS-R-BEST}

\hfill\break

\begin{Shaded}
\begin{Highlighting}[]
\NormalTok{edx }\SpecialCharTok{|\textgreater{}}
  \FunctionTok{group\_by}\NormalTok{(movieId) }\SpecialCharTok{|\textgreater{}}
  \FunctionTok{summarize}\NormalTok{(}\AttributeTok{count =} \FunctionTok{n}\NormalTok{()) }\SpecialCharTok{|\textgreater{}}
  \FunctionTok{ggplot}\NormalTok{(}\FunctionTok{aes}\NormalTok{(}\AttributeTok{x =}\NormalTok{ count)) }\SpecialCharTok{+}
  \FunctionTok{geom\_histogram}\NormalTok{(}\AttributeTok{fill =} \StringTok{"\#8888ff"}\NormalTok{, }\AttributeTok{color =} \StringTok{"\#4020dd"}\NormalTok{) }\SpecialCharTok{+}
  \FunctionTok{ggtitle}\NormalTok{(}\StringTok{"Movies\textquotesingle{} rating histogram"}\NormalTok{) }\SpecialCharTok{+}
  \FunctionTok{xlab}\NormalTok{(}\StringTok{"Rating count"}\NormalTok{) }\SpecialCharTok{+}
  \FunctionTok{ylab}\NormalTok{(}\StringTok{"Number of movies"}\NormalTok{) }\SpecialCharTok{+}
  \FunctionTok{scale\_y\_continuous}\NormalTok{(}\AttributeTok{labels =}\NormalTok{ comma) }\SpecialCharTok{+}
  \FunctionTok{scale\_x\_log10}\NormalTok{(}\AttributeTok{n.breaks =} \DecValTok{10}\NormalTok{) }\SpecialCharTok{+}
  \FunctionTok{theme\_economist}\NormalTok{() }\SpecialCharTok{+}
  \FunctionTok{theme}\NormalTok{(}\AttributeTok{axis.title.x =} \FunctionTok{element\_text}\NormalTok{(}\AttributeTok{vjust =} \SpecialCharTok{{-}}\DecValTok{5}\NormalTok{, }\AttributeTok{face =} \StringTok{"bold"}\NormalTok{), }
        \AttributeTok{axis.title.y =} \FunctionTok{element\_text}\NormalTok{(}\AttributeTok{vjust =} \DecValTok{10}\NormalTok{, }\AttributeTok{face =} \StringTok{"bold"}\NormalTok{), }
        \AttributeTok{plot.margin =} \FunctionTok{margin}\NormalTok{(}\FloatTok{0.7}\NormalTok{, }\FloatTok{0.5}\NormalTok{, }\DecValTok{1}\NormalTok{, }\FloatTok{1.2}\NormalTok{, }\StringTok{"cm"}\NormalTok{))}
\end{Highlighting}
\end{Shaded}

\begin{verbatim}
## `stat_bin()` using `bins = 30`. Pick better value with `binwidth`.
\end{verbatim}

\includegraphics{capstone-movielens-report.draft4_files/figure-latex/unnamed-chunk-15-1.pdf}

\subparagraph{\texorpdfstring{Ratings per
user\autocite{MRS-R-BEST}}{Ratings per user{[}@MRS-R-BEST{]}}}\label{ratings-per-usermrs-r-best}

\hfill\break

User rating count (activity measure)

\hfill\break

\begin{Shaded}
\begin{Highlighting}[]
\FunctionTok{print}\NormalTok{(edx }\SpecialCharTok{|\textgreater{}} 
  \FunctionTok{group\_by}\NormalTok{(userId) }\SpecialCharTok{|\textgreater{}} 
  \FunctionTok{summarize}\NormalTok{(}\AttributeTok{count =} \FunctionTok{n}\NormalTok{()) }\SpecialCharTok{|\textgreater{}}
  \FunctionTok{slice\_head}\NormalTok{(}\AttributeTok{n =} \DecValTok{10}\NormalTok{)}
\NormalTok{)}
\end{Highlighting}
\end{Shaded}

\begin{verbatim}
## # A tibble: 10 x 2
##    userId count
##     <int> <int>
##  1      1    19
##  2      2    17
##  3      3    31
##  4      4    35
##  5      5    74
##  6      6    39
##  7      7    96
##  8      8   727
##  9      9    21
## 10     10   112
\end{verbatim}

User rating summary

\hfill\break

\begin{Shaded}
\begin{Highlighting}[]
\FunctionTok{summary}\NormalTok{(edx }\SpecialCharTok{|\textgreater{}} \FunctionTok{group\_by}\NormalTok{(userId) }\SpecialCharTok{|\textgreater{}} \FunctionTok{summarize}\NormalTok{(}\AttributeTok{count =} \FunctionTok{n}\NormalTok{()) }\SpecialCharTok{|\textgreater{}} \FunctionTok{select}\NormalTok{(count))}
\end{Highlighting}
\end{Shaded}

\begin{verbatim}
##      count       
##  Min.   :  10.0  
##  1st Qu.:  32.0  
##  Median :  62.0  
##  Mean   : 128.8  
##  3rd Qu.: 141.0  
##  Max.   :6616.0
\end{verbatim}

Ratings per user plot

\hfill\break

\begin{Shaded}
\begin{Highlighting}[]
\NormalTok{edx }\SpecialCharTok{|\textgreater{}}
  \FunctionTok{group\_by}\NormalTok{(userId) }\SpecialCharTok{|\textgreater{}}
  \FunctionTok{summarize}\NormalTok{(}\AttributeTok{count =} \FunctionTok{n}\NormalTok{()) }\SpecialCharTok{|\textgreater{}}
  \FunctionTok{ggplot}\NormalTok{(}\FunctionTok{aes}\NormalTok{(}\AttributeTok{x =}\NormalTok{ userId, }\AttributeTok{y =}\NormalTok{ count)) }\SpecialCharTok{+}
  \FunctionTok{geom\_point}\NormalTok{(}\AttributeTok{alpha =} \FloatTok{0.2}\NormalTok{, }\AttributeTok{color =} \StringTok{"\#4020dd"}\NormalTok{) }\SpecialCharTok{+}
  \FunctionTok{geom\_smooth}\NormalTok{(}\AttributeTok{color =} \StringTok{"red"}\NormalTok{) }\SpecialCharTok{+}
  \FunctionTok{ggtitle}\NormalTok{(}\StringTok{"Ratings per user"}\NormalTok{) }\SpecialCharTok{+}
  \FunctionTok{xlab}\NormalTok{(}\StringTok{"Users"}\NormalTok{) }\SpecialCharTok{+}
  \FunctionTok{ylab}\NormalTok{(}\StringTok{"Number of ratings"}\NormalTok{) }\SpecialCharTok{+}
  \FunctionTok{scale\_y\_continuous}\NormalTok{(}\AttributeTok{labels =}\NormalTok{ comma) }\SpecialCharTok{+}
  \FunctionTok{scale\_x\_continuous}\NormalTok{(}\AttributeTok{n.breaks =} \DecValTok{10}\NormalTok{) }\SpecialCharTok{+}
  \FunctionTok{theme\_economist}\NormalTok{() }\SpecialCharTok{+}
  \FunctionTok{theme}\NormalTok{(}\AttributeTok{axis.title.x =} \FunctionTok{element\_text}\NormalTok{(}\AttributeTok{vjust =} \SpecialCharTok{{-}}\DecValTok{5}\NormalTok{, }\AttributeTok{face =} \StringTok{"bold"}\NormalTok{), }
        \AttributeTok{axis.title.y =} \FunctionTok{element\_text}\NormalTok{(}\AttributeTok{vjust =} \DecValTok{10}\NormalTok{, }\AttributeTok{face =} \StringTok{"bold"}\NormalTok{), }
        \AttributeTok{plot.margin =} \FunctionTok{margin}\NormalTok{(}\FloatTok{0.7}\NormalTok{, }\FloatTok{0.5}\NormalTok{, }\DecValTok{1}\NormalTok{, }\FloatTok{1.2}\NormalTok{, }\StringTok{"cm"}\NormalTok{))}
\end{Highlighting}
\end{Shaded}

\begin{verbatim}
## `geom_smooth()` using method = 'gam' and formula = 'y ~ s(x, bs = "cs")'
\end{verbatim}

\includegraphics{capstone-movielens-report.draft4_files/figure-latex/unnamed-chunk-18-1.pdf}

Users' rating histogram

\hfill\break

\begin{Shaded}
\begin{Highlighting}[]
\NormalTok{edx }\SpecialCharTok{|\textgreater{}}
  \FunctionTok{group\_by}\NormalTok{(userId) }\SpecialCharTok{|\textgreater{}}
  \FunctionTok{summarize}\NormalTok{(}\AttributeTok{count =} \FunctionTok{n}\NormalTok{()) }\SpecialCharTok{|\textgreater{}}
  \FunctionTok{ggplot}\NormalTok{(}\FunctionTok{aes}\NormalTok{(}\AttributeTok{x =}\NormalTok{ count)) }\SpecialCharTok{+}
  \FunctionTok{geom\_histogram}\NormalTok{(}\AttributeTok{fill =} \StringTok{"\#8888ff"}\NormalTok{, }\AttributeTok{color =} \StringTok{"\#4020dd"}\NormalTok{) }\SpecialCharTok{+}
  \FunctionTok{ggtitle}\NormalTok{(}\StringTok{"Users\textquotesingle{} rating histogram"}\NormalTok{) }\SpecialCharTok{+}
  \FunctionTok{xlab}\NormalTok{(}\StringTok{"Rating count"}\NormalTok{) }\SpecialCharTok{+}
  \FunctionTok{ylab}\NormalTok{(}\StringTok{"Number of users"}\NormalTok{) }\SpecialCharTok{+}
  \FunctionTok{scale\_y\_continuous}\NormalTok{(}\AttributeTok{labels =}\NormalTok{ comma) }\SpecialCharTok{+}
  \FunctionTok{scale\_x\_log10}\NormalTok{(}\AttributeTok{n.breaks =} \DecValTok{10}\NormalTok{) }\SpecialCharTok{+}
  \FunctionTok{theme\_economist}\NormalTok{() }\SpecialCharTok{+}
  \FunctionTok{theme}\NormalTok{(}\AttributeTok{axis.title.x =} \FunctionTok{element\_text}\NormalTok{(}\AttributeTok{vjust =} \SpecialCharTok{{-}}\DecValTok{5}\NormalTok{, }\AttributeTok{face =} \StringTok{"bold"}\NormalTok{), }
        \AttributeTok{axis.title.y =} \FunctionTok{element\_text}\NormalTok{(}\AttributeTok{vjust =} \DecValTok{10}\NormalTok{, }\AttributeTok{face =} \StringTok{"bold"}\NormalTok{), }
        \AttributeTok{plot.margin =} \FunctionTok{margin}\NormalTok{(}\FloatTok{0.7}\NormalTok{, }\FloatTok{0.5}\NormalTok{, }\DecValTok{1}\NormalTok{, }\FloatTok{1.2}\NormalTok{, }\StringTok{"cm"}\NormalTok{))}
\end{Highlighting}
\end{Shaded}

\begin{verbatim}
## `stat_bin()` using `bins = 30`. Pick better value with `binwidth`.
\end{verbatim}

\includegraphics{capstone-movielens-report.draft4_files/figure-latex/unnamed-chunk-19-1.pdf}

\subsection{Methods / Analysis}\label{methods-analysis}

\begin{noteblock}
All the source code of the R-scripts is available on the project's
\href{https://github.com/AzKurban-edX-DS/Capstone-MovieLens/tree/main/r/src}{GitHub
repository}\autocite{edx_capstone_movielens}.

\end{noteblock}

\subsubsection{Defining Logging and Time Measuring Helper
Functions}\label{defining-logging-and-time-measuring-helper-functions}

\hfill\break

First, let's define some helper functions for logging and time-measuring
features that we will use in our R scripts. Some of them are listed
below:

\begin{Shaded}
\begin{Highlighting}[]
\DocumentationTok{\#\# Logging Helper functions {-}{-}{-}{-}{-}{-}{-}{-}{-}{-}{-}{-}{-}{-}{-}{-}{-}{-}{-}{-}{-}{-}{-}{-}{-}{-}{-}{-}{-}{-}{-}{-}{-}{-}{-}{-}{-}{-}{-}{-}{-}{-}{-}{-}{-}{-}{-}{-}{-}{-}{-}{-}{-}}
\NormalTok{open\_logfile }\OtherTok{\textless{}{-}} \ControlFlowTok{function}\NormalTok{(file\_name)\{}
\NormalTok{  log\_file\_name }\OtherTok{\textless{}{-}} \FunctionTok{as.character}\NormalTok{(}\FunctionTok{Sys.time}\NormalTok{()) }\SpecialCharTok{|\textgreater{}} 
    \FunctionTok{str\_replace\_all}\NormalTok{(}\StringTok{\textquotesingle{}:\textquotesingle{}}\NormalTok{, }\StringTok{\textquotesingle{}\_\textquotesingle{}}\NormalTok{) }\SpecialCharTok{|\textgreater{}} 
    \FunctionTok{str\_replace}\NormalTok{(}\StringTok{\textquotesingle{} \textquotesingle{}}\NormalTok{, }\StringTok{\textquotesingle{}T\textquotesingle{}}\NormalTok{) }\SpecialCharTok{|\textgreater{}}
    \FunctionTok{str\_c}\NormalTok{(file\_name)}
  
  \FunctionTok{log\_open}\NormalTok{(}\AttributeTok{file\_name =}\NormalTok{ log\_file\_name)}
\NormalTok{\}}
\NormalTok{print\_start\_date }\OtherTok{\textless{}{-}} \ControlFlowTok{function}\NormalTok{()\{}
  \FunctionTok{print}\NormalTok{(}\FunctionTok{date}\NormalTok{())}
  \FunctionTok{Sys.time}\NormalTok{()}
\NormalTok{\}}
\NormalTok{put\_start\_date }\OtherTok{\textless{}{-}} \ControlFlowTok{function}\NormalTok{()\{}
  \FunctionTok{put}\NormalTok{(}\FunctionTok{date}\NormalTok{())}
  \FunctionTok{Sys.time}\NormalTok{()}
\NormalTok{\}}
\NormalTok{print\_end\_date }\OtherTok{\textless{}{-}} \ControlFlowTok{function}\NormalTok{(start)\{}
  \FunctionTok{print}\NormalTok{(}\FunctionTok{date}\NormalTok{())}
  \FunctionTok{print}\NormalTok{(}\FunctionTok{Sys.time}\NormalTok{() }\SpecialCharTok{{-}}\NormalTok{ start)}
\NormalTok{\}}
\NormalTok{put\_end\_date }\OtherTok{\textless{}{-}} \ControlFlowTok{function}\NormalTok{(start)\{}
  \FunctionTok{put}\NormalTok{(}\FunctionTok{date}\NormalTok{())}
  \FunctionTok{put}\NormalTok{(}\FunctionTok{Sys.time}\NormalTok{() }\SpecialCharTok{{-}}\NormalTok{ start)}
\NormalTok{\}}

\NormalTok{msg.set\_arg }\OtherTok{\textless{}{-}} \ControlFlowTok{function}\NormalTok{(msg\_template, arg, }\AttributeTok{arg.name =} \StringTok{"\%1"}\NormalTok{) \{}
\NormalTok{  msg\_template }\SpecialCharTok{|\textgreater{}} 
    \FunctionTok{str\_replace\_all}\NormalTok{(arg.name, }\FunctionTok{as.character}\NormalTok{(arg))}
\NormalTok{\}}
\NormalTok{msg.glue }\OtherTok{\textless{}{-}} \ControlFlowTok{function}\NormalTok{(msg\_template, arg, }\AttributeTok{arg.name =} \StringTok{"\%1"}\NormalTok{)\{}
\NormalTok{  msg\_template }\SpecialCharTok{|\textgreater{}}
    \FunctionTok{msg.set\_arg}\NormalTok{(arg, arg.name) }\SpecialCharTok{|\textgreater{}}
    \FunctionTok{str\_glue}\NormalTok{()}
\NormalTok{\}}

\NormalTok{print\_log }\OtherTok{\textless{}{-}} \ControlFlowTok{function}\NormalTok{(msg)\{}
  \FunctionTok{print}\NormalTok{(}\FunctionTok{str\_glue}\NormalTok{(msg))}
\NormalTok{\}}
\NormalTok{put\_log }\OtherTok{\textless{}{-}} \ControlFlowTok{function}\NormalTok{(msg)\{}
  \FunctionTok{put}\NormalTok{(}\FunctionTok{str\_glue}\NormalTok{(msg))}
\NormalTok{\}}

\NormalTok{get\_log1 }\OtherTok{\textless{}{-}} \ControlFlowTok{function}\NormalTok{(msg\_template, arg1) \{}
  \FunctionTok{str\_glue}\NormalTok{(}\FunctionTok{str\_replace\_all}\NormalTok{(msg\_template, }\StringTok{"\%1"}\NormalTok{, }\FunctionTok{as.character}\NormalTok{(arg1)))}
\NormalTok{\}}
\NormalTok{print\_log1 }\OtherTok{\textless{}{-}} \ControlFlowTok{function}\NormalTok{(msg\_template, arg1)\{}
  \FunctionTok{print}\NormalTok{(}\FunctionTok{get\_log1}\NormalTok{(msg\_template, arg1))}
\NormalTok{\}}
\NormalTok{put\_log1 }\OtherTok{\textless{}{-}} \ControlFlowTok{function}\NormalTok{(msg\_template, arg1)\{}
  \FunctionTok{put}\NormalTok{(}\FunctionTok{get\_log1}\NormalTok{(msg\_template, arg1))}
\NormalTok{\}}

\NormalTok{get\_log2 }\OtherTok{\textless{}{-}} \ControlFlowTok{function}\NormalTok{(msg\_template, arg1, arg2) \{}
\NormalTok{  msg\_template }\SpecialCharTok{|\textgreater{}} 
    \FunctionTok{str\_replace\_all}\NormalTok{(}\StringTok{"\%1"}\NormalTok{, }\FunctionTok{as.character}\NormalTok{(arg1)) }\SpecialCharTok{|\textgreater{}}
    \FunctionTok{str\_replace\_all}\NormalTok{(}\StringTok{"\%2"}\NormalTok{, }\FunctionTok{as.character}\NormalTok{(arg2)) }\SpecialCharTok{|\textgreater{}}
    \FunctionTok{str\_glue}\NormalTok{()}
\NormalTok{\}}
\NormalTok{print\_log2 }\OtherTok{\textless{}{-}} \ControlFlowTok{function}\NormalTok{(msg\_template, arg1, arg2)\{}
  \FunctionTok{print}\NormalTok{(}\FunctionTok{get\_log1}\NormalTok{(msg\_template, arg1, arg2))}
\NormalTok{\}}
\NormalTok{put\_log2 }\OtherTok{\textless{}{-}} \ControlFlowTok{function}\NormalTok{(msg\_template, arg1, arg2)\{}
  \FunctionTok{put}\NormalTok{(}\FunctionTok{get\_log1}\NormalTok{(msg\_template, arg1, arg2))}
\NormalTok{\}}

\CommentTok{\# ...}
\end{Highlighting}
\end{Shaded}

\begin{noteblock}
The full source code of these functions is available in the
\href{https://github.com/AzKurban-edX-DS/Capstone-MovieLens/blob/main/r/src/capstone-movielens.main.R\#L20}{Logging
Helper functions} section of the
\href{https://github.com/AzKurban-edX-DS/Capstone-MovieLens/blob/main/r/src/capstone-movielens.main.R}{capstone-movielens.main.R}
script on \emph{GitHub}.

\end{noteblock}

\subsubsection{Preparing train and set
datasets}\label{preparing-train-and-set-datasets}

\hfill\break

We will split the \texttt{edx} dataset into a training set, which we
will use to build and train our models, and a test set in which we will
compute the accuracy of our predictions, the way described in
\href{https://rafalab.dfci.harvard.edu/dsbook-part-2/highdim/regularization.html\#movielens-data}{Section
23.1.1 Movielens data} of the \emph{Course Textbook} mentioned
above\autocite{IDS2_23-1-1}.We will also use the \emph{5-Fold Cross
Validation} method as described in
\href{https://rafalab.dfci.harvard.edu/dsbook-part-2/ml/resampling-methods.html\#cross-validation}{Section
29.6 Cross validation} of the \emph{Course Textbook}. To prepare
datasets for processing, we will use the following functions,
specifically designed for these operations:

\begin{Shaded}
\begin{Highlighting}[]
\FunctionTok{make\_source\_datasets}\NormalTok{()}
\FunctionTok{init\_source\_datasets}\NormalTok{()}
\end{Highlighting}
\end{Shaded}

\begin{noteblock}
The full source code of the function listed above is available in the
\href{https://github.com/AzKurban-edX-DS/Capstone-MovieLens/blob/main/r/src/support-functions/data.helper.functions.R\#L86}{Initialize
input datasets} section of the
\href{https://github.com/AzKurban-edX-DS/Capstone-MovieLens/blob/main/r/src/support-functions/data.helper.functions.R}{data.helper.functions.R}
script on \emph{GitHub}.

\end{noteblock}

\paragraph{\texorpdfstring{The \texttt{make\_source\_datasets}
function}{The make\_source\_datasets function}}\label{the-make_source_datasets-function}

\hfill\break

Let's take a closer look at the objects we will receive as a result of
executing this function.

\begin{Shaded}
\begin{Highlighting}[]
\NormalTok{make\_source\_datasets }\OtherTok{\textless{}{-}} \ControlFlowTok{function}\NormalTok{()\{}
  \CommentTok{\# ...}
  \FunctionTok{list}\NormalTok{(}\AttributeTok{edx\_CV =}\NormalTok{ edx\_CV,}
       \AttributeTok{edx.mx =}\NormalTok{ edx.mx,}
       \AttributeTok{edx.sgr =}\NormalTok{ edx.sgr,}
       \AttributeTok{tuning\_sets =}\NormalTok{ tuning\_sets,}
       \AttributeTok{movie\_map =}\NormalTok{ movie\_map,}
       \AttributeTok{date\_days\_map =}\NormalTok{ date\_days\_map)}
\NormalTok{\}}
\end{Highlighting}
\end{Shaded}

\subparagraph{\texorpdfstring{\texttt{edx.mx} Matrix
Object}{edx.mx Matrix Object}}\label{edx.mx-matrix-object}

\hfill\break

We will use the array representation described in
\href{https://rafalab.dfci.harvard.edu/dsbook-part-2/linear-models/treatment-effect-models.html\#sec-anova}{Section
17.5 of the Textbook}, for the training data: we denote ranking for
movie \(j\) by user \(i\) as \(y_{i,j}\). To create this matrix, we use
\texttt{tidyr::pivot\_wider} function:

\begin{Shaded}
\begin{Highlighting}[]
  \FunctionTok{put\_log}\NormalTok{(}\StringTok{"Function: \textasciigrave{}make\_source\_datasets\textasciigrave{}: Creating Rating Matrix from \textasciigrave{}edx\textasciigrave{} dataset..."}\NormalTok{)}
\NormalTok{  edx.mx }\OtherTok{\textless{}{-}}\NormalTok{ edx }\SpecialCharTok{|\textgreater{}} 
    \FunctionTok{mutate}\NormalTok{(}\AttributeTok{userId =} \FunctionTok{factor}\NormalTok{(userId),}
           \AttributeTok{movieId =} \FunctionTok{factor}\NormalTok{(movieId)) }\SpecialCharTok{|\textgreater{}}
    \FunctionTok{select}\NormalTok{(movieId, userId, rating) }\SpecialCharTok{|\textgreater{}}
    \FunctionTok{pivot\_wider}\NormalTok{(}\AttributeTok{names\_from =}\NormalTok{ movieId, }\AttributeTok{values\_from =}\NormalTok{ rating) }\SpecialCharTok{|\textgreater{}}
    \FunctionTok{column\_to\_rownames}\NormalTok{(}\StringTok{"userId"}\NormalTok{) }\SpecialCharTok{|\textgreater{}}
    \FunctionTok{as.matrix}\NormalTok{()}
  
  \FunctionTok{put\_log}\NormalTok{(}\StringTok{"Function: \textasciigrave{}make\_source\_datasets\textasciigrave{}:}
\StringTok{Matrix created: \textasciigrave{}edx.mx\textasciigrave{} of the following dimentions:"}\NormalTok{)}
\end{Highlighting}
\end{Shaded}

\begin{Shaded}
\begin{Highlighting}[]
  \FunctionTok{str}\NormalTok{(edx.mx)}
\end{Highlighting}
\end{Shaded}

\begin{verbatim}
##  num [1:69878, 1:10677] 5 NA NA NA NA NA NA NA NA NA ...
##  - attr(*, "dimnames")=List of 2
##   ..$ : chr [1:69878] "1" "2" "3" "4" ...
##   ..$ : chr [1:10677] "122" "185" "292" "316" ...
\end{verbatim}

\subparagraph{\texorpdfstring{\texttt{edx.sgr}
Object}{edx.sgr Object}}\label{edx.sgr-object}

\hfill\break

To account for the Movie Genre Effect more accurately, we need a dataset
with split rows for movies belonging to multiple genres:

\begin{Shaded}
\begin{Highlighting}[]
  \FunctionTok{put\_log}\NormalTok{(}\StringTok{"Function: \textasciigrave{}make\_source\_datasets\textasciigrave{}: }
\StringTok{To account for the Movie Genre Effect, we need a dataset with split rows }
\StringTok{for movies belonging to multiple genres."}\NormalTok{)}
\NormalTok{  edx.sgr }\OtherTok{\textless{}{-}} \FunctionTok{splitGenreRows}\NormalTok{(edx)}
\end{Highlighting}
\end{Shaded}

\begin{Shaded}
\begin{Highlighting}[]
\FunctionTok{str}\NormalTok{(edx.sgr)}
\end{Highlighting}
\end{Shaded}

\begin{verbatim}
## tibble [23,371,423 x 6] (S3: tbl_df/tbl/data.frame)
##  $ userId   : int [1:23371423] 1 1 1 1 1 1 1 1 1 1 ...
##  $ movieId  : int [1:23371423] 122 122 185 185 185 292 292 292 292 316 ...
##  $ rating   : num [1:23371423] 5 5 5 5 5 5 5 5 5 5 ...
##  $ timestamp: int [1:23371423] 838985046 838985046 838983525 838983525 838983525 838983421 838983421 838983421 838983421 838983392 ...
##  $ title    : chr [1:23371423] "Boomerang (1992)" "Boomerang (1992)" "Net, The (1995)" "Net, The (1995)" ...
##  $ genres   : chr [1:23371423] "Comedy" "Romance" "Action" "Crime" ...
\end{verbatim}

\begin{Shaded}
\begin{Highlighting}[]
\FunctionTok{summary}\NormalTok{(edx.sgr)}
\end{Highlighting}
\end{Shaded}

\begin{verbatim}
##      userId         movieId          rating        timestamp            title              genres         
##  Min.   :    1   Min.   :    1   Min.   :0.500   Min.   :7.897e+08   Length:23371423    Length:23371423   
##  1st Qu.:18140   1st Qu.:  616   1st Qu.:3.000   1st Qu.:9.472e+08   Class :character   Class :character  
##  Median :35784   Median : 1748   Median :4.000   Median :1.042e+09   Mode  :character   Mode  :character  
##  Mean   :35886   Mean   : 4277   Mean   :3.527   Mean   :1.035e+09                                        
##  3rd Qu.:53638   3rd Qu.: 3635   3rd Qu.:4.000   3rd Qu.:1.131e+09                                        
##  Max.   :71567   Max.   :65133   Max.   :5.000   Max.   :1.231e+09
\end{verbatim}

Note that we use the \texttt{splitGenreRows} function to split rows of
the original dataset:

\begin{Shaded}
\begin{Highlighting}[]
\NormalTok{splitGenreRows }\OtherTok{\textless{}{-}} \ControlFlowTok{function}\NormalTok{(data)\{}
  \FunctionTok{put}\NormalTok{(}\StringTok{"Splitting dataset rows related to multiple genres..."}\NormalTok{)}
\NormalTok{  start }\OtherTok{\textless{}{-}} \FunctionTok{put\_start\_date}\NormalTok{()}
\NormalTok{  gs\_splitted }\OtherTok{\textless{}{-}}\NormalTok{ data }\SpecialCharTok{|\textgreater{}}
    \FunctionTok{separate\_rows}\NormalTok{(genres, }\AttributeTok{sep =} \StringTok{"}\SpecialCharTok{\textbackslash{}\textbackslash{}}\StringTok{|"}\NormalTok{)}
  \FunctionTok{put}\NormalTok{(}\StringTok{"Dataset rows related to multiple genres have been splitted to have single genre per row."}\NormalTok{)}
  \FunctionTok{put\_end\_date}\NormalTok{(start)}
\NormalTok{  gs\_splitted}
\NormalTok{\}}
\end{Highlighting}
\end{Shaded}

\begin{noteblock}
The source code of the function mentioned above is also available in the
\href{https://github.com/AzKurban-edX-DS/Capstone-MovieLens/blob/main/r/src/support-functions/data.helper.functions.R\#L86}{Initialize
input datasets} section of the
\href{https://github.com/AzKurban-edX-DS/Capstone-MovieLens/blob/main/r/src/support-functions/data.helper.functions.R}{data.helper.functions.R}
script on \emph{GitHub}.

\end{noteblock}

\subparagraph{\texorpdfstring{\texttt{movie\_map}
Object}{movie\_map Object}}\label{movie_map-object}

\hfill\break

To be able to map movie IDs to titles we create the following lookup
table:

\begin{Shaded}
\begin{Highlighting}[]
\NormalTok{movie\_map }\OtherTok{\textless{}{-}}\NormalTok{ edx }\SpecialCharTok{|\textgreater{}} \FunctionTok{select}\NormalTok{(movieId, title, genres) }\SpecialCharTok{|\textgreater{}} 
    \FunctionTok{distinct}\NormalTok{(movieId, }\AttributeTok{.keep\_all =} \ConstantTok{TRUE}\NormalTok{)}
  
  \FunctionTok{put\_log}\NormalTok{(}\StringTok{"Function: \textasciigrave{}make\_source\_datasets\textasciigrave{}: Dataset created: movie\_map"}\NormalTok{)}
\end{Highlighting}
\end{Shaded}

\begin{Shaded}
\begin{Highlighting}[]
\FunctionTok{str}\NormalTok{(movie\_map)}
\end{Highlighting}
\end{Shaded}

\begin{verbatim}
## 'data.frame':    10677 obs. of  3 variables:
##  $ movieId: int  122 185 292 316 329 355 356 362 364 370 ...
##  $ title  : chr  "Boomerang (1992)" "Net, The (1995)" "Outbreak (1995)" "Stargate (1994)" ...
##  $ genres : chr  "Comedy|Romance" "Action|Crime|Thriller" "Action|Drama|Sci-Fi|Thriller" "Action|Adventure|Sci-Fi" ...
\end{verbatim}

\begin{Shaded}
\begin{Highlighting}[]
\FunctionTok{summary}\NormalTok{(movie\_map)}
\end{Highlighting}
\end{Shaded}

\begin{verbatim}
##     movieId         title              genres         
##  Min.   :    1   Length:10677       Length:10677      
##  1st Qu.: 2754   Class :character   Class :character  
##  Median : 5434   Mode  :character   Mode  :character  
##  Mean   :13105                                        
##  3rd Qu.: 8710                                        
##  Max.   :65133
\end{verbatim}

Note that titles cannot be considered unique, so we can't use them as
IDs\autocite{IDS2_23-1-1}.

\subparagraph{\texorpdfstring{\texttt{date\_days\_map}
Object}{date\_days\_map Object}}\label{date_days_map-object}

\hfill\break

We have a \texttt{timestamp} field in the \texttt{edx} dataset. To be
able to map the date, year, and number of days since the earliest record
in the \texttt{edx} dataset with the corresponding value in this field,
we create the following lookup table:

\begin{Shaded}
\begin{Highlighting}[]
  \FunctionTok{put\_log}\NormalTok{(}\StringTok{"Function: \textasciigrave{}make\_source\_datasets\textasciigrave{}: Creating Date{-}Days Map dataset..."}\NormalTok{)}
\NormalTok{  date\_days\_map }\OtherTok{\textless{}{-}}\NormalTok{ edx }\SpecialCharTok{|\textgreater{}}
    \FunctionTok{mutate}\NormalTok{(}\AttributeTok{date\_time =} \FunctionTok{as\_datetime}\NormalTok{(timestamp)) }\SpecialCharTok{|\textgreater{}}
    \FunctionTok{mutate}\NormalTok{(}\AttributeTok{date =} \FunctionTok{as\_date}\NormalTok{(date\_time)) }\SpecialCharTok{|\textgreater{}}
    \FunctionTok{mutate}\NormalTok{(}\AttributeTok{year =} \FunctionTok{year}\NormalTok{(date\_time)) }\SpecialCharTok{|\textgreater{}}
    \FunctionTok{mutate}\NormalTok{(}\AttributeTok{days =} \FunctionTok{as.integer}\NormalTok{(date }\SpecialCharTok{{-}} \FunctionTok{min}\NormalTok{(date))) }\SpecialCharTok{|\textgreater{}}
    \FunctionTok{select}\NormalTok{(timestamp, date\_time, date, year, days) }\SpecialCharTok{|\textgreater{}}
    \FunctionTok{distinct}\NormalTok{(timestamp, }\AttributeTok{.keep\_all =} \ConstantTok{TRUE}\NormalTok{)}
  
  \FunctionTok{put\_log}\NormalTok{(}\StringTok{"Function: \textasciigrave{}make\_source\_datasets\textasciigrave{}: Dataset created: date\_days\_map"}\NormalTok{)}
\end{Highlighting}
\end{Shaded}

\begin{Shaded}
\begin{Highlighting}[]
  \FunctionTok{str}\NormalTok{(date\_days\_map)}
\end{Highlighting}
\end{Shaded}

\begin{verbatim}
## 'data.frame':    6519590 obs. of  5 variables:
##  $ timestamp: int  838985046 838983525 838983421 838983392 838984474 838983653 838984885 838983707 838984596 838983834 ...
##  $ date_time: POSIXct, format: "1996-08-02 11:24:06" "1996-08-02 10:58:45" "1996-08-02 10:57:01" "1996-08-02 10:56:32" ...
##  $ date     : Date, format: "1996-08-02" "1996-08-02" "1996-08-02" "1996-08-02" ...
##  $ year     : num  1996 1996 1996 1996 1996 ...
##  $ days     : int  571 571 571 571 571 571 571 571 571 571 ...
\end{verbatim}

\begin{Shaded}
\begin{Highlighting}[]
  \FunctionTok{summary}\NormalTok{(date\_days\_map)}
\end{Highlighting}
\end{Shaded}

\begin{verbatim}
##    timestamp           date_time                           date                 year           days     
##  Min.   :7.897e+08   Min.   :1995-01-09 11:46:49.00   Min.   :1995-01-09   Min.   :1995   Min.   :   0  
##  1st Qu.:9.783e+08   1st Qu.:2001-01-01 05:05:01.75   1st Qu.:2001-01-01   1st Qu.:2001   1st Qu.:2184  
##  Median :1.091e+09   Median :2004-08-03 01:08:18.50   Median :2004-08-03   Median :2004   Median :3494  
##  Mean   :1.066e+09   Mean   :2003-10-10 23:15:02.07   Mean   :2003-10-10   Mean   :2003   Mean   :3196  
##  3rd Qu.:1.152e+09   3rd Qu.:2006-07-04 20:41:57.50   3rd Qu.:2006-07-04   3rd Qu.:2006   3rd Qu.:4194  
##  Max.   :1.231e+09   Max.   :2009-01-05 05:02:16.00   Max.   :2009-01-05   Max.   :2009   Max.   :5110
\end{verbatim}

\subparagraph{\texorpdfstring{\texttt{edx\_CV}
Object}{edx\_CV Object}}\label{edx_cv-object}

\hfill\break

Here we have a list of sample objects we need to perform the
\emph{5-Fold Cross Validation} as explained in
\href{https://rafalab.dfci.harvard.edu/dsbook-part-2/ml/resampling-methods.html\#k-fold-cross-validation}{Section
29.6.1 K-fold cross validation} of the \emph{Course Textbook}:

\begin{Shaded}
\begin{Highlighting}[]
\NormalTok{  start }\OtherTok{\textless{}{-}} \FunctionTok{put\_start\_date}\NormalTok{()}
\NormalTok{  edx\_CV }\OtherTok{\textless{}{-}} \FunctionTok{lapply}\NormalTok{(kfold\_index,  }\ControlFlowTok{function}\NormalTok{(fold\_i)\{}
    
    \FunctionTok{put\_log1}\NormalTok{(}\StringTok{"Method \textasciigrave{}make\_source\_datasets\textasciigrave{}: }
\StringTok{Creating K{-}Fold Cross Validation Datasets, Fold \%1"}\NormalTok{, fold\_i)}
    
    \CommentTok{\#\textgreater{} We split the initial datasets into training sets, which we will use to build }
    \CommentTok{\#\textgreater{} and train our models, and validation sets in which we will compute the accuracy }
    \CommentTok{\#\textgreater{} of our predictions, the way described in the \textasciigrave{}Section 23.1.1 Movielens data\textasciigrave{}}
    \CommentTok{\#\textgreater{} (https://rafalab.dfci.harvard.edu/dsbook{-}part{-}2/highdim/regularization.html\#movielens{-}data) }
    \CommentTok{\#\textgreater{} of the Course Textbook.}
    
\NormalTok{    split\_sets }\OtherTok{\textless{}{-}}\NormalTok{ edx }\SpecialCharTok{|\textgreater{}}
      \FunctionTok{sample\_train\_validation\_sets}\NormalTok{(fold\_i}\SpecialCharTok{*}\DecValTok{1000}\NormalTok{)}
    
\NormalTok{    train\_set }\OtherTok{\textless{}{-}}\NormalTok{ split\_sets}\SpecialCharTok{$}\NormalTok{train\_set}
\NormalTok{    validation\_set }\OtherTok{\textless{}{-}}\NormalTok{ split\_sets}\SpecialCharTok{$}\NormalTok{validation\_set}
    
    \FunctionTok{put\_log}\NormalTok{(}\StringTok{"Function: \textasciigrave{}make\_source\_datasets\textasciigrave{}: }
\StringTok{Sampling 20\% from the split{-}row version of the \textasciigrave{}edx\textasciigrave{} dataset..."}\NormalTok{)}
\NormalTok{    split\_sets.gs }\OtherTok{\textless{}{-}}\NormalTok{ edx.sgr }\SpecialCharTok{|\textgreater{}}
      \FunctionTok{sample\_train\_validation\_sets}\NormalTok{(fold\_i}\SpecialCharTok{*}\DecValTok{2000}\NormalTok{)}

\NormalTok{    train.sgr }\OtherTok{\textless{}{-}}\NormalTok{ split\_sets.gs}\SpecialCharTok{$}\NormalTok{train\_set}
\NormalTok{    validation.sgr }\OtherTok{\textless{}{-}}\NormalTok{ split\_sets.gs}\SpecialCharTok{$}\NormalTok{validation\_set}
    
    \CommentTok{\# put\_log("Function: \textasciigrave{}make\_source\_datasets\textasciigrave{}: Dataset created: validation.sgr")}
    \CommentTok{\# put(summary(validation.sgr))}
    
    \CommentTok{\#\textgreater{} We will use the array representation described in \textasciigrave{}Section 17.5 of the Textbook\textasciigrave{}}
    \CommentTok{\#\textgreater{} (https://rafalab.dfci.harvard.edu/dsbook{-}part{-}2/linear{-}models/treatment{-}effect{-}models.html\#sec{-}anova), }
    \CommentTok{\#\textgreater{} for the training data. }
    \CommentTok{\#\textgreater{} To create this matrix, we use \textasciigrave{}tidyr::pivot\_wider\textasciigrave{} function:}
    
    \FunctionTok{put\_log}\NormalTok{(}\StringTok{"Function: \textasciigrave{}make\_source\_datasets\textasciigrave{}: Creating Rating Matrix from Train Set..."}\NormalTok{)}
\NormalTok{    train\_mx }\OtherTok{\textless{}{-}}\NormalTok{ train\_set }\SpecialCharTok{|\textgreater{}} 
      \FunctionTok{mutate}\NormalTok{(}\AttributeTok{userId =} \FunctionTok{factor}\NormalTok{(userId),}
             \AttributeTok{movieId =} \FunctionTok{factor}\NormalTok{(movieId)) }\SpecialCharTok{|\textgreater{}}
      \FunctionTok{select}\NormalTok{(movieId, userId, rating) }\SpecialCharTok{|\textgreater{}}
      \FunctionTok{pivot\_wider}\NormalTok{(}\AttributeTok{names\_from =}\NormalTok{ movieId, }\AttributeTok{values\_from =}\NormalTok{ rating) }\SpecialCharTok{|\textgreater{}}
      \FunctionTok{column\_to\_rownames}\NormalTok{(}\StringTok{"userId"}\NormalTok{) }\SpecialCharTok{|\textgreater{}}
      \FunctionTok{as.matrix}\NormalTok{()}
    
    \FunctionTok{put\_log}\NormalTok{(}\StringTok{"Function: \textasciigrave{}make\_source\_datasets\textasciigrave{}:}
\StringTok{Matrix created: \textasciigrave{}train\_mx\textasciigrave{} of the following dimentions:"}\NormalTok{)}
    \FunctionTok{put}\NormalTok{(}\FunctionTok{dim}\NormalTok{(train\_mx))}

    \FunctionTok{list}\NormalTok{(}\AttributeTok{train\_set =}\NormalTok{ train\_set,}
         \AttributeTok{train\_mx =}\NormalTok{ train\_mx,}
         \AttributeTok{train.sgr =}\NormalTok{ train.sgr,}
         \AttributeTok{validation\_set =}\NormalTok{ validation\_set)}
\NormalTok{  \})}
  \FunctionTok{put\_end\_date}\NormalTok{(start)}
  \FunctionTok{put\_log}\NormalTok{(}\StringTok{"Function: \textasciigrave{}make\_source\_datasets\textasciigrave{}: }
\StringTok{Set of K{-}Fold Cross Validation datasets created: edx\_CV"}\NormalTok{)}
\end{Highlighting}
\end{Shaded}

\begin{Shaded}
\begin{Highlighting}[]
\FunctionTok{str}\NormalTok{(edx\_CV)}
\end{Highlighting}
\end{Shaded}

\begin{verbatim}
## List of 5
##  $ :List of 4
##   ..$ train_set     :'data.frame':   7172311 obs. of  6 variables:
##   .. ..$ userId   : int [1:7172311] 1 1 1 1 1 1 1 1 1 1 ...
##   .. ..$ movieId  : int [1:7172311] 122 185 292 329 356 362 364 370 420 466 ...
##   .. ..$ rating   : num [1:7172311] 5 5 5 5 5 5 5 5 5 5 ...
##   .. ..$ timestamp: int [1:7172311] 838985046 838983525 838983421 838983392 838983653 838984885 838983707 838984596 838983834 838984679 ...
##   .. ..$ title    : chr [1:7172311] "Boomerang (1992)" "Net, The (1995)" "Outbreak (1995)" "Star Trek: Generations (1994)" ...
##   .. ..$ genres   : chr [1:7172311] "Comedy|Romance" "Action|Crime|Thriller" "Action|Drama|Sci-Fi|Thriller" "Action|Adventure|Drama|Sci-Fi" ...
##   ..$ train_mx      : num [1:69878, 1:10677] 5 NA NA NA NA NA NA NA NA NA ...
##   .. ..- attr(*, "dimnames")=List of 2
##   .. .. ..$ : chr [1:69878] "1" "2" "3" "4" ...
##   .. .. ..$ : chr [1:10677] "122" "185" "292" "329" ...
##   ..$ train.sgr     : tibble [18,669,190 x 6] (S3: tbl_df/tbl/data.frame)
##   .. ..$ userId   : int [1:18669190] 1 1 1 1 1 1 1 1 1 1 ...
##   .. ..$ movieId  : int [1:18669190] 122 122 185 185 292 292 292 292 316 316 ...
##   .. ..$ rating   : num [1:18669190] 5 5 5 5 5 5 5 5 5 5 ...
##   .. ..$ timestamp: int [1:18669190] 838985046 838985046 838983525 838983525 838983421 838983421 838983421 838983421 838983392 838983392 ...
##   .. ..$ title    : chr [1:18669190] "Boomerang (1992)" "Boomerang (1992)" "Net, The (1995)" "Net, The (1995)" ...
##   .. ..$ genres   : chr [1:18669190] "Comedy" "Romance" "Action" "Crime" ...
##   ..$ validation_set:'data.frame':   1827744 obs. of  6 variables:
##   .. ..$ userId   : int [1:1827744] 1 1 1 1 2 2 2 2 3 3 ...
##   .. ..$ movieId  : int [1:1827744] 316 355 377 588 260 376 648 1049 110 1252 ...
##   .. ..$ rating   : num [1:1827744] 5 5 5 5 5 3 2 3 4.5 4 ...
##   .. ..$ timestamp: int [1:1827744] 838983392 838984474 838983834 838983339 868244562 868245920 868244699 868245920 1136075500 1133571071 ...
##   .. ..$ title    : chr [1:1827744] "Stargate (1994)" "Flintstones, The (1994)" "Speed (1994)" "Aladdin (1992)" ...
##   .. ..$ genres   : chr [1:1827744] "Action|Adventure|Sci-Fi" "Children|Comedy|Fantasy" "Action|Romance|Thriller" "Adventure|Animation|Children|Comedy|Musical" ...
##  $ :List of 4
##   ..$ train_set     :'data.frame':   7172306 obs. of  6 variables:
##   .. ..$ userId   : int [1:7172306] 1 1 1 1 1 1 1 1 1 1 ...
##   .. ..$ movieId  : int [1:7172306] 122 185 292 316 329 355 356 364 370 377 ...
##   .. ..$ rating   : num [1:7172306] 5 5 5 5 5 5 5 5 5 5 ...
##   .. ..$ timestamp: int [1:7172306] 838985046 838983525 838983421 838983392 838983392 838984474 838983653 838983707 838984596 838983834 ...
##   .. ..$ title    : chr [1:7172306] "Boomerang (1992)" "Net, The (1995)" "Outbreak (1995)" "Stargate (1994)" ...
##   .. ..$ genres   : chr [1:7172306] "Comedy|Romance" "Action|Crime|Thriller" "Action|Drama|Sci-Fi|Thriller" "Action|Adventure|Sci-Fi" ...
##   ..$ train_mx      : num [1:69878, 1:10677] 5 NA NA NA NA NA NA NA NA NA ...
##   .. ..- attr(*, "dimnames")=List of 2
##   .. .. ..$ : chr [1:69878] "1" "2" "3" "4" ...
##   .. .. ..$ : chr [1:10677] "122" "185" "292" "316" ...
##   ..$ train.sgr     : tibble [18,669,201 x 6] (S3: tbl_df/tbl/data.frame)
##   .. ..$ userId   : int [1:18669201] 1 1 1 1 1 1 1 1 1 1 ...
##   .. ..$ movieId  : int [1:18669201] 122 122 185 185 185 292 292 316 316 329 ...
##   .. ..$ rating   : num [1:18669201] 5 5 5 5 5 5 5 5 5 5 ...
##   .. ..$ timestamp: int [1:18669201] 838985046 838985046 838983525 838983525 838983525 838983421 838983421 838983392 838983392 838983392 ...
##   .. ..$ title    : chr [1:18669201] "Boomerang (1992)" "Boomerang (1992)" "Net, The (1995)" "Net, The (1995)" ...
##   .. ..$ genres   : chr [1:18669201] "Comedy" "Romance" "Action" "Crime" ...
##   ..$ validation_set:'data.frame':   1827749 obs. of  6 variables:
##   .. ..$ userId   : int [1:1827749] 1 1 1 1 2 2 2 2 3 3 ...
##   .. ..$ movieId  : int [1:1827749] 362 520 539 594 539 590 733 1210 1252 1408 ...
##   .. ..$ rating   : num [1:1827749] 5 5 5 5 3 5 3 4 4 3.5 ...
##   .. ..$ timestamp: int [1:1827749] 838984885 838984679 838984068 838984679 868246262 868245608 868244562 868245644 1133571071 1133571145 ...
##   .. ..$ title    : chr [1:1827749] "Jungle Book, The (1994)" "Robin Hood: Men in Tights (1993)" "Sleepless in Seattle (1993)" "Snow White and the Seven Dwarfs (1937)" ...
##   .. ..$ genres   : chr [1:1827749] "Adventure|Children|Romance" "Comedy" "Comedy|Drama|Romance" "Animation|Children|Drama|Fantasy|Musical" ...
##  $ :List of 4
##   ..$ train_set     :'data.frame':   7172307 obs. of  6 variables:
##   .. ..$ userId   : int [1:7172307] 1 1 1 1 1 1 1 1 1 1 ...
##   .. ..$ movieId  : int [1:7172307] 122 185 292 316 329 355 362 370 377 420 ...
##   .. ..$ rating   : num [1:7172307] 5 5 5 5 5 5 5 5 5 5 ...
##   .. ..$ timestamp: int [1:7172307] 838985046 838983525 838983421 838983392 838983392 838984474 838984885 838984596 838983834 838983834 ...
##   .. ..$ title    : chr [1:7172307] "Boomerang (1992)" "Net, The (1995)" "Outbreak (1995)" "Stargate (1994)" ...
##   .. ..$ genres   : chr [1:7172307] "Comedy|Romance" "Action|Crime|Thriller" "Action|Drama|Sci-Fi|Thriller" "Action|Adventure|Sci-Fi" ...
##   ..$ train_mx      : num [1:69878, 1:10677] 5 NA NA NA NA NA NA NA NA NA ...
##   .. ..- attr(*, "dimnames")=List of 2
##   .. .. ..$ : chr [1:69878] "1" "2" "3" "4" ...
##   .. .. ..$ : chr [1:10677] "122" "185" "292" "316" ...
##   ..$ train.sgr     : tibble [18,669,195 x 6] (S3: tbl_df/tbl/data.frame)
##   .. ..$ userId   : int [1:18669195] 1 1 1 1 1 1 1 1 1 1 ...
##   .. ..$ movieId  : int [1:18669195] 122 122 185 185 185 292 292 292 316 329 ...
##   .. ..$ rating   : num [1:18669195] 5 5 5 5 5 5 5 5 5 5 ...
##   .. ..$ timestamp: int [1:18669195] 838985046 838985046 838983525 838983525 838983525 838983421 838983421 838983421 838983392 838983392 ...
##   .. ..$ title    : chr [1:18669195] "Boomerang (1992)" "Boomerang (1992)" "Net, The (1995)" "Net, The (1995)" ...
##   .. ..$ genres   : chr [1:18669195] "Comedy" "Romance" "Action" "Crime" ...
##   ..$ validation_set:'data.frame':   1827748 obs. of  6 variables:
##   .. ..$ userId   : int [1:1827748] 1 1 1 1 2 2 2 2 3 3 ...
##   .. ..$ movieId  : int [1:1827748] 356 364 539 616 590 719 780 786 151 213 ...
##   .. ..$ rating   : num [1:1827748] 5 5 5 5 5 3 3 3 4.5 5 ...
##   .. ..$ timestamp: int [1:1827748] 838983653 838983707 838984068 838984941 868245608 868246191 868244698 868244562 1133571026 1136075789 ...
##   .. ..$ title    : chr [1:1827748] "Forrest Gump (1994)" "Lion King, The (1994)" "Sleepless in Seattle (1993)" "Aristocats, The (1970)" ...
##   .. ..$ genres   : chr [1:1827748] "Comedy|Drama|Romance|War" "Adventure|Animation|Children|Drama|Musical" "Comedy|Drama|Romance" "Animation|Children" ...
##  $ :List of 4
##   ..$ train_set     :'data.frame':   7172311 obs. of  6 variables:
##   .. ..$ userId   : int [1:7172311] 1 1 1 1 1 1 1 1 1 1 ...
##   .. ..$ movieId  : int [1:7172311] 122 185 292 316 329 355 356 362 364 370 ...
##   .. ..$ rating   : num [1:7172311] 5 5 5 5 5 5 5 5 5 5 ...
##   .. ..$ timestamp: int [1:7172311] 838985046 838983525 838983421 838983392 838983392 838984474 838983653 838984885 838983707 838984596 ...
##   .. ..$ title    : chr [1:7172311] "Boomerang (1992)" "Net, The (1995)" "Outbreak (1995)" "Stargate (1994)" ...
##   .. ..$ genres   : chr [1:7172311] "Comedy|Romance" "Action|Crime|Thriller" "Action|Drama|Sci-Fi|Thriller" "Action|Adventure|Sci-Fi" ...
##   ..$ train_mx      : num [1:69878, 1:10677] 5 NA NA NA NA NA NA NA NA NA ...
##   .. ..- attr(*, "dimnames")=List of 2
##   .. .. ..$ : chr [1:69878] "1" "2" "3" "4" ...
##   .. .. ..$ : chr [1:10677] "122" "185" "292" "316" ...
##   ..$ train.sgr     : tibble [18,669,192 x 6] (S3: tbl_df/tbl/data.frame)
##   .. ..$ userId   : int [1:18669192] 1 1 1 1 1 1 1 1 1 1 ...
##   .. ..$ movieId  : int [1:18669192] 122 122 185 185 292 292 316 316 329 329 ...
##   .. ..$ rating   : num [1:18669192] 5 5 5 5 5 5 5 5 5 5 ...
##   .. ..$ timestamp: int [1:18669192] 838985046 838985046 838983525 838983525 838983421 838983421 838983392 838983392 838983392 838983392 ...
##   .. ..$ title    : chr [1:18669192] "Boomerang (1992)" "Boomerang (1992)" "Net, The (1995)" "Net, The (1995)" ...
##   .. ..$ genres   : chr [1:18669192] "Comedy" "Romance" "Action" "Thriller" ...
##   ..$ validation_set:'data.frame':   1827744 obs. of  6 variables:
##   .. ..$ userId   : int [1:1827744] 1 1 1 1 2 2 2 2 3 3 ...
##   .. ..$ movieId  : int [1:1827744] 377 520 588 616 110 648 1049 1356 1148 1276 ...
##   .. ..$ rating   : num [1:1827744] 5 5 5 5 5 2 3 3 4 3.5 ...
##   .. ..$ timestamp: int [1:1827744] 838983834 838984679 838983339 838984941 868245777 868244699 868245920 868244603 1133571121 1133571205 ...
##   .. ..$ title    : chr [1:1827744] "Speed (1994)" "Robin Hood: Men in Tights (1993)" "Aladdin (1992)" "Aristocats, The (1970)" ...
##   .. ..$ genres   : chr [1:1827744] "Action|Romance|Thriller" "Comedy" "Adventure|Animation|Children|Comedy|Musical" "Animation|Children" ...
##  $ :List of 4
##   ..$ train_set     :'data.frame':   7172301 obs. of  6 variables:
##   .. ..$ userId   : int [1:7172301] 1 1 1 1 1 1 1 1 1 1 ...
##   .. ..$ movieId  : int [1:7172301] 122 185 292 316 355 356 364 370 420 466 ...
##   .. ..$ rating   : num [1:7172301] 5 5 5 5 5 5 5 5 5 5 ...
##   .. ..$ timestamp: int [1:7172301] 838985046 838983525 838983421 838983392 838984474 838983653 838983707 838984596 838983834 838984679 ...
##   .. ..$ title    : chr [1:7172301] "Boomerang (1992)" "Net, The (1995)" "Outbreak (1995)" "Stargate (1994)" ...
##   .. ..$ genres   : chr [1:7172301] "Comedy|Romance" "Action|Crime|Thriller" "Action|Drama|Sci-Fi|Thriller" "Action|Adventure|Sci-Fi" ...
##   ..$ train_mx      : num [1:69878, 1:10677] 5 NA NA NA NA NA NA NA NA NA ...
##   .. ..- attr(*, "dimnames")=List of 2
##   .. .. ..$ : chr [1:69878] "1" "2" "3" "4" ...
##   .. .. ..$ : chr [1:10677] "122" "185" "292" "316" ...
##   ..$ train.sgr     : tibble [18,669,194 x 6] (S3: tbl_df/tbl/data.frame)
##   .. ..$ userId   : int [1:18669194] 1 1 1 1 1 1 1 1 1 1 ...
##   .. ..$ movieId  : int [1:18669194] 122 122 185 185 292 292 316 329 329 355 ...
##   .. ..$ rating   : num [1:18669194] 5 5 5 5 5 5 5 5 5 5 ...
##   .. ..$ timestamp: int [1:18669194] 838985046 838985046 838983525 838983525 838983421 838983421 838983392 838983392 838983392 838984474 ...
##   .. ..$ title    : chr [1:18669194] "Boomerang (1992)" "Boomerang (1992)" "Net, The (1995)" "Net, The (1995)" ...
##   .. ..$ genres   : chr [1:18669194] "Comedy" "Romance" "Crime" "Thriller" ...
##   ..$ validation_set:'data.frame':   1827754 obs. of  6 variables:
##   .. ..$ userId   : int [1:1827754] 1 1 1 1 2 2 2 2 3 3 ...
##   .. ..$ movieId  : int [1:1827754] 329 362 377 594 110 376 539 736 1252 1408 ...
##   .. ..$ rating   : num [1:1827754] 5 5 5 5 5 3 3 3 4 3.5 ...
##   .. ..$ timestamp: int [1:1827754] 838983392 838984885 838983834 838984679 868245777 868245920 868246262 868244698 1133571071 1133571145 ...
##   .. ..$ title    : chr [1:1827754] "Star Trek: Generations (1994)" "Jungle Book, The (1994)" "Speed (1994)" "Snow White and the Seven Dwarfs (1937)" ...
##   .. ..$ genres   : chr [1:1827754] "Action|Adventure|Drama|Sci-Fi" "Adventure|Children|Romance" "Action|Romance|Thriller" "Animation|Children|Drama|Fantasy|Musical" ...
\end{verbatim}

\begin{noteblock}
This code snippet is a
\href{https://github.com/AzKurban-edX-DS/Capstone-MovieLens/blob/main/r/src/support-functions/data.helper.functions.R\#L140}{part}
of the
\href{https://github.com/AzKurban-edX-DS/Capstone-MovieLens/blob/main/r/src/support-functions/data.helper.functions.R\#L87}{make\_source\_datasets}
\emph{function} code described above.

\end{noteblock}

Note that we used the \texttt{sample\_train\_validation\_sets} function
call to split the original dataset (\texttt{edx} in this case):

\begin{Shaded}
\begin{Highlighting}[]
\NormalTok{    split\_sets }\OtherTok{\textless{}{-}}\NormalTok{ edx }\SpecialCharTok{|\textgreater{}}
      \FunctionTok{sample\_train\_validation\_sets}\NormalTok{(fold\_i}\SpecialCharTok{*}\DecValTok{1000}\NormalTok{)}
\end{Highlighting}
\end{Shaded}

which returns a pair of train/validation sets:

\begin{Shaded}
\begin{Highlighting}[]
\NormalTok{sample\_train\_validation\_sets }\OtherTok{\textless{}{-}} \ControlFlowTok{function}\NormalTok{(data, seed)\{}
  \FunctionTok{put\_log}\NormalTok{(}\StringTok{"Function: \textasciigrave{}sample\_train\_validation\_sets\textasciigrave{}: Sampling 20\% of the \textasciigrave{}data\textasciigrave{} data..."}\NormalTok{)}
  \FunctionTok{set.seed}\NormalTok{(seed)}
\NormalTok{  validation\_ind }\OtherTok{\textless{}{-}} 
    \FunctionTok{sapply}\NormalTok{(}\FunctionTok{splitByUser}\NormalTok{(data),}
           \ControlFlowTok{function}\NormalTok{(i) }\FunctionTok{sample}\NormalTok{(i, }\FunctionTok{ceiling}\NormalTok{(}\FunctionTok{length}\NormalTok{(i)}\SpecialCharTok{*}\NormalTok{.}\DecValTok{2}\NormalTok{))) }\SpecialCharTok{|\textgreater{}} 
    \FunctionTok{unlist}\NormalTok{() }\SpecialCharTok{|\textgreater{}} 
    \FunctionTok{sort}\NormalTok{()}
  
  \FunctionTok{put\_log}\NormalTok{(}\StringTok{"Function: \textasciigrave{}sample\_train\_validation\_sets\textasciigrave{}: }
\StringTok{Extracting 80\% of the original \textasciigrave{}data\textasciigrave{} not used for the Validation Set, }
\StringTok{excluding data for users who provided no more than a specified number of ratings: \{min\_nratings\}."}\NormalTok{)}
  
\NormalTok{  train\_set }\OtherTok{\textless{}{-}}\NormalTok{ data[}\SpecialCharTok{{-}}\NormalTok{validation\_ind,]}
  
  \FunctionTok{put\_log}\NormalTok{(}\StringTok{"Function: \textasciigrave{}sample\_train\_validation\_sets\textasciigrave{}: Dataset created: train\_set"}\NormalTok{)}
  \FunctionTok{put}\NormalTok{(}\FunctionTok{summary}\NormalTok{(train\_set))}
  
  \FunctionTok{put\_log}\NormalTok{(}\StringTok{"Function: \textasciigrave{}sample\_train\_validation\_sets\textasciigrave{}: }
\StringTok{To make sure we don’t include movies in the Training Set that should not be there, }
\StringTok{we exclude entries using the semi\_join function from the Validation Set."}\NormalTok{)}
\NormalTok{  tmp.data }\OtherTok{\textless{}{-}}\NormalTok{ data[validation\_ind,]}
  
\NormalTok{  validation\_set }\OtherTok{\textless{}{-}}\NormalTok{ tmp.data }\SpecialCharTok{|\textgreater{}} 
    \FunctionTok{semi\_join}\NormalTok{(train\_set, }\AttributeTok{by =} \StringTok{"movieId"}\NormalTok{) }\SpecialCharTok{|\textgreater{}} 
    \FunctionTok{semi\_join}\NormalTok{(train\_set, }\AttributeTok{by =} \StringTok{"userId"}\NormalTok{) }\SpecialCharTok{|\textgreater{}}
    \FunctionTok{as.data.frame}\NormalTok{()}
  
  \CommentTok{\# Add rows excluded from \textasciigrave{}validation\_set\textasciigrave{} into \textasciigrave{}train\_set\textasciigrave{}}
\NormalTok{  tmp.excluded }\OtherTok{\textless{}{-}} \FunctionTok{anti\_join}\NormalTok{(tmp.data, validation\_set)}
\NormalTok{  train\_set }\OtherTok{\textless{}{-}} \FunctionTok{rbind}\NormalTok{(train\_set, tmp.excluded)}
  
  \FunctionTok{put\_log}\NormalTok{(}\StringTok{"Function: \textasciigrave{}sample\_train\_validation\_sets\textasciigrave{}: Dataset created: validation\_set"}\NormalTok{)}
  \FunctionTok{put}\NormalTok{(}\FunctionTok{summary}\NormalTok{(validation\_set))}

  \CommentTok{\# CV train \& test sets Consistency Test}
\NormalTok{  validation.left\_join.Nas }\OtherTok{\textless{}{-}}\NormalTok{ train\_set }\SpecialCharTok{|\textgreater{}}
    \FunctionTok{mutate}\NormalTok{(}\AttributeTok{tst.col =}\NormalTok{ rating) }\SpecialCharTok{|\textgreater{}}
    \FunctionTok{select}\NormalTok{(userId, movieId, tst.col) }\SpecialCharTok{|\textgreater{}}
    \FunctionTok{data.consistency.test}\NormalTok{(validation\_set)}
  
  \FunctionTok{put\_log}\NormalTok{(}\StringTok{"Function: \textasciigrave{}sample\_train\_validation\_sets\textasciigrave{}:}
\StringTok{Below are the data consistency verification results"}\NormalTok{)}
  \FunctionTok{put}\NormalTok{(validation.left\_join.Nas)}
  
  \CommentTok{\# Return result datassets {-}{-}{-}{-}{-}{-}{-}{-}{-}{-}{-}{-}{-}{-}{-}{-}{-}{-}{-}{-}{-}{-}{-}{-}{-}{-}{-}{-}{-}{-}{-}{-}{-}{-}{-}{-}{-}{-}{-}{-}{-}{-}{-}{-}{-}{-}{-}{-}{-}{-}{-}{-}    }
  \FunctionTok{list}\NormalTok{(}\AttributeTok{train\_set =}\NormalTok{ train\_set, }
       \AttributeTok{validation\_set =}\NormalTok{ validation\_set)}
\NormalTok{\}}
\end{Highlighting}
\end{Shaded}

\begin{noteblock}
The
\href{https://github.com/AzKurban-edX-DS/Capstone-MovieLens/blob/main/r/src/support-functions/data.helper.functions.R\#L15}{sample\_train\_validation\_sets}
function is defined in the same script as the
\href{https://github.com/AzKurban-edX-DS/Capstone-MovieLens/blob/main/r/src/support-functions/data.helper.functions.R\#L87}{make\_source\_datasets}`
one, from where it is called.

\end{noteblock}

\paragraph{Common Helper Functions}\label{common-helper-functions}

\hfill\break

For our further analysis, we are going to use the following \emph{common
helper functions}:

\subparagraph{\texorpdfstring{\texttt{clamp}
function}{clamp function}}\label{clamp-function}

\hfill\break

As explained in
\href{https://rafalab.dfci.harvard.edu/dsbook-part-2/highdim/regularization.html\#user-effects}{Section
24.4 User effects} of the \emph{Course Textbook} we know ratings can't
be below 0.5 or above 5. For this reason, we will use the \texttt{clamp}
function described in that section:

\begin{Shaded}
\begin{Highlighting}[]
\NormalTok{clamp }\OtherTok{\textless{}{-}} \ControlFlowTok{function}\NormalTok{(x, }\AttributeTok{min =} \FloatTok{0.5}\NormalTok{, }\AttributeTok{max =} \DecValTok{5}\NormalTok{) }\FunctionTok{pmax}\NormalTok{(}\FunctionTok{pmin}\NormalTok{(x, max), min)}
\end{Highlighting}
\end{Shaded}

\subparagraph{\texorpdfstring{Functions to calculate \emph{(Root) Mean
Squared
Error}}{Functions to calculate (Root) Mean Squared Error}}\label{functions-to-calculate-root-mean-squared-error}

\hfill\break

We will need the following functions to calculate \emph{(R)MSEs}:

\begin{Shaded}
\begin{Highlighting}[]
\NormalTok{mse }\OtherTok{\textless{}{-}} \ControlFlowTok{function}\NormalTok{(r) }\FunctionTok{mean}\NormalTok{(r}\SpecialCharTok{\^{}}\DecValTok{2}\NormalTok{)}

\NormalTok{mse\_cv }\OtherTok{\textless{}{-}} \ControlFlowTok{function}\NormalTok{(r\_list) \{}
\NormalTok{  mses }\OtherTok{\textless{}{-}} \FunctionTok{sapply}\NormalTok{(r\_list, }\FunctionTok{mse}\NormalTok{(r))}
  \FunctionTok{mean}\NormalTok{(mses)}
\NormalTok{\}}

\NormalTok{rmse }\OtherTok{\textless{}{-}} \ControlFlowTok{function}\NormalTok{(r) }\FunctionTok{sqrt}\NormalTok{(}\FunctionTok{mse}\NormalTok{(r))}
\CommentTok{\# rmse\_cv \textless{}{-} function(r\_list) sqrt(mse\_cv(r\_list))}

\NormalTok{rmse2 }\OtherTok{\textless{}{-}} \ControlFlowTok{function}\NormalTok{(true\_ratings, predicted\_ratings) \{}
  \FunctionTok{rmse}\NormalTok{(true\_ratings }\SpecialCharTok{{-}}\NormalTok{ predicted\_ratings)}
\NormalTok{\}}
\end{Highlighting}
\end{Shaded}

\begin{noteblock}
All the \emph{common helper functions}, including those described above,
are defined in the
\href{https://github.com/AzKurban-edX-DS/Capstone-MovieLens/blob/main/r/src/support-functions/common-helper.functions.R}{common-helper.functions.R}
script on \emph{GitHub}.

\end{noteblock}

\subsubsection{Overall Mean Rating (Naive)
Model}\label{overall-mean-rating-naive-model}

\hfill\break

Let's begin our analysis by evaluating the simplest model described in
\href{https://rafalab.dfci.harvard.edu/dsbook-part-2/highdim/regularization.html\#a-first-model}{Section
\emph{23.3 The First Model} of the \emph{Course Textbook}}, and then
gradually refine it through further research. It is about a model that
assumes the same rating for all movies and users with all the
differences explained by random variation would look as follows:

\[
Y_{i,j} = \mu + \varepsilon_{i,j}
\]

with \(\varepsilon_{i,j}\) independent errors sampled from the same
distribution centered at 0 and \(\mu\) the \emph{true} rating for all
movies.

We know that the estimate that minimizes the RMSE is the least squares
estimate of \(\mu\) and, in this case, is the average of all ratings:

\begin{Shaded}
\begin{Highlighting}[]
\NormalTok{mu }\OtherTok{\textless{}{-}} \FunctionTok{mean}\NormalTok{(edx}\SpecialCharTok{$}\NormalTok{rating)}
\FunctionTok{print}\NormalTok{(mu)}
\end{Highlighting}
\end{Shaded}

\begin{verbatim}
## [1] 3.512465
\end{verbatim}

If we predict all unknown ratings with \(\hat{\mu}\), we obtain the
following RMSE:

\begin{Shaded}
\begin{Highlighting}[]
\NormalTok{mu.MSEs }\OtherTok{\textless{}{-}} \FunctionTok{naive\_model\_MSEs}\NormalTok{(mu)}
\FunctionTok{data.frame}\NormalTok{(}\AttributeTok{fold\_No =} \DecValTok{1}\SpecialCharTok{:}\DecValTok{5}\NormalTok{, }\AttributeTok{MSE =}\NormalTok{ mu.MSEs) }\SpecialCharTok{|\textgreater{}}
  \FunctionTok{data.plot}\NormalTok{(}\AttributeTok{title =} \StringTok{"MSE resuls of the 5{-}fold CV method applied to the Overall Mean Rating Model"}\NormalTok{,}
              \AttributeTok{xname =} \StringTok{"fold\_No"}\NormalTok{, }
              \AttributeTok{yname =} \StringTok{"MSE"}\NormalTok{)}
\end{Highlighting}
\end{Shaded}

\includegraphics{capstone-movielens-report.draft4_files/figure-latex/unnamed-chunk-38-1.pdf}

\begin{Shaded}
\begin{Highlighting}[]
\NormalTok{mu.RMSE }\OtherTok{\textless{}{-}} \FunctionTok{sqrt}\NormalTok{(}\FunctionTok{mean}\NormalTok{(mu.MSEs))}
\NormalTok{mu.RMSE}
\end{Highlighting}
\end{Shaded}

\begin{verbatim}
## [1] 1.060346
\end{verbatim}

\begin{noteblock}
For the \emph{Mean Squared Error} data visualization we used
\href{https://github.com/AzKurban-edX-DS/Capstone-MovieLens/blob/main/r/src/support-functions/data.helper.functions.R\#L592}{data.plot}
function{]} defined in the
\href{https://github.com/AzKurban-edX-DS/Capstone-MovieLens/blob/main/r/src/support-functions/data.helper.functions.R\#L447}{Data
Visualization} section of the
\href{https://github.com/AzKurban-edX-DS/Capstone-MovieLens/blob/main/r/src/support-functions/data.helper.functions.R}{data.helper.function.R}
script.

\end{noteblock}

\begin{Shaded}
\begin{Highlighting}[]
\NormalTok{data.plot }\OtherTok{\textless{}{-}} \ControlFlowTok{function}\NormalTok{(data, }
\NormalTok{                      title, }
\NormalTok{                      xname, }
\NormalTok{                      yname, }
                      \AttributeTok{xlabel =} \ConstantTok{NULL}\NormalTok{, }
                      \AttributeTok{ylabel =} \ConstantTok{NULL}\NormalTok{,}
                      \AttributeTok{line\_col =} \StringTok{"blue"}\NormalTok{,}
                      \CommentTok{\# scale = 1,}
                      \AttributeTok{normalize =} \ConstantTok{FALSE}\NormalTok{) \{}
\NormalTok{  y }\OtherTok{\textless{}{-}}\NormalTok{ data[, yname]}
  
  \ControlFlowTok{if}\NormalTok{ (normalize) \{}
\NormalTok{    y }\OtherTok{\textless{}{-}}\NormalTok{ y }\SpecialCharTok{{-}} \FunctionTok{min}\NormalTok{(y)}
\NormalTok{  \}}
  
  \ControlFlowTok{if}\NormalTok{ (}\FunctionTok{is.null}\NormalTok{(xlabel)) \{}
\NormalTok{    xlabel }\OtherTok{=}\NormalTok{ xname}
\NormalTok{  \}}
  \ControlFlowTok{if}\NormalTok{ (}\FunctionTok{is.null}\NormalTok{(ylabel)) \{}
\NormalTok{    ylabel }\OtherTok{=}\NormalTok{ yname}
\NormalTok{  \}}
  
\NormalTok{  aes\_mapping }\OtherTok{\textless{}{-}} \FunctionTok{aes}\NormalTok{(}\AttributeTok{x =}\NormalTok{ data[, xname], }\AttributeTok{y =}\NormalTok{ y)}
  
\NormalTok{  data }\SpecialCharTok{|\textgreater{}} 
    \FunctionTok{ggplot}\NormalTok{(}\AttributeTok{mapping =}\NormalTok{ aes\_mapping) }\SpecialCharTok{+}
    \FunctionTok{ggtitle}\NormalTok{(title) }\SpecialCharTok{+}
    \FunctionTok{xlab}\NormalTok{(xlabel) }\SpecialCharTok{+}
    \FunctionTok{ylab}\NormalTok{(ylabel) }\SpecialCharTok{+}
    \FunctionTok{geom\_point}\NormalTok{() }\SpecialCharTok{+} 
    \FunctionTok{geom\_line}\NormalTok{(}\AttributeTok{color=}\NormalTok{line\_col)}
\NormalTok{\}}
\end{Highlighting}
\end{Shaded}

Here we also used
\href{https://github.com/AzKurban-edX-DS/Capstone-MovieLens/blob/main/r/src/support-functions/common-helper.functions.R\#L54}{naive\_model\_MSEs}
function defined in the
\href{https://github.com/AzKurban-edX-DS/Capstone-MovieLens/blob/main/r/src/support-functions/common-helper.functions.R}{common-helper.functions.R}
script (already mentioned above) to compute \emph{Mean Squared Errors}
using \emph{5-Fold Cross Validation} method:

\begin{Shaded}
\begin{Highlighting}[]
\NormalTok{naive\_model\_MSEs }\OtherTok{\textless{}{-}} \ControlFlowTok{function}\NormalTok{(val) \{}
  \FunctionTok{sapply}\NormalTok{(edx\_CV, }\ControlFlowTok{function}\NormalTok{(cv\_item)\{}
    \FunctionTok{mse}\NormalTok{(cv\_item}\SpecialCharTok{$}\NormalTok{validation\_set}\SpecialCharTok{$}\NormalTok{rating }\SpecialCharTok{{-}}\NormalTok{ val)}
\NormalTok{  \})}
\NormalTok{\}}
\end{Highlighting}
\end{Shaded}

One more function, defined in the
\href{https://github.com/AzKurban-edX-DS/Capstone-MovieLens/blob/main/r/src/support-functions/common-helper.functions.R}{same
script}, that we will need for further analysis of the current model, is
the
\href{https://github.com/AzKurban-edX-DS/Capstone-MovieLens/blob/main/r/src/support-functions/common-helper.functions.R\#L59}{naive\_model\_RMSE}
one:

\begin{Shaded}
\begin{Highlighting}[]
\NormalTok{naive\_model\_RMSE }\OtherTok{\textless{}{-}} \ControlFlowTok{function}\NormalTok{(val)\{}
  \FunctionTok{sqrt}\NormalTok{(}\FunctionTok{mean}\NormalTok{(}\FunctionTok{naive\_model\_MSEs}\NormalTok{(val)))}
\NormalTok{\}}
\end{Highlighting}
\end{Shaded}

\paragraph{\texorpdfstring{Ensure that \texttt{mu.RMSE} value is the
best for the current
model}{Ensure that mu.RMSE value is the best for the current model}}\label{ensure-that-mu.rmse-value-is-the-best-for-the-current-model}

\hfill\break

If we plug in any other number, we will get a higher RMSE. Let's prove
that by the following small investigation:

\begin{Shaded}
\begin{Highlighting}[]
\NormalTok{  deviation }\OtherTok{\textless{}{-}} \FunctionTok{seq}\NormalTok{(}\DecValTok{0}\NormalTok{, }\DecValTok{6}\NormalTok{, }\FloatTok{0.1}\NormalTok{) }\SpecialCharTok{{-}} \DecValTok{3}

\NormalTok{  deviation.RMSE }\OtherTok{\textless{}{-}} \FunctionTok{sapply}\NormalTok{(deviation, }\ControlFlowTok{function}\NormalTok{(delta)\{}
    \FunctionTok{naive\_model\_RMSE}\NormalTok{(mu }\SpecialCharTok{+}\NormalTok{ delta)}
\NormalTok{  \})}
\end{Highlighting}
\end{Shaded}

Let's make a quick investigation of the \texttt{deviation.RMSE} result
we have just got:

\begin{Shaded}
\begin{Highlighting}[]
\FunctionTok{data.frame}\NormalTok{(}\AttributeTok{delta =}\NormalTok{ deviation, }
           \AttributeTok{delta.RMSE =}\NormalTok{ deviation.RMSE) }\SpecialCharTok{|\textgreater{}} 
\FunctionTok{data.plot}\NormalTok{(}\AttributeTok{title =} \FunctionTok{TeX}\NormalTok{(r}\StringTok{\textquotesingle{}[RMSE as a function of deviation ($\textbackslash{}delta$) from the Overall Mean Rating ($\textbackslash{}hat\{mu\}$)]\textquotesingle{}}\NormalTok{),}
              \AttributeTok{xname =} \StringTok{"delta"}\NormalTok{, }
              \AttributeTok{yname =} \StringTok{"delta.RMSE"}\NormalTok{, }
              \AttributeTok{xlabel =} \FunctionTok{TeX}\NormalTok{(r}\StringTok{\textquotesingle{}[$\textbackslash{}delta$]\textquotesingle{}}\NormalTok{), }
              \AttributeTok{ylabel =} \StringTok{"RMSE"}\NormalTok{)}
\end{Highlighting}
\end{Shaded}

\includegraphics{capstone-movielens-report.draft4_files/figure-latex/unnamed-chunk-43-1.pdf}

\begin{Shaded}
\begin{Highlighting}[]
\NormalTok{which\_min\_deviation }\OtherTok{\textless{}{-}}\NormalTok{ deviation[}\FunctionTok{which.min}\NormalTok{(deviation.RMSE)]}
\NormalTok{min\_rmse }\OtherTok{=} \FunctionTok{min}\NormalTok{(deviation.RMSE)}

\FunctionTok{print\_log1}\NormalTok{(}\StringTok{"Minimum RMSE is achieved when the deviation from the mean is: \%1"}\NormalTok{,}
\NormalTok{         which\_min\_deviation)}
\end{Highlighting}
\end{Shaded}

\begin{verbatim}
## Minimum RMSE is achieved when the deviation from the mean is: 0
\end{verbatim}

\begin{Shaded}
\begin{Highlighting}[]
\FunctionTok{print\_log1}\NormalTok{(}\StringTok{"Is the previously computed RMSE the best for the current model: \%1"}\NormalTok{,}
\NormalTok{         mu.RMSE }\SpecialCharTok{==}\NormalTok{ min\_rmse)}
\end{Highlighting}
\end{Shaded}

\begin{verbatim}
## Is the previously computed RMSE the best for the current model: TRUE
\end{verbatim}

\begin{Shaded}
\begin{Highlighting}[]
\NormalTok{RMSEs.ResultTibble.OMR }\OtherTok{\textless{}{-}}\NormalTok{ RMSEs.ResultTibble }\SpecialCharTok{|\textgreater{}} 
  \FunctionTok{RMSEs.AddRow}\NormalTok{(}\StringTok{"Overall Mean Rating Model"}\NormalTok{, mu.RMSE)}
\end{Highlighting}
\end{Shaded}

\begin{Shaded}
\begin{Highlighting}[]
\FunctionTok{RMSE\_kable}\NormalTok{(RMSEs.ResultTibble.OMR)}
\end{Highlighting}
\end{Shaded}

\begin{longtable}[t]{>{\raggedright\arraybackslash}p{35em}>{\centering\arraybackslash}p{10em}>{\raggedright\arraybackslash}p{50em}}
\toprule
\textbf{Method} & \textbf{RMSE} & \textbf{Comment}\\
\midrule
Project Objective & 0.864900 & \\
Overall Mean Rating Model & 1.060346 & \\
\bottomrule
\end{longtable}

To win the grand prize of \$1,000,000, a participating team had to get
an RMSE of at least 0.8563\autocite{BigChaosSln}. So we can definitely
do better!\autocite{IDS2_23-3}

\subsubsection{User Effect Model}\label{user-effect-model}

\hfill\break
To improve our model let's now take into consideration user effects as
explained in
\href{https://rafalab.dfci.harvard.edu/dsbook-part-2/highdim/regularization.html\#user-effects}{Section
\emph{23.4 User effects}} of the \emph{Course Textbook}. If we visualize
the average rating for each user the way the
\href{https://x.com/rafalab}{the author} shows, we can see that there is
substantial variability in the average ratings across users:

\begin{Shaded}
\begin{Highlighting}[]
\FunctionTok{hist}\NormalTok{(edx.user\_mean\_ratings}\SpecialCharTok{$}\NormalTok{mean\_rating, }\AttributeTok{nclass =} \DecValTok{30}\NormalTok{)}
\end{Highlighting}
\end{Shaded}

\includegraphics{capstone-movielens-report.draft4_files/figure-latex/unnamed-chunk-46-1.pdf}

Following the author's further explanation, to account for this
variability, we will use a linear model with a \emph{treatment effect}
\(\alpha_i\) for each user. The sum \(\mu+\alpha_i\) can be interpreted
as the typical rating user \(i\) gives to movies. So we write the model
as follows:

\[
Y_{i,j} = \mu + \alpha_i + \varepsilon_{i,j}
\]

Statistics textbooks refer to the \(\alpha\)s as treatment effects. In
the Netflix challenge papers, they refer to them as
\emph{bias}\autocite{IDS2_23-4,MFT_RS}.

As it is stated here\autocite{IDS2_23-4}, it can be shown that the least
squares estimate \(\hat{\alpha}_i\) is just the average of
\(y_{i,j} - \hat{\mu}\) for each user \(i\). So we can compute them this
way:

\begin{Shaded}
\begin{Highlighting}[]
\NormalTok{a }\OtherTok{\textless{}{-}} \FunctionTok{rowMeans}\NormalTok{(y }\SpecialCharTok{{-}}\NormalTok{ mu, }\AttributeTok{na.rm =} \ConstantTok{TRUE}\NormalTok{)}
\end{Highlighting}
\end{Shaded}

These considerations alows us to compute a \emph{User Mean Ratings} the
following way:

\begin{Shaded}
\begin{Highlighting}[]
\FunctionTok{put\_log}\NormalTok{(}\StringTok{"Computing Average Ratings per User (User Mean Ratings)..."}\NormalTok{)}
\NormalTok{user.mean\_ratings }\OtherTok{\textless{}{-}} \FunctionTok{rowMeans}\NormalTok{(edx.mx, }\AttributeTok{na.rm =} \ConstantTok{TRUE}\NormalTok{)}
\NormalTok{user\_ratings.n }\OtherTok{\textless{}{-}} \FunctionTok{rowSums}\NormalTok{(}\SpecialCharTok{!}\FunctionTok{is.na}\NormalTok{(edx.mx))}
  
\NormalTok{edx.user\_mean\_ratings }\OtherTok{\textless{}{-}} 
  \FunctionTok{data.frame}\NormalTok{(}\AttributeTok{userId =} \FunctionTok{names}\NormalTok{(user.mean\_ratings), }
             \AttributeTok{mean\_rating =}\NormalTok{ user.mean\_ratings,}
             \AttributeTok{n =}\NormalTok{ user\_ratings.n)}
  
\FunctionTok{put\_log}\NormalTok{(}\StringTok{"User Mean Ratings have been computed."}\NormalTok{)}
\end{Highlighting}
\end{Shaded}

\begin{Shaded}
\begin{Highlighting}[]
\FunctionTok{str}\NormalTok{(edx.user\_mean\_ratings)}
\end{Highlighting}
\end{Shaded}

\begin{verbatim}
## 'data.frame':    69878 obs. of  3 variables:
##  $ userId     : chr  "1" "2" "3" "4" ...
##  $ mean_rating: num  5 3.29 3.94 4.06 3.92 ...
##  $ n          : num  19 17 31 35 74 39 96 727 21 112 ...
\end{verbatim}

And then we compute a \emph{User Effect} this way:

\begin{Shaded}
\begin{Highlighting}[]
\FunctionTok{put\_log}\NormalTok{(}\StringTok{"Computing User Effect per users ..."}\NormalTok{)}
\NormalTok{edx.user\_effect }\OtherTok{\textless{}{-}}\NormalTok{ edx.user\_mean\_ratings }\SpecialCharTok{|\textgreater{}}
  \FunctionTok{mutate}\NormalTok{(}\AttributeTok{userId =} \FunctionTok{as.integer}\NormalTok{(userId),}
         \AttributeTok{a =}\NormalTok{ mean\_rating }\SpecialCharTok{{-}}\NormalTok{ mu)}

\FunctionTok{put\_log}\NormalTok{(}\StringTok{"A User Effect Model has been builded"}\NormalTok{)}
\end{Highlighting}
\end{Shaded}

\begin{Shaded}
\begin{Highlighting}[]
\FunctionTok{par}\NormalTok{(}\AttributeTok{cex =} \FloatTok{0.7}\NormalTok{)}
\FunctionTok{hist}\NormalTok{(edx.user\_effect}\SpecialCharTok{$}\NormalTok{a, }\DecValTok{30}\NormalTok{, }\AttributeTok{xlab =} \FunctionTok{TeX}\NormalTok{(r}\StringTok{\textquotesingle{}[$\textbackslash{}hat\{alpha\}\_\{i\}$]\textquotesingle{}}\NormalTok{),}
     \AttributeTok{main =} \FunctionTok{TeX}\NormalTok{(r}\StringTok{\textquotesingle{}[Histogram of $\textbackslash{}hat\{alpha\}\_\{i\}$]\textquotesingle{}}\NormalTok{))}
\end{Highlighting}
\end{Shaded}

\includegraphics{capstone-movielens-report.draft4_files/figure-latex/unnamed-chunk-51-1.pdf}

\begin{Shaded}
\begin{Highlighting}[]
\FunctionTok{str}\NormalTok{(edx.user\_effect)}
\end{Highlighting}
\end{Shaded}

\begin{verbatim}
## 'data.frame':    69878 obs. of  4 variables:
##  $ userId     : int  1 2 3 4 5 6 7 8 9 10 ...
##  $ mean_rating: num  5 3.29 3.94 4.06 3.92 ...
##  $ n          : num  19 17 31 35 74 39 96 727 21 112 ...
##  $ a          : num  1.488 -0.218 0.423 0.545 0.406 ...
\end{verbatim}

\begin{noteblock}
The full source code of the \emph{User Effect} computation is available
in the
\href{https://github.com/AzKurban-edX-DS/Capstone-MovieLens/blob/main/r/src/capstone-movielens.main.R\#L531}{Model
building: User Effect} section of the
\href{https://github.com/AzKurban-edX-DS/Capstone-MovieLens/blob/main/r/src/capstone-movielens.main.R}{capstone-movielens.main.R}
script on \emph{GitHub}.

\end{noteblock}

Finally, we are ready to compute the \texttt{RMSE} (additionally using
the
\href{https://github.com/AzKurban-edX-DS/Capstone-MovieLens/blob/main/r/src/support-functions/common-helper.functions.R\#L4}{clamp}
helper function we defined above to keep predictions in the proper
range):

\begin{Shaded}
\begin{Highlighting}[]
\FunctionTok{put\_log}\NormalTok{(}\StringTok{"Computing the RMSE taking into account user effects..."}\NormalTok{)}
\NormalTok{start }\OtherTok{\textless{}{-}} \FunctionTok{put\_start\_date}\NormalTok{()}
\NormalTok{edx.user\_effect.MSEs }\OtherTok{\textless{}{-}} \FunctionTok{sapply}\NormalTok{(edx\_CV, }\ControlFlowTok{function}\NormalTok{(cv\_fold\_dat)\{}
\NormalTok{  cv\_fold\_dat}\SpecialCharTok{$}\NormalTok{validation\_set }\SpecialCharTok{|\textgreater{}}
    \FunctionTok{left\_join}\NormalTok{(edx.user\_effect, }\AttributeTok{by =} \StringTok{"userId"}\NormalTok{) }\SpecialCharTok{|\textgreater{}}
    \FunctionTok{mutate}\NormalTok{(}\AttributeTok{resid =}\NormalTok{ rating }\SpecialCharTok{{-}} \FunctionTok{clamp}\NormalTok{(mu }\SpecialCharTok{+}\NormalTok{ a)) }\SpecialCharTok{|\textgreater{}} 
    \FunctionTok{pull}\NormalTok{(resid) }\SpecialCharTok{|\textgreater{}} \FunctionTok{mse}\NormalTok{()}
\NormalTok{\})}
\FunctionTok{put\_end\_date}\NormalTok{(start)}

\NormalTok{edx.user\_effect.RMSE }\OtherTok{\textless{}{-}} \FunctionTok{sqrt}\NormalTok{(}\FunctionTok{mean}\NormalTok{(edx.user\_effect.MSEs))}

\NormalTok{RMSEs.ResultTibble.UE }\OtherTok{\textless{}{-}}\NormalTok{ RMSEs.ResultTibble.OMR }\SpecialCharTok{|\textgreater{}} 
  \FunctionTok{RMSEs.AddRow}\NormalTok{(}\StringTok{"User Effect Model"}\NormalTok{, edx.user\_effect.RMSE)}
\end{Highlighting}
\end{Shaded}

\begin{Shaded}
\begin{Highlighting}[]
\FunctionTok{data.frame}\NormalTok{(}\AttributeTok{fold\_No =} \DecValTok{1}\SpecialCharTok{:}\DecValTok{5}\NormalTok{, }\AttributeTok{MSE =}\NormalTok{ edx.user\_effect.MSEs) }\SpecialCharTok{|\textgreater{}}
  \FunctionTok{data.plot}\NormalTok{(}\AttributeTok{title =} \StringTok{"MSE resuls of the 5{-}fold CV method applied to the User Effect Model"}\NormalTok{,}
            \AttributeTok{xname =} \StringTok{"fold\_No"}\NormalTok{, }
            \AttributeTok{yname =} \StringTok{"MSE"}\NormalTok{)}
\end{Highlighting}
\end{Shaded}

\includegraphics{capstone-movielens-report.draft4_files/figure-latex/unnamed-chunk-53-1.pdf}

\begin{Shaded}
\begin{Highlighting}[]
\FunctionTok{RMSE\_kable}\NormalTok{(RMSEs.ResultTibble.UE)}
\end{Highlighting}
\end{Shaded}

\begin{longtable}[t]{>{\raggedright\arraybackslash}p{35em}>{\centering\arraybackslash}p{10em}>{\raggedright\arraybackslash}p{50em}}
\toprule
\textbf{Method} & \textbf{RMSE} & \textbf{Comment}\\
\midrule
Project Objective & 0.8649000 & \\
Overall Mean Rating Model & 1.0603462 & \\
User Effect Model & 0.9697962 & \\
\bottomrule
\end{longtable}

\begin{noteblock}
The full source code of the \emph{User Effect Model RMSE} computation is
available in the
\href{https://github.com/AzKurban-edX-DS/Capstone-MovieLens/blob/main/r/src/capstone-movielens.main.R\#L655}{Compute
RMSE for User Effect Model} section of the
\href{https://github.com/AzKurban-edX-DS/Capstone-MovieLens/blob/main/r/src/capstone-movielens.main.R}{capstone-movielens.main.R}
script on \emph{GitHub}.

\end{noteblock}

\subsubsection{User+Movie Effect (UME)
Model}\label{usermovie-effect-ume-model}

\hfill\break

In
\href{https://rafalab.dfci.harvard.edu/dsbook-part-2/highdim/regularization.html\#movie-effects}{23.5
Movie effects} section of the \emph{Course Textbook} the author draws
our attention to the fact that some movies are generally rated higher
than others. He also explains that a linear model with a \emph{treatment
effect} \(\beta_j\) for each movie can be used in this case, which can
be interpreted as movie effect or the difference between the average
ranking for movie \(j\) and the overall average \(\mu\):

\[
Y_{i,j} = \mu + \alpha_i + \beta_j +\varepsilon_{i,j}
\] The author then shows how to use an approximation by first computing
the least square estimate \(\hat{\mu}\) and \(\hat{\alpha}_i\), and then
estimating \(\hat{\beta}_j\) as the average of the residuals
\(y_{i,j} - \hat{\mu} - \hat{\alpha}_i\):

\begin{Shaded}
\begin{Highlighting}[]
\NormalTok{b }\OtherTok{\textless{}{-}} \FunctionTok{colMeans}\NormalTok{(y }\SpecialCharTok{{-}}\NormalTok{ mu }\SpecialCharTok{{-}}\NormalTok{ a, }\AttributeTok{na.rm =} \ConstantTok{TRUE}\NormalTok{)}
\end{Highlighting}
\end{Shaded}

Inspired by this idea, a few support functions were developed by the
author of this report, which we will use for our further analysis.

\paragraph{UME Model: Support
Functions}\label{ume-model-support-functions}

\hfill\break

\begin{noteblock}
The full source code of the functions described in this section is
available in the
\href{https://github.com/AzKurban-edX-DS/Capstone-MovieLens/blob/main/r/src/support-functions/UM-effect.functions.R\#L1}{User+Movie
Effect Model Functions} section of the
\href{https://github.com/AzKurban-edX-DS/Capstone-MovieLens/blob/main/r/src/support-functions/UM-effect.functions.R}{UM-effect.functions.R}
script on \emph{GitHub}.

\end{noteblock}

\subparagraph{\texorpdfstring{\texttt{train\_user\_movie\_effect}
Function}{train\_user\_movie\_effect Function}}\label{train_user_movie_effect-function}

\hfill\break

We use this function to build and train our model using the
\texttt{train\_set} dataset:

\begin{Shaded}
\begin{Highlighting}[]
\NormalTok{train\_user\_movie\_effect }\OtherTok{\textless{}{-}} \ControlFlowTok{function}\NormalTok{(train\_set, }\AttributeTok{lambda =} \DecValTok{0}\NormalTok{)\{}
  \ControlFlowTok{if}\NormalTok{ (}\FunctionTok{is.na}\NormalTok{(lambda)) \{}
    \FunctionTok{stop}\NormalTok{(}\StringTok{"Function: train\_user\_movie\_effect}
\StringTok{\textasciigrave{}lambda\textasciigrave{} is \textasciigrave{}NA\textasciigrave{}"}\NormalTok{)}
\NormalTok{  \}}

\NormalTok{  UM.effect }\OtherTok{\textless{}{-}}\NormalTok{ train\_set }\SpecialCharTok{|\textgreater{}}
    \FunctionTok{left\_join}\NormalTok{(edx.user\_effect, }\AttributeTok{by =} \StringTok{"userId"}\NormalTok{) }\SpecialCharTok{|\textgreater{}}
    \FunctionTok{mutate}\NormalTok{(}\AttributeTok{resid =}\NormalTok{ rating }\SpecialCharTok{{-}}\NormalTok{ (mu }\SpecialCharTok{+}\NormalTok{ a)) }\SpecialCharTok{|\textgreater{}} 
    \FunctionTok{group\_by}\NormalTok{(movieId) }\SpecialCharTok{|\textgreater{}}
    \FunctionTok{summarise}\NormalTok{(}\AttributeTok{b =} \FunctionTok{mean\_reg}\NormalTok{(resid, lambda), }\AttributeTok{n =} \FunctionTok{n}\NormalTok{())}
  
  \FunctionTok{stopifnot}\NormalTok{(}\SpecialCharTok{!}\FunctionTok{is.na}\NormalTok{(}\FunctionTok{mean}\NormalTok{(UM.effect}\SpecialCharTok{$}\NormalTok{b)))}
\NormalTok{  UM.effect}
\NormalTok{\}}
\end{Highlighting}
\end{Shaded}

\begin{noteblock}
The function described above accepts the \texttt{lambda} parameter,
which we will need later for the \emph{Regularization} method. We also
use the
\href{https://github.com/AzKurban-edX-DS/Capstone-MovieLens/blob/main/r/src/support-functions/common-helper.functions.R\#L63}{mean\_reg}
function call, which we will also need for the \emph{Regularization}. We
will explain that later in the {[}Regularization Method{]} section. For
now, we omit the \texttt{lambda} parameter, accepting its default value
\texttt{lambda\ =\ 0}. In this case, the \texttt{mean\_reg} function is
equivalent to the simple \texttt{mean} one.

\end{noteblock}

\begin{Shaded}
\begin{Highlighting}[]
\DocumentationTok{\#\# Regularization {-}{-}{-}{-}{-}{-}{-}{-}{-}{-}{-}{-}{-}{-}{-}{-}{-}{-}{-}{-}{-}{-}{-}{-}{-}{-}{-}{-}{-}{-}{-}{-}{-}{-}{-}{-}{-}{-}{-}{-}{-}{-}{-}{-}{-}{-}{-}{-}{-}{-}{-}{-}{-}{-}{-}{-}{-}{-}{-}{-}{-}{-}}
\NormalTok{mean\_reg }\OtherTok{\textless{}{-}} \ControlFlowTok{function}\NormalTok{(vals, }\AttributeTok{lambda =} \DecValTok{0}\NormalTok{, }\AttributeTok{na.rm =} \ConstantTok{TRUE}\NormalTok{)\{}
  \ControlFlowTok{if}\NormalTok{ (}\FunctionTok{is.na}\NormalTok{(lambda)) \{}
    \FunctionTok{stop}\NormalTok{(}\StringTok{"Function: mean\_reg}
\StringTok{\textasciigrave{}lambda\textasciigrave{} is \textasciigrave{}NA\textasciigrave{}"}\NormalTok{)}
\NormalTok{  \}}
  
  \FunctionTok{names}\NormalTok{(lambda) }\OtherTok{\textless{}{-}} \ConstantTok{NULL}
\NormalTok{  sums }\OtherTok{\textless{}{-}} \FunctionTok{sum}\NormalTok{(vals, }\AttributeTok{na.rm =}\NormalTok{ na.rm)}
\NormalTok{  N }\OtherTok{\textless{}{-}} \FunctionTok{ifelse}\NormalTok{(na.rm, }\FunctionTok{sum}\NormalTok{(}\SpecialCharTok{!}\FunctionTok{is.na}\NormalTok{(vals)), }\FunctionTok{length}\NormalTok{(vals))}
\NormalTok{  sums}\SpecialCharTok{/}\NormalTok{(N }\SpecialCharTok{+}\NormalTok{ lambda)}
\NormalTok{\}}
\end{Highlighting}
\end{Shaded}

\subparagraph{\texorpdfstring{\texttt{train\_user\_movie\_effect.cv}
Function}{train\_user\_movie\_effect.cv Function}}\label{train_user_movie_effect.cv-function}

\hfill\break

We use the
\href{https://github.com/AzKurban-edX-DS/Capstone-MovieLens/blob/main/r/src/support-functions/UM-effect.functions.R\#L17}{train\_user\_movie\_effect.cv}
function to build and train our model using the
\texttt{5-Fold\ Cross\ Validation} method. Below, we provide the most
important part of the code of that function:

\begin{Shaded}
\begin{Highlighting}[]
\NormalTok{train\_user\_movie\_effect.cv }\OtherTok{\textless{}{-}} \ControlFlowTok{function}\NormalTok{(}\AttributeTok{lambda =} \DecValTok{0}\NormalTok{)\{}
\CommentTok{\# ...}
\NormalTok{  start }\OtherTok{\textless{}{-}} \FunctionTok{put\_start\_date}\NormalTok{()}
\NormalTok{  user\_movie\_effects\_ls }\OtherTok{\textless{}{-}} \FunctionTok{lapply}\NormalTok{(edx\_CV, }\ControlFlowTok{function}\NormalTok{(cv\_fold\_dat)\{}
\NormalTok{    cv\_fold\_dat}\SpecialCharTok{$}\NormalTok{train\_set }\SpecialCharTok{|\textgreater{}} \FunctionTok{train\_user\_movie\_effect}\NormalTok{(lambda)}
\NormalTok{  \})}
  \FunctionTok{put\_end\_date}\NormalTok{(start)}
  \FunctionTok{put\_log}\NormalTok{(}\StringTok{"Function: train\_user\_movie\_effect.cv:}
\StringTok{User+Movie Effect list have been computed"}\NormalTok{)}
  
\NormalTok{  user\_movie\_effects\_united }\OtherTok{\textless{}{-}} \FunctionTok{union\_cv\_results}\NormalTok{(user\_movie\_effects\_ls)}

\NormalTok{  user\_movie\_effect }\OtherTok{\textless{}{-}}\NormalTok{ user\_movie\_effects\_united }\SpecialCharTok{|\textgreater{}}
    \FunctionTok{group\_by}\NormalTok{(movieId) }\SpecialCharTok{|\textgreater{}}
    \FunctionTok{summarise}\NormalTok{(}\AttributeTok{b =} \FunctionTok{mean}\NormalTok{(b), }\AttributeTok{n =} \FunctionTok{mean}\NormalTok{(n))}
  \CommentTok{\# ...}
\NormalTok{  user\_movie\_effect}
\NormalTok{\}}
\end{Highlighting}
\end{Shaded}

\begin{noteblock}
Here we use the function call
\href{https://github.com/AzKurban-edX-DS/Capstone-MovieLens/blob/main/r/src/support-functions/data.helper.functions.R\#L432}{union\_cv\_results},
which is defined in the script
\href{https://github.com/AzKurban-edX-DS/Capstone-MovieLens/blob/main/r/src/support-functions/data.helper.functions.R}{common-helper.functions.R},
to aggregate the \texttt{Cross\ Validation} results.

\end{noteblock}

\begin{Shaded}
\begin{Highlighting}[]
\NormalTok{union\_cv\_results }\OtherTok{\textless{}{-}} \ControlFlowTok{function}\NormalTok{(data\_list) \{}
\NormalTok{  out\_dat }\OtherTok{\textless{}{-}}\NormalTok{ data\_list[[}\DecValTok{1}\NormalTok{]]}
  
  \ControlFlowTok{for}\NormalTok{ (i }\ControlFlowTok{in} \DecValTok{2}\SpecialCharTok{:}\NormalTok{CVFolds\_N)\{}
\NormalTok{    out\_dat }\OtherTok{\textless{}{-}} \FunctionTok{union}\NormalTok{(out\_dat, }
\NormalTok{                     data\_list[[i]])}
\NormalTok{  \}}
  
\NormalTok{  out\_dat}
\NormalTok{\}}
\end{Highlighting}
\end{Shaded}

\subparagraph{\texorpdfstring{\texttt{calc\_user\_movie\_effect\_MSE}
Function}{calc\_user\_movie\_effect\_MSE Function}}\label{calc_user_movie_effect_mse-function}

\hfill\break

The code of the function
\href{https://github.com/AzKurban-edX-DS/Capstone-MovieLens/blob/main/r/src/support-functions/UM-effect.functions.R\#L55}{calc\_user\_movie\_effect\_MSE}
defined in the
\href{https://github.com/AzKurban-edX-DS/Capstone-MovieLens/blob/main/r/src/support-functions/UM-effect.functions.R}{UM-effect.functions.R}
script to calculate the \texttt{Mean\ Squared\ Error\ (MSE)} of the
\texttt{UME\ Model} for the given \texttt{Test\ Set} is provided below:

\begin{Shaded}
\begin{Highlighting}[]
\NormalTok{calc\_user\_movie\_effect\_MSE }\OtherTok{\textless{}{-}} \ControlFlowTok{function}\NormalTok{(test\_set, um\_effect)\{}
\NormalTok{  mse.result }\OtherTok{\textless{}{-}}\NormalTok{ test\_set }\SpecialCharTok{|\textgreater{}}
    \FunctionTok{left\_join}\NormalTok{(edx.user\_effect, }\AttributeTok{by =} \StringTok{"userId"}\NormalTok{) }\SpecialCharTok{|\textgreater{}}
    \FunctionTok{left\_join}\NormalTok{(um\_effect, }\AttributeTok{by =} \StringTok{"movieId"}\NormalTok{) }\SpecialCharTok{|\textgreater{}}
    \FunctionTok{mutate}\NormalTok{(}\AttributeTok{resid =}\NormalTok{ rating }\SpecialCharTok{{-}} \FunctionTok{clamp}\NormalTok{(mu }\SpecialCharTok{+}\NormalTok{ a }\SpecialCharTok{+}\NormalTok{ b)) }\SpecialCharTok{|\textgreater{}} 
    \FunctionTok{pull}\NormalTok{(resid) }\SpecialCharTok{|\textgreater{}} \FunctionTok{mse}\NormalTok{()}
  
  \FunctionTok{stopifnot}\NormalTok{(}\SpecialCharTok{!}\FunctionTok{is.na}\NormalTok{(mse.result))}
\NormalTok{  mse.result}
\NormalTok{\}}
\end{Highlighting}
\end{Shaded}

\subparagraph{\texorpdfstring{\texttt{calc\_user\_movie\_effect\_MSE.cv}
Function}{calc\_user\_movie\_effect\_MSE.cv Function}}\label{calc_user_movie_effect_mse.cv-function}

\hfill\break

The code of the function
\href{https://github.com/AzKurban-edX-DS/Capstone-MovieLens/blob/main/r/src/support-functions/UM-effect.functions.R\#L72}{calc\_user\_movie\_effect\_MSE.cv}
defined in the
\href{https://github.com/AzKurban-edX-DS/Capstone-MovieLens/blob/main/r/src/support-functions/UM-effect.functions.R}{UM-effect.functions.R}
script to calculate the \texttt{5-Fold\ Cross\ Validation} MSE result of
the \texttt{UME\ Model} is provided below:

\begin{Shaded}
\begin{Highlighting}[]
\NormalTok{calc\_user\_movie\_effect\_MSE.cv }\OtherTok{\textless{}{-}} \ControlFlowTok{function}\NormalTok{(um\_effect)\{}
  \FunctionTok{put\_log}\NormalTok{(}\StringTok{"Function: user\_movie\_effects\_MSE.cv:}
\StringTok{Computing the RMSE taking into account User+Movie Effects..."}\NormalTok{)}
\NormalTok{  start }\OtherTok{\textless{}{-}} \FunctionTok{put\_start\_date}\NormalTok{()}
\NormalTok{  user\_movie\_effects\_MSEs }\OtherTok{\textless{}{-}} \FunctionTok{sapply}\NormalTok{(edx\_CV, }\ControlFlowTok{function}\NormalTok{(cv\_fold\_dat)\{}
\NormalTok{    cv\_fold\_dat}\SpecialCharTok{$}\NormalTok{validation\_set }\SpecialCharTok{|\textgreater{}} \FunctionTok{calc\_user\_movie\_effect\_MSE}\NormalTok{(um\_effect)}
\NormalTok{  \})}
  \FunctionTok{put\_end\_date}\NormalTok{(start)}
  
  \FunctionTok{put\_log1}\NormalTok{(}\StringTok{"Function: user\_movie\_effects\_MSE.cv:}
\StringTok{MSE values have been plotted for the \%1{-}Fold Cross Validation samples."}\NormalTok{, }
\NormalTok{           CVFolds\_N)}
  
  \FunctionTok{mean}\NormalTok{(user\_movie\_effects\_MSEs)}
\NormalTok{\}}
\end{Highlighting}
\end{Shaded}

\subparagraph{\texorpdfstring{\texttt{calc\_user\_movie\_effect\_RMSE}
Function}{calc\_user\_movie\_effect\_RMSE Function}}\label{calc_user_movie_effect_rmse-function}

\hfill\break

The code of the function
\href{https://github.com/AzKurban-edX-DS/Capstone-MovieLens/blob/main/r/src/support-functions/UM-effect.functions.R\#L51}{calc\_user\_movie\_effect\_RMSE}
defined in the
\href{https://github.com/AzKurban-edX-DS/Capstone-MovieLens/blob/main/r/src/support-functions/UM-effect.functions.R}{UM-effect.functions.R}
script to calculate the \texttt{Root\ Mean\ Squared\ Error\ (RMSE)} of
the \texttt{UME\ Model} for the given \texttt{Test\ Set} is provided
below:

\begin{Shaded}
\begin{Highlighting}[]
\NormalTok{calc\_user\_movie\_effect\_RMSE }\OtherTok{\textless{}{-}} \ControlFlowTok{function}\NormalTok{(test\_set, um\_effect)\{}
\NormalTok{  mse }\OtherTok{\textless{}{-}}\NormalTok{ test\_set }\SpecialCharTok{|\textgreater{}} \FunctionTok{calc\_user\_movie\_effect\_MSE}\NormalTok{(um\_effect)}
  \FunctionTok{sqrt}\NormalTok{(mse)}
\NormalTok{\}}
\end{Highlighting}
\end{Shaded}

\subparagraph{\texorpdfstring{\texttt{calc\_user\_movie\_effect\_RMSE.cv}
Function}{calc\_user\_movie\_effect\_RMSE.cv Function}}\label{calc_user_movie_effect_rmse.cv-function}

\hfill\break

The code of the function
\href{https://github.com/AzKurban-edX-DS/Capstone-MovieLens/blob/main/r/src/support-functions/UM-effect.functions.R\#L65}{calc\_user\_movie\_effect\_RMSE.cv}
defined in the
\href{https://github.com/AzKurban-edX-DS/Capstone-MovieLens/blob/main/r/src/support-functions/UM-effect.functions.R}{UM-effect.functions.R}
script to calculate the \texttt{5-Fold\ Cross\ Validation} RMSE result
of the \texttt{UME\ Model} is provided below:

\begin{Shaded}
\begin{Highlighting}[]
\NormalTok{calc\_user\_movie\_effect\_RMSE.cv }\OtherTok{\textless{}{-}} \ControlFlowTok{function}\NormalTok{(um\_effect)\{}
\NormalTok{  user\_movie\_effects\_MSE }\OtherTok{\textless{}{-}} \FunctionTok{calc\_user\_movie\_effect\_MSE.cv}\NormalTok{(um\_effect)}
\NormalTok{  um\_effect\_RMSE }\OtherTok{\textless{}{-}} \FunctionTok{sqrt}\NormalTok{(user\_movie\_effects\_MSE)}
  \FunctionTok{put\_log2}\NormalTok{(}\StringTok{"Function: user\_movie\_effects\_RMSE.cv:}
\StringTok{\%1{-}Fold Cross Validation ultimate RMSE: \%2"}\NormalTok{, CVFolds\_N, um\_effect\_RMSE)}
\NormalTok{  um\_effect\_RMSE}
\NormalTok{\}}
\end{Highlighting}
\end{Shaded}

\paragraph{Model Building}\label{model-building}

\hfill\break

\begin{noteblock}
The full source code of builing and training the current model is
available in the
\href{https://github.com/AzKurban-edX-DS/Capstone-MovieLens/blob/main/r/src/capstone-movielens.main.R\#L721}{Model
building: User+Movie Effect} section of the
\href{https://github.com/AzKurban-edX-DS/Capstone-MovieLens/blob/main/r/src/capstone-movielens.main.R}{capstone-movielens.main.R}
script on \emph{GitHub}.

\end{noteblock}

Below, we provide the most significant part of the code for training our
model using the \texttt{5-Fold\ Cross\ Validation} method:

\begin{Shaded}
\begin{Highlighting}[]
\NormalTok{  cv.UM\_effect }\OtherTok{\textless{}{-}} \FunctionTok{train\_user\_movie\_effect.cv}\NormalTok{()}
\end{Highlighting}
\end{Shaded}

\begin{Shaded}
\begin{Highlighting}[]
\FunctionTok{str}\NormalTok{(cv.UM\_effect)}
\end{Highlighting}
\end{Shaded}

\begin{verbatim}
## tibble [10,677 x 3] (S3: tbl_df/tbl/data.frame)
##  $ movieId: int [1:10677] 1 2 3 4 5 6 7 8 9 10 ...
##  $ b      : num [1:10677] 0.335 -0.306 -0.365 -0.598 -0.444 ...
##  $ n      : num [1:10677] 18907 8593 5574 1253 5065 ...
\end{verbatim}

\begin{Shaded}
\begin{Highlighting}[]
\FunctionTok{par}\NormalTok{(}\AttributeTok{cex =} \FloatTok{0.7}\NormalTok{)}
\FunctionTok{hist}\NormalTok{(cv.UM\_effect}\SpecialCharTok{$}\NormalTok{b, }\DecValTok{30}\NormalTok{, }\AttributeTok{xlab =} \FunctionTok{TeX}\NormalTok{(r}\StringTok{\textquotesingle{}[$\textbackslash{}hat\{beta\}\_\{j\}$)]\textquotesingle{}}\NormalTok{),}
     \AttributeTok{main =} \FunctionTok{TeX}\NormalTok{(r}\StringTok{\textquotesingle{}[Histogram of $\textbackslash{}hat\{beta\}\_\{j\}$]\textquotesingle{}}\NormalTok{))}
\end{Highlighting}
\end{Shaded}

\includegraphics{capstone-movielens-report.draft4_files/figure-latex/unnamed-chunk-64-1.pdf}

We can now construct predictors and see how much the \texttt{RMSE}
improves\autocite{IDS2_23-5}:

\begin{Shaded}
\begin{Highlighting}[]
\NormalTok{cv.UM\_effect.RMSE }\OtherTok{\textless{}{-}} \FunctionTok{calc\_user\_movie\_effect\_RMSE.cv}\NormalTok{(cv.UM\_effect)}

\NormalTok{RMSEs.ResultTibble.UME }\OtherTok{\textless{}{-}}\NormalTok{ RMSEs.ResultTibble.UE }\SpecialCharTok{|\textgreater{}} 
  \FunctionTok{RMSEs.AddRow}\NormalTok{(}\StringTok{"User+Movie Effect Model"}\NormalTok{, cv.UM\_effect.RMSE)}
\end{Highlighting}
\end{Shaded}

\begin{Shaded}
\begin{Highlighting}[]
\FunctionTok{RMSE\_kable}\NormalTok{(RMSEs.ResultTibble.UME)}
\end{Highlighting}
\end{Shaded}

\begin{longtable}[t]{>{\raggedright\arraybackslash}p{35em}>{\centering\arraybackslash}p{10em}>{\raggedright\arraybackslash}p{50em}}
\toprule
\textbf{Method} & \textbf{RMSE} & \textbf{Comment}\\
\midrule
Project Objective & 0.8649000 & \\
Overall Mean Rating Model & 1.0603462 & \\
User Effect Model & 0.9697962 & \\
User+Movie Effect Model & 0.8732081 & \\
\bottomrule
\end{longtable}

\subsubsection{Regularizing User+Movie Effect
Model}\label{regularizing-usermovie-effect-model}

\paragraph{Utilizing Penalized least
squares}\label{utilizing-penalized-least-squares}

\hfill\break
\href{https://rafalab.dfci.harvard.edu/dsbook-part-2/highdim/regularization.html\#penalized-least-squares}{Section
\emph{23.6 Penalized least squares} of the \emph{Course Textbook}}
explains why and how we should use \emph{Penalized least squares} to
improve our predictions. The author also explains that the general idea
of penalized regression is to control the total variability of the movie
effects: \(\sum_{j=1}^n \beta_j^2\). Specifically, instead of minimizing
the least squares equation, we minimize an equation that adds a penalty:

\[ 
\sum_{i,j} \left(y_{u,i} - \mu - \alpha_i - \beta_j \right)^2 + \lambda \sum_{j} \beta_j^2
\] The first term is just the sum of squares and the second is a penalty
that gets larger when many \(\beta_i\)s are large. Using calculus, we
can actually show that the values of \(\beta_i\) that minimize this
equation are:

\[
\hat{\beta}_j(\lambda) = \frac{1}{\lambda + n_j} \sum_{i=1}^{n_i} \left(Y_{i,j} - \mu - \alpha_i\right)
\]

where \(n_j\) is the number of ratings made for movie \(j\).

This approach will have our desired effect: when our sample size \(n_j\)
is very large, we obtain a stable estimate and the penalty \(\lambda\)
is effectively ignored since \(n_j+\lambda \approx n_j\). Yet when the
\(n_j\) is small, then the estimate \(\hat{\beta}_i(\lambda)\) is
shrunken towards 0. The larger the \(\lambda\), the more we
shrink\autocite{IDS2_23-6}.

We will implement the \emph{Regularization} method on our models
(starting from the current model) in two steps:

\begin{enumerate}
\def\labelenumi{\arabic{enumi}.}
\item
  \textbf{Preconfiguration:} Preliminary determination of the optimal
  range of \(\lambda\) values for the \texttt{5-Fold\ Cross\ Validation}
  samples;
\item
  \textbf{Fine-tuning:} figuring out the value of \(\lambda\) that
  minimizes the model's RMSE.
\end{enumerate}

\paragraph{Regularization: Common Helper
Functions}\label{regularization-common-helper-functions}

\hfill\break

\begin{noteblock}
The full source code of the functions described below are available in
the
\href{https://github.com/AzKurban-edX-DS/Capstone-MovieLens/blob/main/r/src/support-functions/common-helper.functions.R\#L74}{Model
Tuning} section of the
\href{https://github.com/AzKurban-edX-DS/Capstone-MovieLens/blob/main/r/src/support-functions/common-helper.functions.R}{common-helper.functions.R}
script on \emph{GitHub}.

\end{noteblock}

\subparagraph{\texorpdfstring{\href{https://github.com/AzKurban-edX-DS/Capstone-MovieLens/blob/main/r/src/support-functions/common-helper.functions.R\#L116}{tune.model\_param}
Function}{tune.model\_param Function}}\label{func.tune.model_param}

\hfill\break

The function searches for the parameter value corresponding to the
minimum value of the \texttt{RMSE} from the list of values specified by
the \texttt{param\_values} parameter.

Sugnature

\hfill\break

\begin{Shaded}
\begin{Highlighting}[]
\NormalTok{tune.model\_param }\OtherTok{\textless{}{-}} \ControlFlowTok{function}\NormalTok{(param\_values, }
\NormalTok{                             fn\_tune.test.param\_value, }
                             \AttributeTok{break.if\_min =} \ConstantTok{TRUE}\NormalTok{,}
                             \AttributeTok{steps.beyond\_min =} \DecValTok{2}\NormalTok{)\{}

\CommentTok{\# ...}
  \FunctionTok{list}\NormalTok{(}\AttributeTok{tuned.result =} \FunctionTok{data.frame}\NormalTok{(}\AttributeTok{RMSE =}\NormalTok{ RMSEs\_tmp,}
                                \AttributeTok{parameter.value =}\NormalTok{ param\_vals\_tmp),}
       \AttributeTok{best\_result =}\NormalTok{ param\_values.best\_result)}
\NormalTok{\}  }
\end{Highlighting}
\end{Shaded}

Parameters

\hfill\break

\begin{itemize}
\tightlist
\item
  \textbf{param\_values:} A list of values to search for the value
  corresponding to the minimum value of the \texttt{RMSE} ;
\item
  \textbf{fn\_tune.test.param\_value:} A helper function that calculates
  the value of the \texttt{RMSE} for a given parameter value.;
\item
  \textbf{break.if\_min = TRUE:} A Boolean parameter that determines
  whether the function should terminate after completing the number of
  steps specified by the parameter \texttt{steps.beyond\_min}, after the
  minimum value of the \texttt{RMSE} has been found;
\item
  \textbf{steps.beyond\_min = 2:} (takes effect only if
  \texttt{break.if\_min} parameter is \texttt{TRUE}) Specifies the
  number of steps after finding the minimum value of the \texttt{RMSE},
  upon completion of which the function should terminate.
\end{itemize}

Details

\hfill\break
During execution, the function uses a helper function specified by the
\texttt{fn\_tune.test.param\_value} parameter, which calculates the
\texttt{RMSE} value for the given parameter from the list determined by
the \texttt{param\_values} parameter.

\begin{noteblock}
Note that the algorithm assumes that the dependence of the \texttt{RMSE}
on the input parameter is a monotonically decreasing function until a
minimum is reached and monotonically increasing thereafter. That is, it
is assumed that the function has a single minimum on the given interval.

\end{noteblock}

Value

\hfill\break
The function returns a data structure containing the found value of the
input parameter \texttt{param\_values} for which the \texttt{RMSE} value
is minimal, as well as the minimum \texttt{RMSE} value itself, along
with a sequence of all calculated \texttt{RMSE} values.

Source Code

\hfill\break
Below is the most significant part of the code of the
\href{https://github.com/AzKurban-edX-DS/Capstone-MovieLens/blob/main/r/src/support-functions/common-helper.functions.R\#L116}{tune.model\_param}
function:

\begin{Shaded}
\begin{Highlighting}[]
\NormalTok{tune.model\_param }\OtherTok{\textless{}{-}} \ControlFlowTok{function}\NormalTok{(param\_values, }
\NormalTok{                             fn\_tune.test.param\_value, }
                             \AttributeTok{break.if\_min =} \ConstantTok{TRUE}\NormalTok{,}
                             \AttributeTok{steps.beyond\_min =} \DecValTok{2}\NormalTok{)\{}
\NormalTok{  n }\OtherTok{\textless{}{-}} \FunctionTok{length}\NormalTok{(param\_values)}
\NormalTok{  param\_vals\_tmp }\OtherTok{\textless{}{-}} \FunctionTok{numeric}\NormalTok{()}
\NormalTok{  RMSEs\_tmp }\OtherTok{\textless{}{-}} \FunctionTok{numeric}\NormalTok{()}
\NormalTok{  RMSE\_min }\OtherTok{\textless{}{-}} \ConstantTok{Inf}
\NormalTok{  i\_max.beyond\_RMSE\_min }\OtherTok{\textless{}{-}} \ConstantTok{Inf}
\NormalTok{  prm\_val.best }\OtherTok{\textless{}{-}} \ConstantTok{NA}

  \CommentTok{\# ...}
  
  \ControlFlowTok{for}\NormalTok{ (i }\ControlFlowTok{in} \DecValTok{1}\SpecialCharTok{:}\NormalTok{n) \{}
    \FunctionTok{put\_log1}\NormalTok{(}\StringTok{"Function: \textasciigrave{}tune.model\_param\textasciigrave{}:}
\StringTok{Iteration \%1"}\NormalTok{, i)}
\NormalTok{    prm\_val }\OtherTok{\textless{}{-}}\NormalTok{ param\_values[i]}
\NormalTok{    param\_vals\_tmp[i] }\OtherTok{\textless{}{-}}\NormalTok{ prm\_val}
    
\NormalTok{    RMSE\_tmp }\OtherTok{\textless{}{-}} \FunctionTok{fn\_tune.test.param\_value}\NormalTok{(prm\_val)}
\NormalTok{    RMSEs\_tmp[i] }\OtherTok{\textless{}{-}}\NormalTok{ RMSE\_tmp}

    \FunctionTok{plot}\NormalTok{(param\_vals\_tmp[RMSEs\_tmp], RMSEs\_tmp[RMSEs\_tmp])}

    \ControlFlowTok{if}\NormalTok{(RMSE\_tmp }\SpecialCharTok{\textgreater{}}\NormalTok{ RMSE\_min)\{}
      \FunctionTok{warning}\NormalTok{(}\StringTok{"Function: \textasciigrave{}tune.model\_param\textasciigrave{}:}
\StringTok{\textasciigrave{}RSME\textasciigrave{} reached its minimum: "}\NormalTok{, RMSE\_min, }\StringTok{"}
\StringTok{for parameter value: "}\NormalTok{, prm\_val)}
      \FunctionTok{put\_log2}\NormalTok{(}\StringTok{"Function: \textasciigrave{}tune.model\_param\textasciigrave{}:}
\StringTok{Current \textasciigrave{}RMSE\textasciigrave{} value is \%1 related to parameter value: \%2"}\NormalTok{,}
\NormalTok{               RMSE\_tmp,}
\NormalTok{               prm\_val)}
      
      \ControlFlowTok{if}\NormalTok{ (i }\SpecialCharTok{\textgreater{}}\NormalTok{ i\_max.beyond\_RMSE\_min) \{}
        \FunctionTok{warning}\NormalTok{(}\StringTok{"Function: \textasciigrave{}tune.model\_param\textasciigrave{}:}
\StringTok{Operation is breaked (after \textasciigrave{}RSME\textasciigrave{} reached its minimum) on the following step: "}\NormalTok{, i)}
        \ControlFlowTok{break}
\NormalTok{      \}}
      \ControlFlowTok{next}
\NormalTok{    \}}

\NormalTok{    RMSE\_min }\OtherTok{\textless{}{-}}\NormalTok{ RMSE\_tmp}
\NormalTok{    prm\_val.best }\OtherTok{\textless{}{-}}\NormalTok{ prm\_val}
    
    \ControlFlowTok{if}\NormalTok{ (break.if\_min) \{}
\NormalTok{      i\_max.beyond\_RMSE\_min }\OtherTok{\textless{}{-}}\NormalTok{ i }\SpecialCharTok{+}\NormalTok{ steps.beyond\_min}
\NormalTok{    \}}
\NormalTok{  \}}
  
\NormalTok{  param\_values.best\_result }\OtherTok{\textless{}{-}} \FunctionTok{c}\NormalTok{(}\AttributeTok{param.best\_value =}\NormalTok{ prm\_val.best, }
                                \AttributeTok{best\_RMSE =}\NormalTok{ RMSE\_min)}
  
  
  \FunctionTok{list}\NormalTok{(}\AttributeTok{tuned.result =} \FunctionTok{data.frame}\NormalTok{(}\AttributeTok{RMSE =}\NormalTok{ RMSEs\_tmp,}
                                \AttributeTok{parameter.value =}\NormalTok{ param\_vals\_tmp),}
       \AttributeTok{best\_result =}\NormalTok{ param\_values.best\_result)}
\NormalTok{\}}
\end{Highlighting}
\end{Shaded}

\subparagraph{\texorpdfstring{\href{https://github.com/AzKurban-edX-DS/Capstone-MovieLens/blob/main/r/src/support-functions/common-helper.functions.R\#L198}{model.tune.param\_range}
Function}{model.tune.param\_range Function}}\label{model.tune.param_range-function}

\hfill\break

The function fine-tunes the model by searching for the best possible
value of the input parameter over a given interval for which the
corresponding \texttt{RMSE} value is minimal.

Sugnature

\hfill\break

\begin{Shaded}
\begin{Highlighting}[]
\NormalTok{model.tune.param\_range }\OtherTok{\textless{}{-}} \ControlFlowTok{function}\NormalTok{(loop\_starter,}
\NormalTok{                             tune\_dir\_path,}
\NormalTok{                             cache\_file\_base\_name,}
\NormalTok{                             fn\_tune.test.param\_value,}
                             \AttributeTok{max.identical.min\_RMSE.count =} \DecValTok{4}\NormalTok{,}
                             \AttributeTok{is.cv =} \ConstantTok{TRUE}\NormalTok{,}
                             \AttributeTok{endpoint.min\_diff =} \DecValTok{0}\NormalTok{,}
                             \AttributeTok{break.if\_min =} \ConstantTok{TRUE}\NormalTok{,}
                             \AttributeTok{steps.beyond\_min =} \DecValTok{2}\NormalTok{)\{}
  \CommentTok{\# ...}
  \FunctionTok{list}\NormalTok{(}\AttributeTok{best\_result =}\NormalTok{ param\_values.best\_result,}
       \AttributeTok{param\_values.endpoints =} \FunctionTok{c}\NormalTok{(prm\_val.leftmost, prm\_val.rightmost, seq\_increment),}
       \AttributeTok{tuned.result =} \FunctionTok{data.frame}\NormalTok{(}\AttributeTok{parameter.value =}\NormalTok{ parameter.value,}
                                 \AttributeTok{RMSE =}\NormalTok{ result.RMSE))}
\NormalTok{\}}
\end{Highlighting}
\end{Shaded}

Parameters

\hfill\break

loop\_starter

\hfill\break

A numeric vector of the form \texttt{c(start,\ end,\ dvs)}, where
\texttt{start} and \texttt{end} are the endpoints of the interval on
which the parameter value that minimizes \texttt{RMSE} is sought.
\texttt{dvs} is a divisor for splitting the interval to transform it
into a sequence of values among which the value that minimizes
\texttt{RMSE} is sought. For this purpose, the sequence step is
calculated as follows: \[
step = \frac{end - start}{dvs}
\]

The sequence obtained as a result of the transformation is equivalent to
the one generated by the function \texttt{seq} as follows:

\begin{Shaded}
\begin{Highlighting}[]
\FunctionTok{seq}\NormalTok{(start, end, step)}
\end{Highlighting}
\end{Shaded}

In fact, the \texttt{seq} function is called internally to generate the
sequence during the execution of the \texttt{model.tune.param\_range}
function.

tune\_dir\_path

\hfill\break
To improve performance, the algorithm caches intermediate results in the
file system. This parameter specifies the path to the directory where
the files are cached.

cache\_file\_base\_name

\hfill\break

The algorithm generates unique names for cache files based on this and
the \texttt{loop\_starter} parameter, as well as some other intermediate
values calculated during the execution.

fn\_tune.test.param\_value

\hfill\break

This is a helper function name that is passed to the same-named
parameter of the
\href{https://github.com/AzKurban-edX-DS/Capstone-MovieLens/blob/main/r/src/support-functions/common-helper.functions.R\#L116}{tune.model\_param}
function that is called internally during the execution (see the
description of the \texttt{tune.model\_param} function
\hyperref[func.tune.model_param]{above}).

max.identical.min\_RMSE.count

\hfill\break

\hfill\break

\begin{itemize}
\tightlist
\item
  \textbf{is.cv:} ;
\end{itemize}

\hfill\break

\begin{itemize}
\tightlist
\item
  \textbf{endpoint.min\_diff:} ;
\end{itemize}

\hfill\break

\begin{itemize}
\tightlist
\item
  \textbf{break.if\_min = TRUE:} A Boolean parameter that determines
  whether the function should terminate after completing the number of
  steps
\end{itemize}

\hfill\break

specified by the parameter \texttt{steps.beyond\_min}, after the minimum
value of the \texttt{RMSE} has been found;

\begin{itemize}
\tightlist
\item
  \textbf{steps.beyond\_min = 2:} (takes effect only if
  \texttt{break.if\_min} parameter is \texttt{TRUE}) Specifies the
  number of steps after finding the minimum value of the \texttt{RMSE},
  upon completion of which the function should terminate.
\end{itemize}

\hfill\break

Details

\hfill\break
During execution, the function uses a helper function specified by the
\texttt{fn\_tune.test.param\_value} parameter, which calculates the
\texttt{RMSE} value for the given parameter from the list determined by
the inner \texttt{param\_values} sequence.

\begin{noteblock}
Note

\end{noteblock}

Value

\hfill\break

Source Code

\hfill\break
Below is the simplified version of the code of the
\href{https://github.com/AzKurban-edX-DS/Capstone-MovieLens/blob/main/r/src/support-functions/common-helper.functions.R\#L198}{model.tune.param\_range}
function:

\begin{Shaded}
\begin{Highlighting}[]
\NormalTok{model.tune.param\_range }\OtherTok{\textless{}{-}} \ControlFlowTok{function}\NormalTok{(loop\_starter,}
\NormalTok{                             tune\_dir\_path,}
\NormalTok{                             cache\_file\_base\_name,}
\NormalTok{                             fn\_tune.test.param\_value,}
                             \AttributeTok{max.identical.min\_RMSE.count =} \DecValTok{4}\NormalTok{,}
                             \AttributeTok{is.cv =} \ConstantTok{TRUE}\NormalTok{,}
                             \AttributeTok{endpoint.min\_diff =} \DecValTok{0}\NormalTok{, }\CommentTok{\#1e{-}07,}
                             \AttributeTok{break.if\_min =} \ConstantTok{TRUE}\NormalTok{,}
                             \AttributeTok{steps.beyond\_min =} \DecValTok{2}\NormalTok{)\{}

\NormalTok{  seq\_start }\OtherTok{\textless{}{-}}\NormalTok{ loop\_starter[}\DecValTok{1}\NormalTok{]}
\NormalTok{  seq\_end }\OtherTok{\textless{}{-}}\NormalTok{ loop\_starter[}\DecValTok{2}\NormalTok{]}
\NormalTok{  range\_divider }\OtherTok{\textless{}{-}}\NormalTok{ loop\_starter[}\DecValTok{3}\NormalTok{]}
  
  \ControlFlowTok{if}\NormalTok{ (range\_divider }\SpecialCharTok{\textless{}} \DecValTok{4}\NormalTok{) \{}
\NormalTok{    range\_divider }\OtherTok{\textless{}{-}} \DecValTok{4}
\NormalTok{  \}}
  
\NormalTok{  prm\_val.leftmost }\OtherTok{\textless{}{-}}\NormalTok{ seq\_start}
\NormalTok{  prm\_val.rightmost }\OtherTok{\textless{}{-}}\NormalTok{ seq\_end}
  
\NormalTok{  RMSE.leftmost }\OtherTok{\textless{}{-}} \ConstantTok{NA}
\NormalTok{  RMSE.rightmost }\OtherTok{\textless{}{-}} \ConstantTok{NA}

\NormalTok{  best\_RMSE }\OtherTok{\textless{}{-}} \ConstantTok{NA}
\NormalTok{  param.best\_value }\OtherTok{\textless{}{-}} \DecValTok{0}
  
  
\NormalTok{  param\_values.best\_result }\OtherTok{\textless{}{-}} \FunctionTok{c}\NormalTok{(}\AttributeTok{param.best\_value =}\NormalTok{ param.best\_value, }
                                \AttributeTok{best\_RMSE =}\NormalTok{ best\_RMSE)}
  \CommentTok{\# Start repeat loop}
  \ControlFlowTok{repeat}\NormalTok{\{}
\NormalTok{    seq\_increment }\OtherTok{\textless{}{-}}\NormalTok{ (seq\_end }\SpecialCharTok{{-}}\NormalTok{ seq\_start)}\SpecialCharTok{/}\NormalTok{range\_divider }
    
    \ControlFlowTok{if}\NormalTok{ (seq\_increment }\SpecialCharTok{\textless{}} \FloatTok{0.0000000000001}\NormalTok{) \{}
      \FunctionTok{warning}\NormalTok{(}\StringTok{"Function \textasciigrave{}model.tune.param\_range\textasciigrave{}:}
\StringTok{parameter value increment is too small."}\NormalTok{)}
      \ControlFlowTok{break}
\NormalTok{    \}}
    
\NormalTok{    test\_param\_vals }\OtherTok{\textless{}{-}} \FunctionTok{seq}\NormalTok{(seq\_start, seq\_end, seq\_increment)}
    
\NormalTok{    tuned\_result }\OtherTok{\textless{}{-}} \FunctionTok{tune.model\_param}\NormalTok{(test\_param\_vals, }
\NormalTok{                                  fn\_tune.test.param\_value,}
\NormalTok{                                  break.if\_min,}
\NormalTok{                                  steps.beyond\_min)}
    
\NormalTok{    tuned.result }\OtherTok{\textless{}{-}}\NormalTok{ tuned\_result}\SpecialCharTok{$}\NormalTok{tuned.result}
    \FunctionTok{plot}\NormalTok{(tuned.result}\SpecialCharTok{$}\NormalTok{parameter.value, tuned.result}\SpecialCharTok{$}\NormalTok{RMSE)}
    
\NormalTok{    bound.idx }\OtherTok{\textless{}{-}} \FunctionTok{get\_fine\_tune.param.endpoints.idx}\NormalTok{(tuned.result)}
\NormalTok{    start.idx }\OtherTok{\textless{}{-}}\NormalTok{ bound.idx[}\StringTok{"start"}\NormalTok{]}
\NormalTok{    end.idx }\OtherTok{\textless{}{-}}\NormalTok{ bound.idx[}\StringTok{"end"}\NormalTok{]}
\NormalTok{    best\_RMSE.idx }\OtherTok{\textless{}{-}}\NormalTok{ bound.idx[}\StringTok{"best"}\NormalTok{]}
    
\NormalTok{    prm\_val.leftmost.tmp }\OtherTok{\textless{}{-}}\NormalTok{ tuned.result}\SpecialCharTok{$}\NormalTok{parameter.value[start.idx]}
\NormalTok{    RMSE.leftmost.tmp }\OtherTok{\textless{}{-}}\NormalTok{ tuned.result}\SpecialCharTok{$}\NormalTok{RMSE[start.idx]}

\NormalTok{    prm\_val.rightmost.tmp }\OtherTok{\textless{}{-}}\NormalTok{ tuned.result}\SpecialCharTok{$}\NormalTok{parameter.value[end.idx]}
\NormalTok{    RMSE.rightmost.tmp }\OtherTok{\textless{}{-}}\NormalTok{ tuned.result}\SpecialCharTok{$}\NormalTok{RMSE[end.idx]}
    
\NormalTok{    min\_RMSE }\OtherTok{\textless{}{-}}\NormalTok{ tuned.result}\SpecialCharTok{$}\NormalTok{RMSE[best\_RMSE.idx]}
\NormalTok{    min\_RMSE.prm\_val }\OtherTok{\textless{}{-}}\NormalTok{ tuned.result}\SpecialCharTok{$}\NormalTok{parameter.value[best\_RMSE.idx]}

\NormalTok{    seq\_start }\OtherTok{\textless{}{-}}\NormalTok{ prm\_val.leftmost.tmp}
\NormalTok{    seq\_end }\OtherTok{\textless{}{-}}\NormalTok{ prm\_val.rightmost.tmp}
    
    \ControlFlowTok{if}\NormalTok{ (}\FunctionTok{is.na}\NormalTok{(best\_RMSE)) \{}
\NormalTok{      prm\_val.leftmost }\OtherTok{\textless{}{-}}\NormalTok{ prm\_val.leftmost.tmp}
\NormalTok{      RMSE.leftmost }\OtherTok{\textless{}{-}}\NormalTok{ RMSE.leftmost.tmp}
      
\NormalTok{      prm\_val.rightmost }\OtherTok{\textless{}{-}}\NormalTok{ prm\_val.rightmost.tmp}
\NormalTok{      RMSE.rightmost }\OtherTok{\textless{}{-}}\NormalTok{ RMSE.rightmost.tmp}
      
\NormalTok{      param.best\_value }\OtherTok{\textless{}{-}}\NormalTok{ min\_RMSE.prm\_val}
\NormalTok{      best\_RMSE }\OtherTok{\textless{}{-}}\NormalTok{ min\_RMSE}
\NormalTok{    \}}
    
    \ControlFlowTok{if}\NormalTok{ (RMSE.leftmost.tmp }\SpecialCharTok{{-}}\NormalTok{ min\_RMSE }\SpecialCharTok{\textgreater{}=}\NormalTok{ endpoint.min\_diff) \{}
\NormalTok{      prm\_val.leftmost }\OtherTok{\textless{}{-}}\NormalTok{ prm\_val.leftmost.tmp}
\NormalTok{      RMSE.leftmost }\OtherTok{\textless{}{-}}\NormalTok{ RMSE.leftmost.tmp}
\NormalTok{    \} }
    
    \ControlFlowTok{if}\NormalTok{ (RMSE.rightmost.tmp }\SpecialCharTok{{-}}\NormalTok{ min\_RMSE }\SpecialCharTok{\textgreater{}=}\NormalTok{ endpoint.min\_diff) \{}
\NormalTok{      prm\_val.rightmost }\OtherTok{\textless{}{-}}\NormalTok{ prm\_val.rightmost.tmp}
\NormalTok{      RMSE.rightmost }\OtherTok{\textless{}{-}}\NormalTok{ RMSE.rightmost.tmp}
\NormalTok{    \} }
    
    \ControlFlowTok{if}\NormalTok{ (end.idx }\SpecialCharTok{{-}}\NormalTok{ start.idx }\SpecialCharTok{\textless{}=} \DecValTok{0}\NormalTok{) \{}
      \FunctionTok{warning}\NormalTok{(}\StringTok{"\textasciigrave{}tuned.result$parameter.value\textasciigrave{} sequential start index are the same or greater than end one."}\NormalTok{)}
      \ControlFlowTok{break}
\NormalTok{    \}}
    
    \ControlFlowTok{if}\NormalTok{ (best\_RMSE }\SpecialCharTok{==}\NormalTok{ min\_RMSE) \{}
      \FunctionTok{warning}\NormalTok{(}\StringTok{"Currently computed minimal RMSE equals the previously reached best one: "}\NormalTok{,}
\NormalTok{              best\_RMSE, }\StringTok{"}
\StringTok{Currently computed minial value is: "}\NormalTok{, min\_RMSE)}

      \ControlFlowTok{if}\NormalTok{ (}\FunctionTok{sum}\NormalTok{(tuned.result}\SpecialCharTok{$}\NormalTok{RMSE[tuned.result}\SpecialCharTok{$}\NormalTok{RMSE }\SpecialCharTok{==}\NormalTok{ min\_RMSE]) }\SpecialCharTok{\textgreater{}=}\NormalTok{ max.identical.min\_RMSE.count) \{}
        \FunctionTok{warning}\NormalTok{(}\StringTok{"Minimal \textasciigrave{}RMSE\textasciigrave{}identical values count reached it maximum allowed value: "}\NormalTok{,}
\NormalTok{                max.identical.min\_RMSE.count)}

\NormalTok{        param\_values.best\_result }\OtherTok{\textless{}{-}}
          \FunctionTok{get\_best\_param.result}\NormalTok{(tuned.result}\SpecialCharTok{$}\NormalTok{parameter.value,}
\NormalTok{                                tuned.result}\SpecialCharTok{$}\NormalTok{RMSE)}
        \ControlFlowTok{break}
\NormalTok{      \}}

\NormalTok{    \} }\ControlFlowTok{else} \ControlFlowTok{if}\NormalTok{ (best\_RMSE }\SpecialCharTok{\textless{}}\NormalTok{ min\_RMSE) \{}
      \FunctionTok{stop}\NormalTok{(}\StringTok{"Current minimal RMSE is greater than previously computed best value: "}\NormalTok{,}
\NormalTok{           best\_RMSE, }\StringTok{"}
\StringTok{Currently computed minial value is: "}\NormalTok{, min\_RMSE)}
\NormalTok{    \}}

\NormalTok{    best\_RMSE }\OtherTok{\textless{}{-}}\NormalTok{ min\_RMSE}
\NormalTok{    param.best\_value }\OtherTok{\textless{}{-}}\NormalTok{ min\_RMSE.prm\_val}

\NormalTok{    param\_values.best\_result }\OtherTok{\textless{}{-}} 
      \FunctionTok{get\_best\_param.result}\NormalTok{(tuned.result}\SpecialCharTok{$}\NormalTok{parameter.value, }
\NormalTok{                          tuned.result}\SpecialCharTok{$}\NormalTok{RMSE)}
\NormalTok{  \}}
  \CommentTok{\# End repeat loop}
  
\NormalTok{  n }\OtherTok{\textless{}{-}} \FunctionTok{length}\NormalTok{(tuned.result}\SpecialCharTok{$}\NormalTok{parameter.value)}
\NormalTok{  parameter.value }\OtherTok{\textless{}{-}}\NormalTok{ tuned.result}\SpecialCharTok{$}\NormalTok{parameter.value}
\NormalTok{  result.RMSE }\OtherTok{\textless{}{-}}\NormalTok{ tuned.result}\SpecialCharTok{$}\NormalTok{RMSE}
  
  \ControlFlowTok{if}\NormalTok{ (result.RMSE[}\DecValTok{1}\NormalTok{] }\SpecialCharTok{==}\NormalTok{ best\_RMSE) \{}
\NormalTok{    parameter.value[}\DecValTok{1}\NormalTok{] }\OtherTok{\textless{}{-}}\NormalTok{ prm\_val.leftmost}
\NormalTok{    result.RMSE[}\DecValTok{1}\NormalTok{] }\OtherTok{\textless{}{-}}\NormalTok{ RMSE.leftmost}
\NormalTok{  \}}
  \ControlFlowTok{if}\NormalTok{ (result.RMSE[n] }\SpecialCharTok{==}\NormalTok{ best\_RMSE) \{}
\NormalTok{    parameter.value[n}\SpecialCharTok{+}\DecValTok{1}\NormalTok{] }\OtherTok{\textless{}{-}}\NormalTok{ prm\_val.rightmost}
\NormalTok{    result.RMSE[n}\SpecialCharTok{+}\DecValTok{1}\NormalTok{] }\OtherTok{\textless{}{-}}\NormalTok{ RMSE.rightmost}
    \CommentTok{\# browser()}
\NormalTok{  \}}
  
  \FunctionTok{list}\NormalTok{(}\AttributeTok{best\_result =}\NormalTok{ param\_values.best\_result,}
       \AttributeTok{param\_values.endpoints =} \FunctionTok{c}\NormalTok{(prm\_val.leftmost, prm\_val.rightmost, seq\_increment),}
       \AttributeTok{tuned.result =} \FunctionTok{data.frame}\NormalTok{(}\AttributeTok{parameter.value =}\NormalTok{ parameter.value,}
                                 \AttributeTok{RMSE =}\NormalTok{ result.RMSE))}
\NormalTok{\}}
\end{Highlighting}
\end{Shaded}

\begin{noteblock}
The full version of the code of the
\href{https://github.com/AzKurban-edX-DS/Capstone-MovieLens/blob/main/r/src/support-functions/common-helper.functions.R\#L198}{model.tune.param\_range}
is available in the
\href{https://github.com/AzKurban-edX-DS/Capstone-MovieLens/blob/main/r/src/support-functions/common-helper.functions.R\#L74}{Model
Tuning} section of the
\href{https://github.com/AzKurban-edX-DS/Capstone-MovieLens/blob/main/r/src/support-functions/common-helper.functions.R\#L198}{common-helper.functions.R}
script.

\end{noteblock}

\subparagraph{`` Function}\label{function}

\hfill\break

Sugnature

\hfill\break

Parameters

\hfill\break
- \textbf{fn\_tune.test.param\_value:} A helper function that calculates
the value of the \texttt{RMSE} for a given parameter value.; -
\textbf{break.if\_min = TRUE:} A Boolean parameter that determines
whether the function should terminate after completing the number of
steps specified by the parameter \texttt{steps.beyond\_min}, after the
minimum value of the \texttt{RMSE} has been found; -
\textbf{steps.beyond\_min = 2:} (takes effect only if
\texttt{break.if\_min} parameter is \texttt{TRUE}) Specifies the number
of steps after finding the minimum value of the \texttt{RMSE}, upon
completion of which the function should terminate.

Details

\hfill\break
During execution, the function uses a helper function specified by the
\texttt{fn\_tune.test.param\_value} parameter, which calculates the
\texttt{RMSE} value for the given parameter from the list determined by
the \texttt{param\_values} parameter.

\begin{noteblock}
Note

\end{noteblock}

Value

\hfill\break

Source Code

\hfill\break
Below is the most significant part of the code of the \href{}{function}
function:

\subparagraph{`` Function}\label{function-1}

\hfill\break

Sugnature

\hfill\break

Parameters

\hfill\break
- \textbf{fn\_tune.test.param\_value:} A helper function that calculates
the value of the \texttt{RMSE} for a given parameter value.; -
\textbf{break.if\_min = TRUE:} A Boolean parameter that determines
whether the function should terminate after completing the number of
steps specified by the parameter \texttt{steps.beyond\_min}, after the
minimum value of the \texttt{RMSE} has been found; -
\textbf{steps.beyond\_min = 2:} (takes effect only if
\texttt{break.if\_min} parameter is \texttt{TRUE}) Specifies the number
of steps after finding the minimum value of the \texttt{RMSE}, upon
completion of which the function should terminate.

Details

\hfill\break
During execution, the function uses a helper function specified by the
\texttt{fn\_tune.test.param\_value} parameter, which calculates the
\texttt{RMSE} value for the given parameter from the list determined by
the \texttt{param\_values} parameter.

\begin{noteblock}
Note

\end{noteblock}

Value

\hfill\break

Source Code

\hfill\break
Below is the most significant part of the code of the \href{}{function}
function:

\begin{noteblock}
!!! Note!

\end{noteblock}

\subparagraph{`` Function}\label{function-2}

\hfill\break

Sugnature

\hfill\break

Parameters

\hfill\break
- \textbf{fn\_tune.test.param\_value:} A helper function that calculates
the value of the \texttt{RMSE} for a given parameter value.; -
\textbf{break.if\_min = TRUE:} A Boolean parameter that determines
whether the function should terminate after completing the number of
steps specified by the parameter \texttt{steps.beyond\_min}, after the
minimum value of the \texttt{RMSE} has been found; -
\textbf{steps.beyond\_min = 2:} (takes effect only if
\texttt{break.if\_min} parameter is \texttt{TRUE}) Specifies the number
of steps after finding the minimum value of the \texttt{RMSE}, upon
completion of which the function should terminate.

Details

\hfill\break
During execution, the function uses a helper function specified by the
\texttt{fn\_tune.test.param\_value} parameter, which calculates the
\texttt{RMSE} value for the given parameter from the list determined by
the \texttt{param\_values} parameter.

\begin{noteblock}
Note

\end{noteblock}

Value

\hfill\break

Source Code

\hfill\break
Below is the most significant part of the code of the \href{}{function}
function:

\begin{noteblock}
!!! Note!

\end{noteblock}

\begin{noteblock}
!!! Note!

\end{noteblock}

\begin{noteblock}
!!! Note!

\end{noteblock}

\subparagraph{`` Function}\label{function-3}

\hfill\break

Sugnature

\hfill\break

Parameters

\hfill\break
- \textbf{fn\_tune.test.param\_value:} A helper function that calculates
the value of the \texttt{RMSE} for a given parameter value.; -
\textbf{break.if\_min = TRUE:} A Boolean parameter that determines
whether the function should terminate after completing the number of
steps specified by the parameter \texttt{steps.beyond\_min}, after the
minimum value of the \texttt{RMSE} has been found; -
\textbf{steps.beyond\_min = 2:} (takes effect only if
\texttt{break.if\_min} parameter is \texttt{TRUE}) Specifies the number
of steps after finding the minimum value of the \texttt{RMSE}, upon
completion of which the function should terminate.

Details

\hfill\break
During execution, the function uses a helper function specified by the
\texttt{fn\_tune.test.param\_value} parameter, which calculates the
\texttt{RMSE} value for the given parameter from the list determined by
the \texttt{param\_values} parameter.

\begin{noteblock}
Note

\end{noteblock}

Value

\hfill\break

Source Code

\hfill\break
Below is the most significant part of the code of the \href{}{function}
function:

\paragraph{UME Model Regularization: Support
Function}\label{ume-model-regularization-support-function}

\hfill\break

\begin{noteblock}
The \href{}{regularize.test\_lambda.UM\_effect.cv} function described
below are defined in the \href{}{Regularization} section of the
\href{}{UM-effect.functions.R} script.

\end{noteblock}

\subparagraph{\texorpdfstring{\texttt{regularize.test\_lambda.UM\_effect.cv}
Function}{regularize.test\_lambda.UM\_effect.cv Function}}\label{regularize.test_lambda.um_effect.cv-function}

\hfill\break

This function calculates \emph{RMSE} of the \emph{UME Model} using
\emph{5-Fold CV} method for the given \(\lambda\) parameter value:

\begin{Shaded}
\begin{Highlighting}[]
\NormalTok{regularize.test\_lambda.UM\_effect.cv }\OtherTok{\textless{}{-}} \ControlFlowTok{function}\NormalTok{(lambda)\{}
  \ControlFlowTok{if}\NormalTok{ (}\FunctionTok{is.na}\NormalTok{(lambda)) \{}
    \FunctionTok{stop}\NormalTok{(}\StringTok{"Function: regularize.test\_lambda.UM\_effect.cv}
\StringTok{\textasciigrave{}lambda\textasciigrave{} is \textasciigrave{}NA\textasciigrave{}"}\NormalTok{)}
\NormalTok{  \}}
\NormalTok{  um\_effect }\OtherTok{\textless{}{-}} \FunctionTok{train\_user\_movie\_effect.cv}\NormalTok{(lambda)}
  \FunctionTok{calc\_user\_movie\_effect\_RMSE.cv}\NormalTok{(um\_effect)}
\NormalTok{\}}
\end{Highlighting}
\end{Shaded}

\begin{noteblock}
Note that we reuse the function \href{}{train\_user\_movie\_effect.cv}
calling it from the \href{}{regularize.test\_lambda.UM\_effect.cv}, but
now with the \(\lambda\) parameter different from the default (zero)
value.

\end{noteblock}

Let's now figure out the \(\lambda\) that minimizes the \emph{RMSE}:

\begin{Shaded}
\begin{Highlighting}[]
\CommentTok{\# Here we will simply compute the RMSE for different values of \textasciigrave{}lambda\textasciigrave{} }
\NormalTok{n }\OtherTok{\textless{}{-}} \FunctionTok{colSums}\NormalTok{(}\SpecialCharTok{!}\FunctionTok{is.na}\NormalTok{(y))}

\NormalTok{sums }\OtherTok{\textless{}{-}} \FunctionTok{colSums}\NormalTok{(y }\SpecialCharTok{{-}}\NormalTok{ mu }\SpecialCharTok{{-}}\NormalTok{ a, }\AttributeTok{na.rm =} \ConstantTok{TRUE}\NormalTok{)}
\NormalTok{lambdas }\OtherTok{\textless{}{-}} \FunctionTok{seq}\NormalTok{(}\DecValTok{0}\NormalTok{, }\DecValTok{10}\NormalTok{, }\FloatTok{0.1}\NormalTok{)}

\NormalTok{rmses }\OtherTok{\textless{}{-}} \FunctionTok{sapply}\NormalTok{(lambdas, }\ControlFlowTok{function}\NormalTok{(lambda)\{}
\NormalTok{  b }\OtherTok{\textless{}{-}}\NormalTok{  sums }\SpecialCharTok{/}\NormalTok{ (n }\SpecialCharTok{+}\NormalTok{ lambda)}
  \FunctionTok{reg\_rmse}\NormalTok{(b)}
\NormalTok{\})}

\CommentTok{\# Here is a plot of the RMSE versus \textasciigrave{}lambda\textasciigrave{}:}
\FunctionTok{plot}\NormalTok{(lambdas, rmses, }\AttributeTok{type =} \StringTok{"l"}\NormalTok{)}
\end{Highlighting}
\end{Shaded}

Now we can determine the minimal \emph{RMSE}:

\begin{Shaded}
\begin{Highlighting}[]
\CommentTok{\# print(min(rmses))}
\end{Highlighting}
\end{Shaded}

which is achieved for the following \(\lambda\):

\begin{Shaded}
\begin{Highlighting}[]
\NormalTok{lambda }\OtherTok{\textless{}{-}}\NormalTok{ lambdas[}\FunctionTok{which.min}\NormalTok{(rmses)] }
\FunctionTok{print}\NormalTok{(lambda)}
\end{Highlighting}
\end{Shaded}

Using this \(\lambda\) we can compute the regularized estimates:

\begin{Shaded}
\begin{Highlighting}[]
\NormalTok{b\_reg }\OtherTok{\textless{}{-}}\NormalTok{ sums }\SpecialCharTok{/}\NormalTok{ (n }\SpecialCharTok{+}\NormalTok{ lambda)}

\FunctionTok{str}\NormalTok{(b\_reg)}
\end{Highlighting}
\end{Shaded}

Finally, let's verify that the penalized estimates
\(\hat{b}_i(\lambda)\) we have just computed actually result in the
minimal \emph{RMSE} figured out above:

\begin{Shaded}
\begin{Highlighting}[]
\FunctionTok{reg\_rmse}\NormalTok{(b\_reg)}
\end{Highlighting}
\end{Shaded}

\subsubsection{Accounting for Date
effects}\label{accounting-for-date-effects}

\subparagraph{\texorpdfstring{Yearly rating
count\autocite{MRS-R-BEST}}{Yearly rating count{[}@MRS-R-BEST{]}}}\label{yearly-rating-countmrs-r-best}

\hfill\break

\begin{Shaded}
\begin{Highlighting}[]
\FunctionTok{print}\NormalTok{(edx }\SpecialCharTok{|\textgreater{}} 
  \FunctionTok{mutate}\NormalTok{(}\AttributeTok{year =} \FunctionTok{year}\NormalTok{(}\FunctionTok{as\_datetime}\NormalTok{(timestamp, }\AttributeTok{origin =} \StringTok{"1970{-}01{-}01"}\NormalTok{))) }\SpecialCharTok{|\textgreater{}}
  \FunctionTok{group\_by}\NormalTok{(year) }\SpecialCharTok{|\textgreater{}}
  \FunctionTok{summarize}\NormalTok{(}\AttributeTok{count =} \FunctionTok{n}\NormalTok{())}
\NormalTok{)}
\end{Highlighting}
\end{Shaded}

\begin{verbatim}
## # A tibble: 15 x 2
##     year   count
##    <dbl>   <int>
##  1  1995       2
##  2  1996  942772
##  3  1997  414101
##  4  1998  181634
##  5  1999  709893
##  6  2000 1144349
##  7  2001  683355
##  8  2002  524959
##  9  2003  619938
## 10  2004  691429
## 11  2005 1059277
## 12  2006  689315
## 13  2007  629168
## 14  2008  696740
## 15  2009   13123
\end{verbatim}

\subparagraph{\texorpdfstring{Average rating per year
plot\autocite{MRS-R-BEST}}{Average rating per year plot{[}@MRS-R-BEST{]}}}\label{average-rating-per-year-plotmrs-r-best}

\hfill\break

\begin{Shaded}
\begin{Highlighting}[]
\NormalTok{edx }\SpecialCharTok{|\textgreater{}} 
  \FunctionTok{mutate}\NormalTok{(}\AttributeTok{year =} \FunctionTok{year}\NormalTok{(}\FunctionTok{as\_datetime}\NormalTok{(timestamp, }\AttributeTok{origin =} \StringTok{"1970{-}01{-}01"}\NormalTok{))) }\SpecialCharTok{|\textgreater{}}
  \FunctionTok{group\_by}\NormalTok{(year) }\SpecialCharTok{|\textgreater{}}
  \FunctionTok{summarize}\NormalTok{(}\AttributeTok{rating\_avg =} \FunctionTok{mean}\NormalTok{(rating)) }\SpecialCharTok{|\textgreater{}}
  \FunctionTok{ggplot}\NormalTok{(}\FunctionTok{aes}\NormalTok{(}\AttributeTok{x =}\NormalTok{ year, }\AttributeTok{y =}\NormalTok{ rating\_avg)) }\SpecialCharTok{+}
  \FunctionTok{geom\_bar}\NormalTok{(}\AttributeTok{stat =} \StringTok{"identity"}\NormalTok{, }\AttributeTok{fill =} \StringTok{"\#8888ff"}\NormalTok{) }\SpecialCharTok{+} 
  \FunctionTok{ggtitle}\NormalTok{(}\StringTok{"Average rating per year"}\NormalTok{) }\SpecialCharTok{+}
  \FunctionTok{xlab}\NormalTok{(}\StringTok{"Year"}\NormalTok{) }\SpecialCharTok{+}
  \FunctionTok{ylab}\NormalTok{(}\StringTok{"Average rating"}\NormalTok{) }\SpecialCharTok{+}
  \FunctionTok{scale\_y\_continuous}\NormalTok{(}\AttributeTok{labels =}\NormalTok{ comma) }\SpecialCharTok{+} 
  \FunctionTok{theme\_economist}\NormalTok{() }\SpecialCharTok{+}
  \FunctionTok{theme}\NormalTok{(}\AttributeTok{axis.title.x =} \FunctionTok{element\_text}\NormalTok{(}\AttributeTok{vjust =} \SpecialCharTok{{-}}\DecValTok{5}\NormalTok{, }\AttributeTok{face =} \StringTok{"bold"}\NormalTok{), }
        \AttributeTok{axis.title.y =} \FunctionTok{element\_text}\NormalTok{(}\AttributeTok{vjust =} \DecValTok{10}\NormalTok{, }\AttributeTok{face =} \StringTok{"bold"}\NormalTok{), }
        \AttributeTok{plot.margin =} \FunctionTok{margin}\NormalTok{(}\FloatTok{0.7}\NormalTok{, }\FloatTok{0.5}\NormalTok{, }\DecValTok{1}\NormalTok{, }\FloatTok{1.2}\NormalTok{, }\StringTok{"cm"}\NormalTok{))}
\end{Highlighting}
\end{Shaded}

\includegraphics{capstone-movielens-report.draft4_files/figure-latex/unnamed-chunk-96-1.pdf}

We use the following models to account for the \texttt{date} effect:

\[
Y_{i,j} = \mu + \alpha_i + \beta_j + f(d_{i,j}) + \varepsilon_{i,j}
\]

\subsubsection{Accounting for Genre
effect}\label{accounting-for-genre-effect}

\hfill\break
As mentioned in
\href{https://rafalab.dfci.harvard.edu/dsbook-part-2/highdim/regularization.html\#exercises}{Section
23.7: Exercises} of the \emph{Chapter ``23 Regularization'' of the
Course Textbook} the \texttt{Movielens} dataset also has a genres
column. This column includes every genre that applies to the movie (some
movies fall under several genres)\autocite{IDS2_23-7}.

\paragraph{Genre Data Analysis}\label{genre-data-analysis}

\hfill\break
\#\#\#\#\# Movie Genres Data\\
The following code computes movie rating summaries by popular genres
like Drama, Comedy, Thriller, and Romance:

\begin{Shaded}
\begin{Highlighting}[]
\CommentTok{\#library(stringr)}
\NormalTok{genres }\OtherTok{=} \FunctionTok{c}\NormalTok{(}\StringTok{"Drama"}\NormalTok{, }\StringTok{"Comedy"}\NormalTok{, }\StringTok{"Thriller"}\NormalTok{, }\StringTok{"Romance"}\NormalTok{)}
\FunctionTok{sapply}\NormalTok{(genres, }\ControlFlowTok{function}\NormalTok{(g) \{}
  \FunctionTok{sum}\NormalTok{(}\FunctionTok{str\_detect}\NormalTok{(edx}\SpecialCharTok{$}\NormalTok{genres, g))}
\NormalTok{\})}
\end{Highlighting}
\end{Shaded}

Further, we can find out the movies that have the greatest number of
ratings using the following code:

\begin{Shaded}
\begin{Highlighting}[]
\NormalTok{ordered\_movie\_ratings }\OtherTok{\textless{}{-}}\NormalTok{ edx }\SpecialCharTok{|\textgreater{}} \FunctionTok{group\_by}\NormalTok{(movieId, title) }\SpecialCharTok{|\textgreater{}}
  \FunctionTok{summarize}\NormalTok{(}\AttributeTok{number\_of\_ratings =} \FunctionTok{n}\NormalTok{()) }\SpecialCharTok{|\textgreater{}}
  \FunctionTok{arrange}\NormalTok{(}\FunctionTok{desc}\NormalTok{(number\_of\_ratings))}
\FunctionTok{print}\NormalTok{(}\FunctionTok{head}\NormalTok{(ordered\_movie\_ratings))}
\end{Highlighting}
\end{Shaded}

and figure out the most given ratings in order from most to least:

\begin{Shaded}
\begin{Highlighting}[]
\NormalTok{ratings }\OtherTok{\textless{}{-}}\NormalTok{ edx }\SpecialCharTok{|\textgreater{}}  \FunctionTok{group\_by}\NormalTok{(rating) }\SpecialCharTok{|\textgreater{}}
     \FunctionTok{summarise}\NormalTok{(}\AttributeTok{count =} \FunctionTok{n}\NormalTok{()) }\SpecialCharTok{|\textgreater{}}
     \FunctionTok{arrange}\NormalTok{(}\FunctionTok{desc}\NormalTok{(count))}
\FunctionTok{print}\NormalTok{(ratings)}
\end{Highlighting}
\end{Shaded}

The following code allows us to summarize that in general, half-star
ratings are less common than whole-star ratings (e.g., there are fewer
ratings of 3.5 than there are ratings of 3 or 4, etc.):

\begin{Shaded}
\begin{Highlighting}[]
\FunctionTok{print}\NormalTok{(edx }\SpecialCharTok{|\textgreater{}} \FunctionTok{group\_by}\NormalTok{(rating) }\SpecialCharTok{|\textgreater{}} \FunctionTok{summarize}\NormalTok{(}\AttributeTok{count =} \FunctionTok{n}\NormalTok{()))}
\end{Highlighting}
\end{Shaded}

We can visually see that from the following plot:

\begin{Shaded}
\begin{Highlighting}[]
\NormalTok{edx }\SpecialCharTok{|\textgreater{}}
  \FunctionTok{group\_by}\NormalTok{(rating) }\SpecialCharTok{|\textgreater{}}
  \FunctionTok{summarize}\NormalTok{(}\AttributeTok{count =} \FunctionTok{n}\NormalTok{()) }\SpecialCharTok{|\textgreater{}}
  \FunctionTok{ggplot}\NormalTok{(}\FunctionTok{aes}\NormalTok{(}\AttributeTok{x =}\NormalTok{ rating, }\AttributeTok{y =}\NormalTok{ count)) }\SpecialCharTok{+}
  \FunctionTok{geom\_line}\NormalTok{() }
\end{Highlighting}
\end{Shaded}

\subparagraph{Movie Genres Effect}\label{movie-genres-effect}

\hfill\break

The plot below shows strong evidence of a genre effect (for illustrative
purposes, the plot shows only categories with more than 20, 000
ratings).

\begin{Shaded}
\begin{Highlighting}[]
\CommentTok{\# Preparing data for plotting:}
\NormalTok{genre\_ratins\_grp }\OtherTok{\textless{}{-}}\NormalTok{ train\_set }\SpecialCharTok{|\textgreater{}} 
  \FunctionTok{mutate}\NormalTok{(}\AttributeTok{genre\_categories =} \FunctionTok{as.factor}\NormalTok{(genres)) }\SpecialCharTok{|\textgreater{}}
  \FunctionTok{group\_by}\NormalTok{(genre\_categories) }\SpecialCharTok{|\textgreater{}}
  \FunctionTok{summarize}\NormalTok{(}\AttributeTok{n =} \FunctionTok{n}\NormalTok{(), }\AttributeTok{rating\_avg =} \FunctionTok{mean}\NormalTok{(rating), }\AttributeTok{se =} \FunctionTok{sd}\NormalTok{(rating)}\SpecialCharTok{/}\FunctionTok{sqrt}\NormalTok{(}\FunctionTok{n}\NormalTok{())) }\SpecialCharTok{|\textgreater{}}
  \FunctionTok{filter}\NormalTok{(n }\SpecialCharTok{\textgreater{}} \DecValTok{20000}\NormalTok{) }\SpecialCharTok{|\textgreater{}} 
  \FunctionTok{mutate}\NormalTok{(}\AttributeTok{genres =} \FunctionTok{reorder}\NormalTok{(genre\_categories, rating\_avg)) }\SpecialCharTok{|\textgreater{}}
  \FunctionTok{select}\NormalTok{(genres, rating\_avg, se, n)}

\FunctionTok{dim}\NormalTok{(genre\_ratins\_grp)}
\NormalTok{genre\_ratins\_grp\_sorted }\OtherTok{\textless{}{-}}\NormalTok{ genre\_ratins\_grp }\SpecialCharTok{|\textgreater{}} \FunctionTok{sort\_by.data.frame}\NormalTok{(}\SpecialCharTok{\textasciitilde{}}\NormalTok{ rating\_avg)}
\FunctionTok{print}\NormalTok{(genre\_ratins\_grp\_sorted)}

\CommentTok{\# Creating plot:}
\NormalTok{genre\_ratins\_grp }\SpecialCharTok{|\textgreater{}} 
  \FunctionTok{ggplot}\NormalTok{(}\FunctionTok{aes}\NormalTok{(}\AttributeTok{x =}\NormalTok{ genres, }\AttributeTok{y =}\NormalTok{ rating\_avg, }\AttributeTok{ymin =}\NormalTok{ rating\_avg }\SpecialCharTok{{-}} \DecValTok{2}\SpecialCharTok{*}\NormalTok{se, }\AttributeTok{ymax =}\NormalTok{ rating\_avg }\SpecialCharTok{+} \DecValTok{2}\SpecialCharTok{*}\NormalTok{se)) }\SpecialCharTok{+} 
  \FunctionTok{geom\_point}\NormalTok{() }\SpecialCharTok{+}
  \FunctionTok{geom\_errorbar}\NormalTok{() }\SpecialCharTok{+} 
  \FunctionTok{ggtitle}\NormalTok{(}\StringTok{"Average rating per Genre"}\NormalTok{) }\SpecialCharTok{+}
  \FunctionTok{ylab}\NormalTok{(}\StringTok{"Average rating"}\NormalTok{) }\SpecialCharTok{+}
  \FunctionTok{theme}\NormalTok{(}\AttributeTok{axis.text.x =} \FunctionTok{element\_text}\NormalTok{(}\AttributeTok{angle =} \DecValTok{90}\NormalTok{, }\AttributeTok{hjust =} \DecValTok{1}\NormalTok{))}
\end{Highlighting}
\end{Shaded}

Below are worst and best ratings categories:

\begin{Shaded}
\begin{Highlighting}[]
\FunctionTok{sprintf}\NormalTok{(}\StringTok{"The worst ratings are for the genre category: \%s"}\NormalTok{,}
\NormalTok{        genre\_ratins\_grp}\SpecialCharTok{$}\NormalTok{genres[}\FunctionTok{which.min}\NormalTok{(genre\_ratins\_grp}\SpecialCharTok{$}\NormalTok{genres)])}

\FunctionTok{sprintf}\NormalTok{(}\StringTok{"The best ratings are for the genre category: \%s"}\NormalTok{,}
\NormalTok{        genre\_ratins\_grp}\SpecialCharTok{$}\NormalTok{genres[}\FunctionTok{which.max}\NormalTok{(genre\_ratins\_grp}\SpecialCharTok{$}\NormalTok{genres)])}
\end{Highlighting}
\end{Shaded}

Another way of visualizing a genre effect is shown in the section
\href{https://www.kaggle.com/code/amirmotefaker/movie-recommendation-system-using-r-best/notebook\#Average-rating-for-each-genre}{Average
rating for each genre} of the article ``Movie Recommendation System
using R - BEST'' written by
\href{https://www.kaggle.com/amirmotefaker}{Amir
Moterfaker}\autocite{MRS-R-BEST}:

\begin{Shaded}
\begin{Highlighting}[]
\CommentTok{\# For better visibility, we reduce the data for plotting }
\CommentTok{\# while keeping the worst and best rating rows:}
\NormalTok{plot\_ind }\OtherTok{\textless{}{-}} \FunctionTok{odd}\NormalTok{(}\DecValTok{1}\SpecialCharTok{:}\FunctionTok{nrow}\NormalTok{(genre\_ratins\_grp))}
\NormalTok{plot\_dat }\OtherTok{\textless{}{-}}\NormalTok{ genre\_ratins\_grp\_sorted[plot\_ind,] }

\NormalTok{plot\_dat }\SpecialCharTok{|\textgreater{}}
  \FunctionTok{ggplot}\NormalTok{(}\FunctionTok{aes}\NormalTok{(}\AttributeTok{x =}\NormalTok{ rating\_avg, }\AttributeTok{y =}\NormalTok{ genres)) }\SpecialCharTok{+}
  \FunctionTok{ggtitle}\NormalTok{(}\StringTok{"Genre Average Rating"}\NormalTok{) }\SpecialCharTok{+}
  \FunctionTok{geom\_bar}\NormalTok{(}\AttributeTok{stat =} \StringTok{"identity"}\NormalTok{, }\AttributeTok{width =} \FloatTok{0.6}\NormalTok{, }\AttributeTok{fill =} \StringTok{"\#8888ff"}\NormalTok{) }\SpecialCharTok{+}
  \FunctionTok{xlab}\NormalTok{(}\StringTok{"Average ratings"}\NormalTok{) }\SpecialCharTok{+}
  \FunctionTok{ylab}\NormalTok{(}\StringTok{"Genres"}\NormalTok{) }\SpecialCharTok{+}
  \FunctionTok{scale\_x\_continuous}\NormalTok{(}\AttributeTok{labels =}\NormalTok{ comma, }\AttributeTok{limits =} \FunctionTok{c}\NormalTok{(}\FloatTok{0.0}\NormalTok{, }\FloatTok{5.0}\NormalTok{)) }\SpecialCharTok{+}
  \FunctionTok{theme\_economist}\NormalTok{() }\SpecialCharTok{+}
  \FunctionTok{theme}\NormalTok{(}\AttributeTok{plot.title =} \FunctionTok{element\_text}\NormalTok{(}\AttributeTok{vjust =} \FloatTok{3.5}\NormalTok{),}
        \AttributeTok{axis.title.x =} \FunctionTok{element\_text}\NormalTok{(}\AttributeTok{vjust =} \SpecialCharTok{{-}}\DecValTok{5}\NormalTok{, }\AttributeTok{face =} \StringTok{"bold"}\NormalTok{),}
        \AttributeTok{axis.title.y =} \FunctionTok{element\_text}\NormalTok{(}\AttributeTok{vjust =} \DecValTok{10}\NormalTok{, }\AttributeTok{face =} \StringTok{"bold"}\NormalTok{),}
        \AttributeTok{axis.text.x =} \FunctionTok{element\_text}\NormalTok{(}\AttributeTok{vjust =} \DecValTok{1}\NormalTok{, }\AttributeTok{hjust =} \DecValTok{1}\NormalTok{, }\AttributeTok{angle =} \DecValTok{0}\NormalTok{),}
        \AttributeTok{axis.text.y =} \FunctionTok{element\_text}\NormalTok{(}\AttributeTok{vjust =} \FloatTok{0.25}\NormalTok{, }\AttributeTok{hjust =} \DecValTok{1}\NormalTok{, }\AttributeTok{size =} \DecValTok{8}\NormalTok{),}
        \AttributeTok{plot.margin =} \FunctionTok{margin}\NormalTok{(}\FloatTok{0.7}\NormalTok{, }\FloatTok{0.5}\NormalTok{, }\DecValTok{1}\NormalTok{, }\FloatTok{1.2}\NormalTok{, }\StringTok{"cm"}\NormalTok{))}
\end{Highlighting}
\end{Shaded}

If we define \(g_{i,j}\) as the genre for user's \(i\) rating of movie
\(j\), we can use the following models to account for the \texttt{genre}
effect:

To account for \emph{genre effects} we will use the model suggested in
the
\href{https://rafalab.dfci.harvard.edu/dsbook-part-2/highdim/regularization.html\#exercises}{Section
23.7: Exercises} of the \emph{Chapter ``23 Regularization'' of the
Course Textbook}\autocite{IDS2_23-7}:

\[
Y_{i,j} = \mu + \alpha_i + \beta_j + g_{i,j} + \varepsilon_{i,j}
\]

where \(g_{i,j}\) is an \emph{aggregation function} which is explained
in detail in \emph{Section 22.3: ``Review of Aggregation Functions'' of
``Recommender Systems Handbook''} (\emph{Chapter 22: ``Aggregation of
Preferences in Recommender Systems''}, p.~712)
book\autocite{RRSK_RS_HB}.

In the formula above \(g_{i,j}\) denotes a \emph{genre effect} for
user's \(i\) rating of movie \(j\), so that:

\[
g_{i,j} = \sum_{k=1}^K x_{i,j}^k \gamma_k
\]

with \(x^k_{i,j} = 1\) if \(g_{i,j}\) includes genre \(k\), and
\(x^k_{i,j} = 0\) otherwise.

\[
Y_{i,j} = \mu + \alpha_i + \beta_j + g_{i,j} + f(d_{i,j})
\]

\[
 \sum_{i=1}^{n_i} \left(Y_{i,j} - \mu - \alpha_i\right)
\]

\subsection{Conclusion}\label{conclusion}

Hello Conclusion!

This is a great conclusion, isn't it?!!

\printbibliography

\end{document}
