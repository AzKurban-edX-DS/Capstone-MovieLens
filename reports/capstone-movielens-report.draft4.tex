% Options for packages loaded elsewhere
\PassOptionsToPackage{unicode}{hyperref}
\PassOptionsToPackage{hyphens}{url}
\PassOptionsToPackage{dvipsnames,svgnames,x11names}{xcolor}
%
\documentclass[
]{article}
\usepackage{amsmath,amssymb}
\usepackage{iftex}
\ifPDFTeX
  \usepackage[T1]{fontenc}
  \usepackage[utf8]{inputenc}
  \usepackage{textcomp} % provide euro and other symbols
\else % if luatex or xetex
  \usepackage{unicode-math} % this also loads fontspec
  \defaultfontfeatures{Scale=MatchLowercase}
  \defaultfontfeatures[\rmfamily]{Ligatures=TeX,Scale=1}
\fi
\usepackage{lmodern}
\ifPDFTeX\else
  % xetex/luatex font selection
\fi
% Use upquote if available, for straight quotes in verbatim environments
\IfFileExists{upquote.sty}{\usepackage{upquote}}{}
\IfFileExists{microtype.sty}{% use microtype if available
  \usepackage[]{microtype}
  \UseMicrotypeSet[protrusion]{basicmath} % disable protrusion for tt fonts
}{}
\makeatletter
\@ifundefined{KOMAClassName}{% if non-KOMA class
  \IfFileExists{parskip.sty}{%
    \usepackage{parskip}
  }{% else
    \setlength{\parindent}{0pt}
    \setlength{\parskip}{6pt plus 2pt minus 1pt}}
}{% if KOMA class
  \KOMAoptions{parskip=half}}
\makeatother
\usepackage{xcolor}
\usepackage[margin=1in]{geometry}
\usepackage{color}
\usepackage{fancyvrb}
\newcommand{\VerbBar}{|}
\newcommand{\VERB}{\Verb[commandchars=\\\{\}]}
\DefineVerbatimEnvironment{Highlighting}{Verbatim}{commandchars=\\\{\}}
% Add ',fontsize=\small' for more characters per line
\usepackage{framed}
\definecolor{shadecolor}{RGB}{248,248,248}
\newenvironment{Shaded}{\begin{snugshade}}{\end{snugshade}}
\newcommand{\AlertTok}[1]{\textcolor[rgb]{0.94,0.16,0.16}{#1}}
\newcommand{\AnnotationTok}[1]{\textcolor[rgb]{0.56,0.35,0.01}{\textbf{\textit{#1}}}}
\newcommand{\AttributeTok}[1]{\textcolor[rgb]{0.13,0.29,0.53}{#1}}
\newcommand{\BaseNTok}[1]{\textcolor[rgb]{0.00,0.00,0.81}{#1}}
\newcommand{\BuiltInTok}[1]{#1}
\newcommand{\CharTok}[1]{\textcolor[rgb]{0.31,0.60,0.02}{#1}}
\newcommand{\CommentTok}[1]{\textcolor[rgb]{0.56,0.35,0.01}{\textit{#1}}}
\newcommand{\CommentVarTok}[1]{\textcolor[rgb]{0.56,0.35,0.01}{\textbf{\textit{#1}}}}
\newcommand{\ConstantTok}[1]{\textcolor[rgb]{0.56,0.35,0.01}{#1}}
\newcommand{\ControlFlowTok}[1]{\textcolor[rgb]{0.13,0.29,0.53}{\textbf{#1}}}
\newcommand{\DataTypeTok}[1]{\textcolor[rgb]{0.13,0.29,0.53}{#1}}
\newcommand{\DecValTok}[1]{\textcolor[rgb]{0.00,0.00,0.81}{#1}}
\newcommand{\DocumentationTok}[1]{\textcolor[rgb]{0.56,0.35,0.01}{\textbf{\textit{#1}}}}
\newcommand{\ErrorTok}[1]{\textcolor[rgb]{0.64,0.00,0.00}{\textbf{#1}}}
\newcommand{\ExtensionTok}[1]{#1}
\newcommand{\FloatTok}[1]{\textcolor[rgb]{0.00,0.00,0.81}{#1}}
\newcommand{\FunctionTok}[1]{\textcolor[rgb]{0.13,0.29,0.53}{\textbf{#1}}}
\newcommand{\ImportTok}[1]{#1}
\newcommand{\InformationTok}[1]{\textcolor[rgb]{0.56,0.35,0.01}{\textbf{\textit{#1}}}}
\newcommand{\KeywordTok}[1]{\textcolor[rgb]{0.13,0.29,0.53}{\textbf{#1}}}
\newcommand{\NormalTok}[1]{#1}
\newcommand{\OperatorTok}[1]{\textcolor[rgb]{0.81,0.36,0.00}{\textbf{#1}}}
\newcommand{\OtherTok}[1]{\textcolor[rgb]{0.56,0.35,0.01}{#1}}
\newcommand{\PreprocessorTok}[1]{\textcolor[rgb]{0.56,0.35,0.01}{\textit{#1}}}
\newcommand{\RegionMarkerTok}[1]{#1}
\newcommand{\SpecialCharTok}[1]{\textcolor[rgb]{0.81,0.36,0.00}{\textbf{#1}}}
\newcommand{\SpecialStringTok}[1]{\textcolor[rgb]{0.31,0.60,0.02}{#1}}
\newcommand{\StringTok}[1]{\textcolor[rgb]{0.31,0.60,0.02}{#1}}
\newcommand{\VariableTok}[1]{\textcolor[rgb]{0.00,0.00,0.00}{#1}}
\newcommand{\VerbatimStringTok}[1]{\textcolor[rgb]{0.31,0.60,0.02}{#1}}
\newcommand{\WarningTok}[1]{\textcolor[rgb]{0.56,0.35,0.01}{\textbf{\textit{#1}}}}
\usepackage{graphicx}
\makeatletter
\def\maxwidth{\ifdim\Gin@nat@width>\linewidth\linewidth\else\Gin@nat@width\fi}
\def\maxheight{\ifdim\Gin@nat@height>\textheight\textheight\else\Gin@nat@height\fi}
\makeatother
% Scale images if necessary, so that they will not overflow the page
% margins by default, and it is still possible to overwrite the defaults
% using explicit options in \includegraphics[width, height, ...]{}
\setkeys{Gin}{width=\maxwidth,height=\maxheight,keepaspectratio}
% Set default figure placement to htbp
\makeatletter
\def\fps@figure{htbp}
\makeatother
\setlength{\emergencystretch}{3em} % prevent overfull lines
\providecommand{\tightlist}{%
  \setlength{\itemsep}{0pt}\setlength{\parskip}{0pt}}
\setcounter{secnumdepth}{-\maxdimen} % remove section numbering
\usepackage{awesomebox}
\usepackage{hyperref}
\usepackage[natbib=true, style=numeric, backref=true, sorting=none]{biblatex}
\hypersetup{backref, pdfpagemode=Normal, colorlinks=true, implicit=false}
\ifLuaTeX
  \usepackage{selnolig}  % disable illegal ligatures
\fi
\usepackage[]{biblatex}
\addbibresource{references.bib}
\usepackage{bookmark}
\IfFileExists{xurl.sty}{\usepackage{xurl}}{} % add URL line breaks if available
\urlstyle{same}
\hypersetup{
  pdftitle={Capstone Movielens Report},
  pdfauthor={Azamat Kurbanaev},
  colorlinks=true,
  linkcolor={red},
  filecolor={Maroon},
  citecolor={blue},
  urlcolor={Blue},
  pdfcreator={LaTeX via pandoc}}

\title{Capstone Movielens Report}
\author{Azamat Kurbanaev}
\date{2025-05-05}

\begin{document}
\maketitle

{
\hypersetup{linkcolor=}
\setcounter{tocdepth}{2}
\tableofcontents
}
\newenvironment{infobox}[1]
  {
  \begin{itemize}
  \renewcommand{\labelitemi}{
    \raisebox{-.7\height}[0pt][0pt]{
      {\setkeys{Gin}{width=3em,keepaspectratio}
        \includegraphics{images/#1}}
    }
  }
  \setlength{\fboxsep}{1em}
  \begin{blackbox}
  \item
  }
  {
  \end{blackbox}
  \end{itemize}
  }

\subsection{Introduction / Overview / Executive
Summary}\label{introduction-overview-executive-summary}

The goal of the project is to build a Recommendation System using a
\href{http://grouplens.org/datasets/movielens/10m/}{10M version of the
MovieLens dataset}. Following the
\href{https://archive.nytimes.com/bits.blogs.nytimes.com/2009/09/21/netflix-awards-1-million-prize-and-starts-a-new-contest/index.html}{Netflix
Grand Prize Contest} requirements, we will evaluate the \emph{Root Mean
Square Error} (\emph{RMSE}) score, which, as shown in
\href{https://rafalab.dfci.harvard.edu/dsbook-part-2/highdim/regularization.html\#sec-netflix-loss-function}{Section
23.2 Loss function} of the \emph{Course Textbook}, is defined as: \[
\mbox{RMSE} = \sqrt{\frac{1}{N} \sum_{i,j}^{N} (y_{i,j} - \hat{y}_{i,j})^2}
\]

with \(N\) being the number of user/movie combinations for which we make
predictions and the sum occurring over all these
combinations\autocite{IDS2_23-2}.

Our goal is to achieve a value of less than 0.86490 (compare with the
\emph{Netflix Grand Prize} requirement: of at least
0.8563\autocite{BigChaosSln}).

\subsubsection{Datasets Overview}\label{datasets-overview}

To start with we have to generate two datasets derived from the
\emph{MovieLens} one mentioned above:

\begin{itemize}
\tightlist
\item
  \textbf{edx:} we use it to develop and train our algorithms;
\item
  \textbf{final\_holdout\_test:} according to the course requirements,
  we use it exclusively to evaluate the \emph{\textbf{RMSE}} of our
  final algorithm.
\end{itemize}

For this purpose the following package has been developed by the author
of this report: \texttt{edx.capstone.movielens.data}. The source code of
the package is available
\href{https://github.com/AzKurban-edX-DS/edx.capstone.movielens.data}{on
GitHub}\autocite{edx_capstone_movielens_data}.

Let's install the development version of this package from the GitHub
repository and attach the correspondent library to the global
environment:

\begin{Shaded}
\begin{Highlighting}[]
\ControlFlowTok{if}\NormalTok{(}\SpecialCharTok{!}\FunctionTok{require}\NormalTok{(edx.capstone.movielens.data)) pak}\SpecialCharTok{::}\FunctionTok{pak}\NormalTok{(}\StringTok{"AzKurban{-}edX{-}DS/edx.capstone.movielens.data"}\NormalTok{)}

\FunctionTok{library}\NormalTok{(edx.capstone.movielens.data)}
\NormalTok{edx }\OtherTok{\textless{}{-}}\NormalTok{ edx.capstone.movielens.data}\SpecialCharTok{::}\NormalTok{edx}
\NormalTok{final\_holdout\_test }\OtherTok{\textless{}{-}}\NormalTok{ edx.capstone.movielens.data}\SpecialCharTok{::}\NormalTok{final\_holdout\_test}
\end{Highlighting}
\end{Shaded}

Now, we have the datasets listed above:

\begin{Shaded}
\begin{Highlighting}[]
\FunctionTok{summary}\NormalTok{(edx)}
\end{Highlighting}
\end{Shaded}

\begin{verbatim}
##      userId         movieId          rating        timestamp            title              genres         
##  Min.   :    1   Min.   :    1   Min.   :0.500   Min.   :7.897e+08   Length:9000055     Length:9000055    
##  1st Qu.:18124   1st Qu.:  648   1st Qu.:3.000   1st Qu.:9.468e+08   Class :character   Class :character  
##  Median :35738   Median : 1834   Median :4.000   Median :1.035e+09   Mode  :character   Mode  :character  
##  Mean   :35870   Mean   : 4122   Mean   :3.512   Mean   :1.033e+09                                        
##  3rd Qu.:53607   3rd Qu.: 3626   3rd Qu.:4.000   3rd Qu.:1.127e+09                                        
##  Max.   :71567   Max.   :65133   Max.   :5.000   Max.   :1.231e+09
\end{verbatim}

\begin{Shaded}
\begin{Highlighting}[]
\FunctionTok{summary}\NormalTok{(final\_holdout\_test)}
\end{Highlighting}
\end{Shaded}

\begin{verbatim}
##      userId         movieId          rating        timestamp            title              genres         
##  Min.   :    1   Min.   :    1   Min.   :0.500   Min.   :7.897e+08   Length:999999      Length:999999     
##  1st Qu.:18096   1st Qu.:  648   1st Qu.:3.000   1st Qu.:9.467e+08   Class :character   Class :character  
##  Median :35768   Median : 1827   Median :4.000   Median :1.035e+09   Mode  :character   Mode  :character  
##  Mean   :35870   Mean   : 4108   Mean   :3.512   Mean   :1.033e+09                                        
##  3rd Qu.:53621   3rd Qu.: 3624   3rd Qu.:4.000   3rd Qu.:1.127e+09                                        
##  Max.   :71567   Max.   :65133   Max.   :5.000   Max.   :1.231e+09
\end{verbatim}

\paragraph{\texorpdfstring{\texttt{edx}
Dataset}{edx Dataset}}\label{edx-dataset}

\hfill\break
Let's look into the details of the \texttt{edx} dataset:

\begin{Shaded}
\begin{Highlighting}[]
\FunctionTok{str}\NormalTok{(edx)}
\end{Highlighting}
\end{Shaded}

\begin{verbatim}
## 'data.frame':    9000055 obs. of  6 variables:
##  $ userId   : int  1 1 1 1 1 1 1 1 1 1 ...
##  $ movieId  : int  122 185 292 316 329 355 356 362 364 370 ...
##  $ rating   : num  5 5 5 5 5 5 5 5 5 5 ...
##  $ timestamp: int  838985046 838983525 838983421 838983392 838983392 838984474 838983653 838984885 838983707 838984596 ...
##  $ title    : chr  "Boomerang (1992)" "Net, The (1995)" "Outbreak (1995)" "Stargate (1994)" ...
##  $ genres   : chr  "Comedy|Romance" "Action|Crime|Thriller" "Action|Drama|Sci-Fi|Thriller" "Action|Adventure|Sci-Fi" ...
\end{verbatim}

Note that we have 9000055 rows and six columns in there:

\begin{Shaded}
\begin{Highlighting}[]
\NormalTok{dim\_edx }\OtherTok{\textless{}{-}} \FunctionTok{dim}\NormalTok{(edx)}
\FunctionTok{print}\NormalTok{(dim\_edx)}
\end{Highlighting}
\end{Shaded}

\begin{verbatim}
## [1] 9000055       6
\end{verbatim}

First, let's note that we have 10677 different movies:

\begin{Shaded}
\begin{Highlighting}[]
\NormalTok{n\_movies }\OtherTok{\textless{}{-}} \FunctionTok{n\_distinct}\NormalTok{(edx}\SpecialCharTok{$}\NormalTok{movieId)}
\FunctionTok{print}\NormalTok{(n\_movies)}
\end{Highlighting}
\end{Shaded}

\begin{verbatim}
## [1] 10677
\end{verbatim}

and 69878 different users in the dataset:

\begin{Shaded}
\begin{Highlighting}[]
\NormalTok{n\_users }\OtherTok{\textless{}{-}} \FunctionTok{n\_distinct}\NormalTok{(edx}\SpecialCharTok{$}\NormalTok{userId)}
\FunctionTok{print}\NormalTok{(n\_users)}
\end{Highlighting}
\end{Shaded}

\begin{verbatim}
## [1] 69878
\end{verbatim}

Now, note the expressions below which confirm the fact explained in
\href{https://rafalab.dfci.harvard.edu/dsbook-part-2/highdim/regularization.html\#movielens-data}{Section
\emph{23.1.1 Movielens data}} of the \emph{Course
Textbook}\autocite{IDS2} that not every user rated every movie:

\begin{Shaded}
\begin{Highlighting}[]
\NormalTok{max\_possible\_ratings }\OtherTok{\textless{}{-}}\NormalTok{ n\_movies}\SpecialCharTok{*}\NormalTok{n\_users}
\FunctionTok{sprintf}\NormalTok{(}\StringTok{"Maximum possible ratings: \%s"}\NormalTok{, max\_possible\_ratings)}
\end{Highlighting}
\end{Shaded}

\begin{verbatim}
## [1] "Maximum possible ratings: 746087406"
\end{verbatim}

\begin{Shaded}
\begin{Highlighting}[]
\FunctionTok{sprintf}\NormalTok{(}\StringTok{"Rows in \textasciigrave{}edx\textasciigrave{} dataset: \%s"}\NormalTok{, dim\_edx[}\DecValTok{1}\NormalTok{])}
\end{Highlighting}
\end{Shaded}

\begin{verbatim}
## [1] "Rows in `edx` dataset: 9000055"
\end{verbatim}

\begin{Shaded}
\begin{Highlighting}[]
\FunctionTok{sprintf}\NormalTok{(}\StringTok{"Not every movie was rated: \%s"}\NormalTok{, max\_possible\_ratings }\SpecialCharTok{\textgreater{}}\NormalTok{ dim\_edx[}\DecValTok{1}\NormalTok{])}
\end{Highlighting}
\end{Shaded}

\begin{verbatim}
## [1] "Not every movie was rated: TRUE"
\end{verbatim}

As also explained in that section, we can think of these data as a very
large matrix, with users on the rows and movies on the columns, with
many empty cells. Therefore, we can think of a recommendation system as
filling in the \texttt{NA}s in the dataset for the movies that some or
all the users do not rate. A sample from the \texttt{edx} data below
illustrates this idea\autocite{IDS2_23-1-1}:

\begin{Shaded}
\begin{Highlighting}[]
\NormalTok{keep }\OtherTok{\textless{}{-}}\NormalTok{ edx }\SpecialCharTok{|\textgreater{}} 
\NormalTok{  dplyr}\SpecialCharTok{::}\FunctionTok{count}\NormalTok{(movieId) }\SpecialCharTok{|\textgreater{}} 
  \FunctionTok{top\_n}\NormalTok{(}\DecValTok{4}\NormalTok{, n) }\SpecialCharTok{|\textgreater{}} 
  \FunctionTok{pull}\NormalTok{(movieId)}

\NormalTok{tab }\OtherTok{\textless{}{-}}\NormalTok{ edx }\SpecialCharTok{|\textgreater{}} 
  \FunctionTok{filter}\NormalTok{(movieId }\SpecialCharTok{\%in\%}\NormalTok{ keep) }\SpecialCharTok{|\textgreater{}} 
  \FunctionTok{filter}\NormalTok{(userId }\SpecialCharTok{\%in\%} \FunctionTok{c}\NormalTok{(}\DecValTok{13}\SpecialCharTok{:}\DecValTok{20}\NormalTok{)) }\SpecialCharTok{|\textgreater{}} 
  \FunctionTok{select}\NormalTok{(userId, title, rating) }\SpecialCharTok{|\textgreater{}} 
  \FunctionTok{mutate}\NormalTok{(}\AttributeTok{title =} \FunctionTok{str\_remove}\NormalTok{(title, }\StringTok{", The"}\NormalTok{),}
         \AttributeTok{title =} \FunctionTok{str\_remove}\NormalTok{(title, }\StringTok{":.*"}\NormalTok{)) }\SpecialCharTok{|\textgreater{}}
  \FunctionTok{pivot\_wider}\NormalTok{(}\AttributeTok{names\_from =} \StringTok{"title"}\NormalTok{, }\AttributeTok{values\_from =} \StringTok{"rating"}\NormalTok{)}

\FunctionTok{print}\NormalTok{(tab)}
\end{Highlighting}
\end{Shaded}

\begin{verbatim}
## # A tibble: 5 x 5
##   userId `Pulp Fiction (1994)` `Jurassic Park (1993)` `Silence of the Lambs (1991)` `Forrest Gump (1994)`
##    <int>                 <dbl>                  <dbl>                         <dbl>                 <dbl>
## 1     13                     4                     NA                            NA                    NA
## 2     16                    NA                      3                            NA                    NA
## 3     17                    NA                     NA                             5                    NA
## 4     18                     5                      3                             5                    NA
## 5     19                    NA                      1                            NA                     4
\end{verbatim}

The following plot of the matrix for a random sample of 100 movies and
100 users with yellow indicating a user/movie combination for which we
have a rating shows how \emph{sparse} the matrix is:

\begin{Shaded}
\begin{Highlighting}[]
\NormalTok{users }\OtherTok{\textless{}{-}} \FunctionTok{sample}\NormalTok{(}\FunctionTok{unique}\NormalTok{(edx}\SpecialCharTok{$}\NormalTok{userId), }\DecValTok{100}\NormalTok{)}

\NormalTok{rafalib}\SpecialCharTok{::}\FunctionTok{mypar}\NormalTok{()}
\NormalTok{edx}\SpecialCharTok{|\textgreater{}} 
  \FunctionTok{filter}\NormalTok{(userId }\SpecialCharTok{\%in\%}\NormalTok{ users) }\SpecialCharTok{|\textgreater{}} 
  \FunctionTok{select}\NormalTok{(userId, movieId, rating) }\SpecialCharTok{|\textgreater{}}
  \FunctionTok{mutate}\NormalTok{(}\AttributeTok{rating =} \DecValTok{1}\NormalTok{) }\SpecialCharTok{|\textgreater{}}
  \FunctionTok{pivot\_wider}\NormalTok{(}\AttributeTok{names\_from =}\NormalTok{ movieId, }\AttributeTok{values\_from =}\NormalTok{ rating) }\SpecialCharTok{|\textgreater{}} 
\NormalTok{  (\textbackslash{}(mat) mat[, }\FunctionTok{sample}\NormalTok{(}\FunctionTok{ncol}\NormalTok{(mat), }\DecValTok{100}\NormalTok{)])() }\SpecialCharTok{|\textgreater{}}
  \FunctionTok{as.matrix}\NormalTok{() }\SpecialCharTok{|\textgreater{}} 
  \FunctionTok{t}\NormalTok{() }\SpecialCharTok{|\textgreater{}}
  \FunctionTok{image}\NormalTok{(}\DecValTok{1}\SpecialCharTok{:}\DecValTok{100}\NormalTok{, }\DecValTok{1}\SpecialCharTok{:}\DecValTok{100}\NormalTok{, }\AttributeTok{z =}\NormalTok{ \_ , }\AttributeTok{xlab =} \StringTok{"Movies"}\NormalTok{, }\AttributeTok{ylab =} \StringTok{"Users"}\NormalTok{)}
\end{Highlighting}
\end{Shaded}

\includegraphics[width=0.4\linewidth]{capstone-movielens-report.draft4_files/figure-latex/sparsity-of-movie-recs-1}

Further observations highlighted there that, as we can see from the
distributions the author presented, some movies get rated more than
others, and some users are more active than others in rating movies:

\begin{Shaded}
\begin{Highlighting}[]
\NormalTok{p1 }\OtherTok{\textless{}{-}}\NormalTok{ edx }\SpecialCharTok{|\textgreater{}} 
  \FunctionTok{count}\NormalTok{(movieId) }\SpecialCharTok{|\textgreater{}} 
  \FunctionTok{ggplot}\NormalTok{(}\FunctionTok{aes}\NormalTok{(n)) }\SpecialCharTok{+} 
  \FunctionTok{geom\_histogram}\NormalTok{(}\AttributeTok{bins =} \DecValTok{30}\NormalTok{, }\AttributeTok{color =} \StringTok{"black"}\NormalTok{) }\SpecialCharTok{+} 
  \FunctionTok{scale\_x\_log10}\NormalTok{() }\SpecialCharTok{+} 
  \FunctionTok{ggtitle}\NormalTok{(}\StringTok{"Movies"}\NormalTok{)}

\NormalTok{p2 }\OtherTok{\textless{}{-}}\NormalTok{ edx }\SpecialCharTok{|\textgreater{}} 
  \FunctionTok{count}\NormalTok{(userId) }\SpecialCharTok{|\textgreater{}} 
  \FunctionTok{ggplot}\NormalTok{(}\FunctionTok{aes}\NormalTok{(n)) }\SpecialCharTok{+} 
  \FunctionTok{geom\_histogram}\NormalTok{(}\AttributeTok{bins =} \DecValTok{30}\NormalTok{, }\AttributeTok{color =} \StringTok{"black"}\NormalTok{) }\SpecialCharTok{+} 
  \FunctionTok{scale\_x\_log10}\NormalTok{() }\SpecialCharTok{+} 
  \FunctionTok{ggtitle}\NormalTok{(}\StringTok{"Users"}\NormalTok{)}

\NormalTok{gridExtra}\SpecialCharTok{::}\FunctionTok{grid.arrange}\NormalTok{(p2, p1, }\AttributeTok{ncol =} \DecValTok{2}\NormalTok{)}
\end{Highlighting}
\end{Shaded}

\includegraphics{capstone-movielens-report.draft4_files/figure-latex/movie-id-and-user-hists-1.pdf}

Finally, we can see that no movies have a rating of 0. Movies are rated
from 0.5 to 5.0 in 0.5 increments:

\begin{Shaded}
\begin{Highlighting}[]
\CommentTok{\#library(dplyr)}
\NormalTok{s }\OtherTok{\textless{}{-}}\NormalTok{ edx }\SpecialCharTok{|\textgreater{}} \FunctionTok{group\_by}\NormalTok{(rating) }\SpecialCharTok{|\textgreater{}}
  \FunctionTok{summarise}\NormalTok{(}\AttributeTok{n =} \FunctionTok{n}\NormalTok{())}
\FunctionTok{print}\NormalTok{(s)}
\end{Highlighting}
\end{Shaded}

\begin{verbatim}
## # A tibble: 10 x 2
##    rating       n
##     <dbl>   <int>
##  1    0.5   85374
##  2    1    345679
##  3    1.5  106426
##  4    2    711422
##  5    2.5  333010
##  6    3   2121240
##  7    3.5  791624
##  8    4   2588430
##  9    4.5  526736
## 10    5   1390114
\end{verbatim}

Further analysis of the \texttt{edx} dataset have been also inspired by
the article mentioned above\autocite{MRS-R-BEST}, from which the code
and explanatory notes below were cited.

\subparagraph{\texorpdfstring{Rating distribution
plot\autocite{MRS-R-BEST}}{Rating distribution plot{[}@MRS-R-BEST{]}}}\label{rating-distribution-plotmrs-r-best}

\hfill\break
The code below demonstrates another way of visualizing the rating
distribution:

\begin{Shaded}
\begin{Highlighting}[]
\NormalTok{edx }\SpecialCharTok{|\textgreater{}}
  \FunctionTok{group\_by}\NormalTok{(rating) }\SpecialCharTok{|\textgreater{}}
  \FunctionTok{summarize}\NormalTok{(}\AttributeTok{count =} \FunctionTok{n}\NormalTok{()) }\SpecialCharTok{|\textgreater{}}
  \FunctionTok{ggplot}\NormalTok{(}\FunctionTok{aes}\NormalTok{(}\AttributeTok{x =}\NormalTok{ rating, }\AttributeTok{y =}\NormalTok{ count)) }\SpecialCharTok{+}
  \FunctionTok{geom\_bar}\NormalTok{(}\AttributeTok{stat =} \StringTok{"identity"}\NormalTok{, }\AttributeTok{fill =} \StringTok{"\#8888ff"}\NormalTok{) }\SpecialCharTok{+}
  \FunctionTok{ggtitle}\NormalTok{(}\StringTok{"Rating Distribution"}\NormalTok{) }\SpecialCharTok{+}
  \FunctionTok{xlab}\NormalTok{(}\StringTok{"Rating"}\NormalTok{) }\SpecialCharTok{+}
  \FunctionTok{ylab}\NormalTok{(}\StringTok{"Occurrences Count"}\NormalTok{) }\SpecialCharTok{+}
  \FunctionTok{scale\_y\_continuous}\NormalTok{(}\AttributeTok{labels =}\NormalTok{ comma) }\SpecialCharTok{+}
  \FunctionTok{scale\_x\_continuous}\NormalTok{(}\AttributeTok{n.breaks =} \DecValTok{10}\NormalTok{) }\SpecialCharTok{+}
  \FunctionTok{theme\_economist}\NormalTok{() }\SpecialCharTok{+}
  \FunctionTok{theme}\NormalTok{(}\AttributeTok{axis.title.x =} \FunctionTok{element\_text}\NormalTok{(}\AttributeTok{vjust =} \SpecialCharTok{{-}}\DecValTok{5}\NormalTok{, }\AttributeTok{face =} \StringTok{"bold"}\NormalTok{), }
        \AttributeTok{axis.title.y =} \FunctionTok{element\_text}\NormalTok{(}\AttributeTok{vjust =} \DecValTok{10}\NormalTok{, }\AttributeTok{face =} \StringTok{"bold"}\NormalTok{), }
        \AttributeTok{plot.margin =} \FunctionTok{margin}\NormalTok{(}\FloatTok{0.7}\NormalTok{, }\FloatTok{0.5}\NormalTok{, }\DecValTok{1}\NormalTok{, }\FloatTok{1.2}\NormalTok{, }\StringTok{"cm"}\NormalTok{))}
\end{Highlighting}
\end{Shaded}

\includegraphics{capstone-movielens-report.draft4_files/figure-latex/unnamed-chunk-15-1.pdf}

This graph is another confirmation of what we found out above: rounded
ratings occur more often than half-stared ones. The upward trend
previously discussed is now perfectly clear, although it seems to top
right between the 3 and 4-star ratings lowering the occurrences count
afterward. That might be due to users being more hesitant to rate with
the highest mark for whichever reasons they might
hold\autocite{MRS-R-BEST}.

\subparagraph{Ratings per movie}\label{ratings-per-movie}

\hfill\break

Movie popularity count\autocite{MRS-R-BEST}

\hfill\break

\begin{Shaded}
\begin{Highlighting}[]
\FunctionTok{print}\NormalTok{(edx }\SpecialCharTok{|\textgreater{}} 
  \FunctionTok{group\_by}\NormalTok{(movieId) }\SpecialCharTok{|\textgreater{}} 
  \FunctionTok{summarize}\NormalTok{(}\AttributeTok{count =} \FunctionTok{n}\NormalTok{()) }\SpecialCharTok{|\textgreater{}}
  \FunctionTok{slice\_head}\NormalTok{(}\AttributeTok{n =} \DecValTok{10}\NormalTok{)}
\NormalTok{)}
\end{Highlighting}
\end{Shaded}

\begin{verbatim}
## # A tibble: 10 x 2
##    movieId count
##      <int> <int>
##  1       1 23790
##  2       2 10779
##  3       3  7028
##  4       4  1577
##  5       5  6400
##  6       6 12346
##  7       7  7259
##  8       8   821
##  9       9  2278
## 10      10 15187
\end{verbatim}

\begin{Shaded}
\begin{Highlighting}[]
\FunctionTok{summary}\NormalTok{(edx }\SpecialCharTok{|\textgreater{}} \FunctionTok{group\_by}\NormalTok{(movieId) }\SpecialCharTok{|\textgreater{}} \FunctionTok{summarize}\NormalTok{(}\AttributeTok{count =} \FunctionTok{n}\NormalTok{()) }\SpecialCharTok{|\textgreater{}} \FunctionTok{select}\NormalTok{(count))}
\end{Highlighting}
\end{Shaded}

\begin{verbatim}
##      count        
##  Min.   :    1.0  
##  1st Qu.:   30.0  
##  Median :  122.0  
##  Mean   :  842.9  
##  3rd Qu.:  565.0  
##  Max.   :31362.0
\end{verbatim}

Ratings per movie plot\autocite{MRS-R-BEST}

\hfill\break

\begin{Shaded}
\begin{Highlighting}[]
\NormalTok{edx }\SpecialCharTok{|\textgreater{}}
  \FunctionTok{group\_by}\NormalTok{(movieId) }\SpecialCharTok{|\textgreater{}}
  \FunctionTok{summarize}\NormalTok{(}\AttributeTok{count =} \FunctionTok{n}\NormalTok{()) }\SpecialCharTok{|\textgreater{}}
  \FunctionTok{ggplot}\NormalTok{(}\FunctionTok{aes}\NormalTok{(}\AttributeTok{x =}\NormalTok{ movieId, }\AttributeTok{y =}\NormalTok{ count)) }\SpecialCharTok{+}
  \FunctionTok{geom\_point}\NormalTok{(}\AttributeTok{alpha =} \FloatTok{0.2}\NormalTok{, }\AttributeTok{color =} \StringTok{"\#4020dd"}\NormalTok{) }\SpecialCharTok{+}
  \FunctionTok{geom\_smooth}\NormalTok{(}\AttributeTok{color =} \StringTok{"red"}\NormalTok{) }\SpecialCharTok{+}
  \FunctionTok{ggtitle}\NormalTok{(}\StringTok{"Ratings per movie"}\NormalTok{) }\SpecialCharTok{+}
  \FunctionTok{xlab}\NormalTok{(}\StringTok{"Movies"}\NormalTok{) }\SpecialCharTok{+}
  \FunctionTok{ylab}\NormalTok{(}\StringTok{"Number of ratings"}\NormalTok{) }\SpecialCharTok{+}
  \FunctionTok{scale\_y\_continuous}\NormalTok{(}\AttributeTok{labels =}\NormalTok{ comma) }\SpecialCharTok{+}
  \FunctionTok{scale\_x\_continuous}\NormalTok{(}\AttributeTok{n.breaks =} \DecValTok{10}\NormalTok{) }\SpecialCharTok{+}
  \FunctionTok{theme\_economist}\NormalTok{() }\SpecialCharTok{+}
  \FunctionTok{theme}\NormalTok{(}\AttributeTok{axis.title.x =} \FunctionTok{element\_text}\NormalTok{(}\AttributeTok{vjust =} \SpecialCharTok{{-}}\DecValTok{5}\NormalTok{, }\AttributeTok{face =} \StringTok{"bold"}\NormalTok{), }
        \AttributeTok{axis.title.y =} \FunctionTok{element\_text}\NormalTok{(}\AttributeTok{vjust =} \DecValTok{10}\NormalTok{, }\AttributeTok{face =} \StringTok{"bold"}\NormalTok{), }
        \AttributeTok{plot.margin =} \FunctionTok{margin}\NormalTok{(}\FloatTok{0.7}\NormalTok{, }\FloatTok{0.5}\NormalTok{, }\DecValTok{1}\NormalTok{, }\FloatTok{1.2}\NormalTok{, }\StringTok{"cm"}\NormalTok{))}
\end{Highlighting}
\end{Shaded}

\begin{verbatim}
## `geom_smooth()` using method = 'gam' and formula = 'y ~ s(x, bs = "cs")'
\end{verbatim}

\includegraphics{capstone-movielens-report.draft4_files/figure-latex/unnamed-chunk-18-1.pdf}

Movies' rating histogram\autocite{MRS-R-BEST}

\hfill\break

\begin{Shaded}
\begin{Highlighting}[]
\NormalTok{edx }\SpecialCharTok{|\textgreater{}}
  \FunctionTok{group\_by}\NormalTok{(movieId) }\SpecialCharTok{|\textgreater{}}
  \FunctionTok{summarize}\NormalTok{(}\AttributeTok{count =} \FunctionTok{n}\NormalTok{()) }\SpecialCharTok{|\textgreater{}}
  \FunctionTok{ggplot}\NormalTok{(}\FunctionTok{aes}\NormalTok{(}\AttributeTok{x =}\NormalTok{ count)) }\SpecialCharTok{+}
  \FunctionTok{geom\_histogram}\NormalTok{(}\AttributeTok{fill =} \StringTok{"\#8888ff"}\NormalTok{, }\AttributeTok{color =} \StringTok{"\#4020dd"}\NormalTok{) }\SpecialCharTok{+}
  \FunctionTok{ggtitle}\NormalTok{(}\StringTok{"Movies\textquotesingle{} rating histogram"}\NormalTok{) }\SpecialCharTok{+}
  \FunctionTok{xlab}\NormalTok{(}\StringTok{"Rating count"}\NormalTok{) }\SpecialCharTok{+}
  \FunctionTok{ylab}\NormalTok{(}\StringTok{"Number of movies"}\NormalTok{) }\SpecialCharTok{+}
  \FunctionTok{scale\_y\_continuous}\NormalTok{(}\AttributeTok{labels =}\NormalTok{ comma) }\SpecialCharTok{+}
  \FunctionTok{scale\_x\_log10}\NormalTok{(}\AttributeTok{n.breaks =} \DecValTok{10}\NormalTok{) }\SpecialCharTok{+}
  \FunctionTok{theme\_economist}\NormalTok{() }\SpecialCharTok{+}
  \FunctionTok{theme}\NormalTok{(}\AttributeTok{axis.title.x =} \FunctionTok{element\_text}\NormalTok{(}\AttributeTok{vjust =} \SpecialCharTok{{-}}\DecValTok{5}\NormalTok{, }\AttributeTok{face =} \StringTok{"bold"}\NormalTok{), }
        \AttributeTok{axis.title.y =} \FunctionTok{element\_text}\NormalTok{(}\AttributeTok{vjust =} \DecValTok{10}\NormalTok{, }\AttributeTok{face =} \StringTok{"bold"}\NormalTok{), }
        \AttributeTok{plot.margin =} \FunctionTok{margin}\NormalTok{(}\FloatTok{0.7}\NormalTok{, }\FloatTok{0.5}\NormalTok{, }\DecValTok{1}\NormalTok{, }\FloatTok{1.2}\NormalTok{, }\StringTok{"cm"}\NormalTok{))}
\end{Highlighting}
\end{Shaded}

\begin{verbatim}
## `stat_bin()` using `bins = 30`. Pick better value with `binwidth`.
\end{verbatim}

\includegraphics{capstone-movielens-report.draft4_files/figure-latex/unnamed-chunk-19-1.pdf}

\subparagraph{\texorpdfstring{Ratings per
user\autocite{MRS-R-BEST}}{Ratings per user{[}@MRS-R-BEST{]}}}\label{ratings-per-usermrs-r-best}

\hfill\break

User rating count (activity measure)

\hfill\break

\begin{Shaded}
\begin{Highlighting}[]
\FunctionTok{print}\NormalTok{(edx }\SpecialCharTok{|\textgreater{}} 
  \FunctionTok{group\_by}\NormalTok{(userId) }\SpecialCharTok{|\textgreater{}} 
  \FunctionTok{summarize}\NormalTok{(}\AttributeTok{count =} \FunctionTok{n}\NormalTok{()) }\SpecialCharTok{|\textgreater{}}
  \FunctionTok{slice\_head}\NormalTok{(}\AttributeTok{n =} \DecValTok{10}\NormalTok{)}
\NormalTok{)}
\end{Highlighting}
\end{Shaded}

\begin{verbatim}
## # A tibble: 10 x 2
##    userId count
##     <int> <int>
##  1      1    19
##  2      2    17
##  3      3    31
##  4      4    35
##  5      5    74
##  6      6    39
##  7      7    96
##  8      8   727
##  9      9    21
## 10     10   112
\end{verbatim}

User rating summary

\hfill\break

\begin{Shaded}
\begin{Highlighting}[]
\FunctionTok{summary}\NormalTok{(edx }\SpecialCharTok{|\textgreater{}} \FunctionTok{group\_by}\NormalTok{(userId) }\SpecialCharTok{|\textgreater{}} \FunctionTok{summarize}\NormalTok{(}\AttributeTok{count =} \FunctionTok{n}\NormalTok{()) }\SpecialCharTok{|\textgreater{}} \FunctionTok{select}\NormalTok{(count))}
\end{Highlighting}
\end{Shaded}

\begin{verbatim}
##      count       
##  Min.   :  10.0  
##  1st Qu.:  32.0  
##  Median :  62.0  
##  Mean   : 128.8  
##  3rd Qu.: 141.0  
##  Max.   :6616.0
\end{verbatim}

Ratings per user plot

\hfill\break

\begin{Shaded}
\begin{Highlighting}[]
\NormalTok{edx }\SpecialCharTok{|\textgreater{}}
  \FunctionTok{group\_by}\NormalTok{(userId) }\SpecialCharTok{|\textgreater{}}
  \FunctionTok{summarize}\NormalTok{(}\AttributeTok{count =} \FunctionTok{n}\NormalTok{()) }\SpecialCharTok{|\textgreater{}}
  \FunctionTok{ggplot}\NormalTok{(}\FunctionTok{aes}\NormalTok{(}\AttributeTok{x =}\NormalTok{ userId, }\AttributeTok{y =}\NormalTok{ count)) }\SpecialCharTok{+}
  \FunctionTok{geom\_point}\NormalTok{(}\AttributeTok{alpha =} \FloatTok{0.2}\NormalTok{, }\AttributeTok{color =} \StringTok{"\#4020dd"}\NormalTok{) }\SpecialCharTok{+}
  \FunctionTok{geom\_smooth}\NormalTok{(}\AttributeTok{color =} \StringTok{"red"}\NormalTok{) }\SpecialCharTok{+}
  \FunctionTok{ggtitle}\NormalTok{(}\StringTok{"Ratings per user"}\NormalTok{) }\SpecialCharTok{+}
  \FunctionTok{xlab}\NormalTok{(}\StringTok{"Users"}\NormalTok{) }\SpecialCharTok{+}
  \FunctionTok{ylab}\NormalTok{(}\StringTok{"Number of ratings"}\NormalTok{) }\SpecialCharTok{+}
  \FunctionTok{scale\_y\_continuous}\NormalTok{(}\AttributeTok{labels =}\NormalTok{ comma) }\SpecialCharTok{+}
  \FunctionTok{scale\_x\_continuous}\NormalTok{(}\AttributeTok{n.breaks =} \DecValTok{10}\NormalTok{) }\SpecialCharTok{+}
  \FunctionTok{theme\_economist}\NormalTok{() }\SpecialCharTok{+}
  \FunctionTok{theme}\NormalTok{(}\AttributeTok{axis.title.x =} \FunctionTok{element\_text}\NormalTok{(}\AttributeTok{vjust =} \SpecialCharTok{{-}}\DecValTok{5}\NormalTok{, }\AttributeTok{face =} \StringTok{"bold"}\NormalTok{), }
        \AttributeTok{axis.title.y =} \FunctionTok{element\_text}\NormalTok{(}\AttributeTok{vjust =} \DecValTok{10}\NormalTok{, }\AttributeTok{face =} \StringTok{"bold"}\NormalTok{), }
        \AttributeTok{plot.margin =} \FunctionTok{margin}\NormalTok{(}\FloatTok{0.7}\NormalTok{, }\FloatTok{0.5}\NormalTok{, }\DecValTok{1}\NormalTok{, }\FloatTok{1.2}\NormalTok{, }\StringTok{"cm"}\NormalTok{))}
\end{Highlighting}
\end{Shaded}

\begin{verbatim}
## `geom_smooth()` using method = 'gam' and formula = 'y ~ s(x, bs = "cs")'
\end{verbatim}

\includegraphics{capstone-movielens-report.draft4_files/figure-latex/unnamed-chunk-22-1.pdf}

Users' rating histogram

\hfill\break

\begin{Shaded}
\begin{Highlighting}[]
\NormalTok{edx }\SpecialCharTok{|\textgreater{}}
  \FunctionTok{group\_by}\NormalTok{(userId) }\SpecialCharTok{|\textgreater{}}
  \FunctionTok{summarize}\NormalTok{(}\AttributeTok{count =} \FunctionTok{n}\NormalTok{()) }\SpecialCharTok{|\textgreater{}}
  \FunctionTok{ggplot}\NormalTok{(}\FunctionTok{aes}\NormalTok{(}\AttributeTok{x =}\NormalTok{ count)) }\SpecialCharTok{+}
  \FunctionTok{geom\_histogram}\NormalTok{(}\AttributeTok{fill =} \StringTok{"\#8888ff"}\NormalTok{, }\AttributeTok{color =} \StringTok{"\#4020dd"}\NormalTok{) }\SpecialCharTok{+}
  \FunctionTok{ggtitle}\NormalTok{(}\StringTok{"Users\textquotesingle{} rating histogram"}\NormalTok{) }\SpecialCharTok{+}
  \FunctionTok{xlab}\NormalTok{(}\StringTok{"Rating count"}\NormalTok{) }\SpecialCharTok{+}
  \FunctionTok{ylab}\NormalTok{(}\StringTok{"Number of users"}\NormalTok{) }\SpecialCharTok{+}
  \FunctionTok{scale\_y\_continuous}\NormalTok{(}\AttributeTok{labels =}\NormalTok{ comma) }\SpecialCharTok{+}
  \FunctionTok{scale\_x\_log10}\NormalTok{(}\AttributeTok{n.breaks =} \DecValTok{10}\NormalTok{) }\SpecialCharTok{+}
  \FunctionTok{theme\_economist}\NormalTok{() }\SpecialCharTok{+}
  \FunctionTok{theme}\NormalTok{(}\AttributeTok{axis.title.x =} \FunctionTok{element\_text}\NormalTok{(}\AttributeTok{vjust =} \SpecialCharTok{{-}}\DecValTok{5}\NormalTok{, }\AttributeTok{face =} \StringTok{"bold"}\NormalTok{), }
        \AttributeTok{axis.title.y =} \FunctionTok{element\_text}\NormalTok{(}\AttributeTok{vjust =} \DecValTok{10}\NormalTok{, }\AttributeTok{face =} \StringTok{"bold"}\NormalTok{), }
        \AttributeTok{plot.margin =} \FunctionTok{margin}\NormalTok{(}\FloatTok{0.7}\NormalTok{, }\FloatTok{0.5}\NormalTok{, }\DecValTok{1}\NormalTok{, }\FloatTok{1.2}\NormalTok{, }\StringTok{"cm"}\NormalTok{))}
\end{Highlighting}
\end{Shaded}

\begin{verbatim}
## `stat_bin()` using `bins = 30`. Pick better value with `binwidth`.
\end{verbatim}

\includegraphics{capstone-movielens-report.draft4_files/figure-latex/unnamed-chunk-23-1.pdf}

\subsection{Methods / Analysis}\label{methods-analysis}

\begin{noteblock}
All the source code of the R-scripts is available on the project's
\href{https://github.com/AzKurban-edX-DS/Capstone-MovieLens/tree/main/r/src}{GitHub
repository}\autocite{edx_capstone_movielens}.

\end{noteblock}

\subsubsection{Defining Logging and Mesuring helper
functions}\label{defining-logging-and-mesuring-helper-functions}

\hfill\break

First, let's define some helper functions for logging and time-measuring
features that we will use in our R scripts. Some of them are listed
below:

\begin{Shaded}
\begin{Highlighting}[]
\NormalTok{open\_logfile }\OtherTok{\textless{}{-}} \ControlFlowTok{function}\NormalTok{(file\_name)\{}
\NormalTok{  log\_file\_name }\OtherTok{\textless{}{-}} \FunctionTok{as.character}\NormalTok{(}\FunctionTok{Sys.time}\NormalTok{()) }\SpecialCharTok{|\textgreater{}} 
    \FunctionTok{str\_replace\_all}\NormalTok{(}\StringTok{\textquotesingle{}:\textquotesingle{}}\NormalTok{, }\StringTok{\textquotesingle{}\_\textquotesingle{}}\NormalTok{) }\SpecialCharTok{|\textgreater{}} 
    \FunctionTok{str\_replace}\NormalTok{(}\StringTok{\textquotesingle{} \textquotesingle{}}\NormalTok{, }\StringTok{\textquotesingle{}T\textquotesingle{}}\NormalTok{) }\SpecialCharTok{|\textgreater{}}
    \FunctionTok{str\_c}\NormalTok{(file\_name)}
  
  \FunctionTok{log\_open}\NormalTok{(}\AttributeTok{file\_name =}\NormalTok{ log\_file\_name)}
\NormalTok{\}}
\NormalTok{print\_start\_date }\OtherTok{\textless{}{-}} \ControlFlowTok{function}\NormalTok{()\{}
  \FunctionTok{print}\NormalTok{(}\FunctionTok{date}\NormalTok{())}
  \FunctionTok{Sys.time}\NormalTok{()}
\NormalTok{\}}
\NormalTok{put\_start\_date }\OtherTok{\textless{}{-}} \ControlFlowTok{function}\NormalTok{()\{}
  \FunctionTok{put}\NormalTok{(}\FunctionTok{date}\NormalTok{())}
  \FunctionTok{Sys.time}\NormalTok{()}
\NormalTok{\}}
\NormalTok{print\_end\_date }\OtherTok{\textless{}{-}} \ControlFlowTok{function}\NormalTok{(start)\{}
  \FunctionTok{print}\NormalTok{(}\FunctionTok{date}\NormalTok{())}
  \FunctionTok{print}\NormalTok{(}\FunctionTok{Sys.time}\NormalTok{() }\SpecialCharTok{{-}}\NormalTok{ start)}
\NormalTok{\}}
\NormalTok{put\_end\_date }\OtherTok{\textless{}{-}} \ControlFlowTok{function}\NormalTok{(start)\{}
  \FunctionTok{put}\NormalTok{(}\FunctionTok{date}\NormalTok{())}
  \FunctionTok{put}\NormalTok{(}\FunctionTok{Sys.time}\NormalTok{() }\SpecialCharTok{{-}}\NormalTok{ start)}
\NormalTok{\}}

\NormalTok{print\_log }\OtherTok{\textless{}{-}} \ControlFlowTok{function}\NormalTok{(msg)\{}
  \FunctionTok{print}\NormalTok{(}\FunctionTok{str\_glue}\NormalTok{(msg))}
\NormalTok{\}}
\NormalTok{put\_log }\OtherTok{\textless{}{-}} \ControlFlowTok{function}\NormalTok{(msg)\{}
  \FunctionTok{put}\NormalTok{(}\FunctionTok{str\_glue}\NormalTok{(msg))}
\NormalTok{\}}
\NormalTok{put\_log1 }\OtherTok{\textless{}{-}} \ControlFlowTok{function}\NormalTok{(msg\_template, arg1)\{}
\NormalTok{  msg }\OtherTok{\textless{}{-}} \FunctionTok{str\_replace\_all}\NormalTok{(msg\_template, }\StringTok{"\%1"}\NormalTok{, }\FunctionTok{as.character}\NormalTok{(arg1))}
  \FunctionTok{put}\NormalTok{(}\FunctionTok{str\_glue}\NormalTok{(msg))}
\NormalTok{\}}
\NormalTok{put\_log2 }\OtherTok{\textless{}{-}} \ControlFlowTok{function}\NormalTok{(msg\_template, arg1, arg2)\{}
\NormalTok{  msg }\OtherTok{\textless{}{-}}\NormalTok{ msg\_template }\SpecialCharTok{|\textgreater{}} 
    \FunctionTok{str\_replace\_all}\NormalTok{(}\StringTok{"\%1"}\NormalTok{, }\FunctionTok{as.character}\NormalTok{(arg1)) }\SpecialCharTok{|\textgreater{}}
    \FunctionTok{str\_replace\_all}\NormalTok{(}\StringTok{"\%2"}\NormalTok{, }\FunctionTok{as.character}\NormalTok{(arg2)) }\SpecialCharTok{|\textgreater{}}
    \FunctionTok{str\_glue}\NormalTok{()}
  
  \FunctionTok{put}\NormalTok{(msg)}
\NormalTok{\}}
\end{Highlighting}
\end{Shaded}

\begin{noteblock}
The full source code of these functions is available in the
\href{https://github.com/AzKurban-edX-DS/Capstone-MovieLens/blob/2c5d96ac98a8b8fa4151aa8295f91ac29d9472b3/r/src/capstone-movielens.main.R\#L1}{Logging
Helper function section of the Capstone MovieLens Main R Script}.

\end{noteblock}

\subsubsection{Preparing train and set
datasets}\label{preparing-train-and-set-datasets}

\hfill\break

We will split the \texttt{edx} dataset into a training set, which we
will use to build and train our models, and a test set in which we will
compute the accuracy of our predictions, the way described in
\href{https://rafalab.dfci.harvard.edu/dsbook-part-2/highdim/regularization.html\#movielens-data}{Section
23.1.1 Movielens data} of the \emph{Course Textbook} mentioned
above\autocite{IDS2_23-1-1}.We will also use the \emph{5-Fold Cross
Validation} method the way described in
\href{https://rafalab.dfci.harvard.edu/dsbook-part-2/ml/resampling-methods.html\#cross-validation}{Section
29.6 Cross validation} of the \emph{Course Textbook}. To prepare
datasets for processing, we will use the following functions,
specifically designed for these operations:

\begin{Shaded}
\begin{Highlighting}[]
\FunctionTok{make\_source\_datasets}\NormalTok{()}
\FunctionTok{init\_source\_datasets}\NormalTok{()}
\end{Highlighting}
\end{Shaded}

\begin{itemize}
\tightlist
\item
  \texttt{make\_source\_datasets};
\item
  \texttt{init\_source\_datasets}.
\end{itemize}

\begin{noteblock}
The full source code of the function listed above is available in the
\href{https://github.com/AzKurban-edX-DS/Capstone-MovieLens/blob/2c5d96ac98a8b8fa4151aa8295f91ac29d9472b3/r/src/support-functions/data.helper.functions.R\#L86}{\texttt{Initialize\ input\ datasets}
section of the \texttt{data.helper.functions.R} script on \emph{GitHub}}

\end{noteblock}

We will use the array representation described in
\href{https://rafalab.dfci.harvard.edu/dsbook-part-2/linear-models/treatment-effect-models.html\#sec-anova}{Section
17.5 of the Textbook}, for the training data: we denote ranking for
movie \(j\) by user \(i\) as \(y_{i,j}\). To create this matrix, we use
\texttt{tidyr::pivot\_wider} function:

\begin{Shaded}
\begin{Highlighting}[]
\NormalTok{y }\OtherTok{\textless{}{-}}\NormalTok{ dplyr}\SpecialCharTok{::}\FunctionTok{select}\NormalTok{(train\_set, movieId, userId, rating) }\SpecialCharTok{|\textgreater{}}
\FunctionTok{pivot\_wider}\NormalTok{(}\AttributeTok{names\_from =}\NormalTok{ movieId, }\AttributeTok{values\_from =}\NormalTok{ rating) }\SpecialCharTok{|\textgreater{}}
\FunctionTok{column\_to\_rownames}\NormalTok{(}\StringTok{"userId"}\NormalTok{) }\SpecialCharTok{|\textgreater{}}
\FunctionTok{as.matrix}\NormalTok{()}

\FunctionTok{dim}\NormalTok{(y)}
\end{Highlighting}
\end{Shaded}

\begin{verbatim}
## [1] 24115 10630
\end{verbatim}

To be able to map movie IDs to titles we create the following lookup
table:

\begin{Shaded}
\begin{Highlighting}[]
\NormalTok{movie\_map }\OtherTok{\textless{}{-}}\NormalTok{ train\_set }\SpecialCharTok{|\textgreater{}}\NormalTok{ dplyr}\SpecialCharTok{::}\FunctionTok{select}\NormalTok{(movieId, title, genres) }\SpecialCharTok{|\textgreater{}} 
  \FunctionTok{distinct}\NormalTok{(movieId, }\AttributeTok{.keep\_all =} \ConstantTok{TRUE}\NormalTok{)}

\FunctionTok{summary}\NormalTok{(movie\_map)}
\end{Highlighting}
\end{Shaded}

\begin{verbatim}
##     movieId         title              genres         
##  1      :    1   Length:10630       Length:10630      
##  2      :    1   Class :character   Class :character  
##  3      :    1   Mode  :character   Mode  :character  
##  4      :    1                                        
##  5      :    1                                        
##  6      :    1                                        
##  (Other):10624
\end{verbatim}

Note that titles cannot be considered unique, so we can't use them as
IDs\autocite{IDS2_23-1-1}.

\subsubsection{Naive Model}\label{naive-model}

\hfill\break
Let's begin our analysis by evaluating the simplest model described in
\href{https://rafalab.dfci.harvard.edu/dsbook-part-2/highdim/regularization.html\#a-first-model}{Section
\emph{23.3 The First Model} of the \emph{Course Textbook}}, and then
gradually refine it through further research. It is about a model that
assumes the same rating for all movies and users with all the
differences explained by random variation would look as follows:

\[
Y_{i,j} = \mu + \varepsilon_{i,j}
\]

with \(\varepsilon_{i,j}\) independent errors sampled from the same
distribution centered at 0 and \(\mu\) the \emph{true} rating for all
movies.

We know that the estimate that minimizes the RMSE is the least squares
estimate of \(\mu\) and, in this case, is the average of all ratings:

\begin{Shaded}
\begin{Highlighting}[]
\NormalTok{mu }\OtherTok{\textless{}{-}} \FunctionTok{mean}\NormalTok{(y, }\AttributeTok{na.rm =} \ConstantTok{TRUE}\NormalTok{)}
\FunctionTok{print}\NormalTok{(mu)}
\end{Highlighting}
\end{Shaded}

\begin{verbatim}
## [1] 3.47184
\end{verbatim}

If we predict all unknown ratings with \(\hat{\mu}\), we obtain the
following RMSE:

\begin{Shaded}
\begin{Highlighting}[]
\FunctionTok{rmse}\NormalTok{(test\_set}\SpecialCharTok{$}\NormalTok{rating }\SpecialCharTok{{-}}\NormalTok{ mu)}
\end{Highlighting}
\end{Shaded}

\begin{verbatim}
## [1] 1.054739
\end{verbatim}

If we plug in any other number, we will get a higher RMSE. Let's prove
that by the following small investigation:

\begin{Shaded}
\begin{Highlighting}[]
\NormalTok{deviation }\OtherTok{\textless{}{-}} \FunctionTok{seq}\NormalTok{(}\DecValTok{0}\NormalTok{, }\DecValTok{6}\NormalTok{, }\FloatTok{0.1}\NormalTok{) }\SpecialCharTok{{-}} \DecValTok{3}
\FunctionTok{print}\NormalTok{(deviation)}
\end{Highlighting}
\end{Shaded}

\begin{verbatim}
##  [1] -3.0 -2.9 -2.8 -2.7 -2.6 -2.5 -2.4 -2.3 -2.2 -2.1 -2.0 -1.9 -1.8 -1.7 -1.6 -1.5 -1.4 -1.3 -1.2 -1.1 -1.0 -0.9 -0.8 -0.7 -0.6 -0.5 -0.4 -0.3 -0.2 -0.1  0.0  0.1  0.2  0.3  0.4  0.5
## [37]  0.6  0.7  0.8  0.9  1.0  1.1  1.2  1.3  1.4  1.5  1.6  1.7  1.8  1.9  2.0  2.1  2.2  2.3  2.4  2.5  2.6  2.7  2.8  2.9  3.0
\end{verbatim}

\begin{Shaded}
\begin{Highlighting}[]
\NormalTok{rmse\_value }\OtherTok{\textless{}{-}} \FunctionTok{sapply}\NormalTok{(deviation, }\ControlFlowTok{function}\NormalTok{(diff)\{}
  \FunctionTok{rmse}\NormalTok{(test\_set}\SpecialCharTok{$}\NormalTok{rating }\SpecialCharTok{{-}}\NormalTok{ mu }\SpecialCharTok{+}\NormalTok{ diff)}
\NormalTok{\})}

\FunctionTok{plot}\NormalTok{(deviation, rmse\_value, }\AttributeTok{type =} \StringTok{"l"}\NormalTok{)}
\end{Highlighting}
\end{Shaded}

\includegraphics{capstone-movielens-report.draft4_files/figure-latex/unnamed-chunk-29-1.pdf}

\begin{Shaded}
\begin{Highlighting}[]
\FunctionTok{sprintf}\NormalTok{(}\StringTok{"Minimum RMSE is achieved when the deviation from the mean is: \%s"}\NormalTok{, }
\NormalTok{        deviation[}\FunctionTok{which.min}\NormalTok{(rmse\_value)])}
\end{Highlighting}
\end{Shaded}

\begin{verbatim}
## [1] "Minimum RMSE is achieved when the deviation from the mean is: 0"
\end{verbatim}

To win the grand prize of \$1,000,000, a participating team had to get
an RMSE of at least 0.8563\autocite{BigChaosSln}. So we can definitely
do better!\autocite{IDS2_23-3}

\subsubsection{Taking into account User
effects}\label{taking-into-account-user-effects}

\hfill\break
To improve our model let's now take into consideration user effects as
explained in
\href{https://rafalab.dfci.harvard.edu/dsbook-part-2/highdim/regularization.html\#user-effects}{Section
\emph{23.4 User effects} of the \emph{Course Textbook}}. If we visualize
the average rating for each user the way the
\href{https://x.com/rafalab}{the author} shows, we can see that there is
substantial variability in the average ratings across users:

\begin{Shaded}
\begin{Highlighting}[]
\FunctionTok{hist}\NormalTok{(}\FunctionTok{rowMeans}\NormalTok{(y, }\AttributeTok{na.rm =} \ConstantTok{TRUE}\NormalTok{), }\AttributeTok{nclass =} \DecValTok{30}\NormalTok{)}
\end{Highlighting}
\end{Shaded}

\includegraphics{capstone-movielens-report.draft4_files/figure-latex/unnamed-chunk-30-1.pdf}

Following the author's further explanation, to account for this
variability, we will use a linear model with a \emph{treatment effect}
\(\alpha_i\) for each user. The sum \(\mu+\alpha_i\) can be interpreted
as the typical rating user \(i\) gives to movies. So we write the model
as follows:

\[
Y_{i,j} = \mu + \alpha_i + \varepsilon_{i,j}
\]

Statistics textbooks refer to the \(\alpha\)s as treatment effects. In
the Netflix challenge papers, they refer to them as
\emph{bias}\autocite{IDS2_23-4,MFT_RS}.

As it is stated here\autocite{IDS2_23-4}, it can be shown that the least
squares estimate \(\hat{\alpha}_i\) is just the average of
\(y_{i,j} - \hat{\mu}\) for each user \(i\). So we compute them this
way:

\begin{Shaded}
\begin{Highlighting}[]
\NormalTok{a }\OtherTok{\textless{}{-}} \FunctionTok{rowMeans}\NormalTok{(y }\SpecialCharTok{{-}}\NormalTok{ mu, }\AttributeTok{na.rm =} \ConstantTok{TRUE}\NormalTok{)}
\end{Highlighting}
\end{Shaded}

Finally, we are ready to compute the \texttt{RMSE} (additionally using
the helper function \texttt{clamp} we defined above to keep predictions
in the proper range):

\begin{Shaded}
\begin{Highlighting}[]
\CommentTok{\# Compute the RMSE taking into account user effects:}
\NormalTok{user\_effects\_rmse }\OtherTok{\textless{}{-}}\NormalTok{ test\_set }\SpecialCharTok{|\textgreater{}} 
    \FunctionTok{left\_join}\NormalTok{(}\FunctionTok{data.frame}\NormalTok{(}\AttributeTok{userId =} \FunctionTok{as.integer}\NormalTok{(}\FunctionTok{names}\NormalTok{(a)), }\AttributeTok{a =}\NormalTok{ a), }\AttributeTok{by =} \StringTok{"userId"}\NormalTok{) }\SpecialCharTok{|\textgreater{}}
    \FunctionTok{mutate}\NormalTok{(}\AttributeTok{resid =}\NormalTok{ rating }\SpecialCharTok{{-}} \FunctionTok{clamp}\NormalTok{(mu }\SpecialCharTok{+}\NormalTok{ a)) }\SpecialCharTok{|\textgreater{}} 
    \FunctionTok{filter}\NormalTok{(}\SpecialCharTok{!}\FunctionTok{is.na}\NormalTok{(resid)) }\SpecialCharTok{|\textgreater{}}
    \FunctionTok{pull}\NormalTok{(resid) }\SpecialCharTok{|\textgreater{}} \FunctionTok{rmse}\NormalTok{()}

\FunctionTok{print}\NormalTok{(user\_effects\_rmse)}
\end{Highlighting}
\end{Shaded}

\begin{verbatim}
## [1] 0.9707208
\end{verbatim}

\subsubsection{Taking into account Movie
effects}\label{taking-into-account-movie-effects}

\hfill\break
In
\href{https://rafalab.dfci.harvard.edu/dsbook-part-2/highdim/regularization.html\#movie-effects}{Section
\emph{23.5 Movie effects} of the \emph{Course Textbook}} the author
draws our attention to the fact that some movies are generally rated
higher than others. He also explains that~a linear model with a
\emph{treatment effect} \(\beta_j\) for each movie can be used in this
case, which can be interpreted as movie effect or the difference between
the average ranking for movie \(j\) and the overall average \(\mu\):

\[
Y_{i,j} = \mu + \alpha_i + \beta_j +\varepsilon_{i,j}
\] The author then shows how to use an approximation by first computing
the least square estimate \(\hat{\mu}\) and \(\hat{\alpha}_i\), and then
estimating \(\hat{\beta}_j\) as the average of the residuals
\(y_{i,j} - \hat{\mu} - \hat{\alpha}_i\):

\begin{Shaded}
\begin{Highlighting}[]
\NormalTok{b }\OtherTok{\textless{}{-}} \FunctionTok{colMeans}\NormalTok{(y }\SpecialCharTok{{-}}\NormalTok{ mu }\SpecialCharTok{{-}}\NormalTok{ a, }\AttributeTok{na.rm =} \ConstantTok{TRUE}\NormalTok{)}
\end{Highlighting}
\end{Shaded}

We can now construct predictors and see how much the \texttt{RMSE}
improves\autocite{IDS2_23-5}:

\begin{Shaded}
\begin{Highlighting}[]
\NormalTok{user\_and\_movie\_effects\_rmse }\OtherTok{\textless{}{-}}\NormalTok{ test\_set }\SpecialCharTok{|\textgreater{}} 
    \FunctionTok{left\_join}\NormalTok{(}\FunctionTok{data.frame}\NormalTok{(}\AttributeTok{userId =} \FunctionTok{as.integer}\NormalTok{(}\FunctionTok{names}\NormalTok{(a)), }\AttributeTok{a =}\NormalTok{ a), }\AttributeTok{by =} \StringTok{"userId"}\NormalTok{) }\SpecialCharTok{|\textgreater{}}
    \FunctionTok{left\_join}\NormalTok{(}\FunctionTok{data.frame}\NormalTok{(}\AttributeTok{movieId =} \FunctionTok{as.integer}\NormalTok{(}\FunctionTok{names}\NormalTok{(b)), }\AttributeTok{b =}\NormalTok{ b), }\AttributeTok{by =} \StringTok{"movieId"}\NormalTok{) }\SpecialCharTok{|\textgreater{}}
    \FunctionTok{mutate}\NormalTok{(}\AttributeTok{resid =}\NormalTok{ rating }\SpecialCharTok{{-}} \FunctionTok{clamp}\NormalTok{(mu }\SpecialCharTok{+}\NormalTok{ a }\SpecialCharTok{+}\NormalTok{ b)) }\SpecialCharTok{|\textgreater{}}  
    \FunctionTok{filter}\NormalTok{(}\SpecialCharTok{!}\FunctionTok{is.na}\NormalTok{(resid)) }\SpecialCharTok{|\textgreater{}}
    \FunctionTok{pull}\NormalTok{(resid) }\SpecialCharTok{|\textgreater{}} \FunctionTok{rmse}\NormalTok{()}

\FunctionTok{print}\NormalTok{(user\_and\_movie\_effects\_rmse)}
\end{Highlighting}
\end{Shaded}

\begin{verbatim}
## [1] 0.8662207
\end{verbatim}

\subsubsection{Utilizing Penalized least
squares}\label{utilizing-penalized-least-squares}

\hfill\break
\href{https://rafalab.dfci.harvard.edu/dsbook-part-2/highdim/regularization.html\#penalized-least-squares}{Section
\emph{23.6 Penalized least squares} of the \emph{Course Textbook}}
explains why and how we should use \emph{Penalized least squares} to
improve our predictions. The author also explains that the general idea
of penalized regression is to control the total variability of the movie
effects: \(\sum_{j=1}^n \beta_j^2\). Specifically, instead of minimizing
the least squares equation, we minimize an equation that adds a penalty:

\[ 
\sum_{i,j} \left(y_{u,i} - \mu - \alpha_i - \beta_j \right)^2 + \lambda \sum_{j} \beta_j^2
\] The first term is just the sum of squares and the second is a penalty
that gets larger when many \(\beta_i\)s are large. Using calculus, we
can actually show that the values of \(\beta_i\) that minimize this
equation are:

\[
\hat{\beta}_j(\lambda) = \frac{1}{\lambda + n_j} \sum_{i=1}^{n_i} \left(Y_{i,j} - \mu - \alpha_i\right)
\]

where \(n_j\) is the number of ratings made for movie \(j\).

This approach will have our desired effect: when our sample size \(n_j\)
is very large, we obtain a stable estimate and the penalty \(\lambda\)
is effectively ignored since \(n_j+\lambda \approx n_j\). Yet when the
\(n_j\) is small, then the estimate \(\hat{\beta}_i(\lambda)\) is
shrunken towards 0. The larger the \(\lambda\), the more we
shrink\autocite{IDS2_23-6}.

\paragraph{Support function}\label{support-function}

\hfill\break
We will use the following function to calculate \emph{RMSE} in this
section:

\begin{Shaded}
\begin{Highlighting}[]
\NormalTok{reg\_rmse }\OtherTok{\textless{}{-}} \ControlFlowTok{function}\NormalTok{(b)\{}
\NormalTok{  test\_set }\SpecialCharTok{|\textgreater{}} 
    \FunctionTok{left\_join}\NormalTok{(}\FunctionTok{data.frame}\NormalTok{(}\AttributeTok{userId =} \FunctionTok{as.integer}\NormalTok{(}\FunctionTok{names}\NormalTok{(a)), }\AttributeTok{a =}\NormalTok{ a), }\AttributeTok{by =} \StringTok{"userId"}\NormalTok{) }\SpecialCharTok{|\textgreater{}}
    \FunctionTok{left\_join}\NormalTok{(}\FunctionTok{data.frame}\NormalTok{(}\AttributeTok{movieId =} \FunctionTok{as.integer}\NormalTok{(}\FunctionTok{names}\NormalTok{(b)), }\AttributeTok{b =}\NormalTok{ b), }\AttributeTok{by =} \StringTok{"movieId"}\NormalTok{) }\SpecialCharTok{|\textgreater{}}
    \FunctionTok{mutate}\NormalTok{(}\AttributeTok{resid =}\NormalTok{ rating }\SpecialCharTok{{-}} \FunctionTok{clamp}\NormalTok{(mu }\SpecialCharTok{+}\NormalTok{ a }\SpecialCharTok{+}\NormalTok{ b)) }\SpecialCharTok{|\textgreater{}} 
    \FunctionTok{filter}\NormalTok{(}\SpecialCharTok{!}\FunctionTok{is.na}\NormalTok{(resid)) }\SpecialCharTok{|\textgreater{}}
    \FunctionTok{pull}\NormalTok{(resid) }\SpecialCharTok{|\textgreater{}} \FunctionTok{rmse}\NormalTok{()}
\NormalTok{\}}
\end{Highlighting}
\end{Shaded}

Let's now figure out the \(\lambda\) that minimizes the \emph{RMSE}:

\begin{Shaded}
\begin{Highlighting}[]
\CommentTok{\# Here we will simply compute the RMSE for different values of \textasciigrave{}lambda\textasciigrave{} }
\NormalTok{n }\OtherTok{\textless{}{-}} \FunctionTok{colSums}\NormalTok{(}\SpecialCharTok{!}\FunctionTok{is.na}\NormalTok{(y))}

\NormalTok{sums }\OtherTok{\textless{}{-}} \FunctionTok{colSums}\NormalTok{(y }\SpecialCharTok{{-}}\NormalTok{ mu }\SpecialCharTok{{-}}\NormalTok{ a, }\AttributeTok{na.rm =} \ConstantTok{TRUE}\NormalTok{)}
\NormalTok{lambdas }\OtherTok{\textless{}{-}} \FunctionTok{seq}\NormalTok{(}\DecValTok{0}\NormalTok{, }\DecValTok{10}\NormalTok{, }\FloatTok{0.1}\NormalTok{)}

\NormalTok{rmses }\OtherTok{\textless{}{-}} \FunctionTok{sapply}\NormalTok{(lambdas, }\ControlFlowTok{function}\NormalTok{(lambda)\{}
\NormalTok{  b }\OtherTok{\textless{}{-}}\NormalTok{  sums }\SpecialCharTok{/}\NormalTok{ (n }\SpecialCharTok{+}\NormalTok{ lambda)}
  \FunctionTok{reg\_rmse}\NormalTok{(b)}
\NormalTok{\})}

\CommentTok{\# Here is a plot of the RMSE versus \textasciigrave{}lambda\textasciigrave{}:}
\FunctionTok{plot}\NormalTok{(lambdas, rmses, }\AttributeTok{type =} \StringTok{"l"}\NormalTok{)}
\end{Highlighting}
\end{Shaded}

\includegraphics{capstone-movielens-report.draft4_files/figure-latex/unnamed-chunk-36-1.pdf}
Now we can determine the minimal \emph{RMSE}:

\begin{Shaded}
\begin{Highlighting}[]
\FunctionTok{print}\NormalTok{(}\FunctionTok{min}\NormalTok{(rmses))}
\end{Highlighting}
\end{Shaded}

\begin{verbatim}
## [1] 0.8661143
\end{verbatim}

which is achieved for the following \(\lambda\):

\begin{Shaded}
\begin{Highlighting}[]
\NormalTok{lambda }\OtherTok{\textless{}{-}}\NormalTok{ lambdas[}\FunctionTok{which.min}\NormalTok{(rmses)] }
\FunctionTok{print}\NormalTok{(lambda)}
\end{Highlighting}
\end{Shaded}

\begin{verbatim}
## [1] 3.2
\end{verbatim}

Using this \(\lambda\) we can compute the regularized estimates:

\begin{Shaded}
\begin{Highlighting}[]
\NormalTok{b\_reg }\OtherTok{\textless{}{-}}\NormalTok{ sums }\SpecialCharTok{/}\NormalTok{ (n }\SpecialCharTok{+}\NormalTok{ lambda)}

\FunctionTok{str}\NormalTok{(b\_reg)}
\end{Highlighting}
\end{Shaded}

\begin{verbatim}
##  Named num [1:10630] -0.516 0.327 -0.903 -0.154 -0.29 ...
##  - attr(*, "names")= chr [1:10630] "5" "6" "19" "22" ...
\end{verbatim}

Finally, let's verify that the penalized estimates
\(\hat{b}_i(\lambda)\) we have just computed actually result in the
minimal \emph{RMSE} figured out above:

\begin{Shaded}
\begin{Highlighting}[]
\FunctionTok{reg\_rmse}\NormalTok{(b\_reg)}
\end{Highlighting}
\end{Shaded}

\begin{verbatim}
## [1] 0.8661143
\end{verbatim}

\subsubsection{Accounting for Date
effects}\label{accounting-for-date-effects}

\subparagraph{\texorpdfstring{Yearly rating
count\autocite{MRS-R-BEST}}{Yearly rating count{[}@MRS-R-BEST{]}}}\label{yearly-rating-countmrs-r-best}

\hfill\break

\begin{Shaded}
\begin{Highlighting}[]
\FunctionTok{print}\NormalTok{(edx }\SpecialCharTok{|\textgreater{}} 
  \FunctionTok{mutate}\NormalTok{(}\AttributeTok{year =} \FunctionTok{year}\NormalTok{(}\FunctionTok{as\_datetime}\NormalTok{(timestamp, }\AttributeTok{origin =} \StringTok{"1970{-}01{-}01"}\NormalTok{))) }\SpecialCharTok{|\textgreater{}}
  \FunctionTok{group\_by}\NormalTok{(year) }\SpecialCharTok{|\textgreater{}}
  \FunctionTok{summarize}\NormalTok{(}\AttributeTok{count =} \FunctionTok{n}\NormalTok{())}
\NormalTok{)}
\end{Highlighting}
\end{Shaded}

\begin{verbatim}
## # A tibble: 15 x 2
##     year   count
##    <dbl>   <int>
##  1  1995       2
##  2  1996  942772
##  3  1997  414101
##  4  1998  181634
##  5  1999  709893
##  6  2000 1144349
##  7  2001  683355
##  8  2002  524959
##  9  2003  619938
## 10  2004  691429
## 11  2005 1059277
## 12  2006  689315
## 13  2007  629168
## 14  2008  696740
## 15  2009   13123
\end{verbatim}

\subparagraph{\texorpdfstring{Average rating per year
plot\autocite{MRS-R-BEST}}{Average rating per year plot{[}@MRS-R-BEST{]}}}\label{average-rating-per-year-plotmrs-r-best}

\hfill\break

\begin{Shaded}
\begin{Highlighting}[]
\NormalTok{edx }\SpecialCharTok{|\textgreater{}} 
  \FunctionTok{mutate}\NormalTok{(}\AttributeTok{year =} \FunctionTok{year}\NormalTok{(}\FunctionTok{as\_datetime}\NormalTok{(timestamp, }\AttributeTok{origin =} \StringTok{"1970{-}01{-}01"}\NormalTok{))) }\SpecialCharTok{|\textgreater{}}
  \FunctionTok{group\_by}\NormalTok{(year) }\SpecialCharTok{|\textgreater{}}
  \FunctionTok{summarize}\NormalTok{(}\AttributeTok{rating\_avg =} \FunctionTok{mean}\NormalTok{(rating)) }\SpecialCharTok{|\textgreater{}}
  \FunctionTok{ggplot}\NormalTok{(}\FunctionTok{aes}\NormalTok{(}\AttributeTok{x =}\NormalTok{ year, }\AttributeTok{y =}\NormalTok{ rating\_avg)) }\SpecialCharTok{+}
  \FunctionTok{geom\_bar}\NormalTok{(}\AttributeTok{stat =} \StringTok{"identity"}\NormalTok{, }\AttributeTok{fill =} \StringTok{"\#8888ff"}\NormalTok{) }\SpecialCharTok{+} 
  \FunctionTok{ggtitle}\NormalTok{(}\StringTok{"Average rating per year"}\NormalTok{) }\SpecialCharTok{+}
  \FunctionTok{xlab}\NormalTok{(}\StringTok{"Year"}\NormalTok{) }\SpecialCharTok{+}
  \FunctionTok{ylab}\NormalTok{(}\StringTok{"Average rating"}\NormalTok{) }\SpecialCharTok{+}
  \FunctionTok{scale\_y\_continuous}\NormalTok{(}\AttributeTok{labels =}\NormalTok{ comma) }\SpecialCharTok{+} 
  \FunctionTok{theme\_economist}\NormalTok{() }\SpecialCharTok{+}
  \FunctionTok{theme}\NormalTok{(}\AttributeTok{axis.title.x =} \FunctionTok{element\_text}\NormalTok{(}\AttributeTok{vjust =} \SpecialCharTok{{-}}\DecValTok{5}\NormalTok{, }\AttributeTok{face =} \StringTok{"bold"}\NormalTok{), }
        \AttributeTok{axis.title.y =} \FunctionTok{element\_text}\NormalTok{(}\AttributeTok{vjust =} \DecValTok{10}\NormalTok{, }\AttributeTok{face =} \StringTok{"bold"}\NormalTok{), }
        \AttributeTok{plot.margin =} \FunctionTok{margin}\NormalTok{(}\FloatTok{0.7}\NormalTok{, }\FloatTok{0.5}\NormalTok{, }\DecValTok{1}\NormalTok{, }\FloatTok{1.2}\NormalTok{, }\StringTok{"cm"}\NormalTok{))}
\end{Highlighting}
\end{Shaded}

\includegraphics{capstone-movielens-report.draft4_files/figure-latex/unnamed-chunk-42-1.pdf}

We use the following models to account for the \texttt{date} effect:

\[
Y_{i,j} = \mu + \alpha_i + \beta_j + f(d_{i,j}) + \varepsilon_{i,j}
\]

\subsubsection{Accounting for Genre
effect}\label{accounting-for-genre-effect}

\hfill\break
As mentioned in
\href{https://rafalab.dfci.harvard.edu/dsbook-part-2/highdim/regularization.html\#exercises}{Section
23.7: Exercises} of the \emph{Chapter ``23 Regularization'' of the
Course Textbook} the \texttt{Movielens} dataset also has a genres
column. This column includes every genre that applies to the movie (some
movies fall under several genres)\autocite{IDS2_23-7}.

\paragraph{Genre Data Analysis}\label{genre-data-analysis}

\hfill\break
\#\#\#\#\# Movie Genres Data\\
The following code computes movie rating summaries by popular genres
like Drama, Comedy, Thriller, and Romance:

\begin{Shaded}
\begin{Highlighting}[]
\CommentTok{\#library(stringr)}
\NormalTok{genres }\OtherTok{=} \FunctionTok{c}\NormalTok{(}\StringTok{"Drama"}\NormalTok{, }\StringTok{"Comedy"}\NormalTok{, }\StringTok{"Thriller"}\NormalTok{, }\StringTok{"Romance"}\NormalTok{)}
\FunctionTok{sapply}\NormalTok{(genres, }\ControlFlowTok{function}\NormalTok{(g) \{}
  \FunctionTok{sum}\NormalTok{(}\FunctionTok{str\_detect}\NormalTok{(edx}\SpecialCharTok{$}\NormalTok{genres, g))}
\NormalTok{\})}
\end{Highlighting}
\end{Shaded}

\begin{verbatim}
##    Drama   Comedy Thriller  Romance 
##  3910127  3540930  2325899  1712100
\end{verbatim}

Further, we can find out the movies that have the greatest number of
ratings using the following code:

\begin{Shaded}
\begin{Highlighting}[]
\NormalTok{ordered\_movie\_ratings }\OtherTok{\textless{}{-}}\NormalTok{ edx }\SpecialCharTok{|\textgreater{}} \FunctionTok{group\_by}\NormalTok{(movieId, title) }\SpecialCharTok{|\textgreater{}}
  \FunctionTok{summarize}\NormalTok{(}\AttributeTok{number\_of\_ratings =} \FunctionTok{n}\NormalTok{()) }\SpecialCharTok{|\textgreater{}}
  \FunctionTok{arrange}\NormalTok{(}\FunctionTok{desc}\NormalTok{(number\_of\_ratings))}
\end{Highlighting}
\end{Shaded}

\begin{verbatim}
## `summarise()` has grouped output by 'movieId'. You can override using the `.groups` argument.
\end{verbatim}

\begin{Shaded}
\begin{Highlighting}[]
\FunctionTok{print}\NormalTok{(}\FunctionTok{head}\NormalTok{(ordered\_movie\_ratings))}
\end{Highlighting}
\end{Shaded}

\begin{verbatim}
## # A tibble: 6 x 3
## # Groups:   movieId [6]
##   movieId title                            number_of_ratings
##     <int> <chr>                                        <int>
## 1     296 Pulp Fiction (1994)                          31362
## 2     356 Forrest Gump (1994)                          31079
## 3     593 Silence of the Lambs, The (1991)             30382
## 4     480 Jurassic Park (1993)                         29360
## 5     318 Shawshank Redemption, The (1994)             28015
## 6     110 Braveheart (1995)                            26212
\end{verbatim}

and figure out the most given ratings in order from most to least:

\begin{Shaded}
\begin{Highlighting}[]
\NormalTok{ratings }\OtherTok{\textless{}{-}}\NormalTok{ edx }\SpecialCharTok{|\textgreater{}}  \FunctionTok{group\_by}\NormalTok{(rating) }\SpecialCharTok{|\textgreater{}}
     \FunctionTok{summarise}\NormalTok{(}\AttributeTok{count =} \FunctionTok{n}\NormalTok{()) }\SpecialCharTok{|\textgreater{}}
     \FunctionTok{arrange}\NormalTok{(}\FunctionTok{desc}\NormalTok{(count))}
\FunctionTok{print}\NormalTok{(ratings)}
\end{Highlighting}
\end{Shaded}

\begin{verbatim}
## # A tibble: 10 x 2
##    rating   count
##     <dbl>   <int>
##  1    4   2588430
##  2    3   2121240
##  3    5   1390114
##  4    3.5  791624
##  5    2    711422
##  6    4.5  526736
##  7    1    345679
##  8    2.5  333010
##  9    1.5  106426
## 10    0.5   85374
\end{verbatim}

The following code allows us to summarize that in general, half-star
ratings are less common than whole-star ratings (e.g., there are fewer
ratings of 3.5 than there are ratings of 3 or 4, etc.):

\begin{Shaded}
\begin{Highlighting}[]
\FunctionTok{print}\NormalTok{(edx }\SpecialCharTok{|\textgreater{}} \FunctionTok{group\_by}\NormalTok{(rating) }\SpecialCharTok{|\textgreater{}} \FunctionTok{summarize}\NormalTok{(}\AttributeTok{count =} \FunctionTok{n}\NormalTok{()))}
\end{Highlighting}
\end{Shaded}

\begin{verbatim}
## # A tibble: 10 x 2
##    rating   count
##     <dbl>   <int>
##  1    0.5   85374
##  2    1    345679
##  3    1.5  106426
##  4    2    711422
##  5    2.5  333010
##  6    3   2121240
##  7    3.5  791624
##  8    4   2588430
##  9    4.5  526736
## 10    5   1390114
\end{verbatim}

We can visually see that from the following plot:

\begin{Shaded}
\begin{Highlighting}[]
\NormalTok{edx }\SpecialCharTok{|\textgreater{}}
  \FunctionTok{group\_by}\NormalTok{(rating) }\SpecialCharTok{|\textgreater{}}
  \FunctionTok{summarize}\NormalTok{(}\AttributeTok{count =} \FunctionTok{n}\NormalTok{()) }\SpecialCharTok{|\textgreater{}}
  \FunctionTok{ggplot}\NormalTok{(}\FunctionTok{aes}\NormalTok{(}\AttributeTok{x =}\NormalTok{ rating, }\AttributeTok{y =}\NormalTok{ count)) }\SpecialCharTok{+}
  \FunctionTok{geom\_line}\NormalTok{() }
\end{Highlighting}
\end{Shaded}

\includegraphics{capstone-movielens-report.draft4_files/figure-latex/unnamed-chunk-47-1.pdf}

\subparagraph{Movie Genres Effect}\label{movie-genres-effect}

\hfill\break

The plot below shows strong evidence of a genre effect (for illustrative
purposes, the plot shows only categories with more than 20, 000
ratings).

\begin{Shaded}
\begin{Highlighting}[]
\CommentTok{\# Preparing data for plotting:}
\NormalTok{genre\_ratins\_grp }\OtherTok{\textless{}{-}}\NormalTok{ train\_set }\SpecialCharTok{|\textgreater{}} 
  \FunctionTok{mutate}\NormalTok{(}\AttributeTok{genre\_categories =} \FunctionTok{as.factor}\NormalTok{(genres)) }\SpecialCharTok{|\textgreater{}}
  \FunctionTok{group\_by}\NormalTok{(genre\_categories) }\SpecialCharTok{|\textgreater{}}
  \FunctionTok{summarize}\NormalTok{(}\AttributeTok{n =} \FunctionTok{n}\NormalTok{(), }\AttributeTok{rating\_avg =} \FunctionTok{mean}\NormalTok{(rating), }\AttributeTok{se =} \FunctionTok{sd}\NormalTok{(rating)}\SpecialCharTok{/}\FunctionTok{sqrt}\NormalTok{(}\FunctionTok{n}\NormalTok{())) }\SpecialCharTok{|\textgreater{}}
  \FunctionTok{filter}\NormalTok{(n }\SpecialCharTok{\textgreater{}} \DecValTok{20000}\NormalTok{) }\SpecialCharTok{|\textgreater{}} 
  \FunctionTok{mutate}\NormalTok{(}\AttributeTok{genres =} \FunctionTok{reorder}\NormalTok{(genre\_categories, rating\_avg)) }\SpecialCharTok{|\textgreater{}}
  \FunctionTok{select}\NormalTok{(genres, rating\_avg, se, n)}

\FunctionTok{dim}\NormalTok{(genre\_ratins\_grp)}
\end{Highlighting}
\end{Shaded}

\begin{verbatim}
## [1] 53  4
\end{verbatim}

\begin{Shaded}
\begin{Highlighting}[]
\NormalTok{genre\_ratins\_grp\_sorted }\OtherTok{\textless{}{-}}\NormalTok{ genre\_ratins\_grp }\SpecialCharTok{|\textgreater{}} \FunctionTok{sort\_by.data.frame}\NormalTok{(}\SpecialCharTok{\textasciitilde{}}\NormalTok{ rating\_avg)}
\FunctionTok{print}\NormalTok{(genre\_ratins\_grp\_sorted)}
\end{Highlighting}
\end{Shaded}

\begin{verbatim}
## # A tibble: 53 x 4
##    genres                  rating_avg      se      n
##    <fct>                        <dbl>   <dbl>  <int>
##  1 Comedy|Horror                 2.83 0.00701  24892
##  2 Horror                        2.85 0.00556  48671
##  3 Children|Comedy               2.86 0.00575  40435
##  4 Horror|Sci-Fi                 2.88 0.00787  20524
##  5 Action|Comedy                 2.96 0.00581  34118
##  6 Comedy|Sci-Fi                 3.06 0.00668  27888
##  7 Action|Sci-Fi                 3.08 0.00749  27971
##  8 Horror|Thriller               3.09 0.00543  49751
##  9 Action|Adventure|Comedy       3.17 0.00616  28256
## 10 Comedy                        3.21 0.00167 439352
## # i 43 more rows
\end{verbatim}

\begin{Shaded}
\begin{Highlighting}[]
\CommentTok{\# Creating plot:}
\NormalTok{genre\_ratins\_grp }\SpecialCharTok{|\textgreater{}} 
  \FunctionTok{ggplot}\NormalTok{(}\FunctionTok{aes}\NormalTok{(}\AttributeTok{x =}\NormalTok{ genres, }\AttributeTok{y =}\NormalTok{ rating\_avg, }\AttributeTok{ymin =}\NormalTok{ rating\_avg }\SpecialCharTok{{-}} \DecValTok{2}\SpecialCharTok{*}\NormalTok{se, }\AttributeTok{ymax =}\NormalTok{ rating\_avg }\SpecialCharTok{+} \DecValTok{2}\SpecialCharTok{*}\NormalTok{se)) }\SpecialCharTok{+} 
  \FunctionTok{geom\_point}\NormalTok{() }\SpecialCharTok{+}
  \FunctionTok{geom\_errorbar}\NormalTok{() }\SpecialCharTok{+} 
  \FunctionTok{ggtitle}\NormalTok{(}\StringTok{"Average rating per Genre"}\NormalTok{) }\SpecialCharTok{+}
  \FunctionTok{ylab}\NormalTok{(}\StringTok{"Average rating"}\NormalTok{) }\SpecialCharTok{+}
  \FunctionTok{theme}\NormalTok{(}\AttributeTok{axis.text.x =} \FunctionTok{element\_text}\NormalTok{(}\AttributeTok{angle =} \DecValTok{90}\NormalTok{, }\AttributeTok{hjust =} \DecValTok{1}\NormalTok{))}
\end{Highlighting}
\end{Shaded}

\includegraphics{capstone-movielens-report.draft4_files/figure-latex/unnamed-chunk-48-1.pdf}
Below are worst and best ratings categories:

\begin{Shaded}
\begin{Highlighting}[]
\FunctionTok{sprintf}\NormalTok{(}\StringTok{"The worst ratings are for the genre category: \%s"}\NormalTok{,}
\NormalTok{        genre\_ratins\_grp}\SpecialCharTok{$}\NormalTok{genres[}\FunctionTok{which.min}\NormalTok{(genre\_ratins\_grp}\SpecialCharTok{$}\NormalTok{genres)])}
\end{Highlighting}
\end{Shaded}

\begin{verbatim}
## [1] "The worst ratings are for the genre category: Comedy|Horror"
\end{verbatim}

\begin{Shaded}
\begin{Highlighting}[]
\FunctionTok{sprintf}\NormalTok{(}\StringTok{"The best ratings are for the genre category: \%s"}\NormalTok{,}
\NormalTok{        genre\_ratins\_grp}\SpecialCharTok{$}\NormalTok{genres[}\FunctionTok{which.max}\NormalTok{(genre\_ratins\_grp}\SpecialCharTok{$}\NormalTok{genres)])}
\end{Highlighting}
\end{Shaded}

\begin{verbatim}
## [1] "The best ratings are for the genre category: Comedy|Crime|Drama"
\end{verbatim}

Another way of visualizing a genre effect is shown in the section
\href{https://www.kaggle.com/code/amirmotefaker/movie-recommendation-system-using-r-best/notebook\#Average-rating-for-each-genre}{Average
rating for each genre} of the article ``Movie Recommendation System
using R - BEST'' written by
\href{https://www.kaggle.com/amirmotefaker}{Amir
Moterfaker}\autocite{MRS-R-BEST}:

\begin{Shaded}
\begin{Highlighting}[]
\CommentTok{\# For better visibility, we reduce the data for plotting }
\CommentTok{\# while keeping the worst and best rating rows:}
\NormalTok{plot\_ind }\OtherTok{\textless{}{-}} \FunctionTok{odd}\NormalTok{(}\DecValTok{1}\SpecialCharTok{:}\FunctionTok{nrow}\NormalTok{(genre\_ratins\_grp))}
\NormalTok{plot\_dat }\OtherTok{\textless{}{-}}\NormalTok{ genre\_ratins\_grp\_sorted[plot\_ind,] }

\NormalTok{plot\_dat }\SpecialCharTok{|\textgreater{}}
  \FunctionTok{ggplot}\NormalTok{(}\FunctionTok{aes}\NormalTok{(}\AttributeTok{x =}\NormalTok{ rating\_avg, }\AttributeTok{y =}\NormalTok{ genres)) }\SpecialCharTok{+}
  \FunctionTok{ggtitle}\NormalTok{(}\StringTok{"Genre Average Rating"}\NormalTok{) }\SpecialCharTok{+}
  \FunctionTok{geom\_bar}\NormalTok{(}\AttributeTok{stat =} \StringTok{"identity"}\NormalTok{, }\AttributeTok{width =} \FloatTok{0.6}\NormalTok{, }\AttributeTok{fill =} \StringTok{"\#8888ff"}\NormalTok{) }\SpecialCharTok{+}
  \FunctionTok{xlab}\NormalTok{(}\StringTok{"Average ratings"}\NormalTok{) }\SpecialCharTok{+}
  \FunctionTok{ylab}\NormalTok{(}\StringTok{"Genres"}\NormalTok{) }\SpecialCharTok{+}
  \FunctionTok{scale\_x\_continuous}\NormalTok{(}\AttributeTok{labels =}\NormalTok{ comma, }\AttributeTok{limits =} \FunctionTok{c}\NormalTok{(}\FloatTok{0.0}\NormalTok{, }\FloatTok{5.0}\NormalTok{)) }\SpecialCharTok{+}
  \FunctionTok{theme\_economist}\NormalTok{() }\SpecialCharTok{+}
  \FunctionTok{theme}\NormalTok{(}\AttributeTok{plot.title =} \FunctionTok{element\_text}\NormalTok{(}\AttributeTok{vjust =} \FloatTok{3.5}\NormalTok{),}
        \AttributeTok{axis.title.x =} \FunctionTok{element\_text}\NormalTok{(}\AttributeTok{vjust =} \SpecialCharTok{{-}}\DecValTok{5}\NormalTok{, }\AttributeTok{face =} \StringTok{"bold"}\NormalTok{),}
        \AttributeTok{axis.title.y =} \FunctionTok{element\_text}\NormalTok{(}\AttributeTok{vjust =} \DecValTok{10}\NormalTok{, }\AttributeTok{face =} \StringTok{"bold"}\NormalTok{),}
        \AttributeTok{axis.text.x =} \FunctionTok{element\_text}\NormalTok{(}\AttributeTok{vjust =} \DecValTok{1}\NormalTok{, }\AttributeTok{hjust =} \DecValTok{1}\NormalTok{, }\AttributeTok{angle =} \DecValTok{0}\NormalTok{),}
        \AttributeTok{axis.text.y =} \FunctionTok{element\_text}\NormalTok{(}\AttributeTok{vjust =} \FloatTok{0.25}\NormalTok{, }\AttributeTok{hjust =} \DecValTok{1}\NormalTok{, }\AttributeTok{size =} \DecValTok{8}\NormalTok{),}
        \AttributeTok{plot.margin =} \FunctionTok{margin}\NormalTok{(}\FloatTok{0.7}\NormalTok{, }\FloatTok{0.5}\NormalTok{, }\DecValTok{1}\NormalTok{, }\FloatTok{1.2}\NormalTok{, }\StringTok{"cm"}\NormalTok{))}
\end{Highlighting}
\end{Shaded}

\includegraphics{capstone-movielens-report.draft4_files/figure-latex/unnamed-chunk-50-1.pdf}

If we define \(g_{i,j}\) as the genre for user's \(i\) rating of movie
\(j\), we can use the following models to account for the \texttt{genre}
effect:

To account for \emph{genre effects} we will use the model suggested in
the
\href{https://rafalab.dfci.harvard.edu/dsbook-part-2/highdim/regularization.html\#exercises}{Section
23.7: Exercises} of the \emph{Chapter ``23 Regularization'' of the
Course Textbook}\autocite{IDS2_23-7}:

\[
Y_{i,j} = \mu + \alpha_i + \beta_j + g_{i,j} + \varepsilon_{i,j}
\]

where \(g_{i,j}\) is an \emph{aggregation function} which is explained
in detail in \emph{Section 22.3: ``Review of Aggregation Functions'' of
``Recommender Systems Handbook''} (\emph{Chapter 22: ``Aggregation of
Preferences in Recommender Systems''}, p.~712)
book\autocite{RRSK_RS_HB}.

In the formula above \(g_{i,j}\) denotes a \emph{genre effect} for
user's \(i\) rating of movie \(j\), so that:

\[
g_{i,j} = \sum_{k=1}^K x_{i,j}^k \gamma_k
\]

with \(x^k_{i,j} = 1\) if \(g_{i,j}\) includes genre \(k\), and
\(x^k_{i,j} = 0\) otherwise.

\[
Y_{i,j} = \mu + \alpha_i + \beta_j + g_{i,j} + f(d_{i,j})
\]

\[
 \sum_{i=1}^{n_i} \left(Y_{i,j} - \mu - \alpha_i\right)
\]

\subsection{Conclusion}\label{conclusion}

Hello Conclusion!

This is a great conclusion, isn't it?!!

\printbibliography

\end{document}
